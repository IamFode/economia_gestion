\chapter{Una introducción a la econometría}

\begin{tcolorbox}[colback=white]
    La \textbf{econometría} se trata de cómo podemos usar la teoría y los datos de la economía, los negocios y las ciencias sociales, junto con las herramientas de las estadísticas, para predecir resultados, responder preguntas del tipo "cuánto" y probar hipótesis.
\end{tcolorbox}

\section{El proceso de investigación}
el proceso de investigación,  generalmente sigue estos pasos:
\begin{enumerate}[\bfseries 1.]
    \item La teoría económica nos da una forma de pensar sobre el problema. ¿Qué variables económicas están involucradas y cuál es la posible dirección de la(s) relación(es)? Todo proyecto de investigación, dada la pregunta inicial, comienza construyendo un modelo económico y enumerando las preguntas (hipótesis) de interés. Surgirán más preguntas durante el proyecto de investigación, pero es bueno enumerar aquellas que te motivan al comienzo del proyecto.
    \item El modelo económico de trabajo conduce a un modelo econométrico. Debemos elegir una forma funcional y hacer algunas suposiciones sobre la naturaleza del término de error.
    \item Se obtienen datos de muestra y se elige un método deseable de análisis estadístico, basado en suposiciones iniciales y una comprensión de cómo se recopilaron los datos.
    \item Las estimaciones de los parámetros desconocidos se obtienen con la ayuda de un paquete de software estadístico, se hacen predicciones y se realizan pruebas de hipótesis.
    \item Se realizan diagnósticos del modelo para comprobar la validez de las suposiciones. Por ejemplo, ¿eran relevantes todas las variables explicativas del lado derecho? ¿Se utilizó una forma funcional adecuada?.
    \item Se analizan y evalúan las consecuencias económicas y las implicaciones de los resultados empíricos. ¿Qué resultados de asignación y distribución de recursos económicos están implícitos y cuáles son sus implicaciones en la elección de políticas? ¿Qué preguntas restantes podrían responderse con más estudios o con datos nuevos y mejores?.
\end{enumerate}

\section{Escribir un artículo de investigación empírica}
\subsection{Escribir una propuesta de investigación}
El resumen debe ser breve, por lo general de no más de 500 palabras, y debe incluir lo siguiente:
\begin{enumerate}[\bfseries 1.]
    \item Una declaración concisa del problema.
    \item Comentarios sobre la información disponible, con una o dos referencias clave.
    \item Una descripción de la investigación diseño que incluye
	\begin{enumerate}[\bfseries a.]
	    \item El modelo económico.
	    \item Los métodos econométricos de estimación e inferencia.
	    \item Fuentes de datos.
	    \item Procedimientos de estimación, prueba de hipótesis y predicción, incluido el software econométrico y la versión utilizada 
	\end{enumerate}
    \item La contribución potencial de la investigación.
\end{enumerate}

\subsection{Un formato para escribir un informe de investigación}
\begin{enumerate}[\bfseries 1.]
    \item \textbf{Declaración del problema}. El lugar para comenzar su informe es con un resumen de las preguntas que desea investigar, así como por qué son importantes y quién debería estar interesado en los resultados. Esta sección introductoria no debe ser técnica y debe motivar al lector a continuar leyendo el artículo. También es útil trazar un mapa del contenido de las siguientes secciones del informe. Esta es la primera sección en la que trabajar y también la última. En el mundo ajetreado de hoy, la atención del lector debe captarse muy rápidamente. Una introducción clara, concisa y bien escrita es imprescindible y posiblemente sea la parte más importante del documento. 
    \item \textbf{Revisión de la literatura}. Resuma brevemente la literatura relevante en el área de investigación que ha elegido y aclare cómo su trabajo amplía nuestro conocimiento. Por todos los medios, cite los trabajos de otros que hayan motivado su investigación, pero sea breve. No tienes que repasar todo lo que se ha escrito sobre el tema. 
    \item \textbf{El modelo económico}. Especifique el modelo económico que utilizó y defina las variables económicas. Indique los supuestos del modelo e identifique las hipótesis que desea probar. Los modelos económicos pueden complicarse. Su tarea es explicar el modelo claramente, pero de la manera más breve y sencilla posible. No utilice jerga técnica innecesaria. Use términos simples en lugar de complicados cuando sea posible. Su objetivo es mostrar la calidad de su pensamiento, no la extensión de su vocabulario. 
    \item \textbf{El modelo econométrico}. Discuta el modelo econométrico que corresponde al modelo económico. Asegúrese de incluir una discusión de las variables en el modelo, la forma funcional, las suposiciones de error y cualquier otra suposición que haga. Utilice una notación lo más simple posible y no abarrote el cuerpo del artículo con largas demostraciones o derivaciones; estos pueden ir en un apéndice técnico. 
    \item \textbf{Los datos}. Describa los datos que utilizó, así como la fuente de los datos y cualquier reserva que tenga sobre su idoneidad. 
    \item \textbf{Los procedimientos de estimación e inferencia}. Describe los métodos de estimación que usaste y por qué los elegiste. Explicar los procedimientos de prueba de hipótesis y su uso. Indique el software utilizado y la versión, como Stata 15 o EViews 10. 
    \item \textbf{Los resultados empíricos y las conclusiones Informe las estimaciones de los parámetros, su interpretación y los valores de las estadísticas de prueba.} Comente su significancia estadística, su relación con estimaciones previas y sus implicaciones económicas. 
    \item \textbf{Posibles extensiones y limitaciones del estudio Su investigación generará preguntas sobre el modelo económico, los datos y las técnicas de estimación.} ¿Qué investigaciones futuras sugieren sus hallazgos y cómo podría llevarlas a cabo? 
    \item \textbf{Agradecimientos}. Es apropiado reconocer a quienes han comentado y contribuido a su investigación. Esto puede incluir a su instructor, un bibliotecario que lo ayudó a encontrar datos o un compañero de estudios que leyó y comentó su trabajo. 
    \item \textbf{Referencias}. Una lista alfabética de la literatura que cita en su estudio, así como referencias a las fuentes de datos que utilizó.
\end{enumerate}

\section{Fuentes de datos económicos}
\begin{itemize}
    \item Resources for Economists (RFE) www.rfe.org
    \item National Bureau of Economic Research (NBER) www.nber.org/data
    \item Economagic https://www.economagic.com/
\end{itemize}
