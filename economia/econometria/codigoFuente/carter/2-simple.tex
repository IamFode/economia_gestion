\chapter{El modelo de regresión lineal simple}


\section{Un modelo econométrico}
Sea $e=$ todo lo demás que afecta a la variable dependiente. Es decir, el termino de error, entonces:
\begin{equation}
    y=\beta_1 + \beta_2 x  + e
\end{equation}

El parámetro desconocido $\beta_2$, la propensión marginal a gastar en alimentos a partir de la renta, nos dice la proporción del aumento de la renta utilizada para la compra de alimentos; responde a la pregunta cuánto ¿Cuánto cambiará el gasto en alimentos dado un cambio en los ingresos, manteniendo todo lo demás constante?.\\
El primer supuesto del modelo de regresión lineal simple es que la relación (2.1) se cumple para los miembros de la población bajo consideración. Afirmamos que la regla de comportamiento $y = \beta_1 + \beta_2 x + e$ se cumple para todos los hogares de la población. Cada semana el gasto en alimentos es igual a $\beta_1$, más una proporción $\beta_2$ de los ingresos, más otros factores, $e$.

\subsection{Proceso de generación de datos}
La selección aleatoria de hogares hace que el primer par de observación $(y_1, x_1)$ sea estadísticamente independiente de todos los demás pares de datos, y cada par de observación $(y_i, x_i)$ es estadísticamente independiente de cualquier otro par de datos, $(y_j, x_j)$, donde $i \neq j$. Suponemos además que las variables aleatorias $y_i$ y $x_i$ tienen una función de densidad de probabilidad conjunta $f(y_i , x_i)$ que describe su distribución de valores. A menudo no conocemos la naturaleza exacta de la distribución conjunta, pero se supone que todos los pares extraídos de la misma población siguen la misma distribución conjunta y, por lo tanto, la de los pares de datos no solo son estadísticamente independientes, sino que también están distribuidos de manera idéntica (iid abreviado o iid). Se dice que los pares de datos que son iid son una muestra aleatoria.\\
Si nuestra primera suposición es cierta, que la regla de comportamiento $y = \beta_1 + \beta_2x + e$ se cumple para todos los hogares de la población, entonces reexpresando (2.1) para cada par de datos $(y_i, x_i)$ tenemos,
$$y_i = \beta_1 + \beta_2 x_i + e_i,\qquad i=1,\ldots, N$$
Esto a veces se denomina proceso de generación de datos (DGP) porque asumimos que los datos observables siguen esta relación.

\subsection{El error aleatorio y la exogeneidad estricta}
El segundo supuesto del modelo de regresión simple (2.1) se refiere al término todo lo demás $e$. Las variables $( y_i , x_i )$ son variables aleatorias porque no sabemos qué valores toman hasta que se elige un hogar en particular y se observan. El término de error $e_i$ también es una variable aleatoria. Todos los demás factores que afectan el gasto en alimentos, excepto los ingresos, serán diferentes para cada hogar de la población, por la única razón de que los gustos y preferencias de todos son diferentes. A diferencia de los gastos e ingresos en alimentos, el término de error aleatorio $e_i$ no es observable; es inobservable. No podemos medir gustos y preferencias de forma directa, como tampoco podemos medir directamente la utilidad económica que se deriva de comer un trozo de tarta. \textbf{El segundo supuesto de regresión es que la variable $x$, el ingreso, no se puede utilizar para predecir el valor de $e_i$}, el efecto de la recopilación de todos los demás factores que afectan el gasto en alimentos por parte del i-ésimo hogar. Dado un valor de ingreso $x_i$ para el i-ésimo hogar, el mejor predictor (óptimo) del error aleatorio $e_i$ es la expectativa condicional, o media condicional, $E ( e_i | x_i )$ . \textbf{La suposición de que $x_i$ no se puede usar para predecir $e_i$ es equivalente a decir que $E ( e_i | x_i ) = 0$}. Es decir, dado el ingreso de un hogar, no podemos hacer nada mejor que predecir que el error aleatorio es cero; los efectos de todos los demás factores sobre el gasto en alimentos promedian, de una manera muy específica, a cero. \\
$E ( e_i | x_i ) = 0$ tiene dos implicaciones. El primero es $E ( e_i | x_i ) = 0 \; \Longrightarrow   E ( e_i) = 0$; si el valor esperado condicional del error aleatorio es cero, entonces la expectativa incondicional del error aleatorio también es cero. En la población, el efecto promedio de todos los factores omitidos resumidos por el término de error aleatorio es cero.\\
La segunda implicación es $E ( e_i | x_i ) = 0 \; \Longrightarrow \; cov( e_i , x_i ) = 0$. Si el valor esperado condicional del error aleatorio es cero, entonces $e_i$, el error aleatorio para la i-ésima observación, tiene covarianza cero y correlación cero, con la correspondiente observación $x_i$. En nuestro ejemplo, el componente aleatorio $e_i$, que representa todos los factores que afectan el gasto en alimentos excepto los ingresos del i-ésimo hogar, no está correlacionado con los ingresos de ese hogar. \textbf{Quizás se pregunte cómo podría demostrarse que eso es cierto. Después de todo, $e_i$ es inobservable. Debe convencerse de que todo lo que podría haberse omitido del modelo no está correlacionado con $x_i$}. La herramienta principal es el razonamiento económico: sus propios experimentos intelectuales (es decir, el pensamiento), la lectura de literatura sobre el tema y las discusiones con colegas o compañeros de clase. Y realmente no podemos probar que $E( e_i | x_i ) = 0$ sea cierto con absoluta certeza en la mayoría de los modelos económicos.\\
Notamos que $E ( e_i | x_i ) = 0$ tiene dos implicaciones. Si alguna de las implicaciones no es verdadera, entonces $E ( e_i | x_i ) = 0$ no es verdadera, es decir,
$$E(e_i|x_i) \neq 0 \; \mbox{si}\; E(e_i)\neq 0\; \mbox{o si}\; cov(e_i,x_i)\neq 0$$
Si $cov ( e_i , x_i ) = 0$, se dice que la variable explicativa $x$ es exógena, siempre que se cumpla nuestra primera suposición de que los pares $(y_i, x_i)$ son iid. Cuando $x$ es exógena, el análisis de regresión se puede utilizar con éxito para estimar $\beta_1$ y $\beta_2$. Para diferenciar la condición más débil $cov ( e_i , x_i ) = 0$, exogeneidad simple, de la condición más fuerte $E ( e_i | x_i ) = 0$, decir que $x$ es estrictamente exógena si $E ( e_i | x_i ) = 0$. Si $cov( e_i , x_i ) \neq  0$, entonces se dice que $x$ es endógena . Cuando $x$ es endógeno, es más difícil, a veces mucho más difícil, realizar una inferencia estadística.

\subsection{La función de regresión}
