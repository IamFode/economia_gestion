\setcounter{chapter}{1}
\chapter{El modelo de regresión simple}
\section{Definición del modelo de regresión simple}
\begin{equation}
    y = \beta_o + \beta_1x + u.
\end{equation}

\section{Obtención de las estimaciones de mínimos cuadrados ordinarios}

Sea $\left\{(x_i,y_i): i = 1,\ldots,n\right\}$, una muestra aleatoria de tamaño $n$ tomada de la población. Como estos datos provienen de $(2.1)$ para todo $i$ puede escribirse 
\begin{equation}
    y_i = \beta_o + \beta_1x_i + u_i.
\end{equation}

En tanto el intercepto $\beta_0$ aparezca en la ecuación, nada se altera al suponer que el valor promedio de $u$ en la población, es cero. Es decir, $E(u)=0$.\\
El supuesto crucial es que el valor promedio de $u$ no depende del valor de $x$. Este supuesto se expresa como $u$ $E(u\backslash x) = E(u)$. Esta última ecuación indica que el valor promedio de los factores no observables es el mismo en todas las fracciones de la población determinados por los valores de $x$ y que este promedio común es necesariamente igual al promedio al promedio de $u$ en toda la población. Y por lo tanto $u$ es media independiente de $x$. Combinando la independencia de la media con el supuesto de $E(u)=0$ se obtiene el supuesto de media condicional cero, $E(u\backslash x) = 0$\\\\

En la población, $u$ no está correlacionada con $x$. Por tanto, se tiene que el valor esperado de $u$ es cero y que la covarianza entre $x$ y $y$ es cero:
\begin{equation}
    E(u)=0
\end{equation}
 y 

\begin{equation}
    Cov(x,u) = E(xu) =0.
\end{equation}

\textbf{Covarianza.-} Sean $\mu_x = E(X)$ y $\mu_y = E(Y)$ y considere la variable aleatoria $(X-\mu_x)(Y-\mu_y)$. Si $X$ es mayor a su media y $Y$ es mayor a su media, entonces $(X-\mu_x)(Y-\mu_y)>0$. La covarianza entre dos variables aleatorias $X$ y $Y$ llamada algunas veces covarianza poblacional, para hacer énfasis en que se refiere a la relación entre dos variables que describen una población, está definida como el valor esperado del producto $(X-\mu_x)(Y-\mu_y)$: 

\begin{equation}
    Cov(X,Y) = E\left[(X-\mu_x)(Y-\mu_y)\right]
\end{equation}

que también suele denotarse como $\sigma_{XY}.$  \\
Algunas expresiones para útiles para calcular $Cov(X,Y)$ son las siguientes

\begin{equation}
    Cov(X,Y) = E\left[(X-\mu_x)(Y-\mu_y)\right] = E\left[(X-\mu_X)Y\right] = E\left[X(Y-\mu_Y)\right] = E(XY) - \mu_x \mu_y
\end{equation}

De donde se sigue que si $E(x)=0$ o $E(Y)=0$, entonces $Cov(X,Y) = E(XY)$.\\\\

Luego  
\begin{equation}
    E(y - \beta_0 - \beta_1x) = 0
\end{equation}
 
y 

\begin{equation}
    E\left[x(y - \beta_0 -\beta_i x )= 0\right]
\end{equation}

Como hay que estimar dos parámetros desconocidos, se espera que las dos ecuaciones anteriores puedan servir para obtener buenos estimadores de $\beta_0$ y $\beta_1$. En efecto estas ecuaciones pueden servir para la estimación de estos parámetros.\\\\

Por el método de momentos para la estimación , y las anteriores dos ecuaciones,
\begin{equation}
    n^{-1} \sum\limits_{i=1}^n (y_i - \beta_0 - \beta_1x_i) = 0
\end{equation}
y 
\begin{equation}
    n^{-1} \sum\limits_{i=1}^n x_i(y_i - \beta_0 - \beta_1x_i) = 0
\end{equation}
Luego por la ecuación (2.9) tenemos que 
\begin{equation}
    \overline{y} = \hat{\beta_0} + \hat{\beta_1}\overline{x}.
\end{equation}
donde $\overline{y} = n^{-1}\sum\limits_{i=1}^n y_i$ es el promedio muestral de las $y_i$, y lo mismo ocurre con $\overline{x}$, así, 

\begin{tcolorbox}[colframe=white]
\begin{equation}
    \hat{\beta_0} = \overline{y} - \hat{\beta_1}\overline{x}.
\end{equation}
\end{tcolorbox}
Por último empleando (2.10) y (2.12) para sustituir $\hat{\beta_0}$ se obtiene,
$$\sum_{i=1}^n x_i\left[y_i-(\overline{y}-\hat{\beta_1}\overline{x})-\hat{\beta_1}x_i\right] = 0,$$
de donde, reordenando, tenemos que
$$\sum_{i=1}^n x_i(y_i-\overline{y}) = \hat{\beta_1}\sum_{i=1}^n x_i(x_i-\overline{x}).$$
en consecuencia por las propiedades de la sumatoria, 
$$\sum_{i=1}^n x_i(x_i-\overline{x}) = \sum_{i=1}^n (x_i-\overline{x})^2 \quad y \quad \sum_{i=1}^n x_i(y_i-\overline{y}) = \sum_{i=1}^n (x_i-\overline{x})(y_i-\overline{y})$$\\
ya que $\sum\limits_{i=1}^n (x_i-\overline{x})(y_i-\overline{y})=\sum\limits_{i=1}^n x_i y_i - \overline{y}\sum\limits_{i=1}^n x_i - \overline{x}\sum\limits_{i=1}^n y_i + \overline{y} \overline{x}\sum\limits_{i=1}^n 1 $, luego ya que $\sum\limits_{i=1}^n x_i = n\overline{x}$ entonces $\sum\limits_{i=1}^n x_iy_i - n\overline{yx} - n\overline{xy} + n\overline{xy} = \sum\limits_{i=1}^n (x_i-\overline{x})y_i$
por lo tanto, 
\begin{tcolorbox}[colframe=white]
\begin{equation}
    \hat{\beta_1} = \frac{\sum\limits_{i=1}^n (x_i-\overline{x})(y_i-\overline{y})}{\sum\limits_{i=1}^n (x_i-\overline{x})^2}.
\end{equation}
\end{tcolorbox}
Ésta ecuación no es nada mas que la covarianza muestral en $x$ e $y$ dividida entre la variación muestral de $x$. Esto tiene sentido porque $\beta_1$ es igual a la covarianza poblacional dividida entre la varianza de $x$ cuando $E(u)=0$ y $Cov(x,u) = 0$. Como consecuencia directa se tiene que si en la muestra $x$ e $y$ están correlacionadas positivamente, entonces $\hat{\beta_1}$ es positiva y contrariamente.\\\\
Para todo $\hat{\beta_0}$ y $\hat{\beta_1}$ se define el valor ajustado para $y$ cuando $x=x_i$ como 
\begin{equation}
	\hat{y_i} = \hat{\beta_0} + \hat{\beta_1}x_i.
\end{equation}
Este es el valor que se predice para $y$ cuando $x=x_i$.\\\\
El \textbf{residual} de la observación $i$ es la diferencia entre el verdadero valor $y_i$ y si valor ajustado.\begin{equation} 
    \hat{u} = y_i - \hat{y_i} = y_i - \hat{\beta_0} - \hat{\beta_1}x_i.
\end{equation}
Los residuales no son lo mismo que la ecuación (2.2)\\\\
Supongamos que $\hat{\beta_0}$ y $\hat{\beta_1}$, se eligen de manera que la suma de residuales cuadrados, 
\begin{equation}
    \sum\limits_{i=1}^n \hat{u_i}^2 = \sum\limits_{i=1}^n (y_i - \hat{\beta_0} - \hat{\beta_1}x_i)^2
\end{equation}
sea tan pequeña como sea posible. \\\\

\textbf{Minimización de la suma de los residuos cuadrados.-}
Se mostrará que $\hat{\beta_0}$ y $\hat{\beta_1}$ estimados de MCO minimizan la suma de los residuales cuadrados. Formalmente, el problema es encontrar las soluciones $\hat{\beta_0}$ y $\hat{\beta_1}$ del problema de minimización 
$$\min_{b_0,b_1} \sum_{i=1}^n (y_i-b_0-b_ix_i)^2$$
donde $b_0$ y $b_1$ son argumentos ficticios en el problema de optimización. Para simplificar llámesele a esta función $Q(b_0,b_1)$. Una condición para que $\hat{\beta_0}$ y $\hat{\beta_1}$ sean soluciones del problema de minimización es que las derivadas parciales de $Q(b_0,b_1)$ respecto a $b_0$ y $b_1$ evaluadas en $$\hat{\beta_0},\hat{\beta_1}: \quad \dfrac{\partial Q(\hat{\beta_0},\hat{\beta_1})}{\partial b_0}=0 \qquad y \qquad \dfrac{\partial Q(\hat{\beta_0},\hat{\beta_1})}{\partial b_1}=0$$



