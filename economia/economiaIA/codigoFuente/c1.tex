\chapter{Inteligencia artificial y la paradoja de la moderna: un choque de expectativas y estadísticas}

\section{Fuentes de optimismo tecnológico}
Vemos que la tasa de error al etiquetar el contenido de las fotos en ImageNet, un conjunto de datos de más de diez millones de imágenes, se redujeron de más del $30\%$ en $2010$ a menos del $5\%$ en $2016$ y, más recientemente, al $2.2\%$ con SE-ResNet152 en la competencia ILSVRC2017. Las tasas de error en el reconocimiento de voz en el corpus de grabación de voz de Switchboard, que a menudo se usa para medir el progreso en el reconocimiento de voz, han disminuido del $8.5\%$ al $5.5\%$ en el último año (Saon et al. 2017). El umbral del $5$ por ciento es importante porque ese es aproximadamente el desempeño de los humanos en cada una de estas tareas en los mismos datos de prueba.

\section{La decepcionante realidad reciente}
Aunque las tecnologías discutidas anteriormente tienen un gran potencial, hay pocas señales de que hayan afectado las estadísticas de productividad agregada. Las tasas de crecimiento de la productividad laboral en una amplia franja de economías desarrolladas cayeron a mediados de la primera década del siglo XXI y se han mantenido bajas desde entonces. Las tasas de crecimiento de la productividad laboral en una amplia franja de economías desarrolladas cayeron a mediados de la primera década del siglo XXI y se han mantenido bajas desde entonces. Por ejemplo, el crecimiento de la productividad laboral agregada en los Estados Unidos promedió solo el 1,3 por ciento anual entre 2005 y 2016, menos de la mitad de la tasa de crecimiento anual del 2,8 por ciento sostenida entre 1995 y 2004. Veintiocho de los otros veintinueve países para que la OCDE ha compilado datos de crecimiento de la productividad experimentó desaceleraciones similares. La tasa de crecimiento de la productividad laboral anual promedio no ponderada en estos países fue del $2.3\%$ entre $1995$ y $2004$, pero solo del $1.1\%$ entre $2005$ y $2015$. Además, el ingreso medio real se ha estancado desde fines de la década de $1990$ y las medidas no económicas de bienestar, como la esperanza de vida, han disminuido para algunos grupos.\\
Con lo que algunos han interpretado estos hechos como razones para el pesimismo sobre la capacidad de las nuevas tecnologías como la IA para afectar en gran medida la productividad y los ingresos.

\section{Posibles explicaciones de la paradoja}
Hay cuatro posibles explicaciones principales para la confluencia actual del optimismo tecnológico y el bajo rendimiento de la productividad: 

\begin{enumerate}[(a)]
    \item falsas esperanzas, 
    \item medición errónea, 
    \item distribución concentrada y disipación de rentas, y 
    \item retrasos en la implementación y la reestructuración.
\end{enumerate}

    \subsection{Falsas esperanzas}
    la IA, quizás la tecnología más prometedora de nuestra era, está muy por detrás de la predicción de Marvin Minsky de 1967 de que Dentro de una generación, el problema de crear inteligencia artificial estará sustancialmente resuelto (Minsky 1967, 2).\\
    si bien reconocemos el potencial de exceso de optimismo, y la experiencia con las primeras predicciones para la IA nos recuerda especialmente que debemos ser algo circunspectos.

    \subsection{Medición errónea}
    En este caso, es la lectura pesimista del pasado empírico, no el optimismo sobre el futuro, lo que está equivocado. De hecho, esta explicación implica que los beneficios de productividad de la nueva ola de tecnologías ya se están disfrutando, pero aún no se han medido con precisión. Bajo esta explicación, la desaceleración de la última década es ilusoria. Esta hipótesis de medición incorrecta” se ha presentado en varios trabajos por ejemplo: (Mokyr 2014; Alloway 2015; Feldstein 2015; Hatzius y Dawsey 2015; Smith 2015).\\
    Sin embargo, un conjunto de estudios recientes proporciona buenas razones para pensar que la medición incorrecta no es la explicación completa, ni siquiera sustancial, de la desaceleración. Cardarelli y Lusinyan (2015), Byrne, Fernald y Reinsdorf (2016), Nakamura y Soloveichik (2015) y Syverson (2017), cada uno utilizando diferentes metodologías y datos, presentan evidencia de que la medición incorrecta no es la principal explicación de la desaceleración de la productividad.

    \subsection{Distribución concentrada y disipación de rentas}
    Una versión afirma que los beneficios de las nuevas tecnologías están siendo disfrutados por una fracción relativamente pequeña de la economía, pero la naturaleza rival y de alcance limitado de las tecnologías crea actividades derrochadoras del tipo de la "fiebre del oro". Tanto aquellos que buscan ser uno de los pocos beneficiarios, como aquellos que han obtenido algunas ganancias y buscan bloquear el acceso a otras, se involucran en estos esfuerzos disipativos, destruyendo muchos de los beneficios de las nuevas tecnologías. . Los efectos agregados de la concentración de la industria todavía están en debate, y el mero hecho de que las ganancias de una tecnología no se distribuyan de manera uniforme no garantiza que los recursos se disipen al tratar de capturarlos

    \subsection{Retrasos en la implementación y la reestructuración}
    Lleva un tiempo considerable que una nueva tecnología sea apreciado. Esto es especialmente cierto para aquellas nuevas tecnologías importantes que, en última instancia, tienen un efecto importante en las estadísticas agregadas y el bienestar. Es decir, aquellas con un potencial de aplicación tan amplio que califican como tecnologías de propósito general (GPT). De hecho, cuanto más profunda y de mayor alcance sea la reestructuración potencial, mayor será el tiempo que transcurra entre la invención inicial de la tecnología y su pleno impacto en la economía y la sociedad.\\
    No es hasta que se construye un stock suficiente de la nueva tecnología y ocurre la invención necesaria de procesos y activos complementarios que la promesa de la tecnología realmente florece en datos económicos agregados.\\
    Hay dos fuentes principales de la demora entre el reconocimiento del potencial de una nueva tecnología y sus efectos medibles. Una es que se necesita tiempo para construir el stock de la nueva tecnología a un tamaño suficiente para tener un efecto agregado. La otra es que se necesitan inversiones complementarias para obtener el máximo beneficio de la nueva tecnología, y lleva tiempo descubrir y desarrollar estos complementos e implementarlos.

\section{El argumento a favor de los retrasos en la implementación y la reestructuración}
Comenzamos estableciendo uno de los elementos más básicos de la historia: que el crecimiento lento de la productividad hoy no descarta un crecimiento más rápido de la productividad en el futuro. A pesar del crecimiento constante, históricamente las tasas de crecimiento de la productividad pasadas han sido malos predictores del crecimiento de la productividad futura. En otras palabras, el crecimiento de la productividad de la última década nos dice poco sobre el crecimiento de la productividad para la próxima década. El crecimiento de la productividad de la década anterior no tiene un poder predictivo estadísticamente discernible sobre el crecimiento de la próxima década. El viejo adagio de que “el desempeño pasado no predice los resultados futuros” se aplica bien al tratar de predecir el crecimiento de la productividad en los años venideros.

\section{Un caso impulsado por la tecnología para el optimismo de la productividad}
En lugar de depender únicamente de las estadísticas de productividad pasadas, podemos considerar el entorno tecnológico y de innovación que esperamos ver en el futuro cercano. En particular, necesitamos estudiar y comprender las tecnologías específicas que realmente existen y hacer una evaluación de su potencial.\\
Más allá del ahorro de mano de obra, los avances en IA tienen el potencial de impulsar la productividad total de los factores. En particular, la eficiencia energética y el uso de materiales podrían mejorarse en muchas plantas industriales a gran escala.\\
James Manyika et al. (2017) analizó 2000 tareas y estimó que alrededor del $45\%$ de las actividades por las que se paga a las personas en la economía de los EE. UU. podrían automatizarse utilizando los niveles existentes de IA y otras tecnologías.

\section{La inteligencia artificial es una tecnología de propósito general}
Bresnahan y Trajtenberg (1996) argumentan que una GPT debe ser generalizada, debe poder mejorarse con el tiempo y generar innovaciones complementarias.\\
La máquina de vapor, la electricidad, el motor de combustión interna y las computadoras son ejemplos de importantes tecnologías de propósito general. Cada uno de ellos aumentó la productividad no solo directamente, sino también al impulsar importantes innovaciones complementarias.\\
La inteligencia artificial, y en particular el aprendizaje automático, ciertamente tiene el potencial de ser omnipresente, mejorarse con el tiempo y generar innovaciones complementarias, lo que la convierte en una candidata para un GPT importante.\\
Los sistemas de aprendizaje automático también están diseñados para mejorar con el tiempo. De hecho, lo que las distingue de tecnologías anteriores es que están diseñadas para mejorar ellos mismos en un tiempo extraordinario. En lugar de requerir que un inventor o desarrollador codifique cada paso de un proceso para ser automatizado, un algoritmo de aprendizaje automático puede descubrir por sí mismo una función que conecta un conjunto de entradas $X$ a un conjunto de salidas $Y$ siempre que se le dé un conjunto suficientemente grande de ejemplos etiquetados que mapeen algunas de las entradas a las salidas (Brynjolfsson y Mitchell 2017). Las mejoras reflejan no solo el descubrimiento de nuevos algoritmos y técnicas, particularmente para redes neuronales profundas, sino también sus complementariedades con hardware informático mucho más potente y la disponibilidad de conjuntos de datos digitales mucho más grandes que se pueden utilizar para entrenar los sistemas (Brynjolfsson y McAfee 2017).\\
Gill Pratt señaló que las máquinas tienen una nueva capacidad que ninguna especie biológica tiene: la capacidad de compartir conocimientos y habilidades casi instantáneamente con otros. Específicamente, el auge de la computación en la nube ha facilitado significativamente la ampliación de nuevas ideas a un costo mucho menor que antes. Este es un desarrollo especialmente importante para promover el impacto económico del aprendizaje automático porque permite la robótica en la nube: el intercambio de conocimientos entre robots. Una vez que una máquina aprende una nueva habilidad en un lugar, se puede replicar en otras máquinas a través de redes digitales. Los datos y las habilidades se pueden compartir, lo que aumenta la cantidad de datos que cualquier aprendiz automático puede usar.\\

\section{Por qué el progreso tecnológico futuro es coherente con el bajo crecimiento actual de la productividad}
Al igual que otros GPT, la IA tiene el potencial de ser un importante impulsor de la productividad. Sin embargo, como señalan Jovanovic y Rousseau (2005) (con referencia adicional al ejemplo histórico de David [1991]), un GPT no genera ganancias de productividad inmediatamente después de su llegada (1184). La tecnología puede estar presente y lo suficientemente desarrollada como para permitir cierta noción de sus efectos transformadores, aunque no esté afectando los niveles de productividad actuales de manera notable. Este es precisamente el estado en el que argumentamos que la economía puede estar ahora. Discutimos anteriormente que un GPT puede estar presente en un momento y, sin embargo, no afectar el crecimiento de la productividad actual si existe la necesidad de construir un stock suficientemente grande del nuevo capital, o si los tipos complementarios de capital, tanto tangible como intangible, necesitan ser identificado, producido y puesto en marcha para aprovechar al máximo los beneficios de productividad de GPT.\\
El tiempo necesario para construir un stock de capital suficiente puede ser extenso. Por ejemplo, no fue sino hasta fines de la década de 1980, más de veinticinco años después de la invención del circuito integrado, que el stock de capital de computadoras alcanzó su punto máximo a largo plazo en alrededor del 5 por ciento (al costo histórico) del capital total de equipos no residenciales.\\
David (1991) observa un fenómeno similar en la difusión de la electrificación. Al menos la mitad de los establecimientos manufactureros de EE. UU. permanecieron sin electricidad hasta 1919, unos treinta años después de que comenzara el cambio a la corriente alterna polifásica. Inicialmente, la adopción fue impulsada por simples ahorros de costos en la producción.\\
Este enfoque para organizar las fábricas es obvio en retrospectiva, sin embargo, se necesitaron hasta treinta años para que se adoptara ampliamente. ¿Por qué? Como señaló Henderson (1993, 2006), es exactamente porque los titulares están diseñados en torno a las formas actuales de hacer las cosas y son tan competentes en ellas que están ciegos o son incapaces de absorber los nuevos enfoques y quedan atrapados en el statu quo: sufren la maldición del conocimiento. \\
 Bresnahan, Brynjolfsson y Hitt (2002) encuentran evidencia de complementariedades triples entre TI, capital humano y cambios organizacionales en las decisiones de inversión y los niveles de productividad. Además, Brynjolfsson, Hitt y Yang (2002) muestran que cada dólar de acciones de capital de TI se correlaciona con alrededor de $\$10$ de valor de mercado. Interpretan esto como evidencia de importantes activos intangibles relacionados con TI y muestran que las empresas que combinan inversiones en TI con un  conjunto específico de prácticas organizacionales no solo son más productivas, sino que también tienen valores de mercado desproporcionadamente más altos que las empresas que invierten solo en uno u otro.\\
Las empresas que intentan transformarse a menudo deben reevaluar y reconfigurar no solo sus procesos internos, sino también sus cadenas de suministro y distribución. Estos cambios pueden tomar tiempo, pero los gerentes y empresarios dirigirán la invención de manera que se ahorren los insumos más costosos (Acemoglu y Restrepo 2017). De acuerdo con el principio de LeChatelier (Milgrom y Roberts 1996), las elasticidades tenderán a ser mayores a largo plazo que a corto plazo a medida que se ajusten los factores cuasifijos.\\
Solo muy recientemente el comercio electrónico se convirtió en una fuerza a tener en cuenta para los minoristas generales. ¿Por qué tomó tanto tiempo? Brynjolfsson y Smith (2000) documentan las dificultades que tenían los minoristas establecidos para adaptar sus procesos comerciales para aprovechar al máximo Internet y el comercio electrónico. Se requirieron muchas inversiones complementarias. El sector en su conjunto requería la construcción de una infraestructura de distribución completa. Los clientes tenían que ser "reentrenados". “Nada de esto podría suceder rápidamente. El potencial del comercio electrónico para revolucionar el comercio minorista fue ampliamente reconocido e incluso promocionado a fines de la década de 1990, pero su participación real en el comercio minorista fue minúscula, 0,2 por ciento de todas las ventas minoristas en 1999. El comercio electrónico comienza a acercarse al 10 por ciento de las ventas minoristas totales y empresas como Amazon están teniendo un efecto de primer orden en las ventas y valoraciones bursátiles de los minoristas más tradicionales.\\

\section{Visualización de la paradoja actual a través de la lente de las tecnologías de uso general anteriores}



