\chapter{Los elementos tecnológicos de la inteligencia artificial}

\section{Introducción}
En este capítulo, definiremos un marco para pensar en los ingredientes de esta nueva IA impulsada por ML. Tener una comprensión de las piezas que componen estos sistemas y cómo encajan entre sí es importante para aquellos que construirán negocios en torno a esta tecnología.

\section{Que es IA}
Necesita un banco masivo de datos para poner el sistema en funcionamiento y una estrategia para continuar generando datos para que el sistema pueda responder y aprender. Y finalmente, necesita rutinas de aprendizaje automático que puedan detectar patrones y hacer predicciones a partir de los datos no estructurados.\\
El aprendizaje automático puede hacer cosas fantásticas, pero básicamente se limita a predecir un futuro que se parece principalmente al pasado. Estas son herramientas para el reconocimiento de patrones. Por el contrario, un sistema de IA puede resolver problemas complejos que antes estaban reservados para los humanos.\\
La inteligencia artificial utiliza instancias de aprendizaje automático como componentes del sistema más grande. Estas instancias de ML deben organizarse dentro de una estructura definida por el conocimiento del dominio, y deben recibir datos que les ayuden a completar las tareas de predicción asignadas.\\
La evolución de ML hacia el estatus de tecnología de propósito general es el principal impulsor del actual auge de la IA. Sin embargo, los algoritmos de ML son componentes básicos de la IA dentro de un contexto más amplio. El punto clave aquí es que, si bien las tareas de ML compuestas pueden atacarse con DNN relativamente genéricos, el sistema combinado completo se construye de una manera altamente especializada para la estructura del problema en cuestión.\\
Para lidiar con el mundo real, debe tener una teoría sobre las reglas del juego relevante. Por ejemplo, si desea crear un sistema que pueda comunicarse con los clientes, puede proceder mapeando los deseos e intenciones de los clientes de tal manera que permita diferentes rutinas de aprendizaje automático que generen diálogos para cada uno. O bien, para cualquier sistema de IA que se ocupe del marketing y los precios en un entorno minorista, debe poder utilizar la estructura de un sistema de demanda económica para pronosticar cómo cambiará el precio de un solo artículo (lo que podría, digamos, ser el trabajo). De un solo DNN) afectará los precios óptimos para otros productos y el comportamiento de sus consumidores (quienes podrían ser modelados con DNN).\\
Como detallaremos a continuación, el aprendizaje automático en su forma actual se ha convertido en una tecnología de propósito general (Bresnahan 2010). Estas herramientas se volverán más baratas y rápidas con el tiempo, debido a las innovaciones en el propio ML. Quienes tengan la experiencia que pueda descomponer complejos problemas empresariales humanos en componentes que se puedan resolver con ML tendrán éxito en la construcción de la próxima generación de inteligencia artificial empresarial, que puede hacer más que solo jugar juegos.\\
En muchos de estos escenarios, las ciencias sociales tendrán un papel que desempeñar. La ciencia se trata de poner estructura y teoría en torno a fenómenos que son increíblemente complejos desde el punto de vista de la observación. A menudo se confiará en la economía, como la ciencia social más cercana a los negocios, para proporcionar las reglas para la IA empresarial. Y dado que la IA impulsada por ML se basa en la medición de recompensas y parámetros dentro de su contexto, la econometría desempeñará un papel clave en el puente entre el sistema supuesto y las señales de datos utilizadas para la retroalimentación y el aprendizaje. La obra no se traducirá directamente. Necesitamos construir sistemas que permitan un cierto margen de error en los algoritmos de ML. Esas teorías económicas que se aplican solo a un conjunto muy limitado de condiciones, por ejemplo, en el equilibrio del filo de una navaja, serán demasiado inestables para la IA. Hay un futuro emocionante aquí donde los economistas pueden contribuir a la ingeniería de IA, y tanto la IA como la economía avanzan a medida que aprendemos qué recetas funcionan o no para la IA comercial.\\
Más allá del aprendizaje automático y la estructura del dominio, el tercer pilar de la IA es la generación de datos. Estoy usando el término generación aquí, en lugar de un término más pasivo como colección, para resaltar que los sistemas de IA requieren una estrategia activa para mantener un flujo constante de información nueva y útil que fluye hacia los algoritmos de aprendizaje compuestos. En la mayoría de las aplicaciones de IA habrá dos clases generales de datos.\\
El marco general de los algoritmos de ML que eligen activamente los datos que consumen se conoce como aprendizaje reforzado (RL). Es un aspecto muy importante de la IA impulsada por ML. En algunos escenarios estrechos y altamente estructurados, los investigadores han creado sistemas de aprendizaje de disparo cero en los que la IA puede lograr un alto rendimiento después de comenzar sin ningún dato de entrenamiento estático.\\
Como complemento al trabajo sobre el aprendizaje por refuerzo, hay mucha actividad de investigación en torno a los sistemas de IA que pueden simular datos para que parezcan que provienen de una fuente del mundo real. Esto tiene el potencial de acelerar el entrenamiento del sistema, replicando el éxito que ha tenido el campo con los videojuegos y los juegos de mesa donde la experimentación es prácticamente gratuita (simplemente juegue, nadie pierde dinero ni se lastima). Las redes antagónicas generativas (GAN; Goodfellow et al. 2014) son esquemas en los que una DNN simula datos y otra intenta discernir qué datos son reales y cuáles son simulados.

\section{Aprendizaje automático de uso general}

