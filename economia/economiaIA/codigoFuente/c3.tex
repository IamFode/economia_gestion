\chapter{Predicción, juicio y complejidad. Una teoría de la toma de decisiones y la inteligencia artificial}

\section{Introducción}
La revolución informática ha desdibujado la línea entre la tarea física y mental. Acemoglu y Restrepo (2017) observan que el capital sustituye a la mano de obra en ciertas tareas mientras que al mismo tiempo el progreso tecnológico crea nuevas tareas. Hacen lo que llaman una suposición natural de que solo el trabajo puede realizar las nuevas tareas, ya que son más complejas que las anteriores. Benzell et al. (2015) consideran el impacto del software de manera más explícita. Su entorno tiene dos tipos de mano de obra: alta tecnología (que puede, entre otras cosas, codificar) y baja tecnología (que son empáticos y pueden manejar tareas interpersonales). En este entorno, son los trabajadores de baja tecnología los que no pueden ser reemplazados por máquinas, mientras que los de alta tecnología se emplean inicialmente para crear el código que eventualmente desplazará a los de su clase. Los resultados del modelo dependen, por tanto, de una clase de trabajadores que no puede ser sustituida directamente por capital, pero también de la incapacidad de los propios trabajadores para sustituir entre clases.\\
nuestro enfoque es ahondar en la maleza de lo que está sucediendo actualmente en el campo de la inteligencia artificial (IA). La ola reciente de desarrollos en IA implica avances en el aprendizaje automático. Esos avances permiten una predicción automatizada y barata; es decir, proporcionar un pronóstico (o un pronóstico inmediato) de una variable de interés a partir de los datos disponibles (Agrawal, Gans y Goldfarb 2018b). Hasta ahora, la sustitución entre humanos y máquinas se ha centrado principalmente en consideraciones de costos. ¿Son las máquinas más baratas, más fiables y más escalables (en su forma de software) que los humanos? Este capítulo, sin embargo, considera explícitamente el papel de la predicción en la toma de decisiones y, a partir de ahí, examina las habilidades complementarias que pueden combinarse con la predicción dentro de una tarea.\\
Nuestro enfoque, en este sentido, está en lo que llamamos juicio. Si bien el juicio es un término con un significado amplio, aquí lo usamos para referirnos a una habilidad muy específica. \\
Para intuir la diferencia entre predicción y juicio, considere el ejemplo del fraude con tarjeta de crédito. Un banco observa una transacción con tarjeta de crédito. Esa transacción es legítima o fraudulenta. La decisión es si se aprueba la transacción. Si el banco sabe con certeza que la transacción es legítima, la aprobará. Si el banco sabe con certeza que es fraudulento, rechazará la transacción. ¿Por qué? Porque el banco sabe que la recompensa de aprobar una transacción legítima es mayor que la recompensa de rechazar esa transacción. Las cosas se ponen más interesantes si el banco no está seguro de si la transacción es legítima. La incertidumbre significa que el banco también necesita saber el resultado de rechazar una transacción legítima y de aprobar una transacción fraudulenta. En nuestro modelo, el juicio es el proceso de determinar estos pagos. Es una actividad costosa, en el sentido de que requiere tiempo y esfuerzo. Dado que todos los nuevos desarrollos relacionados con la IA implican hacer que la predicción esté más disponible, nos preguntamos, ¿cómo cambia el juicio y su aplicación endógena el valor de la predicción? ¿Son la predicción y el juicio sustitutos o complementos? ¿Cómo cambia monótonamente el valor de la predicción con la dificultad de aplicar el juicio? En entornos complejos (en lo que respecta a la automatización, la contratación y los límites de la empresa), ¿cómo afectan las mejoras en la predicción al valor del juicio?\\
 
\section{AI y el costo de predicción}
