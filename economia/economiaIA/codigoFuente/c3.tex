\chapter{Predicción, juicio y complejidad. Una teoría de la toma de decisiones y la inteligencia artificial}

\section{Introducción}
La revolución informática ha desdibujado la línea entre la tarea física y mental. Acemoglu y Restrepo (2017) observan que el capital sustituye a la mano de obra en ciertas tareas mientras que al mismo tiempo el progreso tecnológico crea nuevas tareas. Hacen lo que llaman una suposición natural de que solo el trabajo puede realizar las nuevas tareas, ya que son más complejas que las anteriores. Benzell et al. (2015) consideran el impacto del software de manera más explícita. Su entorno tiene dos tipos de mano de obra: alta tecnología (que puede, entre otras cosas, codificar) y baja tecnología (que son empáticos y pueden manejar tareas interpersonales). En este entorno, son los trabajadores de baja tecnología los que no pueden ser reemplazados por máquinas, mientras que los de alta tecnología se emplean inicialmente para crear el código que eventualmente desplazará a los de su clase. Los resultados del modelo dependen, por tanto, de una clase de trabajadores que no puede ser sustituida directamente por capital, pero también de la incapacidad de los propios trabajadores para sustituir entre clases.\\
nuestro enfoque es ahondar en la maleza de lo que está sucediendo actualmente en el campo de la inteligencia artificial (IA). La ola reciente de desarrollos en IA implica avances en el aprendizaje automático. Esos avances permiten una predicción automatizada y barata; es decir, proporcionar un pronóstico (o un pronóstico inmediato) de una variable de interés a partir de los datos disponibles (Agrawal, Gans y Goldfarb 2018b). Hasta ahora, la sustitución entre humanos y máquinas se ha centrado principalmente en consideraciones de costos. ¿Son las máquinas más baratas, más fiables y más escalables (en su forma de software) que los humanos? Este capítulo, sin embargo, considera explícitamente el papel de la predicción en la toma de decisiones y, a partir de ahí, examina las habilidades complementarias que pueden combinarse con la predicción dentro de una tarea.\\
Nuestro enfoque, en este sentido, está en lo que llamamos juicio. Si bien el juicio es un término con un significado amplio, aquí lo usamos para referirnos a una habilidad muy específica. \\
Para intuir la diferencia entre predicción y juicio, considere el ejemplo del fraude con tarjeta de crédito. Un banco observa una transacción con tarjeta de crédito. Esa transacción es legítima o fraudulenta. La decisión es si se aprueba la transacción. Si el banco sabe con certeza que la transacción es legítima, la aprobará. Si el banco sabe con certeza que es fraudulento, rechazará la transacción. ¿Por qué? Porque el banco sabe que la recompensa de aprobar una transacción legítima es mayor que la recompensa de rechazar esa transacción. Las cosas se ponen más interesantes si el banco no está seguro de si la transacción es legítima. La incertidumbre significa que el banco también necesita saber el resultado de rechazar una transacción legítima y de aprobar una transacción fraudulenta. En nuestro modelo, el juicio es el proceso de determinar estos pagos. Es una actividad costosa, en el sentido de que requiere tiempo y esfuerzo. Dado que todos los nuevos desarrollos relacionados con la IA implican hacer que la predicción esté más disponible, nos preguntamos, ¿cómo cambia el juicio y su aplicación endógena el valor de la predicción? ¿Son la predicción y el juicio sustitutos o complementos? ¿Cómo cambia monótonamente el valor de la predicción con la dificultad de aplicar el juicio? En entornos complejos (en lo que respecta a la automatización, la contratación y los límites de la empresa), ¿cómo afectan las mejoras en la predicción al valor del juicio?.
 
\section{AI y el costo de predicción}
En términos más generales, definimos la predicción como la capacidad de tomar información conocida para generar nueva información. \\
El libro de texto Introducción al aprendizaje automático de Alpaydin (2010) cubre la máxima estimación, estimación bayesiana, regresión lineal multivariada, análisis de componentes principales, agrupamiento y regresión no paramétrica. Además, cubre herramientas que pueden ser menos familiares, pero que también usan variables independientes para predecir resultados: árboles de regresión, redes neuronales, modelos ocultos de Markov y aprendizaje por refuerzo. Hastie, Tibshirani y Friedman (2009) tratan temas similares. El simposio sobre macrodatos del Journal of Economic Perspectives de 2014 abarcó varias de estas técnicas de predicción menos conocidas en artículos de Varian (2014) y Belloni, Chernozhukov y Hansen (2014).\\
Si bien muchas de estas técnicas de predicción no son nuevas, los avances recientes en la velocidad de la computadora, la recopilación y el almacenamiento de datos y los métodos de predicción en sí mismos han llevado a mejoras sustanciales. Estas mejoras han transformado el campo de investigación informática de la inteligencia artificial. El Oxford English Dictionary define la inteligencia artificial como: la teoría y el desarrollo de sistemas informáticos capaces de realizar tareas que normalmente requieren inteligencia humana. En las décadas de 1960 y 1970, la investigación de la inteligencia artificial se basaba principalmente en una lógica simbólica basada en reglas. Involucró a expertos humanos que generaron reglas que un algoritmo podría seguir (Domingos 2015, 89). Estas no son tecnologías de predicción. Dichos sistemas se convirtieron en muy buenos jugadores de ajedrez y guiaron a los robots de fábrica en entornos altamente controlados; sin embargo, en la década de 1980 quedó claro que los sistemas basados en reglas no podían manejar la complejidad de muchos escenarios no artificiales. Esto condujo a un invierno de IA en el que la investigación que financiaba proyectos de inteligencia artificial se secó en gran medida (Markov 2015).\\
El conjunto de tecnologías que ha dado lugar al reciente resurgimiento del interés por la inteligencia artificial utiliza datos recopilados de sensores, imágenes, videos, notas escritas o cualquier otra cosa que pueda representarse en bits para completar la información que falta.\\
Nuestro punto es que las tecnologías a las que se les ha dado la etiqueta de inteligencia artificial son tecnologías de predicción. Por lo tanto, para comprender el impacto de estas tecnologías, es importante evaluar el impacto de la predicción en las decisiones.

\section{Caso: Radiología}
El juicio implica la comprensión de la recompensa s. ¿Cuál es el beneficio de realizar una biopsia si la masa es benigna, maligna o no es real? ¿Cuál es la recompensa por no hacer nada en esos tres estados? El problema para los radiólogos en particular es si un radiólogo especialista capacitado está en la mejor posición para hacer este juicio o ocurrirá más adelante en la cadena de toma de decisiones o involucrará nuevas clases de trabajo que fusionen información de diagnóstico, como un radiólogo/patólogo combinado (Jha y Topol 2016).

\section{Modelo de referencia}
Nuestro modelo de línea de base está inspirado en el entorno “bandido” considerado por Bolton y Faure-Grimaud (2009), aunque se aparta significativamente en las preguntas abordadas y los supuestos básicos realizados. Como ellos, en nuestro modelo base, suponemos que hay dos estados del mundo, $\left\{ \theta_1, \theta_2\right\}$ con probabilidades previas de $\left\{ \mu,1 - \mu\right\}$. Hay dos acciones posibles: una acción independiente del estado con pago conocido de $S$ (segura) y una acción dependiente del estado con dos pagos posibles , $R$ o $r$, según sea el caso (arriesgada).\\
Como se señaló en la introducción, una desviación clave de los supuestos habituales de la toma racional de decisiones es que el tomador de decisiones no conoce el beneficio de la acción riesgosa en cada estado y debe aplicar el juicio para determinar ese beneficio. Además, los responsables de la toma de decisiones necesitan ser capaz de hacer un juicio para cada estado que pueda surgir con el fin de formular un plan que sería el equivalente a la maximización de pagos. En ausencia de dicho juicio, la expectativa  antes de que la acción riesgosa sea óptima en cualquier estado es $v$ (que es independiente entre estados). Para hacer las cosas más concretas, asumimos que $R > S > r$.  Por lo tanto, suponemos que $v$ es la probabilidad en cualquier estado de que el pago arriesgado sea $R$ en lugar de $r$. Esta no es una probabilidad condicional del estado. Es una declaración sobre el pago, dado el estado.\\
En ausencia de conocimiento sobre los beneficios específi cos de la acción arriesgada, solo se puede tomar una decisión sobre la base de probabilidades previas. Entonces se elegirá la acción segura si
$$\mu\left[vR+(1-v)r\right]+(1-\mu)\left[vR+(1-v)r\right]=vR+(1-v)r\leq S.$$
De modo que el pago es: $V_o=max\left\{vR+(1-v)r,S\right\}$. Para simplificar las cosas, centraremos nuestra atención en el caso en que la acción segura es, en ausencia de predicción o juicio, el incumplimiento. Es decir, suponemos que
$$(A1)\qquad \qquad \mbox{ (seguro predeterminado) } vR+(1-v)r\leq S.$$
Esta suposición se hace solo por simplicidad y no cambiará las conclusiones cualitativas.7 Bajo (A1), en ausencia de conocimiento de la función de pago o una señal del estado, el tomador de decisiones elegiría $S$.

\subsection{Juicio en ausencia de predicción}
