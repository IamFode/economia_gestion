\chapter{Una primera mirada a la geografía, al comercio y al desarrollo}


\begin{multicols}{2}
\section{Introducción}

    \begin{center}
	¿Por qué la población mundial es desigualmente distribuida?
    \end{center}

    Existen varias razones:\\

    \begin{itemize}
	\item Sociológico: Nos gusta interactuar con otros humanos.
	\item Sicológico: Tenemos temor a estar solos.
	\item Histórico: Tu abuelo vivió donde vives. 
	\item Cultural: El lugar no se compara con ningún lugar en el mundo.
	\item Geográfico: El lugar es impresionante y la playa es increíble.\\
    \end{itemize}

    Enfocaremos la atención en la lógica económica detrás de la agrupación, conocida técnicamente como aglomeración.\\\\

\section{Agrupamiento y la economía mundial}
    La principal razón para observar estos diferentes niveles de agregación es que, al comprender la agrupación, la economía geográfica muestra que, en gran medida, las mismas fuerzas básicas se aplican a todos los niveles de agregación.

    \subsection{La mirada global}
    El Banco Mundial agrega los datos de los países a siete grupos como sigue:

    \begin{enumerate}[\bfseries (i)]

	\item Este de Asia y del pacifico. (EAP; incluido China e Indonesia)
	\item Este de Europa y Asia centra. (ECA; Incluyendo Rusia y Turquia)
	\item Latinoamerica y el Caribe (LAC; Incluyendo Brasil y México)
	\item El Medio Oriente y el Norte de África (MNA; incluyendo Egipto)
	\item Asia del sur (SAS; incluyendo India)
	\item África sub-sahariana (SSA; incluyendo Nigeria y Sud-África) 
	\item Los países de ingresos altos (Altos; incluidos los Estados Unidos, los países de la Unión Europea y Japón).\\

    \end{enumerate}




\end{multicols}
