\chapter{Una primera mirada a la geografía, al comercio y al desarrollo}
\pagenumbering{arabic}

\section{Introducción}

    \begin{center}
	¿Por qué la población mundial es desigualmente distribuida?
    \end{center}

    Existen varias razones:\\

    \begin{itemize}
	\item Sociológico: Nos gusta interactuar con otros humanos.
	\item Sicológico: Tenemos temor a estar solos.
	\item Histórico: Tu abuelo vivió donde vives. 
	\item Cultural: El lugar no se compara con ningún lugar en el mundo.
	\item Geográfico: El lugar es impresionante y la playa es increíble.\\
    \end{itemize}

    Enfocaremos la atención en la lógica económica detrás de la agrupación, conocida técnicamente como aglomeración.\\\\

\section{Agrupamiento y la economía mundial}
    La principal razón para observar estos diferentes niveles de agregación es que, al comprender la agrupación, la economía geográfica muestra que, en gran medida, las mismas fuerzas básicas se aplican a todos los niveles de agregación.

    \subsection{La mirada global}
    El Banco Mundial agrega los datos de los países a siete grupos como sigue:

    \begin{enumerate}[\bfseries (i)]

	\item Este de Asia y del pacifico. (EAP; incluido China e Indonesia)
	\item Este de Europa y Asia centra. (ECA; Incluyendo Rusia y Turquia)
	\item Latinoamerica y el Caribe (LAC; Incluyendo Brasil y México)
	\item El Medio Oriente y el Norte de África (MNA; incluyendo Egipto)
	\item Asia del sur (SAS; incluyendo India)
	\item África sub-sahariana (SSA; incluyendo Nigeria y Sud-África) 
	\item Los países de ingresos altos (Altos; incluidos los Estados Unidos, los países de la Unión Europea y Japón).\\

    \end{enumerate}


\section{Interacción económica}
    La distribución desigual de la actividad económica, y la aparente regularidad en esta distribución, nos tienta a echar un vistazo primero a la estructura de la interacción entre los diferentes centros económicos. Claramente, dicha interacción tiene lugar de muchas maneras diferentes, sobre todo en forma de comercio, ya sea de bienes o servicios, pero también en forma de flujos de capital y mano de obra, o a través de los diversos medios de comunicación modernos, el intercambio de ideas, y la exposición a otras influencias culturales, etc.\\
    La exportación de bienes y servicios de un país a otro implica tiempo y esfuerzo y, por lo tanto, costos.\\\\
    Que puede verse obstaculizada o aliviada por fenómenos geográficos como cadenas montañosas o fácil acceso a buenas vías fluviales, o distancia política, cultural o social, que también requiere tiempo y esfuerzo antes de que uno pueda lograrlo. Participar en negocios internacionales. Usamos el término \textbf{costos de transporte} como una notación abreviada para los dos tipos de distancia descritos anteriormente. \textbf{La presunción es, por supuesto, que a medida que aumentan los costos de transporte se vuelve más difícil intercambiar bienes y servicios entre naciones}. Como un proxy inicial y bastante tosco de los costos de transporte, calculamos la distancia a Alemania para todos los mercados de exportación alemanes. Tomamos las coordenadas del principal centro económico de cada nación como el hipotético centro de actividad económica. Entonces actuamos como si todos los flujos de exportación alemanes fueran simplemente del centro económico alemán al centro económico de cada nación respectiva. Para confirmar la impresión sobre la relación negativa entre la distancia y los flujos comerciales dada, trazamos el logaritmo natural de las exportaciones alemanas contra el logaritmo natural de la distancia a Alemania para 149 socios comerciales alemanes en la figura 1.8a. Esta relación se conoce como la \textbf{ecuación de la gravedad}.
    $$ln(export) = a - b\; ln(distance)$$
   Esta es una primera afirmación de la relación negativa entre la distancia y el flujo de comercio. Obviamente la distancia.\\
   La distancia, obviamente, no es el único determinante de los flujos comerciales. Las exportaciones de Alemania a Italia son mayores que las de Alemania a los Países Bajos, aunque Italia está aproximadamente al doble de distancia utilizando nuestra medida de distancia. Italia, sin embargo, es un país mucho más grande (con una población más alta y un PIB más grande) que los Países Bajos, por lo que la demanda potencial de productos alemanes es, en igualdad de condiciones, mayor en Italia que en los Países Bajos.  Se corrige los flujos de exportación alemanes por este efecto de la demanda, dividiendo las exportaciones por el PIB de un país, y luego retrata nuevamente el impacto de distancia en los flujos comerciales. La relación es claramente más estrecha de esta manera (más cerca de la línea de regresión). Los diversos países de ejemplo (cercanos o con un gran PIB) identificados. (que tienden a estar por encima de la línea de regresión) están todos cerca de la línea de regresión. Una regresión simple que incluye tanto el impacto del ingreso como de la distancia en los flujos de exportación dados:
   $$ln(export) = a + c ln(GDP) - b ln(distancia)$$

\section{Cambio rápido en la distribución de la producción y la población}
    puede surgir la pregunta de si la actividad económica estuvo siempre distribuida de manera desigual. La respuesta es sí y no.\\
    La respuesta es sí en el sentido de que las ciudades, por ejemplo, ya comenzaban a surgir tras la revolución neolítica como consecuencia de un aumento del excedente agrícola. Como tal, las ciudades han dominado en gran medida la cadena de eventos y las decisiones tomadas en muchas áreas diferentes de interés humano a lo largo de la historia.\\\\
    La respuesta es no en el sentido de que la asimetría o "desigualdad" de la distribución de la actividad económica ha cambiado con el tiempo.\\\\
    En una exposición fascinante, amplia y, sin embargo, sorprendentemente detallada, titulada Armas, gérmenes y acero (1997); Jared Diamond explica varios aspectos importantes de las consecuencias de la interacción entre "geografía" y "economía", donde el primer término se refiere a las circunstancias físicas generales (o primera naturaleza) y el segundo término se refiere a las fuerzas asociadas con la interacción humana (o segunda naturaleza).\\
    La distribución de los grandes animales mamíferos es aún más favorable para el continente euroasiático, que pudo domesticar trece especies, en comparación con solo una en las Américas. A pesar de la abundancia de grandes mamíferos en el África subsahariana, ninguno resultó apto para la domesticación.\\
    Otra ventaja importante del continente euroasiático (incluido el norte de África) es su orientación principal de oeste a este en lugar de norte a sur, como es el caso de las Américas y África. Áreas de la misma latitud comparten la misma duración del día y variaciones estacionales, y hasta cierto punto enfermedades, temperaturas y precipitaciones similares. Esto facilitó en gran medida la difusión del conocimiento sobre la producción de alimentos de oeste a este o viceversa, en lugar de norte a sur. Por esa razón, Europa occidental, el valle del Indo y Egipto pudieron disfrutar de los beneficios de las plantas y los animales domesticados por asimilación en lugar de por invención miles de años antes de su invención en las Américas.


