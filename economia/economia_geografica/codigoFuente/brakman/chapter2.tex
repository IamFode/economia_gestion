\chapter{Geografía y teoría económica}

\section{Introdución}

\textbf{La agrupación de actividades económicas se puede encontrar en varios niveles de agregación:} la variación considerable en el tamaño económico de las ciudades o regiones a nivel nacional, o la distribución desigual de la riqueza y la producción a nivel mundial.\\
Surge la pregunta de por qué la ubicación parece ser tan importante para las actividades económicas. Para responder a esta pregunta, necesitamos un marco analítico en el que la geografía juegue un papel de una forma u otra.\\
Las ciudades y regiones varían en tamaño y relevancia. Este es un tema primordial para la economía regional y urbana y también para la geografía económica propiamente dicha, que tradicionalmente se ocupa del análisis teórico de las interdependencias entre ciudades y regiones dentro de un país.\\\\
\begin{center}
    ¿qué tiene que decir cada teoría sobre el papel de la geografía?
\end{center}

\section{La geografía en la economía regional y urbana}
\textbf{La economía regional analiza la dispersión espacial y la coherencia de la actividad económica}
\textbf{La economía regional (también conocida como ciencia regional) se basa en la teoría económica neoclásica y es, en efecto, la sucesora formalizada de la tradición alemana de la economía de ubicación. La geografía económica, por otro lado, es más ecléctica y orientada empíricamente.} Se inspira en teorías económicas heterodoxas y, cada vez más, en áreas externas a la economía, como la sociología, las ciencias políticas y la teoría de la regulación. Comenzamos con una descripción general de un campo de estudio más joven, a saber, \textbf{la economía urbana, que estudia la estructura espacial de las áreas urbanas.} Al igual que la economía regional, la economía urbana se basa en gran medida en las herramientas del análisis neoclásico, de modo que la división entre economía regional y urbana no siempre es clara. 

\subsection{La economía urbana}

\textbf{La distribución desigual de la actividad económica dentro de cada país es el punto de partida de la economía urbana.} El análisis moderno de la aglomeración de empresas y personas en ciudades o áreas metropolitanas se basa en gran medida en \textbf{la economía de la aglomeración, un término que se refiere a la disminución de los costos promedio a medida que se produce una mayor producción dentro de un área geográfica específica. En otras palabras, se basa en rendimientos crecientes a escala.} Antes de entrar en la relevancia de las economías de escala para las ciudades y otras formas de aglomeración, primero analizamos un modelo en el que no hay rendimientos crecientes a escala. Este modelo, el modelo de ciudad monocéntrica. Se justifica una breve discusión, aunque solo sea para poder notar las diferencias con el enfoque de la economía geográfica y dejar claro que, al final, el análisis de las ciudades seguirá siendo bastante limitado mientras no haya rendimientos crecientes a escala.

\subsubsection{El modelo de la ciudad monocéntrica}
El modelo de ciudad monocéntrica asume la existencia de un plano sin rasgos distintivos, perfectamente plano y homogéneo en todos los aspectos. En medio de este plano hay una sola ciudad. Fuera de la ciudad, los agricultores cultivan cultivos que deben vender en la ciudad. Hay costos de transporte positivos asociados con llevar los productos agrícolas a la ciudad, que difieren para los diversos cultivos, al igual que los precios de estos cultivos. Se  analiza cómo los granjeros se ubican a través del plano. \textbf{Cada agricultor quiere estar lo más cerca posible de la ciudad para minimizar sus costos de transporte.} Este incentivo de estar cerca de la ciudad da como resultado mayores rentas de la tierra cerca de la ciudad que en el borde del plano. Por lo tanto, \textbf{cada agricultor se enfrenta a una compensación entre las rentas de la tierra y los costos de transporte.}\\
\textbf{Se demostró que la competencia por las ubicaciones garantiza que la asignación de tierras equilibrada resultante entre los agricultores sea eficiente.} Para cada tipo de cultivo existe una curva de oferta-renta que indica, dependiendo de la distancia a la ciudad, cuánto están dispuestos a pagar los agricultores por la tierra. Dado que las curvas de oferta-renta difieren por cultivo, como resultado de los diferentes precios de esos cultivos en la ciudad y los diferentes costos de transporte, los agricultores de un tipo particular de cultivo pueden superar a sus competidores (es decir, están dispuestos a pagar más) \textbf{para cualquier distancia dada a la ciudad. A medida que nos alejamos del centro de la ciudad,  vemos que, primero, los productores de flores superan la oferta de los otros dos grupos de agricultores, luego que entre los puntos A y B los productores de vegetales están dispuestos a pagar las rentas más altas y que a la derecha del punto B (y por lo tanto el más alejado del centro de la ciudad) los productores de granos pagarán la renta más alta.} Esto da como resultado un patrón de círculos concéntricos de uso de la tierra alrededor de la ciudad, cada anillo consta de granjas que cultivan el mismo cultivo; en secuencia: flores, vegetales y granos.\\
La economía urbana probablemente comenzó como una disciplina separada con William Alonso (1964), quien tomó el modelo de von Thunen y, esencialmente, reemplazó la ciudad por un centro comercial central y los agricultores por viajeros. Los viajeros viajan de ida y vuelta a su trabajo en el centro de negocios, y cada viajero obtiene utilidad de su espacio para vivir, pero también enfrenta costos de transporte. Nuevamente, las rentas de la tierra son las más altas cerca de la ciudad y disminuyen con la distancia. Por lo tanto, se puede aplicar el enfoque de oferta y renta, y la competencia por la tierra entre los viajeros implica una asignación eficiente de la tierra. La eficiencia de la asignación de tierras en el modelo monocéntrico depende del supuesto de que no hay externalidades de ubicación. Combinado con el trabajo de Richard Muth (1969) y Mills (1967), el modelo de Alonso (1964) es sigue siendo la columna vertebral de la economía urbana moderna.\\
Una serie de hechos estilizados sobre la estructura espacial urbana están de acuerdo con el modelo monocéntrico. Primero, la densidad de población disminuye con la distancia de los centros comerciales centrales. En segundo lugar, casi todas las ciudades importantes del mundo occidental se descentralizaron en el siglo XX (a medida que la gente comenzó a ubicarse más lejos del centro de la ciudad), lo que puede estar relacionado con una caída en los costos de transporte. \textbf{El modelo monocéntrico también tiene algunas limitaciones serias. Mencionamos solo dos. Primero, el modelo no tiene en cuenta ninguna interacción entre ciudades; no puede ocuparse de los sistemas urbanos. Segundo, el modelo toma la existencia y ubicación de la ciudad como dadas y se enfoca en la ubicación de agricultores/viajeros fuera de la ciudad.} La pregunta de por qué hay una ciudad para empezar queda sin respuesta. Para hacer frente a estas limitaciones, los economistas urbanos han reconocido durante mucho tiempo que una teoría de las ciudades no puede prescindir de la introducción y el fundamento teórico de algún tipo de rendimientos crecientes a escala. Estos pueden ocurrir a nivel de empresa o a un nivel más agregado (el nivel de la industria o el nivel nacional). 

\paragraph{Economías de escala externas e internas}
\textbf{El término economías de escala o rendimientos crecientes a escala se refiere a una situación en la que un aumento en el nivel de producción implica una disminución en los costos promedio por unidad de producción para la empresa. Se traduce en una curva de costo promedio con pendiente negativa. Para identificar el origen de la caída de los costes medios, se distingue entre economías de escala internas y externas. Con las economías de escala internas, la disminución de los costes medios se produce por un aumento del nivel de producción de la propia empresa.} Cuanto más produce la empresa, mejor puede beneficiarse de las economías de escala y mayor es su ventaja de costos sobre las empresas más pequeñas. \textbf{La estructura de mercado que subyace a las economías de escala internas, típicamente utilizada en la literatura de economía geográfica, debe ser necesariamente de competencia imperfecta, ya que las economías de escala internas implican poder de mercado.} Con \textbf{las economías de escala externas, la disminución de los costes medios se produce a través de un aumento de la producción a nivel de la industria en su conjunto, lo que hace que los costes medios por unidad sean una función de la producción de toda la industria. Scitovsky distingue aquí entre economías externas puras y pecuniarias.}\\
\textbf{Con economías externas puras (o tecnológicas), un aumento en la producción de toda la industria altera la relación tecnológica entre insumos y producción para cada empresa individual. Por lo tanto, tiene un impacto en la función de producción de la empresa.} Un ejemplo de uso frecuente (que se remonta a Alfred Marshall; se refiere a los derrames de información. Un aumento en la producción de la industria aumenta el acervo de conocimiento a través de los efectos indirectos positivos de información para cada empresa, lo que lleva a un aumento en la producción a nivel de empresa. \textbf{En la economía urbana, pero también en la nueva teoría del crecimiento  y la nueva teoría del comercio , se supone que existen economías externas puras. La estructura del mercado puede entonces ser perfectamente competitiva ya que el tamaño de la empresa individual no importa.}\\
\textbf{Las economías externas pecuniarias son transmitidas por el mercado a través de efectos de precio para la empresa individual, lo que puede alterar su decisión de producción.} Dos ejemplos, de nuevo basados en Marshall, son la existencia de un gran mercado local de insumos especializados y la puesta en común del mercado laboral. Una gran industria puede respaldar un mercado de insumos intermedios especializados y un grupo de trabajadores calificados específicos de la industria, lo que beneficia a la empresa individual. \textbf{A diferencia de las economías externas puras, estos efectos indirectos no afectan la relación tecnológica entre insumos y productos (la función de producción). Las externalidades pecuniarias existen en la literatura de economía geográfica a través de un efecto de gusto por la variedad en un gran mercado local. La utilidad de cada consumidor depende positivamente del número de variedades que puede comprar de un bien manufacturado. Los efectos de precio cruciales para las externalidades pecuniarias solo pueden ocurrir con competencia imperfecta. Esto es consistente con el requisito de competencia imperfecta para las economías de escala internas, también utilizado en la literatura de economía geográfica.}\\
Algunas observaciones finales están en orden. Primero, los derrames o externalidades son cruciales para las economías externas. El concepto de derrames a veces se utiliza solo para economías externas puras, refiriéndose a las economías externas pecuniarias como un caso de interdependencia del mercado. Nos ceñimos al uso de derrames o externalidades cuando nos referimos a economías de escala externas en general. De manera similar, el término rendimientos crecientes a veces se usa solo para economías de escala internas. También usamos la frase rendimientos crecientes cuando hablamos de economías externas. Del contexto quedará claro si nos referimos al nivel de la empresa o de la industria.

$$\begin{array}{lll}
    \hline\\
    \mbox{Marshall-Arrow-Romer(MAR) externalidad} & \mbox{Economía de la localización} & \mbox{efectos indirectos específicos (sector)}\\\\
    \hline\\
    \mbox{Externalidades Jacobs}&\mbox{Economía de la urbanización}&\mbox{efectos indirectos específicos (ciudad)}\\\\
    \hline\\
\end{array}$$

\textbf{En segundo lugar, las economías externas pueden aplicarse a un nivel de agregación superior al de la empresa. Este suele ser el nivel de la industria, pero en la teoría moderna del comercio y la teoría moderna del crecimiento también puede ser la economía en su conjunto. Tercero, las economías externas en los modelos son estáticas, mientras que la literatura también considera economías externas dinámicas.} En ese caso, los costos promedio por unidad de producción son una función negativa de la producción acumulada de la industria. Nuevamente, si esto es relevante, quedará claro si nos referimos a economías externas estáticas o dinámicas. \textbf{Cuarto, las economías externas discutidas anteriormente son positivas,  también pueden ser negativos, es decir, un aumento en la producción de una empresa conduce a un aumento en los costos por unidad para otras empresas.}\\
Finalmente, una observación sobre la terminología un tanto confusa con respecto a las externalidades económicas regionales o efectos indirectos. Es habitual distinguir entre las externalidades Marshall-Arrow-Romer (MAR) y las externalidades de Jacobs. En ambos casos, el énfasis está en los efectos indirectos regionales (específicos de la ubicación), es decir, las empresas deben estar ubicadas lo suficientemente cerca unas de otras para beneficiarse de estas externalidades. Las externalidades SAM se centran en los efectos indirectos específicos del sector y también se conocen como economía de localización. Las externalidades de Jacobs se centran en los efectos indirectos específicos de la ciudad que cruzan los límites entre sectores individuales. Estos también se conocen como economía de la urbanización. 


\subsubsection{Economía urbana y rendimientos crecientes}
\textbf{A diferencia del modelo de ciudad monocéntrica, se incluyen rendimientos crecientes a escala. El punto de partida es bastante diferente del modelo monocéntrico. No hay costes de transporte y el interior de una ciudad ya no forma parte del análisis.} En cierto sentido, es un análisis de las ciudades en las que el espacio, es decir, el espacio fuera de las ciudades, no tiene ningún papel que desempeñar. \textbf{La justificación de este descuido geográfico del espacio no urbano es que, en los países industrializados modernos, una gran parte de la actividad económica general y de la población se encuentra en áreas urbanas, de modo que la relevancia de lo urbano frente a lo no urbano Se supone que las transacciones son limitadas.} En cambio, \textbf{el análisis se centra en las fuerzas que determinan el tamaño de las ciudades y las interacciones entre ellas. Las fuerzas de aglomeración en el modelo de Henderson son economías de escala externas positivas que son específicas de la industria. Esto último significa que hay derrames positivos cuando una empresa de una industria en particular se ubica en una ciudad donde se encuentran otras empresas de la misma industria.} Usando una categorización bien conocida que se remonta a los trabajos de Marshall, estos pueden deberse a:
\begin{enumerate}[\bfseries (i)]
    \item El intercambio de información, 
    \item la existencia de una gran cantidad de mano de obra, o
    \item la existencia de especialistas proveedores.
\end{enumerate}
Por lo tanto, las economías externas pueden, en principio, involucrar economías externas puras (como en el enfoque original de Henderson) o economías externas pecuniarias. \\
Las fuerzas de expansión son economías de escala externas negativas dentro de la ciudad, como la congestión, que es una función del tamaño total de la ciudad. Una ciudad grande implica costos de transporte y rentas de la tierra relativamente altos. Las deseconomías de escala no dependen del tipo de producción que se lleva a cabo en la ciudad, por lo que dependen únicamente del tamaño total de una ciudad. Junto con las economías externas específicas de la industria, esto tiene dos implicaciones importantes. En primer lugar, puede racionalizar los sistemas de ciudades (diferentes tamaños de ciudades que satisfacen las necesidades de diferentes industrias). Basado en el supuesto de que los efectos secundarios positivos de la ubicación son específicos de la industria, cada industria tiene su propio tamaño óptimo. Ciudades de diferentes tamaños comercian entre sí. En comparación con el marco de von Thunen o Alonso, la economía urbana moderna es mucho menos ad-hoc porque puede brindar una base teórica para los rendimientos crecientes que impulsan la existencia de las ciudades, y produce conocimientos sobre los sistemas urbanos. 

\subsubsection{¿Qué tipo de economías de escala externas?}
Existe un amplio respaldo empírico para la idea de que los efectos indirectos específicos de la industria son importantes para las ciudades. \textbf{Estas economías externas específicas de la industria se conocen como economías de localización, a diferencia de las economías de urbanización. Las últimas son economías externas que se aplican a empresas de todos los sectores y captan la noción de efectos indirectos positivos para una empresa como resultado de la actividad económica total en una ciudad. Ambos tipos de economías externas a menudo se relacionan con la ubicación de las ciudades en un sentido estático, pero también se aplican en un contexto dinámico (¿cómo se desarrollan las ciudades con el tiempo?).} Con respecto al crecimiento de las ciudades en los Estados Unidos, no encuentran apoyo para la hipótesis de que las ciudades especializadas en ciertas industrias crecen más rápidamente en promedio. En cambio, concluyen que \textbf{si las economías externas son importantes, probablemente sea más importante tener una variedad de industrias diversificadas en una ciudad.} Si este último es el caso, surge la pregunta de por qué tantas ciudades están especializadas en industrias particulares. Se sugieren que tanto las economías de localización como las de urbanización son relevantes (aunque al final favorecen las economías de urbanización), mientras que otros argumentan que en un contexto dinámico las economías de localización son más relevantes.\\
Desde un punto de vista teórico, cabe destacar que el enfoque de los sistemas urbanos de Henderson no da por sentada la existencia de la ciudad, como hacía el modelo monocéntrico. También proporciona una teoría de las interacciones entre ciudades. \textbf{El problema con el enfoque es que el espacio fuera de las ciudades (deliberadamente) no forma parte del análisis. Esto es problemático si uno quiere poder decir dónde están ubicadas las ciudades en relación con otras y la parte no urbana de la geografía:} La literatura sobre sistemas de ciudades ha enfatizado el espacio urbano pero ha descuidado el espacio nacional. Como veremos, \textbf{la ubicación de la actividad manufacturera y la relación entre estas ubicaciones y el resto del espacio es un tema clave en la economía geográfica. Para analizar esta relación, los costos de transporte deben ser parte del análisis, ya que son cruciales para determinar el equilibrio entre las fuerzas de aglomeración y expansión.}\\
En sus estudios de las teorías de la aglomeración, que incluye la economía urbana, \textbf{se analizan tres enfoques básicos: rendimientos crecientes, externalidades y competencia espacial.} Su uso de rendimientos crecientes y externalidades corresponde a nuestra definición de economías externas puras y economías externas pecuniarias, respectivamente. Ambos tipos de economías externas son importantes. Esto deja la competencia espacial, lo que significa que \textbf{la competencia entre empresas es casi automáticamente de naturaleza oligopólica cuando se toma en consideración el espacio.} La competencia está restringida por la distancia; Por lo general, se piensa que una empresa compite solo con sus empresas vecinas. \textbf{La competencia espacial está, por tanto, intrínsecamente ligada al comportamiento estratégico de las empresas. La razón es simplemente que en la economía geográfica y en particular en la versión de competencia monopolística, que caracteriza la estructura del mercado en nuestro modelo central de economía geográfica, el comportamiento estratégico no es tenido en cuenta.} Las empresas toman el comportamiento (fijación de precios) de las demás como dado. Además de los tres enfoques mencionados dan dos razones adicionales para la aglomeración (urbana): la existencia de un espacio no homogéneo y economías de escala internas en un proceso de producción. Con el primero se puede racionalizar la aglomeración sin ninguna forma de rendimientos crecientes a escala (piense en las diferencias en la geografía física real que da lugar, por ejemplo, a un puerto natural y la aglomeración correspondiente).

\subsection{Economía regional}
\textbf{La economía regional analiza la organización espacial de los sistemas económicos (y no solo de las ciudades) y de alguna manera también debe dar cuenta de la distribución desigual en el espacio.} Todas las contribuciones alemanas toman en consideración el espacio nacional o de toda la economía para analizar dónde se ubican las actividades económicas. Esta es una pregunta relevante ya que el movimiento de bienes y personas no es gratuito y la producción suele estar sujeta a alguna forma de rendimientos crecientes. Sin embargo, los padres fundadores de la economía regional se centran en diferentes aspectos de la ubicación de la actividad económica. Como vimos, von Thunen, por ejemplo, enfatizó las decisiones de ubicación tomadas por los agricultores, mientras que Weber analizó la ubicación óptima y el tamaño de la planta para las empresas manufactureras. Esta subsección se centra en las ideas presentadas (y probadas) por primera vez por Christaller y Losch, quienes trataron no solo de explicar la ubicación de las ciudades sino también de diferenciarlas por las diversas funciones que desempeñan y de tratar las relaciones entre las ciudades y el medio ambiente. No ciudades. Este enfoque se conoce como la teoría del lugar central, que muestra que diferentes puntos o ubicaciones en el panorama económico tienen diferentes niveles de centralidad y que los bienes y servicios se proporcionan de manera eficiente sobre una base jerárquica.\\

\subsubsection{Teoría del lugar central}
\textbf{Dada una distribución uniforme de consumidores idénticos en un plano homogéneo, la teoría del lugar central sostiene que las ubicaciones difieren en centralidad y que esta centralidad determina el tipo de bienes que proporciona la ubicación.} La provisión de estos bienes está determinada por rendimientos internos crecientes a escala, mientras que la ubicación es relevante porque los consumidores incurren en costos de transporte. Para minimizar estos costos, los consumidores quieren tener acceso a proveedores de bienes cercanos. Para algunos tipos de bienes, como el pan, esto es más fácil que para otros, como los televisores, porque los rendimientos crecientes a escala son relativamente limitados. Por lo tanto, la economía puede sustentar muchos lugares relativamente pequeños (pueblos) donde los panaderos están activos para suministrar pan. Por el contrario, solo puede haber relativamente pocos lugares (ciudades pequeñas, los lugares centrales) donde las empresas de electrónica vendan televisores, que la gente compra con menos frecuencia. Para minimizar los costos de transporte, ambos tipos de ubicaciones se distribuyen de manera bastante uniforme en el espacio. Además, obtenemos una jerarquía de lugares donde la ciudad realiza todas las funciones (vende pan y televisores), mientras que el pueblo realiza solo algunas funciones (vende solo pan). Donde el lugar central equidistante está rodeado por seis equidistantes ciudades más pequeñas, que juntas forman un hexágono. Cada ciudad pequeña, a su vez, está rodeada por seis aldeas equidistantes.\\
El hecho de que se ocupe explícitamente de la ubicación de la actividad económica es una ventaja importante de la teoría del lugar central. El principal problema con el enfoque es que la lógica económica detrás de las decisiones de los consumidores y las empresas sigue sin estar clara. ¿Qué tipo de comportamiento de los agentes individuales conduce a un resultado de lugar central? Los rendimientos crecientes a nivel de empresa requieren alguna forma de competencia imperfecta, un análisis que falta. En consecuencia, la teoría del lugar central, especialmente la versión gráfica que todavía se encuentra en la mayoría de los libros de texto de introducción a la geografía económica, es más una historia descriptiva que un modelo causal.\\
Por supuesto, los científicos regionales y los geógrafos económicos también han sido conscientes de las limitaciones de esta versión de la teoría del lugar central, que durante los últimos treinta años ha recibido menos interés, particularmente dentro de la geografía económica. Para una base teórica, los geógrafos económicos han comenzado a buscar en otra parte. Y también explicamos por qué los geógrafos económicos modernos son bastante críticos con el trabajo realizado por los economistas geográficos (y viceversa). Sin embargo, siguiendo a Walter Isard (1956, 1960), los científicos regionales han tratado de construir sobre las ideas básicas de la teoría del lugar central para dar una base económica teórica (a menudo altamente formalizada) a esta teoría. Estos modelos son en su mayoría de naturaleza de equilibrio parcial, explicando algunos aspectos del sistema de lugar central mientras ignoran otros. \textbf{Por lo general, un modelo en esta tradición no trata con empresas o consumidores individuales, sino que formaliza esencialmente el patrón geométrico de un sistema de lugar central}. El resultado del lugar central es, por lo tanto, meramente racionalizado y no explicado por el comportamiento subyacente de los consumidores y productores, ni por sus decisiones e interacciones (de mercado). Por ejemplo, la curva de demanda que enfrenta una empresa en un lugar particular no se deriva de los primeros principios, sino que simplemente se supone. \textbf{La economía geográfica intenta llenar este vacío en la literatura dando una base microeconómica a la jerarquía de los lugares centrales.} 

\paragraph{Teoría del lugar central en un pólder holandés}
Entre 1937 y 1942, se recuperó del mar un área de 48 000 hectáreas (120 000 acres) y se convirtió en un pólder en el centro de los Países Bajos. El nuevo pólder, llamado Noord-Oost Polder (Polder del noreste), fue y sigue siendo utilizado principalmente para la agricultura. Las autoridades holandesas también planificaron el establecimiento de una serie de (pequeños) pueblos y aldeas en este pólder, y su planificación estuvo explícitamente influenciada por el trabajo de Christaller y Losch en lugares centrales. El pólder cumplió claramente con algunos de los supuestos de la teoría del lugar central: la tierra es extremadamente plana y casi perfectamente homogénea en todos los demás aspectos. Los colonos iniciales en el pólder (los granjeros) también estaban distribuidos uniformemente por todo el pólder, y no era demasiado descabellado suponer que estos granjeros tenían preferencias idénticas. El desembolso de las nuevas ubicaciones en el pólder, se parecía mucho al desembolso del lugar central. Había un lugar central (la ciudad de Emmeloord), que quedó rodeada durante un período de diez años por varios lugares más pequeños (casi) equidistantes. Estas ubicaciones más pequeñas se diseñaron explícitamente para suministrar solo bienes de orden inferior, mientras que Emmeloord se diseñó para suministrar  bienes de orden superior (para el desembolso de los distintos lugares en el pólder). Con base en esta idea central de la teoría del lugar central, las autoridades hicieron proyecciones sobre el tamaño de cada lugar. Estas proyecciones se eliminaron de la realidad después de sesenta años. El lugar central se había vuelto mucho más grande de lo previsto, algunas ubicaciones son casi correctas y algunas ubicaciones se han vuelto más pequeñas de lo esperado. 

\subsubsection{Potencial de mercado}
\textbf{Las observaciones anteriores sobre el fundamento de la teoría del lugar central se aplican de manera más general. Hay más ejemplos de teorías que, como la teoría del lugar central, intentan enfrentarse a una regularidad espacial pero que carecen de un fundamento económico-teórico convincente.} A diferencia de la teoría económica (neoclásica), existe una tendencia a dar simplemente una representación, utilizando, por ejemplo, ecuaciones simples, de la regularidad sin conexión con un modelo del comportamiento individual subyacente de los agentes económicos. Otros ejemplos de modelos utilizados para describir o imitan regularidades espaciales empíricas particulares son (i) las ecuaciones subyacentes a la distribución de rango-tamaño , (ii) el modelo de gravedad del comercio , y (iii) \textbf{el análisis del potencial de mercado . Este último, debido a Chauncy Harris (1954), es ampliamente utilizado en economía regional. Para el caso de los Estados Unidos, y utilizando el valor de las ventas minoristas por condado de los EE. UU., Harris encuentra que el potencial de mercado de cualquier ubicación se puede describir mediante}

\begin{equation}
    MP_i = \sum_{j=1}^n \dfrac{M_j}{D_{ij}}
\end{equation}

Donde $MP_i$ es el mercado potencial de la ubicación $i$, $M_j$ es la demanda de la ubicación $j$ por los bienes de la ubicación $i$, y $D_{ij}$ es la distancia entre las ubicaciones $i$ y $j$.\\
Por lo tanto, la ecuación del potencial de mercado proporciona una indicación de la proximidad general de una ubicación en relación con la demanda total. Harris (1954), y muchos economistas regionales desde entonces, \textbf{han descubierto que el potencial de mercado (y por lo tanto la demanda) suele ser alto en aquellas áreas donde también se encuentra realmente la producción. Esto respalda la noción de agrupamiento de la actividad económica e indica que las decisiones de aglomeración y ubicación en las que se basa no son solo un problema del lado de la oferta, sino que la demanda también juega su papel. De hecho, la idea de que la producción tiene lugar donde la demanda es alta también puede revertirse. La demanda es alta donde se ubica la producción como resultado del poder adquisitivo de los trabajadores que hacen posible la producción en ese lugar.} Aunque convincente desde un punto de vista empírico, \textbf{el análisis del potencial de mercado carece de una base teórica (y por lo tanto también carece de contenido: ¿qué representa MPi?). Esto no es muy sorprendente, porque la teoría del crecimiento y el comercio económico tienen grandes dificultades para explicar cualquier fenómeno en el que la geografía juega un papel.} En particular, la variable de distancia $D_{ij}$ es difícil de reconciliar con la teoría económica. \textbf{Sin embargo, las ideas detrás del análisis del potencial de mercado juegan un papel destacado en la economía geográfica, el modelo central de la economía geográfica puede interpretarse como un intento de proporcionar una base teórica para la ecuación (2.1). El trabajo empírico en economía geográfica también utiliza el enfoque del mercado potencial}.\\
Ambos ejemplos (la teoría del lugar central y el enfoque del potencial de mercado) ilustran algunos puntos que se aplican de manera más general. Los enfoques teóricos, como la teoría del lugar central, y los enfoques de inspiración más empírica, como el análisis del potencial de mercado, tratan aspectos importantes del espacio espacial. Organización de la actividad económica. Sin embargo, el marco de análisis en general no cumple con los estándares de la teoría económica dominante, que requiere que las conclusiones se basen en las acciones e interacciones de los agentes económicos individuales en el mercado. Esto requiere el análisis del comportamiento individual del consumidor y del productor, la estructura del mercado y los equilibrios resultantes. Tal fundamento microeconómico de la geografía no existe (o simplemente no se consideró útil para empezar) en algunos rincones de la economía regional.  

\subsection{¿Es nueva la economía geográfica? La mirada desde la economía urbana y regional}
\textbf{La economía geográfica puede verse como una nueva geografía económica en la medida en que combina percepciones espaciales bien establecidas de la economía regional y urbana con el marco de equilibrio general de la teoría económica dominante.} Intenta no sólo poner más teoría económica en geografía pero, sobre todo, más geografía en la economía convencional, mientras se utiliza el conjunto de herramientas de la economía convencional. Si este intento produce nuevos conocimientos sobre las relaciones entre la geografía y la economía o solo fundamenta el trabajo existente (y, para algunos geógrafos económicos, obsoleto) en un marco analítico diferente, es una cuestión diferente. Antes de volver a la corriente principal de la teoría económica del comercio y el crecimiento y su análisis (o falta de él) de la geografía en el resto de este capítulo, es útil hacer las siguientes dos observaciones sobre el legado de la economía urbana y regional para la economía geográfica y su intento de fundamentar el análisis de la distribución de la actividad económica a través del espacio en un marco de equilibrio general.\\
\textbf{La primera observación es que el marco estándar de equilibrio general no funcionará si queremos que importen la geografía o el espacio. La existencia de costos positivos de transporte o comercio dará como resultado un equilibrio en el que no habrá transporte ni comercio de bienes.}\\
Como veremos en la siguiente sección, la opción de la no homogeneidad del espacio es la ruta tradicionalmente tomada por la teoría del comercio internacional. Las teorías sobre economía urbana y regional discutidas en la presente sección toman la segunda ruta. Con von Thunen, la fricción o la no divisibilidad requerida para llegar a un resultado de equilibrio significativo en un mundo de costos de transporte positivos es la suposición de que la producción debe venderse en el mercado central. En la economía urbana y regional moderna, suele ser la introducción de alguna forma de rendimientos crecientes la que desempeña este papel. Fujita y Thisse se refieren a \textbf{las fuerzas económicas rendimientos crecientes: costos de transporte como la compensación fundamental para una economía espacial. Como veremos en los capítulos 3 y 4, esto también es válido para la economía geográfica.} Con rendimientos crecientes pero sin costos de transporte, a las empresas les puede resultar rentable producir en una sola planta, pero no les importa la ubicación de esta planta. Como argumentamos anteriormente, en ausencia de rendimientos crecientes (o espacio no homogéneo) pero con costos de transporte, terminaremos en una situación de capitalismo de traspatio.\\
\textbf{La segunda observación relacionada es que, debido a que resultó difícil combinar los costos de transporte, los rendimientos crecientes a escala y la competencia imperfecta en un marco de equilibrio general, tomó bastante tiempo desde Koopmans (1957), pasando por Starrett (1978) ) para llegar al primer modelo de economía geográfica, de Krugman (1991a).} Dicho esto, también está claro que muchos de los ingredientes de la economía geográfica no son nuevos y ya eran bien conocidos en 1991, cuando Krugman publicó su modelo en The Journal of Political Economy. En su estudio de la teoría de la aglomeración, Ottaviano y Thisse (2004) plantean la pregunta ¿Dónde estábamos en 1990?, que es anterior a Krugman (1991a). Observan que, con una excepción crucial, todos los elementos importantes ya estaban flotando en la teoría de ubicación existente. Cada ubicación preferirá la autarquía para ahorrar en costos de transporte y, si cada actividad económica es perfectamente (y sin costo) divisible en el espacio, la autarquía o el capitalismo de traspatio con consumidores/trabajadores autoproductivos es el único resultado factible. Esto va, por supuesto, en contra de los hechos más básicos, a saber, que efectivamente observamos comercio entre ubicaciones y la ausencia de capitalismo de traspatio. Por lo tanto, Duranton (2008) concluye: Si uno toma los costos de transporte como un hecho inevitable de la vida, debe asumir alguna no homogeneidad del espacio o alguna no convexidad de los conjuntos de producción. más específicamente, el legado de la teoría de la ubicación se puede resumir en los siguientes cinco puntos (Ottaviano y Thisse, 2004: 2576; énfasis en el original):

\begin{enumerate}[\bfseries (i)]
    \item El espacio económico es el resultado de una compensación entre varias formas de rendimientos crecientes y diferentes tipos de costos de movilidad;
    \item la competencia de precios, los altos costos de transporte y el uso del suelo fomentan la dispersión de la producción y el consumo; por lo tanto;
    \item es probable que las empresas se concentren en grandes áreas metropolitanas cuando venden productos diferenciados y los costos de transporte son bajos; 
    \item las ciudades ofrecen una amplia gama de bienes finales y mercados laborales especializados que las hacen atractivas para los consumidores/trabajadores; y
    \item las aglomeraciones son el resultado de procesos acumulativos que involucran tanto a la oferta como a la demanda.
\end{enumerate}

\textbf{En consecuencia, la economía espacial debe entenderse como el resultado de la interacción entre las fuerzas de aglomeración y dispersión,} una idea propuesta por geógrafos y científicos regionales hace mucho tiempo, dentro de un marco de equilibrio general que explica explícitamente las fallas del mercado.\\
\textbf{Estos cinco puntos también están en el corazón de la economía geográfica, lo que plantea la pregunta de cuál es la diferencia clave entre la economía geográfica al estilo de Krugman (1991a) y las teorías de ubicación existentes de los enfoques urbano y regional.  De hecho, la respuesta es que lo que faltaba era un marco de equilibrio general con competencia imperfecta que conectara estos diversos conocimientos y permitiera un estudio detallado de sus interacciones. La principal contribución de la economía geográfica es, por lo tanto, combinar los elementos existentes en un solo marco analítico}. 

\paragraph{El teorema de la imposibilidad espacial}
\textbf{El teorema de la imposibilidad espacial establece que en una economía con un número finito de ubicaciones y un número finito de consumidores y empresas, en la que el espacio es homogéneo y el transporte es costoso, no existe un equilibrio perfectamente competitivo en el que tenga lugar el transporte real.} Esto es intuitivamente fácil de entender, ya que en tal economía los costos de transporte siempre pueden evitarse porque la producción y el consumo pueden tener lugar a un nivel arbitrariamente pequeño, sin costos adicionales (una situación denominada capitalismo de traspatio en la literatura). En tal mundo hipotético de perfecta divisibilidad, sería imposible explicar por qué ocurre la agrupación o aglomeración de actividades (como observamos en la realidad). \textbf{Sólo si hay indivisibilidades, o costos adicionales involucrados si la producción se divide, la ubicación de las actividades económicas se vuelve importante;} Starrett (1978: 27) afirma: “Mientras haya algunas indivisibilidades en el sistema (de modo que las operaciones individuales deban ocupar espacio), entonces un conjunto suficientemente complicado de actividades interrelacionadas generará costos de transporte. Este principio se conoce como el teorema de imposibilidad espacial de Starrett; \\
Es interesante notar que Tjalling Koopmans señaló hace medio siglo que solo podemos comenzar a comprender la importancia de la ubicación o la geografía para la economía si reconocemos el hecho de que las actividades económicas no son infinitamente divisibles; o, en palabras de Koopmans (1957: 154; énfasis en el original), Sin reconocer las indivisibilidades en la persona humana, en las residencias, plantas, equipos y en el transporte, los patrones de ubicación, hasta los de la aldea más pequeña, no se pueden entender.\\
\textbf{Cada ubicación preferirá la autarquía para ahorrar en costos de transporte y, si cada actividad económica es perfectamente (y sin costo) divisible en el espacio, la autarquía o el capitalismo de patio trasero con consumidores/trabajadores autoproductivos es el único resultado factible.} Esto va, por supuesto, en contra de los hechos más básicos, a saber, que efectivamente observamos comercio entre ubicaciones y la ausencia de capitalismo de traspatio. Por lo tanto, Duranton (2008) concluye: \textbf{Si uno toma los costos de transporte como un hecho inevitable de la vida, debe asumir alguna no homogeneidad del espacio o alguna no convexidad de los conjuntos de producción}.

\section{Teoría del comercio internacional}
En esta sección se analiza el papel de la geografía en la teoría del comercio internacional. \textbf{El modelo central de la economía geográfica tiene sus raíces firmemente en la teoría del comercio internacional. En muchos sentidos es una extensión de la llamada nueva teoría del comercio, específicamente de Krugman (1979, 1980)}. Estos dos artículos, junto con Krugman (1991a) el modelo básico de economía geográfica– le valieron el Premio Nobel de economía en 2008.

\subsection{Teoría neoclásica del comercio}
\textbf{La etiqueta teoría neoclásica del comercio se refiere a teorías en las que los flujos comerciales entre naciones se basan en la ventaja comparativa, resultante de las diferencias tecnológicas (David Ricardo) o de la abundancia de factores.} En el modelo de abundancia de factores, desarrollado por Eli Heckscher, Ohlin y Paul Samuelson, la ventaja comparativa está determinada, como sugiere el nombre, por las diferencias entre países en la abundancia relativa de las dotaciones de factores. Basta con pensar en el modelo simple de abundancia de factores (dos bienes, dos países y dos factores de producción), que sigue siendo la columna vertebral de cualquier curso de introducción al comercio internacional. Fue ampliamente utilizado, por ejemplo, en el debate sobre los efectos de la globalización para los mercados laborales de la Organización para la Cooperación y el Desarrollo Económicos (OCDE).\\
Suponga que hay dos países (Norte y Sur), dos bienes transables (ropa y maquinaria) y dos factores de producción (mano de obra altamente calificada y mano de obra poco calificada). Suponga que el Norte está relativamente bien dotado de mano de obra altamente calificada y el Sur de mano de obra poco calificada. La producción de ambos conjuntos de bienes requiere ambos insumos, pero la producción de maquinaria es relativamente intensiva en habilidades. Los consumidores del Norte y del Sur tienen preferencias idénticas y consumen ambos bienes. En ausencia de comercio, el Norte, que abunda en mano de obra altamente calificada, puede fabricar maquinaria más fácilmente que el Sur, porque la producción de maquinaria requiere mucha destreza. En la autarquía, esto da como resultado un precio relativamente bajo para la maquinaria en el Norte y la ropa en el Sur. Una vez que el Norte y el Sur comiencen a comerciar, los precios se igualarán, lo que resultará en un precio más alto para la maquinaria en el Norte y un precio más alto para la ropa en el Sur. Como consecuencia, North tendrá una incentivo para especializarse (parcialmente) en la producción de maquinaria. Un razonamiento similar es válido para el sur con respecto a la ropa. Los flujos comerciales resultantes son del tipo interindustrial (comercio de maquinaria para prendas de vestir). Además, \textbf{los precios de los factores se igualarán entre el Norte y el Sur como resultado del comercio.}\\
\textbf{El modelo de abundancia de factores utiliza algunos supuestos específicos adicionales, como competencia perfecta, bienes homogéneos, producción con rendimientos constantes a escala, sin costos de transporte asociados con el comercio de bienes y movilidad de los factores de producción entre industrias, pero no entre países.} Está claro que \textbf{varios de estos supuestos están en desacuerdo con los supuestos clave en la economía regional y urbana, en la que tenemos rendimientos crecientes a escala externos y/o internos, competencia imperfecta, costos de transporte positivos y movilidad de los factores de producción (y las empresas)}. Estos ingredientes son necesarios para dar cuenta de los patrones económicos espaciales. ¿Significa esto que la geografía o la ubicación de la actividad económica no es un problema en la teoría neoclásica del comercio? Bueno, sí y no.\\
Para explicar esta respuesta, es útil distinguir entre la primera y la segunda naturaleza de la economía de la ubicación. \textbf{La ubicación de la actividad económica es relevante en el modelo de abundancia de factores en lo que respecta a la distribución desigual de las dotaciones de factores.} Esta distribución está dada y, por lo tanto, es un determinante de ubicación de primera naturaleza en la terminología de Krugman. \textbf{En nuestro ejemplo, el Norte se especializa en maquinaria y el Sur en ropa como resultado de la distribución geográfica de las dotaciones, lo que se traduce en una distribución desigual de la actividad económica en el espacio global. En este sentido restringido, la geografía importa.}\\
Sin embargo, nos gustaría llegar a esta conclusión de otra manera, a saber, mostrando cómo la relevancia de la ubicación se deriva de las decisiones tomadas por los agentes económicos y sus interacciones y no (solo) de alguna diferencia exógena en la dotación de factores. En otras palabras, \textbf{la ubicación de la producción debería ser una variable endógena, un determinante de ubicación de segunda naturaleza en la terminología de Krugman.} Esta segunda naturaleza está claramente ausente en la teoría de la abundancia de factores. \textbf{La endogenización de las decisiones de ubicación es necesaria para producir la aglomeración de la actividad económica.} Las diferencias en la dotación de factores no pueden implicar un patrón de producción centro-periferia; simplemente conducen a la especialización (en oposición a la aglomeración). El comercio entre países no puede conducir a la desigualdad en el sentido de que la maquinaria y la ropa no pueden aglomerarse en el Norte.\\
El modelo de abundancia de factores conduce a la igualación de los salarios de alta calificación entre el Norte y el Sur (y de manera similar para los salarios de baja calificación). Esta teoría económica se ha utilizado para analizar un fenómeno con un componente geográfico evidente: la globalización y su supuesto impacto en la asignación de la producción y la renta en las economías industrializadas occidentales (Norte). \textbf{La globalización puede definirse como la creciente interdependencia entre países a través del aumento del comercio y/o el aumento de la movilidad de los factores.}\\
Las diferencias relativas en las dotaciones de factores se pueden utilizar para dar una justificación teórica de las diferencias en los patrones de especialización entre países. Otras versiones de la teoría neoclásica del comercio tienen implicaciones similares en lo que se refiere a la relevancia de la geografía. En el modelo ricardiano, la ventaja comparativa, y por lo tanto el patrón comercial, está determinado por las diferencias tecnológicas exógenas entre países. \textbf{Los países se especializan en la producción de aquellos bienes en los que tienen una productividad comparativamente alta, lo que determina la ubicación de la producción.} Nuestra principal objeción al modelo de abundancia de factores también es válida para el modelo ricardiano: en la medida limitada en que la geografía importa, esta relevancia se da de manera exógena. Naturalmente, las diferencias en la dotación de factores o la tecnología pueden ser el resultado de diferencias geográficas. Consideremos, por ejemplo, la tierra como un factor de producción, como en la tradición de von Thunen en economía urbana. La disponibilidad de tierra (fértil) da forma a la ventaja comparativa. De manera similar, la geografía física de un país (acceso al mar, altitud, clima, etc.) también puede ser un determinante subyacente de la ventaja comparativa, que sin duda es válida para el stock de recursos naturales. Se  muestran que tales diferencias geográficas entre países ayudan a explicar las diferencias en el desarrollo económico.\\
El papel limitado de la geografía en la teoría neoclásica del comercio quizás se ilustre mejor con el llamado modelo de factores específicos. Parte de la dotación de factores de un país (mano de obra, por ejemplo) es entonces móvil internacionalmente, mientras que otras partes (tierra y capital, por ejemplo) no lo son. La producción de un determinado bien requiere insumos del factor móvil, así como de un factor inmóvil particular o específico (es decir, normalmente tierra o capital específico del sector). Las diferencias en la dotación de los factores específicos influyen así en el patrón de producción y comercio, con un país que se especializa, ceteris paribus, en la producción del bien que requiere el insumo del factor específico con el que el país está relativamente bien dotado. Existe un vínculo geográfico en la medida en que la distribución de las dotaciones inmóviles está determinada por las condiciones geográficas. Una vez más, tal conexión es indirecta en el mejor de los casos, y el impacto de la geografía se determina fuera del modelo comercial.\\
Para resumir, se afirman correctamente que el espacio no homogéneo, también conocido como espacio no neutral, se ha invocado tradicionalmente para explicar la distribución desigual de la actividad económica en la economía internacional. Estos dan lugar a distintas fuentes de ventaja comparativa, que imposibilitan la expansión de la actividad económica. Por lo tanto, \textbf{la relevancia de la ubicación existe solo por suposición, y no hay interdependencia entre la geografía y la economía. En particular, la ubicación de equilibrio de la actividad económica no es el resultado del comportamiento subyacente de los agentes económicos.} El equilibrio (comercial) suele ser único y totalmente determinado por fuerzas exógenas. Más importante aún, la teoría neoclásica del comercio no permite el establecimiento de un equilibrio centro-periferia, lo que presenta un problema en vista de los muchos ejemplos dados en el capítulo 1. Para permitir la aglomeración de la actividad económica, algunos de los supuestos subyacentes a la teoría neoclásica del comercio han de ser cambiado. Un candidato obvio es la introducción de rendimientos internos crecientes a escala y, por lo tanto, de competencia imperfecta.\\
\textbf{Una observación final sobre la relación entre la geografía y la teoría neoclásica del comercio es que, sin una distribución desigual de los recursos y, por lo tanto, sin una ventaja comparativa, ceteris paribus ya no es una razón fundamental para el comercio, y la geografía deja de ser un problema.} Se puede llegar a una conclusión similar incluso si existen ventajas comparativas. Si estos costos son lo suficientemente altos, la producción de bienes estará perfectamente dispersa en el espacio. La economía consistirá entonces en muchas pequeñas empresas, produciendo para su propio consumo: capitalismo de traspatio nuevamente. Esto se relaciona directamente con la discusión del teorema de imposibilidad espacial, donde concluimos que una salida era introducir un espacio no homogéneo (la ruta tradicionalmente tomada en la teoría del comercio internacional) y otra salida era introducir un espacio no homogéneo. -convexidades en la producción, de las cuales los rendimientos crecientes a escala son el mejor ejemplo. A este respecto, la nueva teoría del comercio que se presenta a continuación tiene más en común con la economía urbana y regional que con la teoría neoclásica del comercio.

\paragraph{Globalización, abundancia de factores y agrupamiento}
Supongamos que el mundo se puede caracterizar con un modelo de abundancia de factores con dos países (Norte y Sur), dos bienes manufactureros (máquinas y prendas de vestir) y un bien no comercializable y no manufacturero. El norte está relativamente bien dotado de mano de obra altamente calificada, el sur con mano de obra poco calificada. Ambos factores de producción son necesarios para la producción de todos los bienes, pero la producción mecánica es relativamente intensiva en habilidades. Por lo tanto, el norte tiene una ventaja comparativa en la producción de máquinas. Ambos países producen bienes transables, hay competencia perfecta, la tecnología es fija y no hay movilidad laboral transfronteriza. Suponga que inicialmente, debido a los costos de transacción muy altos, ambos países no comercian en absoluto. Ahora, los costos de transacción disminuyen y el comercio se abre. La caída en los costos de transacción (que sirve como un indicador de la globalización) puede ser impulsada por políticas (reducción de tarifas y similares) o impulsada por la tecnología (tecnologías mejoradas de transporte y comunicación). ¿Cuáles son los principales efectos del comercio para el Norte? Se especializará en la producción de maquinaria y comenzará a importar prendas de vestir, es decir, el sector de maquinaria se expande y el sector de prendas de vestir se contrae. Esto tiene las siguientes implicaciones para el norte.
\begin{enumerate}[\bfseries (1)]
    \item Habrá un salario de alta calificación y un salario de baja calificación (el teorema de igualación del precio de los factores). Dado que los salarios se determinan en el mercado mundial, los cambios en la oferta nacional de factores ya no tienen ningún impacto en los salarios.
    \item El norte se enfrenta a un aumento de la producción mundial de prendas de vestir, lo que se traduce en una caída del precio relativo de las prendas. Esto perjudicará a la mano de obra poco calificada en el norte, utilizada intensivamente en la producción de prendas de vestir, al reducir su salario real (teorema de Stolper-Samuelson).
    \item La expansión del sector de la maquinaria en el Norte aumenta la demanda relativa de mano de obra altamente calificada, elevando así los salarios de los trabajadores altamente calificados en relación con los salarios de los trabajadores poco calificados en el Norte. Esto induce a las empresas del norte a sustituir la mano de obra altamente calificada y disminuye la intensidad de mano de obra calificada de la producción manufacturera en el norte.
    \item La contracción del sector de la confección en el norte no solo cambia la combinación de la producción manufacturera, sino que también implica una contracción del sector manufacturero en su conjunto, porque parte de la mano de obra liberada del sector de la confección se empleará en el sector no comercializable. Sector servicios. En consecuencia, el sector no manufacturero se expande y el norte se enfrenta a la desindustrialización.
\end{enumerate}
Por lo tanto, se puede recurrir al modelo de abundancia de factores para brindar una base teórica a la idea de que la globalización (aumento de las importaciones por parte del norte de bienes intensivos poco calificados del Sur) puede perjudicar a los trabajadores poco calificados, al reducir sus salarios relativos, y puede conducir a desindustrialización. La principal dimensión geográfica de este análisis de la globalización es la implicación de que el Norte (los países de la OCDE) se especializan en la producción intensiva en mano de obra calificada, mientras que el Sur se especializa en la producción intensiva en mano de obra poco calificada (el sudeste de Asia, América Latina o las economías en transición en Europa del Este).\\
Una pregunta importante para esta versión del debate sobre la globalización es, por supuesto, si existe alguna evidencia empírica que respalde el modelo de abundancia de factores en general para validar las implicaciones 1 a 4 anteriores. El modelo de abundancia de factores ha sido objeto recientemente de una impresionante cantidad de investigación empírica.  Aunque existe cierto desacuerdo, el consenso general parece ser que se necesitan suposiciones adicionales relativamente sólidas para cerrar la brecha entre la teoría neoclásica y los datos. En cualquier caso, las cuatro implicaciones del modelo de abundancia de factores no están fundamentadas de manera convincente por la evidencia empírica. Por lo tanto, es dudoso que el modelo de abundancia de factores (solo) pueda explicar el empeoramiento de la posición de la mano de obra poco calificada en el Norte.

\subsection{Nueva teoría comercial}
Desde finales de la década de 1970 en adelante, la teoría neoclásica del comercio ha sido desafiada por el desarrollo de la nueva teoría del comercio, que ahora es complementaria a la teoría neoclásica del comercio y forma parte de casi todos los libros de texto sobre comercio internacional. \textbf{La razón del comercio entre países en la nueva teoría comercial no depende de la ventaja comparativa.} De hecho, Krugman (1979, 1980) ha desarrollado un modelo (ahora estándar) en el que los países participan en el comercio que mejora el bienestar incluso cuando no existe ninguna ventaja comparativa. El punto de partida de la nueva teoría del comercio fue el hecho estilizado de que \textbf{una gran parte del comercio internacional tiene lugar entre países con dotaciones de factores muy similares}. Este comercio no es, como predeciría la teoría neoclásica del comercio, comercio interindustrial (exportación de cereales a cambio de automóviles) sino comercio intraindustrial (exportación de automóviles a cambio de automóviles). La relevancia empírica del comercio intraindustrial era, por supuesto, bien conocida, pero la base teórica de este tipo de comercio requería una clase de modelos en los que algunos de los componentes básicos de la teoría neoclásica del comercio tenían que ser derribados.

\subsubsection{Krugman (1979)}
Las ideas básicas de Krugman (1979) se pueden ilustrar de la siguiente manera. Supongamos que hay dos países con el mismo tamaño de mercado, Oeste y Este, que tienen las mismas dotaciones, usan la misma tecnología y ambos tienen una empresa (inmóvil) productora de automóviles. En el modelo de abundancia de factores, estos países no comerciarían. Ambas firmas fabrican varios tipos de automóviles bajo rendimientos crecientes a escala para cada tipo. En la autarquía, las empresas producen tres tipos de automóviles, a saber, los tipos X, Y y Z en el Oeste y los tipos A, B y C en el Este. Así, hay una industria que produce seis tipos o variedades de automóviles. Los consumidores (trabajadores) en el Oeste y el Este están inmóviles, distribuidos uniformemente y tienen preferencias idénticas. \textbf{Las variedades son sustitutos imperfectos y las preferencias son tales que los consumidores siempre prefieren más variedades de un automóvil a menos} (este es el efecto de los gustos por la variedad . \textbf{La clave para comprender la justificación del comercio en este modelo es la combinación de rendimientos crecientes a escala a nivel de empresa (economías de escala internas; y el efecto de los gustos por la variedad en las preferencias de los consumidores, que es una externalidad no tenida en cuenta por las empresas. Al pasar de la autarquía al libre comercio, estos dos supuestos aseguran que el comercio tendrá lugar y mejorará el bienestar.}\\
\textbf{La medida en que cada empresa puede explotar los rendimientos crecientes a escala está determinada por el tamaño del mercado. La apertura del comercio amplía el tamaño del mercado para cada tipo de automóvil.} Dado que cada variedad se produce con rendimientos crecientes a escala, este mercado más grande permite a las empresas explotar mejor los rendimientos crecientes. \textbf{La apertura del comercio significa que la producción de automóviles por variedad puede aumentar a medida que el mercado más grande hace que sea rentable expandir la escala de producción. Al hacerlo, los precios por variedad disminuirán. Para que esto sea posible en el mercado integrado de Occidente y Oriente, el número total de variedades producidas debe disminuir. Para ver esto, tenga en cuenta que las dotaciones totales (Oeste + Este) y el tamaño total del mercado son fijos, de modo que no es posible aumentar simultáneamente la producción de las seis variedades. En libre comercio, los dos países juntos producen menos de seis variedades, digamos cuatro (X, Y, A y B). Hay entonces dos efectos positivos sobre el bienestar. Primero, la disminución de los precios provocada por la mayor escala de producción implica que los trabajadores/consumidores terminan con un salario real más alto. En segundo lugar, después del comercio, los consumidores pueden consumir cuatro variedades en lugar de tres, y esto aumenta el bienestar a través del efecto  por el gusto de las variedades}.\\
Aunque las ideas básicas de Krugman (1979) son fáciles de entender, \textbf{la introducción de rendimientos crecientes a escala implica una estructura de mercado de competencia imperfecta.} El desafío teórico era, por lo tanto, proporcionar un modelo de comercio con competencia imperfecta, un desafío real en vista de la discusión sobre economía urbana y regional. Afortunadamente, Krugman podía basarse en un modelo de competencia monopolística que acababa de publicarse. El enfoque Dixit-Stiglitz ahora se usa ampliamente en muchos campos, incluida la economía geográfica. En vista de la dificultad de lidiar con la competencia imperfecta, \textbf{no sorprende que la nueva teoría del comercio también incluya modelos con economías de escala puramente externas, en lugar de internas (Helpman, 1984; Helpman y Krugman, 1985), ya que permite a los economistas geográficos ceñirse a una estructura de mercado de competencia perfecta.}\\
La pregunta ahora es si la nueva teoría del comercio tiene algo que decir sobre el papel de la geografía. En el modelo de Krugman (1979), la respuesta es simple. La ubicación de la actividad económica no es realmente un problema. Los costos comerciales son cero, por lo que las empresas son indiferentes con respecto a la ubicación de sus sitios de producción. Incluso si hay costos comerciales positivos, el tamaño del mercado (exógeno) se distribuye uniformemente entre los dos países, lo que impide cualquier aglomeración de la actividad económica. Es indeterminado qué país termina produciendo qué variedades. Todo lo que se puede decir es que los países producen diferentes variedades y el patrón de comercio es indeterminado. Sin embargo, \textbf{este modelo es importante como base del modelo central de la economía geográfica, por ejemplo, con respecto al análisis del comportamiento del productor y del consumidor. Con las economías de escala externas tampoco se aborda la ubicación de la actividad económica.} Se podría argumentar que los efectos de bloqueo en algunos de estos modelos permiten que las condiciones iniciales desempeñen un papel en la determinación de la asignación de la producción. Al igual que con la teoría del comercio neoclásica, este papel de la geografía se determina fuera del modelo.

\paragraph{Nueva teoría del comercio y economías externas}
Supongamos que una industria, digamos la industria de las computadoras personales (PC), se caracteriza por economías de escala externas puras, que surgen, por ejemplo, de los derrames de información cuando el aumento en la producción de una sola empresa aumenta el conocimiento de producción para todas las empresas en la PC. Esto implica no solo que los costos promedio por PC para cada empresa son una función decreciente de la producción de la industria, sino también que todavía podemos usar la competencia perfecta. No hay ninguna ventaja para una empresa en ser grande (en vista de las economías de escala externas), por lo que normalmente la economía ahora consta de muchas empresas pequeñas. Bajo competencia perfecta, el precio es igual al costo promedio de cada empresa. Finalmente, suponga que hay dos países, A y B, y que los consumidores en ambos países tienen preferencias idénticas. Como en el caso de las economías de escala internas, el comercio intraindustrial puede desarrollarse entre los dos países, con ambos países produciendo y exportando variedades de PC. Sin embargo, con economías externas, también podemos tener un equilibrio en el que un país produce la demanda mundial total de PC. Si, por alguna razón histórica, la industria de PC se establece inicialmente en el país A, las economías externas pueden convertir esta distribución inicial de la producción en un equilibrio duradero incluso si la industria de PC en el país B fuera más eficiente (efecto lock-in). Dos cuestiones, también parte de la economía geográfica, son relevantes en este caso. Primero, las condiciones iniciales pueden determinar el resultado del equilibrio (estable). Dependiendo de qué empresa ingrese primero al mercado, cualquiera de los dos países podría terminar siendo el productor mundial de PC. En segundo lugar, el equilibrio comercial resultante puede ser ineficiente.\\
Un ejemplo simple ilustra la posibilidad de un equilibrio malo. Supongamos que el país A es el primero en establecer una industria de PC y produce 500.000 PC. A este nivel de producción de la industria, se puede cobrar un precio (igual al costo promedio) de \$1 000 por PC para satisfacer la demanda mundial, que se fija en 500 000 unidades por simplicidad. Suponga que la industria de PC en el país B podría producir 500 000 PC de manera más eficiente, digamos por \$750 por unidad de producción. Esto no implica que el país B comenzará a producir PC, ya que estos costos se aplican a toda la industria. En ausencia de una industria de PC en el país B, y por lo tanto en ausencia de economías externas positivas, una sola empresa en el país B puede producir 500 PC solo por un precio superior a \$ 1,000, es decir, a un costo más alto que en país A, por lo que no vale la pena instalarse en el país B.\\
Las economías de escala externas son importantes en este ejemplo. Dado que la industria del país A satisface la demanda mundial a un costo promedio de \$1,000, la empresa individual en el país B solo puede producir a un costo promedio más alto. Esto es cierto para todas las empresas en el país B. Solo cuando todas las empresas en B deciden conjuntamente comenzar a producir PC, pueden hacerse cargo del mercado de PC, ya que esto reduce el costo promedio por debajo de \$ 1,000 como resultado de las economías de escala externas. Sin embargo, no hay nada que induzca a las empresas de B a tomar tal decisión, porque la empresa individual solo se enfrenta al hecho de que sus costes medios superan el precio de mercado prevaleciente. Este problema no ocurriría con economías de escala internas, en las que los costos promedio de una empresa caen a medida que la empresa produce más.

\subsubsection{Krugman (1980)}
Krugman (1980) es un paso crucial desde el nuevo modelo comercial inicial de Krugman (1979) hasta el modelo central de la economía geográfica. \textbf{La justificación del comercio intraindustrial es la misma que en el modelo de 1979, con algunas diferencias notables. En primer lugar, en el modelo de 1980, la apertura del comercio y, por tanto, el aumento del tamaño del mercado, no conduce a un aumento de la escala de producción, a pesar de los rendimientos crecientes a escala a nivel de empresa. En cambio, el volumen de producción de cada variedad (a nivel de empresa) es el mismo bajo la autarquía y el comercio, y los precios no cambian. Las ganancias del comercio ahora se deben completamente al efecto del gusto por la variedad, ya que los consumidores pueden elegir entre más variedades bajo comercio que autarquía.} El modelo central de la economía geográfica coincide con Krugman (1980) en este importante tema. \textbf{En segundo lugar, en el modelo de 1980, el comercio entre naciones incurre en costos de transporte, lo que, obviamente, es relevante desde el punto de vista geográfico. Tercero, en el modelo de 1980, la demanda por variedad ya no es simétrica ya que los países difieren en el tamaño del mercado.}\\
\textbf{Esta distribución desigual del tamaño del mercado se vuelve importante cuando se combina con costos de transporte positivos, porque un país producirá aquellas variedades para las cuales la demanda en el país es relativamente alta. En este sentido, la localización de la producción importa y la concentración de la actividad económica puede ser un resultado del modelo.} El razonamiento es simple: dada la distribución desigual de la demanda, las empresas, que aún están inmóviles, minimizan los costos de transporte si producen aquellas variedades para las que la demanda interna es relativamente fuerte. Además, y a diferencia del modelo sin costos de transporte, la dirección del comercio ya no es indeterminada, porque la concentración de la producción implica que los países serán exportadores netos de aquellas variedades para las cuales la demanda interna es relativamente alta. Como dice Krugman (1980: 955): \textbf{Los países tenderán a exportar ese tipo de productos para los que tienen una demanda interna relativamente grande. Observe que este argumento depende por completo de los rendimientos crecientes; en un mundo de rendimientos decrecientes, la fuerte demanda interna de un bien tiende a convertirlo en una importación en lugar de una exportación”. Este fenómeno se conoce como el efecto del mercado interno.}\\
En un intento por probar el efecto del mercado interno, Donald Davis y David Weinstein (1999) se refieren a Krugman (1980) como un modelo de geografía económica. Esto sugiere que no hay una diferencia fundamental entre este modelo y el modelo central del capítulo 3. Sin embargo, no estamos de acuerdo con ese punto de vista por tres razones. Primero, ni las empresas ni los trabajadores deciden nada sobre la ubicación en Krugman (1980). No hay movilidad de las empresas ni de los factores de producción. Dada su ubicación (exógena), las empresas toman una decisión solo sobre las variedades que quieren producir. En segundo lugar, la concentración de la producción de variedades (y, por supuesto, de la demanda) no permite la aglomeración de la actividad económica. Los equilibrios centro-periferia no son posibles porque la concentración de la demanda en el primer país, digamos para las variedades X, se refleja en una concentración similar de la demanda para las variedades (1 – X) en el otro país. En este sentido, ambos países se caracterizan por una concentración geográfica de la industria. En tercer lugar, la asignación del tamaño del mercado para las variedades no es un resultado del modelo, sino que simplemente se da (por lo tanto, también se dan los ingresos). Esto está íntimamente ligado a la inmovilidad de los trabajadores (que demandan los bienes producidos) y de las empresas. En estos aspectos, \textbf{la ubicación en Krugman (1980) todavía se determina fuera del modelo.}

\subsubsection{Krugman y Venables (1990)}
Las consecuencias de Krugman (1980) se analizan en Krugman y Venables (1990). Permiten que los países difieran en tamaño, desarrollando así un modo que se parece mucho al modelo central de la economía geográfica un año antes de que Krugman publicara este modelo central en 1991. Es un modelo de dos países, en el que el país 1 es grande: tiene más dotaciones de factores (capital y trabajo) y un mercado más grande que el país 2. En la parte principal de su estudio, las dotaciones relativas son las mismas entre los dos países, por lo que no existe una ventaja comparativa y el comercio es del tipo intraindustrial. Hay dos sectores en ambos países, uno perfectamente competitivo y otro imperfectamente competitivo (manufacturas), ambos productores de bienes transables. El país 1 también tiene un mayor número de empresas en el sector manufacturero. Este sector produce productos diferenciados bajo rendimientos crecientes a escala y competencia monopolística. Se permite la entrada y salida de empresas, pero las empresas no pueden moverse entre países. Esto último también es válido para los factores de producción. Tanto para las empresas como para los factores de producción, sólo existe movilidad intersectorial. La pregunta central es cómo un aumento en el grado de integración económica (utilizando una caída en los costos de transporte como proxy) afecta el centro (país 1) y la periferia (país 2).\\
En la autarquía (cuando los altos costos de transporte prohíben el comercio), ambos países tienen una participación en el sector manufacturero igual a su participación en las dotaciones mundiales. La diferencia de dotaciones viene dada por los segmentos A y B. Resulta que, para un rango intermedio de costos de transporte, la integración económica fortalece el centro: la participación del centro en la industria mundial, S1, es mayor que su participación en las dotaciones mundiales (esta última es 0,6), y viceversa para la periferia ( S2 < 0,4). Nuevas empresas ingresan al sector manufacturero en el país 1, mientras que algunas empresas salen de este sector en el país 2. Dado el mercado más grande en el país 1 y la minimización de los costos de transporte, las nuevas empresas prefieren el país 1 aunque los salarios sean más altos. A medida que los costos de transporte continúan cayendo, la participación del núcleo de la industria mundial finalmente comienza a disminuir nuevamente. A costos de transporte muy bajos, la ventaja de producir en el país con el mercado más grande se vuelve pequeña, lo que, combinado con la competencia más dura del mercado laboral en el país 1 (más empresas compiten por los factores de producción del país, lo que eleva los precios de los factores), implica que nuevos a las empresas les resulta rentable iniciar la producción en el país 2, donde los salarios son más bajos. En el caso extremo de cero costos de transporte, los salarios serán iguales y la participación de cada país en las manufacturas volverá a su participación en las dotaciones mundiales. Por lo tanto, no es lineal relación entre la participación de un país en la industria mundial y los costos de transporte, las acciones siempre suman uno.\\
¿Por qué el análisis es ¿interesante? En primer lugar, se ocupa de la aglomeración de la actividad económica, porque permite una distribución global desigual de la actividad manufacturera. Recuérdese que este no es el caso en el modelo de Krugman (1980), en el que hay una concentración geográfica de una sola industria pero no hay una concentración de la producción manufacturera en su conjunto. En segundo lugar, el patrón en forma de U  presagia importantes resultados teóricos y empíricos en la literatura económica geográfica. Tercero,  se basa en ejemplos numéricos y se utilizan para analizar los efectos de la integración económica en el centro y la periferia. Esto se asemeja a la estrategia de la economía geográfica de utilizar simulaciones por computadora para analizar la aglomeración de la actividad económica. Por lo tanto, surge la pregunta: ¿es este un modelo de economía geográfica completo? La respuesta es que da para mucho, pero al final no lo es. La razón principal es que la existencia de centro y periferia no se deriva del modelo en sí. La suposición de que el los tamaños de mercado difieren plantea la pregunta de por qué debería haber un centro y una periferia para empezar. Además de fijar el tamaño del mercado, se supone que los trabajadores están inmóviles. La movilidad de los trabajadores, y por ende de la demanda, iría en contra de la idea de que se podría fijar a priori el tamaño relativo del mercado de los dos países. El modelo central de los capítulos 3 y 4 agrega la endogenización del tamaño del mercado (y la movilidad de los consumidores/trabajadores, que determinan el tamaño del mercado) al modelo de Krugman y Venables (1990).\\
En la introducción a su libro, Fujita, Krugman y Venables (1999) enumeran cuatro características principales de la economía geográfica. Dos de estas características (competencia monopolística Dixit-Stiglitz y costos de transporte tipo iceberg) también están presentes en Krugman (1980). El modelo de Krugman y Venables (1990) agrega un (crudo) intento de usar simulaciones. En nuestra opinión, \textbf{la verdadera novedad de la economía geográfica se encuentra en la cuarta característica, la dinámica, que aborda la cuestión de cómo abordar la movilidad de los agentes económicos (principalmente empresas y trabajadores), que endogeniza el tamaño del mercado. El énfasis subraya el hecho de que esta es la diferencia clave entre la nueva teoría del comercio y la economía geográfica.} Como en nuestra discusión  sobre las similitudes entre la economía urbana y regional, por un lado, y la economía geográfica, por el otro, hemos llegado a la conclusión de que, además de un elemento crucial: la dinámica, los nuevos La teoría del comercio en 1990 era muy similar a la economía geográfica de Krugman (1991a). Desde la perspectiva de la nueva teoría del comercio, Head y Mayer (2004a) lo resumen acertadamente. \textbf{Enumeran los siguientes cinco ingredientes esenciales de la economía geográfica (2613-4):}
\begin{enumerate}
    \item Rendimientos crecientes a escala que son internos a la empresa; 
    \item competencia imperfecta;
    \item costos positivos de transporte o comercio; 
    \item ubicaciones de empresas endógenas; y 
    \item la ubicación endógena de la demanda, ya sea a través de trabajadores móviles (Krugman, 1991; véase el capítulo 3) o empresas que utilizan la producción de su sector como insumos intermedios (Venables, 1996 y Krugman y Venables, 1995; véase el capítulo 4).
\end{enumerate}

Cómo la introducción de la condición (5) puede poner en marcha un proceso de causalidad circular; el punto a enfatizar aquí es el importante legado de la nueva teoría del comercio para la economía geográfica. \\

\section{Crecimiento económico y desarrollo}
\textbf{La teoría del comercio se ocupa, sobre todo, de la cuestión de cómo el comercio internacional determina la distribución de la actividad económica entre los países.} Como tal, no aborda la cuestión dinámica del crecimiento económico y el desarrollo a lo largo del tiempo. Un descuido geográfico del crecimiento económico no sería un problema en el contexto de nuestro estudio si los países experimentaran un proceso más o menos similar de crecimiento económico y convergieran aproximadamente a los mismos niveles de bienestar económico. Un vistazo rápido a los datos deja claro que no es así. Las tasas de crecimiento del PIB per cápita varían considerablemente entre países, al igual que el nivel del PIB per cápita. Además, los datos sugieren que puede haber un componente geográfico involucrado. Los países de alto y bajo crecimiento suelen estar concentrados geográficamente; piense en el sudeste asiático y el África subsahariana, respectivamente. Los niveles altos y bajos de PIB per cápita claramente tampoco están distribuidos al azar en el espacio: uno observa grupos de países ricos y pobres. En esta sección nos preguntamos si las teorías sobre el crecimiento económico y el desarrollo tienen algo que decir sobre la relación entre la geografía y el crecimiento económico. No intentamos una encuesta, sino que simplemente nos enfocamos en conocimientos básicos (y bien conocidos) para evaluar la relevancia geográfica de la teoría del crecimiento.\\
\textbf{La geografía no es realmente relevante en la teoría neoclásica del crecimiento. A corto plazo, se puede lograr una tasa de crecimiento positivo de la producción per cápita mediante la acumulación de capital o el progreso tecnológico. Sin embargo, dado que la acumulación de capital está sujeta a la ley de los rendimientos decrecientes, solo a través del progreso tecnológico se puede mantener una tasa de crecimiento positivo de la producción per cápita a largo plazo.} El progreso tecnológico es exógeno, de modo que al final deja sin explicación el crecimiento de la producción per cápita. Se cree que las diferencias entre países en el nivel de producción per cápita son temporales. Suponiendo que los países tengan acceso a la misma tecnología y sean iguales en todos los demás aspectos (estructurales o institucionales), la teoría neoclásica del crecimiento predice que los países convergerán al mismo nivel de producción per cápita a largo plazo. El stock de capital (per cápita) será bajo para los países inicialmente pobres, lo que implica un alto retorno de la inversión (acumulación de capital), lo que favorece el proceso de convergencia.\\
Habrá una convergencia absoluta: los países terminarán con el mismo nivel de equilibrio de producción (y capital) per cápita. Aunque la convergencia puede ser lenta, el modelo de crecimiento neoclásico predice que los países pobres se pondrán al día, y que es mejor pensar que las diferencias reales en las tasas de crecimiento reflejan este proceso de convergencia. En un mundo así, la aglomeración espacial del PIB per cápita alto o bajo no merece mucha atención. La versión básica del modelo de crecimiento neoclásico tiene dificultades para explicar los hechos estilizados del crecimiento. O la convergencia es extremadamente lenta, o la principal predicción de la teoría, la convergencia absoluta, es errónea. Hay dos opciones para mejorar este estado de cosas: (i) adaptar el modelo de crecimiento neoclásico para permitir diferencias persistentes, o (ii) proponer una teoría alternativa del crecimiento económico.\\
La opción (i) podría incluir la introducción de un tercer factor de producción, el capital humano, además del trabajo y el capital físico, o el uso de la convergencia condicional en lugar de la absoluta. Esta segunda posibilidad requiere alguna explicación. Bajo la convergencia condicional, los países ya no tienen acceso a la misma tecnología o pueden tener diferentes características institucionales. En consecuencia, no es necesario que los países converjan al mismo nivel de producción per cápita de equilibrio a largo plazo. En cambio, la convergencia está condicionada a las características de un país. Esto permite un vínculo débil entre la geografía y la teoría neoclásica del crecimiento: en la medida en que las diferencias entre países en tecnología o instituciones sean específicas de la ubicación, la geografía importa. Gallup, Sachs y Mellinger (1998) dan apoyo empírico en un escenario a través del país y Duncan Black y Henderson (2003) para el crecimiento de las ciudades de EE. UU., al mostrar que \textbf{las diferencias geográficas físicas, como el clima o el acceso al mar, tienen un fuerte impacto en el crecimiento económico}. Al igual que con la teoría neoclásica del comercio, el papel de la geografía en la teoría neoclásica del crecimiento es limitado e indirecto. Su impacto se determina fuera del modelo y no hay retroalimentación de las variables de crecimiento a las variables de ubicación. \\
La opción (ii) requiere el desarrollo de un modelo teórico alternativo a la teoría neoclásica del crecimiento. Desde el trabajo seminal de Paul Romer (1986, 1990; véase también Robert Lucas, 1988), esta vía de investigación se conoce como “nueva teoría del crecimiento”. Aunque los modelos pueden variar considerablemente, dos diferencias cruciales (entrelazadas) con el modelo de crecimiento neoclásico son los intentos de endogenizar el crecimiento económico y prescindir del supuesto de rendimientos decrecientes de los factores acumulables (van Marrewijk, 1999). Con respecto al uso de economías de escala, existen varias opciones en la nueva literatura sobre crecimiento. Inicialmente, la mayoría de los modelos usaban economías externas puras a nivel nacional o industrial, en lugar de economías internas a nivel de empresa. La investigación posterior también utilizó economías internas positivas bajo competencia imperfecta, similar a la nueva teoría comercial. Las economías externas positivas pueden dar lugar a efectos indirectos positivos y complementariedad estratégica y, por lo tanto, a equilibrios múltiples.\\
También existen otras posibilidades, como el célebre modelo AK con rendimientos constantes del capital a nivel de toda la economía, lo que implica que los países no convergen al mismo equilibrio de largo plazo. La endogenización del proceso de crecimiento se centra en el progreso tecnológico, una función positiva del stock general de capital o mano de obra, o de los gastos en I+D. Sea Y la producción, K el stock de capital, L el trabajo y A(.) la función tecnológica. Considere la siguiente función de producción:
\begin{equation}
Y=A(.)K^{1-b} L^b \qquad 0<b<1
\end{equation}
Esta función de producción se parece mucho a la conocida función de producción de Cobb-Douglas que suele subyacer en el modelo de crecimiento neoclásico. De hecho, si asumimos que A es una constante, tenemos el Cobb-Douglas función de producción, y se puede ver fácilmente que la producción per cápita, $Y/L$, está determinada por el stock de capital por trabajador, $K/L$. El crecimiento de este último está sujeto a rendimientos decrecientes (siempre que 0 < b < 1). Volviendo a la ecuación (2.2), las economías de escala no decrecientes pueden incorporarse especificando A(.) como una función del stock de capital agregado, el capital humano o alguna función dinámica de "innovación", que captura la acumulación de conocimiento agregado en el economía. El kid de todos estos intentos es garantizar que ya no haya rendimientos decrecientes para los factores de producción acumulables.\\
No es indiscutible si la nueva teoría del crecimiento es realmente diferente de la teoría neoclásica del crecimiento (Solow, 1994). Nuestro principal interés aquí está en el posible papel de la geografía en la nueva teoría del crecimiento. En muchas versiones de la nueva teoría del crecimiento no existe tal función. Para permitir que la ubicación sea relevante, los países deben diferir en algún aspecto. Tomemos, por ejemplo, las economías externas a escala. Si estos son los mismos para todos los países, la economía puede describirse mediante una función de producción global uniforme con rendimientos crecientes a escala. La ubicación no es irrelevante si los efectos indirectos asociados con las economías externas están de alguna manera localizados. Gene Grossman y Helpman (1991) analizan los efectos indirectos localizados si las externalidades positivas asociadas con la I+D o, de manera más general, con el conocimiento existen solo dentro de un determinado grupo de países. Este modelo está muy cerca en varios aspectos del modelo central de la economía geográfica (ver Brakman y Garretsen, 1993: 179). La existencia de externalidades localizadas y, por lo tanto, el rango geográfico limitado de la difusión del conocimiento puede deberse a diferencias culturales, políticas e institucionales, todo lo cual puede contribuir a la localización de estas economías externas. Pueden ayudar a explicar por qué algunos (grupos de) países no solo tienen una tasa de crecimiento y un nivel de producción per cápita más altos que otros, sino también por qué esta diferencia podría no disminuir con el tiempo, haciendo posible el equilibrio centro-periferia.\\
Los nuevos modelos de crecimiento pueden así dar cuenta de la aglomeración de la actividad económica. El problema es cómo se analiza la ubicación en sí. La introducción de la ubicación en la nueva teoría del crecimiento tiene un gran parecido con la relevancia de la ubicación en los modelos de crecimiento neoclásicos que permiten la convergencia condicional. En ambos casos, el papel de la ubicación no se deriva del modelo en sí, y en ambos casos se estipula, ya sea teórica o empíricamente, que la tasa de crecimiento del progreso tecnológico de un país depende de la ubicación de ese país. La conclusión debe ser que la ubicación todavía no forma parte del análisis y que la endogenización del crecimiento económico no se extiende al papel de la geografía. \textbf{Aunque algunas versiones de la nueva teoría del crecimiento son bastante similares en varios aspectos a los modelos de economía geográfica (rendimientos crecientes a escala, competencia imperfecta, productos diferenciados, equilibrios múltiples), la nueva teoría del crecimiento no ofrece una teoría de la ubicación.}

\paragraph{La relevancia de la geografía física}
\textbf{En su trabajo sobre la concentración geográfica de las industrias estadounidenses, Glenn Ellison y Glaeser (1997, 1999) argumentan que esta concentración puede darse básicamente por dos motivos. El primero es la existencia de tecnologías de rendimientos crecientes y otras economías de escala. Podríamos llamar a esto el papel de la economía en la geografía. De esto se tratan, ante todo, los modelos de economía geográfica. La segunda razón para la concentración es la existencia de ventajas de costos naturales que se deben a las diferencias en la geografía física real.} Ellison y Glaeser (1999: 315) concluyen que alrededor del 20 por ciento de la aglomeración observada de industrias estadounidenses puede explicarse por variables que miden las ventajas naturales. De manera similar, Jan Haaland, Hans Kind, Karen Midelfart Knarvik y Johan Torstensson (1999) concluyen en un contexto diferente que la concentración de la industria en Europa está significativamente determinada por las diferencias en las dotaciones en Europa. Aunque los efectos del mercado interno son aún más importantes para explicar la concentración geográfica de la industria, la relevancia de las dotaciones también implica (indirectamente) que la geografía física puede ser importante, ya que las diferencias en las dotaciones pueden deberse a diferencias en la geografía física.\\
\textbf{En su estudio sobre la evolución de las ciudades estadounidenses, Black y Henderson (2003) afirman que el desempeño del crecimiento de las ciudades puede diferir por dos razones: diferencias en la geografía física y diferencias debidas a los efectos de concentración enfatizados por la economía geográfica.} Estos últimos los miden a través del mercado potencial de cada ciudad. Con respecto a la geografía física, encuentran que las ciudades en climas cálidos (menos grados día de calentamiento) y más secos (menos precipitaciones) en la costa de hecho crecen más rápido. En cualquier caso, estos estudios apuntan a la relevancia de la geografía física para explicar la aglomeración de la actividad económica. \\
\textbf{En un gran estudio entre países, Gallup, Sachs y Mellinger (1998) investigan el impacto de la geografía física en el crecimiento económico. El punto de partida de su análisis es la observación de que prácticamente todos los países de los trópicos son pobres, mientras que los países situados fuera de los trópicos son casi invariablemente relativamente ricos (en PIB per cápita).} Además, los países costeros generalmente tienen ingresos más altos que los países sin buen acceso al mar. En una serie de estimaciones transversales, hacen retroceder el crecimiento económico sobre varios indicadores de la geografía física (controlando los determinantes estándar del crecimiento). Ellos también encuentran que la geografía física importa, no necesariamente de manera directa (la ubicación en los trópicos no es significativa pero la presencia de malaria reduce significativamente el crecimiento), pero la importancia de la ubicación de un país frente a la mar (ya sea por ser un país costero o un país con vías navegables que conducen al mar) da un impacto directo directo (positivo) de la geografía física en el crecimiento económico. La evidencia adicional de Mellinger, \textbf{Gallup y Sachs (2000) respalda la idea de que la geografía física es importante para explicar las diferencias de crecimiento e ingresos en todo el mundo.}\\
El punto que queremos enfatizar es que \textbf{la geografía física es importante para la ubicación de la actividad económica.} En la terminología utilizada en este capítulo: los aspectos de primera naturaleza de las elecciones de ubicación son importantes. Es necesario enfatizar esto, porque la economía geográfica, y por lo tanto \textbf{gran parte del resto de este libro, se enfoca en la segunda naturaleza de la elección de la ubicación, asumiendo a menudo el espacio (geografía física) es homogéneo.} Estos enfoques no están en conflicto entre sí, como también reconocen Gallup, Sachs y Mellinger (1998: 132): Los dos enfoques, por supuesto, pueden ser complementarios: una ciudad puede surgir debido a las ventajas de costos que surgen de la geografía diferenciada, pero continuar prosperar debido a las economías de aglomeración incluso cuando las ventajas de costos han desaparecido. El trabajo empírico debe apuntar a desentrañar las fuerzas de la geografía diferenciada y las economías de aglomeración autoorganizadas.

\subsection{Desarrollo económico}
Hasta ahora, hemos usado los términos crecimiento económico y desarrollo económico indistintamente. Sin embargo, dentro de la teoría económica, \textbf{el análisis del desarrollo económico generalmente se refiere a las condiciones bajo las cuales los países en desarrollo pueden lograr el crecimiento económico.} Este es un alcance algo más limitado que el análisis del crecimiento económico, que debería aplicarse a todos los países, pero se ocupa más del crecimiento continuo y menos de las condiciones previas para el crecimiento. Hoy en día, los estudios de desarrollo económico hacen un uso extensivo de la teoría del crecimiento económico (antigua y nueva). En este sentido, \textbf{la economía del desarrollo es en gran medida una aplicación de la economía convencional a los países en desarrollo.} No siempre ha sido así. Especialmente en las décadas de 1950 y 1960, economistas del desarrollo como Gunnar Myrdal (1957) y Albert Hirschman (1958), y otros como Perroux (1955), siguiendo a Rosenstein-Rodan (1943), utilizaron un marco analítico muy diferente al neoclásico. Enfoque que ahora es dominante en la economía del desarrollo. Este marco fue atacado por su (supuesta) falta de coherencia analítica, lo que no fue del todo sorprendente, porque estos autores, a veces explícitamente pero más a menudo implícitamente, se basó en economías externas y competencia imperfecta, conceptos que no se invocaban con frecuencia en la economía dominante en esos días. Las teorías utilizadas por estos economistas del desarrollo son interesantes no solo porque trataron de explicar las (ausentes) condiciones para el crecimiento económico en los países en desarrollo, sino porque también tenían un buen ojo para la dimensión geográfica del desarrollo económico, tanto dentro de los países en desarrollo como entre ellos. Países ricos y pobres. Esto explica por qué los geógrafos económicos hasta el día de hoy, al discutir el desarrollo económico, todavía se refieren al trabajo de estos economistas de desarrollo más antiguos.\\
Una vez más, no es nuestro objetivo examinar este campo. Por lo tanto, nos limitamos a una breve discusión de los conceptos principales. En el influyente análisis big push de Rosenstein-Rodan (1943), \textbf{un tamaño insuficiente del mercado local es visto como la principal causa del subdesarrollo de una región o un país.} La solución al subdesarrollo se encuentra en una expansión coordinada (es decir, dirigida por el gobierno) de la inversión que permita a las empresas cosechar los beneficios de las economías de escala (externas e internas), fomentando así la industrialización de la región o país atrasado. Una empresa individual no tiene incentivos para expandir su nivel de producción debido a la ausencia de rendimientos crecientes a escala a nivel de empresa. La expansión de la producción solo se vuelve rentable, y aquí las economías externas entran en escena, si un número suficiente de otras empresas también expanden su producción (de ahí el término gran impulso). La industrialización en la región o país atrasado también requiere la expansión de la mano de obra manufacturera. Si la mano de obra es inmóvil entre países, la expansión de la mano de obra manufacturera tiene que ocurrir atrayendo mano de obra de otros sectores de la economía (típicamente el sector agrícola), lo que requiere una oferta de mano de obra suficientemente elástica. Sin un gran impulso en la inversión, la periferia no puede alcanzar al centro, por así decirlo.\\
Myrdal (1957) también describe la sostenibilidad de los patrones de desarrollo económico centro-periferia. No enfatiza tanto las condiciones bajo las cuales el país o región atrasada puede iniciar un proceso de desarrollo económico. En cambio, argumenta que, si, por la razón que sea, una región o un país obtiene una ventaja inicial en términos de desarrollo económico, es muy probable que esta ventaja se refuerce a sí misma. Myrdal introduce el concepto de causalidad circular o acumulativa para describir este proceso. Una vez que un país o región toma la delantera en el desarrollo económico, las economías externas en este país o región asegurarán que se vuelva más y no, como predice el modelo de crecimiento neoclásico, menos interesante para que las empresas inviertan y para que la mano de obra se establezca en la región de más rápido crecimiento. La existencia de fuertes derrames localizados conduce al establecimiento de un centro (con el mercado relativamente más grande) y una periferia.\\
Hirschman (1958) también se centra en la naturaleza autorreforzante del (diferencias en) el desarrollo económico. Su uso de vínculos hacia atrás y hacia adelante puede considerarse como un medio para ilustrar cómo las empresas, al ubicar la producción en una región en particular, aumentan la rentabilidad de otras empresas que lo hacen. En la terminología moderna, las ideas presentadas por Hirschman tienen un claro sabor a rendimientos crecientes a escala. Cabe señalar que el uso de rendimientos crecientes a escala en los escritos de Rosenstein-Rodan, Myrdal o Hirschman es (en el mejor de los casos) indirecto, en el sentido de que no analizan del todo la relevancia de las economías de escala para el propio desarrollo económico. Esta relevancia se destila de sus trabajos a través de los ojos de la teoría económica moderna. De hecho, las dificultades para analizar el papel de los rendimientos crecientes y la competencia imperfecta aseguraron que el apogeo de esta rama de la economía del desarrollo fuera bastante efímero. La teoría neoclásica del desarrollo económico con su énfasis en la competencia perfecta y los rendimientos decrecientes siguió siendo mucho más influyente.\\
Hoy en día, esta última afirmación ya no se sostiene. Con el surgimiento de la nueva teoría del comercio, la nueva teoría del crecimiento y, por supuesto, la economía geográfica, \textbf{los rendimientos crecientes y la competencia imperfecta se han convertido más en una regla que en una excepción en la teoría económica.} Cuando se trata del desarrollo económico, estas nuevas teorías formalizan las ideas de Rosenstein Rodan cum suis y les dan a estas ideas una base microeconómica.\\

\section{Conclusiones}
¿Cuál es el mensaje principal de este capítulo? Básicamente, que todos los enfoques teóricos discutidos en las secciones anteriores tienen algo útil que decir sobre la relación entre geografía y economía, pero que cada enfoque también tiene sus limitaciones. Sin exagerar demasiado, \textbf{se puede argumentar que en la economía regional y urbana hay un amplio espacio para la geografía o el espacio en el análisis, pero que estos enfoques a veces carecen de la base microeconómica del comportamiento individual y de la estructura de equilibrio general que constituye la columna vertebral de la corriente principal.} Teoría económica en la actualidad. \textbf{Por el contrario, tanto para la antigua como para la nueva teoría del comercio y el crecimiento, existe tal base microeconómica y una estructura de equilibrio general, pero el problema es que la geografía suele ser casi irrelevante o, cuando importa, su función no lo es (suficientemente suficiente).}  Vinculado al comportamiento subyacente de los agentes económicos. Desde nuestro punto de vista, la economía geográfica puede verse como un intento de romper aún más la valla entre la geografía y la economía. Al hacerlo, tiene sus raíces firmemente en la teoría económica dominante, por lo que es en particular un intento de traer más geografía a la economía. Por esa razón, preferimos el término economía geográfica a nueva geografía económica.\\


