\chapter{Geografía y teoría económica}

\section{Introdución}

    La agrupación de actividades económicas se puede encontrar en varios niveles de agregación: la variación considerable en el tamaño económico de las ciudades o regiones a nivel nacional, o la distribución desigual de la riqueza y la producción a nivel mundial.\\
    Por supuesto, surge la pregunta de por qué la ubicación parece ser tan importante para las actividades económicas.

\section{La geografía en la economía regional y urbana}
La economía regional y la geografía económica ahora difieren. La economía regional (también conocida como ciencia regional) se basa en la teoría económica neoclásica y es, en efecto, la sucesora formalizada de la tradición alemana de la economía de ubicación. La geografía económica, por otro lado, es más ecléctica y orientada empíricamente. Se inspira en teorías económicas heterodoxas y, cada vez más, en áreas externas a la economía, como la sociología, las ciencias políticas y la teoría de la regulación.\\
Comenzamos con una descripción general de un campo de estudio más joven, a saber, la economía urbana, que estudia la estructura espacial de las áreas urbanas.

\subsection{La economía urbana}
La distribución desigual de la actividad económica dentro de cada país es el punto de partida para la economía urbana. El análisis moderno de la aglomeración de empresas y personas en ciudades o áreas metropolitanas se basa en gran medida en la economía de la aglomeración, un término que se refiere a la disminución de los costos promedio a medida que se produce una mayor producción dentro de un área geográfica específica. En otras palabras se base en rendimientos crecientes a escala. Antes de entrar en la relevancia de las economías de escala para las ciudades y otras formas de aglomeración, primero discutimos un modelo en el que no hay rendimientos crecientes a escala en absoluto. Este modelo, el modelo de ciudad monocéntrica, se origina con von Thu¨nen (1826) y sigue siendo un modelo de referencia para la economía urbana (y regional) hasta el día de hoy. Se justifica una breve discusión, aunque solo sea para poder notar las diferencias con el enfoque de la economía geográfica y dejar claro que, al final, el análisis de las ciudades seguirá siendo bastante limitado mientras no haya rendimientos crecientes a escala.

\subsubsection{El modelo de la ciudad monocéntrica}
El modelo de ciudad monocéntrica asume la existencia de un plano sin rasgos distintivos, perfectamente plano y homogéneo en todos los aspectos. En medio de este plano hay una sola ciudad. Donde existen  agricultores que quieren estar lo más cerca posible de la ciudad para minimizar sus costos de transporte. Este incentivo de estar cerca de la ciudad da como resultado mayores rentas de la tierra cerca de la ciudad que en el borde del plano. Por lo tanto, cada agricultor se enfrenta a una compensación entre las rentas de la tierra y los costos de transporte. Para cada tipo de cultivo existe una curva de oferta-renta que indica, dependiendo de la distancia a la ciudad, cuánto están dispuestos a pagar los agricultores por la tierra.\\
La eficiencia de la asignación de tierras en el modelo monocéntrico depende del supuesto de que no hay externalidades de ubicación.\\
Una serie de hechos estilizados sobre la estructura espacial urbana están de acuerdo con el modelo monocéntrico. Primero, la densidad de población disminuye con la distancia de los centros comerciales centrales. En segundo lugar, casi todas las ciudades importantes del mundo occidental se descentralizaron en el siglo XX (a medida que la gente comenzó a ubicarse más lejos del centro de la ciudad), lo que puede estar relacionado con una caída en los costos de transporte. El modelo monocéntrico también tiene algunas limitaciones serias. Mencionamos solo dos. Primero, el modelo no tiene en cuenta ninguna interacción entre ciudades; no puede ocuparse de los sistemas urbanos. Segundo, el modelo toma la existencia y ubicación de la ciudad como dadas y se enfoca en la ubicación de agricultores/viajeros fuera de la ciudad. \\\\
El término "economías de escala" o "rendimientos crecientes a escala" se refiere a una situación en la que un aumento en el nivel de producción implica una disminución en los costos promedio por unidad de producción para la empresa. Se traduce en una curva de costo promedio con pendiente negativa. Para identificar el origen de la caída de los costes medios, Se distingue entre economías de escala internas y externas. Con las economías de escala internas, la disminución de los costes medios se produce por un aumento del nivel de producción de la propia empresa. Cuanto más produce la empresa, mejor puede beneficiarse de las economías de escala y mayor es su ventaja de costos sobre las empresas más pequeñas. La estructura de mercado que subyace a las economías de escala internas, típicamente utilizada en la literatura de economía geográfica, debe ser necesariamente de competencia imperfecta, ya que las economías de escala internas implican poder de mercado.\\
Con las economías de escala externas, la disminución de los costes medios se produce a través de un aumento de la producción a nivel de la industria en su conjunto, lo que hace que los costes medios por unidad sean una función de la producción de toda la industria. Se distingue aquí entre economías externas puras y pecuniarias. Con economías externas puras (o tecnológicas), un aumento en la producción de toda la industria altera la relación tecnológica entre insumos y producción para cada empresa individual. Por lo tanto, tiene un impacto en la función de producción de la empresa. Un ejemplo de uso frecuente (que se remonta a Alfred Marshall; se refiere a los derrames de información (El derrame de conocimiento ocurre cuando las empresas receptoras explotan el conocimiento que ha sido desarrollado originalmente por una empresa de origen). Estas empresas receptoras pueden ser socios de la alianza, competidores directos de la empresa de origen o empresas de otros sectores industriales). Un aumento en la producción de la industria aumenta el acervo de conocimiento a través de los efectos indirectos positivos de información para cada empresa, lo que lleva a un aumento en la producción a nivel de empresa. En la economía urbana, pero también en la nueva teoría del crecimiento y la nueva teoría del comercio se supone que existen economías externas puras. La estructura del mercado puede entonces ser perfectamente competitiva ya que el tamaño de la empresa individual no importa.\\
Las economías externas pecuniarias son transmitidas por el mercado a través de efectos de precio para la empresa individual, lo que puede alterar su decisión de producción. Dos ejemplos, de nuevo basados en Marshall, son la existencia de un gran mercado local de insumos especializados y la puesta en común del mercado laboral. Una gran industria puede respaldar un mercado de insumos intermedios especializados y un grupo de trabajadores calificados específicos de la industria, lo que beneficia a la empresa individual. A diferencia de las economías externas puras, estos efectos indirectos no afectan la relación tecnológica entre insumos y productos (la función de producción). Las externalidades pecuniarias existen en la literatura de economía geográfica a través de un efecto de amor por la variedad en un gran mercado local. La utilidad de cada consumidor depende positivamente del número de variedades que puede comprar de un bien manufacturado. Los efectos de precio cruciales para las externalidades pecuniarias solo pueden ocurrir con competencia imperfecta. Esto es consistente con el requisito de competencia imperfecta para las economías de escala internas, también utilizado en la literatura de economía geográfica. \\
Algunas observaciones finales están en orden. Primero, los derrames o externalidades son cruciales para las economías externas. El concepto de derrames a veces se utiliza solo para economías externas puras, refiriéndose a las economías externas pecuniarias como un caso de interdependencia del mercado. Nos ceñimos al uso de derrames o externalidades cuando nos referimos a economías de escala externas en general. De manera similar, el término "rendimientos crecientes" a veces se usa solo para economías de escala internas. También usamos la frase “rendimientos crecientes” cuando hablamos de economías externas. Del contexto quedará claro si nos referimos al nivel de la empresa o de la industria. En segundo lugar, las economías externas pueden aplicarse a un nivel de agregación superior al de la empresa. Este suele ser el nivel de la industria, pero en la teoría moderna del comercio y la teoría moderna del crecimiento también puede ser la economía en su conjunto. Tercero, las economías externas en los modelos son estáticas, mientras que la literatura también considera economías externas dinámicas. En ese caso, los costos promedio por unidad de producción son una función negativa de la producción acumulada de la industria. Nuevamente, si esto es relevante, quedará claro si nos referimos a economías externas estáticas o dinámicas. Cuarto, las economías externas discutidas anteriormente son positivas pero también pueden ser negativos, es decir, un aumento en la producción de una empresa conduce a un aumento en los costos por unidad para otras empresas. Finalmente, una observación sobre la terminología un tanto confusa con respecto a las externalidades económicas regionales o efectos indirectos. Es habitual distinguir entre las externalidades Marshall-Arrow-Romer (MAR) y las externalidades de Jacobs. En ambos casos, el énfasis está en los efectos indirectos regionales (específicos de la ubicación), es decir, las empresas deben estar ubicadas lo suficientemente cerca unas de otras para beneficiarse de estas externalidades. Las externalidades SAM se centran en los efectos indirectos específicos del sector y también se conocen como economía de localización. Las externalidades de Jacobs se centran en los efectos indirectos específicos de la ciudad que cruzan los límites entre sectores individuales. Estos también se conocen como economía de la urbanización. Dado que la distinción entre las externalidades MAR y Jacobs como tales no es descriptiva (y por lo tanto no informativa) y la distinción de localización/urbanización puede ser confusa (ya que ambas son específicas de la ubicación), tal vez sea mejor hablar de sector específico y de ciudad.

\begin{center}
    \includegraphics[scale=0.5]{imagen/imagen1.png}
\end{center}

\subsubsection{Economía urbana y rendimientos crecientes}
El punto de partida es bastante diferente del modelo monocéntrico. No hay costes de transporte y el interior de una ciudad ya no forma parte del análisis. En cierto sentido, es un análisis de las ciudades en las que el espacio, es decir, el espacio fuera de las ciudades, no tiene ningún papel que desempeñar. La justificación de este descuido geográfico del espacio no urbano es que, en los países industrializados modernos, una gran parte de la actividad económica general y de la población se encuentra en áreas urbanas, de modo que la relevancia de lo urbano frente a lo no urbano Se supone que las transacciones son limitadas. En cambio, el análisis se centra en las fuerzas que determinan el tamaño de las ciudades y las interacciones entre ellas. La justificación de este descuido geográfico del espacio no urbano es que, en los países industrializados modernos, una gran parte de la actividad económica general y de la población se encuentra en áreas urbanas, de modo que se supone que la relevancia de las transacciones urbanas frente a las no urbanas es limitada. En cambio, el análisis se centra en las fuerzas que determinan el tamaño de las ciudades y las interacciones entre ellas.\\
Las fuerzas de aglomeración en el modelo de Henderson son economías de escala externas positivas que son específicas de la industria. Esto último significa que hay excedentes positivos cuando una empresa de una industria en particular se ubica en una ciudad donde se encuentran otras empresas de la misma industria. \\
Usando una categorización bien conocida que se remonta a los trabajos de Marshall, estos pueden deberse a (i) el intercambio de información, (ii) la existencia de una gran cantidad de mano de obra, o (iii) la existencia de especialistas proveedores. Por lo tanto, las economías externas pueden, en principio, involucrar economías externas puras (como en el enfoque original de Henderson) o economías externas pecuniarias.

\subsubsection{¿rue tipo de economías de escala externas?}
\textbf{Estas economías externas específicas de la industria se conocen como economías de localización, a diferencia de las economías de urbanización.} Si las economías externas son importantes, probablemente sea más importante tener una variedad de industrias diversificadas en una ciudad. Si este último es el caso, surge la pregunta de por qué tantas ciudades están especializadas en industrias particulares. Glaeser et al. (1992: 1148-50) sugieren que tanto las economías de localización como las de urbanización son relevantes (aunque al final favorecen las economías de urbanización), mientras que Duncan Black y Henderson (1999a) argumentan que en un contexto dinámico las economías de localización son más relevantes.\\
Desde un punto de vista teórico, cabe destacar que el enfoque de los sistemas urbanos de Henderson no da por sentada la existencia de la ciudad, como hacía el modelo monocéntrico. También proporciona una teoría de las interacciones entre ciudades. El problema con el enfoque es que el espacio fuera de las ciudades (deliberadamente) no forma parte del análisis. Esto es problemático si uno quiere poder decir dónde están ubicadas las ciudades en relación con otras y la parte no urbana de la geografía: La literatura sobre sistemas de ciudades ha enfatizado el espacio urbano pero ha descuidado el espacio nacional. Como veremos, la ubicación de la actividad manufacturera y la relación entre estas ubicaciones y el resto del espacio es un tema clave en la economía geográfica. Para analizar esta relación, los costos de transporte deben ser parte del análisis, ya que son cruciales para determinar el equilibrio entre las fuerzas de aglomeración y expansión. \\
El uso de rendimientos crecientes y externalidades corresponde a nuestra definición de economías externas puras y economías externas pecuniarias, respectivamente. Ambos tipos de economías externas son importantes. Esto deja la competencia espacial, lo que significa que la competencia entre empresas es casi automáticamente de naturaleza oligopólica cuando se toma en consideración el espacio. La competencia está restringida por la distancia; Por lo general, se piensa que una empresa compite solo con sus empresas vecinas. La competencia espacial está, por tanto, intrínsecamente ligada al comportamiento estratégico de las empresas.  La razón es simplemente que en la economía geográfica y en particular en la versión de competencia monopolística de Avinash Dixit y Joseph Stiglitz (1977) , que caracteriza la estructura del mercado en nuestro modelo central de economía geográfica, el comportamiento estratégico no es tenido en cuenta. Las empresas toman el comportamiento (fijación de precios) de las demás como dado. Además de los tres enfoques mencionados por Fujita y Thisse (1996), Anas, Arnott y Small (1998) dan dos razones adicionales para la aglomeración (urbana): la existencia de un espacio no homogéneo y economías de escala internas en un proceso de producción. Con el primero se puede racionalizar la aglomeración sin ninguna forma de rendimientos crecientes a escala (piense en las diferencias en la geografía física real que da lugar, por ejemplo, a un puerto natural y la aglomeración correspondiente). 

\subsection{Economía regional}
\textbf{La economía regional analiza la organización espacial de los sistemas económicos} (y no solo de las ciudades) y de alguna manera también debe dar cuenta de la distribución desigual en el espacio. von Thu¨nen (1826), Wilhelm Launhardt (1885), Weber (1909), Christaller (1933) y Lo¨sch (1940). Todas estas contribuciones alemanas toman en consideración el espacio nacional o de toda la economía para analizar dónde se ubican las actividades económicas. \\
Esta es una pregunta relevante ya que el movimiento de bienes y personas no es gratuito y la producción suele estar sujeta a alguna forma de rendimientos crecientes. Sin embargo, los padres fundadores de la economía regional se centran en diferentes aspectos de la ubicación de la actividad económica.\\
Von Thu¨nen, por ejemplo, enfatizó las decisiones de ubicación tomadas por los agricultores, mientras que Weber analizó la ubicación óptima y el tamaño de la planta para las empresas manufactureras. Esta sub-sección se centra en las ideas presentadas (y probadas) por primera vez por Christaller y Lo¨sch, \textbf{quienes trataron no solo de explicar la ubicación de las ciudades sino también de diferenciarlas por las diversas funciones que desempeñan y de tratar las relaciones entre las ciudades y el medio ambiente. $"$no ciudades$"$. Este enfoque se conoce como la teoría del lugar central}, que muestra que diferentes puntos o ubicaciones en el panorama económico tienen diferentes niveles de centralidad y que los bienes y servicios se proporcionan de manera eficiente sobre una base jerárquica.

\subsubsection{Teoría del lugar central}
Dada una distribución uniforme de consumidores idénticos en un plano homogéneo, \textbf{la teoría del lugar central sostiene que las ubicaciones difieren en centralidad y que esta centralidad determina el tipo de bienes que proporciona la ubicación.} La provisión de estos bienes está determinada por rendimientos internos crecientes a escala, mientras que la ubicación es relevante porque los consumidores incurren en costos de transporte. Para minimizar estos costos, los consumidores quieren tener acceso a proveedores de bienes cercanos. Para algunos tipos de bienes, como el pan, esto es más fácil que para otros. Por lo tanto, la economía puede sustentar muchos lugares relativamente pequeños (pueblos) donde los panaderos están activos para suministrar pan. Por el contrario, solo puede haber relativamente pocos lugares (ciudades pequeñas, los lugares centrales) donde las empresas de electrónica vendan televisores, que la gente compra con menos frecuencia. Para minimizar los costos de transporte, ambos tipos de ubicaciones se distribuyen de manera bastante uniforme en el espacio. Además, obtenemos una jerarquía de lugares donde la ciudad realiza todas las funciones (vende pan y televisores), mientras que el pueblo realiza solo algunas funciones (vende solo pan). 
El hecho de que se ocupe explícitamente de la ubicación de la actividad económica es una ventaja importante de la teoría del lugar central. El principal problema con el enfoque es que la lógica económica detrás de las decisiones de los consumidores y las empresas sigue sin estar clara. ¿Qué tipo de comportamiento de los agentes individuales conduce a un resultado de lugar central? Los rendimientos crecientes a nivel de empresa requieren alguna forma de competencia imperfecta, un análisis que falta. \\
Los científicos regionales y los geógrafos económicos también han sido conscientes de las limitaciones de esta versión de la teoría del lugar central, que durante los últimos treinta años ha recibido menos interés, particularmente dentro de la geografía económica. Para una base teórica, los geógrafos económicos han comenzado a buscar en otra parte. Sin embargo, siguiendo a Walter Isard (1956, 1960), los científicos regionales han tratado de construir sobre las ideas básicas de la teoría del lugar central para dar una base económica teórica (a menudo altamente formalizada) a esta teoría. Estos modelos son en su mayoría de naturaleza de equilibrio parcial, explicando algunos aspectos del sistema de lugar central mientras ignoran otros. Por lo general, un modelo en esta tradición no trata con empresas o consumidores individuales, sino que formaliza esencialmente el patrón geométrico de un sistema de lugar central. El resultado del lugar central es, por lo tanto, meramente racionalizado y no explicado por el comportamiento subyacente de los consumidores y productores, ni por sus decisiones e interacciones (de mercado). Por ejemplo, la curva de demanda que enfrenta una empresa en un lugar particular no se deriva de los primeros principios, sino que simplemente se supone. 

\subsubsection{Potencial de mercado}
