\chapter{El modelo central de la economía geográfica}

\section{Introducción}
Hace mucho tiempo observó que los campos de la teoría del comercio por un lado y la economía regional y urbana por el otro tenían, en principio, los mismos objetivos de investigación. Ambas áreas de investigación quieren responder a las preguntas: "¿Quién produce qué, dónde y por qué?" A pesar de la observación de Ohlin, cada campo ha seguido su propio camino desde el siglo XIX. De todas formas La teoría del comercio y la economía regional y urbana se combinan productivamente en la economía geográfica.\\
Se analizará un pequeño modelo de equilibrio general desarrollado por Krugman (1991a, 1991b). Como veremos, las ecuaciones de equilibrio de este modelo no son lineales. Esto significa que los pequeños cambios en los parámetros no siempre producir los mismos efectos; a veces los efectos son pequeños, a veces son grandes. 

\section{Un ejemplo de economía geográfica}
Es posible construir un ejemplo simple para ilustrar algunos de los principales hallazgos del enfoque de la economía geográfica. Supongamos que hay dos regiones (o países), Norte y Sur, y dos sectores de producción, manufactura y agricultura. La industria manufacturera produce variedades, es decir, productos diferenciados, bajo economías internas de escala. Por lo tanto, el costo por unidad de producción cae a medida que una empresa expande su nivel de producción. Como resultado, cada empresa produce solo una variedad. Una empresa puede residir en el Norte o en el Sur, es decir, una empresa tiene que decidir dónde producir. Esta decisión de ubicación diferencia esencialmente el ejemplo de la nueva teoría comercial.\\
La demanda total de cada variedad de manufacturas en este ejemplo es exógena. Suponemos que cada empresa vende cuatro unidades a los trabajadores de la industria manufacturera y seis unidades a los agricultores. Por lo tanto, la demanda total de cada variedad es diez. La producción de la agricultura, y por lo tanto la demanda que genera, es específica de la ubicación. Su distribución espacial está dada exógenamente; suponemos que se venden cuatro unidades en el Norte y dos unidades en el Sur. La ubicación de los trabajadores en el sector manufacturero y, por tanto, las cuatro unidades que demandan en esa ubicación, no es exógena. El papel de los trabajadores inmóviles es importante, ya que aseguran que siempre haya una demanda positiva en ambas regiones. Finalmente, los costos de transporte entre el Norte y el Sur son $1$ dolar por unidad. Las empresas deciden su ubicación para minimizar los costos de transporte.\\
Ahora podemos determinar la decisión de ubicación de cada empresa. Primero, podemos calcular las ventas regionales de cada empresa, dada la ubicación de las otras empresas. Se dan tres posibilidades (no exhaustivas): todas las empresas en el Norte, todas las empresas en el Sur y el 25 por ciento de todas las empresas en el Norte y el 75 por ciento de todas las empresas en el Sur. Las ventas en cada región son iguales a las ventas a los trabajadores en la manufactura más las ventas a los agricultores. La empresa vende cinco unidades en North, a saber, cuatro a los agricultores ubicados en North más  unidad a los trabajadores de manufactura ubicados en North. De manera similar, la empresa vende cinco unidades en el Sur, es decir, dos unidades a los agricultores ubicados en el Sur más tres unidades a los trabajadores de manufactura ubicados en el Sur.\\
En segundo lugar, podemos construir una tabla de decisión, calculando los costos de transporte en función de la decisión de ubicación de la empresa, dada la ubicación de las otras empresas. Suponga, por ejemplo, que todas las empresas están ubicadas en el norte. Entonces los costos de transporte para una empresa ubicada en el Sur serán entonces ocho dolares, es decir, 4 dolares para las ventas a los agricultores del Norte y 4 dolares para las ventas a todos los trabajadores de la industria manufacturera ubicada en el Norte. De manera similar, si la empresa se ubica en el norte, los costos de transporte serían solo dos dolares para las ventas a los agricultores del sur. Dado que los costos de transporte se minimizan al ubicarse en el Norte si todas las demás empresas están ubicadas en el Norte, la empresa decide ubicar también la producción en el Norte. Una empresa se ubicará en el Sur si todas las demás empresas también se ubican allí, mientras que a la empresa le es indiferente ubicarse en el Norte o en el Sur (ya que los costos de transporte son los mismos si la empresa se ubica en el Sur). En cualquier región si el 25 por ciento de las empresas están ubicadas en el Norte y el 75 por ciento en el Sur.\\
Sobre la base de este ejemplo, podemos ilustrar algunas características distintivas del enfoque de la economía geográfica.\\
En primer lugar, el concepto de causalidad acumulativa. \textbf{Si, por alguna razón, una ubicación ha atraído a más empresas que la otra ubicación, una nueva empresa tiene un incentivo para ubicarse donde están las otras empresas.} Si todas las empresas existentes están ubicadas en el norte, la nueva empresa también debería ubicarse allí si desea minimizar sus costos de transporte.\\
En segundo lugar. La aglomeración de todas las empresas en el Norte o en el Sur es un equilibrio. Sin embargo, no podemos determinar de antemano dónde ocurrirá la aglomeración. Esto depende críticamente de las condiciones iniciales, es decir, las decisiones previas de ubicación de otras empresas.\\
Tercero, un equilibrio puede ser estable o inestable. Si una sola empresa decide mudarse, esto decisión no influiría en las decisiones de ubicación de las otras empresas. Si una sola empresa decide trasladarse, la nueva ubicación se volverá inmediatamente más atractiva para todas las demás empresas. Esto desencadenará un efecto bola de nieve: todas las empresas seguirán al pionero. En este ejemplo, solo la aglomeración es un equilibrio estable.\\
En cuarto lugar, observamos que un equilibrio estable puede no ser óptimo. Si todas las empresas están ubicadas en el norte, los costos de transporte son solo e2. Si todas las empresas están ubicadas en el sur, los costos de transporte son e4 (ver las entradas en negrita en la tabla 3.2). Por lo tanto, los costos de transporte para la economía en su conjunto se minimizan si todas las empresas se aglomeran en el Norte, mientras que la aglomeración en el Sur sigue siendo un equilibrio estable.\\
En quinto lugar, el ejemplo ilustra la interacción de la aglomeración y los flujos comerciales. Con la aglomeración completa, es decir, todas las manufacturas se producen en una sola región, el comercio entre regiones será de tipo interindustrial (alimentos para manufacturas). De hecho, este equilibrio también refleja el llamado efecto del mercado interior; la combinación de economías de escala y costos de transporte es responsable del agrupamiento de toda la actividad libre en un solo lugar. Debido a esta combinación, los costos de transporte pueden minimizarse. La gran región termina con un gran mercado para la fabricación de bienes, que pueden venderse sin incurrir en costos de transporte. La consecuencia es que esta región se convierte en exportadora de productos manufacturados; \textbf{Las grandes regiones tienden a convertirse en exportadoras de aquellos bienes para los que tienen un gran mercado local, de ahí el término efecto del mercado interno.} Si la industria manufacturera está ubicada en ambas regiones, el comercio también será del tipo intraindustrial. Además de intercambiar bienes manufacturados por productos agrícolas, se intercambiarán diferentes variedades de productos manufacturados diferenciados entre ambas regiones.\\
El ejemplo es útil, ya que ilustra algunos aspectos importantes de la economía geográfica. Sin embargo, un ejemplo es solo un ejemplo y no es un sustituto de un modelo bien especificado. ¿Qué le falta?
\begin{itemize}
    \item En primer lugar, falta la interacción entre los costos de transporte, el comportamiento de fijación de precios y la elección de la ubicación. 
    \item Además, es un modelo de equilibrio parcial, en el sentido de que las empresas no se preocupan por la mano de obra necesaria; donde sea que decidan ubicarse, la disponibilidad de mano de obra no es el problema. 
\end{itemize}

\section{La estructura del modelo}


