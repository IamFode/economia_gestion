\chapter{El modelo central de la economía geográfica}

\section{Introducción}
Hace mucho tiempo observó que los campos de la teoría del comercio por un lado y la economía regional y urbana por el otro tenían, en principio, los mismos objetivos de investigación. Ambas áreas de investigación quieren responder a las preguntas: "¿Quién produce qué, dónde y por qué?" A pesar de la observación de Ohlin, \textbf{cada campo ha seguido su propio camino desde el siglo XIX. De todas formas La teoría del comercio y la economía regional y urbana se combinan productivamente en la economía geográfica.}\\
Se analizará un pequeño modelo de equilibrio general desarrollado por Krugman (1991a, 1991b). Como veremos, las ecuaciones de equilibrio de este modelo no son lineales. Esto significa que los pequeños cambios en los parámetros no siempre producir los mismos efectos; a veces los efectos son pequeños, a veces son grandes. 

\section{Un ejemplo de economía geográfica}
Supongamos que hay dos regiones (o países), Norte y Sur, y dos sectores de producción, manufactura y agricultura. La industria manufacturera produce variedades, es decir, productos diferenciados, bajo economías internas de escala. Por lo tanto, el costo por unidad de producción cae a medida que una empresa expande su nivel de producción. Como resultado, cada empresa produce solo una variedad. Una empresa puede residir en el Norte o en el Sur, es decir, una empresa tiene que decidir dónde producir. Esta decisión de ubicación diferencia esencialmente el ejemplo de la nueva teoría comercial.\\
La demanda total de cada variedad de manufacturas en este ejemplo es exógena. Suponemos que cada empresa vende cuatro unidades a los trabajadores de la industria manufacturera y seis unidades a los agricultores. Por lo tanto, la demanda total de cada variedad es diez. La producción de la agricultura, y por lo tanto la demanda que genera, es específica de la ubicación. Su distribución espacial está dada exógenamente; suponemos que se venden cuatro unidades en el Norte y dos unidades en el Sur. La ubicación de los trabajadores en el sector manufacturero y, por tanto, las cuatro unidades que demandan en esa ubicación, no es exógena. El papel de los trabajadores inmóviles es importante, ya que aseguran que siempre haya una demanda positiva en ambas regiones. Finalmente, los costos de transporte entre el Norte y el Sur son $1$ dolar por unidad. Las empresas deciden su ubicación para minimizar los costos de transporte.\\
Ahora podemos determinar la decisión de ubicación de cada empresa. Primero, podemos calcular las ventas regionales de cada empresa, dada la ubicación de las otras empresas. Se dan tres posibilidades (no exhaustivas): todas las empresas en el Norte, todas las empresas en el Sur y el 25 por ciento de todas las empresas en el Norte y el 75 por ciento de todas las empresas en el Sur. Las ventas en cada región son iguales a las ventas a los trabajadores en la manufactura más las ventas a los agricultores. La empresa vende cinco unidades en North, a saber, cuatro a los agricultores ubicados en North más  unidad a los trabajadores de manufactura ubicados en North. De manera similar, la empresa vende cinco unidades en el Sur, es decir, dos unidades a los agricultores ubicados en el Sur más tres unidades a los trabajadores de manufactura ubicados en el Sur.\\
En segundo lugar, podemos construir una tabla de decisión, calculando los costos de transporte en función de la decisión de ubicación de la empresa, dada la ubicación de las otras empresas. Suponga, por ejemplo, que todas las empresas están ubicadas en el norte. Entonces los costos de transporte para una empresa ubicada en el Sur serán entonces ocho dolares, es decir, 4 dolares para las ventas a los agricultores del Norte y 4 dolares para las ventas a todos los trabajadores de la industria manufacturera ubicada en el Norte. De manera similar, si la empresa se ubica en el norte, los costos de transporte serían solo dos dolares para las ventas a los agricultores del sur. Dado que los costos de transporte se minimizan al ubicarse en el Norte si todas las demás empresas están ubicadas en el Norte, la empresa decide ubicar también la producción en el Norte. Una empresa se ubicará en el Sur si todas las demás empresas también se ubican allí, mientras que a la empresa le es indiferente ubicarse en el Norte o en el Sur (ya que los costos de transporte son los mismos si la empresa se ubica en el Sur). En cualquier región si el 25 por ciento de las empresas están ubicadas en el Norte y el 75 por ciento en el Sur.\\
Sobre la base de este ejemplo, podemos ilustrar algunas características distintivas del enfoque de la economía geográfica.\\
En primer lugar, el concepto de causalidad acumulativa. \textbf{Si, por alguna razón, una ubicación ha atraído a más empresas que la otra ubicación, una nueva empresa tiene un incentivo para ubicarse donde están las otras empresas.} Si todas las empresas existentes están ubicadas en el norte, la nueva empresa también debería ubicarse allí si desea minimizar sus costos de transporte.\\
En segundo lugar. La aglomeración de todas las empresas en el Norte o en el Sur es un equilibrio. Sin embargo, no podemos determinar de antemano dónde ocurrirá la aglomeración. Esto depende críticamente de las condiciones iniciales, es decir, las decisiones previas de ubicación de otras empresas.\\
Tercero, un equilibrio puede ser estable o inestable. Si una sola empresa decide mudarse, esto decisión no influiría en las decisiones de ubicación de las otras empresas. Si una sola empresa decide trasladarse, la nueva ubicación se volverá inmediatamente más atractiva para todas las demás empresas. Esto desencadenará un efecto bola de nieve: todas las empresas seguirán al pionero. En este ejemplo, solo la aglomeración es un equilibrio estable.\\
En cuarto lugar, observamos que un equilibrio estable puede no ser óptimo. Si todas las empresas están ubicadas en el norte, los costos de transporte son solo 2 euros. Si todas las empresas están ubicadas en el sur, los costos de transporte son 4 euros. Por lo tanto, los costos de transporte para la economía en su conjunto se minimizan si todas las empresas se aglomeran en el Norte, mientras que la aglomeración en el Sur sigue siendo un equilibrio estable.\\
En quinto lugar, el ejemplo ilustra la interacción de la aglomeración y los flujos comerciales. Con la aglomeración completa, es decir, todas las manufacturas se producen en una sola región, el comercio entre regiones será de tipo interindustrial (alimentos para manufacturas). De hecho, este equilibrio también refleja el llamado efecto del mercado interior; la combinación de economías de escala y costos de transporte es responsable del agrupamiento de toda la actividad libre en un solo lugar. Debido a esta combinación, los costos de transporte pueden minimizarse. La gran región termina con un gran mercado para la fabricación de bienes, que pueden venderse sin incurrir en costos de transporte. La consecuencia es que esta región se convierte en exportadora de productos manufacturados; \textbf{Las grandes regiones tienden a convertirse en exportadoras de aquellos bienes para los que tienen un gran mercado local, de ahí el término efecto del mercado interno.} Si la industria manufacturera está ubicada en ambas regiones, el comercio también será del tipo intraindustrial. Además de intercambiar bienes manufacturados por productos agrícolas, se intercambiarán diferentes variedades de productos manufacturados diferenciados entre ambas regiones.\\
El ejemplo es útil, ya que ilustra algunos aspectos importantes de la economía geográfica. Sin embargo, un ejemplo es solo un ejemplo y no es un sustituto de un modelo bien especificado. ¿Qué le falta?
\begin{itemize}
    \item En primer lugar, falta la interacción entre los costos de transporte, el comportamiento de fijación de precios y la elección de la ubicación. 
    \item Además, es un modelo de equilibrio parcial, en el sentido de que las empresas no se preocupan por la mano de obra necesaria; donde sea que decidan ubicarse, la disponibilidad de mano de obra no es el problema. 
\end{itemize}

\section{La estructura del modelo}

El modelo central identifica dos regiones, etiquetadas como 1 y 2. Hay dos sectores en la economía, el sector manufacturero y el sector alimentario. Los consumidores en ambas regiones consisten en trabajadores agrícolas y trabajadores de manufactura. Los trabajadores agrícolas obtienen sus ingresos trabajando en las granjas de su región. El flujo de ingresos de los trabajadores agrícolas es parte de una transferencia bilateral: obtienen un ingreso a cambio de su oferta de mano de obra.\\
El sector manufacturero consta de empresas $N_1$ en la región 1 y empresas $N_2$ en la región 2. Cada empresa manufacturera produce un producto diferenciado, es decir, produce una variedad única de manufacturas. Utiliza solo mano de obra en el proceso de producción, que se caracteriza por economías internas de escala. Esto implica que las empresas tienen poder de monopolio, que utilizan para determinar el precio de su producto. Además, los costos de transporte están involucrados en la venta de un bien manufacturado en otra región. Estos costos no surgen si el bien manufacturado se vende en la región en la que se produce. Como resultado de los costos de transporte, las empresas exportadoras cobrarán un precio más alto en la región extranjera que en la región de origen. Los trabajadores manufactureros obtienen sus ingresos (la tasa salarial manufacturera) al suministrar mano de obra a las empresas del sector manufacturero ubicadas en la región de origen.\\
Los consumidores gastan sus ingresos tanto en alimentos como en manufacturas. Dado que los alimentos son un bien homogéneo, no les importa si se producen en la región 1 o en la región 2. Como los alimentos no tienen costos de transporte, tienen el mismo precio en ambas regiones (lo que implica que los agricultores ganan el mismo salario en ambas regiones). El gasto de los consumidores en manufacturas debe distribuirse entre las muchas variedades producidas en las regiones 1 y 2. En igualdad de condiciones, consumir variedades importadas es más costoso que consumir variedades nacionales, como resultado de los costos de transporte de los productos manufacturados. Sin embargo, dado que las variedades son productos diferenciados y los consumidores tienen gusto por la variedad, siempre consumirán algunas unidades de todas las variedades producidas, ya sea en el país o en el extranjero.\\

\section{Demanda}
\subsection{Gasto en alimentos y manufacturas (a)}
La economía tiene dos sectores de bienes, manufacturas, $M$, y alimentos, $F$. Aunque las manufacturas consisten en muchas variedades diferentes, podemos definir un índice de precios exacto para representarlas como un grupo. A este índice de precios de manufacturas lo llamamos $I$. Si un consumidor obtiene un ingreso $Y$ (por trabajar en el sector alimentario o en el sector manufacturero), tiene que decidir cuánto de este ingreso se gasta en alimentos y cuánto en manufacturas. La solución a este problema depende de las preferencias del consumidor, que se suponen de la especificación Cobb-Douglas para todos los consumidores, en la que $F$ representa el consumo de alimentos y $M$ representa el consumo de manufacturas:
$$U=F^{1-\delta}M^\delta\quad 0<\delta<1$$
de donde el consumidor debe satisfacer la restricción presupuestaria en la ecuación.
$$F+I\cdot M = Y$$
Nótese la ausencia del precio de los alimentos en esta ecuación. Esto es el resultado de elegir los alimentos como numerario, lo que implica que el ingreso $Y$ se mide en términos de alimentos. Por lo tanto, solo el índice de precios de las manufacturas $I$ aparece en la ecuación. Para decidir sobre la asignación óptima de ingresos sobre la compra de alimentos y manufacturas, el consumidor ahora tiene que resolver un problema de optimización simple, a saber, maximizar la utilidad como se indica en la ecuación $U=F^{1-\delta}M^\delta\quad 0<\delta<1$, sujeto a la restricción presupuestaria de la ecuación $F+I\cdot M = Y$. La solución a este problema se da en la siguiente ecuación y se derivada como:
$$F=(1-\delta)Y;\quad IM=\delta Y$$
De ahora en adelante nos referiremos al parámetro $\delta$ dado en la ecuación $U=F^{1-\delta}M^\delta\quad 0<\delta<1$ como la fracción del ingreso gastado en manufacturas.\\\\

Para maximizar la ecuación $U=F^{1-\delta}M^\delta\quad 0<\delta<1$ sujeta a la restricción presupuestaria 
$F+I\cdot M = Y$, definimos el Lagrangiano $\Gamma$, usando el multiplicador $k$:
$$\Gamma = F_{1-\delta M^{\delta}}+k[Y-IM]$$
La diferenciación de $\Gamma$ con respecto a $F$ y $M$ da las condiciones de primer orden
$$(1-\delta)F^{-\delta}M^{\delta};\quad \delta F^{1-\delta}M^{\delta-1} = kI$$
Tomando la relación de las condiciones de primer orden se obtiene
$$\dfrac{\delta F^{1-\delta} M^{\delta-1}}{(1-\delta)F^{-\delta} M^{\delta}}=\dfrac{kI}{k}; \quad o \quad IM=\dfrac{\delta}{1-\delta}F$$
Sustituyendo este último en la ecuación presupuestaria se obtiene
$$Y=F+IM=F+\dfrac{\delta}{1-\delta}F;\quad F=(1-\delta)Y$$

\subsection{Gasto en la fabricación de variedades (b)}
Todavía tenemos que decidir cómo se distribuye este gasto entre las diferentes variedades de manufacturas, es decir, \textbf{tenemos que asignar de manera óptima el gasto entre el consumo de una cantidad de bienes que se pueden consumir}. Este problema sólo puede resolverse si especificamos cómo las preferencias por el consumo agregado de manufacturas $M$ dependen del consumo de variedades particulares de manufacturas. A este respecto, el modelo central de la economía geográfica aplica fructíferamente un modelo de competencia monopolística. Sea $c_i$ el nivel de consumo de una determinada variedad $i$ de manufacturas, y sea $N$ el número total de variedades disponibles. El enfoque de Dixit-Stiglitz utiliza una función de elasticidad de sustitución constante (CES) para construir el consumo agregado de manufacturas $M$ en función del consumo $c_i$ de las $N$ variedades:

$$M=\left(\sum_{i=1}^N c_i^{\rho} \right)^{1/\rho}\; ; \; 0<p<1$$\\

Tenga en cuenta que el consumo de todas las variedades entra en ésta ecuación simétricamente. El parámetro $\rho$, representa el efecto del gusto por la variedad de los consumidores. Si $\rho = 1$, la esta ecuación se simplifica a $M = \sum_i c_i$ y la variedad como tal no importa para la utilidad (tener 100 unidades de una variedad da la misma utilidad que una unidad de 100 variedades). Por lo tanto los productos son entonces sustitutos perfectos. Así necesitamos $\rho < 1$ para asegurar que las variedades de productos sean sustitutos imperfectos. Además, necesitamos $\rho > 0$ para asegurar que las variedades individuales sean sustitutos y no complementos entre sí, lo que permite un comportamiento de fijación de precios basado en el poder de monopolio.\\
Ahora suponga que todos los $c_i$ se consumen en cantidades iguales, es decir, $c_i = c$ para todos los $i$. Entonces podemos reescribir la ecuación dada como,
$$M=\left(\sum_{i=1}^N c^p\right)^{1/\rho}=(Nc^\rho)^{1/\rho}c=N^{(1/\rho) - 1}[Nc]$$
Como $0 < \rho < 1$, el término $(1/\rho) - 1$ es mayor que cero. Entonces queda inmediatamente claro que tener 100 unidades de una variedad $(Nc = C, N = 1)$ le da al consumidor menos utilidad que una unidad de 100 variedades $(Nc = C, N = 100)$. En muchos modelos, incluidos muchos nuevos modelos de crecimiento y modelos de economía geográfica, el término $Nc$  corresponde a un reclamo sobre recursos reales, porque $Nc$ tiene que ser producido en primer lugar, mientras que el número de variedades disponibles $N$ representa una externalidad o la extensión del mercado. El término $N^{(1/\rho)-1}$ representa una bonificación para grandes mercados, de ahí el término efecto de gusto por la variedad. En este sentido, \textbf{un aumento en la extensión del mercado, que aumenta el número de variedades $N$ entre las que el consumidor puede elegir, aumenta más que proporcionalmente la utilidad.}\\
Volvamos al tema central: ¿cómo distribuye el consumidor el gasto en manufacturas entre las diversas variedades?. Sea $p_i$ el precio de la variedad $i$ para $i = 1, \ldots, N$. Naturalmente, los fondos $p_ic_i$ gastados en la variedad $i$ no pueden gastarse simultáneamente en la variedad $j$, como se indica en la restricción presupuestaria para manufacturas en la siguiente ecuación:\\
$$\sum_{i=1}^N p_ic_i=\delta Y$$
Ahora debemos resolver un problema de optimización algo más complicado, a saber, la maximización de la utilidad derivada del consumo de manufacturas dada en la ecuación de $M$, sujeta a la restricción presupuestaria de la ecuación que acabamos de dar. La solución a este problema se da en las siguientes ecuaciones:
$$c_j=P_j^{-\epsilon}[I^{\epsilon-1}\delta Y],\quad \mbox{donde}\; I\equiv \left[\sum_{i=1}^N P_i^{1-\epsilon}\right]^{1/(1-\epsilon)}\quad \mbox{para}\; j=1,\ldots,N$$
$$M=\delta Y/I,\; \mbox{y}\; \epsilon \equiv \dfrac{1}{1-\rho}$$

En este punto, lo importante es enfatizar que la ecuación primera de estas ecuaciones que da la curva de demanda. Concluimos simplemente señalando que hemos derivado la demanda para cada variedad de manufacturas, lo que explica el inciso (b).\\\\

\subsection{Efectos de la demanda: renta, precio, elasticidad e (c) y el índice de precios I}
La parte 3.4.2 derivó la demanda de variedades de fabricación. La demanda de la variedad $1$, por ejemplo, está dada por $c_1 = p_1^{-\epsilon}[I^{\epsilon-1} \delta Y]$. Esta demanda parece estar influenciada por cuatro cosas:

\begin{enumerate}[\bfseries (i)]
    \item  El ingreso $\delta Y$ gastado en manufacturas en general,
    \item  el precio $p_1$ del bien $1$,
    \item  algún parámetro $\epsilon$, y 
    \item  el índice de precios $I$.
\end{enumerate}

\textbf{El punto (i) es sencillo. Cuanto más gasta el consumidor en manufacturas en general, más gasta en la variedad $1$.} En igualdad de condiciones, un aumento del $10$ por ciento en el gasto en manufacturas da como resultado un aumento del $10$ por ciento en la demanda para todas las variedades de manufacturas.\\
\textbf{El punto (ii) también es sencillo, pero muy importante. Es sencillo en el sentido de que obviamente esperamos que la demanda de la variedad $1$ sea una función del precio cobrado por la empresa que produce la variedad $1$}. Es muy importante en vista de cómo la demanda de la variedad $1$ depende del precio $p_1$. Donde depende del índice de precios de las manufacturas $I$ y del ingreso $\delta Y$ gastado por los consumidores en manufacturas en general. Ambas son entidades macroeconómicas que la empresa que produce la variedad $1$ tomará como dadas, es decir, asumirá que no tiene control sobre estas variables. En ese caso, podemos simplificar la demanda de la variedad $1$, definiendo $constante_1 = [I^{\epsilon-1} \delta Y]$ como $c_1 = constante_1 p_1^{-\epsilon}$. Esto, a su vez, implica que la elasticidad precio de la demanda de la variedad $1$ es constante e igual al parámetro $\epsilon > 1$, es decir, $-(\partial c_1/\partial p_1)(p_1/c_1) =\epsilon$;  Esta simple elasticidad precio de la demanda es la principal ventaja del enfoque de Dixit-Stiglitz ya que simplifica en gran medida el comportamiento de fijación de precios de las empresas monopolísticamente competitivas. Tenga en cuenta que la demanda de una variedad cae mucho más rápidamente como resultado de un pequeño aumento de precio, digamos de $1$ a $1.5$, si la elasticidad precio de la demanda es alta.\\
El punto (iii) queda claro después de la discusión en el punto (ii). Hemos definido el parámetro $\epsilon$ no solo para simplificar la notación de la ecuación, sino también porque es un parámetro económico importante, ya que mide la elasticidad precio de la demanda de una variedad de bienes manufacturados. Además, como se discutió en la parte anterior, este parámetro mide la elasticidad de sustitución entre dos variedades diferentes, que es decir, cuán difícil es sustituir una variedad de manufacturas por otra variedad de manufacturas. Evidentemente, la elasticidad precio de la demanda y la elasticidad de sustitución están relacionadas en el enfoque de Dixit-Stiglitz.\\
Finalmente, \textbf{el punto (iv) indica que la demanda de la variedad 1 depende del índice de precios I}. Si el índice de precios $I$ aumenta, lo que implica que en promedio los precios de las variedades de fabricación que compiten con la variedad $1$ están aumentando, entonces la demanda de la variedad $1$ es creciente. Las variedades son, por lo tanto, sustitutos económicos entre sí (si el precio de una variedad en particular aumenta, su propia demanda cae y la demanda de todas las demás variedades aumenta).\\


Tenga en cuenta que, aunque pueda parecer un poco engorroso a primera vista, el índice de precios $I$ en la ecuación principal se define de manera análoga a la función en la ecuación (3.4) que especifica la preferencia por las variedades, con 1 – e en (3.6) jugando el papel principal. papel de q en (3.4). De hecho, si usamos esta información para calcular la elasticidad de sustitución de los precios en la ecuación (3.6) obtenemos 1/[1 (1 e)] ¼ 1/e, la inversa de la elasticidad de sustitución de las variedades. Esto no es una coincidencia, ya que indica que si la elasticidad de sustitución de variedades es alta, un pequeño cambio de precio puede tener grandes efectos, y viceversa si es pequeño.9 Sin embargo, lo más importante es tener en cuenta que la definición del índice de precios I implica que M ¼ dY/I (ver ecuación (3.7)). El índice de precios I da así una representación exacta de la utilidad derivada del consumo de manufacturas; esta utilidad aumenta si, y solo si, el gasto en manufacturas aumenta más rápidamente que el índice de precios I. También tenga en cuenta que se requiere IM ¼ dY para justificar nuestras acciones en la subsección 3.4.1, donde usamos el índice de precios I para derivar la división de los ingresos sobre el consumo de alimentos y trabajo consumo. De lo contrario, nuestros cálculos allí no habrían sido consistentes.\\
Con frecuencia usamos el índice de precios I para derivar los salarios reales en el modelo. Por lo tanto, vale la pena echar un vistazo más de cerca a la definición de índices de precios basados en el consumo (ver también Obstfeld y Rogoff, 1996: 226). Puede hacerse la pregunta "¿Cuál es la cantidad mínima de gasto requerida para comprar una unidad de utilidad?" Sea I este gasto mínimo en manufacturas, tal que M ¼1. Entonces llamamos índice de precios exacto o basado en el consumo. De esta definición se sigue directamente de las ecuaciones (3.5) y (3.7) que I es de hecho tal índice.10 Es obvio que un aumento en el número de variedades disminuye I. Ya hemos explicado que un aumento en el número de variedades aumenta más que proporcionalmente la subutilidad de las manufacturas. Este efecto tiene una imagen especular en el índice de precios I; más variedades reducen I, porque se requiere menos gasto para M = 1. Además, el término “I” nos permitió escribir las ecuaciones de demanda de manera más eficiente como cj = pj e [Ie 1 dY].\\
Para concluir nuestra discusión sobre la estructura de demanda del modelo central, debemos hacer dos comentarios. El primero es relativamente corto. Podríamos usar el mismo procedimiento aplicado en la subsección 3.4.2 para derivar el índice de precios exacto para la asignación entre variedades para derivar dicho índice de precios para el problema de la subsección 3.4.1, asignación de ingresos entre alimentos y manufacturas. Como el lector querrá comprobar, el resultado sería 11d I d ¼ I d , en el que el “1” de la izquierda representa el precio de los alimentos, el cual se iguala a uno por ser el numerario . Por lo tanto, la utilidad del consumidor aumenta si, y solo si, Y/I d aumenta, es decir, si el nivel de ingresos aumenta más rápidamente que el índice de precios exacto I d . Por lo tanto, podemos definir el ingreso real y como una representación exacta de las preferencias de un consumidor (ver el recuadro 3.1 y la sección 3.8; ver la ecuación (3.8)). De manera similar, si la tasa salarial es W, podemos definir el salario real w también utilizando el índice de precios exacto (ver nuevamente la ecuación (3.8)). Además, si un consumidor individual solo tiene renta salarial, es decir, si Y ¼ W, entonces la renta real individual y es equivalente al salario real w\\
$$\mbox{Ingreso real:}\quad y=YI^{-\delta}; \quad \mbox{salario real:}\quad w=WI^{-\delta}$$
La segunda observación se refiere al punto (ii) anterior, en el que argumentamos que la (propia) elasticidad precio de la demanda del productor de la variedad 1 es igual a $\epsilon$. Recuerde la especificación de la función de demanda: $c_1 = p_1^{-\epsilon} [I^{\epsilon-1} \delta Y]$. Argumentamos que el término entre corchetes es tratado como una constante por el productor porque se trata de entidades macroeconómicas. Si bien esto es cierto, se pasa por alto un pequeño detalle: uno de los términos en la especificación del índice de precios de las manufacturas $I$ es el precio $p_1$. Por lo tanto, un productor verdaderamente racional también tendría en cuenta este minúsculo efecto sobre el índice de precios agregado.\\
Por esa razón, a menudo se supone que el número de variedades $N$ producidas es grande, es decir, si nuestro productor es una de las 80 000 empresas, podemos ignorar este efecto con seguridad, donde representamos la curva de demanda a la que se enfrenta el productor de una variedad si asume que no puede influir en el índice de precios de las manufacturas, y la demanda real teniendo en cuenta este efecto sobre el índice de precios. Claramente, la suposición es una mala aproximación si solo hay dos empresas , pero nadie sugiere que debas usar la competencia monopolística en un duopolio. Si hay veinte empresas la aproximación ya es mucho mejor, si hay 200 empresas la desviación es prácticamente indetectable, mientras que es inobservable si hay 2.000 empresas . Esto sugiere que podemos ignorar con seguridad este detalle para un número razonablemente grande de variedades.

\section{Oferta}
\subsection{Estructura de producción (d)}
Comenzamos el análisis del lado de la oferta del modelo central con una descripción de la estructura de producción de alimentos y manufacturas (ver también la figura 3.1). La producción de alimentos se caracteriza por rendimientos constantes a escala, y los alimentos se producen en condiciones de competencia perfecta. Se supone que los trabajadores de esta industria están inmóviles. Como se mencionó en la sección 3.3, el sector alimentario es, por lo tanto, el candidato natural para ser utilizado como numerario. Dada la fuerza laboral total L, se supone que una fracción (1c) trabaja en el sector alimentario. Por lo tanto, la fuerza de trabajo en la industria manufacturera es cL. La producción en el sector alimentario, F, es igual, por elección de unidades, al empleo alimentario:
$$F=(1-\lambda)L;\quad 0<\lambda<1$$
Dado que a los trabajadores agrícolas se les paga el valor del producto marginal, esta elección de unidades implica que el salario de los trabajadores agrícolas es uno, porque la comida es el numerario.
La producción en el sector manufacturero se caracteriza por economías de escala internas, lo que significa que existe una competencia imperfecta en este sector (ver recuadro 2.1). Las variedades en la industria manufacturera son simétricos y se producen con la misma tecnología. Tenga en cuenta que incluso en este punto ya hemos introducido un elemento de ubicación. Las economías de escala internas significan que cada variedad es producida por una sola empresa; la empresa con las mayores ventas siempre puede superar a un competidor potencial. Una vez que hayamos introducido más ubicaciones, cada empresa tiene que decidir dónde producir. Las economías de escala se modelizan de la forma más sencilla posible:
$$l_i=a+\beta x_i$$
donde li es la cantidad de trabajo necesaria para producir xi de la variedad i. Los coeficientes a y b describen los requisitos de insumos de mano de obra fijos y marginales, respectivamente. El insumo de trabajo fijo a en (3.10) asegura que a medida que la producción se expande, se necesita menos trabajo para producir una unidad de xi, lo que significa que hay economías de escala internas. Esto se ilustra en la figura 3.5, que muestra la mano de obra total requerida para producir una cierta cantidad de producto y la cantidad promedio de trabajo requerida para producir esa cantidad de producto. Esto explica la leyenda d en la figura 3.1.

\subsubsection{Fijación de precios y beneficios cero (e)}
Cada empresa manufacturera produce una variedad única bajo rendimientos internos a escala. Esto implica que la empresa tiene poder de monopolio, que utilizará para maximizar sus beneficios. Por lo tanto, tenemos que determinar la fijación de precios comportamiento de cada empresa. El modelo de competencia monopolística de Dixit-Stiglitz hace dos supuestos a este respecto. En primer lugar, se supone que cada empresa toma como dado el comportamiento de fijación de precios de otras empresas; es decir, si la empresa 1 cambia su precio, asumirá que los precios de las otras N–1 variedades seguirán siendo los mismos. En segundo lugar, se supone que la empresa ignora el efecto de cambiar su propio precio sobre el índice de precios I de las manufacturas. Ambos supuestos parecen razonables si el número de variedades N es grande, como también se analiza en la subsección 3.4.3. Para facilitar la notación, eliminamos el subíndice de la empresa en esta sección. Tenga en cuenta que una empresa que produce x unidades de producción usando la función de producción en la ecuación (3.10) obtendrá ganancias p dadas en la ecuación (3.11) si la tasa de salario que tiene que pagar es W.
$$\pi=px-W(a+\beta x)$$
Naturalmente, la empresa tendrá que vender las unidades de producción x que está produciendo, es decir, estas ventas deben ser consistentes con la demanda de una variedad de manufacturas derivadas en la sección 3.4. Aunque esta demanda se derivó para un consumidor arbitrario, la característica más importante de la demanda de una variedad, a saber, la elasticidad precio constante de la demanda e, también se cumple cuando combinamos la demanda de muchos consumidores con la misma estructura de preferencias (véase también el ejercicio 3.4). ). Si la demanda x de una variedad tiene una elasticidad precio constante de la demanda e, la maximización de los beneficios dada en la ecuación (3.11) conduce a una regla de fijación de precios óptima muy simple, conocida como precio de sobreprecio, como se indica en la ecuación (3.12) y se deriva en la nota técnica 3.3:
$$p(1-1/\epsilon)=\beta W \quad (o\; \beta W/ \rho$$
El término "precio de margen" es obvio. Los costos marginales de producir una unidad adicional de producto es igual a bW, mientras que el precio p que cobra la empresa es mayor que este costo marginal. Cuánto más alto depende crucialmente de la elasticidad precio de la demanda. Si la demanda es bastante inelástica, digamos e = 2, el margen de beneficio es alto (en este caso, 100 por ciento). Si la demanda es más bien elástica, digamos e = 5, el margen de beneficio es más bajo (en este caso, 20 por ciento). Tenga en cuenta que la empresa debe cobrar un precio más alto que el costo marginal para recuperar los costos fijos de la mano de obra aW. Debido a que la elasticidad precio de la demanda e es constante, el margen del precio sobre el costo marginal también es constante y, por lo tanto, invariante a la escala de producción. Nótese que el precio es fijo si el salario es fijo, como en Krugman (1980).\\
Ahora que hemos determinado el precio óptimo que cobrará una empresa para maximizar las ganancias, podemos calcular esas ganancias (si conocemos la constante en la nota técnica 3.3). Aquí es donde entra en juego otra característica importante de la competencia monopolística. Si las ganancias son positivas (a veces denominadas ganancias en exceso), aparentemente es muy atractivo instalarse en el sector manufacturero. Entonces, se esperaría que nuevas empresas ingresaran al mercado y comenzaran a producir diferentes variedades. Esto implica, por supuesto, que el consumidor destinará su gasto a más variedades de manufacturas. Dado que todas las variedades son sustitutos entre sí, la entrada de nuevas empresas en el sector manufacturero implica que las ganancias de las empresas existentes caerán. Este proceso de entrada de nuevas empresas continuará hasta que las ganancias en el sector manufacturero se reduzcan a cero. Un proceso inverso, con empresas que abandonan el sector manufacturero, operaría si las ganancias fueran negativas. Por lo tanto, la competencia monopolística en el sector manufacturero impone como condición de equilibrio que las ganancias sean cero. Si hacemos eso en la ecuación (3.11), podemos calcular la escala a la que operará una empresa que produce una variedad en el sector manufacturero, ecuación (3.13), cuánto trabajo se necesita para producir esta cantidad de producción, ecuación (3.14), y cuántas variedades N se producen en la economía en función de la mano de obra disponible en el sector manufacturero, ecuación (3.15); ver nota técnica 3.4.
$$x=\dfrac{a(\epsilon-1)}{\beta}$$
$$l_i=a\epsilon$$
$$N=\lambda L/l_i = \lambda L/a\epsilon$$

La ecuación (3.13), que da la escala de producción de una empresa individual, puede parecer extraña a primera vista. Pase lo que pase, la producción por empresa se mantiene fija en el equilibrio. La elasticidad precio constante de la demanda en conjunto con la función de producción es responsable de este resultado. Implica que el sector manufacturero en su conjunto se expande y contrae solo al producir más o menos variedades, ya que el nivel de producción por variedad no cambia. De (3.15) vemos que un mercado más grande causado, por ejemplo, por la apertura de las fronteras o el aumento del comercio internacional afecta solo al número de variedades. Como resultado de las economías de escala, no es rentable tener la misma variedad producida por más de una empresa; cada empresa producirá sólo una variedad.\\
Puede surgir la pregunta de dónde están las economías de escala; ya no importan? Hay otra forma de ver el parámetro e. En equilibrio, también se utiliza como medida de economías de escala. Las economías de escala se pueden medir de varias maneras, pero una medida específica de las economías de escala es la siguiente: costos promedio divididos por costos marginales; si los costes marginales son inferiores a los costes medios, un aumento de la producción reducirá los costes medios. Para el modelo central podemos calcular esta medida para el nivel de producción de equilibrio. El requerimiento de mano de obra es ae (ver ecuación (3.14)), el nivel de producción es a(e 1)/b por lo que los costes laborales medios son ae /[a(e 1)/b] ¼ be/(e 1). Los costes laborales marginales son simplemente b, por lo que esta medida de economías de escala se reduce a costes medios / costes marginales ¼ e/(e1), que en equilibrio depende únicamente del parámetro de elasticidad de sustitución e (nótese en particular que los parámetros a y b de la función de producción no entran). Para un valor bajo de e esta medida de economías de escala es alta, mientras que para valores altos esta medida es baja. Esto último significa que las variedades se están convirtiendo en sustitutos cada vez más perfectos. En el límite, solo sobrevive una sola variedad. La producción de esta única variedad se lleva a cabo a la mayor escala posible (toda la mano de obra de fabricación se emplea en la producción de la única variedad), lo que deja menos espacio para las economías de escala que si e es bajo, y muchas empresas diferentes producen muchas variedades diferentes (ver Hanoch, 1975). Recuérdese, finalmente, que esta medida indica únicamente el nivel de economías de escala en equilibrio. Las economías de escala internas no están ausentes, por lo tanto, sino que aparecen de una manera bastante especial en el modelo Dixit-Stiglitz de competencia monopolística.

\subsection{Costos de transporte: icebergs en geografía}
l objetivo del modelo central de la economía geográfica es introducir la geografía de una manera no trivial. Es decir, el modelo debe mostrar cómo la geografía afecta las decisiones de los consumidores y productores individuales y cómo estas decisiones, a su vez, dan forma a la distribución espacial de la actividad económica. Para poder hacerlo, se deben introducir los costos de transporte. Solo si es costoso mover productos y personas por el espacio, la geografía tiene sentido en el modelo central.\\
Los gastos de transporte que presentamos son especiales. En principio, se podría modelar un sector de transporte y agregarlo al modelo, pero esto sería muy engorroso. Cada costo es también una ganancia para otra persona, y los costos de transporte son ingresos para el sector del transporte, por lo que uno debe lidiar con el gasto de este sector. Además, la decisión de ubicación del sector transporte puede ser diferente de las decisiones de ubicación de los otros sectores. Es por estas razones que Samuelson (1952) ha introducido el concepto de costos de transporte del iceberg (llamada f). En el contexto del modelo central de economía geográfica, los costos de transporte iceberg implican que una fracción de los bienes manufacturados no llega a su destino cuando los bienes se envían entre regiones. La fracción que no llega representa el costo de transporte. El modelo central usa T como parámetro para representar estos costos, donde T se define como la cantidad de bienes que deben enviarse para garantizar que llegue una unidad por unidad de distancia. Supongamos, por ejemplo, que la unidad de distancia es igual a la distancia de Naaldwijk, en el centro de la aglomeración hortícola holandesa, a París, y que se envían 107 flores desde Holanda a Francia, mientras que sólo 100 llegan ilesas a París y pueden ser vendido. Entonces T = 1,07. Es como si algunos bienes se hubieran derretido en tránsito, de ahí el nombre de costos de "iceberg". Esta forma de modelar los costos de transporte sin introducir un sector de transporte es muy atractiva, porque no tenemos que modelar un sector de transporte separado. Esto implica que no tenemos que lidiar con preguntas como “¿Quién trabaja en este sector?” y “¿Dónde viven y gastan sus ingresos los trabajadores del sector del transporte?” Esto explica la llamada f en la figura 3.1. El Recuadro 3.3 analiza la relevancia de los costos de transporte. \\
 lo largo del resto del libro, el parámetro T denota la cantidad de bienes que deben enviarse para garantizar que llegue una unidad de una variedad de manufacturas por unidad de distancia, mientras que Trs se define como la cantidad de bienes que deben enviarse desde región r para asegurar que una unidad llegue a la región s. Suponemos que esto es proporcional a la distancia entre las regiones r y s. Si Drs denota la distancia entre la región r y la región s (que es cero si r = s), tenemos
 $$T_{rs} = T^{D_n}, \; mbox{para}\; r,s=1,2;\quad \mbox{nota}: T_{rs}=T_{rs}, \; \mbox{y} \; T_{rr}=T^0=1$$
 Estas definiciones facilitan la notación en las ecuaciones a continuación y nos permiten distinguir entre cambios en el parámetro T (es decir, un cambio general en la tecnología [de transporte] que se aplica a todas las regiones) y cambios en la "distancia" Drs entre regiones, que pueden resultado de un cambio de política, como cambios de tarifas, un tratado cultural, nueva infraestructura, para el  modelo central de dos regiones discutido aquí, siempre asumimos que la distancia entre las dos regiones es uno. La ecuación (3.16) y las siguientes ecuaciones aún no utilizan este hecho para desarrollar un modelo general de múltiples regiones al mismo tiempo. 

 \section{Equilibro}
 Dados los detalles del modelo central de la economía geográfica tal como se expuso en las secciones anteriores, ya hemos explicado una parte significativa de la estructura de este modelo, como se muestra en la figura 3.1 en la sección 3.3. Lo que hay que hacer ahora es establecer las relaciones de equilibrio, que efectivamente atarán todos los cabos sueltos. En particular, tenemos que determinar la forma en que las relaciones de equilibrio junto con los cuadros sombreados en la figura 3.1 determinan en última instancia la distribución espacial de la actividad económica. Estos recuadros sombreados (llamado g en la figura 3.1) se refieren a la movilidad de los trabajadores y las empresas manufactureras entre las dos regiones. Como ya se ha explicado detalladamente en el capítulo 2, esta movilidad diferencia realmente a la economía geográfica de la nueva teoría del comercio, en la que se basa gran parte del modelo central.\\
 Para comprender la determinación del equilibrio y el papel de la movilidad de los factores y las empresas en el mismo, procedemos en tres pasos. En primer lugar, esta sección se concentra en las relaciones de equilibrio a corto plazo, es decir, proporciona el análisis de equilibrio para una distribución dada exógenamente de la fuerza laboral manufacturera. Por lo tanto, se supone que la mano de obra manufacturera no es móvil entre regiones a corto plazo. La distribución espacial de los trabajadores y las empresas manufactureras aún no está determinada por el modelo en sí, sino que simplemente se impone sobre el modelo. En segundo lugar, la siguiente sección aborda brevemente el tema de la dinámica, es decir, cómo nos movemos a través de una secuencia de equilibrios a corto plazo (sin movilidad de factores) a lo largo del tiempo hasta un equilibrio a largo plazo (con movilidad de factores). Esto es crucial para el enfoque de la economía geográfica. En tercer lugar, el análisis de los equilibrios tanto a corto como a largo plazo resulta ser tan complicado que lo trataremos por separado en el próximo capítulo.

 \subsection{Equilibrio a corto plazo}
 Esta subsección resume las relaciones de equilibrio a corto plazo de toda la economía, es decir, las ecuaciones para ambas regiones en un entorno de dos regiones, y reúne y analiza brevemente las tres ecuaciones de equilibrio a corto plazo para la región 1. ¿Cuáles son las ecuaciones a corto plazo? ejecutar relaciones de equilibrio? Da la casualidad de que, de hecho, ya hemos usado algunos de estos sin decirlo explícitamente. Por ejemplo, ya hemos asumido que los mercados laborales se equilibran, es decir, (i) todos los trabajadores agrícolas tienen un trabajo y (ii) todos los trabajadores de manufactura tienen un trabajo. El punto (i) ha determinado el nivel de producción de alimentos en cada región, en conjunto con la función de producción de alimentos y la competencia perfecta en el sector alimentario. El punto (ii) ha determinado el número de variedades manufactureras producidas en cada región, junto con la función de producción de manufacturas, el comportamiento de fijación de precios de las empresas y la entrada o salida de empresas en el sector manufacturero hasta que las ganancias sean cero. Evidentemente, no hay beneficios para las empresas del sector manufacturero (debido a la entrada y salida), ni para los agricultores (debido a los rendimientos constantes a escala y la competencia perfecta). Esto implica que todos los ingresos obtenidos en la economía para que los consumidores los gasten se derivan de los salarios que ganan en sus respectivos sectores.\\
 Esto nos lleva a la siguiente relación de equilibrio: cómo determinar el ingreso en cada región. En vista de lo anterior, esto es simple. Hay f1(1c)L trabajadores agrícolas en la región 1, cada uno de los cuales gana un salario agrícola de uno (alimentos es el numerario), y hay k1 c L trabajadores industriales en la región 1, cada uno de los cuales gana un salario W1. Como no hay ganancias u otros factores de producción, este es el único ingreso generado en la región 1. Si dejamos que Yi denote el ingreso generado en la región i, esto implica
 $$T_1=\lambda_1 W_1 \lambda L + \phi(1-\lambda)L$$
donde el primer término del lado derecho representa el ingreso de los trabajadores manufactureros y el segundo término refleja el ingreso de los trabajadores agrícolas. Como se discutió en las secciones 3.6 y 3.7, la cantidad real de costos de transporte entre regiones para el sector manufacturero está dada por T 1. Dado que todas las empresas en una región enfrentan costos marginales de producción idénticos y la misma elasticidad constante de la demanda (ver también a continuación), todos cobran el mismo precio a los productores locales, digamos p1 para los productores de la región 1 y p2 para los productores de la región 2 (ver también el ejercicio 3.4). Este precio de fábrica, o f.o.b. El precio de una variedad producida en la región 1 cobrado a los consumidores de la región 1 está relacionado con los costos marginales de producción en la región 1 a través de la condición de precio óptimo (3.12): p1¼ bW1/q. Esto indica que el valor f.o.b. el precio es directamente proporcional al salario. El precio de una variedad producida en la región 1 después de ser entregada en la región 2 es Tp1, que es el valor c.i.f. precio de esta variedad (véase también el recuadro 3.3 sobre precios f.o.b. y c.i.f.). Además, recuerde de la sección 3.7 que, dado que hay k1cL trabajadores manufactureros en la región 1, de la ecuación (3.15) se deduce que el número de empresas N1 ubicadas en la región 1 es igual a N1¼ k1cL/ae. Es decir, la cantidad de empresas ubicadas en la región 1 es directamente proporcional a k1, la cantidad de trabajadores manufactureros ubicados en la región 1. Los tres aspectos discutidos anteriormente, a saber (i) los precios de los bienes producidos localmente son directamente proporcionales al salario local tasa, (ii) los precios cobrados en la otra región son más altos por los costos de transporte entre regiones, y (iii) la cantidad de variedades producidas en una región es directamente proporcional a la cantidad de trabajadores manufactureros en una región: son importantes para comprender por eso el índice de precios puedo tener un valor diferente en ambas regiones. Para la región 1 el índice de precios I1 es (ver nota técnica 3.5 para más detalles)


