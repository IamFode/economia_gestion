\chapter{El modelo central de la economía geográfica}

\section{Introducción}
Hace mucho tiempo observó que los campos de la teoría del comercio por un lado y la economía regional y urbana por el otro tenían, en principio, los mismos objetivos de investigación. Ambas áreas de investigación quieren responder a las preguntas: "¿Quién produce qué, dónde y por qué?" A pesar de la observación de Ohlin, cada campo ha seguido su propio camino desde el siglo XIX. De todas formas La teoría del comercio y la economía regional y urbana se combinan productivamente en la economía geográfica.\\
Se analizará un pequeño modelo de equilibrio general desarrollado por Krugman (1991a, 1991b). Como veremos, las ecuaciones de equilibrio de este modelo no son lineales. Esto significa que los pequeños cambios en los parámetros no siempre producir los mismos efectos; a veces los efectos son pequeños, a veces son grandes. 

\section{Un ejemplo de economía geográfica}
Es posible construir un ejemplo simple para ilustrar algunos de los principales hallazgos del enfoque de la economía geográfica. Supongamos que hay dos regiones (o países), Norte y Sur, y dos sectores de producción, manufactura y agricultura. La industria manufacturera produce variedades, es decir, productos diferenciados, bajo economías internas de escala. Por lo tanto, el costo por unidad de producción cae a medida que una empresa expande su nivel de producción. Como resultado, cada empresa produce solo una variedad. Una empresa puede residir en el Norte o en el Sur, es decir, una empresa tiene que decidir dónde producir. Esta decisión de ubicación diferencia esencialmente el ejemplo de la nueva teoría comercial.\\
La demanda total de cada variedad de manufacturas en este ejemplo es exógena. Suponemos que cada empresa vende cuatro unidades a los trabajadores de la industria manufacturera y seis unidades a los agricultores. Por lo tanto, la demanda total de cada variedad es diez. La producción de la agricultura, y por lo tanto la demanda que genera, es específica de la ubicación. Su distribución espacial está dada exógenamente; suponemos que se venden cuatro unidades en el Norte y dos unidades en el Sur. La ubicación de los trabajadores en el sector manufacturero y, por tanto, las cuatro unidades que demandan en esa ubicación, no es exógena. El papel de los trabajadores inmóviles es importante, ya que aseguran que siempre haya una demanda positiva en ambas regiones. Finalmente, los costos de transporte entre el Norte y el Sur son $1$ dolar por unidad. Las empresas deciden su ubicación para minimizar los costos de transporte.\\
Ahora podemos determinar la decisión de ubicación de cada empresa. Primero, podemos calcular las ventas regionales de cada empresa, dada la ubicación de las otras empresas. Se dan tres posibilidades (no exhaustivas): todas las empresas en el Norte, todas las empresas en el Sur y el 25 por ciento de todas las empresas en el Norte y el 75 por ciento de todas las empresas en el Sur. Las ventas en cada región son iguales a las ventas a los trabajadores en la manufactura más las ventas a los agricultores. La empresa vende cinco unidades en North, a saber, cuatro a los agricultores ubicados en North más  unidad a los trabajadores de manufactura ubicados en North. De manera similar, la empresa vende cinco unidades en el Sur, es decir, dos unidades a los agricultores ubicados en el Sur más tres unidades a los trabajadores de manufactura ubicados en el Sur.\\
En segundo lugar, podemos construir una tabla de decisión, calculando los costos de transporte en función de la decisión de ubicación de la empresa, dada la ubicación de las otras empresas. Suponga, por ejemplo, que todas las empresas están ubicadas en el norte. Entonces los costos de transporte para una empresa ubicada en el Sur serán entonces ocho dolares, es decir, 4 dolares para las ventas a los agricultores del Norte y 4 dolares para las ventas a todos los trabajadores de la industria manufacturera ubicada en el Norte. De manera similar, si la empresa se ubica en el norte, los costos de transporte serían solo dos dolares para las ventas a los agricultores del sur. Dado que los costos de transporte se minimizan al ubicarse en el Norte si todas las demás empresas están ubicadas en el Norte, la empresa decide ubicar también la producción en el Norte. Una empresa se ubicará en el Sur si todas las demás empresas también se ubican allí, mientras que a la empresa le es indiferente ubicarse en el Norte o en el Sur (ya que los costos de transporte son los mismos si la empresa se ubica en el Sur). En cualquier región si el 25 por ciento de las empresas están ubicadas en el Norte y el 75 por ciento en el Sur.\\
Sobre la base de este ejemplo, podemos ilustrar algunas características distintivas del enfoque de la economía geográfica.\\
En primer lugar, el concepto de causalidad acumulativa. \textbf{Si, por alguna razón, una ubicación ha atraído a más empresas que la otra ubicación, una nueva empresa tiene un incentivo para ubicarse donde están las otras empresas.} Si todas las empresas existentes están ubicadas en el norte, la nueva empresa también debería ubicarse allí si desea minimizar sus costos de transporte.\\
En segundo lugar. La aglomeración de todas las empresas en el Norte o en el Sur es un equilibrio. Sin embargo, no podemos determinar de antemano dónde ocurrirá la aglomeración. Esto depende críticamente de las condiciones iniciales, es decir, las decisiones previas de ubicación de otras empresas.\\
Tercero, un equilibrio puede ser estable o inestable. Si una sola empresa decide mudarse, esto decisión no influiría en las decisiones de ubicación de las otras empresas. Si una sola empresa decide trasladarse, la nueva ubicación se volverá inmediatamente más atractiva para todas las demás empresas. Esto desencadenará un efecto bola de nieve: todas las empresas seguirán al pionero. En este ejemplo, solo la aglomeración es un equilibrio estable.\\
En cuarto lugar, observamos que un equilibrio estable puede no ser óptimo. Si todas las empresas están ubicadas en el norte, los costos de transporte son solo e2. Si todas las empresas están ubicadas en el sur, los costos de transporte son e4 (ver las entradas en negrita en la tabla 3.2). Por lo tanto, los costos de transporte para la economía en su conjunto se minimizan si todas las empresas se aglomeran en el Norte, mientras que la aglomeración en el Sur sigue siendo un equilibrio estable.\\
En quinto lugar, el ejemplo ilustra la interacción de la aglomeración y los flujos comerciales. Con la aglomeración completa, es decir, todas las manufacturas se producen en una sola región, el comercio entre regiones será de tipo interindustrial (alimentos para manufacturas). De hecho, este equilibrio también refleja el llamado efecto del mercado interior; la combinación de economías de escala y costos de transporte es responsable del agrupamiento de toda la actividad libre en un solo lugar. Debido a esta combinación, los costos de transporte pueden minimizarse. La gran región termina con un gran mercado para la fabricación de bienes, que pueden venderse sin incurrir en costos de transporte. La consecuencia es que esta región se convierte en exportadora de productos manufacturados; \textbf{Las grandes regiones tienden a convertirse en exportadoras de aquellos bienes para los que tienen un gran mercado local, de ahí el término efecto del mercado interno.} Si la industria manufacturera está ubicada en ambas regiones, el comercio también será del tipo intraindustrial. Además de intercambiar bienes manufacturados por productos agrícolas, se intercambiarán diferentes variedades de productos manufacturados diferenciados entre ambas regiones.\\
El ejemplo es útil, ya que ilustra algunos aspectos importantes de la economía geográfica. Sin embargo, un ejemplo es solo un ejemplo y no es un sustituto de un modelo bien especificado. ¿Qué le falta?
\begin{itemize}
    \item En primer lugar, falta la interacción entre los costos de transporte, el comportamiento de fijación de precios y la elección de la ubicación. 
    \item Además, es un modelo de equilibrio parcial, en el sentido de que las empresas no se preocupan por la mano de obra necesaria; donde sea que decidan ubicarse, la disponibilidad de mano de obra no es el problema. 
\end{itemize}

\section{La estructura del modelo}
El modelo central identifica dos regiones, etiquetadas como 1 y 2. Hay dos sectores en la economía, el sector manufacturero y el sector alimentario. Los consumidores en ambas regiones consisten en trabajadores agrícolas y trabajadores de manufactura. Los trabajadores agrícolas obtienen sus ingresos trabajando en las granjas de su región. El flujo de ingresos de los trabajadores agrícolas es parte de una transferencia bilateral: obtienen un ingreso a cambio de su oferta de mano de obra.\\
El sector manufacturero consta de empresas $N_1$ en la región 1 y empresas $N_2$ en la región 2. Cada empresa manufacturera produce un producto diferenciado, es decir, produce una variedad única de manufacturas. Utiliza solo mano de obra en el proceso de producción, que se caracteriza por economías internas de escala. Esto implica que las empresas tienen poder de monopolio, que utilizan para determinar el precio de su producto. Además, los costos de transporte están involucrados en la venta de un bien manufacturado en otra región. Estos costos no surgen si el bien manufacturado se vende en la región en la que se produce. Como resultado de los costos de transporte, las empresas exportadoras cobrarán un precio más alto en la región extranjera que en la región de origen. Los trabajadores manufactureros obtienen sus ingresos (la tasa salarial manufacturera) al suministrar mano de obra a las empresas del sector manufacturero ubicadas en la región de origen.\\
Los consumidores gastan sus ingresos tanto en alimentos como en manufacturas. Dado que los alimentos son un bien homogéneo, no les importa si se producen en la región 1 o en la región 2. Como los alimentos no tienen costos de transporte, tienen el mismo precio en ambas regiones (lo que implica que los agricultores ganan el mismo salario en ambas regiones). El gasto de los consumidores en manufacturas debe distribuirse entre las muchas variedades producidas en las regiones 1 y 2. En igualdad de condiciones, consumir variedades importadas es más costoso que consumir variedades nacionales, como resultado de los costos de transporte de los productos manufacturados. Sin embargo, dado que las variedades son productos diferenciados y los consumidores tienen gusto por la variedad, siempre consumirán algunas unidades de todas las variedades producidas, ya sea en el país o en el extranjero.\\

\section{Demanda}
\subsection{Gasto en alimentos y manufacturas (a)}
La economía tiene dos sectores de bienes, manufacturas, $M$, y alimentos, $F$. Aunque las manufacturas consisten en muchas variedades diferentes, podemos definir un índice de precios exacto para representarlas como un grupo. A este índice de precios de manufacturas lo llamamos $I$. Si un consumidor obtiene un ingreso $Y$ (por trabajar en el sector alimentario o en el sector manufacturero), tiene que decidir cuánto de este ingreso se gasta en alimentos y cuánto en manufacturas. La solución a este problema depende de las preferencias del consumidor, que se suponen de la especificación Cobb-Douglas para todos los consumidores, en la que F representa el consumo de alimentos y M representa el consumo de manufacturas:
$$U=F^{1-\delta}M^\delta\quad 0<\delta<1$$
de donde el consumidor debe satisfacer la restricción presupuestaria en la ecuación.
$$F+I\cdot M = Y$$
Nótese la ausencia del precio de los alimentos en esta ecuación. Esto es el resultado de elegir los alimentos como numerario, lo que implica que el ingreso $Y$ se mide en términos de alimentos. Por lo tanto, solo el índice de precios de las manufacturas $I$ aparece en la ecuación. Para decidir sobre la asignación óptima de ingresos sobre la compra de alimentos y manufacturas, el consumidor ahora tiene que resolver un problema de optimización simple, a saber, maximizar la utilidad como se indica en la ecuación $U=F^{1-\delta}M^\delta\quad 0<\delta<1$, sujeto a la restricción presupuestaria de la ecuación $F+I\cdot M = Y$. La solución a este problema se da en la siguiente ecuación y se derivada como:
$$F=(1-\delta)Y;\quad IM=\delta Y$$
De ahora en adelante nos referiremos al parámetro $\delta$ dado en la ecuación $U=F^{1-\delta}M^\delta\quad 0<\delta<1$ como la fracción del ingreso gastado en manufacturas.\\\\

Para maximizar la ecuación $U=F^{1-\delta}M^\delta\quad 0<\delta<1$ sujeta a la restricción presupuestaria 
$F+I\cdot M = Y$, definimos el Lagrangiano $\Gamma$, usando el multiplicador $k$:
$$\Gamma = F_{1-\delta M^{\delta}}+k[Y-IM]$$
La diferenciación de $\Gamma$ con respecto a $F$ y $M$ da las condiciones de primer orden
$$(1-\delta)F^{-\delta}M^{\delta};\quad \delta F^{1-\delta}M^{\delta-1} = kI$$
Tomando la relación de las condiciones de primer orden se obtiene
$$\dfrac{\delta F^{1-\delta} M^{\delta-1}}{(1-\delta)F^{-\delta} M^{\delta}}; \quad o \quad IM\dfrac{\delta}{1-\delta}F$$
Sustituyendo este último en la ecuación presupuestaria se obtiene
$$Y=F+IM=F+\dfrac{\delta}{1-\delta}F;\quad F=(1-\delta)Y$$

\subsection{Gasto en la fabricación de variedades (b)}
Todavía tenemos que decidir cómo se distribuye este gasto entre las diferentes variedades de manufacturas, es decir, \textbf{tenemos que asignar de manera óptima el gasto entre el consumo de una cantidad de bienes que se pueden consumir}. Este problema sólo puede resolverse si especificamos cómo las preferencias por el consumo agregado de manufacturas $M$ dependen del consumo de variedades particulares de manufacturas. A este respecto, el modelo central de la economía geográfica aplica fructíferamente un modelo de competencia monopolística. Sea $c_i$ el nivel de consumo de una determinada variedad $i$ de manufacturas, y sea $N$ el número total de variedades disponibles. El enfoque de Dixit-Stiglitz utiliza una función de elasticidad de sustitución constante (CES) para construir el consumo agregado de manufacturas $M$ en función del consumo $c_i$ de las $N$ variedades:

$$M=\left(\sum_{i=1}^N c_i^p\right)^{1/p}\; ; \; 0<p<1$$

Tenga en cuenta que el consumo de todas las variedades entra en la ecuación (3.4) simétricamente. Esto simplifica enormemente el análisis en el resto del capítulo. El parámetro q, discutido más adelante, representa el efecto de amor por la variedad de los consumidores. Si q = 1, la ecuación (3.4) se simplifica a M = Ri ci y la variedad como tal no importa para la utilidad (tener 100 unidades de una variedad da la misma utilidad que una unidad de 100 variedades). Los productos son entonces sustitutos perfectos (una unidad menos de una variedad puede compensarse exactamente con una unidad más de otra variedad). Por lo tanto, necesitamos q < 1 para asegurar que las variedades de productos sean sustitutos imperfectos. Además, necesitamos q > 0 para asegurar que las variedades individuales sean sustitutos (y no complementos) entre sí, lo que permite un comportamiento de fijación de precios basado en el poder de monopolio.\\
ale la pena detenerse un poco más en la especificación de (3.4). Suponga que todos los ci se consumen en cantidades iguales, es decir, ci¼ c para todos los i. Entonces podemos reescribir la ecuación (3.4) como
$$M=\left(\sum_{i=1}^N c^p\right)^{1/p}=(Nc^p)^{1/p}$$
Como 0 < q < 1, el término (1/q) 1 es mayor que cero. Entonces, a partir de (3.40) queda inmediatamente claro que tener 100 unidades de una variedad (Nc ¼ C, N ¼ 1) le da al consumidor menos utilidad que una unidad de 100 variedades (Nc ¼ C, N ¼ 100). En muchos modelos, incluidos muchos nuevos modelos de crecimiento y modelos de economía geográfica, el término Nc en la ecuación (3.40) corresponde a un reclamo sobre recursos reales, porque Nc tiene que ser producido en primer lugar, mientras que el número de variedades disponibles N representa una externalidad o la extensión del mercado. El término N(1/q)1 representa una bonificación para grandes mercados, de ahí el término "efecto de amor por la variedad". En este sentido, un aumento en la extensión del mercado, que aumenta el número de variedades N entre las que el consumidor puede elegir, aumenta más que proporcionalmente la utilidad.\\
Después de nuestra breve digresión sobre el efecto del amor por la variedad, es hora de volver al problema que nos ocupa: ¿cómo distribuye el consumidor el gasto en manufacturas entre las diversas variedades? Sea pi el precio de la variedad i para i ¼ 1, . . . , N. Naturalmente, los fondos pici gastados en la variedad i no pueden gastarse simultáneamente en la variedad j, como se indica en la restricción presupuestaria para manufacturas en la ecuación (3.5):\\
$$\sum_{i=1}^N p_ic_i=\delta Y$$
Para derivar la demanda de un consumidor, ahora debemos resolver un problema de optimización algo más complicado, a saber, la maximización de la utilidad derivada del consumo de manufacturas dada en la ecuación (3.4), sujeta a la restricción presupuestaria de la ecuación (3.5). La solución a este problema se da en las ecuaciones (3.6) y (3.7), y se deriva de la nota técnica 3.2.
$$c_j=P_j^{-\epsilon}[I^{\epsilon-1}\delta Y],\quad \mbox{donde}\; I\equiv \left[\sum_{i=1}^N P_i^{1-\epsilon}\right]^{1/(1-\epsilon)}\quad \mbox{para}\; j=1,\ldots,N$$
$$M=\delta Y/I,\; \mbox{y}\; \epsilon \equiv \dfrac{1}{1-\rho}$$

La discusión y explicación del significado de las ecuaciones (3.6) y (3.7) ciertamente están justificadas; hacemos esto en la siguiente subsección. En este punto, lo importante es enfatizar que la ecuación (3.6) da la curva de demanda. Concluimos esta subsección simplemente señalando que hemos derivado la demanda para cada variedad de manufacturas, lo que explica la llamada b en la figura 3.1.\\\\

A menudo se ha dicho que solo hay una forma de que la competencia sea perfecta, pero muchas formas de ser imperfecta. En consecuencia, existen muchos modelos en competencia para describir la competencia imperfecta, investigando muchos casos y supuestos diferentes con respecto al comportamiento del mercado, el tipo de bien, la interacción estratégica entre empresas, las preferencias de los consumidores, etc. Ese también fue el caso de la competencia monopolística (ver , por ejemplo, Tirole, 1988), hasta que Dixit y Stiglitz publicaron en 1977 un artículo titulado “Competencia monopolística y la diversidad óptima de productos”, en The American Economic Review, que revolucionaría la construcción de modelos en al menos cuatro campos de la economía: el comercio teoría, organización industrial, teoría del crecimiento y economía geográfica.6 El gran paso adelante fue hacer algunos supuestos heroicos sobre la simetría de las nuevas variedades y la forma estructural, lo que permitió una manera elegante y consistente de modelar la producción a nivel de empresa, beneficiando de economías internas de escala en conjunción con una estructura de mercado de competencia monopolística, sin empantanarse resumido en una taxonomía de modelos de oligopolio. Estos factores son los responsables de la actual popularidad del modelo Dixit-Stiglitz. En todos los campos que ahora usan intensamente la formulación de Dixit-Stiglitz, los investigadores eran conscientes de que la competencia imperfecta era relevante como una característica esencial de muchos fenómenos observados empíricamente. Esto significó que el modelo fue inmediatamente aceptado como el nuevo estándar para modelar la competencia monopolística; su desarrollo fue ciertamente muy oportuno. En la teoría del comercio internacional, la introducción del modelo de competencia monopolística permitió a los economistas internacionales explicar y comprender el comercio intraindustrial, que hasta entonces había sido observado empíricamente pero nunca explicado satisfactoriamente (Krugman, 1979, 1980). En la organización industrial, ayudó a deshacerse de muchos supuestos ad hoc, que dificultaron el desarrollo de muchos modelos de organización industrial (Tirole, 1988). El modelo Dixit-Stiglitz también se utilizó para explorar el papel de los bienes intermedios diferenciados en los modelos de comercio internacional. Esta reformulación del modelo estándar de Dixit-Stiglitz juega un papel importante en el vínculo entre el comercio internacional y el crecimiento económico (ver, por ejemplo, Grossman y Helpman, 1991). El modelo también resulta muy útil para explicar la decisión de exportar IED si las empresas son heterogéneas (véanse Melitz, 2003 y Helpman, Melitz y Yeaple, 2004). Finalmente, el modelo se usa intensamente en economía geográfica, el tema de este libro.

