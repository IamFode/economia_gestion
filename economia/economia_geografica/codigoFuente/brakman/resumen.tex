\section{Geografía y teoría económica}
Se puede argumentar que en la economía regional y urbana hay un amplio espacio para la geografía o el espacio en el análisis, pero que estos enfoques a veces carecen de la base microeconómica del comportamiento individual y de la estructura de equilibrio general que constituye la columna vertebral de la corriente principal. Teoría económica en la actualidad. Por el contrario, tanto para la antigua como para la nueva teoría del comercio y el crecimiento, existe tal base microeconómica y una estructura de equilibrio general, pero el problema es que la geografía suele ser casi irrelevante o, cuando importa, su función no lo es. Vinculado al comportamiento subyacente de los agentes económicos. Desde nuestro punto de vista, la economía geográfica puede verse como un intento de romper aún más la valla entre la geografía y la economía. Al hacerlo, tiene sus raíces firmemente en la teoría económica dominante, por lo que es en particular un intento de traer más geografía a la economía. Por esa razón, preferimos el término economía geográfica a nueva geografía económica.

\section{Modelo principal}
Comenzamos nuestra discusión sobre el modelo central de la economía geográfica con un ejemplo en la sección  y concluimos con una advertencia: un ejemplo es solo un ejemplo. Luego proporcionamos un modelo completo que llenó los espacios en blanco del ejemplo. Ahora podemos comprobar si las cinco características importantes del ejemplo están realmente presentes en el modelo más sofisticado.
\begin{enumerate}
    \item Causalidad acumulativa.  si, por alguna razón, una región ha atraído a más empresas manufactureras que la otra región, una nueva empresa tiene un incentivo para ubicarse donde están las otras empresas. En el modelo central esto es claramente visible,  ya que un aumento en el ingreso local conduce a un mayor aumento en la demanda que el mismo aumento en el ingreso en la otra región.
    \item Equilibrios múltiples. Se señaló  que puede haber múltiples equilibrios a largo plazo; en particular, aglomeración de todas las empresas en el Norte, aglomeración de todas las empresas en el Sur y una distribución de 1:3 de empresas entre el Norte y el Sur. De manera similar, en la discusión del modelo central,  se analizaron tres equilibrios a corto plazo, a saber, la aglomeración en la región 1, la aglomeración en la región 2 y la distribución de la actividad manufacturera entre las dos regiones. Los comentarios sobre la dinámica muestran que estos tres equilibrios a corto plazo también son equilibrios a largo plazo.
    \item Equilibrios estables e inestables. Aunque nunca analizamos un sistema dinámico, argumentamos en el ejemplo de la sección 3.2 que puede haber equilibrios estables e inestables. En particular, la aglomeración de empresas en el Norte o el Sur es un equilibrio estable, mientras que la distribución 1:3 de empresas entre el Norte y el Sur es un equilibrio inestable. Una observación similar es válida para el modelo central. Sin embargo, hay tres calificaciones importantes, a saber (i) introducimos específicamente un sistema dinámico,  (ii) tanto la aglomeración como la dispersión de la actividad manufacturera pueden ser un equilibrio estable.
    \item Equilibrios no óptimos. Vimos que un equilibrio a largo plazo podría ser mejor que otro equilibrio a largo plazo desde el punto de vista del bienestar. Tenga en cuenta que tenemos cuatro agentes económicos diferentes en el modelo central , a saber, los trabajadores de manufactura en ambas regiones y los agricultores en ambas regiones. Esto reduce a tres agentes económicos en el equilibrio a largo plazo, en el que el salario real es el mismo para todos los trabajadores de la industria. Supongamos que echamos un vistazo más de cerca a los equilibrios de aglomeración.  Supongamos por el momento que los equilibrios de aglomeración son estables :\\\\
	Los agricultores se dividen por igual entre las dos regiones. Desde el punto de vista del bienestar, entonces no importa si la manufactura está aglomerada en la región 1 o en la región 2.\\ 
	Suponga que la manufactura está aglomerada en la región 1. Esta es una buena noticia para los agricultores de la región 1 (su bienestar = 10 ), porque tienen fácil acceso a una gran cantidad de variedades producidas localmente. Son malas noticias para los agricultores de la región 2 (su bienestar = 5), porque tienen que importar todas las variedades de fabricación de la región 1. Ahora supongamos que toda la fabricación está aglomerada en la región 2. Esta vez son buenas noticias para los agricultores. En la región 2, y malas noticias para los agricultores de la región 1. A los obreros manufactureros no les importa dónde estén aglomerados, pues su salario real es el mismo . Dado que el bienestar total es esencialmente la suma del salario real de todos los agentes económicos, y el número de agricultores se distribuye de manera exactamente uniforme entre las dos regiones, no importa desde el punto de vista del bienestar total si la manufactura está aglomerada en la región 1 o en región 2 (claramente sí importa para los agentes económicos individuales, en particular los agricultores; bienestar total = 171 en cualquier caso).
    \item Interacción de la aglomeración y los flujos comerciales. Notamos que el llamado efecto del mercado interno era un aspecto crucial de la economía geográfica. Observan la tendencia de las industrias de rendimientos crecientes, en igualdad de condiciones, a concentrarse cerca de sus mercados más grandes y exportar a mercados más pequeños. Este efecto es causado por la interacción de economías de escala externas e internas. Las empresas quieren ubicarse cerca de la demanda, donde puedan beneficiarse de un mercado más grande y posibles derrames. Estas economías de escala dan como resultado que las actividades de rendimientos crecientes sean atraídas hacia los grandes mercados, y más aún si estos lugares tienen un buen acceso a otros mercados. Debido a que las actividades son atraídas hacia los lugares preferidos, los salarios (reales) aumentan y la mano de obra tiene un incentivo para migrar hacia estos lugares, aumentando aún más el atractivo. Al final, una parte desproporcionadamente grande de la actividad termina en estos lugares privilegiados, y esta región se convierte en exportadora neta de manufacturas. De ahí el nombre de efecto del mercado interno.

\end{enumerate}


Es importante ver que la introducción de los costos de transporte es el factor determinante para este efecto. Una comparación entre Krugman (1980) y su artículo de 1979 aclara este punto. ¿Qué sucede si se permite el comercio entre un país grande y uno pequeño en ausencia de costos de transporte? Uno podría esperar un efecto de mercado interno, con el país más grande exportando manufacturas e importando el producto homogéneo. Este no es el caso. El libre comercio asegura que los precios de todas las variedades se igualen entre países y, en ausencia de costos de transporte, esto también iguala los salarios. Además, los costos de producción no se ven afectados por la presencia de otras empresas en el mismo lugar. Por lo tanto, la estructura comercial es indeterminada: la producción de manufacturas puede tener lugar en cualquier parte; ningún lugar tiene una ventaja de costo natural sobre el otro. Sin embargo, con el libre comercio, aumenta el número total de variedades disponibles para los consumidores en cada país. En esencia, esto determina las ganancias del comercio en el modelo. La introducción de los costos de transporte cambia esto drásticamente al crear el efecto del mercado interno, como ya se mostró en Krugman (1980). Concentrar la producción de manufacturas en el mercado más grande permite beneficiarse de economías de escala y, al mismo tiempo, economizar costos de transporte. Esto aumenta los salarios reales de los trabajadores manufactureros en el mercado más grande, lo que hace de esta región el lugar más atractivo para vivir.\\
Finalmente, ya podemos decir algo sobre la estructura del comercio entre las regiones comparando el equilibrio de expansión y el equilibrio de aglomeración entre sí. Supongamos, por ejemplo, que la producción manufacturera está aglomerada en la región 1. En ese caso, los agricultores de la región 2 tienen que importar todas las variedades manufactureras de la región 1. La región 2, por lo tanto, exporta alimentos a cambio de variedades manufactureras, un ejemplo de comercio interindustrial puro. – es decir, el comercio entre regiones de diferentes tipos de bienes. Por el contrario, suponga que la producción manufacturera se distribuye uniformemente entre las dos regiones, al igual que la producción de alimentos (el equilibrio simétrico), usando el especificación de la sección 3.8. Allí se muestra que Y1¼ Y2¼ 1/2 en este caso, lo que implica que la demanda de alimentos en cualquier región es igual a (1 d) Y1¼ (1 d)/2. Dado que el nivel de producción de alimentos en, por ejemplo, la región 1 es igual a f1(1 c)L ¼ (1/2)(1 d)1¼ (1 d)/2 (ver sección 3.8), se deduce que el total la demanda de alimentos en la región 1 es igual a la producción total de alimentos en la región 1. En el equilibrio de expansión, por lo tanto, no hay comercio de alimentos entre las dos regiones. Dado que todos los consumidores gastarán dinero en todas las variedades de fabricación, incluidas las producidas en la otra región, habrá comercio de variedades de fabricación entre las dos regiones. Este es, por lo tanto, un ejemplo de comercio intraindustrial puro, es decir, comercio entre regiones de tipos similares de bienes (importando variedades de fabricación a cambio de otras variedades de fabricación).



