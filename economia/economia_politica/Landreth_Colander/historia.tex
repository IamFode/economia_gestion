\documentclass[10pt]{book}
\usepackage[text=17cm,left=2.5cm,right=2.5cm, headsep=20pt, top=2.5cm, bottom = 2cm,letterpaper,showframe = false]{geometry} %configuración página
\usepackage{latexsym,amsmath,amssymb,amsfonts} %(símbolos de la AMS).7
\parindent = 0cm  %sangria
\usepackage[T1]{fontenc} %acentos en español
\usepackage[spanish]{babel} %español capitulos y secciones
\usepackage{graphicx} %gráficos y figuras.

%---------------FORMATO de letra--------------------%

\usepackage{lmodern} % tipos de letras
\usepackage{titlesec} %formato de títulos
\usepackage[backref=page]{hyperref} %hipervinculos
\usepackage{multicol} %columnas
\usepackage{tcolorbox, empheq} %cajas
\usepackage{enumerate} %indice enumerado
\usepackage{marginnote}%notas en el margen
\tcbuselibrary{skins,breakable,listings,theorems}
\usepackage[Bjornstrup]{fncychap}%diseño de portada de capitulos
\usepackage[all]{xy}%flechas
\counterwithout{footnote}{chapter}
\usepackage{xcolor}
\usepackage[htt]{hyphenat}
%--------------------GRÀFICOS--------------------------

\usepackage{tkz-fct}

%---------------------------------

\titleformat*{\section}{\bfseries\rmfamily}
\titleformat*{\subsection}{\bfseries\rmfamily}
\titleformat*{\subsubsection}{\bfseries\rmfamily}
\titleformat*{\paragraph}{\bfseries\rmfamily}
\titleformat*{\subparagraph}{\bfseries\rmfamily}

%------------------------------------------

\renewcommand{\labelenumi}{\Roman{enumi}.}%primer piso II) enumerate
\renewcommand{\labelenumii}{\arabic{enumii}$)$}%segundo piso 2)
\renewcommand{\labelenumiii}{\alph{enumiii}$)$}%tercer piso a)
\renewcommand{\labelenumiv}{$\bullet$}%cuarto piso (punto)

%----------Formato título de capítulos-------------

\usepackage{titlesec}
\renewcommand{\thechapter}{\arabic{chapter}}
\titleformat{\chapter}[display]
{\titlerule[2pt]
\vspace{4ex}\bfseries\sffamily\huge}
{\filleft\Huge\thechapter}
{2ex}
{\filleft}

\begin{document}

\normalfont
\input xy
\xyoption{all}
\author{\Large Apuntes por FODE}
\title{\small Landreth / Colander \\ \vspace{1cm} \large HISTORIA DEL PENSAMIENTO ECONÓMICO}
\date{}
\pagestyle{empty}
\maketitle
\thispagestyle{empty}
\let\cleardoublepage\clearpage
\tableofcontents								%indice


%------------------------------------------
 
\let\cleardoublepage\clearpage

\chapter{Introducción}
\section{El principal tema de interés del pensamiento económico moderno}
Los individuos desean consumir más bienes y servicios de los que existen lo que provoca una escasez relativa. Es decir, la insuficiencia de recursos en relación con una demanda. A través de la historia se utilizaron cuatro mecanismos para resolver el problema de la escasez:

\begin{enumerate}[1.]
    \item Fuerza bruta.
    \item Tradiciones.
    \item La autoridad (instituciones cómo el estado o la iglesia).
    \item El mercado (Principal mecanismo de asignación).
\end{enumerate}

Cabe destacar que estos mecanismos no son mutuamente excluyentes. Y que no siempre se asignó al mercado como principal mecanismo de asignación. A esto, decir que el mercado es el único mecanismo es totalmente falso, ya que las sociedades modernas utilizan tanto la fuerza, la tradición y la autoridad para resolver el problema de la escasez. Independientemente de cual sea el mecanismo a utilizar para asignar los recursos, la raíz del problema de la escasez obliga a dejar algunos deseos sin satisfacer. Por lo que, estos  mecanismos tienen la función de decidir quién recibe los recursos y quien no.

\subsection{Las Divisiones de la teoría económica moderna}
para abordar el tema de escasez relativa, se centran en la microeconomía y la macroeconomía. La microeconomía normalmente analiza las cuestiones de:
\begin{itemize}
    \item Asignación y
	\begin{itemize}
	    \item Qué se produce y
	    \item Cómo se produce.
	\end{itemize}
    \item Distribución
	\begin{itemize}
	    \item Cómo se reparte la renta real entre los miembros de la sociedad.
	\end{itemize}
\end{itemize}

Centrando el estudio desde el individuo hasta la sociedad, utilizando los principales instrumentos de análisis económico: la oferta y la demanda; con el fin de explicar las fuerzas que determinan los precios relativos para asignar los recursos y la distribución de la renta.\\

La macroeconomía analiza las cuestiones de:
\begin{itemize}
    \item estabilidad y 
    \item crecimiento.
\end{itemize}

Que comienza a analizar la sociedad hasta llegar al individuo, utilizando variables agregadas de toda la economía.


\section{Nuestro enfoque de la historia del pensamiento económico}
\subsection{Enfoque relativista y absolutista}
Dos enfoques que pueden responder a la pregunta ¿Cómo surge la teoría económica?. El primero, toma mucho en cuenta la historia como tal. Y el segundo, pone más enfoque en las fuerzas internas, cómo también en las últimas investigaciones. A esto, no podremos decir que estas posturas son convincentes en sí mismas ni por si mismas, por lo que es más fructífero concebir la historia del pensamiento económico como un proceso dinámico de interacción entre las fuerzas externas e internas de la disciplina que dan origen a nuevos avances teóricos.

\subsection{Economistas ortodoxos y heterodoxos}
Una manera de comprender las cuestiones que separan a estos economistas, es examinar las preguntas a las que trataban de responder. Mientras los ortodoxos se ocuparon de los cuatros cuestiones (asignación, distribución, estabilidad y crecimiento), los economistas heterodoxos estudiaron las fuerzas que provocan cambios en la sociedad y la economía. Es decir, lo que los autores ortodoxos consideran dados (ceteris paribus), es lo que los heterodoxos tratan de explicar. Y de aquí, que se forma la escuela dominante y no dominante, según si es o no aceptado por los académicos. Tengamos en cuenta que las escuelas que no forman parte del pensamiento dominante desempeñan un importante papel en la evolución de la disciplina: polinizan la teoría dominante y la obligan a ser sincera al señalar sus puntos débiles o sus incoherencias.\\

Entre las escuelas heterodoxas se encuentran:

\begin{itemize}
    \item los austriacos,
    \item los institucionalistas,
    \item los postkeynesianos y 
    \item los radicales.
\end{itemize}

\section{Cuestiones metodológicas}
\subsection{La economía como arte y como ciencia}
Tal vez la distinción más importante en el pensamiento económico sea la que se hace entre:

\begin{itemize}
    \item El arte de la economía, 
    \item la economía positiva y 
    \item la economía normativa.
\end{itemize}

La economía positiva se ocupa de las fuerzas que gobiernan la actividad económica, realizando preguntas como:

\begin{itemize}
    \item ¿Cómo funciona la economía?,
    \item ¿qué fuerzas determinan la distribución de la renta?.
\end{itemize}

La economía normativa se ocupa explícitamente de cómo debería ser. Es la rama de filosófica de la economía que integra la economía y la ética.\\

El arte de la economía se ocupa de cuestiones relacionadas con la política económica. Relacionando la economía normativa con la ciencia económica, haciendo preguntas como:

\begin{itemize}
    \item Si éstos son mis objetivos normativos y si ésta es la forma en que funciona la economía, ¿cuál es la mejor manera de lograr estos objetivos?.
\end{itemize}

A esto, la metodología positiva es formal y abstracta, el cual trata de separar las fuerzas económicas de las fuerzas políticas y sociales. Por otro lado, el arte de la economía es más compleja, ya que se ocupa de la política económica y debe abordar las relaciones entre la política, las fuerzas sociales y las fuerzas económicas. En el arte de la economía hay que integrar todas las dimensiones de un problema de las que se hace abstracción en la economía positiva. Pero veremos que los escritos metodológicos modernos giraron en torno a la economía positiva.


\subsection{La importancia de la verificación empírica}
Cómo respondemos a las preguntas de:

\begin{itemize}
    \item ¿Qué sabemos?.
    \item ¿Cómo sabemos que lo que sabemos es correcto?.
\end{itemize}

Dependerá de la respuesta a la pregunta

\begin{itemize}
    \item ¿Existe una verdad última que los científicos estén tratando de revelar (POSTURA ABSOLUTISTA) o no existe ninguna verdad subyacente (POSTURA RELATIVISTA)?, si existe una verdad última ¿cómo la encontramos?, si no existe ninguna ¿hay algunas proposiciones más verdaderas que otras?.
    \item Ahora, si existe una verdad última, el problema es cómo saber que ya la hemos descubierto.
\end{itemize}

Para el análisis económico definiremos tres términos que desempeñaron un papel importante 

\begin{itemize}
    \item Inductivo
    \item deductivo y
    \item abductivo.
\end{itemize}


\subsection{La evolución del pensamiento metodológico}
\subsubsection{La aparición del positivismo lógico}
En el positivismo lógico se unieron el razonamiento deductivo y el deseo positivista de dejar que los hechos hablaran por sí mismos por lo que los análisis normativos se desterraron de la economía por no considerarse científicos.

\subsubsection{Del positivismo lógico al falsacionismo}
Donde mejor se expresa esta preocupación es en los escritos de Karl Popper, quien en la década de 1930 afirmó que las verificaciones empíricas no establecen la verdad de una teoría, sólo su falsedad; esa es la razón por la que el enfoque de Popper se denomina a veces falsacionismo. Según Popper, nunca es posible “verificar” una teoría, ya que no es posible realizar todas las contrastaciones posibles de la teoría. El progreso de la ciencia depende, según Popper, de la continua falsación de las teorías. La teoría que impere será la que explique la mayor variedad de observaciones empíricas y que aún no se haya falsado.\\

El rechazo moderno de la teoría de Popper no es infundado: el falsacionismo tiene algunos problemas serios. En primer lugar, las predicciones empíricas de algunas teorías no pueden contrastarse porque no existe la tecnología necesaria para contrastarlas. ¿Qué debe hacerse con esas teorías? En segundo lugar, es difícil saber cuándo se ha falsado o no una teoría. Por ejemplo, si una contrastación empírica no produce los resultados esperados, el investigador puede atribuir, y a me- nudo atribuye, el fracaso a fallos del procedimiento de contrastación o a algún factor exógeno. Por tanto, una sola contrastación empírica negativa a menudo no invalida la teoría. El tercer problema se debe a la manera de pensar de los investigadores, que pueden no contrastar las implicaciones de una teoría establecida, suponiendo que son verdaderas. Ese modo de pensar puede llevar a no aceptar teorías nuevas y posiblemente más defendibles.


\subsubsection{Del falsacionismo a los paradigmas}
En respuesta en parte a estos problemas, Thomas Kuhn desbancó la metodología del falsacionismo en The Structure of Scientific Revolutions (1962) introduciendo en el debate el concepto de paradigma. Que se refiere al enfoque y acervo de conocimientos dados por los investigadores que se ajustan al pensamiento científico dominante. Donde, los demás estudios llevan a menudo a descubrir anomalías que el paradigma no es capaz de explicar, pero la existencia de esas anomalías no es suficiente para desechar el paradigma dominante; sólo puede desecharlo otro paradigma que sea capaz de abordar mejor las anomalías.\\

Una vez que se ha desarrollado ese paradigma superior, es posible una revolución científica. En la ciencia revolucionaria, primero el paradigma existente es rechazado por una parte de la comunidad científica y después el antiguo y el nuevo paradigma comienzan a competir y la comunicación entre los investigadores de los bandos contrarios se hace difícil. Al final, si la revolución tiene éxito, se plantean nuevas preguntas en el nuevo marco y se desarrolla una nueva ciencia normal.

\subsubsection{De los paradigmas a los programas de investigación}
Lakatos, estableció la idea de que la teoría podía no contener la verdad. Donde, se tiene a científicos rivales, que analizan e intentan falsar, pero aceptar un conjunto de postulados lógicos que constituyen el núcleo duro. Sólo si se falsan suficientes implicaciones, se reconsiderarán los supuestos del núcleo duro. Lakato llamo \textbf{progresivos} si el proceso de falsación progresaba y \textbf{degnerativos} en caso contrario.

La obra de Lakatos tiene dos características significativas: 
\begin{enumerate}[1.]
    \item  reconoce la complejidad del proceso por el que se falsa una teoría; y 
    \item mientras que los análisis anteriores exigían que predominara una teoría, Lakatos prevé la existencia simultánea de múltiples teorías viables cuyos méritos relativos no son fáciles de discernir.
\end{enumerate}

\subsubsection{De los programas de investigación a los enfoques de la metodología basados en la sociología y la retórica}
Una obra que se distancia de una forma mucho más radical de la metodología anterior es Against Method: An Outline of an Anarchistic Theory of Knowledge (1975) de Paul Feyerabend. Este autor sostiene que la aceptación de cualquier método limita la creatividad en la resolución de los problemas y que la mejor ciencia es, pues, la que no utiliza ningún método; en otras palabras, todo vale.\\

El enfoque retórico de la metodología pone el acento en la persuasión del lenguaje y sostiene que una teoría puede aceptarse, no porque sea inherentemente verdadera sino porque sus defensores consiguen convencer a otros de su valor por medio de su retórica superior. El enfoque sociológico examina las restricciones sociales e institucionales que influyen en la aceptabilidad de una teoría. La financiación, el empleo y el control de las revistas pueden influir tanto en qué teoría se acepta como en la capacidad de la teoría para explicar exactamente los fenómenos.

\subsection{Conclusiones metodológicas}
Aunque raras veces se habla de metodología, es la metodología la que explica en última instancia las diferencias entre los economistas. Los formalistas tienden más a utilizar una metodología positivista lógica o falsacionista y a creer en un enfoque absolutista; el resto tiende más a utilizar un enfoque sociológico o retórico y a creer en un enfoque relativista.


\part{La economía preclásica}

\chapter{Los comienzos del pensamiento económico preclásico}

\end{document}
