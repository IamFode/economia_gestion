\documentclass[10pt]{book}
\usepackage[text=17cm,left=2.5cm,right=2.5cm, headsep=20pt, top=2.5cm, bottom = 2cm,letterpaper,showframe = false]{geometry} %configuración página
\usepackage{latexsym,amsmath,amssymb,amsfonts} %(símbolos de la AMS).7
\parindent = 0cm  %sangria
\usepackage[T1]{fontenc} %acentos en español
\usepackage[spanish]{babel} %español capitulos y secciones
\usepackage{graphicx} %gráficos y figuras.

%---------------FORMATO de letra--------------------%

\usepackage{lmodern} % tipos de letras
\usepackage{titlesec} %formato de títulos
\usepackage[backref=page]{hyperref} %hipervinculos
\usepackage{multicol} %columnas
\usepackage{tcolorbox, empheq} %cajas
\usepackage{enumerate} %indice enumerado
\usepackage{marginnote}%notas en el margen
\tcbuselibrary{skins,breakable,listings,theorems}
\usepackage[Bjornstrup]{fncychap}%diseño de portada de capitulos
\usepackage[all]{xy}%flechas
\counterwithout{footnote}{chapter}
\usepackage{xcolor}
\usepackage[htt]{hyphenat}
%--------------------GRÀFICOS--------------------------

\usepackage{tkz-fct}

%---------------------------------

\titleformat*{\section}{\LARGE\bfseries\rmfamily}
\titleformat*{\subsection}{\Large\bfseries\rmfamily}
\titleformat*{\subsubsection}{\large\bfseries\rmfamily}
\titleformat*{\paragraph}{\normalsize\bfseries\rmfamily}
\titleformat*{\subparagraph}{\small\bfseries\rmfamily}

%------------------------------------------

\renewcommand{\labelenumi}{\Roman{enumi}.}%primer piso II) enumerate
\renewcommand{\labelenumii}{\arabic{enumii}$)$}%segundo piso 2)
\renewcommand{\labelenumiii}{\alph{enumiii}$)$}%tercer piso a)
\renewcommand{\labelenumiv}{$\bullet$}%cuarto piso (punto)

%----------Formato título de capítulos-------------

\usepackage{titlesec}
\renewcommand{\thechapter}{\arabic{chapter}}
\titleformat{\chapter}[display]
{\titlerule[2pt]
\vspace{4ex}\bfseries\sffamily\huge}
{\filleft\Huge\thechapter}
{2ex}
{\filleft}

\begin{document}

\normalfont
\input xy
\xyoption{all}
\author{\Large Apuntes por FODE}
\title{\small Landreth / Colander \\ \vspace{1cm} \large HISTORIA DEL PENSAMIENTO ECONÓMICO}
\date{}
\pagestyle{empty}
\maketitle
\thispagestyle{empty}
\let\cleardoublepage\clearpage
\tableofcontents								%indice


%------------------------------------------
 
\let\cleardoublepage\clearpage

\chapter{Introducción}
\section*{El principal tema de interés del pensamientoe económico moderno}
A principios de la década de 1900 algunas sociedades adoptaron un sistema de planificación central, que implicaba el control estatal de la asignación de los recursos. En Europa oriental, se observan movimientos para pasar de una economía autoritaria a una economía de mercado, cuyos resultados son inciertos.\\
Las sociedades modernas de mercado utilizan \textbf{la fuerza, la tradición y la autoridad, además de los mercados.}\\
\subsection*{Divisiones de la teoría económica moderna}
En el pensamiento económico moderno, los problemas relacionados con la escasez relativa generalmente se dividen en microeconomía y macroeconomía. La microeconomía analiza las cuestiones de la asignación y la distribución. . La macroeconomía analiza las cuestiones de la estabilidad y el crecimiento. 
\section*{Nuestro enfoque de la historia del pensamiento económico}
\subsection*{Enfoque relativista y absolutista}
A los historiadores relativistas les interesan las fuerzas históricas, económicas, sociológicas y políticas que llevaron a los hombres y a las mujeres a examinar ciertas cuestiones económicas y  el modo en que estas fuerzas determinaron el contenido de la teoría emergente. Sostienen que la historia desempeña un papel importante en el desarrollo de todas las teorías económicas. Un relativista haría hincapié, por ejemplo, en las relaciones entre la aparición y el contenido de la economía clásica y la industrialización de Inglaterra, entre la economía ricardiana y el conflicto entre los terratenientes y los capitalistas ingleses y entre la economía keynesiana y la Gran Depresión de los años 30.\\
Los historiadores absolutistas ponen el acento en las fuerzas internas, como la creciente profesionalización de la economía, para explicar el desarrollo de la teoría económica. Los absolutistas sostienen que el progreso de la teoría no refleja meramente las circunstancias históricas sino que depende del descubrimiento y la explicación de problemas o paradojas sin resolver por parte de profesionales formados que reaccionan a los avances intelectuales que surgen en el seno de la profesión. Según este enfoque, es posible ordenar las teorías en términos absolutos según su valor; lo más probable es que la teoría más reciente contenga menos errores y se aproxime más a la verdad que las teorías anteriores.\\
Es más fructífero concebir la historia del pensamiento económico como un proceso dinámico de interacción entre las fuerzas externas e internas de la disciplina que dan origen a nuevos avances teóricos.
\subsection*{Economistas ortodoxos y heterodoxos}
Una manera de comprender las cuestiones que separan a los autores ortodoxos de los heterodoxos es examinar las preguntas a las que trataban de responder. Mientras que los teóricos ortodoxos modernos se han ocupado principalmente de los cuatro problemas de la asignación, la distribución, la estabilidad y el crecimiento, los economistas heterodoxos han estudiado las fuerzas que provocan cambios en la sociedad y la economía.
\section*{El papel de los economistas heterodoxos}
\subsection*{Cómo influyen los economistas discrepantes en el pensamiento económico y en la profesión}
Una manera de comprender el papel de los economistas discrepantes es examinar un segmento de la historia del pensamiento económico. 

\subsection*{La economía como arte y como ciencia}
Tal vez la distinción más importante en el pensamiento económico sea la que se hace entre el arte de la economía, la economía positiva y la economía normativa. La economíapositiva se ocupa de las fuerzas que gobiernan la actividad económica.\\
La economía normativa se ocupa explícitamente de qué debe ser.\\
La distinción es importante porque la economía positiva y el arte de la economía tienen metodologías muy distintas. La metodología de la economía positiva es formal y abstracta; trata de separar las fuerzas económicas de las fuerzas políticas y sociales. La metodología del arte de la economía es más compleja, ya que se ocupa de la política económica y debe abordar las relaciones entre la política, las fuerzas sociales y las fuerzas económicas.\\
\subsection*{La importancia de la verificación empírica}
$“$Abductivo$”$ es el nombre que dio un filósofo pragmático, Charles Peirce, a una combinación específica del enfoque inductivo y el deductivo. El concepto abductivo es importante para la economía y otros estudios de sistemas complejos. El razonamiento abductivo utiliza tanto la deducción como la inducción para dar una explicación razonable a lo que ocurre. Conjuga la historia, las instituciones y el estudio empírico para comprenderlo; sin embargo, no pretende ofrecer una teoría definitiva, ya que, cuando se trata de un complejo sistema, no es posible llegar a tener una teoría definitiva.\\
\part{La economía preclásica} 

\chapter*{Los comienzos del pensamiento económico preclásico}
Los pensadores chinos, griegos, árabe-islámicos y escolásticos no analizaron la economía como una disciplina independiente; estaban interesados en cuestiones mucho más amplias y filosóficas. Y como la actividad económica que observaron en esos primeros tiempos no estaba organizada en un sistema de mercado como el que conocemos hoy, no se ocuparon de la naturaleza y el significado de un sistema de precios sino de cuestiones éticas relacionadas con la justicia y la equidad. Sin embargo, sus ideas sobre algunos fenómenos económicos sirvieron de base a pensadores posteriores. La excepción a esta generalización es Guan Zyong, cuyas obras, aunque adelantadas a su tiempo, eran desconocidas en Occidente.\\
Los pensadores griegos, especialmente Hesiodo y Jenofonte, estudiaron la administración de los recursos en el ámbito del hogar y del productor y extrajeron sus conclusiones sobre la eficiencia y su relación con una división correcta del trabajo. Aristóteles y otros griegos examinaron el papel de la propiedad privada y de los incentivos. En su análisis de las necesidades y los deseos, Aristóteles planteó cuestiones eternas sobre el fin de la vida, cuestiones que se convirtieron en el tema de interés en los análisis posteriores de los escolásticos.\\
Durante la Edad Media, se tradujeron muchos escritos griegos al árabe y del árabe al latín. Los estudiosos árabes influyeron, pues, en el pensamiento escolástico en los campos de la filosofía, la ética, las ciencias y la economía hasta un grado que no se ha reconocido totalmente hasta los últimos cincuenta años. Y aunque la doctrina religiosa musulmana y la cristiana eran esencialmente hostiles a la actividad económica, no pudieron eliminar todas las actividades económicas. Al-Ghazali e Ibn Khaldun, al tratar de comprender su época, consiguieron, pues, aportar algunas ideas útiles sobre la actividad económica y contribuyeron así al largo proceso histórico de construcción de
los cimientos del conocimiento de la economía.

\chapter{El mercatilismo, la fisiocracia y otros precursores del pensamiento económico clásico.}
\section*{El mercatilismo}
\subsection*{Aportaciones teóricas del mercantilismo}
El logro más importante de los últimos mercantilistas posiblemente fuera el reconocimiento explícito de la posibilidad de analizar la economía. Este avance representó la transferencia a las ciencias sociales de actitudes que imperaban por entonces en las ciencias físicas. Se materializó plenamente tras el periodo en que vivió Isaac Newton (1642–1727) y sus efectos aún se sienten hoy. La sustitución del análisis moral de los escolásticos por el análisis de causa–efecto no representa, sin embargo, una clara ruptura con el pasado, ya que el análisis lógico fue utilizado por algunos escolásticos y la moralización aún existe en la literatura económica moderna. Pero la idea de que las leyes de la economía podían descubrirse por medio de los mismos métodos que revelaron las leyes de la física fue un paso importante para el desarrollo posterior de la teoría económica. Muchos mercantilistas veían una causalidad muy mecánica en la economía y creían que si se comprendían las reglas de esta causalidad, se podría controlar la economía. Por tanto, con una acertada legislación sería posible influir positivamente en el curso de los acontecimientos económicos y el análisis económico indicaría que tipos de intervención del Estado lograrían el fin perseguido. Los mercantilistas se dieron cuenta, sin embargo, de que la interferencia del Estado no debe ser caprichosa o complicar verdades económicas básicas como la ley de la oferta y la demanda. Algunos dedujeron correctamente, por ejemplo, que si se fijaban unos precios máximos inferiores a los de equilibrio, había exceso de demanda y escasez. Los mercantilistas posteriores aplicaron frecuentemente los conceptos de hombre económico y el motivo de los beneficios para estimular la actividad económica. Sostenían que los gobiernos no pueden cambiar la naturaleza básica de los seres humanos, especialmente sus impulsos egoístas. El político considera dados estos factores e intenta crear una serie de leyes e instituciones que canalicen estos impulsos para aumentar el poder y la prosperidad de la nación. Como veremos, muchos de los mercantilistas posteriores se dieron cuenta de los graves errores analíticos de sus predecesores. Reconocieron, por ejemplo, que la cantidad de dinero no es una medida de la riqueza de una nación, que todas las naciones no podían tener una balanza comercial favorable, que ningún país podía tener una balanza comercial favorable a largo plazo, que el comercio puede ser mutuamente beneficioso para las naciones y que la especialización y la división del trabajo beneficiarían a las naciones que las practicaran. Un creciente número de autores recomendó una reducción del grado de intervención del Estado. La literatura mercantilista contiene, pues, afirmaciones en las que se observa un incipiente liberalismo clásico.

\section*{Precursores influyentes del pensamiento clásico}
\subsection*{Thomas Mun}
Su pensamiento era típicamente mercantilista, en el sentido de que confundía la riqueza de una nación con sus reservas de metales preciosos y, por tanto, abogaba por una balanza comercial favorable y la entrada de oro y plata para saldarla. Creía que el gobierno debía regular el comercio exterior para conseguir una balanza favorable, fomentar la importación de materias primas baratas y la exportación de bienes manufacturados, aprobar aranceles protectores sobre los bienes manufacturados importados y adoptar otras medidas para aumentar la población y mantener los salarios en un nivel bajo y competitivo.\\
Cuando se publicó la última edición del famoso libro de Mun en 1755, muchos de los mercantilistas más perspicaces estaban dándose cuenta de los graves errores del paradigma mercantilista. Estos mercantilistas liberales estaban comenzando a formular los fundamentos intelectuales de la obra Wealth of Nations de Smith.

\subsection*{William Petty}
Es el primer autor que defendió la medición de las variables económicas. \\
\textbf{Political Arithmetic} de Petty se escribió en 1676, pero no se publicó hasta 1690. Petty parecía ser consciente de que estaba abriendo nuevos caminos al analizar la metodología de la aritmética política.\\
Petty parece que fue quien primero abogó explícitamente por el uso de lo que llamaríamos técnicas estadísticas para medir los fenómenos sociales. Trató de medir la población, la renta nacional, las exportaciones, las importaciones y el stock de capital de una nación. Sus métodos eran increíblemente rudimentarios, lo que llevó a Adam Smith a decir que la aritmética política le parecía de poca utilidad.\\

\subsection*{Bernard Mandeville}
Su obra \textbf{Fable of the Bees; Or, Private Vices, Publick Benefits (1714)} no sólo provocó a sus contemporáneos sino que ha continuado siendo de interés para los estudiosos de la literatura, la filosofía, la psicología y la economía. Keynes dedica dos páginas de la General Theory a analizar positivamente a Mandeville.\\
Los impulsos egoístas racionales de los seres humanos promovían el bien social porque el sentimiento moral atemperaba el egoísmo y permitía comprender la diferencia entre el bien y el mal y elegir el camino correcto. Mandeville sostenía que el egoísmo era un vicio moral, pero que de los actos egoístas podía surgir el bien social si estos actos eran debidamente canalizados por el gobierno.\\

\subsection*{David Hume}
Hume abrazó las ideas de John Locke, que pensaba que el nivel de actividad económica de una economía depende de la cantidad de dinero y de su velocidad, y realizó una descripción razonablemente completa de las relaciones entre la balanza comercial de un país, la cantidad de dinero y el nivel general de precios.\\
Según Hume, una economía no podía mantener continuamente una balanza comercial favorable, como defendían muchos mercantilistas. Una balanza comercial favorable provocaba un aumento de la cantidad de oro y plata (metales preciosos) dentro de la economía. El aumento de la cantidad de dinero provocaba una subida del nivel de precios en la economía que tenía la balanza comercial favorable. Si un país tenía una balanza comercial favorable, algún otro u otros tenían que tener una balanza desfavorable.\\
Los mercantilistas habían afirmado que las variaciones de la oferta monetaria podían aumentar la producción real. Los clásicos sostenían que la producción real no dependía de la cantidad de dinero sino de fuerzas reales: la oferta de trabajo, los recursos naturales, los bienes de capital y la estructura institucional.\\
\textbf{Hume buscó una conexión entre la libertad económica –la libertad para vender nuestros recursos, ya fuera trabajo o no, en el momento, en el lugar y al precio que quisiéramos; la libertad para producir y vender los frutos de nuestras actividades; y la libertad para comprar productos o factores sin limitaciones impuestas por fuerzas externas– y la libertad política. Hume sostenía que el aumento de la libertad económica y el aumento de la libertad política iban unidos.}\\
Hume fue un precursor de la distinción que hicieron más tarde Nassau Senior, John Neville Keynes y Lionel Robbins entre las afirmaciones positivas y las normativas. Que lo que debe ser (afirmaciones normativas) no puede derivarse de lo que es (afirmaciones positivas) se conoce con el nombre de Máxima de Hume.

\subsection*{Richard Cantillon}
Dados unos mercados competitivos en los que los empresarios buscan clientes en los mercados de bienes finales y compiten entre sí en los mercados de factores, Cantillon fue capaz de señalar los procesos de ajuste que se producen cuando cambian las demandas, los costes, la tecnología u otros factores.\\
Dividió la economía en sectores y analizó el flujo de renta entre ellos; aunque no formuló explícitamente una tabla económica para representar estos flujos, influyó claramente en Quesnay, quien sí lo hizo.

\section*{La fisiogracia}

\end{document}
