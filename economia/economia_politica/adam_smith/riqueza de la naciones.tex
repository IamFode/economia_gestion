\documentclass[10pt]{article}
\usepackage[text=17cm,left=2.5cm,right=2.5cm, headsep=20pt, top=2.5cm, bottom = 2cm,letterpaper,showframe = false]{geometry} 	
\usepackage{latexsym,amsmath,amssymb,amsfonts}	%(símbolos de la AMS).7
\parindent = 0cm 								%sangria
\usepackage{lmodern}							% tipos de letras
\usepackage[T1]{fontenc}						%acentos en español
\usepackage[spanish]{babel}
\usepackage{titlesec} %formato de títulos
\pagestyle{empty}								%elimina numeración de página
\usepackage{multicol}
\usepackage{xcolor}



\begin{document}
\begin{center}
\huge Investigación y causa de la riqueza de las naciones\\
\vspace*{0.5cm}
\large Adam Smith.\\
\vspace{1cm}
\Large Apuntes por Fode.
\vspace{1.5cm}
\end{center}
\section*{Introducción y plan de la obra}
Las causas de este ventajoso adelantamiento en las facultades ó principios productivos del trabajo, y el orden con que se distribuye su producto en las diferentes clases y condiciones de la sociedad son el asunto del Libro primero de esta Investigación.\\
El segundo Libro trata de la naturaleza del fondo capital, del modo con que se va aumentando ó acumulando gradualmente, y de las diferentes cantidades ó porciones de trabajo que se ponen en movimiento según los diferentes modos de emplearlo.\\ 
Desde la ruina del Imperio Romano la política de Europea ha sido mas favorable a las artes, manufacturas, y comercio, que pertenecen á la industria urbana, que á la agricultura , que es la rústica. Las circunstancias pues que han inducido á esta política se explican en el Libro tercero.\\
En el Libro cuarto se procura explicar con la claridad posible , y examinar a fondo aquellos diferentes sistemas, y los principales efectos que han producido en distintas épocas y Naciones.\\
El quinto y último trata de las rentas del Soberano ó de la República.   
\part*{\center Libro I}
\begin{multicols}{2}
\section*{De la división del trabajo}
Los mayores adelantamientos en las facultades, ó principios productivos del trabajo, y la destreza, pericia, y acierto con que éste se aplica y dirige en la sociedad no parecen efectos de otra causa que de la división del trabajo mismo.\\
la división del trabajo no puede ser tan obvia; y por con siguiente es siempre menos  considerada.\\
Esta separación se ve con mas generalidad y perfección en los países que están elevados á mas alto grado, de industria y cultura , siendo por lo común obra de muchos en un estado culto lo que de uno solo en una sociedad ruda y poco cultivada.\\
Este considerable aumento que un mismo número de manos puede producir en la cantidad de la Obra en consecuencia dé- la división dél trabajó nace de tres circunstancias diferentes:
\begin{enumerate}
\item De la mayor destreza de cada operario particular,
\item Del ahorro de aquel tiempo que comúnmente se pierde en pasar de una operación á otra de distinta especie y
\item de la invención de un número grande de máquinas que facilitan y abrevian el trabajo, habilitando á un hombre para hacer la labor de muchos.
\end{enumerate}
Yo he visto á varios mozos de edad como de veinte años, que por no haber tenido otro oficio que el hacer clavos , cuando lo ejercían, podían cada uno hacer al día mas de dos mil y trecientos.\\
El uno provee al otro de lo que le hace falta , y este a aquel recíprocamente, y de este modo viene a difundir en todas las clases de la sociedad una plenitud general y admirable.\\
El pastor , el que separa las clases de lanas , el cardador , el tintorero , el hilandero, al tejedor, el batanero, el sastre , y otros muchos , todos tienen que juntar sus operacines para llegar a completar una producción tan. grosera y tan basta.\\
\section*{Del principio que motiva la división del trabajo}
Es una consecuencia necesaria, aunque lenta y gradual de cierta propensión genial del hombre que tiene por objeto una utilidad menos extensiva; la propensión, es a saber, de negociar, cambiar o permutar una cosa por otra. \\
Para el hombre se halla siempre constituido, según la ordinaria providencia, en la necesidad de la ayuda de su semejante.\\
Cualquiera que en materia de intereses estipula de otro, se propone hacer esto: Dame tu lo que me hace falta y yo te daré lo que te falta a ti. En este entorno nunca hablamos de nuestras necesidades, si no de sus ventajas. \textbf{Solo el mendigo confía toda su subsistencia principalmente a la benevolencia y compasión de sus conciudadamos} y aún el mendigo no pone toda su confianza. Lo que motiva a la división del trabajo es la variedad de talentos que existe en ella.
\section*{Que la división del trabajo tiene sus límites según la extensión del mercado público}
Imposible es que en semejantes lugares pueda mantenerse un artífice con un sólo labor.\\
De todos los países que se extendían por las cosas del mediterráneo, Egipto según parece fue el primero en que se cultivaron y recibieron con alguna perfección las manufacturas y la agricultura. Es muy verosímil que la extensión y comodidad de esta navegación interna fuse una de las causas principales de unos progresos tan tempranos como los de Egipto.\\
  Los de la agricultura y manufactura parece también haber sido muy antiguos en las provincias de Bengala en la India Oriental, y en algunas también del Imperio de la China. 
\section*{Del origen y uso de la Moneda}
El producto propio es muy poco lo que puede suministrar al hombre de tantas cosas como necesarias. Para subvenir a la mayor parte de sus necesidades tiene que permutar o cambiar aquella porción sobrante del producto de su trabajo, o la que excede de su consumo, por otra porción del ajeno. Es decir el hombre vive de la permutación.\\
En las edades mas rudas de la sociedad, se dice, haber sido el ganado el instrumento común del comercio. Como dice Homero, las armas de Dyomedes no costaron mas de nueve bueyes, pero las de Glauco ciento. En la abissinia, se asegura haber sido la sal el instrumento del comercio, en la india cierto genero de conchas, pescado salado en Newfundlandia, el tabaco en la Virginia, el azúcar en algunas colonias Inglesas de las Indias Occidentales, los cueros en algunos otros países.\\
Los metales, el medio de cambio que no hay otra cosa mas durable. \\
El hierro fue entre los Espartanos el instrumento común del comercio, el cobre entre los antiguos Romanos, y el oro y la plata entre las naciones ricas y comerciante.\\
Por autoridad de un antiguo escritor  llamado Timéo, que hasta tiempo de Servio Tullio no tuvieron los Romanos moneda acuñada, sino que usaron de barras de cobre sin marca para comprar cuanto necesitaban. Estas barras rudas y groseras hacían en aquellos tiempos las funciones de moneda.\\
{\color{blue} Abraham pesó a Ephrón los cuatrcientos siclos de plata que se convino a pagar por el campo de Machpelah}. Las rentas de los antiguos Reyes Anglosajones se dice haber pagado en especies. Guillelmo el Conquistador introdujo en aquel reino la costumbre de que se pagasen en moneda. \\
Al disminuir el peso de cada moneda en el tiempo pero no así su valor los príncipes y estados soberanos que las hicieron se habilitaron en la apariencia para pagar sus deudas, y cumplir con sus contraídas obligaciones con una cantidad menor que la que en otros casos hubieran necesitado. Pero solo fue apariencia ya que todos los deudores del estado gozarían también del mismo privilegio. \\
{\color{blue} La palabra VALOR tiene dos distintas inteligencias, algunas veces significa la utilidad de algún objeto particular o VALOR DE UTILIDAD, y otras aquella aptitud o poder que tiene para cambiarse por otros bienes a voluntad del que posee la cosa, o también se podría llamar  VALOR DE CAMBIO.} Muchas cosas que tienen más del de utilidad suelen tener menos del de valor, y por el contrario a veces las que tiene mas de este tienen muy poco, o ninguno del otro. Pro ejemplo no hay cosa mas útil que el agua y apenas con ella se podrá comprar otra alguna, ni habrá cosa que pueda darse por ella a cambio. Por el contrario un diamante apenas tiene valor intrínseco de utilidad, y por lo común pueden permutarse pro él muchos bienes de gran valor.
\section*{Del precio real y nominal de toda mercadería, o del precio en trabajo, y precio en moneda.}
{\color{blue}Todo hombre es rico o pobre según el grado en que puede gozar por si de las cosas necesarias y deleitables para la vida humana. Será pobre o rico, a medida de la cantidad de ajeno trabajo que el puede tener a su disposición, o adquirir de otro.} El trabajo pues en la medida o mensura real del valor permutable de toda mercadería. El trabajo fue el precio primitivo, donde se compro originalmente todo genero de riqueza.\\
{\color{blue} La riqueza, como dice Mr. Hobbes, es cierta especie de poder.}\\
Para el cambio mas bien se compara una mercadería con otra que con el trabajo. Pero de estos el dinero es el mejor estimador de valor permutable. {\color{blue} En todo tiempo, y en todo lugar aquello es mas caro realmente que cuesta mas trabajo adquirirlo, aquello es mas barato que se adquiere con mas facilidad y menos trabajo.} Este pues, como que nunca varía en su valor propio, e intrínseco, es el único precio, ultimo, real y estable porque deben estimarse, y con que compararse deben los valores de las mercaderías en todo tiempo y lugar. Este es su precio real, y el de la moneda precio nominal solamente.\\
{\color{blue}El trabajo tiene también precio real y nominal. El real se deberá decir que consiste en la cantidad de las cosas necesarias y útiles que por él se reporta o adquiere, y el nominal en la del dinero. En cuyo supuesto el trabajador será rico o pobre bien o mal remunerado a proporción del precio real, no nominal de su trabajo.}
Mientras permanece cierta proporción fija entre diferentes metales, o sus respectivos valores en moneda, el valor del mas precioso es el que regula el de las demás monedas.
\section*{De las partes integrantes o componentes del precio de toda mercadería}
La única circunstancia que puede dar regla para la permutación reciproca de unas cosas por otras de distinta especia para ser la proporción entre las diferentes cantidades de trabajo que se necesitan para adquirirlas.\\
En el estado mas culto de la sociedad la consideración o las circunstancias de superior fatiga y mayor destreza se aplica regularmente a los salarios del trabajo.
\end{multicols}
\end{document}