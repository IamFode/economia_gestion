\documentclass[10pt]{article}
\usepackage[text=17cm,left=2.5cm,right=2.5cm, headsep=20pt, top=2.5cm, bottom = 2cm,letterpaper,showframe = false]{geometry} 	
\usepackage{latexsym,amsmath,amssymb,amsfonts}	%(símbolos de la AMS).7
\parindent = 0cm 								%sangria
\usepackage{lmodern}							% tipos de letras
\usepackage[T1]{fontenc}						%acentos en español
\usepackage[spanish]{babel}
\usepackage{titlesec} %formato de títulos
\pagestyle{empty}								%elimina numeración de página
\usepackage{multicol}
\usepackage{enumerate}

\begin{document}
\begin{center}
\huge Ensayo sobre la Naturaleza del comercio en general\\
\vspace*{0.5cm}
\large Richard Cantillon\\
\vspace{1cm}
\Large Apuntes por Fode.
\vspace{1.5cm}
\end{center}


\section*{Prefacio}
Con la única excepción de los impuestos, convierten a la obra de Cantillon en "un producto cultual tan valioso como el descubrimiento de la circulación de la sangre, por Harvey", según la feliz frase de Henry Higgs. Admirará el lector de esta obra la justeza de muchas afirmaciones hechas por Cantillon hace dos siglos, pero adaptables precisamente a las circunstancias tan nuevas y tan viejas de la actualidad.\\
En esta obra vemos anticipados muchos de nuestros presentes problemas monetarios, y en ella encontramos el hilo luminoso para salir con gracia de los peores laberintos ideológicos y reales en Economía.
\part*{\center Primera Parte}
\begin{multicols}{1}
\section*{De la riqueza}
La tierra es la fuente o materia de donde se extrae la riqueza, y el trabajo del hombre es la forma de producirla.
\section*{De las sociedades humanas}
Sea cualquiera la manera de formarse una sociedad humana, la propiedad de las tierras donde se asienta pertenecerá necesariamente a un pequeño número de personas.\\
cuando un príncipe, a la cabeza de un ejército, ha conquistado un país, distribuye las tierras entre sus oficiales o favoritos, de acuerdo con los méritos respectivos o siguiendo un arbitrario designio (en este caso se halló originariamente Francia) ; establece leyes para asegurar la propiedad de esas tierras para ellos o sus descendientes ; o bien se reserva la propiedad de las tierras, empleando a sus oficiales o favoritos en el empeño de hacerlas producir; o las cede a condición de que le paguen sobre ellas todos los años un cierto censo o canon; o las entrega reservándose la libertad de gravarlas todos los años, según sus necesidades propias y la capacidad de sus vasallos.\\
En esta economía es preciso que los colonos y labradores encuentren su sustento; tal cosa es absolutamente indispensable ya se exploten las tierras por cuenta del propietario mismo o por la del colono. El excedente del producto de la tierra queda a disposición del propietario; éste transfiere, a su vez, una parte al príncipe o al Gobierno, o bien el colono entrega dicha porción directamente al príncipe, deduciéndola del canon del propietario.
\section*{De los Pueblos}
Cualquiera que sea el empleo que se haga de la tierra pastos, cereales, varias los colonos o agricultores que trabajan en ellas deben residir en sus cercanías; de otro modo el tiempo necesario para ir a sus campos y retornar a sus casas consumiría una porción muy importante de la jornada.
\section*{De los burgos}
Existen pueblos donde se han establecido mercados, en inter de algún propietario o señor cortesano. Estos mercados, Tienen se celebran una o dos veces por semana, animan a muchos pequeños artesanos y mercaderes a establecerse en el lugar.\\
Los precios van fijándose en el mercado conforme a la proporción de los artículos que se ofrecen en venta y del dinero dispuesto a comprarlos ; todo ello ocurre en el mismo lugar, a la vista de todos los aldeanos de diversos poblados y de los mercaderes o empresarios del burgo. Una vez determinado el precio entre algunos, los otros lo siguen sin dificultad, estableciéndose así el precio del mercado para aquel día.\\
Cuando los pueblos de la circunscripción de un burgo, cuyos habitantes llevan ordinariamente sus artículos al respectivo mercado, sean importantes y dispongan de abundantes productos, el burgo adquirirá también importancia y grandeza proporcionales; pero cuando los pueblos circundantes cuenten con escasos productos, el burgo será también pobre y miserable.
\section*{De las ciudades}
Crecerá más la ciudad si el Rey o el Gobierno establece en ella tribunales de justicia, ante los cuales eleven sus recursos los habitantes de los burgos y aldeas de la provincia.
\section*{De las ciudades capitales}
El Rey o el Gobierno supremo la convierten en residencia suya, y en ella gastan las rentas del Estado; allí se emplazan en última instancia los Tribunales de Justicia; ese es el centro de las modas, y todas las provincias lo toman por modelo.
\section*{El trabajo de un labrador vale menos que el de un artesano}
Así pues quienes emplean artesanos o gente de oficio, necesariamente deben pagar por su trabajo un precio más elevado que el de un labrador u obrero manual ; y este trabajo será necesariamente caro, en proporción al tiempo que se pierda en aprenderlo, al gasto y al riesgo precisos para perfeccionarse en 61.
\section*{Los artesanos ganan, unos más, otros menos, según los distintos casos y circunstancias}
Los oficios que reclaman más tiempo para perfeccionarse en ellos, o más habilidad y esfuerzo, deben ser, naturalmente, los mejor pagados.
\section*{El número de labradores, artesanos y otros, que trabajan en un Estado, guarda relación, naturalmente, con la necesidad que de ellos se tiene}
Ocurre a menudo que los labradores y artesanos no tienen ocupación suficiente cuando existen en número excesivo para repartirse el trabajo. También sucede que se ven privados de su habitual ocupación por accidentes o por una variación en el consumo; puede acontecer también que el trabajo abunde y aun sea excesivo, según los casos y circunstancias. Sea como quiera, cuando carecen de trabajo abandonan los pueblos, burgos o ciudades donde residen, en número tal que los que permanezcan en el poblado guarden constantemente proporción con el empleo suficiente para permitirles subsistir; y cuando sobreviene un aumento constante de trabajo, hay algo que ganar, y otros afluyen para compartir la tarea.
\section*{El precio y el valor intrínseco de una cosa en general es la medida de la tierra y del trabajo que interviene en su producción}
El precio o valor intrínseco de una cosa es la medida de la cantidad de tierra y de trabajo que intervienen en su producción, teniendo en cuenta la fertilidad o producto de la tierra, y la calidad del trabajo. Pero ocurre a menudo que muchas cosas, actualmente dotadas de un cierto valor intrínseco, no se venden en el mercado conforme a ese valor : ello depende del humor y la fantasía de los hombres y del consumo que de tales productos se hace.\\
Si los campesinos de un Estado siembran más trigo que de ordinario, es decir mucho más del que hace falta para el consumo del año, el valor intrínseco y real del trigo corresponderá a la tierra y al trabajo que intervinieron en su producción ; pero a causa de esta excesiva abundancia, y existiendo más vendedores que compradores, el precio del trigo en el mercado descenderá necesariamente por debajo del precio o valor intrínseco. Si, a la inversa, los agricultores siembran menos trigo del necesario para el consumo, habrá más compradores que vendedores, y el precio del trigo en el mercado se elevará por encima de su valor intrínseco.
\section*{De la paridad o relación entre el valor de la tierra y el valor del trabajo}
No parece que la Providencia haya dado el derecho de posesión de las tierras a un hombre, con preferencia a otro. Los títulos más antiguos están fundados en la violencia y la conquista.\\
La pérdida de un artesano sera más onerosa que la de un trabajador, lo cual obliga a tener más cuidado de aquéllos, atendiendo a lo que cuesta siempre que alguien aprenda un oficio, para reemplazarlos.\\
Los trabajadores o artesanos, cuando disponen libremente de su doble porción, si son casados emplearán una parte para su propio sustento, y la otra para el de sus hijos. Si son solteros, dejarán de lado una pequeña parte.\\
Por ejemplo el trabajador casado se contentará viviendo a base de pan, queso, legumbres, etc.; raras veces comerá carne; beberá poco vino o cerveza, no dispondrá sino de vestidos viejos y de mala calidad, que usará el mayor tiempo posible: el remanente de su doble porción lo destinará a la crianza y sustento de sus hijos ; en cambio, el trabajador soltero comerá carne siempre que pueda, se procurará trajes nuevos, y por consiguiente empleará su doble porción para el propio sustento, con lo cual consumirá., en su persona, doble cantidad de productos de la tierra que el trabajador casado. Se infiere que a la Naturaleza le es indiferente que las tierras produzcan hierba, bosques o cereales, y que en ellas pueda nutrirse un número grande o pequeño de vegetales, animales u hombres.\\
Como los granjeros y maestros artesanos en Europa son todos empresarios y trabajan a su propio riesgo, unos se enriquecen y ganan más que el doble de su subsistencia, otros se arruinan y quiebran.\\
El dinero o la moneda, que encuentra en el cambio las proporciones de valor, es la medida más certera para juzgar de la paridad entre la tierra y el trabajo, y de la relación que uno y otro tienen en los diferentes países, variando dicha paridad según la mayor o menor cantidad de producto de la tierra que se atribuye a los que la trabajan. Por ejemplo, si un hombre gana una onza de plata, diariamente, con su trabajo, y otro no gana más que media onza en el mismo lugar, se puede concluir que el primero tiene disponible el doble de producto de la tierra que el segundo.
\section*{Todas las clases y todos los hombres de un Estado subsisten o se enriquecen a costa de los propietarios de tierras}
Sólo el príncipe y los propietarios de las tierras viven con independencia; todas las demás clases y todos los habitantes están contratados o son empresarios. Si el príncipe y los propietarios de las tierras cercaran sus haciendas, y no quisieran dejar trabajar a nadie en ellas, es evidente que no habría alimento ni vestido para ninguno de los habitantes del Estado: por consiguiente no sólo todos los habitantes del Estado subsisten a base del producto de la tierra que por cuenta de los propietarios se cultiva, sino también a expensas de los mismos propietarios de las fincas de las cuales derivan todos sus haberes.\\
El propietario recibe ordinariamente el tercio del producto de su tierra, y a base de este tercio no solamente procura sustento a todos los artesanos y otras personas a las que da empleo en la ciudad, sino también a los carreteros que llevan los productos del campo a las ciudades.\\
La tierra pertenece a los propietarios, pero sería inútil para ellos si no se cultivase. Cuanto más se la trabaje, en igualdad de circunstancias, mayor será la cuantía de sus productos ; y cuanto más se elaboran estos productos, siendo iguales todas las cosas, mayor valor poseerán como mercancías.
En cuanto a los que ejercen profesiones que no son necesarias, como los bailarines, comediantes, pintores, músicos, etc., sólo se les mantiene en el Estado para diversión u ornato, y su número es siempre muy reducido, en comparación con el resto de los habitantes.
\section*{La circulación y el trueque de bienes y mercaderías, lo mismo que su producción, se realiza en Europa por empresarios a riesgo suyo}
El colono es un empresario que promete pagar al propietario, por su granja o su tierra, una suma fija de dinero (ordinariamente se la supone equivalente, en valor, al tercio del producto de la tierra) sin tener la certeza del beneficio que obtendrá de esta empresa. Emplea parte de la tierra en criar ganados, en producir cereales, vino, heno, etc., a su buen juicio, sin posibilidad de prever cuál de estos artículos le permitirá obtener el mejor precio. El precio de estos productos dependerá, en parte, del tiempo, y, en parte, del consumo; si hay abundancia de trigo en relación con el consumo, el precio se envilecerá; si hay escasez el precio será más caro. o cual significa que conduce la empresa de su granja con incertidumbre.\\
El empresario o comerciante que acarrea los productos del campo a la ciudad no puede permanecer en ella para venderlos al menudeo, esperando que sean solicitados para el consumo: ninguna de las familias de la ciudad soportará por sí misma la compra inmediata de los productos necesarios para una temporada, ya que cada familia puede aumentar o disminuir su cifra, y el volumen de consumo, o, por lo menos, escoger a su gusto el tipo de mercaderías a consumir. En las familias apenas si se hace provisiones copiosa de otro artículo que del vino. Sea como fuere, la mayoría de los ciudadanos viven al día, y, sin embargo, son los que representan la mayor parte del consumo, pero no pueden hacer provisión alguna de productos del campo. Por esta razón muchas gentes en la ciudad se convierten en comerciantes o empresarios, comprando los productos del campo a quienes los traen a ella, o bien trayendolos por su cuenta: pagan así, por ellos un precio cierto, según el del lugar donde los compran, revendiéndolos al por mayor, o al menudeo, a un precio incierto.\\
En un Estado va siendo su número proporcionado a su clientela, o al consumo que ésta hace. Si existen sombrereros en exceso en una ciudad o en una calle, para el número de personas que en ella compran sombreros, algunos de los menos acreditados ante la clientela caerán en bancarrota; si el número es escaso, otros sombrereros considerarán ventajosa la empresa de abrir una tienda, y así es como los empresarios de todo género se ajustan y proporcionan automáticamente a los riesgos, en un Estado.\\
Por todas estas inducciones y por otras muchas que dirían hacerse acerca de un tema cuyo objeto son todos los habitantes de un Estado, cabe afirmar que si se exceptúan el príncipe y los terratenientes, todos los habitantes de un Estado son dependientes; que pueden, éstos, dividirse en dos clases:
\begin{itemize}
\item Empresarios.
\item Gente asalariada.
\end{itemize}
Que los empresarios viven, por decirlo así, de ingresos inciertos, y todos los demás cuentan con ingresos ciertos durante el tiempo que de ellos gozan, aunque sus funciones y su rango sean muy desiguales. En resumen, todos los habitantes de un Estado derivan su sustento y sus ventajas del fondo de los propietarios de tierras, y son dependientes.\\
Es cierto, sin embargo, que si algún habitante percibe altos emolumentos, o un empresario poderoso ha ahorrado capital o riqueza, es decir, si tiene almacenes de trigo, lana, cobre, oro o plata, o de alguna otra mercadería o artículo de uso o consumo constante en un Estado, y posee un valor intrínseco real, podrá considerársela, con razón, como independiente, por la cuantía de ese caudal los productos y mercaderías, incluso el oro y la plata, se hallan mucho más sujetos a accidentes y pérdidas que la propiedad de las tierras.\\
Estableceré, pues, el principio de que los propietarios de tierras son los únicos individuos naturalmente independientes en un Estado; que todas las clases restantes son dependientes, ya sean empresarios o asalariados, y que todo el trueque y la circulación del Estado se realiza por mediación de estos empresarios.
\section*{Las fantasias, modos y maneras de vivir del príncipe, y en particular de los propietarios de las tierras, determinan los usos a que esas tierras se destinan en un Estado, y causan, en el mercado, las variaciones de los precios de todas las cosas}
Si el propietario de un latifundio (y quiero proceder en mi argumentación como si no hubiera ningún otro en el mundo) lo cultiva por sí mismo, procederá a su arbitrio en cuanto a la utilización de las tierras.\\
\begin{enumerate}
\item Destinará necesariamente una parte al cultivo de cereales, para el mantenimiento de todos los agricultores, artesanos y mayordomos que trabajan para él; otra parte se aplicará a alimentar los bueyes, carneros y otros animales necesarios para su vestido y alimento, o para otras comoodidades, según sus gustos.
\item Dedicará una porción de sus tierras a parques, jardines y árboles frutales, o a vifiedos, según su inclinación, y a praderas para procurar pasto a los caballos, de los cuales se sirva para su recreo, etc.
\end{enumerate}
Si un señor o un propietario, que ha dado todas sus tierras en arriendo, tiene el capricho de cambiar su régimen de vida; si, por ejemplo, disminuye el número de sus criados y aumenta el de sus caballos, sus criados no sólo se verán obligados a abandonar la hacienda de este señor, sino que también habrán de hacerlo, en proporción, los artesanos y labradores antes ocupados en procurarles su sustento : la porción de tierra que se empleaba en mantenerlos será utilizada en mayor escala como praderas para los caballos, y si todos los propietarios de un Estado procediesen del mismo modo, pronto se multiplicaría el número de caballos y disminuiría el de los habitantes. Cuando un propietario ha despedido un gran número de criados y aumentado el número de sus caballos, habrá demasiado trigo para el consumo de los habitantes, y, por consiguiente, el trigo se venderá a bajo precio; en cambio, el heno será caro.\\
Si todos los terratenientes, en un Estado, administraran por sí mismos las tierras, las emplearían en producir lo que les agradara; y como las variaciones del consumo están principalmente motivadas por su régimen de vida, los precios que ofrecen en el mercado deciden a los colonos a todas las variaciones introducidas en el empleo y uso de las tierras.
\section*{La multiplicación y el descenso en el número de habitantes de un Estado dependen principalmente de la voluntad, de los modos y maneras de vivir de los terratenientes}
En una palabra, podríamos multiplicar todo género de animales, hasta la cifra deseada, y aun al infinito, si se dispusiera, hasta el infinito también, de tierras adecuadas para nutrirlos. La multiplicación de los animales no tiene otros límites que los medios más o menos abundantes que se destinan a alimentarlos. Indudablemente si todas las tierras se destina al mero sustento del hombre, la especie humana se multiplicaría hasta la cifra que esas tierras podrían sustentar.\\
No hay país donde la población se multiplique tan copiosamente como en China. Las gentes pobres viven, allí, únicamente de arroz y agua de arroz; trabajan casi desnudas, y en las provincias meridionales levantan tres abundantes cosechas de arroz, cada año, gracias al gran desvelo de sus habitantes por la agricultura. La tierra no descansa jamás y da, cada vez, más de ciento por uno; quienes cubren su cuerpo con vestidos, los llevan en su mayor parte de algodón, planta que exige tan poca tierra para crecer, que un acre posiblemente puede producir la cantidad de algodón suficiente para vestir cinco personas adultas.\\
Todos se casan, pues así lo manda su religión, y crían tantos hijos como pueden alimentar. Consideran como un crimen el empleo de las tierras para parques o jardines de placer, como si de este modo se arrebatara a los hombres la posibilidad de su sustento. Llevan a los viajeros en sillas de manos, y ahorran el trabajo de los caballos en todo cuanto puede atenderse mediante el esfuerzo humano. Su número es increíble, según las relaciones de viaje ; sin embargo, están obligados a hacer morir a muchos de sus hijos en la misma cuna, cuando no ven el modo de criarlos, conservando sólo el número de los que pueden alimentar. Mediante un trabajo rudo y obstinado extraen de los ríos una extraordinaria cantidad de pescado, y de la tierra todo cuanto se puede obtener de ella. Sin embargo cuando llegan años estériles mueren de hambre por millares, a pesar de los desvelos del Emperador, que almacena arroz en grandes cantidades para trances semejantes. Aun siendo, como son, numerosos los habitantes de la China, necesariamente guardan proporción con los medios de subsistencia, y no rebasan la cifra de los que el país puede sustentar, según el género de vida que les es propio ; y sobre este pie, un solo acre de tierra basta para alimentar a varios de ellos.\\
De otro lado no hay país donde la multiplicación de las gentes sea más limitada que entre los salvajes del interior de América. Menosprecian la agricultura, viven en los bosques y hallan su sustento en la caza de animales allí comunes. Como los árboles consumen el jugo y substancia de la tierra, hay poca hierba para alimentar a esos animales ; y como cada indio consume varios al año, de cincuenta a cien acres, no dan alimento bastante para un solo indio. Uno de estos pequeños poblados de indios suele disponer de unas cuarenta leguas cuadradas como coto de caza. Entre ellos se riñen guerras crueles y constantes por cuestión de límites, y el número de los habitantes se proporciona a los medios que encuentran de subsistir a base de la caza.\\
Los habitantes de Europa cultivan la tierra y producen cereales para su subsistencia. La lana de sus carneros les permite vestirse. El trigo es el grano de que se alimenta la mayor parte de sus gentes, aunque muchos aldeanos hacen su pan de centeno, y en el Norte, de cebada y de avena. La cantidad de alimento de los aldeanos y del resto del pueblo no es la misma en todos los lugares de Europa, pues las tierras son a menudo diferentes en cuanto a excelencia y fertilidad.\\
Mediante esos datos comprobaremos que un hombre que vive con pan, ajo y tubérculos, que va vestido de cáriamo, usa ropa interior muy burda, se calza con zuecos y no bebe más que agua, como es el caso de muchos aldeanos en las regiones meridionales de Francia, puede subsistir a base del producto de un acre y medio de tierra de calidad mediana, que rinde seis veces la semilla y descansa una vez cada tres años. De otro lado, un hombre adulto, calzado con zapatos de cuero y medias, que lleva vestidos de lana, vive en una casa y muda su ropa interior, posee un lecho, sillas, una mesa y otras cosas necesarias, que bebe moderadamente cerveza o vino y come todos los días carne, manteca, queso, pan, legumbres, etc., todo ello en cantidad suficiente pero moderada, puede procurarse todo esto con el producto de cuatro o cinco acres de tierra de mediana calidad. Es cierto que en estos cálculos no se reserva ninguna tierra para el mantenimiento de las caballerías, sólo se trata de las necesarias para labrar la tierra y para el transporte de los productos alimenticios a diez millas de distancia.\\
Si los propietarios de tierra tuviesen en cuenta el aumento de población y se estimulara a los aldeanos a casarse jóvenes, y a tener hijos, con la promesa de proveer a su subsistencia, destinando las tierras solamente a esto, sin duda se multiplicarían hasta el número que las tierras pudiesen soportar, de acuerdo con los productos de las parcelas necesarias a la subsistencia de cada uno, ya sea un acre y medic), o cuatro a cinco acres por persona.\\
Si los propietarios de las tierras y los señores de Polonia se avinieran a consumir en un principio manufacturas de su propio Estado, por deficientes que fueran, poco a poco harían mejorar su calidad, y ocuparían en su producción un mayor número de sus conciudadanos, en lugar de dar esta ventaja a los extranjeros: y si todos los Estados mostraran un parecido empeño en no dejarse engañar por los demás en el comercio, cada Estado adquiriría importancia en proporción a sus productos y a la laboriosidad de sus habitantes.\\ 
Cuando he dicho que los propietarios de tierras podrían multiplicar los habitantes en proporción al número de los que dichas tierras pueden mantener, supongo que la mayor parte de los hombres no desean cosa mejor que casarse, si pueden hallarse en condiciones de mantener sus familias, con el régimen de vida que ellos mismos disfrutan, es decir que si un hombre se contenta con el producto de un acre y medio de tierra, contraerá matrimonio siempre que esté seguro de tener lo bastante para mantener a su familia del mismo modo; pero si aspira a vivir del producto de cinco a diez acres, no se apresurará a casarse, a menos que no piense sostener a su familia en un nivel más bajo. Los hijos de la nobleza, en Europa, se educan en la abundancia, y como se da ordinariamente la mayor parte del patrimonio a los primogénitos, los segundones no tienen prisa por casarse; en su mayoría permanecen solteros, ya sea en el ejército o en los claustros, pero raramente se encontrará quienes no estén dispuestos a casarse, si les ofrecen herederas y fortunas, es decir, el medio de mantener una familia en el pie de vida que han conocido, y sin el cual correrían el peligro de hacer a sus hijos desgraciados.\\
Cuando los labradores y artesanos no se casan, es porque esperan ahorrar lo suficiente para ponerse en situación de constituir una familia, o de encontrar alguna muchacha que lleve a la misma una pequeña dote; y proceden así porque ven a diario muchos otros de su clase que, por no tomar las precauciones más elementales, forman un hogar y caen en la más espantosa miseria, viéndose obligados a privarse de su propio sustento para alimentar a sus hijos.\\
Me parece así bastante claro que el número de habitantes de un Estado dependa de los medios a ellos asignados para su sustento ; y como los medios de subsistencia dependen del método de cultivar la tierra, y el uso de ésta depende, a su vez, de la voluntad, del gusto y del género de vida de los propietarios de la misma, es evidente que de ellos depende la multiplicación o decrecimiento de la población de los países.
La multiplicación del número de habitantes, o incremento de la población, puede acelerarse sobre todo en los países cuyos habitantes se contentan con vivir más pobremente y gastar el mínimo del producto de la tierra; pero en los países en que todos los aldeanos y labriegos tienen por costumbre comer a menudo carne, o beber vino o cerveza, no es posible que se dé sustento a tantos habitantes.\\
Cuando las guerras han aniquilado o disminuido a los habitantes de un país, los salvajes, y las naciones civilizadas pronto las repueblan en los días de paz, sobre todo cuando el príncipe o los propietarios de las tierras procuran el necesario estímulo.\\
Pero todas estas ventajas son refinamientos y casos accidentales a los cuales no aludo aquí más que de pasada. El procedimiento natural y constante de aumentar el número de habitantes de un Estado es darles empleo en él y hacer que las tierras produzcan lo necesario para sostenerlos. Es también un problema al margen de mi investigación saber si vale más tener una gran cantidad de habitantes pobres y mal alimentados que un número más pequeño pero mejor atendido. Un millón de habitantes que consumen el producto de seis acres por cabeza, o cuatro millones que viven del de un acre y medio.
\section*{Cuanto más trabajo hay en un Estado tanto más rico se considera, naturalmente}
Cabe presumir que una tercera parte de los habitantes de un Estado son demasiado jóvenes o demasiado viejos para el trabajo cotidiano, y una sexta parte está compuesta de propietarios de tierras, enfermos y diferentes clases de empresarios que no contribuyen con su trabajo a las diferentes necesidades de las empresas. Todo esto implica que una mitad de los habitantes no trabajan o, por lo menos, no desarrollan actividad alguna en el aspecto de que estamos tratando. Así que si veinticinco personas hacen todo el trabajo necesario para sustentar a otras cien, existirán veinticinco personas de las cien, que se hallan en condiciones de trabajar, pero que no hacen nada.\\
Si por costumbre se atrae oro y plata del extranjero mediante la exportación de artículos y productos del Estado, como trigo, vinos, lanas, etc., ello permitirá enriquecer al Estado a expensas de la disminución del número de habitantes; pero si el oro y la plata se obtienen del extranjero a cambio del trabajo de los habitantes, así como de las manufacturas y artículos donde interviene pequeña cantidad de productos de la propia tierra, esto engrandecerá al Estado en forma útil y sustancial.\\
A fin de que el consumo de manufacturas de un Estado llegue a adquirir importancia en el extranjero, es preciso hacerlas buenas y estimables mediante un gran consumo en el interior del propio Estado ; hace falta también desacreditar en el propio país las mercaderías extranjeras, y dar mucho trabajo a los conciudadanos.\\
Un Estado no se considera rico por las mil futesas que afectan a la elegancia de las damas y de los hombres, que sirven para juegos y diversiones, sino por las mercaderías que son útiles y cómodas.\\
La experiencia permite observar que los Estados que abrazaron el protestantismo y no tienen ni monjes ni mendigos, se han convertido visiblemente en los más poderosos. Disfrutan también de la ventaja de haber suprimido un gran número de fiestas en las que el trabajo se interrumpe, en los países católicos, romanos, donde la laboriosidad de los habitantes sufre sustanciales interrupciones.
\section*{De los metales y de las minas y particularmente del oro y de la plata}
El trabajo de las minas de plata es caro por razón de la mortalidad que causa, ya que los obreros apenas si resisten cinco o seis años en este trabajo.\\
El valor de los metales en el mercado, lo mismo que el de todas las mercaderías o artículos, unas veces está por encima y otras por debajo del valor intrínseco, y varía en proporción a su abundancia o escasez, según el consumo que de ellos se hace.\\
El hierro no sólo es útil para los usos de la vida común; podría decirse que, en cierto modo, es necesario, y si los americanos, que no se servían de a antes del descubrimiento de su Continente hubiesen descubierto las minas y conocido las aplicaciones de este metal, sin duda hubiesen trabajado en la producción del mismo, por costosa que hubiera sido. ya en los historiadores más antiguos encontramos que desde tiempo inmemorial los pueblos se servían del oro y de la plata para fines monetarios, en Egipto y en Asia, y el Génesis nos dice que ya en tiempos de Abraham se acuñaban monedas de plata.
Como este metal iba siendo cada vez más estimado en la Mesopotamia puesto que los grandes propietarios compraban grandes copas de plata, y las clases subalternas, según sus recursos y ahorros, podían comprar pequeños cubiletes de ese metal el empresario de la mina, viendo que su producto tenía una salida constante, procedió a asignarle un valor, proporcional a su calidad o a su peso, en relación con todas las demás mercaderías o artículos que recibía en cambio.\\
Supongamos ahora que más allá del río Tigris, y, por consiguiente, fuera de Mesopotamia, se descubriese una mina de plata, cuyas vetas resultaran ser incomparablemente más ricas y abundantes que las del Monte Niphates, y que el trabajo de esa nueva mina, fácil de drenar, resultara menor que el de la primera. Es natural creer que el empresario de esa nueva mina se encontraría en disposición de suministrar plata a precio más bajo que la del Monte Niphates; y que los habitantes de Mesopotamia, deseosos de poseer piezas y objetos de plata, encontrarían más conveniente para ellos transportar sus mercaderías fuera del país, y cederlas al empresario de la nueva mina a cambio de ese metal, en vez de recurrir al antiguo empresario. Este, encontrando menos salida a su producción, forzosamente disminuiría sus precios; pero si el empresario nuevo bajase, en proporción, el suyo, el antiguo necesariamente habría de cesar en sus labores, y entonces el precio de la plata, como el de las demás mercancías y artículos, se regularía necesariamente a base del que estableciera la mina nueva. La plata costaría entonces menos a los habitantes de allende el Tigris que a los de Mesopotamia, puesto que éstos estaban obligados a incurrir en los gastos de un largo transporte de sus artículos y mercaderías, para adquirir la plata.
De ahí ha procedido la costumbre de regular el valor de las cosas, en proporción de su cantidad, es decir de su peso, con referencia a todos los demás artículos y mercaderías. Pero como la plata se puede alear con el hierro, el plomo, el estaño, el cobre, etc., que son metales menos raros y cuya extracción de las minas se efectúa con menor gasto, el trueque de la plata estuvo sujeto a frecuentes fraudes, y esto hizo que diversos reinos establecieran Casas de Moneda para certificar, mediante una acuñación pública, la verdadera cantidad de plata que cada moneda contenía, y entregar a los particulares que a dichas Casas llevaban barras o lingotes de plata, la misma cantidad de piezas, provistas de una impronta o certificado de la verdadera cantidad de plata que contenían.\\
El ensayo del oro se hace del mismo modo, con la única diferencia de que los grados de finura o pureza del oro se dividen en veinticuatro partes, a las que se llama quilates, porque el oro es más precioso. Estos quilates se dividen en treintaidosavos (mientras que los grados de finura de la plata se dividen en doce partes llamadas dineros, y estos dineros en veinticuatro granos cada uno).\\
Sin embargo, en este ensayo me he servido siempre del término $"$valor intrínseco$"$ con referencia a la cantidad de trabajo que entra en la producción de las cosas, porque no he encontrado término más apropiado para expresar mi pensamiento.\\
En las colonias de América se han utilizado como moneda el tabaco, el azúcar y el cacao, pero estas mercancías son demasiado voluminosas, perecederas y de calidad desigual ; por consiguiente son poco adecuadas para servir de moneda o de medida común del valor.\\
Dice Locke que el consentimiento de los hombres ha dado un valor al oro y a la plata.
\end{multicols}
\part*{\center Segunda Parte}
\begin{multicols}{1}
\section*{Del trueque}
En la primera parte hemos intentado probar que el valor de todas las cosas usadas por los hombres es proporcional a la cantidad de tierra empleada para producirlas y para el sustento de las gentes que las elaboran. mostraremos mediante una confrontación de los cambios que podrían hacerse: vino por tejidos, trigo por zapatos, sombreros, etc., y por la dificultad que causaría el transporte de estas diferentes mercancías o artículos alimenticios la imposibilidad de fijar su respectivo valor intrínseco, y la necesidad absoluta, para el hombre, de ballar sustancias de fácil transporte, no perecederas, susceptibles de tener, en su peso, una proporción o un valor igual a los diferentes artículos alimenticios y a las mercaderías, tan necesarias como convenientes. De ahí se ha derivado la elección del oro y la plata para el gran comercio, y del cobre para las pequeñas transacciones; estos metales no sólo son duraderos y de fácil transporte, sino que, además, requieren utilizar, para producirlos, una extensa superficie de tierra, circunstancia que les da el valor real deseable en el cambio.\\
si en un mercado hay dos veces más trigo del que en a se consume, y comparamos la cantidad total de trigo con la de plata, el trigo sería proporcionalmente más abundante que el dinero destinado a adquirirlo; sin embargo, el precio del mercado se sostendrá, como si sólo existiera la mitad de la cantidades trigo, porque la otra mitad puede y debe ser enviada a la ciudad, y los gastos de acarreo se incluirán en el precio de venta en la ciudad misma, que es siempre más alto si se compara con el de la aldea.
\section*{De los precios de los mercados}
El carnicero sostiene su precio según el número de compradores que se presentan; los compradores, por su parte, ofrecen un precio menor cuando creen que el carnicero tendrá menos ventas: el precio establecido por algunos es ordinariamente seguido por otros.\\
Unos son más hábiles para mantener un elevado precio por su mercancía; otros, para rebajarlo. Aunque este método de fijar los precios de las cosas en el mercado no tenga ningún fundamento justo o geométrico, ya que a menudo depende de la prisa o del temperamento expeditivo de un pequeño número de compradores o de vendedores.\\
Es evidente que la cantidad de artículos alimenticios o mercancías ofrecidas en venta, proporcionada a la demanda o al número de compradores, es la base sobre la cual se fija o se pretende fijar los precios actuales en los mercados, y en general estos precios no suelen alejarse mucho del valor intrínseco.\\
Los mercados distantes pueden influir siempre sobre el precio del mercado propio : si el trigo está muy caro en Francia, su precio se elevará en Inglaterra y en otros países vecinos.
\section*{De la circulación del dinero}
Las cosas necesarias para la subsistencia son los alimentos, el vestido y la habitación.
Si en las zonas campesinas se hacen telas bastas y burda ropa blanca, si se construyen casas, como habitualmente sucede, el trabajo necesario para todo ello puede pagarse por vía de trueque mediante evaluación, sin que sea necesario dinero en efectivo.\\
El dinero circula siempre en pago de servicios.\\
A base de lo antedicho se comprenderá que debe existir la proporción cuantitativa de dinero en efectivo necesaria para la circulación de un Estado, y que esta cantidad puede ser mayor o menor en los Estados, según el ritmo que se siga y la velocidad de los pagos.\\
Así, para estimar la cantidad de dinero circulante, hay que considerar siempre la velocidad de su circulación.\\
No hace falta incremento alguno en el dinero circulante en un Estado, respecto al comercio con el extranjero, cuando la balanza mercantil está equilibrada.
\section*{De la desigualdad de la circulación del dinero efectivo en un Estado}
Se puede decir así, que todos los distritos rurales y todas las ciudades de un Estado deben constante y anualmente un saldo, o deuda, a la capital. Ahora bien, como este saldo se paga en dinero, es evidente que las provincias deben sumas considerables a la capital, porque los productos y mercaderías que las provincias envían a la capital se venden en ella por dinero, y con él se paga la deuda o saldo en cuestión.\\
Convendría instalar manufacturas en las provincias alejadas de la capital, para aumentar su importancia y para determinar una circulación de dinero proporcionalmente menos desigual que la de la capital misma.\\
En la actualidad, si un Estado o un reino que suministra productos de sus manufacturas a los países extranjeros hace ese comercio de tal suerte que todos los afios obtiene del extranjero un saldo constante de dinero, la circulación sera en el propio país más rápida que en los de fuera, el dinero abundará más también, y en consecuencia la tierra y el trabajo se pagarán insensiblemente a más alto precio. Esto hará que en todas las ramas del comercio el Estado en cuestión cambie con el extranjero una cantidad menor de tierra y de trabajo por otra más grande, mientras duran estas circunstancias.
\section*{Del aumento y la disminución de la cantidad de dinero efectivo en un Estado}
l aumento de dinero provocará un aumento de los gastos, y esto último, a su vez, traerá consigo un aumento considerable de los precios del mercado en los años más favorables del cambio, y otro relativamente menor en los de nivel más bajo. Todo el mundo reconoce que la abundancia de dinero o su aumento en el cambio encarece el precio de todas las cosas. La cantidad de dinero que se ha traído de América a Europa durante los dos últimos siglos justifica esta verdad por la experiencia.
\section*{Continuación del mismo tema del aumento y de la disminución de la cantidad de dinero en un Estado}
Induzco que cuando se introduce doble cantidad de dinero en un Estado no siempre se duplica el precio de los productos y mercaderías. La proporción de carestía que el aumento y la cantidad de dinero introducen en un Estado dependerá del rumbo que este dinero imprima al consumo y a la circulación. Los precios de mercado se encarecerán más para ciertas especies que para otras, por abundante que sea el dinero.\\
El incremento de dinero no aumenta el precio de los productos y mercaderías sino por la diferencia de los gastos de transporte, cuando este transporte es viable.\\
Infiero que un aumento de dinero efectivo en un Estado provoca siempre, en él, un aumento de consumo y la costumbre de un más elevado nivel de gastos. Pero la carestía originada por ese incremento de dinero no se distribuye por igual entre todas las especies de productos y mercaderías, proporcionalmente a la cantidad de dinero incrementado, a menos que dicho incremento penetre por los mismos canales de circulación que el dinero primitivo. \\
Cuando en un Estado se introduce una respetable cantidad de dinero excedente, este dinero nuevo dé un nuevo giro al consumo, e incluso una nueva velocidad a la circulación.
\section*{Otra reflexión sobre el aumento y sobre la disminución de la cantidad de dinero efectivo en un Estado}
Parecería así que cuando un Estado se extiende mediante el comercio y la abundancia de dinero elevando el precio de la tierra y del trabajo, el príncipe o el poder legislativo deberán retirar dinero de la circulación, guardarlo para casos imprevistos. Para juzgar de la abundancia y de la rareza de dinero en circulación, no conozco mejor módulo que el de los alquileres y rentas de los propietarios de tierras. Cuando se arriendan tierras a elevado precio es serial de que el dinero abunda en el Estado ; pero cuando los propietarios se ven obligados a arrendarlas a un precio mucho más bajo, esto quiere decir que permaneciendo inalterables todos los demás factores el dinero escasea. \\
la abundancia del dinero hace que los precios se eleven a un nivel respetable, los habitantes se apresuran a trabajar para adquirirlo, si bien no tienen la misma urgencia por poseer ciertos artículos o mercaderías más allá de lo que es preciso para su sustento. \\
El último medio imaginable para aumentar la cantidad de dinero efectivo en la circulación de un Estado es el recurso a la violencia y a las armas, medio que a menudo se mezcla con los otros, puesto que en todos los tratados de paz por lo común se procura asegurar el derecho a comerciar y las ventajas inherentes a él. Una de las causas de la caída de roma fue por causa del lujo.\\
El lujo de los romanos que no se inició sino después de la derrota de Antíoco, rey de Asia, hacia el año 564 de la fundación de Roma se limitó al producto y al trabajo de sus vastos dominios, la circulación del dinero no hizo más que aumentar en vez de disminuir. El erario público estaba en posesión de todas las minas de oro, plata y cobre existentes en el Imperio. Poseía minas de oro en Asia, Macedonia, Aquilea, etc., y ricos yacimientos, tanto de oro como de plata, en Espafia y en otros muchos lugares. Tenían varias casas de Moneda, donde se realizaban acuñaciones de oro, plata y cobre.
\section*{Del interés del dinero y sus causas}
El interés del dinero en un Estado se determina por la proporción numérica entre prestamistas y prestatarios. En las clases más bajas el interés es siempre más alto, en proporción al mayor riesgo, y que disminuye de clase en clase, hasta la más elevada, que es la de los negociantes ricos, a quienes se reputa solventes.
\section*{De las causas del aumento y de la disminución del interés del dinero de un Estado}
El aumento de la cantidad de dinero efectivo en un Estado disminuye el precio del interés, porque cuando el dinero abunda es más fácil encontrar alguien que lo preste. Esta idea no siempre es verdadera ni justa. Si la abundancia de dinero en el Estado viene a través de las gentes que lo prestan, disminuirá, sin duda, el interés corriente, conforme aumenta el número de prestamistas; pero si llega por mediación de personas que lo gastan, tendrá el efecto inverso, y elevará el tipo de interés aumentando el número de empresarios que encontrarán trabajo como consecuencia de este aumento en los gastos, viéndose obligados a tomar dinero a préstamo, para equipar su industria, en todas clases de interés.\\
La causa más constante de elevación del tipo de interés en un Estado es el gasto cuantioso de los nobles
y propietarios de tierras, o de otras gentes ricas. Los empresarios y maestros artesanos se hallan en condiciones de proveer las grandes casas en todos sus renglones de gastos. Estos empresarios tienen casi siempre necesidad de tomar dinero a préstamo para regularizar sus suministros. Y cuando los nobles consumen sus rentas por anticipado y toman dinero a préstamo, contribuyen doblemente a elevar la tasa de interés.
\end{multicols}
\part*{\center Tercera Parte}
\begin{multicols}{1}
\section*{Del comercio con el extranjero}
Cuando un Estado cambia una pequeña cantidad de productos de la tierra contra otra cantidad mayor de
productos en sus tratos con el extranjero, seguramente lleva ventaja en este comercio ; y si por añadidura el dinero corriente abunda más en el propio Estado que en el exterior, cambiará siempre una cantidad menor de productos de la tierra por otra más grande.\\
Para los holandeses es preferible enriquecer a las Indias y no a sus propios vecinos, quienes podrían aprovecharse de esta coyuntura para oprimirlos. Además, venden a otros habitantes de Europa telas y baratijas de su propio país, a precio mucho más alto que el de las manufacturas vendidas a las Indias para su consumo en aquellas lejanas tierras. Errarían Inglaterra y Francia imitando en esto a los holandeses. Estos dos últimos reinos tienen en su propio país medios sobrados para procurar telas con que vestir a sus mujeres; y aunque resultan a precio más elevado que las manufacturas de las Indias, deben obligar a sus habitantes a no vestirse con tejidos extranjeros ; tampoco habrán de permitir la disminución de sus propios artículos y manufacturas, ni prestarse a caer en dependencia de los extranjeros, y mucho menos se avendrán a ceder dinero.
\section*{De los cambios y su naturaleza}
Cuando se pagan noventa y ocho libras en una localidad para recibir cien libras en otra, se dice que el cambio está a 2 \% por debajo de la par, poco más o menos: cuando se pagan ciento dos libras en una localidad, y no se reciben más que cien en otra, se dice que el cambio está a 2 To, exactamente, por encima de la par; cuando se dan cien libras en una localidad para recibir cien en la otra, se dice que el cambio está a la par.
\section*{Otras explicaciones para el conocimiento de la naturaleza de los cambios}
Si Inglaterra debe a Francia cien mil onzas de plata por el saldo comercial, si Francia debe cien mil onzas a Holanda, y Holanda cien mil onzas a Inglaterra, estas tres sumas podrán compensarse mediante letras de cambio entre los banqueros respectivos de los tres Estados, sin que sea necesario enviar dinero alguno por ningún lado.\\
Supongamos que Portugal consume anualmente y de modo constante cantidades considerables de manufacturas de lana y otros artículos de Inglaterra, tanto para sus propios habitantes como para los de Brasil; que de estas sumas paga una parte en vino, aceites, etc., pero que por el excedente del pago, existe un saldo comercial constante que precisa enviar de Lisboa a Londres. Si el rey de Portugal, bajo la pena, no solamente de confiscación, sino aun de perder la vida., prohíbe con todo rigor transportar metal de oro o de plata fuera de su territorio, el terror a estas prohibiciones impedirá por le pronto que los banqueros se
entremezclen en las remesas de esos saldos. El precio de las mercaderías inglesas quedará disponible en
efectivo en Lisboa. Los mercaderes ingleses, no pudiendo recibir sus fondos de Lisboa, las telas se encarecerán de un modo extraordinario; sin embargo, los tejidos no han subido de precio en Inglaterra, sino que los comerciantes se abstienen tan sólo de enviarlos a Lisboa puesto que no puede disponerse de su importe. Para tener telas inglesas, la nobleza portuguesa y otras personas, que no se avienen a prescindir de ellas, ofrecerán el doble del precio usual; pero como no podría obtenerse bastante cantidad sino enviando dinero fuera de Portugal. el aumento del precio constituirá el beneficio de quien, contraviniendo las prohibiciones, envíe el oro y la plata, fuera del reino. Este incentivo animará a muchos judíos y otras personas a trasladar oro y plata a los barcos ingleses surtos en la rada de Lisboa, aun con riesgo de la vida. Ganarán por lo pronto de cien a ciento cincuenta por ciento en esta operación, y el beneficio será pagado por los portugueses en el elevado precio que ofrecerán por las telas. Poco a poco se familiarizarán con estos manejos, después de haberlos practicado a menudo con éxito, y con el tiempo podrá situarse dinero a bordo de los arcos ingleses con un recargo de un dos o un uno por ciento.\\
El rey de Portugal hace la ley o la prohibición. Sus súbditos, incluso sus cortesanos, pagan los gastos del riesgo que se corre por soslayar y eludir la prohibición Semejante ley carece, por consiguiente, de eficacia; antes bien representa un efectivo perjuicio para Portugal, porque da lugar a que salga mucho más dinero del Estado del que saldría si semejante ley no existiese.\\
En efecto, los que se benefician con semejante maniobra siendo judíos o gentes de otro origen, no dejan de enviar sus beneficios a países extranjeros, y cuando ya han reunido cantidad suficiente o les invade el miedo, ellos mismos corren detrás de su dinero.\\
Si algunos de estos delincuentes fueran sorprendidos infraganti, confiscados sus bienes y aun condenados a perder la vida, esta circunstancia y esta ejecución, en lugar de impedir la salida de dinero, no haría sino aumentarla, porque los que antes se conformaban con una tasa de uno o dos por ciento en ese tipo de operaciones querrían tener veinte o cincuenta por ciento, con lo que siempre será necesaria una exportación de dinero en cantidad bastante para pagar el saldo.
\section*{De las variaciones en la proporción de valores, con respecto a los metales que sirven como moneda}
Los metales que más abundan y que menos cuesta
producir son, también, los más baratos. El hierro parece ser el más necesario, pero como su extracción se logra comúnmente en Europa con menos pena y trabajo que el cobre, su baratura es mayor. El valor del cobre, y el de todas las demás cosas, está proporcionado a la cantidad de tierra y de mano de obra que intervienen en su producción. Durante los cinco primeros siglos, en Roma no se utilizaba otra moneda. La plata sólo empezó a usarse en los cambios en el año 484. La proporción del cobre a la plata se fijó entonces, en las monedas, de 72 a 1 ; en la acuñación de 512, como de 80 a 1 ; en la de 537, como 64 a 1 ; en la de 586, de 48 a 1; en la de 663, de Druso, y en la de Sila, de 672, en 53 1 /3 a 1 ; en la de Marco Antonio, de 712, y en la de Augusto, de 724, de 56 a 1 ; en la de Nerón. del año 54 d. c., de 60 a 1 ; en la de Antonino, del año 160, de 64 a 1; en tiempo de Constantino, año 330 d. c., de 120 a 125 a 1 ; en el siglo de Justiniano, alrededor de 550, de 100 a 1 ; posteriormente ha ido variando por debajo de la proporción de 100 a 1 en las monedas de Europa. En el año 310 de la fundación de Roma precisaban en Grecia trece onzas de plata para pagar una onza de oro, es decir, que el oro estaba con respecto a la plata en la proporción de 1 a 13 ; el año 400 poco más o menos, como de 1 a 12 ; el año 460, como de 1 a 10, tanto en Grecia como en Italia, y en el resto de Europa. Esta proporción de 1 a 10 parece haber continuado constantemente durante tres siglos, hasta la muerte de Augusto, en el año 767 de la fundación de Roma, o sea el 14 de la Era Cristiana. En tiempo de Tiberio el oro se hizo más raro o la plata más abundante, habiendo subido poco a poco la proporción a la de 1 a 12, 12 1/2 y 13. Bajo Constantino, en el año de gracia 330, y bajo Justiniano, en el 550, fue de 1 a 142/ 5 . Luego la historia se hace más obscura; algunos creen que la proporción vino a ser de 1 a 18 en tiempo de ciertos reyes de Francia. En el año de gracia de 840, durante el reinado de Carlos el Calvo, se acuñaron monedas de oro y plata, y la proporción se estimó de 1 a 12. Bajo el reinado de San Luis, que murió en 1270, la proporción era de 1 a 10; en 1371, como de 1 a 12 ; en 1421, por encima de 1 a 11 ; en 1500, por debajo de 1 a 12; hacia 1600, como de 1 a 12; en 1641, como de 1 a 14; en 1700, como de 1 a 15 ; en 1730 como de 1 a 14 1/2.\\
La cantidad de oro y de plata que se había traído de México y del Perú durante el pasado siglo, no sólo ha hecho más abundantes estos metales sino que incluso ha elevado el valor del oro con respecto a la plata recibida en mayor cantidad, de manera que la proporción que se fija en las monedas de España, según los precios del mercado, es como de 1 a 16; los otros Estados de Europa han seguido bastante cerca los precios de España en sus monedas, estableciéndolos unos como de 1 a 15 7/8, otros como de 1 a 15 3/4, a 15 5/8, etc., según las ideas y opiniones de los directores de las Casas de Moneda.\\
En el Japón, donde existen minas de plata bastante ricas, la proporción del oro a la plata es, en la actualidad, como de 1 a 8; en la China, como de 1 a 10; en los otros países de aquende de las Indias, como de 1 a 11, de 1 a 12, de I. a 13 y de 1 a 14, a medida que uno se aproxima al Occidente y a Europa. Pero si las minas del Brasil continúan suministrando tanto oro, la proporción podrá bajar, a la larga, hasta situarse en la de 1 a 10, incluso en Europa, cosa que me parece la más natural si es que esta proporción ha de guiarse por cosa distinta del azar. Es evidente que durante la época en que todas las minas de oro y de plata, en Europa, en Asia y en África se explotaban por cuenta de la República Romana, la proporción de 1 a 10 era la más constante.\\
Solo el precio del mercado puede restituir la proporción de valor del oro a la plata, lo mismo que todas las proporciones de valores.\\
En términos generales el valor de la plata es más permanente y el del oro se halla más sujeto a variación.
\section*{Del aumento y de la disminución de valor de las especies amonedadas en denominación determinada}
Conforme a los principios que hemos establecido, las cantidades de dinero que circulan en los cambios fijan y determinan los precios de todas las cosas en un Estado, teniendo en cuenta la rapidez o la lentitud de la circulación.
\section*{De los Bancos y su crédito}
En estas circunstancias el banquero podrá prestar a menudo noventa mil onzas de plata (de las cien mil que debe) durante todo el año, y no tendrá necesidad de guardar en caja más de diez mil onzas, para hacer frente a los reintegros que puedan solicitarle. Sus negocios son con personas opulentas y económicas; a medida que le piden mil onzas por un lado, le llevan ordinariamente mil onzas, por otro. Basta pues, por lo común, mantener en efectivo la décima parte de sus depósitos. Ejemplos y experiencias de esta forma de operar se han podido reunir en Londres. Esto hace que en lugar de que los particulares guarden en sus áreas durante todo el año la mayor parte de las cien mil onzas, se acostumbren a depositarlas en manos de un banquero, y que noventa mil de esas cien mil onzas se pongan en circulación. Tal es, primordialmente, la idea que podemos formarnos de la utilidad de esta clase de Bancos ; los banqueros u orfebres contribuyen a acelerar la circulación del dinero. Lo prestan a interés, a su propio riesgo y peligro, y sin embargo siempre están o deben estar dispuestos a pagar los billetes a voluntad del depositante, y contra su presentación.\\
Si un particular tiene que pagar mil onzas a otro, le daría en pago un billete del banquero, por dicha suma. Posiblemente esta otra persona no irá a reclamar al banquero el pago respectivo ; guardará el billete y lo dará, en ocasión oportuna, en pago a un tercero, y así el billete en cuestión podrá pasar por muchas manos en los grandes pagos, sin que durante largo tiempo se piense en requerir su pago al banquero. Apenas si habrá alguno que, no teniendo una confianza completa o necesitando pagar sumas pequeñas, solicitar del reintegro. En este primer caso el dinero efectivo de un banquero no representa sino la décima parte de sus operaciones.\\
Si dispone de gran copia de depósitos y de un elevado crédito, verá aumentar la confianza que se tiene en sus billetes, y las gentes mostrarán menos prisa por reclamar el pago. Pero el pago sólo se difiere unos cuantos días o semanas cuando los billetes caen en manos de personas que no están acostumbradas a tratar con él, y debe guiarse siempre por las costumbres de quienes suelen confiarle su dinero. Si sus billetes caen en manos de gentes de su mismo oficio mostrarán éstas una gran prisa en retirarle el dinero.
\section*{De los refinamientos del crédito de los Bancos generales}
El Banco nacional de Londres está integrado por un gran número de accionistas que designan directores para la gerencia de las operaciones. Su primordial ventaja consistía en hacer una distribución anual de los beneficios obtenidos por vía de interés sobre el dinero prestado a base de los fondos depositados en el Banco; posteriormente se incorporó la Deuda pública, sobre la cual el Estado paga un interés anual.\\
Es pues indudable que un Banco, en complicidad con el ministro, es capaz de elevar y sostener el precio de los fondos públicos y de reducir la tasa de interés en el Estado, al arbitrio del ministro, cuando las operaciones se llevan a cabo con discreción, y de este modo se liberan las deudas del Estado. Pero estos refinamientos, que abren la puerta para realizar grandes fortunas, sólo en contados casos se aplican para la utilidad exclusiva del Estado, y los que participan en ellos se corrompen con frecuencia. Los billetes de Banco redundantes, fabricados y emitidos en estas ocasiones, no perjudican la circulación, porque aplicándose a la compra y venta de fondos de capital no sirven para el gasto de las familias, y por consiguiente no se cambian por plata. Pero si en virtud de algún temor o accidente imprevisto los tenedores de billetes solicitaran la plata del Banco, la bomba explotaría y se pondría de manifiesto que estas operaciones son por demás peligrosas.
\end{multicols}
\end{document}