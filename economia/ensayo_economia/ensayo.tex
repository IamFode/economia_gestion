\documentclass[10pt]{book}
\usepackage[text=17cm,left=2.5cm,right=2.5cm, headsep=20pt, top=2.5cm, bottom = 2cm,letterpaper,showframe = false]{geometry} %configuración página
\usepackage{latexsym,amsmath,amssymb,amsfonts} %(símbolos de la AMS).7
\parindent = 0cm  %sangria
\usepackage[T1]{fontenc} %acentos en español
\usepackage[spanish]{babel} %español capitulos y secciones
\usepackage{graphicx} %gráficos y figuras.

%---------------FORMATO de letra--------------------%

\usepackage{lmodern} % tipos de letras
\usepackage{titlesec} %formato de títulos
\usepackage[backref=page]{hyperref} %hipervinculos
\usepackage{multicol} %columnas
\usepackage{tcolorbox, empheq} %cajas
\usepackage{enumerate} %indice enumerado
\usepackage{marginnote}%notas en el margen
\tcbuselibrary{skins,breakable,listings,theorems}
\usepackage[Bjornstrup]{fncychap}%diseño de portada de capitulos
\usepackage[all]{xy}%flechas
\counterwithout{footnote}{chapter}
\usepackage{xcolor}

%--------------------GRÀFICOS--------------------------

\usepackage{tkz-fct}

%----------Formato título de capítulos-------------

\usepackage{titlesec}
\renewcommand{\thechapter}{\arabic{chapter}}
\titleformat{\chapter}[display]
{\titlerule[2pt]
\vspace{4ex}\bfseries\sffamily\huge}
{\filleft\Huge\thechapter}
{2ex}
{\filleft}

\usepackage[htt]{hyphenat}

\begin{document}

%------------------------------------------
 

\chapter*{¿Cuáles son las causas del subdesarrollo en Bolivia?}

\begin{center}
Por Christian L. Paredes Aguilera.
\end{center}

\vspace{1.5cm}

\begin{multicols}{2}

\section*{Introducción}
Si partimos de la idea de que Bolivia es un país subdesarrollado independientemente de que en él haya riqueza, entonces se puede sostener que el país adolece de condiciones estructurales que le impiden transformar el excedente en un sistema óptimo de producción, economía y rentabilidad. Por lo tanto, ¿que nos impide dicha transformación?.\\
Se podría atacar esta pregunta mencionando temas inherentes a extractivismo, corruptela, insuficiente productividad  u otros, pero debido a un mundo mas complejo y acelerado, tomare tres directrices que responderá a esta pregunta y por ende, de manera alternativa las causas del subdesarrollo en el país.\\
Estas directrices según Mario Weitz darán o están dando tres ventajas competitivas, escalables y complementarias entre si como son la educación, innovación y ciencia y tecnología. Acordemos que cada uno es directamente proporcional al anterior.

\section*{Causas}

\subsection*{Educación}
Tal vez peque de redundante al identificar una educación obsoleta, sabiendo de la información que se puede obtener hoy en día. Así que me enfocaré en temas de investigación, más específicamente en la investigación científica\\
La investigación científica como pilar y pieza fundamental de la educación que engrana de manera única con los otras directrices, se fundamenta en la compresión del mundo que nos rodea, por lo tanto nos dará una ventaja competitiva. Sabiendo esto, algunos estudios coinciden que el número de profesionales dedicados a la investigación y la producción científica boliviana es baja \cite{uno}. Consideramos importante destacar algunas limitantes de dicho fenómeno: El país en sus indicadores de desarrollo, como en los referidos al complejo de investigación, demuestran la persistencia de fallas estructurales en el diseño de políticas públicas; falta de redes internacionales de colaboración entre países latinoamericanos, los cuales comparten una problemática en común y la no existencia de un adecuado fortalecimiento de centros impulsores, generadores de información y conocimiento.\\
El país solo invierte un 0,08 \% de su PIB en investigación, valor que nos coloca muy por debajo del promedio de todos los países de América Latina y el Caribe (0,62 \%). Esta situación es similar a los valiosos y escasos recursos humanos, con solo 0,4 \% de investigadores por cada 1000 integrantes de la población económicamente activa \cite{dos}, por ello se tendría una fuga de capital humano o talento como también poco incentivo para el individuo y su compromiso para tal efecto. 

\subsection*{Innovación}
Gran parte de lo que en Bolivia se entiende por innovación tiene que ver con la compra o adquisición de bienes de capital. Esto concluye como irrelevante ya que se está hablando demasiado del aparato productivo que tenemos en Bolivia. Dicho de otra forma una parte significativa de gasto de innovación tiene que ver con procesos, procedimientos o ideas nuevas, lo que se ve muy poco en Bolivia. Vayamos a hechos mas concretos.\\
Las empresas innovan poco porque es muy difícil aprovechar los resultados. Por ende se tiene un problema de apropiabilidad, lo cual es muy serio porque está vinculado a una externalidad negativa, tal vez  falla de gobierno. Cabe recalcar que en Bolivia, los derechos de propiedad son muy difíciles de hacer cumplir.\\
En la mayoría de los países, la innovación se lleva vía subsidios estatales o vía programas de asociaciones público privadas que permiten que las empresas lleven a cabo las innovaciones con beneficios impositivos, subsidios directos de los impuestos y otros. En Bolivia es que no ha habido ningún incentivo desde el sector público al privado y esto retrasa el desarrollo de nuevos productos y sectores.\cite{tres}\\

\subsection*{Ciencia y Tecnología}
Partamos del hecho directo entre lo que un país invierte en ciencia y tecnología y el nivel promedio de riqueza; es decir mientras más produce una nación por el uso de tecnología (por ejemplo, para crear vacunas), mayores son los beneficios sociales, la salud de sus habitantes, su productividad y nivel de vida \cite{cinco}. Sabiendo esto estoy en capacidad de dar dos ejemplos concretos y relacionarlos con nuestra actualidad.\\
En los años 70, China tuvo una revolución cultural. Decidieron que los investigadores y académicos tenían que salir a las granjas y fábricas, ya que la ciencia era burguesa y los científicos la burocracia dorada. Después, en los años 90 se dieron cuenta de ese error, y le dieron la importancia que tiene a la ciencia y la tecnología, invirtieron en estos dos rubros, y hoy está en camino de ser potencia mundial en tecnología.\\
Un ejemplo más lo constituye Israel, un país que trataba de sobrevivir como podía, con guerras continuas, y ahora es una potencia. Hizo producir el desierto, inventaron el riego por goteo y tienen tecnología que incluso venden a otros países.\\
Entonces a pesar de un crecimiento lineal a partir del 2016 en las exportaciones de productos de alta tecnología \cite{seis}, y el crecimiento exponecial en materia de inversión tecnológica que se experimenta \cite{siete}, aún no podemos compararnos con los grandes tenedores de tecnología, como se cito anteriormente. Y por lo tanto aún no podemos adquirir una ventaja competitiva para liderar o por lo menos estar a la par de dichos tenedores.  


\section*{Conclusión}
Como resultado de los avances científicos en la informática, Bolivia debe enfrentar el reto que representa la presencia de las nuevas tecnologías en un nuevo ámbito en el que se debe buscar ser competitivos. Los ejecutivos de las instituciones, ya sean estas privadas o estatales, podrían comprender mejor el hecho de que el avanzado mundo de servicios es más eficiente cuando se asignan más recursos financieros y tecnológicos para incrementar el nivel de competividad, lo que significaría buscar nuevas alternativas para mejorar el nivel y la calidad de los servicios ofrecidos.\\
Por ello es necesario el diseño de una política clara para establecer una normativa legal para incentivar el uso de la tecnología, incentivar la investigación, desarrollo e innovación, tomando en cuenta las experiencias de otros países y adecuándolas a nuestro contexto. Se debe recalcar que resulta esencial instituir leyes para proteger y garantizar los datos, la privacidad, la soberanía nacional y la identidad cultural. Además se deberá elaborar una ley que regule las redes telemáticas para garantizar el cumplimiento de servicio al público.\\
En un mundo cada vez mas complejo, pero con mas oportunidades, se debe pensar en formar cultura de investigación e innovación.\\ 
Recordemos que una mente Boliviana es muy similar a una mente de cualquier potencia, no hay excusas. 


\end{multicols}

\begin{thebibliography}{10}


    \bibitem{uno} Eróstegui Revilla C, De Pardo Ghetti E, Baumann-Pinto GA, Suárez Barrientos EL (2011). Evaluación de la difusión de la producción científica en Bolivia. 

    \bibitem{dos} Camacho Salinas R, Villegas M, Mendizabal Ch (2015). Bolivia entre la realidad económica y la utopía académica.

    \bibitem{tres} Gabriela Espinoza (2019). Solamente adquirir bienes de capital es irrelevante. 

    \bibitem{cuatro} Javier Bellott (2019). No existen políticas públicas para motivar la innovación.

    \bibitem{cinco} Dr. Ricardo Arechavala Vargas (2020). Ciencia y tecnología, claves para salir del subdesarrollo.

    \bibitem{seis} Banco Mundial (2019). Exportaciones de productos de alta tecnología (US\$ a precios actuales)

    \bibitem{siete} COBIS (2020), Perspectivas en innovación digital financiera.



%----------sin numeración

    \bibitem{}  Mauricio Ríos García (2016). Las causas del subdesarrollo en Bolivia.

    \bibitem{}  Osvaldo Sunkel (1970), subdesarrollo latinoamericano y la teoría del desarrollo.

    \bibitem{}	Carlos Herrera (2011), Las causas del subdesarrollo. 

\end{thebibliography}


\end{document}
