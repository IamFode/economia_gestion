\documentclass[10pt]{book}
\usepackage[text=17cm,left=2.5cm,right=2.5cm, headsep=20pt, top=2.5cm, bottom = 2cm,letterpaper,showframe = false]{geometry} %configuración página
\usepackage{latexsym,amsmath,amssymb,amsfonts} %(símbolos de la AMS).7
\parindent = 0cm  %sangria
\usepackage[T1]{fontenc} %acentos en español
\usepackage[spanish]{babel} %español capitulos y secciones
\usepackage{graphicx} %gráficos y figuras.

%---------------FORMATO de letra--------------------%

\usepackage{lmodern} % tipos de letras
\usepackage{titlesec} %formato de títulos
\usepackage[backref=page]{hyperref} %hipervinculos
\usepackage{multicol} %columnas
\usepackage{tcolorbox, empheq} %cajas
\usepackage{enumerate} %indice enumerado
\usepackage{marginnote}%notas en el margen
\tcbuselibrary{skins,breakable,listings,theorems}
\usepackage[Bjornstrup]{fncychap}%diseño de portada de capitulos
\usepackage[all]{xy}%flechas
\counterwithout{footnote}{chapter}
\usepackage{xcolor}

%--------------------GRÀFICOS--------------------------

\usepackage{tkz-fct}

%----------Formato título de capítulos-------------

\usepackage{titlesec}
\renewcommand{\thechapter}{\arabic{chapter}}
\titleformat{\chapter}[display]
{\titlerule[2pt]
\vspace{4ex}\bfseries\sffamily\huge}
{\filleft\Huge\thechapter}
{2ex}
{\filleft}

\usepackage[htt]{hyphenat}

\begin{document}
\normalfont
\input xy
\xyoption{all}
\author{Ensayo por Christian Paredes Aguilera}
\title{¿Cuáles son las causas del subdesarrollo en Bolivia?}
\date{}
\pagestyle{empty}
\maketitle
\thispagestyle{empty}
\let\cleardoublepage\clearpage


%------------------------------------------
 
\let\cleardoublepage\clearpage


Fernando Molina define el subdesarrollo como una condición estructural de una sociedad que no puede prever el futuro de una una manera optimista por que en cualquier momento puede volver a caer. \\
Si partimos de la idea de que Bolivia es un país subdesarrollado independientemente de que en él haya riqueza, su condición de mediterraneidad que le impide comerciar abiertamente y directamente con el mundo, contrario a la tesis de Jeff Sachs $"$El fin de la pobreza$"$ donde se sigue planteando en el paradigma exclusivo de cosas y objetos materiales y no así  sobre los medios para alcanzar los fines que hayan elegido en función de sus juicios de valor. Entonces se puede sostener que el país adolece de condiciones estructurales que le impiden transformar el excedente en un sistema óptimo de producción, economía y rentabilidad, de donde, probablemente, el problema no proviene de crecimiento, ni riqueza. Esto lo demuestra décadas de subdesarrollo. \\
Por otro lado el sector privado y publico siempre ha exportado ahorro durante toda la historia del país, que ha generado un excedente en el país gracias a los recursos naturales, pero ese capital no se ha invertido en el país. Estas exportaciones se vienen dando hasta nuestros días con una cultura extractivista arraigada.

Un problema no menor que se tiene es el capital humano, 

Un grupo de países de la periferia de los centros coloniales en el sur y en Centroamérica, con economías del tipo S, continúa basando su actividad económica principalmente en una agricultura de subsistencia bastante amplia y en el desarrollo de algunas limitadas actividades de agricultura tropical de exportación. La mayor parte de estos países — Bolivia, Ecuador, Colombia y los centroamericanos— que no desarrollaron nuevas e importantes actividades productivas ni establecieron estrechas vinculaciones con una nueva metrópoli, también caen en un anárquico proceso bastante prolongado, similar al de Perú y México. En casi todos estos casos contribuyen a intensificar la anarquía guerras como la de Chile contra la Confederación PeruanoBoliviana ( 1836-1839), y las sucesivas agresiones externas que sufrió México
( 1846-1848, 1862). 

(subdesarrollo latinoamericano y la teoría del desarrollo OSVALDO SUNKEL)
(Mauricio Ríos García página web www.riosmauricio.com, las causas del subdesarrollo en Bolivia)


\end{document}
