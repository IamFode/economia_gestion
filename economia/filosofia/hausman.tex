\documentclass[10pt]{book}
\usepackage[text=17cm,left=2.5cm,right=2.5cm, headsep=20pt, top=2.5cm, bottom = 2cm,letterpaper,showframe = false]{geometry} %configuración página
\usepackage{latexsym,amsmath,amssymb,amsfonts} %(símbolos de la AMS).7
\parindent = 0cm  %sangria
\usepackage[T1]{fontenc} %acentos en español
\usepackage[spanish]{babel} %español capitulos y secciones
\usepackage{graphicx} %gráficos y figuras.

%---------------FORMATO de letra--------------------%

\usepackage{lmodern} % tipos de letras
\usepackage{titlesec} %formato de títulos
\usepackage[backref=page]{hyperref} %hipervinculos
\usepackage{multicol} %columnas
\usepackage{tcolorbox, empheq} %cajas
\usepackage{enumerate} %indice enumerado
\usepackage{marginnote}%notas en el margen
\tcbuselibrary{skins,breakable,listings,theorems}
\usepackage[Bjornstrup]{fncychap}%diseño de portada de capitulos
\usepackage[all]{xy}%flechas
\counterwithout{footnote}{chapter}
\usepackage{xcolor}
\usepackage[htt]{hyphenat}
%--------------------GRÀFICOS--------------------------

\usepackage{tkz-fct}

%---------------------------------

\titleformat*{\section}{\bfseries\rmfamily}
\titleformat*{\subsection}{\bfseries\rmfamily}
\titleformat*{\subsubsection}{\bfseries\rmfamily}
\titleformat*{\paragraph}{\bfseries\rmfamily}
\titleformat*{\subparagraph}{\bfseries\rmfamily}

%------------------------------------------

\renewcommand{\labelenumi}{\Roman{enumi}.}%primer piso II) enumerate
\renewcommand{\labelenumii}{\arabic{enumii}$)$}%segundo piso 2)
\renewcommand{\labelenumiii}{\alph{enumiii}$)$}%tercer piso a)
\renewcommand{\labelenumiv}{$\bullet$}%cuarto piso (punto)

%----------Formato título de capítulos-------------

\usepackage{titlesec}
\renewcommand{\thechapter}{\arabic{chapter}}
\titleformat{\chapter}[display]
{\titlerule[2pt]
\vspace{4ex}\bfseries\sffamily\huge}
{\filleft\Huge\thechapter}
{2ex}
{\filleft}

\begin{document}

\normalfont
\input xy
\xyoption{all}
\author{\Large Apuntes por FODE}
\title{\small Daniel M. Hausman \\ \vspace{1cm} \large La filosofía de la economía: Una ontología}
\date{}
\pagestyle{empty}
\maketitle
\thispagestyle{empty}
\let\cleardoublepage\clearpage
\tableofcontents								%indice





\chapter*{Introducción}

\section{Una introducción a la filosofía de la ciencia}
Los temas de los que se ha ocupado la filosofía de la ciencia que son más relevantes para la economía se pueden dividir en cinco grupos:

\begin{enumerate}[1.]
    \item \textit{Objetivos} ¿Cuáles son los objetivos de la ciencia y de la teoría científica? ¿Es la ciencia principalmente una actividad práctica que apunta a descubrir generalizaciones útiles, o debería la ciencia buscar explicaciones y verdades?.
    \item \textit{Explicación} ¿Qué es una explicación científica?.
    \item \textit{Teorías} ¿Qué son las teorías, modelos y leyes? ¿Cómo se relacionan entre sí? ¿Cómo se descubren o construyen?.
    \item \textit{Prueba, inducción y demarcación} ¿Cómo se prueban y confirman o refutan las teorías, modelos y leyes científicas? ¿Cuáles son las diferencias entre las actitudes y prácticas de los científicos y las de los miembros de otras disciplinas?.
    \item ¿Las respuestas a estas cuatro preguntas son las mismas para todas las ciencias en todo momento? ¿Se pueden estudiar las acciones e instituciones humanas de la misma manera que se estudia la naturaleza?.
\end{enumerate}

\subsection{Los objetivos de la ciencia}
Hay dos escuelas principales de pensamiento

\begin{enumerate}[1.]
    \item Los realistas científicos.- la ciencia debe descubrir verdades sobre el mundo y explicar los fenómenos. (Ven a las teorías como aproximaciones de la realidad).
    \item Los antirrealistas que puede ser instrumentalistas que consideran que los objetivos de la ciencia son exclusivamente prácticos, o los antirrealistas pueden estar en desacuerdo con los realistas principalmente sobre si existen los no observables postulados para las teorías científicas, si las afirmaciones sobre ellos son verdaderas o falsas y si la evidencia observable puede establecer afirmaciones sobre inobservables. (Ven a las teorías como herramientas)
\end{enumerate}

Los antirrealistas están de acuerdo que las teorías son importantes, pero ubican su importancia exclusivamente a su papel de ayudar a las personas a anticipar y controlar los fenómenos. Puede verse en "La metodología de la Economía positiva" de Milton Friedman.


\subsection{Explicación científica}
Las explicaciones responden a la pregunta:
\begin{center}
    ¿Por qué?
\end{center}

Carl Hempel, desarrolla dos modelos principales de explicación científica:

\begin{itemize}
    \item Deductivo-nomológico.- Un enunciado de lo que se va a explicar se deduce de un conjunto de enunciados verdaderos que incluye esencialmente al menos una ley.
    \item Inductivo-estadístico.- Este último, como sugiere su nombre, se ocupa de las explicaciones probabilísticas e intenta extender la intuición básica del modelo deductivo-nomológico (D-N).
\end{itemize}

El modelo D-N es una explicación determinista o no estadística. Si sólo se dispone de una regularidad estadística, no se podrá deducir lo que se quiere explicar, pero se podrá demostrar que es altamente probable, que es lo que exige el modelo inductivo-estadístico de Hempel. \\
Parece que las explicaciones de los acontecimientos y estados de cosas suelen citar sus causar, pero existe dos problemas.
\begin{enumerate}
    \item Primero, aunque la mayoría de las explicaciones de eventos y estados de cosas son explicaciones causales, no todas lo son. 
    \item En segundo lugar, decir que las explicaciones citan causas no es en sí mismo muy informativo. Sin una teoría de la causalidad, una teoría causal de la explicación está vacía, e incluso con una teoría de la causalidad, solo araña la superficie para sostener que explicar es citar una causa. 
\end{enumerate}

\subsection{Teorías científicas y leyes}
La mayoría de los filósofos han argumentado que la ciencia procede mediante el descubrimiento de teorías y leyes, pero los economistas se sienten más cómodos hablando de modelos que de leyes y teorías.\\

Las leyes de las ciencias no son, por supuesto, leyes prescriptivas que dictan cómo deberían ser las cosas. Las leyes científicas son, en cambio, (hablando en términos generales) regularidades en la naturaleza. Pero quizás el concepto de ley no sea útil para aquellos interesados en la metodología económica.\\

De acuerdo con el modelo de explicación deductivo-nomológico, los economistas pueden usar generalizaciones como la ley de la demanda para explicar fenómenos económicos solo si esas generalizaciones son genuinamente leyes.\\

Los positivistas lógicos precisaron la noción de “trabajar juntos”, argumentando que las teorías forman sistemas deductivos. Según los positivistas, las teorías son principalmente objetos “sintácticos”, cuyos términos y afirmaciones se interpretan por medio de reglas de “correspondencia”. Estos esperaban expresar teorías científicas en lenguajes formales, pero los positivistas se dieron cuenta de que la relación entre teoría y observación es más compleja.\\

Por tal razón se usan modelos, las cuales se manipulan, exploran y modifican, afirmando que los predicados (se afirma algo como: es un sistema de consumo de dos productos básicos) construyen o determinan que son verdaderos o falsos para los sistemas de cosas del mundo. Ahora bien, por qué los filósofos serios defienden la visión predicada de los modelos?.
Existen dos razones

\begin{itemize}
    \item En primer lugar, si uno espera ser capaz de reconstruir las afirmaciones de la ciencia formalmente, el punto de vista del predicado tiene importantes ventajas técnicas.
    \item En segundo lugar, el punto de vista del predicado ofrece una forma útil de esquematizar los dos tipos de logros que implica la construcción de una teoría científica. (Modelo de Hempel).
\end{itemize}

Una parte absolutamente crucial del quehacer científico es la construcción de nuevos conceptos, de nuevas formas de clasificar los fenómenos, que es común en los economistas.

\subsection{Evaluación y Demarcación}

La opinión kantiana de que hay verdades sintéticas a priori (que pueden ser demostrados puramente por razonamiento lógico y deductivo, sin necesidad de recurrir a la experiencia empírica) todavía tiene partidarios entre los llamados economistas austriacos, especialmente Ludwig von Mises y sus seguidores. Argumentan que los postulados fundamentales de la economía son verdades sintéticas a priori. 

Hume, lanza un desafío: muéstrame un buen argumento cuya conclusión sea alguna generalización o alguna afirmación sobre algo no observado y cuyas premisas incluyan solo informes de experiencias sensoriales. Tal argumento no puede ser un argumento deductivo, porque tales inferencias son falibles. Tampoco lo hará un argumento “inductivo”, ya que solo tenemos motivos inductivos y, por lo tanto, que plantean dudas para creer que tales argumentos son buenos.\\

El problema de la demarcación se refiere a la distinción entre teorías científicas y otros tipos de teorías.\\

De lo que debería ocuparse la filosofía de la ciencia, según Lakatos, no son las reglas para evaluar teorías, sino las reglas para modificar y comparar teorías.

\subsection{La unidad de la ciencia}
Uno se pregunta si, dado el libre albedrío, el comportamiento humano es intrínsecamente impredecible y, por lo tanto, no está sujeto a ninguna ley. Tan tentador como esta línea de pensamiento puede ser, es un error. Incluso si no hay leyes deterministas del comportamiento humano, hay, de hecho, muchas regularidades en la acción humana.\\

Porque las personas pueden, como señala Frank Knight, cometer errores o no reconocer las cosas. Como primera aproximación, los economistas se abstraen de tales dificultades. Suponen que las personas tienen información perfecta. Al suponer que la gente cree cualesquiera que sean los hechos, los economistas pueden evitar preocuparse por lo que la gente realmente cree.\\

Un punto de vista es que la economía sirve a la política de la misma manera que las ciencias naturales guían las políticas, es decir, ayudando a los formuladores de políticas a elegir los medios que lograrán sus fines. Tal papel práctico para el conocimiento científico no parece problemático. Los agentes tienen algún objetivo que quieren lograr, y el científico proporciona el "saber hacer" necesario. Desde este punto de vista, la economía importa a la política sólo como una fuente de información descriptiva o “libre de valores”.

\section{Una introducción a la economía}
La economía comienza en el siglo XVIII con los escritos de los fisiócratas franceses, de Cantillon y Hume, y especialmente de Adam Smith. Lo que diferenció a estos pensadores de sus predecesores fue su creciente reconocimiento de la existencia de mecanismos mediante los cuales las acciones individuales tendrían consecuencias sistemáticas sin necesidad de que el gobierno controlara los procesos. Smith y otros llegaron a ver la economía en gran medida como un sistema autorregulador. La economía surgió cuando se comprendió que había mecanismos y sistemas económicos para estudiar. \\

La economía se ha preocupado principalmente por comprender cómo funciona un sistema económico capitalista. (Un sistema económico capitalista es una economía de mercado en la que los medios de producción son en su mayor parte de propiedad privada, y los trabajadores son libres de aceptar o rechazar ofertas de empleo). Muchos economistas creen que sus teorías también se aplican a otros arreglos económicos, y se ha trabajado mucho en otros tipos de economías. Pero el núcleo de la teoría económica se ha dedicado a comprender las economías capitalistas.\\

 Los economistas “clásicos”, de los cuales Adam Smith, David Ricardo y John Stuart Mill son los más destacados, no tenían mucho que decir sobre las elecciones de los consumidores. Su énfasis estaba en la producción y en los factores que influyen en la oferta de bienes de consumo. Consideraron que los agentes buscaban maximizar sus ganancias financieras y dividieron tanto a los agentes como a los insumos básicos en tres clases principales:
\begin{itemize}
    \item Capitalistas con su capital (que concibieron como existencias de bienes acumulados o el valor de los mismos), 
    \item terratenientes con su tierra y
    \item trabajadores con su capital.
\end{itemize}
Estos ofrecieron dos generalizaciones principales con respecto a la producción
\begin{itemize}
    \item Asumieron que en cualquier momento dado todos los bienes reproducibles (excluyendo así cosas como la pintura) podrían producirse en cualquier cantidad por el mismo costo por unidad.
    \item Luego, descubrieron los rendimientos decreciente. A menos que haya alguna innovación tecnológica, a medida que se dedica más y más trabajo a una cantidad fija de tierra la cantidad que aumenta la producción cuando se emplea un trabajador adicional eventualmente disminuirá.
\end{itemize}

A diferencia de Ricardo, a finales del siglo XIX, los economistas reconocieron que la población no tenía por qué crecer en respuesta a los salarios más altos. Además, las mejores tecnológicas provocaron aumentos en la productividad. Luego llegaron los neo-clásicos o marginales, que centraron su atención en la elección y el intercambio individuales, mediante las preferencias del consumidor, el intercambio y la demanda de mercancías.\\


Muchos de los primeros economistas neo-clásicos fueron influenciados por el utilitarismo, una teoría ética expuesta por Jeremy Bentham y John Stuart Mill. Según los utilitaristas, las preguntas de política social deben responderse calculando las consecuencias de las alternativas para el felicidad total de los individuos. La política que maximiza la suma de las utilidades individuales es la moralmente correcta. Bentham sostuvo que la utilidad de algo para un individuo es una sensación que, en principio, podría cuantificarse y medirse. También creía que los individuos actúan para maximizar su propia utilidad. Jevons desarrolló la noción esencialmente benthamita de una función de utilidad. Cada opción abierta a un resultado individual en una cierta cantidad de utilidad para esa persona. Entonces se puede aclarar la noción de racionalidad manteniendo que las personas actúan para maximizar alguna función de utilidad consistente. Además, los economistas neoclásicos asumieron que los consumidores generalmente no están satisfechos, que siempre preferirán un paquete $x$ de bienes o servicios a otro paquete y si $x$ es inequívocamente mayor que $y$. La insaciabilidad es tanto una primera aproximación plausible como articula la noción de interés propio. Todo lo que importa a los agentes son los paquetes de mercancías y servicios que están entregando o recibiendo. \\

Con la adición de una generalización más, se tiene el núcleo de la teoría económica moderna. Los primeros economistas neo-clásicos notaron que a medida que uno consume más de cualquier producto o servicio, cada unidad adicional aumenta la utilidad de uno a una tasa decreciente, llamada ley de utilidad marginal decreciente. Esta teoría se ha refinado enormemente, donde "utilidad" es solo otra forma de hablar sobre preferencias. En términos generales, uno puede reemplazar la "ley" de la utilidad marginal decreciente con la generalización de que las personas están dispuestas a pagar menos por unidades adicionales de productos básicos que ya tienen mucho que por productos básicos de los que tiene muy poco. A pesar de estos refinamientos la teoría dominante sigue siendo reconociblemente la teoría desarrollada por los primeros economistas neoclásicos.\\

Luego de los años 30, Keynes desafió la teoría dominante, enfatizando la importancia de la liquidez tanto para las empresa como para los individuos cuando se enfrentan a las incertidumbres. Los precios, especialmente los salarios, no caen fácilmente, y los salarios más bajos pueden generar menos gastos, lo que suprimiría la demanda de productos básicos y conduciría a una caída aún más profunda. Para ello Keynes mencionó que el estado podría aumentar la demanda agregada de de materias primas y alentar la inversión y de esa manera sacar la economía del desempleo. De todas formas, Keynes no sacudió los fundamentos de la teoría neo-clásica. Pero las versiones actualizadas de la economía Keynesiana, siguen siendo influyentes. Siendo así, la macroeconomía un área inestable de la economía.\\

La econometría es una rama de la estadística aplicada, así como una rama de la economía. A partir de la década de 1930, se esperaba que las afirmaciones de los teóricos económicos pudieran probarse y refinarse con la ayuda de técnicas estadísticas. Desde entonces, las técnicas econométricas se han vuelto mucho más sofisticadas. Exactamente lo que este trabajo significa para la teoría económica (a diferencia de las investigaciones prácticas de enfoque limitado) es controvertido, y algunos economistas prominentes argumentan que la econometría es incapaz de proporcionar buenas razones para creer o no creer en afirmaciones causales significativas.

Por otro lado, de acuerdo con el materialismo histórico de Marx, las relaciones entre las personas en el curso de sus actividades productivas son las relaciones sociales más fundamentales. Marx, considera al capitalismo, a pesar de las miserias que puede causar un enorme adelanto para el ser humano. Pero, como argumentos en su ensayo inicial "trabajo enajenado", no permite que las personas decidan racionalmente y conscientemente, como se debe desarrollar la sociedad y la naturaleza humana. Por lo que Marx cree que las personas pueden trascender el capitalismo y organizar la producción y la distribución de alguna manera racional. Pero no tuvo existe, ya que las teorías económicas no se ciernen sobre las olas políticas. \\

Luego apareció la economía institucionalista, que no ignoran las tomas de decisiones individuales, pero enfatizan las restricciones en evolución sobre los agentes que ocupan roles económicos específicos.\\

La tercera alternativa contemporánea a la economía dominante, es la economía del comportamiento, incluida la neuroeconomía. \\

También se encuentran los neorricardianos, que creen que se puede comprender mejor la economía reformulando ecuaciones matemáticas. Los economistas Austriacos que están de acuerdo con los economistas neo-clásicos pero enfatizan la importancia de la incertidumbre, desequilibrio y un punto de vista subjetivo. Los economistas postkeynesianos a menudo ofrecen críticas similares en la alta teoría, pero a diferencia de los austriacos, tienden a defender las políticas intervencionalistas.  Los pronosticadores económicos a menudo dependen muy poco de cualquier teoría específica. 


\section{Una introducción a la metodología de la economía}

Knight y los austriacos están de acuerdo en que tan pronto como se abandona el punto de vista subjetivo y se piensa en la economía como si fuera una ciencia natural, se pierden de vista los rasgos centrales del tema. Luego Hutchison argumenta que la economía, como otras ciencias, debe formular generalizaciones comprobadas y someterlas a pruebas serias. Como también Milton Friedman intenta mostrar que la economía satisface estándares ampliamente positivistas, que se convirtió en el trabajo más influyente sobre metodología económica del siglo XX.


\part{Discusiones clásicas}

\chapter{Sobre la definición y el método de la economía política\\ John Stuart Mill}

Mill fue uno de los primeros defensores de los derechos de la mujer y de un socialismo democrático moderado.\\

La economía política, no trata de toda la naturaleza del hombre modificada por el estado social, ni de toda la conducta del hombre en sociedad. Si no se refiere a él únicamente y exclusivamente como un ser que desea poseer riquezas y que es capaz de juzgar la eficacia comparativa de los medios para obtener ese fin. Bajo esa influencia de deseo, muestrea a la humanidad acumulando riqueza y empleando esa riqueza en la producción de otra riqueza, sancionando de común acuerdo la institución de la propiedad, estableciendo leyes para evitar que los individuos invadan las propiedades de otros por la fuerza o el fraude, adoptando diversos artilugios para aumentar la productividad de su trabajo, resolviendo la división del producto por acuerdo, bajo la influencia de la competencia. Todo esto y más para facilitar la producción. Ahora bien, cuando un efecto depende de una concurrencia de causas, esas causas deben estudiarse una a la vez, y sus leyes deben investigarse por separado, esto si deseamos a través de las causas, obtener el poder de predecir o controlar el efecto, ya que la ley del efecto se compone de las leyes de todas las causas que lo determinan. Para juzgar cómo el ser humano actuará bajo la variedad de deseos y aversiones que simultáneamente operan sobre él, debemos saber cómo actuaría bajo la influencia exclusiva de cada uno en particular. \\

El economista político, se pregunta cuáles son las acciones que producirían este deseo, si, dentro de los departamentos en cuestión, no estuviera impedido por ningún otro. Pero con respecto a aquellas partes de la conducta humana en las que la riqueza ni siquiera es el objeto principal, la Economía Política no pretende que sus conclusiones sean aplicables. Sólo en algunos casos llamativos, como el principio de población serán tomados en cuenta y deben ser corregidos teniendo debidamente en cuenta los efectos de cualquier impulso de una descripción diferente, que puede demostrarse que interfiere con el resultado en cualquier caso particular. Las conclusiones de la Economía Política dejarán de ser aplicables a la explicación o predicción de los acontecimientos reales, hasta que sean modificadas por una correcta asignación del grado de influencia ejercido por la otra causa.\\

Por lo tanto, la definición de economía política sera:

\begin{center}
    \textit{La ciencia que traza las leyes de los fenómenos de la sociedad que surgen de las operaciones combinadas de la humanidad para la producción de riqueza, en la medida en que esos fenómenos no son modificados por la búsqueda de cualquier otro objeto.}
\end{center}

Tengamos en cuenta que las diferencias de principio entre ciencias difieren no sólo en lo que creen ver, si no en el lugar de donde obtuvieron la luz con la que creen verlo. La más universal de las formas en que suele presentarse esta diferencia de método, es la antigua disputa entre lo que se llama:

\begin{itemize}
\item teoría y
\item práctica o experiencia. 
\end{itemize}

De todas formas las dos partes necesitarán de experiencia y teoría a diferencia que los que se hacen llamar prácticos, requerirán experiencia específica, y argumentarán totalmente hacia arriba desde hechos particulares hasta una conclusión general; mientras que aquellos que se llaman teóricos, pretenderán abarcar un campo más amplio de experiencia, y habiendo argumentado hacia arriba desde hechos particulares hasta un principio general que incluye un rango mucho más amplio que el de la cuestión en discusión, argumentaran hacia abajo desde ese principio general hasta una variedad de conclusiones específicas. Estos métodos  se pueden llamar: 
\begin{itemize}
    \item a posteriori.- entendemos el que requiere como base de sus conclusiones, no la mera experiencia, si no la experiencia concreta.
    \item a priori.- entendemos con el razonamiento a partir de una hipótesis supuesta.
\end{itemize}
Verificar, la hipótesis misma a posteriori, no forman parte en absoluto del quehacer científico, sino de la aplicación de la ciencia.\\

Con respecto a la economía política, lo hemos caracterizado como una ciencia abstracta y un método a priori. Ya que, supone una definición arbitraria de hombre, como un ser que invariablemente hace aquello por lo que puede obtener la mayor cantidad de necesidades, comodidades y lujos, con la menor cantidad de trabajo y abnegación física con la que puede obtenerse en el estado de conocimiento existente. Ahora, bien nadie que esté familiarizado con los tratados sistemáticos de economía política pondrá en duda que siempre que un economista ha demostrado que actuando de una manera determinada, un trabajador puede obtener salarios más altos, o un capitalista tener mayores beneficios. Esto ya que, el economista político razona a partir de premisas asumidas, que pueden carecer de fundamentos en los hechos. Por consiguiente, las conclusiones de la economía política, como de la geometría, solo son verdades en abstracto, es decir son verdaderas bajo ciertas suposiciones, en las que sólo se tienen en cuenta causas generales, causas comunes a toda la clase de casos considerados. El método a priori que se le imputa, como si su empleo demostrara que toda su ciencia carece de valor, es, como demostraremos más adelante, el único método por el cual puede alcanzarse la verdad en cualquier departamento de la ciencia social. Siempre que a medida en que los hechos reales se alejen de la hipótesis deban permitirse una desviación correspondiente de la letra estricta de su conclusión, de lo contrario sólo será verdad para las cosas que ha supuesto arbitrariamente, no para las cosas que existen. Cuando una determinada causa existe realmente, y si se la dejara sola produciría infaliblemente un determinado efecto, ese mismo efecto, modificado por todas las demás causas concurrentes, corresponderá correctamente al resultado realmente producido.\\

Luego, el método a posteriori será útil siempre que este en relación con el método a priori, ya que por si solo no podremos hacer experimentos morales o de ética. A estos experimentos se le denominan \textit{experimentos cruciales}. Por ejemplo, en asuntos de estado, ¿cómo podemos obtener un experimentos crucial sobre el efecto de una política comercial restrictiva sobre la riqueza nacional?. Tendremos que encontrar dos naciones iguales en todos los demás aspectos, o al menos que posean, en un grado exactamente igual, todo lo que conduce a la opulencia nacional, y que adopten exactamente la misma política en todos sus otros asuntos, pero que difieran en esto solamente, en que una de ellas adopta un sistema de restricciones comerciales, y la otra adopta el libre comercio. Este sería un experimento decisivo, similar a los que casi siempre podemos obtener en física experimental. Por lo tanto, ya sea en la política económica o en las ciencias sociales, mientras vemos los hechos concretos revestidos de complejidad, no queda otro método que el a priori o el de la especulación abstracta.\\

Ahora, las causas pueden ser de experimentación específica, que son las leyes de la naturaleza humana y las circunstancias externas capaces de excitar la voluntad humana a la acción, que están al alcance de nuestra observación. También podemos observar cuales son las causas que excitan estos deseos. Si al final la suposición es correcta hasta donde llega, y no difiere de la verdad más que como una parte difiere del todo, entonces las conclusiones que se deducen correctamente de la suposición constituyen una verdad abstracta; y cuando se completan añadiendo o sustrayendo el efecto de las circunstancias no calculadas, son verdaderas en lo concreto, y pueden aplicarse a la práctica. De este carácter es la ciencia de la economía política, para hacerla perfecta como ciencia abstracta, las combinaciones de circunstancias que asume, para rastrear sus efectos, deben incorporar todas las circunstancias que son comunes a todos los casos, y del mismo modo todas las circunstancias que son comunes a cualquier clase importante de casos. Las conclusiones correctamente deducidas de estas suposiciones serían tan verdaderas en abstracto como las de las matemáticas; y sería una aproximación lo más cercana posible a la verdad abstracta, a la verdad en lo concreto.\\

Cuando los principios de la economía política se van a aplicar a un caso en particular, entonces es necesario tener en cuenta todas las circunstancias individuales de ese caso; no sólo examinando a cuál de los conjuntos de circunstancias contemplados por la ciencia abstracta corresponden las circunstancias del caso en cuestión, sino también qué otras circunstancias pueden existir en ese caso, que no siendo comunes a él con ninguna clase grande y fuertemente marcada de casos, no han caído bajo el conocimiento de la ciencia. Estas circunstancias se han denominado causas perturbadoras. Estas causas tiene sus leyes que a priori de las leyes de las causas perturbadoras, la naturaleza y la cantidad de la perturbación puede predecirse a priori. Por lo tanto, el efecto de las causas especiales debe entonces sumarse o restarse del efecto de las generales. \\

Habiendo demostrado ahora que el método a priori en Economía Política, y en todas las demás ramas de la ciencia moral, es el único modo cierto o científico de investigación, y que el método a posteriori, o el de la experiencia específica, como medio de llegar a la verdad, es inaplicable a estos temas, podremos demostrar que este último método es, no obstante, de gran valor en las ciencias morales; a saber, no como medio de descubrir la verdad, sino de verificarla, y de reducir al punto más bajo esa incertidumbre antes aludida como derivada de la complejidad de cada caso particular, y de la dificultad (por no decir imposibilidad) de que se nos asegure a priori que hemos tenido en cuenta todas las circunstancias materiales.\\

Si pudiéramos estar completamente seguros de que conocemos todos los hechos del caso particular, podríamos obtener poca ventaja adicional de la experiencia específica y podremos llamarnos profetas, pero estas no son reveladas, por lo que deben ser recogidas por la observación. Ahora, algunas de las causas pueden estar más allá de la observación. Donde corremos el riesgo de solo prestar atención a una parte de las causas. Pero a la vez tender a deducirlas lo más exactas posibles, haciendo abstractas de todas las circunstancias excepto de las forman parte de la hipótesis. Así, menos probable sospechemos que estamos en un error.\\

El verdadero estadista práctico es aquel que combina esta experiencia con un profundo conocimiento de la filosofía política abstracta. Cualquier adquisición, sin la otra, lo deja cojo e impotente si es consciente de la deficiencia; lo vuelve obstinado y presuntuoso si, como es más probable, es completamente inconsciente de ello.\\

El método del filósofo práctico consiste, en dos procesos:

\begin{enumerate}[1.]
    \item analítico,
    \item sintético.
\end{enumerate}

Debe analizar el estado actual de la sociedad en sus elementos, sin dejar caer ni perder ninguno de ellos por el camino. Después de referirse a la experiencia del hombre individual para aprender la ley de cada uno de estos elementos, es decir, para aprender cuáles son sus efectos naturales, y cuánto del efecto se sigue de tanto de la causa cuando no es contrarrestado por ninguna otra causa, queda una operación de síntesis; poner todos estos efectos juntos, y, de lo que son por separado, recoger lo que sería el efecto de todas las causas actuando a la vez. Si estas diversas operaciones pudieran realizarse correctamente, el resultado sería la profecía; pero como sólo se pueden realizar con una cierta aproximación de corrección, la humanidad nunca puede predecir con absoluta certeza, sino sólo con una menor o menor precisión con mayor o menor grado de probabilidad. \\

Si la fuerza que, siendo la menos conspicua de las dos, se llama fuerza perturbadora, prevalece lo suficiente sobre la otra fuerza en algún caso, para constituir ese caso lo que comúnmente se llama una excepción, la misma fuerza perturbadora probablemente actúa como causa modificadora en muchos otros casos que nadie llamará excepciones.. Si se dijera que es una ley de la naturaleza que todos los cuerpos pesados caen a tierra, probablemente se diría que la resistencia de la atmósfera, que impide que un globo caiga, constituye al globo una excepción a esa pretendida ley de la naturaleza. Pero la verdadera ley es que todos los cuerpos pesados tienden a caer, y a esto no hay excepción, ni siquiera el sol y la luna, porque incluso ellos, como todo astrónomo sabe, tienden hacia la tierra con una fuerza exactamente igual a la que la tierra tiende hacia ellos. La resistencia de la atmósfera podría, en el caso particular del globo, por una mala interpretación de lo que es la ley de la gravitación, decirse que prevalece sobre la ley; pero su efecto perturbador es tan real en cualquier otro caso, ya que aunque no impide, retarda la caída de todos los cuerpos. La regla y la llamada excepción no dividen los casos entre sí; cada una de ellas es una regla general que se extiende a todos los casos. Llamar a uno de estos principios concurrentes excepción del otro, es superficial y contrario a los principios correctos de nomenclatura y ordenación. Un efecto exactamente de la misma clase, y derivado de la misma causa, no debe colocarse en dos categorías diferentes, por el mero hecho de que exista o no otra causa preponderante sobre él.


\chapter{Objetividad y compresión de la economía \\ Max Weber}

Toda reflexión seria sobre los elementos últimos de la conducta humana significativa se orienta principalmente en términos de las categorías:

\begin{itemize}
    \item Fines,
    \item Medios.
\end{itemize}

En la medida en que somos capaces de determinar (dentro de los límites actuales de nuestro conocimiento) qué medios para la consecución de un fin propuesto son apropiados o inapropiados, podemos de este modo estimar las posibilidades de alcanzar un fin determinado con ciertos medios disponibles. De este modo, podemos criticar indirectamente la fijación del fin en sí como prácticamente significativo (sobre la base de la situación histórica existente) o como carente de sentido con referencia a las condiciones existentes. Además, cuando parece existir la posibilidad de alcanzar un fin propuesto, podemos determinar (naturalmente dentro de los límites de nuestro conocimiento existente) las consecuencias que la aplicación de los medios a utilizar producirá además de la eventual consecución del fin propuesto, como resultado de la interdependencia de todos los acontecimientos. De este modo, podemos proporcionar a la persona que actúa la capacidad de sopesar y comparar las consecuencias indeseables frente a las deseables de su acción. Así, podemos responder a la pregunta: ¿cuánto 'costará' la consecución de un fin deseado en términos de la pérdida previsible de otros valores?.\\

El tipo de ciencia social que nos interesa es una ciencia empírica de la realidad concreta. Queremos comprender, por un lado, las relaciones y el significado cultural de los acontecimientos individuales en sus manifestaciones contemporáneas y, por otro, las causas de que históricamente sean así y no de otro modo. Ahora bien, en cuanto intentamos reflexionar sobre el modo en que la vida nos enfrenta en situaciones concretas inmediatas, nos presenta una multiplicidad infinita de acontecimientos que surgen y desaparecen sucesiva y coexistentemente, tanto 'dentro' como 'fuera' de nosotros mismos. Todo el análisis de la realidad infinita que la mente humana finita puede llevar a cabo descansa sobre el supuesto tácito de que sólo una porción finita de esta realidad constituye el objeto de la investigación científica, y que sólo ella es 'importante' en el sentido de ser "digna de ser conocida". Pero, cuales son los criterios por lo que se selecciona este segmento?; Tan pronto como hayamos demostrado que alguna relación causal es una 'ley', es decir, si hemos demostrado que es universalmente válida por medio de una inducción histórica exhaustiva o la hemos hecho inmediata y tangiblemente plausible según nuestra experiencia subjetiva, un gran número de casos similares se ordenan bajo la fórmula así alcanzada. Aquellos elementos de cada suceso individual que quedan sin explicar por la selección de sus elementos subsumibles bajo la 'ley' se consideran residuos científicamente no integrados de los que se ocupará el perfeccionamiento ulterior del sistema de 'leyes'. Alternativamente, se considerarán 'accidentales' y, por lo tanto, sin importancia científica porque no encajan en la estructura de la 'ley'; en otras palabras, no son típicos del acontecimiento y, por lo tanto, sólo pueden ser objeto de 'curiosidad ociosa'. En consecuencia, el ideal al que sirven todas las ciencias, incluidas las culturales es un sistema de proposiciones a partir del cual puede deducirse la verdad.\\

Hemos denominado 'ciencias de la cultura' a aquellas disciplinas que analizan los fenómenos de la vida en función de su significación cultural. El concepto de cultura es un concepto de valor. La realidad empírica se convierte en cultura para nosotros porque la relacionamos con ideas de valor en la medida que lo hacemos y se volvieron significativos para nosotros. Es significativa porque revela relaciones que son importantes debido a su conexión con nuestros valores. Sin embargo, no podemos descubrir lo que tiene sentido para nosotros mediante una investigación sin supuestos de los datos empíricos. Más bien, la percepción de su significado para nosotros es la presuposición de que se convierta en objeto de investigación. Naturalmente, el sentido no coincide con las leyes como tales, y cuanto más general es la ley, menor es la coincidencia. Porque el significado específico que un fenómeno tiene para nosotros no se encuentra naturalmente en las relaciones que comparte con muchos otros fenómenos. ...\\

¿Cuál es la consecuencia de todo esto?\\
Naturalmente, ello no implica que el conocimiento de proposiciones universales, la construcción de conceptos abstractos, el conocimiento de regularidades y el intento de formular 'leyes' carezcan de justificación científica en las ciencias de la cultura. Muy al contrario, si el conocimiento causal de los historiadores consiste en la imputación de efectos concretos a causas concretas, una imputación válida de cualquier efecto individual sin la aplicación del conocimiento 'nomológico' -es decir, el conocimiento de secuencias causales recurrentes- sería en general imposible. En otras palabras, los efectos 'adecuados' de los elementos causales implicados deben tenerse en cuenta para llegar a cualquier conclusión de este tipo. \\
En todas partes, sin embargo, y por tanto también en la esfera de los procesos económicos complicados, cuanto más cierto y amplio sea nuestro conocimiento general, mayor será la certeza de la imputación. No se trata  aquí de leyes económicas, sino de relaciones adecuadas expresadas en regla y con la aplicación de la categoría de 'posibilidad objetiva'. El establecimiento de tales regularidades no es el fin, sino más bien el medio del conocimiento. El que una relación causal de la experiencia cotidiana que se repite con regularidad deba formularse como una "ley" es una cuestión de conveniencia que debe resolverse por separado en cada caso concreto. \\

Ahora, un análisis objetivo no carece de sentido, porque:

\begin{enumerate}[1.]
    \item el conocimiento de las leyes sociales no es conocimiento de la realidad social, sino más bien uno de los diversos medios de los que se sirve nuestra mente para alcanzar este fin.
    \item el conocimiento de los acontecimientos culturales sólo es concebible a partir del significado que las constelaciones concretas de la realidad tiene para nosotros en determinadas situaciones concretas individuales.
\end{enumerate}

Todo conocimiento de la realidad cultural, como puede verse, es siempre un conocimiento desde puntos de vista particulares.\\

Entre las muchas confusiones, apareció el intento de refutar la concepción materialista de la historia, afirmando que puesto que toda la vida económica debe tener lugar en formas legal o convencionalmente reguladas, todo desarrollo económico debe adoptar la forma de esfuerzo por la creación de nuevas formas legales. Por lo tanto, se dice que sólo es inteligible a través de máximas éticas y que, por este motivo, es esencialmente diferente de cualquier tipo de desarrollo 'natural'. En consecuencia, se dice que el conocimiento del desarrollo económico es de carácter 'teleológico'.  Quienes consideren de importancia secundaria la laboriosa tarea de comprender causalmente la realidad histórica pueden prescindir de ella, pero es imposible suplantarla por cualquier tipo de 'teleología'. Desde nuestro punto de vista, la 'finalidad' es la concepción de un efecto que se convierte en causa de una acción. Puesto que tenemos en cuenta toda causa que produce o puede producir un efecto significativo, también consideramos ésta. Su significación específica consiste únicamente en el hecho de que no sólo observamos la conducta humana, sino que podemos y deseamos comprenderla.\\

La elección del objeto de investigación y la extensión o profundidad con que esta investigación intenta penetrar en la infinita red causal, están determinadas por las ideas valorativas que dominan al investigador y su época.\\

Podemos finalmente pasar a la cuestión que es metodológicamente relevante en la consideración de la 'objetividad' del conocimiento cultural. La pregunta: ¿cuál es la función lógica y la estructura de los conceptos que nuestra ciencia, como todas las demás, utiliza? Replanteada con especial referencia al problema decisivo, la pregunta es: ¿cuál es el significado de la teoría y la conceptualización teórica para nuestro conocimiento de la realidad cultural?\\
La economía fue originalmente -como ya hemos visto- una "técnica", al menos en el foco central de su atención. Con esto queremos decir que consideraba la realidad desde un punto de vista evaluativo práctico, al menos ostensiblemente inequívoco y estable: a saber, el aumento de la "riqueza" de la población. Por otra parte, fue desde el principio algo más que una técnica, ya que se integró en el gran esquema del derecho natural (teoría filosófica y jurídica que sostiene la existencia de principios y normas universales e inmutables que se consideran intrínsecos a la naturaleza humana y que, por lo tanto, son inherentes a todas las sociedades y sistemas legales.) y de la Weltanschauung racionalista del siglo XVIII (visión del mundo que se basa principalmente en la razón y la lógica como herramientas fundamentales para entender la realidad y resolver problemas).\\

 El resultado, en la medida en que nos interesa, es que a pesar de la poderosa resistencia a la infiltración del dogma naturalista debido al idealismo alemán desde Fichte y el logro de la Escuela Histórica Alemana en derecho y economía y en parte por el trabajo mismo de la Escuela Histórica, el punto de vista naturalista en ciertos problemas decisivos aún no ha sido superado. Entre estos problemas encontramos la relación entre “teoría” e “historia”, que aún es problemática en nuestra disciplina.\\

 La única forma exacta de conocimiento -la formulación de leyes inmediata e intuitivamente evidentes- es, sin embargo, al mismo tiempo la única que ofrece acceso a acontecimientos que no han sido observados directamente. De ahí que, al menos en lo que se refiere a los fenómenos fundamentales de la vida económica, la construcción de un sistema de proposiciones abstractas y, por tanto, puramente formales, análogas a las de las ciencias naturales exactas, sea el único medio de analizar y dominar intelectualmente la complejidad de la vida social. La teoría económica exacta se ocupa del funcionamiento de un motivo psíquico, las demás teorías tienen como tarea formular el comportamiento de todos los demás motivos en tipos similares de proposiciones que gocen de validez hipotética. En consecuencia, en ocasiones se ha hecho la fantástica afirmación respecto a las teorías económicas -por ejemplo, las teorías abstractas del precio, el interés, la renta, etc.- de que pueden, siguiendo ostensiblemente la analogía de las proposiciones de la ciencia física, aplicarse válidamente a la derivación de conclusiones cuantitativamente enunciadas a partir de premisas reales dadas, ya que, dados los fines, el comportamiento económico respecto a los medios está inequívocamente determinado. Esta afirmación omite observar que para poder llegar a este resultado incluso en el caso más simple, la totalidad de la realidad histórica existente, incluyendo cada una de sus relaciones causales, debe ser asumida como dada y presupuesta como conocida. \\

 La teoría abstracta pretendía basarse en axiomas psicológicos y, en consecuencia, los historiadores han reclamado una psicología empírica para demostrar la invalidez de esos axiomas y derivar el curso de los acontecimientos económicos a partir de principios psicológicos.  En el establecimiento de las proposiciones de la teoría abstracta, se trata sólo aparentemente de "deducciones" a partir de motivos psicológicos fundamentales. En realidad, las primeras son un caso especial de un tipo de construcción de conceptos que es peculiar y, hasta cierto punto, indispensable, para las ciencias de la cultura. En este punto merece la pena describirlo con más detalle, ya que así podemos acercarnos más a la cuestión fundamental del significado de la teoría en las ciencias sociales. Tenemos en la teoría económica abstracta una ilustración de esas construcciones sintéticas que se han designado como "ideas" de los fenómenos históricos. Nos ofrece una imagen ideal de los acontecimientos del mercado de mercancías en las condiciones de una sociedad organizada según los principios de la economía de intercambio, la libre competencia y una conducta rigurosamente racional. 



 \chapter{La naturaleza y el significado de la ciencia económica} 

 \section{Capítulo 1: La asignatura de economía}

 Los fines son varios, los medios son limitados. Además los fines tiene significado diferente. Ahora bien, la multiplicidad de fines por sí misma no tiene interés necesario para el economista. Si quiero hacer dos cosas, y tengo tiempo y medios amplios para hacerlas, y no necesito el tiempo ni los medios para otra cosa, entonces mi conducta no asume ninguna de las formas que son objeto de la ciencia económica. La mera limitación de medios tampoco es suficiente por sí misma para dar lugar a fenómenos económicos. Si los medios de satisfacción no tienen uso alternativo, entonces pueden ser escasos, pero no pueden ser economizados. El Maná que cayó del cielo pudo haber sido escaso, pero, si era imposible cambiarlo por otra cosa o posponer su uso, no era objeto de ninguna actividad con un aspecto económico. Tampoco es la aplicabilidad alternativa de medios escasos una condición completa de la existencia del tipo de fenómenos que estamos analizando. Si el sujeto económico tiene dos fines y un medio para satisfacerlos, y los dos fines son de igual importancia, su posición será como la posición del asno en la fábula, paralizado a medio camino entre los dos paquetes de heno igualmente atractivos. Pero cuando el tiempo y los medios para alcanzar los fines son limitados y susceptibles de aplicación alternativa; es decir que existen otras opciones o soluciones que podrían ser consideradas en lugar de las ya propuestas o utilizadas., y los fines pueden distinguirse por orden de importancia, entonces el comportamiento asume necesariamente la forma de elección. Todo acto en el que se emplean tiempo y medios escasos para la consecución de un fin implica la renuncia a su uso para la consecución de otro. Tiene un aspecto económico. La escasez de medios para satisfacer fines de importancia variable es una condición casi omnipresente del comportamiento humano. \\
 La economía es la ciencia que estudia el comportamiento humano como una relación entre fines y medios escasos que tienen usos alternativos.

 \section{Capítulo 4: La naturaleza de las generalizaciones económicas}

 Las proposiciones más fundamentales del análisis económico son las proposiciones de la teoría general del valor. Podemos proceder a preguntar, de que depende su validez. No debería ser necesario dedicar mucho tiempo a demostrar que no puede basarse en una mera apelación histórica, ya que no habría razón alguna para suponer que la historia se repetiría; porque si hay algo que demuestra la historia, no menos que la lógica elemental, es que la inducción histórica, sin ayuda del juicio analítico, es la peor base posible para la profecía.\\
 Es también evidente que nuestra creencia no se basa en los resultados de un experimento controlado. Pero nuestra creencia en las proposiciones, hablando de la teoría del valor, es completa como la creencia basada en cualquier número de experimentos controlados. Pero ¿de qué depende?. Nos damos cuenta de que el fundamento de la teoría del valor es la suposición de las diferentes cosas que el individuo desea hacer tiene una importancia diferente para él y, por tanto, pueden ordenarse de una determinada manera. Que se deduce a que podamos juzgar si diferentes experiencias posibles son de importancia equivalente o mayor o menor para nosotros. De este hecho derivamos la idea de la sustituibilidad de bienes entre diferentes usos, de la demanda de un bien en términos de otro, de una distribución equilibrada de biene entre diferentes usos, del equilibrio del intercambio y de la formación de precios. Supones también una distribución inicial de la propiedad. Pero siempre el principal supuesto de fondo es el de los sistemas de valoración de los distintos agentes económicos.\\

 Como es bien sabido, el principal principio de explicación, complementario de los principios de valoración subjetiva asumidos en la teoría más restringida del valor y el intercambio, es el principio a veces descrito como la Ley de los Rendimientos Decrecientes. La ley de los rendimientos decrecientes es simplemente una forma de expresar el hecho obvio de que los diferentes factores de producción son sustitutos imperfectos entre sí. Si se aumenta la cantidad de trabajo sin aumentar la cantidad de tierra, el producto aumentará, pero no aumentará proporcionalmente. Para duplicar el producto, si no se duplican tanto la tierra como el trabajo, hay que duplicar con creces cualquiera de los dos factores. También se deduce de consideraciones más íntimamente relacionadas con nuestras concepciones fundamentales.Una clase de factores escasos debe definirse como constituida por aquellos factores que son sustitutos perfectos.Es decir, las diferencias de factores se definen esencialmente como sustituibilidad imperfecta. La ley de los rendimientos decrecientes se desprende, por tanto, de la suposición de que existe más de una clase de factores de producción escasos. El principio complementario de que, dentro de unos límites, los rendimientos pueden aumentar, se desprende igualmente de la suposición de que los factores son relativamente indivisibles. Sobre la base de estos principios y con la ayuda de supuestos subsidiarios del tipo ya mencionado (la naturaleza de los mercados y el marco jurídico de la producción, etc.), es posible construir una teoría del equilibrio de la producción.\\

 El propósito de este ejemplo y muchos en la teoría económica es señalar la evidente deducción de una serie de postulados, que son supuestos que implican de alguna manera hechos simples e indiscutibles de la experiencia, relativos a la forma en que la escasez de bienes, que es el objetivo de nuestra ciencias, se manifiesta en el mundo de la realidad. En la teoría del valor el postulado es el hecho de que los individuos pueden ordenar sus preferencias. El postulado principal de la teoría de la producción es el hecho de que hay más de un factor de producción, el principal postulado de la teoría de la dinámica es el hecho de que no tenemos certeza sobre las escaseces futuras. Basta con enunciarlos para reconocerlos como evidentes.\\

 Ahora, bien la verdad de las deducciones de esta estructura depende, como siempre, de su coherencia lógica. Al igual que en las ciencias sociales debemos tener cuidado de investigar la naturaleza de nuestro objeto. Pero si bien es importante darse cuenta de cuántos son los supuestos secundarios que surgen necesariamente a medida que nuestra teoría se vuelve más y más complicada, es igualmente importante darse cuenta de cuán ampliamente aplicables son los supuestos principales sobre los que descansa. Como hemos visto, los principales son aplicables siempre y dondequiera que se den las condiciones que dan lugar a los fenómenos económicos. \\

 Estas consideraciones, deberían permitirnos detectar la falacia implícita siguiente: A veces se ha afirmado que las generalizaciones de la Economía tienen un carácter esencialmente 'histórico-relativo', que su validez se limita a determinadas condiciones históricas y que, fuera de ellas, carecen de relevancia para el análisis de los fenómenos sociales. Es cierto que, para aplicar de forma fructífera las proposiciones más generales de la Economía, es importante complementarlas con una serie de postulados auxiliares extraídos del análisis de lo que a menudo puede denominarse legítimamente material histórico-relativo. Pero no es cierto que los postulados principales sean histórico-relativos en el mismo sentido. Es cierto que se basan en la experiencia, que se refieren a la realidad. Pero se trata de una experiencia de un grado de generalidad tan amplio que los sitúa en una clase bastante diferente de los supuestos histórico-relativos. Nadie podría en duda los supuestos principales, y nadie que haya examinado estos supuestos puede dudar de la utilidad de partir de este plano. Pero si, como ha ocurrido notoriamente en la historia de las grandes controversias metodológicas, se interpretan en el sentido de que las conclusiones generales que se derivan del análisis general son tan limitadas como sus aplicaciones particulares, que las generalizaciones de la economía política sólo eran aplicables al estado de Inglaterra a principios del reinado de la reina Victoria, y argumentos similares, son totalmente engañosos. Por lo tanto, necesitamos un nuevo término para designar lo que se suele llamar histórico relativo. En Economía pura examinamos la implicación de la existencia de medios escasos con usos alternativos. Como hemos visto, el supuesto de las valoraciones relativas (cuál valoración ofrece las conclusiones más sólidas ) es la base de todas las complicaciones posteriores. \\

 Dada la diversas teorías y confusiones como el hedonismo psicológico en la teoría económica, es fundamentalmente importante distinguir entre la práctica real de los economistas, y la lógica que ésta implica y su ocasional apología ex post facto (defensa después de lo ocurrido). Inspeccionan con celo superfluo la fachada externa, pero evitan el trabajo intelectual de examinar la estructura interna. Por instancia, nadie que conociera la reciente teoría del valor podría seguir sosteniendo honestamente que tiene alguna conexión esencial con el hedonismo psicológico. Lo único que debemos suponer como economistas es el hecho evidente de que las distintas posibilidades ofrecen distintos incentivos y que estos incentivos pueden ordenarse según su importancia. El método científico exige que no se tenga en cuenta nada que no pueda observarse directamente. Podemos tener en cuenta la demanda tal como se manifiesta en el comportamiento observable del mercado. Pero no podemos ir más allá. La valoración es un proceso subjetivo. No podemos observar la valoración. Por lo tanto, no tiene cabida en una explicación científica. Nuestras construcciones teóricas deben partir de datos observables. A primera vista, esto parece muy plausible. El argumento de que no debemos hacer nada que no se haga en las ciencias físicas es muy seductor. Pero es dudoso que esté realmente justificado. Al fin y al cabo, lo nuestro es explicar ciertos aspectos de la conducta. Y es muy dudoso que esto pueda hacerse en términos que no impliquen ningún elemento psíquico.\\

 Pero incluso si restringimos el objeto de la Economía a la explicación de cosas tan observables como los precios, encontraremos que de hecho es imposible explicarlos a menos que invoquemos elementos de naturaleza subjetiva o psicológica. Si suponemos que tal sistema sólo tiene en cuenta los datos observables, nos engañamos a nosotros mismos. De ello se deduce que, si queremos hacer nuestro trabajo como economistas, si queremos ofrecer una explicación suficiente de cuestiones que toda definición de nuestra materia abarca necesariamente, debemos incluir elementos psicológicos. No podemos dejarlos fuera si queremos que nuestra explicación sea adecuada. Ahora, no es posible entender los conceptos de elección, de relación de medios y fines, los conceptos centrales de nuestra ciencia, en términos de observación de datos externos. \\

 Lo relevante para las ciencias sociales no es si los juicios de valor individuales son correctos en el sentido último de la filosofía del valor, sino si se realizan y si son eslabones esenciales en la cadena de explicación causal.

 \section{Capítulo 5: Generalizaciones Económicas y Realidad}

 Si la 'situación dada' se ajusta a un determinado patrón, también deben estar presentes otras características, ya que su presencia es 'deducible' del patrón postulado originalmente. Para ello, el método analítico es un instrumento para 'sacudir' todas las implicaciones de unos supuestos dados. Concedida la correspondencia de sus supuestos originales y los hechos, sus conclusiones son inevitables e ineludibles. Todo esto resulta especialmente claro si consideramos el procedimiento del análisis diagramático, que a explicado las implicaciones ocultas. Por ejemplo, que queremos mostrar los efectos sobre el precio de la imposición de un pequeño impuesto. Hacemos ciertas suposiciones en cuanto a la elasticidad de la demanda, ciertas suposiciones en cuanto a las funciones de coste, las plasmamos en el diagrama habitual, y podemos leer inmediatamente, por así decirlo, los efectos sobre el precio.\\
 Acá podemos sacar una función más para la investigación empírica, ésta puede sacar a la luz los hechos cambiantes que hacen posible la predicción en una situación dada. Éste es sin duda uno de los principales usos de los estudios aplicados: no desenterrar leyes 'empíricas' en un ámbito en el que no cabe esperar tales leyes, sino proporcionar de un momento a otro algún conocimiento de los datos fluctuantes en los que, en la situación dada, puede basarse la predicción. No puede sustituir al análisis formal. Pero puede sugerir en distintas situaciones qué análisis formal es apropiado, y puede proporcionar en ese momento algún contenido para las categorías formales. Por supuesto, si otras cosas no permanecen inalteradas, las consecuencias predichas no se derivan necesariamente de ello. Esta obviedad elemental, necesariamente implícita en cualquier predicción científica, debe mantenerse especialmente en primer plano de atención cuando se habla de este tipo de prognosis. Por lo tanto, las falsas afirmaciones de una ciencia que no tiene en cuenta los hechos queda al descubierto.

 \section{Capítulo 6: La importancia de la ciencia económica}
 La teoría del intercambio supone que puedo comparar la importancia que tiene para mí el pan a 6 unidades monetarias por barra y estas 6 gastadas en otras alternativas presentadas por las oportunidades del mercado. Y asume que el orden de mis preferencias así expuesto puede compararse con el orden de preferencias del panadero. Pero no asume que, en ningún momento, sea necesario comparar la satisfacción que obtengo del gasto de 6 unidades monetarias en pan con la satisfacción que obtiene el panadero al recibirlo. Esa comparación es de una naturaleza completamente diferente. Es una comparación que nunca es necesaria en la teoría del equilibrio y que nunca está implícita en los supuestos de dicha teoría. Es una comparación que queda necesariamente fuera del alcance de cualquier ciencia positiva. Afirmar que la preferencia de A está por encima de la de B en orden de importancia es totalmente distinto de afirmar que A prefiere n a m y B prefiere n y m en distinto orden. Implica un elemento de valoración convencional. Por tanto, es esencialmente normativa. No tiene cabida en la ciencia pura. Por lo tanto, dentro de su propia estructura de generalizaciones, no proporciona normas vinculantes en la práctica, es incapaz de decidir sobre la conveniencia de diferentes fines y es fundamentalmente distinta de la Ética. \\

 Para ser completamente racionales, debemos saber qué es lo que preferimos, debemos ser conscientes de las implicaciones alternativas Pues la racionalidad en la elección no es ni más ni menos que la elección con plena conciencia de las alternativas rechazadas. Y es justo aquí donde la Economía adquiere su significado práctico. Puede aclararnos las implicaciones de los distintos fines que podemos elegir. Nos permite querer sabiendo qué es lo que queremos. Nos permite elegir un sistema de fines que sean coherentes entre sí. Un ejemplo lo aclarará, No es racional querer un fin determinado si no se es consciente del sacrificio que implica la consecución de ese fin. Y, en esta suprema ponderación de alternativas, sólo una conciencia completa de las implicaciones del análisis económico moderno puede conferir la capacidad de juzgar racionalmente.

 Es muy posible que en la sociedad moderna existan diferencias en cuanto a los fines últimos que hagan inevitables algunos conflictos. Pero está claro que muchas de nuestras dificultades más acuciantes surgen, no por esta razón, sino porque nuestros objetivos no están coordinados. Como consumidores queremos lo barato, como productores elegimos la seguridad. Valoramos una distribución de los factores de producción como gastadores y ahorradores privados. Como ciudadanos públicos sancionamos los acuerdos que frustran la consecución de esta distribución. Pedimos dinero barato y precios más bajos, menos importaciones y un mayor volumen de comercio. Las diferentes 'organizaciones de voluntad' de la sociedad, aunque compuestas por los mismos individuos, formulan preferencias diferentes. A tal situación, la economía nos permite armonizar nuestras distintas opciones. No puede eliminar las limitaciones últimas de la acción humana. Pero sí hace posible que, dentro de esas limitaciones, actuemos con coherencia. Sirve al habitante del mundo moderno, con sus infinitas interconexiones y relaciones, como una extensión de su aparato perceptivo. Proporciona una técnica de acción racional. Así, podemos decir que la economía asume verdaderamente la reacionalidad de la sociedad humana. No pretende, como se ha afirmado tantas veces, que la acción sea necesariamente racional en el sentido de que los fines perseguidos no sean incompatibles entre sí. No se basa en el supuesto de que los individuos siempre actúen racionalmente. Pero sí depende para su razón de ser práctica de la suposición de que es deseable que lo hagan. Supone que, dentro de los límites de la necesidad, es deseable elegir fines que puedan alcanzarse armoniosamente. La afirmación de que la reacionalidad y la capacidad de elegir con conocimiento son deseables.  Si la irracionalidad, si la entrega a la fuerza ciega de los estímulos externos y al impulso descoordinado en cada momento es un bien que debe preferirse por encima de todos los demás, entonces es cierto que desaparece la razón de ser de la Economía. 

 \chapter{Economia y acción humana\\ Frank Knight}

 Si se abadona la explicación del comportamiento en términos de razones detrás de las decisiones económicas que toman los individuos. Se abren varias posibilidades alternativas. La más sencilla sea la analogía a la tendencia de la física: prescindir de toda 'explicación' y limitarse a formular leyes empíricas; el resultado es la teoría económica estadística, cuyo contenido se limita a los fenómenos objetivos de las mercancías y los precios.

 Otro podría ser el control social de la vida económica, cómo el socialismo de cátedra en Alemania o el fabianismo y liberalismo de izquierdas en Inglaterra o la fase de la economía institucionalista en Estados Unidos.\\

 Una tercera alternativa a la teoría explicativa es la de tratar los fenómenos económicos como esencialmente históricos. La economía histórica se destacan en:

 \begin{enumerate}[1]
     \item La primera, trata de la historia, en la medida de lo posible, en términos objetivos y empíricos, y puede utilizarse la estadística para descubrir y analizar tendencias. Análoga a las ciencias naturales.
     \item La segunda, utiliza las ideas humanísticas más familiares de la historia política y social: Ambición, esfuerzo y fracasos individuales en un entorno sociopsicológico determinado.
 \end{enumerate}

 A medida que la economía histórica llega a la generalización, puede describirse como economía institucional. En Alemania se lo denomina economía neohistórica o sociológica, con Sombart y Max Weber como sus lidere más destacados.\\

En la raíz de las diferencias y disputas entre la vieja y la nueva economía, así como entre las tres nuevas líneas de desarrollo teórico señaladas anteriormente, se encuentran dos problemas:

 \begin{itemize}
     \item  la relación entre descripción y explicación y 
     \item la relación entre exposición de hechos y evaluación crítica.
 \end{itemize}

 El primero, ineludible en cualquier pensamiento sobre la conducta humana, es fundamentalmente el problema de la realidad de la elección, o 'libertad de la voluntad'. Implica la esencia del problema del valor en el sentido de los valores individuales, y es en el fondo el problema de la relación entre el hombre individual y la naturaleza. El segundo problema básico tiene que ver con la relación entre el hombre individual y la sociedad. El hecho crucial en relación con el primer problema es que, si al motivo o al fin en cualquiera de sus formas se le concede algún papel real en la conducta, no puede ser el de una causa en el sentido de causalidad de la ciencia natural. Esta es la limitación suprema tanto de la economía estadística como de la histórica. En efecto, si se utiliza un motivo o un fin para explicar un comportamiento, éste debe, a su vez, ponerse en relación con los acontecimientos y las condiciones que lo preceden, y entonces el motivo resulta innecesario; el comportamiento se explicará plenamente por estos antecedentes. El motivo no puede tratarse como un acontecimiento natural. El contraste fundamental entre la causa y el efecto en la naturaleza y el fin y los medios en el comportamiento humano es la esencia de los hechos que plantean el problema de la interpretación del comportamiento. No parece posible hacer realidad los problemas humanos sin ver en la actividad humana un elemento de esfuerzo, contingencia y, lo que es más importante, de error, que por las mismas razones debe suponerse ausente de los procesos naturales.\\

 Así pues, el motivo o la intención se imponen en cualquier debate relevante sobre la actividad humana. Pero el tema del comportamiento no puede simplificarse hasta el punto de reducirlo a un dualismo. Es preciso introducir al menos tres principios básicos en su interpretación. 
\begin{itemize}
    \item La acción humana típica se explica en parte por la causalidad natural, 
    \item en parte, por una intención o deseo que es un dato absoluto y por tanto un hecho aunque no sea un acontencimiento o condición natural,
    \item y en parte por un impulso a realizar valores que no puede reducirse enteramente a deseos fácticos porque este impulso no tiene objetos literlmente descriptibles.
\end{itemize}

El segundo principio de explicación es quizás el más vulnerable de los tres. Es decir, es dudoso que un deseo sea absoluto, que exista algún deseo que no busque el logro de algún cambio en un sistema creciente de significado y valores. Todo acto, en el sentido económico, cambia la configuración de la materia en el espacio. Pero esto no excluye la posibilidad de 'actos' que cambien el significado y los valores sin cambiar la configuración natural, ya que la reflexión puede producir una nueva percepción y efectuar un cambio en los gustos personales. Más fundamentalmente, es dudoso que una configuración sea en sí misma preferible a otra.\\

Las personas manifiestan y sienten dos tipos diferentes de motivación para sus actos, 
\begin{itemize}
    \item Por un lado, el deseo o la preferencia, que el actor y las personas ajenas a él tratan como algo definitivo, como un hecho bruto.
    \item Por otro lado, las personas emiten juicios de valor de diversa índole para explicar sus actos, y la explicación desemboca en la justificación.
\end{itemize}
En otras palabras nadie puede tratar el motivo de forma objetiva o describir un motivo sin implicaciones de bueno y malo. Así, los hombres no sólo desean más o menos distinto de valorar, sino que desean porque valoran y también valoran sin desear. De hecho, la mayor parte de las valoraciones humanas, en relación con la verdad, la belleza y la moral, son en gran medida o totalmente independientes del deseo de cualquier cosa o resultado concreto. El hecho de que la motivación económica individual implique en sí misma una valoración y no un mero deseo queda establecido por otras dos consideraciones:
\begin{itemize}
    \item En primer lugar, lo que se elige en una transacción económica se quiere generalmente como medio para otro cosa, lo que implica un juicio de que realmente es un medio para el resultado en cuestión. 
    \item En segundo lugar, lo que se quiere en última instancia por sí mismo rara vez, o nunca, puede describirse finalmente en términos de configuración física, sino que debe definirse en relación con un universo de significados y valores. 
\end{itemize}
Así pues, hay un elemento de valoración en la noción de eficacia en la realización de un fin determinado; y, además, el fin real contiene como elemento un concepto de valor.\\

La concepción dual que se encuentra en la motivación se refleja también en el concepto más estrictamente económico de valor. Este último contiene definitivamente algo más que la noción de una cualidad medida por el precio; siempre se mide imperfectamente en condiciones reales. El precio 'tiende' a coincidir con el valor, pero la noción de valor también implica una norma a la que se ajustaría el precio en unas condiciones ideales. Esta normalmente incluye dos ideas:
\begin{itemize}
    \item La de un objetivo al que se aspira pero que sólo se alcanza de forma más o menos aproximada debido a errores de diversa índole;
    \item y la de un objetivo de acción correcto en contraste con los objetivos incorrectos, así como con el objetivo real.
\end{itemize}
En una sociedad basada en la competencia como principio aceptado, el precio competitivo, o precio igual a los costes necesarios de producción, es el valor verdadero en ambos sentidos;  las aberraciones deben atribuirse a dos grupos de casos:
\begin{itemize}
    \item errores de cálculo accidentales y,
    \item objetivos de acción incorrectos.
\end{itemize}

Para aclarar el punto principal es necesario observar la diferencia en la concepción de las condiciones ideales en economía y en mecánica. En este último campo, la más notable de las condiciones ideales es la ausencia de fricción; una concepción aparentemente similar de las condiciones ideales es una de las características familiares, casi un cliché, de la teoría económica. Como descripción generalizada, la concepción de la competencia perfecta, a la que se llega por abstracción de las características de la situación económica que hacen que la competencia sea imperfecta, es como las concepciones de la mecánica sin fricción y está igualmente justificada. Pero suponer que la abstracción de la teoría de la competencia perfecta tiene la misma relación con el comportamiento que la fricción con el proceso mecánico sería totalmente engañoso. La fricción en mecánica implica una transformación de energía de una forma a otra, de acuerdo con una ley tan rígida y un principio de conservación tan definido como la ley y el principio de conservación que son válidos para los cambios mecánicos en los que no desaparece la energía. No hay nada de esto en el proceso económico. Lo que se abstrae en la teoría de los precios de equilibrio es el hecho del error en el comportamiento económico.  No puede tratarse como una tendencia hacia un resultado objetivo, sino sólo como una tendencia a la conformidad con la intención del comportamiento, intención que no puede medirse ni identificarse ni definirse en términos de ningún dato experimental. Las condiciones ideales de la economía implican una valoración perfecta en un sentido limitado, un comportamiento económico perfecto que asume el fin o la intención como dados.\\

Hasta ahora se han examinado dos niveles de interpretación del comportamiento económico. 
\begin{itemize}
    \item El primero es aquel en el que el comportamiento se reduce en la medida de lo posible a principios de regularidad mediante un procedimiento estadístico, 
    \item el segundo es la interpretación del comportamiento en términos de motivación, que debe centrarse en la diferencia entre motivo y acto y en el hecho del error. 
    \item Es en el tercer nivel de interpretación donde el fin intencional de la acción en sí se somete a valoración o crítica desde algún punto de vista.
\end{itemize}
Aquí la relación entre individuo y sociedad, el segundo problema principal sugerido anteriormente, y el concepto de valor en relación con la política social se convierten en temas centrales de discusión. De hecho, incluso en el segundo nivel deben reconocerse dos formas de referencia social:
\begin{itemize}
    \item Los fines individuales, tal como se dan, son principalmente sociales en su origen y contenido, y
    \item en las sociedades en las que el pensamiento económico tiene alguna relevancia existe una gran aceptación y aprobación ético-social de la motivación individual en abstracto.
\end{itemize}
La sociedad moderna, por ejemplo, obtiene la libertad individual siendo un valor social y no un mero hecho. Si la noción de comportamiento económico se separa efectivamente del proceso mecánico, si los fines se consideran como fines y no como meros efectos físicos, la discusión se sitúa ya en gran parte en el tercer nivel. Los fines físicos como deseados no pueden mantenerse a menos que se les otorgue un gran elemento de valoración además del deseo. No se puede sostener que los "deseos" de bienes y servicios económicos sean definitivos o tengan una realidad autónoma e independiente. El menor escrutinio muestra que son, en gran medida, manifestaciones más bien accidentales del deseo de algo de la naturaleza de la libertad o el poder. Pero tales objetos de deseo son formas de relación social y no cosas, y la noción de eficiencia económica sólo tiene una aplicabilidad limitada a su búsqueda y consecución. El tratamiento de tales actividades, si ha de tener algún atractivo general y serio, debe ser una discusión de política social relativa a los fines o normas sociales y al procedimiento social para realizarlos.\\

En este campo predomina el interés por los valores y, sobre todo, por la política social. Así, la teoría económica, que creció en un ambiente de reacción contra el control, puso claramente demasiado énfasis en este lado del asunto y descuidó el otro. Ahora es igual de obvio que existen limitaciones igualmente amplias y complejas al principio de libertad en el sentido económico, es decir, a la organización de la vida económica exclusivamente a través del libre contrato entre individuos que utilizan determinados recursos para alcanzar determinados fines individuales. La sociedad no puede aceptar los fines individuales y los medios individuales como datos o como objetivos principales de su propia política. En primer lugar, simplemente no son datos, sino que se crean históricamente en el propio proceso social y se ven inevitablemente afectados por la política social. En segundo lugar, la sociedad no puede ser ni siquiera relativamente indiferente al funcionamiento del proceso. Hacerlo sería, en última instancia, destructivo tanto para la sociedad como para el individuo. Esta conclusión se ve fuertemente reforzada por el hecho de que el interés inmediato del individuo es en gran medida competitivo, centrado en su propio avance social en relación con otros individuos.\\

Estas reflexiones apuntan a un error lógico subyacente en la teoría del valor típica de los economistas clásicos.  No sostenían ostensiblemente que la libertad como tal fuera un bien. Notoriamente, eran hedonistas; su argumento a favor de la libertad la hacía instrumental para el placer, basándose en que el individuo es mejor juez que los funcionarios del gobierno sobre los medios para su felicidad. Ciertamente, un individuo puede desear la libertad y reclamar el derecho a ella sin sostener que tomará decisiones uniformemente más sabias que las que se tomarían por él, desde el punto de vista de su propia comodidad y seguridad material. Y con la misma certeza se puede sostener que el individuo debe, dentro de unos límites, tomar sus propias decisiones y atenerse a sus consecuencias, aunque no decida hacerlo. En otras palabras, los economistas clásicos no se dieron cuenta, y el espíritu 'científico' de la época ha hecho que los economistas en general se resistan a admitir que la libertad es esencialmente un valor social, al menos cuando se defiende o se opone a ella, como lo es cualquier otro sistema social o relación social.\\

Los intereses o deseos reales expresados en el comportamiento económico son, en su inmensa mayoría, sociales en su génesis y en su contenido; por consiguiente, no pueden describirse al margen de un sistema de relaciones sociales que, a su vez, no puede tratarse en términos puramente objetivos y reales. Hasta cierto punto, un individuo puede concebirlos en tales términos. Pero las partes de una comunicación de este tipo se sitúan en el papel de espectadores más que en el de miembros de la sociedad o participantes en el fenómeno. \\
En este conflicto entre el interés del espectador por ver y comprender y el interés del participante por actuar y cambiar, el filósofo o el metodólogo no pueden tomar partido. A la pregunta de si la economía como tal debe ser una cosa o la otra sólo se puede responder reconociendo que debe ser ambas cosas, con más o menos énfasis en un sentido o en otro según los objetivos de un tratamiento concreto; pero siempre, implícitamente, debe ser ambas cosas, por muy unilateral que sea el énfasis, ya que cada interés presupone y es relativo al otro, y todo escritor y lector, como ser humano, está motivado por ambos intereses. Lo deseable es que en cualquier declaración quede clara la relación entre los dos grupos de intereses. Pero lo que tiende a ocurrir es lo contrario: aquel cuyo interés es principalmente la verdad tiende a reforzar sus afirmaciones identificando verdad y valor, y aquel cuyo interés son los valores tiende a reforzar sus afirmaciones dándoles la cualidad de verdad.\\

Mientras que en el período de desarrollo de la economía clásica el interés práctico social se centraba casi exclusivamente en la liberación de un sistema anticuado de control, en la actualidad el péndulo ha oscilado definitivamente en sentido contrario. La sociedad busca positivamente una base de unidad y orden en lugar de intentar negativamente abandonar una base insatisfactoria. Además, las normas actuales de pensamiento han caído bajo el dominio extremo del ideal científico, que tiene poca o ninguna aplicabilidad al problema. No hay solución intelectual para los conflictos de intereses. Sólo pueden discutirse los valores, pero la discusión no conduce necesariamente a un acuerdo; y el desacuerdo sobre los principios parece requerir moralmente una apelación a la fuerza. También es interesante señalar que la tendencia a la "racionalización" hace que el conflicto de intereses y el desacuerdo sobre los principios adquieran cada uno la cualidad de su opuesto, y que en la práctica se mezclen inseparablemente.\\

Tanto las escuelas 'fascistas' como las 'comunistas' se inclinan a tratar la verdad o falsedad de las proposiciones en economía como una cuestión indiferente o incluso ilusoria, juzgando las doctrinas sólo por su conductividad hacia el establecimiento del tipo de orden social deseado. Este punto de vista es, por supuesto, "falso" desde un punto de vista "científico" más restringido; en cualquier orden social, los resultados de ciertas elecciones que afectan a la producción y al consumo, las haga quien las haga, se rigen por ciertos principios abstractos, esencialmente matemáticos, que expresan la diferencia entre economía y despilfarro. \\

En el otro extremo -en el primer y segundo nivel de interpretación indicados anteriormente- existe un movimiento igualmente enérgico en aras de un tratamiento rigurosamente 'científico' de la economía. El análisis en el primer nivel, que prescinde de la motivación y sólo considera los resultados de la acción en forma de estadísticas de mercancías, no deja lugar real a ningún concepto de economía. Además, no puede llevarse a cabo ni siquiera literalmente, ya que las mercancías deben nombrarse y clasificarse y el tratamiento debe tener en cuenta las similitudes y diferencias de uso, así como las características físicas. Y la economía en el segundo nivel, tratando los deseos como hechos, está sujeta a limitaciones muy estrechas. En realidad, los deseos no tienen un contenido muy definido, y de lo que tienen los estudiantes no pueden tener un conocimiento definitivo. La concepción puede convertirse en la base de una teoría puramente abstracta, pero tiene poca aplicación a la realidad. Para dar contenido a los datos, los deseos deben identificarse con los bienes y servicios en los que se expresan, y el segundo método se reduce entonces a la identidad con el primero. Además, los únicos deseos que pueden asimilarse a los datos científicos son puramente individuales, y cualquier debate sobre política social debe basarse en valores o ideales totalmente ajenos a este sistema.

\chapter{Textos selectos sobre economía, historia y Ciencias Sociales\\ Karl Marx}

\end{document}
