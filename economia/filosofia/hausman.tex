\documentclass[10pt]{book}
\usepackage[text=17cm,left=2.5cm,right=2.5cm, headsep=20pt, top=2.5cm, bottom = 2cm,letterpaper,showframe = false]{geometry} %configuración página
\usepackage{latexsym,amsmath,amssymb,amsfonts} %(símbolos de la AMS).7
\parindent = 0cm  %sangria
\usepackage[T1]{fontenc} %acentos en español
\usepackage[spanish]{babel} %español capitulos y secciones
\usepackage{graphicx} %gráficos y figuras.

%---------------FORMATO de letra--------------------%

\usepackage{lmodern} % tipos de letras
\usepackage{titlesec} %formato de títulos
\usepackage[backref=page]{hyperref} %hipervinculos
\usepackage{multicol} %columnas
\usepackage{tcolorbox, empheq} %cajas
\usepackage{enumerate} %indice enumerado
\usepackage{marginnote}%notas en el margen
\tcbuselibrary{skins,breakable,listings,theorems}
\usepackage[Bjornstrup]{fncychap}%diseño de portada de capitulos
\usepackage[all]{xy}%flechas
\counterwithout{footnote}{chapter}
\usepackage{xcolor}
\usepackage[htt]{hyphenat}
%--------------------GRÀFICOS--------------------------

\usepackage{tkz-fct}

%---------------------------------

\titleformat*{\section}{\bfseries\rmfamily}
\titleformat*{\subsection}{\bfseries\rmfamily}
\titleformat*{\subsubsection}{\bfseries\rmfamily}
\titleformat*{\paragraph}{\bfseries\rmfamily}
\titleformat*{\subparagraph}{\bfseries\rmfamily}

%------------------------------------------

\renewcommand{\labelenumi}{\Roman{enumi}.}%primer piso II) enumerate
\renewcommand{\labelenumii}{\arabic{enumii}$)$}%segundo piso 2)
\renewcommand{\labelenumiii}{\alph{enumiii}$)$}%tercer piso a)
\renewcommand{\labelenumiv}{$\bullet$}%cuarto piso (punto)

%----------Formato título de capítulos-------------

\usepackage{titlesec}
\renewcommand{\thechapter}{\arabic{chapter}}
\titleformat{\chapter}[display]
{\titlerule[2pt]
\vspace{4ex}\bfseries\sffamily\huge}
{\filleft\Huge\thechapter}
{2ex}
{\filleft}

\begin{document}

\normalfont
\input xy
\xyoption{all}
\author{\Large Apuntes por FODE}
\title{\small Daniel M. Hausman \\ \vspace{1cm} \large La filosofía de la economía: Una ontología}
\date{}
\pagestyle{empty}
\maketitle
\thispagestyle{empty}
\let\cleardoublepage\clearpage
\tableofcontents								%indice





\chapter*{Introducción}

\section{Una introducción a la filosofía de la ciencia}
Los temas de los que se ha ocupado la filosofía de la ciencia que son más relevantes para la economía se pueden dividir en cinco grupos:

\begin{enumerate}[1.]
    \item \textit{Objetivos} ¿Cuáles son los objetivos de la ciencia y de la teoría científica? ¿Es la ciencia principalmente una actividad práctica que apunta a descubrir generalizaciones útiles, o debería la ciencia buscar explicaciones y verdades?.
    \item \textit{Explicación} ¿Qué es una explicación científica?.
    \item \textit{Teorías} ¿Qué son las teorías, modelos y leyes? ¿Cómo se relacionan entre sí? ¿Cómo se descubren o construyen?.
    \item \textit{Prueba, inducción y demarcación} ¿Cómo se prueban y confirman o refutan las teorías, modelos y leyes científicas? ¿Cuáles son las diferencias entre las actitudes y prácticas de los científicos y las de los miembros de otras disciplinas?.
    \item ¿Las respuestas a estas cuatro preguntas son las mismas para todas las ciencias en todo momento? ¿Se pueden estudiar las acciones e instituciones humanas de la misma manera que se estudia la naturaleza?.
\end{enumerate}

\subsection{Los objetivos de la ciencia}
Hay dos escuelas principales de pensamiento

\begin{enumerate}[1.]
    \item Los realistas científicos.- la ciencia debe descubrir verdades sobre el mundo y explicar los fenómenos. (Ven a las teorías como aproximaciones de la realidad).
    \item Los antirrealistas que puede ser instrumentalistas que consideran que los objetivos de la ciencia son exclusivamente prácticos, o los antirrealistas pueden estar en desacuerdo con los realistas principalmente sobre si existen los no observables postulados para las teorías científicas, si las afirmaciones sobre ellos son verdaderas o falsas y si la evidencia observable puede establecer afirmaciones sobre inobservables. (Ven a las teorías como herramientas)
\end{enumerate}

Los antirrealistas están de acuerdo que las teorías son importantes, pero ubican su importancia exclusivamente a su papel de ayudar a las personas a anticipar y controlar los fenómenos. Puede verse en "La metodología de la Economía positiva" de Milton Friedman.


\subsection{Explicación científica}
Las explicaciones responden a la pregunta:
\begin{center}
    ¿Por qué?
\end{center}

Carl Hempel, desarrolla dos modelos principales de explicación científica:

\begin{itemize}
    \item Deductivo-nomológico.- Un enunciado de lo que se va a explicar se deduce de un conjunto de enunciados verdaderos que incluye esencialmente al menos una ley.
    \item Inductivo-estadístico.- Este último, como sugiere su nombre, se ocupa de las explicaciones probabilísticas e intenta extender la intuición básica del modelo deductivo-nomológico (D-N).
\end{itemize}

El modelo D-N es una explicación determinista o no estadística. Si sólo se dispone de una regularidad estadística, no se podrá deducir lo que se quiere explicar, pero se podrá demostrar que es altamente probable, que es lo que exige el modelo inductivo-estadístico de Hempel. \\
Parece que las explicaciones de los acontecimientos y estados de cosas suelen citar sus causar, pero existe dos problemas.
\begin{enumerate}
    \item Primero, aunque la mayoría de las explicaciones de eventos y estados de cosas son explicaciones causales, no todas lo son. 
    \item En segundo lugar, decir que las explicaciones citan causas no es en sí mismo muy informativo. Sin una teoría de la causalidad, una teoría causal de la explicación está vacía, e incluso con una teoría de la causalidad, solo araña la superficie para sostener que explicar es citar una causa. 
\end{enumerate}

\subsection{Teorías científicas y leyes}
La mayoría de los filósofos han argumentado que la ciencia procede mediante el descubrimiento de teorías y leyes, pero los economistas se sienten más cómodos hablando de modelos que de leyes y teorías.\\

Las leyes de las ciencias no son, por supuesto, leyes prescriptivas que dictan cómo deberían ser las cosas. Las leyes científicas son, en cambio, (hablando en términos generales) regularidades en la naturaleza. Pero quizás el concepto de ley no sea útil para aquellos interesados en la metodología económica.\\

De acuerdo con el modelo de explicación deductivo-nomológico, los economistas pueden usar generalizaciones como la ley de la demanda para explicar fenómenos económicos solo si esas generalizaciones son genuinamente leyes.\\

Los positivistas lógicos precisaron la noción de “trabajar juntos”, argumentando que las teorías forman sistemas deductivos. Según los positivistas, las teorías son principalmente objetos “sintácticos”, cuyos términos y afirmaciones se interpretan por medio de reglas de “correspondencia”. Estos esperaban expresar teorías científicas en lenguajes formales, pero los positivistas se dieron cuenta de que la relación entre teoría y observación es más compleja.\\

Por tal razón se usan modelos, las cuales se manipulan, exploran y modifican, afirmando que los predicados (se afirma algo como: es un sistema de consumo de dos productos básicos) construyen o determinan que son verdaderos o falsos para los sistemas de cosas del mundo. Ahora bien, por qué los filósofos serios defienden la visión predicada de los modelos?.
Existen dos razones

\begin{itemize}
    \item En primer lugar, si uno espera ser capaz de reconstruir las afirmaciones de la ciencia formalmente, el punto de vista del predicado tiene importantes ventajas técnicas.
    \item En segundo lugar, el punto de vista del predicado ofrece una forma útil de esquematizar los dos tipos de logros que implica la construcción de una teoría científica. (Modelo de Hempel).
\end{itemize}

Una parte absolutamente crucial del quehacer científico es la construcción de nuevos conceptos, de nuevas formas de clasificar los fenómenos, que es común en los economistas.

\subsection{Evaluación y Demarcación}

La opinión kantiana de que hay verdades sintéticas a priori (que pueden ser demostrados puramente por razonamiento lógico y deductivo, sin necesidad de recurrir a la experiencia empírica) todavía tiene partidarios entre los llamados economistas austriacos, especialmente Ludwig von Mises y sus seguidores. Argumentan que los postulados fundamentales de la economía son verdades sintéticas a priori. 

Hume, lanza un desafío: muéstrame un buen argumento cuya conclusión sea alguna generalización o alguna afirmación sobre algo no observado y cuyas premisas incluyan solo informes de experiencias sensoriales. Tal argumento no puede ser un argumento deductivo, porque tales inferencias son falibles. Tampoco lo hará un argumento “inductivo”, ya que solo tenemos motivos inductivos y, por lo tanto, que plantean dudas para creer que tales argumentos son buenos.\\

El problema de la demarcación se refiere a la distinción entre teorías científicas y otros tipos de teorías.\\

De lo que debería ocuparse la filosofía de la ciencia, según Lakatos, no son las reglas para evaluar teorías, sino las reglas para modificar y comparar teorías.

\subsection{La unidad de la ciencia}
Uno se pregunta si, dado el libre albedrío, el comportamiento humano es intrínsecamente impredecible y, por lo tanto, no está sujeto a ninguna ley. Tan tentador como esta línea de pensamiento puede ser, es un error. Incluso si no hay leyes deterministas del comportamiento humano, hay, de hecho, muchas regularidades en la acción humana.\\

Porque las personas pueden, como señala Frank Knight, cometer errores o no reconocer las cosas. Como primera aproximación, los economistas se abstraen de tales dificultades. Suponen que las personas tienen información perfecta. Al suponer que la gente cree cualesquiera que sean los hechos, los economistas pueden evitar preocuparse por lo que la gente realmente cree.\\

Un punto de vista es que la economía sirve a la política de la misma manera que las ciencias naturales guían las políticas, es decir, ayudando a los formuladores de políticas a elegir los medios que lograrán sus fines. Tal papel práctico para el conocimiento científico no parece problemático. Los agentes tienen algún objetivo que quieren lograr, y el científico proporciona el "saber hacer" necesario. Desde este punto de vista, la economía importa a la política sólo como una fuente de información descriptiva o “libre de valores”.

\section{Una introducción a la economía}
La economía comienza en el siglo XVIII con los escritos de los fisiócratas franceses, de Cantillon y Hume, y especialmente de Adam Smith. Lo que diferenció a estos pensadores de sus predecesores fue su creciente reconocimiento de la existencia de mecanismos mediante los cuales las acciones individuales tendrían consecuencias sistemáticas sin necesidad de que el gobierno controlara los procesos. Smith y otros llegaron a ver la economía en gran medida como un sistema autorregulador. La economía surgió cuando se comprendió que había mecanismos y sistemas económicos para estudiar. \\

La economía se ha preocupado principalmente por comprender cómo funciona un sistema económico capitalista. (Un sistema económico capitalista es una economía de mercado en la que los medios de producción son en su mayor parte de propiedad privada, y los trabajadores son libres de aceptar o rechazar ofertas de empleo). Muchos economistas creen que sus teorías también se aplican a otros arreglos económicos, y se ha trabajado mucho en otros tipos de economías. Pero el núcleo de la teoría económica se ha dedicado a comprender las economías capitalistas.\\

 Los economistas “clásicos”, de los cuales Adam Smith, David Ricardo y John Stuart Mill son los más destacados, no tenían mucho que decir sobre las elecciones de los consumidores. Su énfasis estaba en la producción y en los factores que influyen en la oferta de bienes de consumo. Consideraron que los agentes buscaban maximizar sus ganancias financieras y dividieron tanto a los agentes como a los insumos básicos en tres clases principales:
\begin{itemize}
    \item Capitalistas con su capital (que concibieron como existencias de bienes acumulados o el valor de los mismos), 
    \item terratenientes con su tierra y
    \item trabajadores con su capital.
\end{itemize}
Estos ofrecieron dos generalizaciones principales con respecto a la producción
\begin{itemize}
    \item Asumieron que en cualquier momento dado todos los bienes reproducibles (excluyendo así cosas como la pintura) podrían producirse en cualquier cantidad por el mismo costo por unidad.
    \item Luego, descubrieron los rendimientos decreciente. A menos que haya alguna innovación tecnológica, a medida que se dedica más y más trabajo a una cantidad fija de tierra la cantidad que aumenta la producción cuando se emplea un trabajador adicional eventualmente disminuirá.
\end{itemize}

A diferencia de Ricardo, a finales del siglo XIX, los economistas reconocieron que la población no tenía por qué crecer en respuesta a los salarios más altos. Además, las mejores tecnológicas provocaron aumentos en la productividad. Luego llegaron los neo-clásicos o marginales, que centraron su atención en la elección y el intercambio individuales, mediante las preferencias del consumidor, el intercambio y la demanda de mercancías.\\


Muchos de los primeros economistas neo-clásicos fueron influenciados por el utilitarismo, una teoría ética expuesta por Jeremy Bentham y John Stuart Mill. Según los utilitaristas, las preguntas de política social deben responderse calculando las consecuencias de las alternativas para el felicidad total de los individuos. La política que maximiza la suma de las utilidades individuales es la moralmente correcta. Bentham sostuvo que la utilidad de algo para un individuo es una sensación que, en principio, podría cuantificarse y medirse. También creía que los individuos actúan para maximizar su propia utilidad. Jevons desarrolló la noción esencialmente benthamita de una función de utilidad. Cada opción abierta a un resultado individual en una cierta cantidad de utilidad para esa persona. Entonces se puede aclarar la noción de racionalidad manteniendo que las personas actúan para maximizar alguna función de utilidad consistente. Además, los economistas neoclásicos asumieron que los consumidores generalmente no están satisfechos, que siempre preferirán un paquete $x$ de bienes o servicios a otro paquete y si $x$ es inequívocamente mayor que $y$. La insaciabilidad es tanto una primera aproximación plausible como articula la noción de interés propio. Todo lo que importa a los agentes son los paquetes de mercancías y servicios que están entregando o recibiendo. \\

Con la adición de una generalización más, se tiene el núcleo de la teoría económica moderna. Los primeros economistas neo-clásicos notaron que a medida que uno consume más de cualquier producto o servicio, cada unidad adicional aumenta la utilidad de uno a una tasa decreciente, llamada ley de utilidad marginal decreciente. Esta teoría se ha refinado enormemente, donde "utilidad" es solo otra forma de hablar sobre preferencias. En términos generales, uno puede reemplazar la "ley" de la utilidad marginal decreciente con la generalización de que las personas están dispuestas a pagar menos por unidades adicionales de productos básicos que ya tienen mucho que por productos básicos de los que tiene muy poco. A pesar de estos refinamientos la teoría dominante sigue siendo reconociblemente la teoría desarrollada por los primeros economistas neoclásicos.\\

Luego de los años 30, Keynes desafió la teoría dominante, enfatizando la importancia de la liquidez tanto para las empresa como para los individuos cuando se enfrentan a las incertidumbres. Los precios, especialmente los salarios, no caen fácilmente, y los salarios más bajos pueden generar menos gastos, lo que suprimiría la demanda de productos básicos y conduciría a una caída aún más profunda. Para ello Keynes mencionó que el estado podría aumentar la demanda agregada de de materias primas y alentar la inversión y de esa manera sacar la economía del desempleo. De todas formas, Keynes no sacudió los fundamentos de la teoría neo-clásica. Pero las versiones actualizadas de la economía Keynesiana, siguen siendo influyentes. Siendo así, la macroeconomía un área inestable de la economía.\\

La econometría es una rama de la estadística aplicada, así como una rama de la economía. A partir de la década de 1930, se esperaba que las afirmaciones de los teóricos económicos pudieran probarse y refinarse con la ayuda de técnicas estadísticas. Desde entonces, las técnicas econométricas se han vuelto mucho más sofisticadas. Exactamente lo que este trabajo significa para la teoría económica (a diferencia de las investigaciones prácticas de enfoque limitado) es controvertido, y algunos economistas prominentes argumentan que la econometría es incapaz de proporcionar buenas razones para creer o no creer en afirmaciones causales significativas.

Por otro lado, de acuerdo con el materialismo histórico de Marx, las relaciones entre las personas en el curso de sus actividades productivas son las relaciones sociales más fundamentales. Marx, considera al capitalismo, a pesar de las miserias que puede causar un enorme adelanto para el ser humano. Pero, como argumentos en su ensayo inicial "trabajo enajenado", no permite que las personas decidan racionalmente y conscientemente, como se debe desarrollar la sociedad y la naturaleza humana. Por lo que Marx cree que las personas pueden trascender el capitalismo y organizar la producción y la distribución de alguna manera racional. Pero no tuvo existe, ya que las teorías económicas no se ciernen sobre las olas políticas. \\

Luego apareció la economía institucionalista, que no ignoran las tomas de decisiones individuales, pero enfatizan las restricciones en evolución sobre los agentes que ocupan roles económicos específicos.\\

La tercera alternativa contemporánea a la economía dominante, es la economía del comportamiento, incluida la neuroeconomía. \\

También se encuentran los neorricardianos, que creen que se puede comprender mejor la economía reformulando ecuaciones matemáticas. Los economistas Austriacos que están de acuerdo con los economistas neo-clásicos pero enfatizan la importancia de la incertidumbre, desequilibrio y un punto de vista subjetivo. Los economistas postkeynesianos a menudo ofrecen críticas similares en la alta teoría, pero a diferencia de los austriacos, tienden a defender las políticas intervencionalistas.  Los pronosticadores económicos a menudo dependen muy poco de cualquier teoría específica. 


\section{Una introducción a la metodología de la economía}

Knight y los austriacos están de acuerdo en que tan pronto como se abandona el punto de vista subjetivo y se piensa en la economía como si fuera una ciencia natural, se pierden de vista los rasgos centrales del tema. Luego Hutchison argumenta que la economía, como otras ciencias, debe formular generalizaciones comprobadas y someterlas a pruebas serias. Como también Milton Friedman intenta mostrar que la economía satisface estándares ampliamente positivistas, que se convirtió en el trabajo más influyente sobre metodología económica del siglo XX.


\part{Discusiones clásicas}

\chapter{Sobre la definición y el método de la economía política\\ John Stuart Mill}

Mill fue uno de los primeros defensores de los derechos de la mujer y de un socialismo democrático moderado.\\

La economía política, no trata de toda la naturaleza del hombre modificada por el estado social, ni de toda la conducta del hombre en sociedad. Si no se refiere a él únicamente y exclusivamente como un ser que desea poseer riquezas y que es capaz de juzgar la eficacia comparativa de los medios para obtener ese fin. Bajo esa influencia de deseo, muestrea a la humanidad acumulando riqueza y empleando esa riqueza en la producción de otra riqueza, sancionando de común acuerdo la institución de la propiedad, estableciendo leyes para evitar que los individuos invadan las propiedades de otros por la fuerza o el fraude, adoptando diversos artilugios para aumentar la productividad de su trabajo, resolviendo la división del producto por acuerdo, bajo la influencia de la competencia. Todo esto y más para facilitar la producción. Ahora bien, cuando un efecto depende de una concurrencia de causas, esas causas deben estudiarse una a la vez, y sus leyes deben investigarse por separado, esto si deseamos a través de las causas, obtener el poder de predecir o controlar el efecto, ya que la ley del efecto se compone de las leyes de todas las causas que lo determinan. Para juzgar cómo el ser humano actuará bajo la variedad de deseos y aversiones que simultáneamente operan sobre él, debemos saber cómo actuaría bajo la influencia exclusiva de cada uno en particular. \\

El economista político, se pregunta cuáles son las acciones que producirían este deseo, si, dentro de los departamentos en cuestión, no estuviera impedido por ningún otro. Pero con respecto a aquellas partes de la conducta humana en las que la riqueza ni siquiera es el objeto principal, la Economía Política no pretende que sus conclusiones sean aplicables. Sólo en algunos casos llamativos, como el principio de población serán tomados en cuenta y deben ser corregidos teniendo debidamente en cuenta los efectos de cualquier impulso de una descripción diferente, que puede demostrarse que interfiere con el resultado en cualquier caso particular. Las conclusiones de la Economía Política dejarán de ser aplicables a la explicación o predicción de los acontecimientos reales, hasta que sean modificadas por una correcta asignación del grado de influencia ejercido por la otra causa.\\

Por lo tanto, la definición de economía política sera:

\begin{center}
    \textit{La ciencia que traza las leyes de los fenómenos de la sociedad que surgen de las operaciones combinadas de la humanidad para la producción de riqueza, en la medida en que esos fenómenos no son modificados por la búsqueda de cualquier otro objeto.}
\end{center}

Tengamos en cuenta que las diferencias de principio entre ciencias difieren no sólo en lo que creen ver, si no en el lugar de donde obtuvieron la luz con la que creen verlo. La más universal de las formas en que suele presentarse esta diferencia de método, es la antigua disputa entre lo que se llama:

\begin{itemize}
\item teoría y
\item práctica o experiencia. 
\end{itemize}

De todas formas las dos partes necesitarán de experiencia y teoría a diferencia que los que se hacen llamar prácticos, requerirán experiencia específica, y argumentarán totalmente hacia arriba desde hechos particulares hasta una conclusión general; mientras que aquellos que se llaman teóricos, pretenderán abarcar un campo más amplio de experiencia, y habiendo argumentado hacia arriba desde hechos particulares hasta un principio general que incluye un rango mucho más amplio que el de la cuestión en discusión, argumentaran hacia abajo desde ese principio general hasta una variedad de conclusiones específicas. Estos métodos  se pueden llamar: 
\begin{itemize}
    \item a posteriori.- entendemos el que requiere como base de sus conclusiones, no la mera experiencia, si no la experiencia concreta.
    \item a priori.- entendemos con el razonamiento a partir de una hipótesis supuesta.
\end{itemize}
Verificar, la hipótesis misma a posteriori, no forman parte en absoluto del quehacer científico, sino de la aplicación de la ciencia.\\

Con respecto a la economía política, lo hemos caracterizado como una ciencia abstracta y un método a priori. Ya que, supone una definición arbitraria de hombre, como un ser que invariablemente hace aquello por lo que puede obtener la mayor cantidad de necesidades, comodidades y lujos, con la menor cantidad de trabajo y abnegación física con la que puede obtenerse en el estado de conocimiento existente. Ahora, bien nadie que esté familiarizado con los tratados sistemáticos de economía política pondrá en duda que siempre que un economista ha demostrado que actuando de una manera determinada, un trabajador puede obtener salarios más altos, o un capitalista tener mayores beneficios. Esto ya que, el economista político razona a partir de premisas asumidas, que pueden carecer de fundamentos en los hechos. Por consiguiente, las conclusiones de la economía política, como de la geometría, solo son verdades en abstracto, es decir son verdaderas bajo ciertas suposiciones, en las que sólo se tienen en cuenta causas generales, causas comunes a toda la clase de casos considerados. El método a priori que se le imputa, como si su empleo demostrara que toda su ciencia carece de valor, es, como demostraremos más adelante, el único método por el cual puede alcanzarse la verdad en cualquier departamento de la ciencia social. Siempre que a medida en que los hechos reales se alejen de la hipótesis deban permitirse una desviación correspondiente de la letra estricta de su conclusión, de lo contrario sólo será verdad para las cosas que ha supuesto arbitrariamente, no para las cosas que existen. Cuando una determinada causa existe realmente, y si se la dejara sola produciría infaliblemente un determinado efecto, ese mismo efecto, modificado por todas las demás causas concurrentes, corresponderá correctamente al resultado realmente producido.\\

Luego, el método a posteriori será útil siempre que este en relación con el método a priori, ya que por si solo no podremos hacer experimentos morales o de ética. A estos experimentos se le denominan \textit{experimentos cruciales}. Por ejemplo, en asuntos de estado, ¿cómo podemos obtener un experimentos crucial sobre el efecto de una política comercial restrictiva sobre la riqueza nacional?. Tendremos que encontrar dos naciones iguales en todos los demás aspectos, o al menos que posean, en un grado exactamente igual, todo lo que conduce a la opulencia nacional, y que adopten exactamente la misma política en todos sus otros asuntos, pero que difieran en esto solamente, en que una de ellas adopta un sistema de restricciones comerciales, y la otra adopta el libre comercio. Este sería un experimento decisivo, similar a los que casi siempre podemos obtener en física experimental. Por lo tanto, ya sea en la política económica o en las ciencias sociales, mientras vemos los hechos concretos revestidos de complejidad, no queda otro método que el a priori o el de la especulación abstracta.\\

Ahora, las causas pueden ser de experimentación específica, que son las leyes de la naturaleza humana y las circunstancias externas capaces de excitar la voluntad humana a la acción, que están al alcance de nuestra observación. También podemos observar cuales son las causas que excitan estos deseos. Si al final la suposición es correcta hasta donde llega, y no difiere de la verdad más que como una parte difiere del todo, entonces las conclusiones que se deducen correctamente de la suposición constituyen una verdad abstracta; y cuando se completan añadiendo o sustrayendo el efecto de las circunstancias no calculadas, son verdaderas en lo concreto, y pueden aplicarse a la práctica. De este carácter es la ciencia de la economía política, para hacerla perfecta como ciencia abstracta, las combinaciones de circunstancias que asume, para rastrear sus efectos, deben incorporar todas las circunstancias que son comunes a todos los casos, y del mismo modo todas las circunstancias que son comunes a cualquier clase importante de casos. Las conclusiones correctamente deducidas de estas suposiciones serían tan verdaderas en abstracto como las de las matemáticas; y sería una aproximación lo más cercana posible a la verdad abstracta, a la verdad en lo concreto.

pag48 ultimo parrafo


\end{document}
