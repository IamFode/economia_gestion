\documentclass[10pt]{book}
\usepackage[text=17cm,left=2.5cm,right=2.5cm, headsep=20pt, top=2.5cm, bottom = 2cm,letterpaper,showframe = false]{geometry} %configuración página
\usepackage{latexsym,amsmath,amssymb,amsfonts} %(símbolos de la AMS).7
\parindent = 0cm  %sangria
\usepackage[T1]{fontenc} %acentos en español
\usepackage[spanish]{babel} %español capitulos y secciones
\usepackage{graphicx} %gráficos y figuras.

%---------------FORMATO de letra--------------------%

\usepackage{lmodern} % tipos de letras
\usepackage{titlesec} %formato de títulos
\usepackage[backref=page]{hyperref} %hipervinculos
\usepackage{multicol} %columnas
\usepackage{tcolorbox, empheq} %cajas
\usepackage{enumerate} %indice enumerado
\usepackage{marginnote}%notas en el margen
\tcbuselibrary{skins,breakable,listings,theorems}
\usepackage[Bjornstrup]{fncychap}%diseño de portada de capitulos
\usepackage[all]{xy}%flechas
\counterwithout{footnote}{chapter}
\usepackage{xcolor}
\usepackage[htt]{hyphenat}
%--------------------GRÀFICOS--------------------------

\usepackage{tkz-fct}

%---------------------------------

\titleformat*{\section}{\bfseries\rmfamily}
\titleformat*{\subsection}{\bfseries\rmfamily}
\titleformat*{\subsubsection}{\bfseries\rmfamily}
\titleformat*{\paragraph}{\bfseries\rmfamily}
\titleformat*{\subparagraph}{\bfseries\rmfamily}

%------------------------------------------

\renewcommand{\labelenumi}{\Roman{enumi}.}%primer piso II) enumerate
\renewcommand{\labelenumii}{\arabic{enumii}$)$}%segundo piso 2)
\renewcommand{\labelenumiii}{\alph{enumiii}$)$}%tercer piso a)
\renewcommand{\labelenumiv}{$\bullet$}%cuarto piso (punto)

%----------Formato título de capítulos-------------

\usepackage{titlesec}
\renewcommand{\thechapter}{\arabic{chapter}}
\titleformat{\chapter}[display]
{\titlerule[2pt]
\vspace{4ex}\bfseries\sffamily\huge}
{\filleft\Huge\thechapter}
{2ex}
{\filleft}

\begin{document}

\normalfont
\input xy
\xyoption{all}
\author{\Large Apuntes por FODE}
\title{\small Daniel M. Hausman \\ \vspace{1cm} \large La filosofía de la economía: Una ontología}
\date{}
\pagestyle{empty}
\maketitle
\thispagestyle{empty}
\let\cleardoublepage\clearpage
\tableofcontents								%indice





\chapter*{Introducción}

\section{Una introducción a la filosofía de la ciencia}
Los temas de los que se ha ocupado la filosofía de la ciencia que son más relevantes para la economía se pueden dividir en cinco grupos:

\begin{enumerate}[1.]
    \item \textit{Objetivos} ¿Cuáles son los objetivos de la ciencia y de la teoría científica? ¿Es la ciencia principalmente una actividad práctica que apunta a descubrir generalizaciones útiles, o debería la ciencia buscar explicaciones y verdades?.
    \item \textit{Explicación} ¿Qué es una explicación científica?.
    \item \textit{Teorías} ¿Qué son las teorías, modelos y leyes? ¿Cómo se relacionan entre sí? ¿Cómo se descubren o construyen?.
    \item \textit{Prueba, inducción y demarcación} ¿Cómo se prueban y confirman o refutan las teorías, modelos y leyes científicas? ¿Cuáles son las diferencias entre las actitudes y prácticas de los científicos y las de los miembros de otras disciplinas?.
    \item ¿Las respuestas a estas cuatro preguntas son las mismas para todas las ciencias en todo momento? ¿Se pueden estudiar las acciones e instituciones humanas de la misma manera que se estudia la naturaleza?.
\end{enumerate}

\subsection{Los objetivos de la ciencia}
Hay dos escuelas principales de pensamiento

\begin{enumerate}[1.]
    \item Los realistas científicos.- la ciencia debe descubrir verdades sobre el mundo y explicar los fenómenos. (Ven a las teorías como aproximaciones de la realidad).
    \item Los antirrealistas que puede ser instrumentalistas que consideran que los objetivos de la ciencia son exclusivamente prácticos, o los antirrealistas pueden estar en desacuerdo con los realistas principalmente sobre si existen los no observables postulados para las teorías científicas, si las afirmaciones sobre ellos son verdaderas o falsas y si la evidencia observable puede establecer afirmaciones sobre inobservables. (Ven a las teorías como herramientas)
\end{enumerate}

Los antirrealistas están de acuerdo que las teorías son importantes, pero ubican su importancia exclusivamente a su papel de ayudar a las personas a anticipar y controlar los fenómenos. Puede verse en "La metodología de la Economía positiva" de Milton Friedman.


\subsection{Explicación científica}
Las explicaciones responden a la pregunta:
\begin{center}
    ¿Por qué?
\end{center}

Carl Hempel, desarrolla dos modelos principales de explicación científica:

\begin{itemize}
    \item Deductivo-nomológico.- Un enunciado de lo que se va a explicar se deduce de un conjunto de enunciados verdaderos que incluye esencialmente al menos una ley.
    \item Inductivo-estadístico.- Este último, como sugiere su nombre, se ocupa de las explicaciones probabilísticas e intenta extender la intuición básica del modelo deductivo-nomológico (D-N).
\end{itemize}

El modelo D-N es una explicación determinista o no estadística. Si sólo se dispone de una regularidad estadística, no se podrá deducir lo que se quiere explicar, pero se podrá demostrar que es altamente probable, que es lo que exige el modelo inductivo-estadístico de Hempel. \\
Parece que las explicaciones de los acontecimientos y estados de cosas suelen citar sus causar, pero existe dos problemas.
\begin{enumerate}
    \item Primero, aunque la mayoría de las explicaciones de eventos y estados de cosas son explicaciones causales, no todas lo son. 
    \item En segundo lugar, decir que las explicaciones citan causas no es en sí mismo muy informativo. Sin una teoría de la causalidad, una teoría causal de la explicación está vacía, e incluso con una teoría de la causalidad, solo araña la superficie para sostener que explicar es citar una causa. 
\end{enumerate}

\subsection{Teorías científicas y leyes}
La mayoría de los filósofos han argumentado que la ciencia procede mediante el descubrimiento de teorías y leyes, pero los economistas se sienten más cómodos hablando de modelos que de leyes y teorías.\\

Las leyes de las ciencias no son, por supuesto, leyes prescriptivas que dictan cómo deberían ser las cosas. Las leyes científicas son, en cambio, (hablando en términos generales) regularidades en la naturaleza. Pero quizás el concepto de ley no sea útil para aquellos interesados en la metodología económica.\\

De acuerdo con el modelo de explicación deductivo-nomológico, los economistas pueden usar generalizaciones como la ley de la demanda para explicar fenómenos económicos solo si esas generalizaciones son genuinamente leyes.\\

Los positivistas lógicos precisaron la noción de “trabajar juntos”, argumentando que las teorías forman sistemas deductivos. Según los positivistas, las teorías son principalmente objetos “sintácticos”, cuyos términos y afirmaciones se interpretan por medio de reglas de “correspondencia”. Estos esperaban expresar teorías científicas en lenguajes formales, pero los positivistas se dieron cuenta de que la relación entre teoría y observación es más compleja.\\

Por tal razón se usan modelos, las cuales se manipulan, exploran y modifican, afirmando que los predicados (se afirma algo como: es un sistema de consumo de dos productos básicos) construyen o determinan que son verdaderos o falsos para los sistemas de cosas del mundo. Ahora bien, por qué los filósofos serios defienden la visión predicada de los modelos?.
Existen dos razones

\begin{itemize}
    \item En primer lugar, si uno espera ser capaz de reconstruir las afirmaciones de la ciencia formalmente, el punto de vista del predicado tiene importantes ventajas técnicas.
    \item En segundo lugar, el punto de vista del predicado ofrece una forma útil de esquematizar los dos tipos de logros que implica la construcción de una teoría científica. (Modelo de Hempel).
\end{itemize}

Una parte absolutamente crucial del quehacer científico es la construcción de nuevos conceptos, de nuevas formas de clasificar los fenómenos, que es común en los economistas.

\subsection{Evaluación y Demarcación}

La opinión kantiana de que hay verdades sintéticas a priori (que pueden ser demostrados puramente por razonamiento lógico y deductivo, sin necesidad de recurrir a la experiencia empírica) todavía tiene partidarios entre los llamados economistas austriacos, especialmente Ludwig von Mises y sus seguidores. Argumentan que los postulados fundamentales de la economía son verdades sintéticas a priori. 

Hume, lanza un desafío: muéstrame un buen argumento cuya conclusión sea alguna generalización o alguna afirmación sobre algo no observado y cuyas premisas incluyan solo informes de experiencias sensoriales. Tal argumento no puede ser un argumento deductivo, porque tales inferencias son falibles. Tampoco lo hará un argumento “inductivo”, ya que solo tenemos motivos inductivos y, por lo tanto, que plantean dudas para creer que tales argumentos son buenos.\\

El problema de la demarcación se refiere a la distinción entre teorías científicas y otros tipos de teorías.\\

De lo que debería ocuparse la filosofía de la ciencia, según Lakatos, no son las reglas para evaluar teorías, sino las reglas para modificar y comparar teorías.

\subsection{La unidad de la ciencia}
Uno se pregunta si, dado el libre albedrío, el comportamiento humano es intrínsecamente impredecible y, por lo tanto, no está sujeto a ninguna ley. Tan tentador como esta línea de pensamiento puede ser, es un error. Incluso si no hay leyes deterministas del comportamiento humano, hay, de hecho, muchas regularidades en la acción humana.\\

Porque las personas pueden, como señala Frank Knight, cometer errores o no reconocer las cosas. Como primera aproximación, los economistas se abstraen de tales dificultades. Suponen que las personas tienen información perfecta. Al suponer que la gente cree cualesquiera que sean los hechos, los economistas pueden evitar preocuparse por lo que la gente realmente cree.\\

Un punto de vista es que la economía sirve a la política de la misma manera que las ciencias naturales guían las políticas, es decir, ayudando a los formuladores de políticas a elegir los medios que lograrán sus fines. Tal papel práctico para el conocimiento científico no parece problemático. Los agentes tienen algún objetivo que quieren lograr, y el científico proporciona el "saber hacer" necesario. Desde este punto de vista, la economía importa a la política sólo como una fuente de información descriptiva o “libre de valores”.

\section{Una introducción a la economía}
La economía comienza en el siglo XVIII con los escritos de los fisiócratas franceses, de Cantillon y Hume, y especialmente de Adam Smith. Lo que diferenció a estos pensadores de sus predecesores fue su creciente reconocimiento de la existencia de mecanismos mediante los cuales las acciones individuales tendrían consecuencias sistemáticas sin necesidad de que el gobierno controlara los procesos. Smith y otros llegaron a ver la economía en gran medida como un sistema autorregulador. La economía surgió cuando se comprendió que había mecanismos y sistemas económicos para estudiar. \\

La economía se ha preocupado principalmente por comprender cómo funciona un sistema económico capitalista. (Un sistema económico capitalista es una economía de mercado en la que los medios de producción son en su mayor parte de propiedad privada, y los trabajadores son libres de aceptar o rechazar ofertas de empleo). Muchos economistas creen que sus teorías también se aplican a otros arreglos económicos, y se ha trabajado mucho en otros tipos de economías. Pero el núcleo de la teoría económica se ha dedicado a comprender las economías capitalistas.\\

 Los economistas “clásicos”, de los cuales Adam Smith, David Ricardo y John Stuart Mill son los más destacados, no tenían mucho que decir sobre las elecciones de los consumidores. Su énfasis estaba en la producción y en los factores que influyen en la oferta de bienes de consumo. Consideraron que los agentes buscaban maximizar sus ganancias financieras y dividieron tanto a los agentes como a los insumos básicos en tres clases principales:
\begin{itemize}
    \item Capitalistas con su capital (que concibieron como existencias de bienes acumulados o el valor de los mismos), 
    \item terratenientes con su tierra y
    \item trabajadores con su capital.
\end{itemize}
Estos ofrecieron dos generalizaciones principales con respecto a la producción
\begin{itemize}
    \item Asumieron que en cualquier momento dado todos los bienes reproducibles (excluyendo así cosas como la pintura) podrían producirse en cualquier cantidad por el mismo costo por unidad.
    \item Luego, descubrieron los rendimientos decreciente. A menos que haya alguna innovación tecnológica, a medida que se dedica más y más trabajo a una cantidad fija de tierra la cantidad que aumenta la producción cuando se emplea un trabajador adicional eventualmente disminuirá.
\end{itemize}

A diferencia de Ricardo, a finales del siglo XIX, los economistas reconocieron que la población no tenía por qué crecer en respuesta a los salarios más altos. Además, las mejores tecnológicas provocaron aumentos en la productividad. Luego llegaron los neo-clásicos o marginales, que centraron su atención en la elección y el intercambio individuales, mediante las preferencias del consumidor, el intercambio y la demanda de mercancías.\\


Muchos de los primeros economistas neo-clásicos fueron influenciados por el utilitarismo, una teoría ética expuesta por Jeremy Bentham y John Stuart Mill. Según los utilitaristas, las preguntas de política social deben responderse calculando las consecuencias de las alternativas para el felicidad total de los individuos. La política que maximiza la suma de las utilidades individuales es la moralmente correcta. Bentham sostuvo que la utilidad de algo para un individuo es una sensación que, en principio, podría cuantificarse y medirse. También creía que los individuos actúan para maximizar su propia utilidad. Jevons desarrolló la noción esencialmente benthamita de una función de utilidad. Cada opción abierta a un resultado individual en una cierta cantidad de utilidad para esa persona. Entonces se puede aclarar la noción de racionalidad manteniendo que las personas actúan para maximizar alguna función de utilidad consistente. Además, los economistas neoclásicos asumieron que los consumidores generalmente no están satisfechos, que siempre preferirán un paquete $x$ de bienes o servicios a otro paquete y si $x$ es inequívocamente mayor que $y$. La insaciabilidad es tanto una primera aproximación plausible como articula la noción de interés propio. Todo lo que importa a los agentes son los paquetes de mercancías y servicios que están entregando o recibiendo. \\

Con la adición de una generalización más, se tiene el núcleo de la teoría económica moderna. Los primeros economistas neo-clásicos notaron que a medida que uno consume más de cualquier producto o servicio, cada unidad adicional aumenta la utilidad de uno a una tasa decreciente, llamada ley de utilidad marginal decreciente. Esta teoría se ha refinado enormemente, donde "utilidad" es solo otra forma de hablar sobre preferencias. En términos generales, uno puede reemplazar la "ley" de la utilidad marginal decreciente con la generalización de que las personas están dispuestas a pagar menos por unidades adicionales de productos básicos que ya tienen mucho que por productos básicos de los que tiene muy poco. A pesar de estos refinamientos la teoría dominante sigue siendo reconociblemente la teoría desarrollada por los primeros economistas neoclásicos.\\

Luego de los años 30, Keynes desafió la teoría dominante, enfatizando la importancia de la liquidez tanto para las empresa como para los individuos cuando se enfrentan a las incertidumbres. Los precios, especialmente los salarios, no caen fácilmente, y los salarios más bajos pueden generar menos gastos, lo que suprimiría la demanda de productos básicos y conduciría a una caída aún más profunda. Para ello Keynes mencionó que el estado podría aumentar la demanda agregada de de materias primas y alentar la inversión y de esa manera sacar la economía del desempleo. De todas formas, Keynes no sacudió los fundamentos de la teoría neo-clásica. Pero las versiones actualizadas de la economía Keynesiana, siguen siendo influyentes. Siendo así, la macroeconomía un área inestable de la economía.\\

La econometría es una rama de la estadística aplicada, así como una rama de la economía. A partir de la década de 1930, se esperaba que las afirmaciones de los teóricos económicos pudieran probarse y refinarse con la ayuda de técnicas estadísticas. Desde entonces, las técnicas econométricas se han vuelto mucho más sofisticadas. Exactamente lo que este trabajo significa para la teoría económica (a diferencia de las investigaciones prácticas de enfoque limitado) es controvertido, y algunos economistas prominentes argumentan que la econometría es incapaz de proporcionar buenas razones para creer o no creer en afirmaciones causales significativas.

Por otro lado, de acuerdo con el materialismo histórico de Marx, las relaciones entre las personas en el curso de sus actividades productivas son las relaciones sociales más fundamentales. Marx, considera al capitalismo, a pesar de las miserias que puede causar un enorme adelanto para el ser humano. Pero, como argumentos en su ensayo inicial "trabajo enajenado", no permite que las personas decidan racionalmente y conscientemente, como se debe desarrollar la sociedad y la naturaleza humana. Por lo que Marx cree que las personas pueden trascender el capitalismo y organizar la producción y la distribución de alguna manera racional. Pero no tuvo existe, ya que las teorías económicas no se ciernen sobre las olas políticas. \\

Luego apareció la economía institucionalista, que no ignoran las tomas de decisiones individuales, pero enfatizan las restricciones en evolución sobre los agentes que ocupan roles económicos específicos.\\

La tercera alternativa contemporánea a la economía dominante, es la economía del comportamiento, incluida la neuroeconomía. \\

También se encuentran los neorricardianos, que creen que se puede comprender mejor la economía reformulando ecuaciones matemáticas. Los economistas Austriacos que están de acuerdo con los economistas neo-clásicos pero enfatizan la importancia de la incertidumbre, desequilibrio y un punto de vista subjetivo. Los economistas postkeynesianos a menudo ofrecen críticas similares en la alta teoría, pero a diferencia de los austriacos, tienden a defender las políticas intervencionalistas.  Los pronosticadores económicos a menudo dependen muy poco de cualquier teoría específica. 


\section{Una introducción a la metodología de la economía}

Knight y los austriacos están de acuerdo en que tan pronto como se abandona el punto de vista subjetivo y se piensa en la economía como si fuera una ciencia natural, se pierden de vista los rasgos centrales del tema. Luego Hutchison argumenta que la economía, como otras ciencias, debe formular generalizaciones comprobadas y someterlas a pruebas serias. Como también Milton Friedman intenta mostrar que la economía satisface estándares ampliamente positivistas, que se convirtió en el trabajo más influyente sobre metodología económica del siglo XX.


\part{Discusiones clásicas}

\chapter{Sobre la definición y el método de la economía política\\ John Stuart Mill}

Mill fue uno de los primeros defensores de los derechos de la mujer y de un socialismo democrático moderado.\\

La economía política, no trata de toda la naturaleza del hombre modificada por el estado social, ni de toda la conducta del hombre en sociedad. Si no se refiere a él únicamente y exclusivamente como un ser que desea poseer riquezas y que es capaz de juzgar la eficacia comparativa de los medios para obtener ese fin. Bajo esa influencia de deseo, muestrea a la humanidad acumulando riqueza y empleando esa riqueza en la producción de otra riqueza, sancionando de común acuerdo la institución de la propiedad, estableciendo leyes para evitar que los individuos invadan las propiedades de otros por la fuerza o el fraude, adoptando diversos artilugios para aumentar la productividad de su trabajo, resolviendo la división del producto por acuerdo, bajo la influencia de la competencia. Todo esto y más para facilitar la producción. Ahora bien, cuando un efecto depende de una concurrencia de causas, esas causas deben estudiarse una a la vez, y sus leyes deben investigarse por separado, esto si deseamos a través de las causas, obtener el poder de predecir o controlar el efecto, ya que la ley del efecto se compone de las leyes de todas las causas que lo determinan. Para juzgar cómo el ser humano actuará bajo la variedad de deseos y aversiones que simultáneamente operan sobre él, debemos saber cómo actuaría bajo la influencia exclusiva de cada uno en particular. \\

El economista político, se pregunta cuáles son las acciones que producirían este deseo, si, dentro de los departamentos en cuestión, no estuviera impedido por ningún otro. Pero con respecto a aquellas partes de la conducta humana en las que la riqueza ni siquiera es el objeto principal, la Economía Política no pretende que sus conclusiones sean aplicables. Sólo en algunos casos llamativos, como el principio de población serán tomados en cuenta y deben ser corregidos teniendo debidamente en cuenta los efectos de cualquier impulso de una descripción diferente, que puede demostrarse que interfiere con el resultado en cualquier caso particular. Las conclusiones de la Economía Política dejarán de ser aplicables a la explicación o predicción de los acontecimientos reales, hasta que sean modificadas por una correcta asignación del grado de influencia ejercido por la otra causa.\\

Por lo tanto, la definición de economía política sera:

\begin{center}
    \textit{La ciencia que traza las leyes de los fenómenos de la sociedad que surgen de las operaciones combinadas de la humanidad para la producción de riqueza, en la medida en que esos fenómenos no son modificados por la búsqueda de cualquier otro objeto.}
\end{center}

Tengamos en cuenta que las diferencias de principio entre ciencias difieren no sólo en lo que creen ver, si no en el lugar de donde obtuvieron la luz con la que creen verlo. La más universal de las formas en que suele presentarse esta diferencia de método, es la antigua disputa entre lo que se llama:

\begin{itemize}
\item teoría y
\item práctica o experiencia. 
\end{itemize}

De todas formas las dos partes necesitarán de experiencia y teoría a diferencia que los que se hacen llamar prácticos, requerirán experiencia específica, y argumentarán totalmente hacia arriba desde hechos particulares hasta una conclusión general; mientras que aquellos que se llaman teóricos, pretenderán abarcar un campo más amplio de experiencia, y habiendo argumentado hacia arriba desde hechos particulares hasta un principio general que incluye un rango mucho más amplio que el de la cuestión en discusión, argumentaran hacia abajo desde ese principio general hasta una variedad de conclusiones específicas. Estos métodos  se pueden llamar: 
\begin{itemize}
    \item a posteriori.- entendemos el que requiere como base de sus conclusiones, no la mera experiencia, si no la experiencia concreta.
    \item a priori.- entendemos con el razonamiento a partir de una hipótesis supuesta.
\end{itemize}
Verificar, la hipótesis misma a posteriori, no forman parte en absoluto del quehacer científico, sino de la aplicación de la ciencia.\\

Con respecto a la economía política, lo hemos caracterizado como una ciencia abstracta y un método a priori. Ya que, supone una definición arbitraria de hombre, como un ser que invariablemente hace aquello por lo que puede obtener la mayor cantidad de necesidades, comodidades y lujos, con la menor cantidad de trabajo y abnegación física con la que puede obtenerse en el estado de conocimiento existente. Ahora, bien nadie que esté familiarizado con los tratados sistemáticos de economía política pondrá en duda que siempre que un economista ha demostrado que actuando de una manera determinada, un trabajador puede obtener salarios más altos, o un capitalista tener mayores beneficios. Esto ya que, el economista político razona a partir de premisas asumidas, que pueden carecer de fundamentos en los hechos. Por consiguiente, las conclusiones de la economía política, como de la geometría, solo son verdades en abstracto, es decir son verdaderas bajo ciertas suposiciones, en las que sólo se tienen en cuenta causas generales, causas comunes a toda la clase de casos considerados. El método a priori que se le imputa, como si su empleo demostrara que toda su ciencia carece de valor, es, como demostraremos más adelante, el único método por el cual puede alcanzarse la verdad en cualquier departamento de la ciencia social. Siempre que a medida en que los hechos reales se alejen de la hipótesis deban permitirse una desviación correspondiente de la letra estricta de su conclusión, de lo contrario sólo será verdad para las cosas que ha supuesto arbitrariamente, no para las cosas que existen. Cuando una determinada causa existe realmente, y si se la dejara sola produciría infaliblemente un determinado efecto, ese mismo efecto, modificado por todas las demás causas concurrentes, corresponderá correctamente al resultado realmente producido.\\

Luego, el método a posteriori será útil siempre que este en relación con el método a priori, ya que por si solo no podremos hacer experimentos morales o de ética. A estos experimentos se le denominan \textit{experimentos cruciales}. Por ejemplo, en asuntos de estado, ¿cómo podemos obtener un experimentos crucial sobre el efecto de una política comercial restrictiva sobre la riqueza nacional?. Tendremos que encontrar dos naciones iguales en todos los demás aspectos, o al menos que posean, en un grado exactamente igual, todo lo que conduce a la opulencia nacional, y que adopten exactamente la misma política en todos sus otros asuntos, pero que difieran en esto solamente, en que una de ellas adopta un sistema de restricciones comerciales, y la otra adopta el libre comercio. Este sería un experimento decisivo, similar a los que casi siempre podemos obtener en física experimental. Por lo tanto, ya sea en la política económica o en las ciencias sociales, mientras vemos los hechos concretos revestidos de complejidad, no queda otro método que el a priori o el de la especulación abstracta.\\

Ahora, las causas pueden ser de experimentación específica, que son las leyes de la naturaleza humana y las circunstancias externas capaces de excitar la voluntad humana a la acción, que están al alcance de nuestra observación. También podemos observar cuales son las causas que excitan estos deseos. Si al final la suposición es correcta hasta donde llega, y no difiere de la verdad más que como una parte difiere del todo, entonces las conclusiones que se deducen correctamente de la suposición constituyen una verdad abstracta; y cuando se completan añadiendo o sustrayendo el efecto de las circunstancias no calculadas, son verdaderas en lo concreto, y pueden aplicarse a la práctica. De este carácter es la ciencia de la economía política, para hacerla perfecta como ciencia abstracta, las combinaciones de circunstancias que asume, para rastrear sus efectos, deben incorporar todas las circunstancias que son comunes a todos los casos, y del mismo modo todas las circunstancias que son comunes a cualquier clase importante de casos. Las conclusiones correctamente deducidas de estas suposiciones serían tan verdaderas en abstracto como las de las matemáticas; y sería una aproximación lo más cercana posible a la verdad abstracta, a la verdad en lo concreto.\\

Cuando los principios de la economía política se van a aplicar a un caso en particular, entonces es necesario tener en cuenta todas las circunstancias individuales de ese caso; no sólo examinando a cuál de los conjuntos de circunstancias contemplados por la ciencia abstracta corresponden las circunstancias del caso en cuestión, sino también qué otras circunstancias pueden existir en ese caso, que no siendo comunes a él con ninguna clase grande y fuertemente marcada de casos, no han caído bajo el conocimiento de la ciencia. Estas circunstancias se han denominado causas perturbadoras. Estas causas tiene sus leyes que a priori de las leyes de las causas perturbadoras, la naturaleza y la cantidad de la perturbación puede predecirse a priori. Por lo tanto, el efecto de las causas especiales debe entonces sumarse o restarse del efecto de las generales. \\

Habiendo demostrado ahora que el método a priori en Economía Política, y en todas las demás ramas de la ciencia moral, es el único modo cierto o científico de investigación, y que el método a posteriori, o el de la experiencia específica, como medio de llegar a la verdad, es inaplicable a estos temas, podremos demostrar que este último método es, no obstante, de gran valor en las ciencias morales; a saber, no como medio de descubrir la verdad, sino de verificarla, y de reducir al punto más bajo esa incertidumbre antes aludida como derivada de la complejidad de cada caso particular, y de la dificultad (por no decir imposibilidad) de que se nos asegure a priori que hemos tenido en cuenta todas las circunstancias materiales.\\

Si pudiéramos estar completamente seguros de que conocemos todos los hechos del caso particular, podríamos obtener poca ventaja adicional de la experiencia específica y podremos llamarnos profetas, pero estas no son reveladas, por lo que deben ser recogidas por la observación. Ahora, algunas de las causas pueden estar más allá de la observación. Donde corremos el riesgo de solo prestar atención a una parte de las causas. Pero a la vez tender a deducirlas lo más exactas posibles, haciendo abstractas de todas las circunstancias excepto de las forman parte de la hipótesis. Así, menos probable sospechemos que estamos en un error.\\

El verdadero estadista práctico es aquel que combina esta experiencia con un profundo conocimiento de la filosofía política abstracta. Cualquier adquisición, sin la otra, lo deja cojo e impotente si es consciente de la deficiencia; lo vuelve obstinado y presuntuoso si, como es más probable, es completamente inconsciente de ello.\\

El método del filósofo práctico consiste, en dos procesos:

\begin{enumerate}[1.]
    \item analítico,
    \item sintético.
\end{enumerate}

Debe analizar el estado actual de la sociedad en sus elementos, sin dejar caer ni perder ninguno de ellos por el camino. Después de referirse a la experiencia del hombre individual para aprender la ley de cada uno de estos elementos, es decir, para aprender cuáles son sus efectos naturales, y cuánto del efecto se sigue de tanto de la causa cuando no es contrarrestado por ninguna otra causa, queda una operación de síntesis; poner todos estos efectos juntos, y, de lo que son por separado, recoger lo que sería el efecto de todas las causas actuando a la vez. Si estas diversas operaciones pudieran realizarse correctamente, el resultado sería la profecía; pero como sólo se pueden realizar con una cierta aproximación de corrección, la humanidad nunca puede predecir con absoluta certeza, sino sólo con una menor o menor precisión con mayor o menor grado de probabilidad. \\

Si la fuerza que, siendo la menos conspicua de las dos, se llama fuerza perturbadora, prevalece lo suficiente sobre la otra fuerza en algún caso, para constituir ese caso lo que comúnmente se llama una excepción, la misma fuerza perturbadora probablemente actúa como causa modificadora en muchos otros casos que nadie llamará excepciones.. Si se dijera que es una ley de la naturaleza que todos los cuerpos pesados caen a tierra, probablemente se diría que la resistencia de la atmósfera, que impide que un globo caiga, constituye al globo una excepción a esa pretendida ley de la naturaleza. Pero la verdadera ley es que todos los cuerpos pesados tienden a caer, y a esto no hay excepción, ni siquiera el sol y la luna, porque incluso ellos, como todo astrónomo sabe, tienden hacia la tierra con una fuerza exactamente igual a la que la tierra tiende hacia ellos. La resistencia de la atmósfera podría, en el caso particular del globo, por una mala interpretación de lo que es la ley de la gravitación, decirse que prevalece sobre la ley; pero su efecto perturbador es tan real en cualquier otro caso, ya que aunque no impide, retarda la caída de todos los cuerpos. La regla y la llamada excepción no dividen los casos entre sí; cada una de ellas es una regla general que se extiende a todos los casos. Llamar a uno de estos principios concurrentes excepción del otro, es superficial y contrario a los principios correctos de nomenclatura y ordenación. Un efecto exactamente de la misma clase, y derivado de la misma causa, no debe colocarse en dos categorías diferentes, por el mero hecho de que exista o no otra causa preponderante sobre él.


\chapter{Objetividad y compresión de la economía \\ Max Weber}

Toda reflexión seria sobre los elementos últimos de la conducta humana significativa se orienta principalmente en términos de las categorías:

\begin{itemize}
    \item Fines,
    \item Medios.
\end{itemize}

En la medida en que somos capaces de determinar (dentro de los límites actuales de nuestro conocimiento) qué medios para la consecución de un fin propuesto son apropiados o inapropiados, podemos de este modo estimar las posibilidades de alcanzar un fin determinado con ciertos medios disponibles. De este modo, podemos criticar indirectamente la fijación del fin en sí como prácticamente significativo (sobre la base de la situación histórica existente) o como carente de sentido con referencia a las condiciones existentes. Además, cuando parece existir la posibilidad de alcanzar un fin propuesto, podemos determinar (naturalmente dentro de los límites de nuestro conocimiento existente) las consecuencias que la aplicación de los medios a utilizar producirá además de la eventual consecución del fin propuesto, como resultado de la interdependencia de todos los acontecimientos. De este modo, podemos proporcionar a la persona que actúa la capacidad de sopesar y comparar las consecuencias indeseables frente a las deseables de su acción. Así, podemos responder a la pregunta: ¿cuánto 'costará' la consecución de un fin deseado en términos de la pérdida previsible de otros valores?.\\

El tipo de ciencia social que nos interesa es una ciencia empírica de la realidad concreta. Queremos comprender, por un lado, las relaciones y el significado cultural de los acontecimientos individuales en sus manifestaciones contemporáneas y, por otro, las causas de que históricamente sean así y no de otro modo. Ahora bien, en cuanto intentamos reflexionar sobre el modo en que la vida nos enfrenta en situaciones concretas inmediatas, nos presenta una multiplicidad infinita de acontecimientos que surgen y desaparecen sucesiva y coexistentemente, tanto 'dentro' como 'fuera' de nosotros mismos. Todo el análisis de la realidad infinita que la mente humana finita puede llevar a cabo descansa sobre el supuesto tácito de que sólo una porción finita de esta realidad constituye el objeto de la investigación científica, y que sólo ella es 'importante' en el sentido de ser "digna de ser conocida". Pero, cuales son los criterios por lo que se selecciona este segmento?; Tan pronto como hayamos demostrado que alguna relación causal es una 'ley', es decir, si hemos demostrado que es universalmente válida por medio de una inducción histórica exhaustiva o la hemos hecho inmediata y tangiblemente plausible según nuestra experiencia subjetiva, un gran número de casos similares se ordenan bajo la fórmula así alcanzada. Aquellos elementos de cada suceso individual que quedan sin explicar por la selección de sus elementos subsumibles bajo la 'ley' se consideran residuos científicamente no integrados de los que se ocupará el perfeccionamiento ulterior del sistema de 'leyes'. Alternativamente, se considerarán 'accidentales' y, por lo tanto, sin importancia científica porque no encajan en la estructura de la 'ley'; en otras palabras, no son típicos del acontecimiento y, por lo tanto, sólo pueden ser objeto de 'curiosidad ociosa'. En consecuencia, el ideal al que sirven todas las ciencias, incluidas las culturales es un sistema de proposiciones a partir del cual puede deducirse la verdad.\\

Hemos denominado 'ciencias de la cultura' a aquellas disciplinas que analizan los fenómenos de la vida en función de su significación cultural. El concepto de cultura es un concepto de valor. La realidad empírica se convierte en cultura para nosotros porque la relacionamos con ideas de valor en la medida que lo hacemos y se volvieron significativos para nosotros. Es significativa porque revela relaciones que son importantes debido a su conexión con nuestros valores. Sin embargo, no podemos descubrir lo que tiene sentido para nosotros mediante una investigación sin supuestos de los datos empíricos. Más bien, la percepción de su significado para nosotros es la presuposición de que se convierta en objeto de investigación. Naturalmente, el sentido no coincide con las leyes como tales, y cuanto más general es la ley, menor es la coincidencia. Porque el significado específico que un fenómeno tiene para nosotros no se encuentra naturalmente en las relaciones que comparte con muchos otros fenómenos. ...\\

¿Cuál es la consecuencia de todo esto?\\
Naturalmente, ello no implica que el conocimiento de proposiciones universales, la construcción de conceptos abstractos, el conocimiento de regularidades y el intento de formular 'leyes' carezcan de justificación científica en las ciencias de la cultura. Muy al contrario, si el conocimiento causal de los historiadores consiste en la imputación de efectos concretos a causas concretas, una imputación válida de cualquier efecto individual sin la aplicación del conocimiento 'nomológico' -es decir, el conocimiento de secuencias causales recurrentes- sería en general imposible. En otras palabras, los efectos 'adecuados' de los elementos causales implicados deben tenerse en cuenta para llegar a cualquier conclusión de este tipo. \\
En todas partes, sin embargo, y por tanto también en la esfera de los procesos económicos complicados, cuanto más cierto y amplio sea nuestro conocimiento general, mayor será la certeza de la imputación. No se trata  aquí de leyes económicas, sino de relaciones adecuadas expresadas en regla y con la aplicación de la categoría de 'posibilidad objetiva'. El establecimiento de tales regularidades no es el fin, sino más bien el medio del conocimiento. El que una relación causal de la experiencia cotidiana que se repite con regularidad deba formularse como una "ley" es una cuestión de conveniencia que debe resolverse por separado en cada caso concreto. \\

Ahora, un análisis objetivo no carece de sentido, porque:

\begin{enumerate}[1.]
    \item el conocimiento de las leyes sociales no es conocimiento de la realidad social, sino más bien uno de los diversos medios de los que se sirve nuestra mente para alcanzar este fin.
    \item el conocimiento de los acontecimientos culturales sólo es concebible a partir del significado que las constelaciones concretas de la realidad tiene para nosotros en determinadas situaciones concretas individuales.
\end{enumerate}

Todo conocimiento de la realidad cultural, como puede verse, es siempre un conocimiento desde puntos de vista particulares.\\

Entre las muchas confusiones, apareció el intento de refutar la concepción materialista de la historia, afirmando que puesto que toda la vida económica debe tener lugar en formas legal o convencionalmente reguladas, todo desarrollo económico debe adoptar la forma de esfuerzo por la creación de nuevas formas legales. Por lo tanto, se dice que sólo es inteligible a través de máximas éticas y que, por este motivo, es esencialmente diferente de cualquier tipo de desarrollo 'natural'. En consecuencia, se dice que el conocimiento del desarrollo económico es de carácter 'teleológico'.  Quienes consideren de importancia secundaria la laboriosa tarea de comprender causalmente la realidad histórica pueden prescindir de ella, pero es imposible suplantarla por cualquier tipo de 'teleología'. Desde nuestro punto de vista, la 'finalidad' es la concepción de un efecto que se convierte en causa de una acción. Puesto que tenemos en cuenta toda causa que produce o puede producir un efecto significativo, también consideramos ésta. Su significación específica consiste únicamente en el hecho de que no sólo observamos la conducta humana, sino que podemos y deseamos comprenderla.\\

La elección del objeto de investigación y la extensión o profundidad con que esta investigación intenta penetrar en la infinita red causal, están determinadas por las ideas valorativas que dominan al investigador y su época.\\

Podemos finalmente pasar a la cuestión que es metodológicamente relevante en la consideración de la 'objetividad' del conocimiento cultural. La pregunta: ¿cuál es la función lógica y la estructura de los conceptos que nuestra ciencia, como todas las demás, utiliza? Replanteada con especial referencia al problema decisivo, la pregunta es: ¿cuál es el significado de la teoría y la conceptualización teórica para nuestro conocimiento de la realidad cultural?\\
La economía fue originalmente -como ya hemos visto- una "técnica", al menos en el foco central de su atención. Con esto queremos decir que consideraba la realidad desde un punto de vista evaluativo práctico, al menos ostensiblemente inequívoco y estable: a saber, el aumento de la "riqueza" de la población. Por otra parte, fue desde el principio algo más que una técnica, ya que se integró en el gran esquema del derecho natural (teoría filosófica y jurídica que sostiene la existencia de principios y normas universales e inmutables que se consideran intrínsecos a la naturaleza humana y que, por lo tanto, son inherentes a todas las sociedades y sistemas legales.) y de la Weltanschauung racionalista del siglo XVIII (visión del mundo que se basa principalmente en la razón y la lógica como herramientas fundamentales para entender la realidad y resolver problemas).\\

 El resultado, en la medida en que nos interesa, es que a pesar de la poderosa resistencia a la infiltración del dogma naturalista debido al idealismo alemán desde Fichte y el logro de la Escuela Histórica Alemana en derecho y economía y en parte por el trabajo mismo de la Escuela Histórica, el punto de vista naturalista en ciertos problemas decisivos aún no ha sido superado. Entre estos problemas encontramos la relación entre “teoría” e “historia”, que aún es problemática en nuestra disciplina.\\

 La única forma exacta de conocimiento -la formulación de leyes inmediata e intuitivamente evidentes- es, sin embargo, al mismo tiempo la única que ofrece acceso a acontecimientos que no han sido observados directamente. De ahí que, al menos en lo que se refiere a los fenómenos fundamentales de la vida económica, la construcción de un sistema de proposiciones abstractas y, por tanto, puramente formales, análogas a las de las ciencias naturales exactas, sea el único medio de analizar y dominar intelectualmente la complejidad de la vida social. La teoría económica exacta se ocupa del funcionamiento de un motivo psíquico, las demás teorías tienen como tarea formular el comportamiento de todos los demás motivos en tipos similares de proposiciones que gocen de validez hipotética. En consecuencia, en ocasiones se ha hecho la fantástica afirmación respecto a las teorías económicas -por ejemplo, las teorías abstractas del precio, el interés, la renta, etc.- de que pueden, siguiendo ostensiblemente la analogía de las proposiciones de la ciencia física, aplicarse válidamente a la derivación de conclusiones cuantitativamente enunciadas a partir de premisas reales dadas, ya que, dados los fines, el comportamiento económico respecto a los medios está inequívocamente determinado. Esta afirmación omite observar que para poder llegar a este resultado incluso en el caso más simple, la totalidad de la realidad histórica existente, incluyendo cada una de sus relaciones causales, debe ser asumida como dada y presupuesta como conocida. \\

 La teoría abstracta pretendía basarse en axiomas psicológicos y, en consecuencia, los historiadores han reclamado una psicología empírica para demostrar la invalidez de esos axiomas y derivar el curso de los acontecimientos económicos a partir de principios psicológicos.  En el establecimiento de las proposiciones de la teoría abstracta, se trata sólo aparentemente de "deducciones" a partir de motivos psicológicos fundamentales. En realidad, las primeras son un caso especial de un tipo de construcción de conceptos que es peculiar y, hasta cierto punto, indispensable, para las ciencias de la cultura. En este punto merece la pena describirlo con más detalle, ya que así podemos acercarnos más a la cuestión fundamental del significado de la teoría en las ciencias sociales. Tenemos en la teoría económica abstracta una ilustración de esas construcciones sintéticas que se han designado como "ideas" de los fenómenos históricos. Nos ofrece una imagen ideal de los acontecimientos del mercado de mercancías en las condiciones de una sociedad organizada según los principios de la economía de intercambio, la libre competencia y una conducta rigurosamente racional. 



 \chapter{La naturaleza y el significado de la ciencia económica} 

 \section{Capítulo 1: La asignatura de economía}

 Los fines son varios, los medios son limitados. Además los fines tiene significado diferente. Ahora bien, la multiplicidad de fines por sí misma no tiene interés necesario para el economista. Si quiero hacer dos cosas, y tengo tiempo y medios amplios para hacerlas, y no necesito el tiempo ni los medios para otra cosa, entonces mi conducta no asume ninguna de las formas que son objeto de la ciencia económica. La mera limitación de medios tampoco es suficiente por sí misma para dar lugar a fenómenos económicos. Si los medios de satisfacción no tienen uso alternativo, entonces pueden ser escasos, pero no pueden ser economizados. El Maná que cayó del cielo pudo haber sido escaso, pero, si era imposible cambiarlo por otra cosa o posponer su uso, no era objeto de ninguna actividad con un aspecto económico. Tampoco es la aplicabilidad alternativa de medios escasos una condición completa de la existencia del tipo de fenómenos que estamos analizando. Si el sujeto económico tiene dos fines y un medio para satisfacerlos, y los dos fines son de igual importancia, su posición será como la posición del asno en la fábula, paralizado a medio camino entre los dos paquetes de heno igualmente atractivos. Pero cuando el tiempo y los medios para alcanzar los fines son limitados y susceptibles de aplicación alternativa; es decir que existen otras opciones o soluciones que podrían ser consideradas en lugar de las ya propuestas o utilizadas., y los fines pueden distinguirse por orden de importancia, entonces el comportamiento asume necesariamente la forma de elección. Todo acto en el que se emplean tiempo y medios escasos para la consecución de un fin implica la renuncia a su uso para la consecución de otro. Tiene un aspecto económico. La escasez de medios para satisfacer fines de importancia variable es una condición casi omnipresente del comportamiento humano. \\
 La economía es la ciencia que estudia el comportamiento humano como una relación entre fines y medios escasos que tienen usos alternativos.

 \section{Capítulo 4: La naturaleza de las generalizaciones económicas}

 Las proposiciones más fundamentales del análisis económico son las proposiciones de la teoría general del valor. Podemos proceder a preguntar, de que depende su validez. No debería ser necesario dedicar mucho tiempo a demostrar que no puede basarse en una mera apelación histórica, ya que no habría razón alguna para suponer que la historia se repetiría; porque si hay algo que demuestra la historia, no menos que la lógica elemental, es que la inducción histórica, sin ayuda del juicio analítico, es la peor base posible para la profecía.\\
 Es también evidente que nuestra creencia no se basa en los resultados de un experimento controlado. Pero nuestra creencia en las proposiciones, hablando de la teoría del valor, es completa como la creencia basada en cualquier número de experimentos controlados. Pero ¿de qué depende?. Nos damos cuenta de que el fundamento de la teoría del valor es la suposición de las diferentes cosas que el individuo desea hacer tiene una importancia diferente para él y, por tanto, pueden ordenarse de una determinada manera. Que se deduce a que podamos juzgar si diferentes experiencias posibles son de importancia equivalente o mayor o menor para nosotros. De este hecho derivamos la idea de la sustituibilidad de bienes entre diferentes usos, de la demanda de un bien en términos de otro, de una distribución equilibrada de biene entre diferentes usos, del equilibrio del intercambio y de la formación de precios. Supones también una distribución inicial de la propiedad. Pero siempre el principal supuesto de fondo es el de los sistemas de valoración de los distintos agentes económicos.\\

 Como es bien sabido, el principal principio de explicación, complementario de los principios de valoración subjetiva asumidos en la teoría más restringida del valor y el intercambio, es el principio a veces descrito como la Ley de los Rendimientos Decrecientes. La ley de los rendimientos decrecientes es simplemente una forma de expresar el hecho obvio de que los diferentes factores de producción son sustitutos imperfectos entre sí. Si se aumenta la cantidad de trabajo sin aumentar la cantidad de tierra, el producto aumentará, pero no aumentará proporcionalmente. Para duplicar el producto, si no se duplican tanto la tierra como el trabajo, hay que duplicar con creces cualquiera de los dos factores. También se deduce de consideraciones más íntimamente relacionadas con nuestras concepciones fundamentales.Una clase de factores escasos debe definirse como constituida por aquellos factores que son sustitutos perfectos.Es decir, las diferencias de factores se definen esencialmente como sustituibilidad imperfecta. La ley de los rendimientos decrecientes se desprende, por tanto, de la suposición de que existe más de una clase de factores de producción escasos. El principio complementario de que, dentro de unos límites, los rendimientos pueden aumentar, se desprende igualmente de la suposición de que los factores son relativamente indivisibles. Sobre la base de estos principios y con la ayuda de supuestos subsidiarios del tipo ya mencionado (la naturaleza de los mercados y el marco jurídico de la producción, etc.), es posible construir una teoría del equilibrio de la producción.\\

 El propósito de este ejemplo y muchos en la teoría económica es señalar la evidente deducción de una serie de postulados, que son supuestos que implican de alguna manera hechos simples e indiscutibles de la experiencia, relativos a la forma en que la escasez de bienes, que es el objetivo de nuestra ciencias, se manifiesta en el mundo de la realidad. En la teoría del valor el postulado es el hecho de que los individuos pueden ordenar sus preferencias. El postulado principal de la teoría de la producción es el hecho de que hay más de un factor de producción, el principal postulado de la teoría de la dinámica es el hecho de que no tenemos certeza sobre las escaseces futuras. Basta con enunciarlos para reconocerlos como evidentes.\\

 Ahora, bien la verdad de las deducciones de esta estructura depende, como siempre, de su coherencia lógica. Al igual que en las ciencias sociales debemos tener cuidado de investigar la naturaleza de nuestro objeto. Pero si bien es importante darse cuenta de cuántos son los supuestos secundarios que surgen necesariamente a medida que nuestra teoría se vuelve más y más complicada, es igualmente importante darse cuenta de cuán ampliamente aplicables son los supuestos principales sobre los que descansa. Como hemos visto, los principales son aplicables siempre y dondequiera que se den las condiciones que dan lugar a los fenómenos económicos. \\

 Estas consideraciones, deberían permitirnos detectar la falacia implícita siguiente: A veces se ha afirmado que las generalizaciones de la Economía tienen un carácter esencialmente 'histórico-relativo', que su validez se limita a determinadas condiciones históricas y que, fuera de ellas, carecen de relevancia para el análisis de los fenómenos sociales. Es cierto que, para aplicar de forma fructífera las proposiciones más generales de la Economía, es importante complementarlas con una serie de postulados auxiliares extraídos del análisis de lo que a menudo puede denominarse legítimamente material histórico-relativo. Pero no es cierto que los postulados principales sean histórico-relativos en el mismo sentido. Es cierto que se basan en la experiencia, que se refieren a la realidad. Pero se trata de una experiencia de un grado de generalidad tan amplio que los sitúa en una clase bastante diferente de los supuestos histórico-relativos. Nadie podría en duda los supuestos principales, y nadie que haya examinado estos supuestos puede dudar de la utilidad de partir de este plano. Pero si, como ha ocurrido notoriamente en la historia de las grandes controversias metodológicas, se interpretan en el sentido de que las conclusiones generales que se derivan del análisis general son tan limitadas como sus aplicaciones particulares, que las generalizaciones de la economía política sólo eran aplicables al estado de Inglaterra a principios del reinado de la reina Victoria, y argumentos similares, son totalmente engañosos. Por lo tanto, necesitamos un nuevo término para designar lo que se suele llamar histórico relativo. En Economía pura examinamos la implicación de la existencia de medios escasos con usos alternativos. Como hemos visto, el supuesto de las valoraciones relativas (cuál valoración ofrece las conclusiones más sólidas ) es la base de todas las complicaciones posteriores. \\

 Dada la diversas teorías y confusiones como el hedonismo psicológico en la teoría económica, es fundamentalmente importante distinguir entre la práctica real de los economistas, y la lógica que ésta implica y su ocasional apología ex post facto (defensa después de lo ocurrido). Inspeccionan con celo superfluo la fachada externa, pero evitan el trabajo intelectual de examinar la estructura interna. Por instancia, nadie que conociera la reciente teoría del valor podría seguir sosteniendo honestamente que tiene alguna conexión esencial con el hedonismo psicológico. Lo único que debemos suponer como economistas es el hecho evidente de que las distintas posibilidades ofrecen distintos incentivos y que estos incentivos pueden ordenarse según su importancia. El método científico exige que no se tenga en cuenta nada que no pueda observarse directamente. Podemos tener en cuenta la demanda tal como se manifiesta en el comportamiento observable del mercado. Pero no podemos ir más allá. La valoración es un proceso subjetivo. No podemos observar la valoración. Por lo tanto, no tiene cabida en una explicación científica. Nuestras construcciones teóricas deben partir de datos observables. A primera vista, esto parece muy plausible. El argumento de que no debemos hacer nada que no se haga en las ciencias físicas es muy seductor. Pero es dudoso que esté realmente justificado. Al fin y al cabo, lo nuestro es explicar ciertos aspectos de la conducta. Y es muy dudoso que esto pueda hacerse en términos que no impliquen ningún elemento psíquico.\\

 Pero incluso si restringimos el objeto de la Economía a la explicación de cosas tan observables como los precios, encontraremos que de hecho es imposible explicarlos a menos que invoquemos elementos de naturaleza subjetiva o psicológica. Si suponemos que tal sistema sólo tiene en cuenta los datos observables, nos engañamos a nosotros mismos. De ello se deduce que, si queremos hacer nuestro trabajo como economistas, si queremos ofrecer una explicación suficiente de cuestiones que toda definición de nuestra materia abarca necesariamente, debemos incluir elementos psicológicos. No podemos dejarlos fuera si queremos que nuestra explicación sea adecuada. Ahora, no es posible entender los conceptos de elección, de relación de medios y fines, los conceptos centrales de nuestra ciencia, en términos de observación de datos externos. \\

 Lo relevante para las ciencias sociales no es si los juicios de valor individuales son correctos en el sentido último de la filosofía del valor, sino si se realizan y si son eslabones esenciales en la cadena de explicación causal.

 \section{Capítulo 5: Generalizaciones Económicas y Realidad}

 Si la 'situación dada' se ajusta a un determinado patrón, también deben estar presentes otras características, ya que su presencia es 'deducible' del patrón postulado originalmente. Para ello, el método analítico es un instrumento para 'sacudir' todas las implicaciones de unos supuestos dados. Concedida la correspondencia de sus supuestos originales y los hechos, sus conclusiones son inevitables e ineludibles. Todo esto resulta especialmente claro si consideramos el procedimiento del análisis diagramático, que a explicado las implicaciones ocultas. Por ejemplo, que queremos mostrar los efectos sobre el precio de la imposición de un pequeño impuesto. Hacemos ciertas suposiciones en cuanto a la elasticidad de la demanda, ciertas suposiciones en cuanto a las funciones de coste, las plasmamos en el diagrama habitual, y podemos leer inmediatamente, por así decirlo, los efectos sobre el precio.\\
 Acá podemos sacar una función más para la investigación empírica, ésta puede sacar a la luz los hechos cambiantes que hacen posible la predicción en una situación dada. Éste es sin duda uno de los principales usos de los estudios aplicados: no desenterrar leyes 'empíricas' en un ámbito en el que no cabe esperar tales leyes, sino proporcionar de un momento a otro algún conocimiento de los datos fluctuantes en los que, en la situación dada, puede basarse la predicción. No puede sustituir al análisis formal. Pero puede sugerir en distintas situaciones qué análisis formal es apropiado, y puede proporcionar en ese momento algún contenido para las categorías formales. Por supuesto, si otras cosas no permanecen inalteradas, las consecuencias predichas no se derivan necesariamente de ello. Esta obviedad elemental, necesariamente implícita en cualquier predicción científica, debe mantenerse especialmente en primer plano de atención cuando se habla de este tipo de prognosis. Por lo tanto, las falsas afirmaciones de una ciencia que no tiene en cuenta los hechos queda al descubierto.

 \section{Capítulo 6: La importancia de la ciencia económica}
 La teoría del intercambio supone que puedo comparar la importancia que tiene para mí el pan a 6 unidades monetarias por barra y estas 6 gastadas en otras alternativas presentadas por las oportunidades del mercado. Y asume que el orden de mis preferencias así expuesto puede compararse con el orden de preferencias del panadero. Pero no asume que, en ningún momento, sea necesario comparar la satisfacción que obtengo del gasto de 6 unidades monetarias en pan con la satisfacción que obtiene el panadero al recibirlo. Esa comparación es de una naturaleza completamente diferente. Es una comparación que nunca es necesaria en la teoría del equilibrio y que nunca está implícita en los supuestos de dicha teoría. Es una comparación que queda necesariamente fuera del alcance de cualquier ciencia positiva. Afirmar que la preferencia de A está por encima de la de B en orden de importancia es totalmente distinto de afirmar que A prefiere n a m y B prefiere n y m en distinto orden. Implica un elemento de valoración convencional. Por tanto, es esencialmente normativa. No tiene cabida en la ciencia pura. Por lo tanto, dentro de su propia estructura de generalizaciones, no proporciona normas vinculantes en la práctica, es incapaz de decidir sobre la conveniencia de diferentes fines y es fundamentalmente distinta de la Ética. \\

 Para ser completamente racionales, debemos saber qué es lo que preferimos, debemos ser conscientes de las implicaciones alternativas Pues la racionalidad en la elección no es ni más ni menos que la elección con plena conciencia de las alternativas rechazadas. Y es justo aquí donde la Economía adquiere su significado práctico. Puede aclararnos las implicaciones de los distintos fines que podemos elegir. Nos permite querer sabiendo qué es lo que queremos. Nos permite elegir un sistema de fines que sean coherentes entre sí. Un ejemplo lo aclarará, No es racional querer un fin determinado si no se es consciente del sacrificio que implica la consecución de ese fin. Y, en esta suprema ponderación de alternativas, sólo una conciencia completa de las implicaciones del análisis económico moderno puede conferir la capacidad de juzgar racionalmente.

 Es muy posible que en la sociedad moderna existan diferencias en cuanto a los fines últimos que hagan inevitables algunos conflictos. Pero está claro que muchas de nuestras dificultades más acuciantes surgen, no por esta razón, sino porque nuestros objetivos no están coordinados. Como consumidores queremos lo barato, como productores elegimos la seguridad. Valoramos una distribución de los factores de producción como gastadores y ahorradores privados. Como ciudadanos públicos sancionamos los acuerdos que frustran la consecución de esta distribución. Pedimos dinero barato y precios más bajos, menos importaciones y un mayor volumen de comercio. Las diferentes 'organizaciones de voluntad' de la sociedad, aunque compuestas por los mismos individuos, formulan preferencias diferentes. A tal situación, la economía nos permite armonizar nuestras distintas opciones. No puede eliminar las limitaciones últimas de la acción humana. Pero sí hace posible que, dentro de esas limitaciones, actuemos con coherencia. Sirve al habitante del mundo moderno, con sus infinitas interconexiones y relaciones, como una extensión de su aparato perceptivo. Proporciona una técnica de acción racional. Así, podemos decir que la economía asume verdaderamente la reacionalidad de la sociedad humana. No pretende, como se ha afirmado tantas veces, que la acción sea necesariamente racional en el sentido de que los fines perseguidos no sean incompatibles entre sí. No se basa en el supuesto de que los individuos siempre actúen racionalmente. Pero sí depende para su razón de ser práctica de la suposición de que es deseable que lo hagan. Supone que, dentro de los límites de la necesidad, es deseable elegir fines que puedan alcanzarse armoniosamente. La afirmación de que la reacionalidad y la capacidad de elegir con conocimiento son deseables.  Si la irracionalidad, si la entrega a la fuerza ciega de los estímulos externos y al impulso descoordinado en cada momento es un bien que debe preferirse por encima de todos los demás, entonces es cierto que desaparece la razón de ser de la Economía. 

 \chapter{Economia y acción humana\\ Frank Knight}

 Si se abadona la explicación del comportamiento en términos de razones detrás de las decisiones económicas que toman los individuos. Se abren varias posibilidades alternativas. La más sencilla sea la analogía a la tendencia de la física: prescindir de toda 'explicación' y limitarse a formular leyes empíricas; el resultado es la teoría económica estadística, cuyo contenido se limita a los fenómenos objetivos de las mercancías y los precios.

 Otro podría ser el control social de la vida económica, cómo el socialismo de cátedra en Alemania o el fabianismo y liberalismo de izquierdas en Inglaterra o la fase de la economía institucionalista en Estados Unidos.\\

 Una tercera alternativa a la teoría explicativa es la de tratar los fenómenos económicos como esencialmente históricos. La economía histórica se destacan en:

 \begin{enumerate}[1]
     \item La primera, trata de la historia, en la medida de lo posible, en términos objetivos y empíricos, y puede utilizarse la estadística para descubrir y analizar tendencias. Análoga a las ciencias naturales.
     \item La segunda, utiliza las ideas humanísticas más familiares de la historia política y social: Ambición, esfuerzo y fracasos individuales en un entorno sociopsicológico determinado.
 \end{enumerate}

 A medida que la economía histórica llega a la generalización, puede describirse como economía institucional. En Alemania se lo denomina economía neohistórica o sociológica, con Sombart y Max Weber como sus lidere más destacados.\\

En la raíz de las diferencias y disputas entre la vieja y la nueva economía, así como entre las tres nuevas líneas de desarrollo teórico señaladas anteriormente, se encuentran dos problemas:

 \begin{itemize}
     \item  la relación entre descripción y explicación y 
     \item la relación entre exposición de hechos y evaluación crítica.
 \end{itemize}

 El primero, ineludible en cualquier pensamiento sobre la conducta humana, es fundamentalmente el problema de la realidad de la elección, o 'libertad de la voluntad'. Implica la esencia del problema del valor en el sentido de los valores individuales, y es en el fondo el problema de la relación entre el hombre individual y la naturaleza. El segundo problema básico tiene que ver con la relación entre el hombre individual y la sociedad. El hecho crucial en relación con el primer problema es que, si al motivo o al fin en cualquiera de sus formas se le concede algún papel real en la conducta, no puede ser el de una causa en el sentido de causalidad de la ciencia natural. Esta es la limitación suprema tanto de la economía estadística como de la histórica. En efecto, si se utiliza un motivo o un fin para explicar un comportamiento, éste debe, a su vez, ponerse en relación con los acontecimientos y las condiciones que lo preceden, y entonces el motivo resulta innecesario; el comportamiento se explicará plenamente por estos antecedentes. El motivo no puede tratarse como un acontecimiento natural. El contraste fundamental entre la causa y el efecto en la naturaleza y el fin y los medios en el comportamiento humano es la esencia de los hechos que plantean el problema de la interpretación del comportamiento. No parece posible hacer realidad los problemas humanos sin ver en la actividad humana un elemento de esfuerzo, contingencia y, lo que es más importante, de error, que por las mismas razones debe suponerse ausente de los procesos naturales.\\

 Así pues, el motivo o la intención se imponen en cualquier debate relevante sobre la actividad humana. Pero el tema del comportamiento no puede simplificarse hasta el punto de reducirlo a un dualismo. Es preciso introducir al menos tres principios básicos en su interpretación. 
\begin{itemize}
    \item La acción humana típica se explica en parte por la causalidad natural, 
    \item en parte, por una intención o deseo que es un dato absoluto y por tanto un hecho aunque no sea un acontencimiento o condición natural,
    \item y en parte por un impulso a realizar valores que no puede reducirse enteramente a deseos fácticos porque este impulso no tiene objetos literlmente descriptibles.
\end{itemize}

El segundo principio de explicación es quizás el más vulnerable de los tres. Es decir, es dudoso que un deseo sea absoluto, que exista algún deseo que no busque el logro de algún cambio en un sistema creciente de significado y valores. Todo acto, en el sentido económico, cambia la configuración de la materia en el espacio. Pero esto no excluye la posibilidad de 'actos' que cambien el significado y los valores sin cambiar la configuración natural, ya que la reflexión puede producir una nueva percepción y efectuar un cambio en los gustos personales. Más fundamentalmente, es dudoso que una configuración sea en sí misma preferible a otra.\\

Las personas manifiestan y sienten dos tipos diferentes de motivación para sus actos, 
\begin{itemize}
    \item Por un lado, el deseo o la preferencia, que el actor y las personas ajenas a él tratan como algo definitivo, como un hecho bruto.
    \item Por otro lado, las personas emiten juicios de valor de diversa índole para explicar sus actos, y la explicación desemboca en la justificación.
\end{itemize}
En otras palabras nadie puede tratar el motivo de forma objetiva o describir un motivo sin implicaciones de bueno y malo. Así, los hombres no sólo desean más o menos distinto de valorar, sino que desean porque valoran y también valoran sin desear. De hecho, la mayor parte de las valoraciones humanas, en relación con la verdad, la belleza y la moral, son en gran medida o totalmente independientes del deseo de cualquier cosa o resultado concreto. El hecho de que la motivación económica individual implique en sí misma una valoración y no un mero deseo queda establecido por otras dos consideraciones:
\begin{itemize}
    \item En primer lugar, lo que se elige en una transacción económica se quiere generalmente como medio para otro cosa, lo que implica un juicio de que realmente es un medio para el resultado en cuestión. 
    \item En segundo lugar, lo que se quiere en última instancia por sí mismo rara vez, o nunca, puede describirse finalmente en términos de configuración física, sino que debe definirse en relación con un universo de significados y valores. 
\end{itemize}
Así pues, hay un elemento de valoración en la noción de eficacia en la realización de un fin determinado; y, además, el fin real contiene como elemento un concepto de valor.\\

La concepción dual que se encuentra en la motivación se refleja también en el concepto más estrictamente económico de valor. Este último contiene definitivamente algo más que la noción de una cualidad medida por el precio; siempre se mide imperfectamente en condiciones reales. El precio 'tiende' a coincidir con el valor, pero la noción de valor también implica una norma a la que se ajustaría el precio en unas condiciones ideales. Esta normalmente incluye dos ideas:
\begin{itemize}
    \item La de un objetivo al que se aspira pero que sólo se alcanza de forma más o menos aproximada debido a errores de diversa índole;
    \item y la de un objetivo de acción correcto en contraste con los objetivos incorrectos, así como con el objetivo real.
\end{itemize}
En una sociedad basada en la competencia como principio aceptado, el precio competitivo, o precio igual a los costes necesarios de producción, es el valor verdadero en ambos sentidos;  las aberraciones deben atribuirse a dos grupos de casos:
\begin{itemize}
    \item errores de cálculo accidentales y,
    \item objetivos de acción incorrectos.
\end{itemize}

Para aclarar el punto principal es necesario observar la diferencia en la concepción de las condiciones ideales en economía y en mecánica. En este último campo, la más notable de las condiciones ideales es la ausencia de fricción; una concepción aparentemente similar de las condiciones ideales es una de las características familiares, casi un cliché, de la teoría económica. Como descripción generalizada, la concepción de la competencia perfecta, a la que se llega por abstracción de las características de la situación económica que hacen que la competencia sea imperfecta, es como las concepciones de la mecánica sin fricción y está igualmente justificada. Pero suponer que la abstracción de la teoría de la competencia perfecta tiene la misma relación con el comportamiento que la fricción con el proceso mecánico sería totalmente engañoso. La fricción en mecánica implica una transformación de energía de una forma a otra, de acuerdo con una ley tan rígida y un principio de conservación tan definido como la ley y el principio de conservación que son válidos para los cambios mecánicos en los que no desaparece la energía. No hay nada de esto en el proceso económico. Lo que se abstrae en la teoría de los precios de equilibrio es el hecho del error en el comportamiento económico.  No puede tratarse como una tendencia hacia un resultado objetivo, sino sólo como una tendencia a la conformidad con la intención del comportamiento, intención que no puede medirse ni identificarse ni definirse en términos de ningún dato experimental. Las condiciones ideales de la economía implican una valoración perfecta en un sentido limitado, un comportamiento económico perfecto que asume el fin o la intención como dados.\\

Hasta ahora se han examinado dos niveles de interpretación del comportamiento económico. 
\begin{itemize}
    \item El primero es aquel en el que el comportamiento se reduce en la medida de lo posible a principios de regularidad mediante un procedimiento estadístico, 
    \item el segundo es la interpretación del comportamiento en términos de motivación, que debe centrarse en la diferencia entre motivo y acto y en el hecho del error. 
    \item Es en el tercer nivel de interpretación donde el fin intencional de la acción en sí se somete a valoración o crítica desde algún punto de vista.
\end{itemize}
Aquí la relación entre individuo y sociedad, el segundo problema principal sugerido anteriormente, y el concepto de valor en relación con la política social se convierten en temas centrales de discusión. De hecho, incluso en el segundo nivel deben reconocerse dos formas de referencia social:
\begin{itemize}
    \item Los fines individuales, tal como se dan, son principalmente sociales en su origen y contenido, y
    \item en las sociedades en las que el pensamiento económico tiene alguna relevancia existe una gran aceptación y aprobación ético-social de la motivación individual en abstracto.
\end{itemize}
La sociedad moderna, por ejemplo, obtiene la libertad individual siendo un valor social y no un mero hecho. Si la noción de comportamiento económico se separa efectivamente del proceso mecánico, si los fines se consideran como fines y no como meros efectos físicos, la discusión se sitúa ya en gran parte en el tercer nivel. Los fines físicos como deseados no pueden mantenerse a menos que se les otorgue un gran elemento de valoración además del deseo. No se puede sostener que los "deseos" de bienes y servicios económicos sean definitivos o tengan una realidad autónoma e independiente. El menor escrutinio muestra que son, en gran medida, manifestaciones más bien accidentales del deseo de algo de la naturaleza de la libertad o el poder. Pero tales objetos de deseo son formas de relación social y no cosas, y la noción de eficiencia económica sólo tiene una aplicabilidad limitada a su búsqueda y consecución. El tratamiento de tales actividades, si ha de tener algún atractivo general y serio, debe ser una discusión de política social relativa a los fines o normas sociales y al procedimiento social para realizarlos.\\

En este campo predomina el interés por los valores y, sobre todo, por la política social. Así, la teoría económica, que creció en un ambiente de reacción contra el control, puso claramente demasiado énfasis en este lado del asunto y descuidó el otro. Ahora es igual de obvio que existen limitaciones igualmente amplias y complejas al principio de libertad en el sentido económico, es decir, a la organización de la vida económica exclusivamente a través del libre contrato entre individuos que utilizan determinados recursos para alcanzar determinados fines individuales. La sociedad no puede aceptar los fines individuales y los medios individuales como datos o como objetivos principales de su propia política. En primer lugar, simplemente no son datos, sino que se crean históricamente en el propio proceso social y se ven inevitablemente afectados por la política social. En segundo lugar, la sociedad no puede ser ni siquiera relativamente indiferente al funcionamiento del proceso. Hacerlo sería, en última instancia, destructivo tanto para la sociedad como para el individuo. Esta conclusión se ve fuertemente reforzada por el hecho de que el interés inmediato del individuo es en gran medida competitivo, centrado en su propio avance social en relación con otros individuos.\\

Estas reflexiones apuntan a un error lógico subyacente en la teoría del valor típica de los economistas clásicos.  No sostenían ostensiblemente que la libertad como tal fuera un bien. Notoriamente, eran hedonistas; su argumento a favor de la libertad la hacía instrumental para el placer, basándose en que el individuo es mejor juez que los funcionarios del gobierno sobre los medios para su felicidad. Ciertamente, un individuo puede desear la libertad y reclamar el derecho a ella sin sostener que tomará decisiones uniformemente más sabias que las que se tomarían por él, desde el punto de vista de su propia comodidad y seguridad material. Y con la misma certeza se puede sostener que el individuo debe, dentro de unos límites, tomar sus propias decisiones y atenerse a sus consecuencias, aunque no decida hacerlo. En otras palabras, los economistas clásicos no se dieron cuenta, y el espíritu 'científico' de la época ha hecho que los economistas en general se resistan a admitir que la libertad es esencialmente un valor social, al menos cuando se defiende o se opone a ella, como lo es cualquier otro sistema social o relación social.\\

Los intereses o deseos reales expresados en el comportamiento económico son, en su inmensa mayoría, sociales en su génesis y en su contenido; por consiguiente, no pueden describirse al margen de un sistema de relaciones sociales que, a su vez, no puede tratarse en términos puramente objetivos y reales. Hasta cierto punto, un individuo puede concebirlos en tales términos. Pero las partes de una comunicación de este tipo se sitúan en el papel de espectadores más que en el de miembros de la sociedad o participantes en el fenómeno. \\
En este conflicto entre el interés del espectador por ver y comprender y el interés del participante por actuar y cambiar, el filósofo o el metodólogo no pueden tomar partido. A la pregunta de si la economía como tal debe ser una cosa o la otra sólo se puede responder reconociendo que debe ser ambas cosas, con más o menos énfasis en un sentido o en otro según los objetivos de un tratamiento concreto; pero siempre, implícitamente, debe ser ambas cosas, por muy unilateral que sea el énfasis, ya que cada interés presupone y es relativo al otro, y todo escritor y lector, como ser humano, está motivado por ambos intereses. Lo deseable es que en cualquier declaración quede clara la relación entre los dos grupos de intereses. Pero lo que tiende a ocurrir es lo contrario: aquel cuyo interés es principalmente la verdad tiende a reforzar sus afirmaciones identificando verdad y valor, y aquel cuyo interés son los valores tiende a reforzar sus afirmaciones dándoles la cualidad de verdad.\\

Mientras que en el período de desarrollo de la economía clásica el interés práctico social se centraba casi exclusivamente en la liberación de un sistema anticuado de control, en la actualidad el péndulo ha oscilado definitivamente en sentido contrario. La sociedad busca positivamente una base de unidad y orden en lugar de intentar negativamente abandonar una base insatisfactoria. Además, las normas actuales de pensamiento han caído bajo el dominio extremo del ideal científico, que tiene poca o ninguna aplicabilidad al problema. No hay solución intelectual para los conflictos de intereses. Sólo pueden discutirse los valores, pero la discusión no conduce necesariamente a un acuerdo; y el desacuerdo sobre los principios parece requerir moralmente una apelación a la fuerza. También es interesante señalar que la tendencia a la "racionalización" hace que el conflicto de intereses y el desacuerdo sobre los principios adquieran cada uno la cualidad de su opuesto, y que en la práctica se mezclen inseparablemente.\\

Tanto las escuelas 'fascistas' como las 'comunistas' se inclinan a tratar la verdad o falsedad de las proposiciones en economía como una cuestión indiferente o incluso ilusoria, juzgando las doctrinas sólo por su conductividad hacia el establecimiento del tipo de orden social deseado. Este punto de vista es, por supuesto, "falso" desde un punto de vista "científico" más restringido; en cualquier orden social, los resultados de ciertas elecciones que afectan a la producción y al consumo, las haga quien las haga, se rigen por ciertos principios abstractos, esencialmente matemáticos, que expresan la diferencia entre economía y despilfarro. \\

En el otro extremo -en el primer y segundo nivel de interpretación indicados anteriormente- existe un movimiento igualmente enérgico en aras de un tratamiento rigurosamente 'científico' de la economía. El análisis en el primer nivel, que prescinde de la motivación y sólo considera los resultados de la acción en forma de estadísticas de mercancías, no deja lugar real a ningún concepto de economía. Además, no puede llevarse a cabo ni siquiera literalmente, ya que las mercancías deben nombrarse y clasificarse y el tratamiento debe tener en cuenta las similitudes y diferencias de uso, así como las características físicas. Y la economía en el segundo nivel, tratando los deseos como hechos, está sujeta a limitaciones muy estrechas. En realidad, los deseos no tienen un contenido muy definido, y de lo que tienen los estudiantes no pueden tener un conocimiento definitivo. La concepción puede convertirse en la base de una teoría puramente abstracta, pero tiene poca aplicación a la realidad. Para dar contenido a los datos, los deseos deben identificarse con los bienes y servicios en los que se expresan, y el segundo método se reduce entonces a la identidad con el primero. Además, los únicos deseos que pueden asimilarse a los datos científicos son puramente individuales, y cualquier debate sobre política social debe basarse en valores o ideales totalmente ajenos a este sistema.

\chapter{Textos selectos sobre economía, historia y Ciencias Sociales\\ Karl Marx}
Aquí se reproducen tres textos. El primero, 'Trabajo enajenado' de los Manuscritos económicos y filosóficos de Marx de 1844, brinda una visión general de su visión de la forma en que las relaciones económicas entre las personas y los productos de esas relaciones dominan a las mismas personas que crean y mantienen esas relaciones. El segundo, el 'Prefacio' de Marx a Una contribución a la crítica de la economía política, esboza muy brevemente el materialismo histórico de Marx, según el cual el estado de la tecnología determina las relaciones económicas entre las personas, que a su vez determinan las relaciones legales y políticas y el curso de la historia. El tercero, 'El método de la economía política', que es una sección de la 'Introducción' a Una contribución a la crítica de la economía política, contiene la discusión más explícita y sostenida de Marx sobre la metodología económica.

\section{Trabajo enajenado}
Hemos partido de las premisas de la economía política. Hemos aceptado su lenguaje y sus leyes. Hemos supuesto la propiedad privada; la separación entre trabajo, capital y tierra, y entre salario, beneficio y capital; la división del trabajo; la competencia; la concepción del valor de cambio, etc. A partir de la propia economía política, utilizando sus propias palabras, hemos demostrado que el trabajador se reduce al nivel de una mercancía, y más aún la mercancía más miserable de todas; que la miseria del trabajador es inversamente proporcional a la potencia y al volumen de su producción; que la consecuencia necesaria de la competencia es la acumulación del capital en pocas manos y, por consiguiente, el restablecimiento del monopolio bajo una forma más terrible; y que, finalmente, la distinción entre capitalista y propietario, entre obrero agrícola y obrero industrial, desaparece y toda la sociedad debe dividirse en las dos clases de propietarios y obreros sin propiedad.\\

La economía política procede y capta tomando como leyes el proceso de la propiedad privada. No la explica ni la comprende. Tampoco explica la relación entre trabajo y capital. La economía política no nos enseña nada sobre hasta que punto las circunstancias externas y aparentemente accidentales son sólo la expresión de un desarrollo necesario. Tales podrían ser el intercambio, la relación trabajo y capital, y los ya mencionados. Las únicas piezas que la economía política pone en movimiento son la codicia y la guerra de los avaros: la competencia.\\

La competencia, la libertad artesanal y la división de la propiedad territorial sólo se desarrollaron y concibieron como consecuencias accidentales, deliberadas y violentas del monopolio, de los gremios y de la propiedad feudal, y no como sus consecuencias necesarias, inevitables y naturales. Ahora, debemos comprender la conexión esencial entre la propiedad privada, la codicia, la separación del trabajo, el capital, y la propiedad de la tierra, el intercambio y la competencia, el valor y la devaluación del hombre, el monopolio y la competencia, etc. Es decir, la conexión entre todo este sistema de enajenación y el sistema monetario.\\

Dado que la economía política asume como un hecho en forma de historia lo que debe de explicar, debemos partir de un hecho económico real. \\

El trabajador se convierte en una mercancía más barata cuantas más mercancías produce. El trabajo no sólo produce mercancías; también se produce a sí mismo y a los trabajadores como mercancía y lo hace en la misma proporción en que produce mercancías en general. Lo que significa que el objeto que el trabajo produce, su producto, se opone a él como algo ajeno, como un poder independiente del productor. El producto del trabajo es el trabajo encarnado y materializado en un objeto, es la materialización del trabajo. Hasta tal punto aparece la realización del trabajo como pérdida de realidad que el trabajador pierde su realidad hasta el punto de morir de hambruna. Hasta tal punto aparece la realización del trabajo como pérdida de realidad que el trabajador pierde su realidad hasta el punto de morir de hambruna. El trabajo mismo se convierte en un objeto que sólo puede obtener mediante un enorme esfuerzo y con interrupciones espasmódicas. Hasta tal punto la apropiación del objeto aparece como enajenación que cuantos más objetos produce el trabajador menos puede poseer y más cae bajo la dominación de su producto, del capital. \\

Todas estas consecuencias están contenidas en esta característica: que el trabajador se relaciona con el producto del trabajo como con un objeto extraño. Pues es evidente que, según esta premisa, cuanto más se esfuerza el trabajador en su labor, tanto más poderoso se vuelve el mundo ajeno y objeto que él hace nacer frente a sí mismo, tanto más se empobrecen él y su mundo interior, y tanto menos le pertenecen. La exteriorización del trabajador en su producto significa no sólo que su trabajo se convierte en un objeto, en una existencia externa, sino que existe fuera de él, independientemente de él y ajeno a él, y comienza a enfrentarse a él como un poder autónomo; que la vida que él ha otorgado al objeto se enfrenta a él como hostil y ajena.\\

Veamos ahora más de cerca la objetivación, la producción del trabajador, y el extrañamiento, la pérdida del objeto, de su producto, que ello conlleva.\\

El trabajador no puede hacer nada sin la naturaleza, por medio del cual produce. Pero así como la naturaleza proporciona al trabajo los medios de vida, en el sentido de que el trabajo no puede vivir sin objetos sobre los que ejercitarse, también proporciona los medios de vida en sentido estricto, es decir, los medios de subsistencia física del trabajador. Cuanto más se apropia el trabajador de la naturaleza por medio de su trabajo, tanto más se priva a si mismo de los medios de vida, esto se desglosa en dos aspectos:
\begin{enumerate}[1.]
    \item  El mundo exterior sensual se convierte cada vez menos en un objeto perteneciente a su trabajo.
    \item Se convierte cada vez menos en un medio de vida en sentido inmediato, en un medio para la subsistencia física del trabajador.
\end{enumerate}

En estos dos aspectos, pues, el trabajador se convierte en esclavo de su objeto; en primer lugar, en cuanto que recibe un objeto de trabajo, es decir, recibe trabajo, y, en segundo lugar, en cuanto que recibe medios de subsistencia. En primer lugar, pues, para que pueda existir como trabajador, y en segundo lugar como sujeto físico. La culminación de esta esclavitud es que sólo como trabajador puede mantenerse como sujeto físico y sólo como sujeto físico es trabajador.\\

El extrañamiento del trabajador en su objeto se expresa según las leyes de la economía política de la siguiente manera:

\begin{enumerate}[1.]
    \item cuanto más produce el trabajador, menos tiene que consumir;
    \item cuanto más valor crea, más inútil se vuelve;
    \item cuanto más moldeado está su producto, más deformado está el trabajador;
    \item cuanto más civilizado es su objeto, más salvaje es el trabajador;
    \item cuanto más poderoso es el trabajo, más impotente es el trabajador;
    \item cuanto más inteligente es el trabajo, más embrutecido es el trabajador y más se convierte en esclavo de la naturaleza.
\end{enumerate}

La economía política oculta el distanciamiento en la naturaleza del trabajo ignorando la relación directa entre el trabajador (trabajo) y la producción. Por lo tanto, cuando preguntamos cuál es la relación esencial del trabajo, estamos preguntando por la relación del trabajador con la producción.\\

Hasta ahora, hemos considerado el extrañamiento, la alienación del trabajador, sólo desde un aspecto, es decir, la relación del trabajador con los productos de su trabajo. Pero el extrañamiento se manifiesta no sólo en el resultado, sino también en el acto de producción, dentro de la propia actividad de producción. De modo tal que si el producto del trabajo es enajenación, la producción misma debe ser enajenación activa.\\

¿En que consiste la enajenación del trabajo?

En primer lugar, el hecho de que el trabajo es exterior al trabajador; es decir, no pertenece a su ser esencial; se siente miserable. Por lo que se sentirá a si mismo cuando no esté trabajando. Así, su trabajo no es voluntario. No es pues la satisfacción de una necesidad, si no un mero medio para satisfacer necesidades ajenas a él. Queda claramente demostrado que mientras no existe coacción física, se rechaza como peste. Por último, el carácter externo del trabajo para el obrero queda demostrado por el hecho de que no le pertenece a él, sino a otro. El resultado es que el hombre (el trabajador) siente que actúa libremente sólo en sus funciones animales - comer, beber y procrear, o a lo sumo en su vivienda y adorno, mientras que en sus funciones humanas no es más que animal.\\

Hemos considerado el cato de extrañamiento de la actividad humana práctica, del trabajo, desde dos aspecto:

\begin{enumerate}[1.]
    \item La relación del trabajador con el producto del trabajo como objeto extraño que tiene poder sobre él. Esta relación es, al mismo tiempo, la relación con el mundo exterior sensual, con los objetos naturales, como un mundo ajeno que se enfrenta a él, en oposición hostil. 
    \item La relación del trabajo con el acto de producción dentro del trabajo. Esta relación es la relación del trabajador con su propia actividad como algo ajeno, como una actividad dirigida contra sí mismo, que es independiente de él y no le pertenece.
\end{enumerate}

De las dos características del trabajo enajenado tenemos que deducir una tercera.\\

El hombre es un ser-especie, no sólo porque práctica y teóricamente hace de las especies su objeto, si también porque se considera a sí mismo como la especie viva presente, ya que se considera a sí mismo como un ser universal y por tanto libre.\\

La universalidad del hombre se manifiesta en la práctica en esa universalidad que hace de toda la naturaleza su cuerpo inorgánico, 

\begin{enumerate}[1.]
    \item como medio directo de vida y 
    \item como materia, objeto e instrumento de su actividad vital.
\end{enumerate}

El hombre vive de la naturaleza -es decir, la naturaleza es su cuerpo- y debe mantener un diálogo continuo con ella si no quiere morir. Decir que la vida física y mental del hombre está ligada a la naturaleza significa simplemente que la naturaleza está ligada a sí misma, ya que el hombre forma parte de la naturaleza.\\

El trabajo enajenado no sólo 

\begin{enumerate}[1.]
    \item aleja a la naturaleza del hombre y 
    \item aleja al hombre de sí mismo, de su propia función, de su actividad vital; 
\end{enumerate}
por ello, también aleja al hombre de su especie. Convierte su vida-especie en un medio para su vida individual. En primer lugar, separa la vida de la especie de la vida individual y, en segundo lugar, convierte esta última, en su forma abstracta, en el fin de la primera, también en su forma abstracta y separada.

Porque, el trabajo, la actividad vital, la vida productiva en si misma, se presenta al hombre sólo como un medio para la satisfacción de una necesidad, la necesidad de preservar la existencia física. Pero la vida productiva es vida-especie. Es vida que produce vida. El hombre hace de su propia actividad vital un objeto de su voluntad y de su conciencia. Tiene una actividad vital consciente, es esto lo que distingue al animal. Por lo que es un ser consciente; es decir, su propia vida es un objeto para él. Así su actividad es libre. Pero, el trabajo enajenado invierte esta relación, de modo que el hombre, por el hecho de ser un ser consciente, convierte su actividad vital, su ser esencial, e un medio para su existencia. \\

Por lo tanto, el objeto del trabajo es la objetivación de la vida-especie del hombre: porque el hombre se produce a sí mismo no sólo intelectualmente, en su conciencia, sino activa y realmente, y por ello puede contemplarse a sí mismo en un mundo que él mismo ha creado.  Al arrancarle al hombre el objeto de su producción, el trabajo enajenado le arranca su vida de especie, su verdadera objetividad de especie, y transforma su ventaja sobre los animales en la desventaja de que se le arrebata su cuerpo inorgánico, la naturaleza. \\

Del mismo modo que el trabajo enajenado reduce la actividad espontánea y libre a un medio, convierte la vida de especie del hombre en un medio de su existencia física. La conciencia, que el hombre tiene de su especie, se transforma mediante el trabajo enajenado de modo que la vida de especie se convierte para él en un medio.

\begin{enumerate}[1.]
    \item[3.] El trabajo enajenado, por tanto, convierte al ser-especie del hombre es un ser ajeno a él y en medio de su existencia individual. Aleja al hombre de su propio cuerpo de la naturaleza tal que como existe fuera de él, de su esencia espiritual, de sus existencia humana.
    \item[4.] Una consecuencia inmediata del alejamiento del hombre del producto de su trabajo, de su actividad vital, de su ser-especie, es el alejamiento del hombre del hombre. Cuando el hombre se enfrenta a sí mismo, se enfrenta también a los demás hombres. Lo que es cierto de la relación del hombre con su trabajo, con el producto de su trabajo y consigo mismo, también lo es de su relación con otros hombres y con el trabajo y el objeto del trabajo de otros hombres. En general, la proposición de que el hombre está alejado de su especie significa que cada hombre está alejado de los demás y que todos están alejados de la esencia del hombre.\\
\end{enumerate}

En la relación de trabajo enajenado, cada hombre considera, pues, al otro de acuerdo con la norma y la situación en que, como trabajador, se encuentra.\\

Partimos de un hecho económico, el distanciamiento del trabajador y de su producción. Hemos dado a este hecho una forma conceptual: el trabajo enajenado, alienado. Hemos analizado este concepto, y al hacerlo nos hemos limitado a analizar un hecho económico.\\
Veamos ahora cómo debe expresarse y presentarse en la realidad el concepto de trabajo enajenado y alienado.\\

Si el producto del trabajo me es ajeno y se enfrenta a mí como un poder extraño, ¿a quién pertenece entonces?, ¿a un ser distinto? ¿quién es ese ser?.\\

El ser extraño a quien pertenecen el trabajo y el producto del trabajo, a cuyo servicio se realiza el trabajo y para cuyo disfrute se crea el producto del trabajo, no puede ser otro que el hombre mismo. Si el producto del trabajo no pertenece al trabajador, y si se enfrenta a él como un poder ajeno, esto sólo es posible porque pertenece a un hombre distinto del trabajador. Si su actividad es un tormento para él, debe proporcionar placer y disfrute a alguien más. No los dioses, no la naturaleza, sino sólo el hombre mismo puede ser este poder ajeno sobre los hombres. Ésta proposición estará enunciada como:
\begin{center}
    La relación del hombre consigo mismo sólo se hace objetiva y real para él a través de su relación con otros hombres.
\end{center}

Si consideramos el producto de su trabajo, su trabajo objetivado, como un objeto ajeno, hostil y poderoso que es independiente de él, entonces su relación con ese objeto es tal que otro hombre - ajeno, hostil, poderoso e independiente de él - es su amo. Si se relaciona con su propia actividad como actividad no libre, entonces se relaciona con ella como actividad al servicio, bajo el dominio, la coacción y el yugo de otro hombre.\\
Del mismo modo que crea su propia producción como una pérdida de realidad, un castigo, y su propio producto como una pérdida, un producto que no le pertenece, crea la dominación del no productor sobre la producción y su producto. Del mismo modo que aleja de sí mismo su propia actividad, confiere al extraño una actividad que no le pertenece.\\

La relación del trabajador con el trabajo crea la relación del capitalista -o cualquier otra palabra que se elija para designar al amo del trabajo-con ese trabajo. La propiedad privada es, por tanto, el producto, el resultado y la consecuencia necesaria del trabajo enajenado, de la relación externa del trabajador con la naturaleza y consigo mismo. La propiedad privada deriva, pues, del análisis del concepto de trabajo enajenado, es decir, del hombre enajenado, del trabajo enajenado, de la vida enajenada, del hombre enajenado. Pero del análisis de este concepto se desprende claramente que, aunque la propiedad privada aparece como la base y la causa del trabajo enajenado, es su consecuencia.\\
Sólo cuando el desarrollo de la propiedad privada llega a su punto de culminación, resurge éste, su secreto, a saber, que es
\begin{enumerate}[a)]
    \item el producto del trabajo enajenado, y
    \item el medio a través del cual se enajena el trabajo, la realización de esta enajenación.
\end{enumerate}

Esta evolución arroja luz sobre una serie de controversias hasta ahora no resueltas.


\begin{enumerate}[(1)]
    \item La economía política parte del trabajo como verdadera alma de la producción y, sin embargo, no concede nada al trabajo y todo a la propiedad privada. Proudhon ha resuelto esta contradicción pronunciándose a favor del trabajo y en contra de la propiedad privada. Pero hemos visto que esta contradicción aparente es la contradicción del trabajo enajenado consigo mismo y que la economía política no ha hecho más que formular leyes del trabajo enajenado. Por consiguiente, para nosotros, el salario y la propiedad privada son idénticos: pues allí el producto, el objeto del trabajo, paga el trabajo mismo, el salario no es más que una consecuencia necesaria de la enajenación del trabajo; del mismo modo, cuando se trata del salario, el trabajo no aparece como un fin en sí mismo, sino como el servidor del salario. Luego, un aumento forzoso de los salarios no sería más que una mejor paga para los esclavos y no significaría un aumento de la  trascendencia humana ni de la dignidad ni para el trabajador ni para la mano de obra. El salario es una consecuencia inmediata del trabajo enajenado, y el trabajo enajenado es la causa inmediata de la propiedad privada. Si uno cae, entonces el otro debe caer también.
    \item De la relación del trabajo enajenado con la propiedad privada se deduce además que la emancipación de la sociedad de la propiedad privada, etc., de la servidumbre, se expresa en la forma política de la emancipación de los trabajadores. No porque se trate sólo de su emancipación, sino porque en su emancipación está contenida la emancipación humana universal. La razón de esta universalidad es que toda la servidumbre humana está implicada en la relación del trabajador con la producción, y todas las relaciones de servidumbre no son más que modificaciones y consecuencias de esta relación.
\end{enumerate}

Así como hemos llegado al concepto de propiedad privada a través del análisis del concepto de trabajo enajenado y alienado, con la ayuda de estos dos factores es posible evolucionar todas las categorías económicas, y en cada una de estas categorías -por ejemplo, comercio, competencia, capital, dinero- identificaremos sólo una expresión particular y desarrollada de estos constituyentes básicos. 

Pero, antes de pasar a considerar esta configuración, tratemos de resolver otros dos problemas.

\begin{enumerate}[(1)]
    \item Tenemos que determinar la naturaleza general de la propiedad privada, tal como ha surgido del trabajo enajenado, en su relación con la propiedad verdaderamente humana y social.
    \item Hemos tomado la enajenación del trabajo, su alienación, como un hecho y hemos analizado ese hecho. ¿Cómo, nos preguntamos ahora, llega el hombre a alienar su trabajo, a enajenarlo? ¿Cómo se fundamenta este distanciamiento en la naturaleza del desarrollo humano? Ya hemos avanzado mucho en la solución de este problema al transformar la cuestión del origen de la propiedad privada en la cuestión de la relación del trabajo enajenado con el curso del desarrollo humano. En efecto, cuando se habla de propiedad privada, se piensa que se trata de algo exterior al hombre. Al hablar de trabajo, se trata inmediatamente del hombre mismo. Esta nueva manera de formular el problema contiene ya su solución. \\
ad (1): El carácter general de la propiedad privada y su relación con la propiedad verdaderamente humana.
\end{enumerate}

El trabajo enajenado se ha resuelto para nosotros en dos partes componentes, que se condicionan mutuamente, o que no son más que expresiones diferentes de una misma relación. La apropiación aparece como extrañamiento, como alienación; y la alienación aparece como apropiación, el extrañamiento como verdadera admisión a la ciudadanía.\\

Hemos considerado el único aspecto -el trabajo enajenado en relación con el propio trabajador-, es decir, la relación del trabajo enajenado consigo mismo. Y como producto, como consecuencia necesaria de esta relación, hemos encontrado la relación de propiedad del no-trabajador con el trabajador y con el trabajo. La propiedad privada como expresión material y resumida del trabajo enajenado abarca ambas relaciones: la relación del trabajador con el trabajo y con el producto de su trabajo y con los no trabajadores, y la relación del no trabajador con el trabajador y con el producto de su trabajo.\\

Ya hemos visto que, en relación con el trabajador que se apropia de la naturaleza mediante su trabajo, la apropiación aparece como extrañamiento, la autoactividad como actividad para otro y de otro, la vitalidad como sacrificio de la vida, la producción de un objeto como pérdida de ese objeto en favor de un poder extraño, de un hombre extraño. Consideremos ahora la relación entre este hombre, ajeno al trabajo y al obrero, y el obrero, el trabajo y el objeto del trabajo. Lo primero que hay que señalar es que todo lo que aparece para el trabajador como una actividad de alienación, de extrañamiento, aparece para el no-trabajador como una situación de alienación, de extrañamiento.\\

En segundo lugar, la actitud real y práctica del trabajador en la producción y ante el producto (como estado de ánimo) aparece para el no-trabajador que se enfrenta a él como una actitud teórica. \\

En tercer lugar, el no-trabajador hace contra el trabajador todo lo que el trabajador hace contra sí mismo, pero no hace contra sí mismo lo que hace contra el trabajador.\\

Veamos más de cerca estas tres relaciones

\section{El método de la economía política}

Cuando examinamos un país determinado desde el punto de vista de la economía política, empezamos por su población, la división de la población en clases, la ciudad y el campo, el mar, las diferentes ramas de la producción, la exportación y la importación, la producción y el consumo anuales, los precios, etc. Pero esto es erróneo, ya que la población es una abstracción si, se prescinde de las clases que la componen. Estas clases a su vez, siguen siendo términos vacíos si no se conocen los factores de los que dependen; Es decir, el trabajo asalariado, el capital, etc. En resumen si se toma la población como punto de partida, se trataría de una noción muy vaga de un todo complejo y de términos concretos imaginarios se pasaría a abstracciones cada vez más tenues hasta llegar a las definiciones más simples. A partir de ahí habría que hacer de nuevo el recorrido en sentido inverso hasta llegar de nuevo al concepto de población, que esta vez no es una noción vaga de un todo, sino una totalidad que comprende muchas determinaciones y relaciones.  Los economistas del siglo XVII, por ejemplo, tomaron siempre como punto de partida el organismo vivo, la población, la nación, el Estado, varios Estados, etc., pero el análisis les condujo siempre al final al descubrimiento de unas pocas relaciones generales abstractas decisivas, como la división del trabajo, el dinero y el valor. Cuando se dedujeron y establecieron más o menos claramente estos factores separados, se desarrollaron sistemas económicos que, partiendo de conceptos simples, como trabajo, división del trabajo, demanda, valor de cambio, avanzaron hasta categorías como Estado, intercambio internacional y mercado mundial.\\

Hegel concibió la idea falaz de que el mundo real es el resultado del pensamiento que provoca su propia síntesis, su propia profundización y su propio movimiento; mientras que el método de avanzar de lo abstracto a lo concreto es simplemente el modo en que el pensamiento asimila lo concreto y lo reproduce como categoría mental concreta. Sin embargo, esto no es en absoluto el proceso de evolución del propio mundo concreto. Por ejemplo, la categoría económica más simple, el valor de cambio, presupone una población, una población que produce en condiciones definidas, así como un tipo distinto de familia, de comunidad, de Estado, etc. El valor de cambio sólo puede existir como relación abstracta y unilateral de un todo orgánico concreto ya existente.\\

El sujeto, la sociedad, debe ser considerado siempre como la condición previa de la comprensión, incluso cuando se emplea el método teórico.\\

La categoría más simple aparece, pues, como una relación de simples comunidades familiares o tribales con la propiedad. En las sociedades que han alcanzado un estadio superior, la categoría aparece como una relación comparativamente simple existente en una comunidad más avanzada. No obstante, se puede concluir que las categorías simples representan relaciones o condiciones que pueden reflejar la situación concreta inmadura sin plantear todavía la relación o condición más compleja que se expresa conceptualmente en la categoría más concreta.\\
El procedimiento de razonamiento abstracto que avanza de los conceptos más simples a los más complejos se ajusta en esa medida al desarrollo histórico real. \\

Es cierto, por otra parte, que hay ciertas formaciones sociales muy desarrolladas, pero sin embargo históricamente inmaduras, que emplean algunas de las formas económicas más avanzadas. Aunque la categoría más simple, por lo tanto, puede haber existido históricamente antes que la categoría más concreta, su completo desarrollo intensivo y extensivo puede, sin embargo, ocurrir en una formación social compleja, mientras que la categoría más concreta puede haber evolucionado completamente en una formación social más primitiva.\\

El trabajo parece ser una categoría muy simple. La noción de trabajo en esta forma universal, como trabajo en general, es también extremadamente antigua. Sin embargo, el 'trabajo' en su simplicidad se considera económicamente una categoría tan moderna como las relaciones que dan lugar a esta simple abstracción.  El sistema monetario, por ejemplo, sigue considerando la riqueza de forma bastante objetiva como una cosa que existe independientemente en forma de dinero.\\

Las abstracciones más generales surgen en conjunto sólo cuando el desarrollo concreto es más profuso, de modo que una cualidad específica se ve como común a muchos fenómenos, o común a todos. Entonces ya no se percibe únicamente bajo una forma particular. El trabajo, no sólo como categoría sino en realidad, se ha convertido en un medio para crear riqueza en general, y ha dejado de estar vinculado como atributo a un individuo concreto.\\
El ejemplo del trabajo demuestra de manera sorprendente cómo incluso las categorías más abstractas, a pesar de su validez en todas las épocas -precisamente porque son abstracciones- son igualmente un producto de las condiciones históricas incluso en la forma específica de abstracciones, y conservan su plena validez sólo para y en el marco de estas condiciones.\\

Al igual que en general cuando se examina cualquier ciencia histórica o social, también en el caso del desarrollo de las categorías económicas es necesario recordar siempre que el sujeto, en este contexto la sociedad burguesa contemporánea, se presupone tanto en la realidad como en la mente, y que, por tanto, las categorías expresan formas de existencia y condiciones de existencia -y a veces simplemente aspectos separados- de esta sociedad particular, el sujeto; así pues, la categoría, incluso desde el punto de vista científico, no comienza en absoluto en el momento en que se discute como tal. 

En todas las formas en las que la propiedad de la tierra es el factor decisivo, siguen predominando las relaciones naturales; en las formas en las que el factor decisivo es el capital, predominan los elementos sociales, históricamente evolucionados. La renta no puede entenderse sin el capital, pero el capital puede entenderse sin la renta. El capital es el poder económico que lo domina todo en la sociedad burguesa. Debe constituir tanto el punto de partida como la conclusión y tiene que ser expuesto antes que la propiedad inmobiliaria. Después de analizar el capital y la propiedad inmobiliaria por separado, hay que examinar su interconexión.\\

Es precisamente el predominio de los pueblos agrícolas en el mundo antiguo lo que hizo que las naciones mercantiles - fenicios, cartagineses - se desarrollaran con tal pureza (precisión abstracta). Pues el capital en forma de capital mercantil o monetario aparece en esa forma abstracta en la que el capital aún no se ha convertido en el factor dominante de la sociedad.\\

Es evidente que la disposición del material debe hacerse de tal manera que [la sección] uno comprenda definiciones generales abstractas, que por lo tanto pertenecen en cierta medida a todas las formaciones sociales, pero en el sentido expuesto anteriormente. Dos, las categorías que constituyen la estructura interna de la sociedad burguesa y en las que se basan las principales clases. Capital, trabajo asalariado, propiedad de la tierra y sus relaciones mutuas. La ciudad y el campo. Las tres grandes clases sociales; el intercambio entre ellas. La circulación. El sistema de crédito (privado). En tercer lugar, el Estado como personificación de la sociedad burguesa. Análisis de sus relaciones consigo mismo. Las clases "improductivas". Los impuestos. La deuda pública. El crédito público. La población. Colonias. Emigración. Cuatro, condiciones internacionales de producción. División internacional del trabajo. Intercambio internacional. Exportación e importación. Tipo de cambio. Cinco, mercado mundial y crisis.


\chapter{Las limitaciones de la utilidad marginal \\ Thorstein Veblen}

Las limitaciones de la economía de la utilidad marginal son agudas y característicos. Es, de principio a fin, una doctrina del valor y, desde el punto de vista de la forma y el método, una teoría de la valoración. Todo el sistema, por lo tanto, se encuentra dentro del campo teórico de la distribución, y no tiene más que una relación secundaria con cualquier otro fenómeno económico que no sea el de la distribución. Dentro de este rango limitado, la teoría de la utilidad marginal tiene un carácter totalmente estadístico. En todo esto, la escuela de la utilidad marginal coincide sustancialmente con la economía clásica del siglo XIX, con la diferencia de que la primera está confinada dentro de unos límites más estrechos y se atiene con mayor coherencia a sus premisas teleológicas. Ambas son teleológicas, y ninguna puede admitir de forma coherente argumentos de causa a efecto en la formulación de sus principales artículos teóricos. Ninguno de los dos puede tratar teóricamente los fenómenos de cambio, sino, a lo sumo, la adaptación racional al cambio que puede suponerse sobrevenido.\\

Para el científico moderno, los fenómenos de crecimiento y cambio son los hechos más obstrusivos y más importantes de la vida económica. Para comprender la vida económica moderna, el avance tecnológico de los dos últimos siglos -por ejemplo, el crecimiento de las artes industriales- es de la máxima importancia; pero la teoría de la utilidad marginal no tiene nada que ver con este asunto, ni este asunto tiene que ver con la teoría de la utilidad marginal. No tiene nada que decir sobre el crecimiento de los usos y conveniencias comerciales o sobre los cambios concomitantes en los principios de conducta que gobiernan las relaciones pecuniarias de los hombres, que condicionan y son condicionados por estas relaciones alteradas de la vida comercial o que las llevan a cabo.\\

Es característico de la escuela que siempre que un elemento del tejido cultural, una institución o cualquier fenómeno institucional, está implicado en los hechos de los que se ocupa la teoría, tales hechos institucionales se dan por sentados, se niegan o se explican.\\

La debilidad de este esquema teórico reside en sus postulados, que limitan la investigación a generalizaciones de orden teleológico o 'deductivo'. La escuela de la utilidad marginal los comparte con otros economistas de la línea clásica, ya que esta escuela no es más que una rama o derivación de los economistas clásicos ingleses del siglo XIX. La diferencia sustancial entre esta escuela y la generalidad de los economistas clásicos radica principalmente en el hecho de que en la economía de la utilidad marginal los postulados comunes se cumplen de forma más coherente, al tiempo que se definen con mayor nitidez y se comprenden más adecuadamente sus limitaciones. Tanto la escuela clásica en general como su variante especializada, la escuela de la utilidad marginal, en particular, toman como punto de partida común la psicología tradicional de los hedonistas de principios del siglo XIX, que se acepta como algo natural o de notoriedad común y se sostiene de forma bastante acrítica. El cálculo hedonista en lo que respecta a la conducta económica, se trata de una respuesta racional y desprejuiciada al estímulo del placer y el dolor anticipados, siendo típicamente y en su mayor parte, una respuesta a los impulsos del placer anticipado, ya que los hedonistas del siglo XIX y de la escuela de la utilidad marginal son en su mayor parte de temperamento optimista. Tal teoría sólo puede tener en cuenta la conducta en la medida en que sea una conducta racional, guiada por una elección deliberada y exhaustivamente inteligente, una sabia adaptación a las exigencias de la oportunidad principal.\\

Las circunstancias externas que condicionan la conducta son variables, por supuesto, y por lo tanto tendrán un efecto variable sobre la conducta; pero su variación es, en efecto, interpretada como de tal carácter sólo como para variar el grado de tensión al que el agente humano está sujeto por el contacto con estas circunstancias externas. Los elementos culturales implicados en el esquema teórico, elementos que son de la naturaleza de las instituciones, las relaciones humanas gobernadas por el uso y la costumbre en cualquier tipo y conexión, no están sujetos a investigación, sino que se dan por sentados como preexistentes en una forma acabada y típica, y como constituyentes de una situación económica normal y definitiva, bajo la cual y en términos de la cual se desarrollan necesariamente las relaciones humanas. Es decir, no incluye nada de las consecuencias o efectos causados por estos elementos institucionales.\\
Los elementos culturales tan tácitamente postulados como condiciones inmutables que preceden a la vida económica son la propiedad y el libre contrato, junto con otras características del esquema de derechos naturales que están implícitas en el ejercicio de los mismos. A los efectos de la teoría, se considera que estos productos culturales se dan a priori con una fuerza sin paliativos. Forman parte de la naturaleza de las cosas, de modo que no hay necesidad de dar cuenta de ellos o de indagar en ellos.\\

Ahora, las premisas en cuestión, en la medida en que son peculiares de la economía hedonista, son
\begin{enumerate}[(a)]
    \item Una determinada situación institucional, cuya característica sustancial es el derecho natural de propiedad.
    \item El cálculo Hedonista.- Que se refiere a los métodos de inferencia, a partir de la razón suficiente y de la causa eficiente están desconectados entre sí y no hay transición de uno a otro.
\end{enumerate}

La consecuencia inmediata es que la teoría económica resultante es de carácter teleológico - 'deductivo' o 'a priori', como se la suele llamar- en lugar de estar trazada en términos de causa y efecto. La relación que busca esta teoría entre los hechos de los que se ocupa es el control que ejercen los acontecimientos futuros (aprehendidos) sobre la conducta presente. La relación de razón suficiente va por el camino de la discriminación interesada, la previsión, de un agente que piensa en el futuro y guía su actividad presente teniendo en cuenta este futuro. La relación de razón suficiente sólo va del futuro (aprehendido) al presente, y es únicamente de carácter y fuerza intelectual, subjetiva, personal y teleológica; mientras que la relación de causa y efecto sólo va en la dirección contraria, y es únicamente de carácter y fuerza objetiva, impersonal y materialista.\\

Aún así, sucede que la relación de razón suficiente entra muy sustancialmente en la conducta humana. Pero al mismo tiempo no es menos cierto que la conducta humana, económica o de otro tipo, está sujeta a la secuencia de causa y efecto, por la fuerza de elementos tales como la habituación y los requisitos convencionales. Pero los hechos de este orden, que son para la ciencia moderna de mayor interés que los detalles teleológicos de la conducta, quedan necesariamente fuera de la atención del economista hedonista, porque no pueden interpretarse en términos de razón suficiente, como exigen sus postulados, ni encajar en un esquema de doctrinas teleológicas.\\

La aceptación por parte de los economistas de estos u otros elementos institucionales como dados e inmutables limita su investigación de una manera particular y decisiva. Bloquea la investigación en el punto en el que surge el interés científico moderno. Las instituciones en cuestión son sin duda buenas para su propósito como instituciones, pero no son buenas como premisas para una investigación científica sobre la naturaleza, el origen, el crecimiento y los efectos de estas instituciones y de las mutaciones que experimentan y que llevan a cabo en el esquema de vida de la comunidad.\\

Para cualquier científico moderno interesado en los fenómenos económicos, la cadena de causas y efectos en la que está implicada cualquier fase dada de la cultura humana, así como los cambios acumulativos forjados en el tejido de la propia conducta humana por la actividad habitual de la humanidad, son cuestiones de interés más absorbente y más duradero que el método de inferencia por el cual se supone que un individuo equilibra invariablemente el placer y el dolor en determinadas condiciones que se suponen normales e invariables. Las primeras son cuestiones de la historia vital de la raza o de la comunidad, cuestiones de crecimiento cultural y de la fortuna de las generaciones; mientras que las segundas son una cuestión de casuística individual ante una situación dada que puede surgir en el curso de este crecimiento cultural. Las primeras tienen que ver con la continuidad y las mutaciones de ese esquema de conducta por el que la humanidad se ocupa de sus medios materiales de vida; las segundas, si se conciben en términos hedonistas, se refieren a un episodio inconexo de la experiencia sensorial de un miembro individual de esa comunidad.\\

En la medida en que la investigación es ciencia económica, específicamente, la atención convergerá en el esquema de la vida material y abarcará otras fases de la civilización sólo en su correlación con el esquema de la civilización material.\\

Como toda cultura humana, esta civilización material es un sistema de instituciones: tejido institucional y crecimiento institucional. Pero las instituciones son una consecuencia de la costumbre. El crecimiento de la cultura es una secuencia acumulativa de costumbre, y sus formas y medios son la respuesta habitual de la naturaleza humana a exigencias que varían incontinentemente, acumulativamente, pero con algo de secuencia consistente en las variaciones acumulativas que así avanzan, - incontinentemente, porque cada nuevo movimiento crea una nueva situación que induce una nueva variación en la forma habitual de respuesta; acumulativamente, porque cada nueva situación es una variación de lo que la ha precedido e incorpora como factores causales todo lo que ha sido efectuado por lo que la precedió; coherentemente, porque los rasgos subyacentes de la naturaleza humana (propensiones, aptitudes, etc.) por la fuerza de los cuales tiene lugar la respuesta, y sobre la base de los cuales tiene efecto la habituación, permanecen sustancialmente inalterados.\\

una teoría adecuada de la conducta económica, incluso con fines estadísticos, no puede elaborarse simplemente en términos del individuo -como es el caso de la economía de la utilidad marginal- porque no puede elaborarse simplemente en términos de los rasgos subyacentes de la naturaleza humana; ya que la respuesta que conforma la conducta humana tiene lugar bajo normas institucionales y sólo bajo estímulos que tienen una influencia institucional; porque la situación que provoca e inhibe la acción en un caso dado es en gran parte de derivación institucional, cultural. Además, los fenómenos de la vida humana sólo se producen como fenómenos de la vida de un grupo o comunidad: sólo bajo los estímulos debidos al contacto con el grupo y sólo bajo el control (habitual) ejercido por los cánones de conducta impuestos por el sistema de vida del grupo. No sólo la conducta del individuo está rodeada y dirigida por sus relaciones habituales con sus semejantes en el grupo, sino que estas relaciones, al ser de carácter institucional, varían a medida que varía el sistema institucional. Los deseos y las aspiraciones, el fin y el objetivo, los medios y los modos, la amplitud y la deriva de la conducta del individuo son funciones de una variable institucional de carácter muy complejo y totalmente inestable.\\

El crecimiento y las mutaciones del tejido institucional son un resultado de la conducta de los miembros individuales del grupo, ya que es de la experiencia de los individuos, a través de la habituación de los individuos, de donde surgen las instituciones; y es en esta misma experiencia donde estas instituciones actúan para dirigir y definir los objetivos y el fin de la conducta. \\

Los postulados de la utilidad marginal, y los preconceptos hedonistas en general, fracasan en la medida en que limitan la atención a aquellos aspectos de la conducta económica que se considera que no están condicionados por las normas y los ideales habituales y que no tienen ningún efecto de habituación. Hacen caso omiso o se abstraen de la secuencia causal de la propensión y la habituación en la vida económica y excluyen de la investigación teórica todo interés en los hechos del crecimiento cultural, con el fin de atender a las características del caso que se consideran ociosas a este respecto. Todos estos hechos de fuerza y crecimiento institucional se dejan de lado por no ser relevantes para la teoría pura. Ciertos fenómenos institucionales, es cierto, se incluyen entre las premisas de los hedonistas, como se ha señalado anteriormente; pero se incluyen como postulados a priori. Así, la institución de la propiedad se incluye en la investigación no como un factor de crecimiento o un elemento sujeto a cambio, sino como uno de los hechos primordiales e inmutables del orden de la naturaleza, subyacente al cálculo hedonista.  Mientras que la institución de la propiedad se incluye de esta manera entre los postulados de la teoría, e incluso se presume que está siempre presente en la situación económica, no se permite que tenga ninguna fuerza en la configuración de la conducta económica, que se concibe para seguir su curso hasta su resultado hedonista como si ningún factor institucional interviniera entre el impulso y su realización.\\

La situación económica moderna es una situación empresarial, en el sentido de que la actividad económica de todo tipo suele estar controlada por consideraciones empresariales. Las exigencias de la vida moderna son comúnmente exigencias pecuniarias. Es decir, son exigencias de la propiedad. Tanto la eficiencia productiva como la ganancia distributiva se valoran en términos de precio. \\

Aunque el esquema institucional del sistema de precios domina visiblemente el pensamiento de la comunidad moderna en asuntos que están fuera del interés económico, los economistas hedonistas insisten, en efecto, en que este esquema institucional debe considerarse sin efecto dentro de esa gama de actividad a la que debe su génesis, crecimiento y persistencia.\\

El dinero y el recurso habitual a su uso se conciben simplemente como los medios por los que se adquieren los bienes consumibles y, por lo tanto, como un método conveniente para procurarse las sensaciones placenteras del consumo; siendo estas últimas, en la teoría hedonista, el único y manifiesto fin de todo esfuerzo económico. Los valores monetarios no tienen, por tanto, otro significado que el de poder adquisitivo sobre los bienes consumibles, y el dinero es simplemente un instrumento de cálculo. La inversión, la concesión de créditos, los préstamos de todo tipo y grado, con el pago de intereses y demás, se consideran simplemente como pasos intermedios entre las sensaciones placenteras del consumo y los esfuerzos inducidos por la anticipación de estas sensaciones, sin tener en cuenta otros aspectos del caso. 

\part{OPINIONES POSITIVISTAS Y POPPERIANAS
}

\chapter{La metodología de la economía positiva \\ Milton Friedman}

En su admirable libro The Scope and Method of Political Economy, John Neville Keynes distingue entre 'una ciencia positiva...[,] un conjunto de conocimientos sistematizados sobre lo que es; una ciencia normativa o reguladora...[,] un conjunto de conocimientos sistematizados que discuten criterios sobre lo que debería ser...; un arte...[,] un sistema de reglas para alcanzar un fin determinado'. ...; comenta que 'la confusión entre ellas es común y ha sido fuente de muchos errores malintencionados'; e insta a la importancia de 'reconocer una ciencia positiva distinta de la economía política'.\\

\section{La relación entre la economía positiva y normativa}
La confunción entre economía positiva y normativa es, en cierta medida, inevitable. La opinión de los 'expertos' difícilmente podría aceptarse sólo por fe, incluso si los "expertos" fueran casi unánimes y claramente desinteresados. Las conclusiones de la economía positiva parecen ser, y son, inmediatamente relevantes para importantes problemas normativos, para cuestiones sobre lo que debería hacerse y cómo puede alcanzarse cualquier objetivo dado. Tanto los profanos como los expertos se ven inevitablemente tentados a adaptar las conclusiones positivas a prejuicios normativos muy arraigados y a rechazar las conclusiones positivas si sus implicaciones normativas o lo que se dice que son sus implicaciones normativas, son desagradables. La economía positiva es, en principio, independiente de cualquier posición ética o juicio normativo particular. Como dice Keynes se ocupa de lo que es, no de lo que debería ser. Su tarea es proporcionar un sistema de generalizaciones que pueda utilizarse para hacer predicciones correctas sobre las consecuencias de cualquier cambio en las circunstancias. Su rendimiento debe juzgarse por la precisión, el alcance y la conformidad con la experiencia de las predicciones que arroja. En resumen, la economía positiva es, o puede ser, una ciencia 'objetiva', precisamente en el mismo sentido que cualquiera de las ciencias físicas.\\
Por otra parte, la economía normativa y el arte de la economía no pueden ser independientes de la economía positiva. Cualquier conclusión política descansa necesariamente en una predicción sobre las consecuencias de hacer una cosa en lugar de otra, una predicción que debe basarse -implícita o explícitamente- en la economía positiva. Un ejemplo obvio y no carente de importancia es la legislación sobre el salario mínimo. Por debajo del cúmulo de argumentos que se ofrecen a favor y en contra de dicha legislación existe un consenso subyacente sobre el objetivo de lograr un 'salario digno' para todos.  La diferencia de opinión se basa en gran medida en una diferencia implícita o explícita en las predicciones sobre la eficacia de este medio concreto para promover el fin acordado. Los defensores creen (predicen) que los salarios mínimos legales disminuyen la pobreza al aumentar los salarios de los que reciben menos del salario mínimo, así como de los que reciben más del salario mínimo, sin que se produzca un aumento compensatorio en el número de personas totalmente desempleadas o empleadas de forma menos ventajosa de lo que lo estarían en caso contrario. Los que se oponen creen (predicen) que los salarios mínimos legales aumentan la pobreza al aumentar el número de personas desempleadas o empleadas de forma menos ventajosa y que esto compensa con creces cualquier efecto favorable sobre los salarios de los que siguen empleados. Es posible que un acuerdo sobre las consecuencias económicas de la legislación no produzca un acuerdo completo sobre su conveniencia, ya que podrían subsistir diferencias sobre sus consecuencias políticas o sociales; pero, dado el acuerdo sobre los objetivos, sin duda contribuiría en gran medida a producir un consenso. Una de las principales razones para distinguir claramente la economía positiva de la economía normativa es precisamente la contribución que puede hacer a un acuerdo sobre la política.

\section{Economía positiva}
El objetivo último de una ciencia positiva es el desarrollo de una 'teoría' o 'hipótesis' que produzca predicciones válidas y significativas (es decir, no trucadas) sobre fenómenos aún no observados. Una teoría de este tipo es, en general, una compleja mezcla de dos elementos. En parte, es un 'lenguaje' diseñado para promover 'métodos sistemáticos y organizados de razonamiento'. En parte, es un conjunto de hipótesis sustantivas (posibles respuestas acerca de la realidad, que deben ser sometidas a verificación empírica) diseñadas para abstraer características esenciales de la realidad compleja. Para ello, sólo los cánones de la lógica formal pueden demostrar si un lenguaje concreto es completo y coherente; es decir, si las proposiciones del lenguaje son correctas o incorrectas. El sencillo ejemplo de la 'oferta' y la 'demanda' consideradas como elementos del lenguaje de la teoría económica, son las dos grandes categorías en las que se clasifican los factores que afectan a los precios relativos de los productos o factores de producción. La simple e incluso obvia medida de clasificar los factores relevantes bajo los epígrafes de 'oferta' y 'demanda' supone una gran simplificación del problema y constituye una eficaz salvaguarda contra las falacias que, de otro modo, tienden a producirse. Pero la generalización no siempre es válida. Por ejemplo, no es válida para las fluctuaciones diarias de los precios en un mercado principalmente especulativo.\\

Vista como un cuerpo de hipótesis sustantivas, la teoría debe juzgarse por su poder predictivo para la clase de fenómenos que pretende 'explicar'. La única prueba relevante de la validez de una hipótesis es la comparación de sus predicciones con la experiencia. La hipótesis se rechaza si sus predicciones se contradicen ('frecuentemente' o con más frecuencia que las predicciones de una hipótesis alternativa); se acepta si sus predicciones no se contradicen; se le otorga una gran confianza si ha sobrevivido a muchas oportunidades de contradicción. Las pruebas objetivas nunca pueden 'demostrar' una hipótesis; sólo pueden no refutarla, que es a lo que nos referimos generalmente cuando decimos, de forma un tanto inexacta, que la hipótesis ha sido 'confirmada' por la experiencia.\\

Para evitar confusiones, quizá habría que señalar explícitamente que las 'predicciones' mediante las que se comprueba la validez de una hipótesis no tienen por qué referirse a fenómenos que aún no se han producido; es decir, no tienen por qué ser previsiones de acontecimientos futuros; pueden referirse a fenómenos que se han producido pero sobre los que aún no se han realizado observaciones o que la persona que hace la predicción desconoce. Por ejemplo, una hipótesis puede implicar que tal o cual cosa debió ocurrir en 1906, dadas otras circunstancias conocidas. Si una búsqueda en los registros revela que tal cosa ocurrió, la predicción se confirma; si revela que tal cosa no ocurrió, la predicción se contradice.\\

La evidencia adicional con la que la hipótesis debe ser consistente puede descartar algunas de las muchas categorías posibles (mercados competitivos, monopolistas, costes marginales constantes, curva de oferta horizonta,etc); nunca puede reducirlas a una única posibilidad capaz de ser consistente con la evidencia finita. La elección entre hipótesis alternativas igualmente coherentes con las pruebas disponibles debe ser hasta cierto punto arbitraria, aunque existe un acuerdo general en que las consideraciones pertinentes vienen sugeridas por los criterios de 'simplicidad' y 'eficacia', nociones que desafían una especificación completamente objetiva. Una teoría es 'más simple' cuanto menor es el conocimiento inicial necesario para hacer una predicción dentro de un determinado campo de fenómenos; es más 'afectiva' cuanto más precisa es la predicción resultante, más amplia es el área dentro de la cual la teoría arroja predicciones y más líneas adicionales de investigación sugiere. La exhaustividad y la coherencia lógicas son importantes, pero desempeñan un papel secundario; su función es garantizar que la hipótesis dice lo que pretende decir y lo hace de la misma manera para todos los usuarios; desempeñan aquí el mismo papel que las comprobaciones de la exactitud aritmética en los cálculos estadísticos.

Cabe señalar, que ningún experimento puede controlarse por completo, y toda experiencia está parcialmente controlada, en el sentido de que algunas influencias perturbadoras son relativamente constantes en su transcurso.\\
Las pruebas aportadas por la experiencia son abundantes y, con frecuencia, tan concluyentes como las de los experimentos realizados; por lo tanto, la incapacidad de llevar a cabo experimentos no es un obstáculo fundamental para poner a prueba las hipótesis mediante el éxito de sus predicciones. Muchas ocasiones estas experiencias no son tan directas o convincentes como las que podrían proporcionar los experimentos controlados. Por supuesto, ocasionalmente, la experiencia arroja pruebas directas. Por ejemplo, la evidencia de las inflaciones sobre la hipótesis de que un aumento sustancial de la cantidad de dinero en un período relativamente corto va acompañado de un aumento sustancial de los precios.\\

Una de las consecuencias de la dificultad de poner a prueba las hipótesis económicas de fondo ha sido la tendencia al análisis puramente formal o tautológico. Como ya se ha señalado, las tautologías ocupan un lugar muy importante en la economía y otras ciencias como lenguaje especializado o 'sistema de archivo analítico'. Más allá de esto, la lógica formal y las matemáticas, que son ambas tautologías, son ayudas esenciales para comprobar la corrección del razonamiento, descubrir las implicaciones de las hipótesis y determinar si hipótesis supuestamente diferentes pueden no ser realmente equivalentes o dónde radican las diferencias.\\
Pero la teoría económica debe ser algo más que una estructura de tautologías si quiere ser capaz de predecir y no simplemente describir las consecuencias de la acción; si quiere ser algo distinto de matemáticas disfrazadas. Y la eficacia de las propias tautologías depende en última instancia, como ya se ha señalado, de la aceptabilidad de las hipótesis de fondo que sugieren las categorías particulares en las que organizan los fenómenos empíricos refractarios.\\

Un efecto más grave de la dificultad de comprobar las hipótesis económicas por sus predicciones es fomentar la incomprensión del papel de la evidencia empírica en el trabajo teórico. La evidencia empírica es vital en dos etapas diferentes, aunque estrechamente relacionadas: 
\begin{itemize}
    \item en la construcción de hipótesis y 
    \item en la comprobación de su validez.
\end{itemize}

Las pruebas completas y exhaustivas sobre los fenómenos que una hipótesis debe generalizar o 'explicar', además de su valor obvio para sugerir nuevas hipótesis, son necesarias para garantizar que una hipótesis explica lo que pretende explicar, es decir, que sus implicaciones para dichos fenómenos no son contradichas de antemano por la experiencia ya observada. Dado que la hipótesis es coherente con las pruebas disponibles, su comprobación posterior consiste en deducir de ella nuevos hechos observables pero desconocidos hasta ahora y contrastarlos con pruebas empíricas adicionales. Para que esta prueba sea relevante, los hechos deducidos deben referirse a la clase de fenómenos que la hipótesis pretende explicar, y deben estar lo suficientemente bien definidos como para que la observación pueda demostrar que son erróneos.\\

Las dos etapas de elaboración de hipótesis y comprobación de su validez están relacionadas en dos aspectos diferentes. En primer lugar, los hechos concretos que entran en juego en cada etapa son en parte un accidente de la recogida de datos y de los conocimientos del investigador concreto. Los hechos que sirven para poner a prueba las implicaciones de una hipótesis también podrían haber formado parte de la materia prima utilizada para construirla, y viceversa. En segundo lugar, el proceso nunca empieza de cero; la denominada 'etapa inicial' siempre implica la comparación de las implicaciones de un conjunto de hipótesis anteriores con la observación; la contradicción de estas implicaciones es el estímulo para la construcción de nuevas hipótesis o la revisión de las antiguas. Así pues, las dos etapas, metodológicamente distintas, se desarrollan siempre conjuntamente.

Los malentendidos sobre este proceso aparentemente sencillo se centran en la frase 'la clase de fenómenos que la hipótesis pretende explicar'. La dificultad en las ciencias sociales de obtener nueva evidencia para esta clase de fenómenos y de juzgar su conformidad con las implicaciones de la hipótesis hace que sea tentador suponer que otra evidencia, más fácilmente disponible, es igualmente relevante para la validez de la hipótesis a suponer. Que las hipótesis no solo tienen 'implicaciones' sino también 'suposiciones' y que la conformidad de estas 'suposiciones' con la 'realidad' es una prueba de la validez de la hipótesis distinta o adicional a la prueba de las implicaciones, esta opinión generalizada es fundamentalmente erróneo y productivo de mucho daño.\\

En la medida en que pueda decirse que una teoría tiene 'supuestos', y en la medida en que su 'realismo' pueda juzgarse independientemente de la validez de las predicciones, la relación entre la importancia de una teoría y el realismo de sus supuestos es casi la opuesta a la sugerida por el punto de vista criticado. Se descubrirá que las hipótesis verdaderamente importantes y significativas tienen 'supuestos' que son representaciones descriptivas tremendamente inexactas de la realidad y, en general, cuanto más significativa es la teoría, más irreales son los supuestos (en este sentido). La razón es sencilla. Una hipótesis es importante si explica mucho por poco, es decir, si abstrae los elementos comunes y cruciales de la masa de circunstancias complejas y detalladas que rodean a los fenómenos a explicar y permite predicciones válidas basándose únicamente en ellos. Para ser importante, por tanto, una hipótesis debe ser descriptivamente falsa en sus supuestos; no tiene en cuenta, ni explica, ninguna de las muchas otras circunstancias concurrentes, ya que su propio éxito demuestra que son irrelevantes para los fenómenos que hay que explicar.\\
Para decirlo de un modo menos paradójico, lo que hay que preguntarse sobre los 'supuestos' de una teoría no es si son descriptivamente 'realistas', porque nunca lo son, sino si son aproximaciones suficientemente buenas para el objetivo que se persigue. Y esta pregunta sólo puede responderse viendo si la teoría funciona, lo que significa si produce predicciones suficientemente precisas. Así pues, las dos pruebas supuestamente independientes se reducen a una sola.

\section{¿Se puede probar una hipótesis por el realismo de sus supuestos?}





\end{document}
