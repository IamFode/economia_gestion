\documentclass[a4paper,fleqn]{cas-sc}

%Packages
\usepackage[authoryear,longnamesfirst]{natbib}
\usepackage[spanish]{babel}


%%%Author macros
\def\tsc#1{\csdef{#1}{\textsc{\lowercase{#1}}\xspace}}
\tsc{WGM}
\tsc{QE}
%%%

\newtheorem{theorem}{Theorem}
\newtheorem{lemma}[theorem]{Lemma}
\newdefinition{rmk}{Remark}
\newproof{pf}{Proof}
\newproof{pot}{Proof of Theorem \ref{thm}}


\begin{document}
\let\WriteBookmarks\relax
\def\floatpagepagefraction{1}
\def\textpagefraction{.001}

% Short title
\shorttitle{}    

% Short author
\shortauthors{Christian Paredes}  

% Main title of the paper
\title[mode = title]{La cantidad de dinero y el nivel de precios. Un análisis de Wavelets}  

% Title footnote mark
\tnotemark[1]

% Title footnote 1.
\tnotetext[1]{Trabajo Fin de Máster}

% First author

\author[1]{Christian L. Paredes Aguilera}%[
	%type=editor,
	%style=Español,
	%auid=001,
	%bioid=1,
	%prefix=Sir,
	%twitter=www.twitter.com/juanfidelalvarez,
	%linkedin=www.linkedin.com/in/juanfidelalvarez,
	%gplus=001]
%]

% Corresponding author indication
%\cormark[1]

% Footnote of the first author
%\fnmark[]

% Email id of the first author
% \ead{soyfode@gmail.com}

% URL of the first author
% \ead[url]{www.christianparedes.com}

% Credit authorship
%\credit{Conceptualization of this study, Methodology, Software}

% Address/affiliation
\affiliation[2]{organization={Universidad de Vigo},
            addressline={R\'ua as Pedreiras, 2}, 
            city={Vigo},
%           citysep={}, % Uncomment if no comma needed between city and postcode
            postcode={36310}, 
            %state={},
            country={España}}

%\credit{Data curation, Writing - Original draft preparation}

% For a title note without a number/mark
%\nonumnote{}

% Here goes the abstract
\begin{abstract}
En construcción\\
En construcción\\
En construcción\\
\end{abstract}

% Use if graphical abstract is present
%\begin{graphicalabstract}
%\includegraphics{}
%\end{graphicalabstract}

% Research highlights
%\begin{highlights}
%\item 
%\item 
%\item 
%\end{highlights}

% Keywords
% Each keyword is seperated by \sep
\begin{keywords}
    Cantidad de Dinero \sep
    Nivel de Precios \sep
    Wavelets
\end{keywords}

\maketitle


% Main text
\section{Propuesta de trabajo de fin de máster}

Según \cite{Ramsey2016}, las wavelets son una generalización del análisis de Fourier en el que la estacionariedad de la serie temporal ya no es crítica y se puede lograr la localización de una señal. En este sentido, tomando prestada una gran idea de Strang (véase \cite{Strang1996}, el análisis de Fourier representa mejor las funciones compuestas por combinaciones lineales de entradas estacionarias, pero las wavelets son como una notación musical en la que cada nota se caracteriza por su frecuencia, su posición en el tiempo y su duración. Es decir, nos permitirá estudiar la evolución de una, dos o más series temporales en el tiempo y en el espacio frecuencia (corto y largo plazo). 

Ahora, surge la siguiente duda: ¿De que manera podría ayudarnos las wavelets a analizar la relación entre la expansión monetaria y el crecimiento de los precios?. Como menciona \cite{AguiarConraria2008} y \cite{Jiang2015}, podríamos utilizar herramientas como: la coherencia wavelet\footnote[1]{Analiza el coeficiente de correlación localizado en el espacio tiempo-frecuencia}, la diferencias de fase\footnote[2]{Proporciona información sobre el retardo, o sincronización, entre las oscilaciones de las dos series temporales.} 

En fin, las Wavelets y las herramientas mencionadas nos permitirán analizar la relación entre la cantidad de dinero y el nivel de precios en el tiempo y en el espacio frecuencia al mismo tiempo.


\begin{enumerate}

    \item  \textbf{Wavelets cruzadas:}  Que estudia directamente las interacciones entre dos series temporales a diferentes frecuencias (corto, largo plazo) y cómo evolucionan con el tiempo. 

    \item  \textbf{Espectro de potencia wavelet:}  Describe la evolución de la varianza de una serie temporal en las diferentes frecuencias, con periodos de gran varianza asociados a periodos de gran potencia (mucha fluctuación) en las diferentes escalas, 

\item  \textbf{Potencia cross-wavelet} (de dos series temporales)\textbf{:} describe la covarianza local. entre las series temporales 

    \item  \textbf{Coherencia wavelet:}  Que .

    \item  \textbf{Diferencia de fase:} Proporciona información sobre el retardo, o sincronización, entre las oscilaciones de las dos series temporales.

\end{enumerate}

% To print the credit authorship contribution details
\printcredits

\bibliographystyle{cas-model2-names}

% Loading bibliography database
\bibliography{refs}

% Biography
\bio{}
% Here goes the biography details.
\endbio


\end{document}

