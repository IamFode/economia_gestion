\documentclass[a4paper,fleqn]{cas-sc}

%Packages
\usepackage[authoryear,longnamesfirst]{natbib}
\usepackage[spanish]{babel}


%%%Author macros
\def\tsc#1{\csdef{#1}{\textsc{\lowercase{#1}}\xspace}}
\tsc{WGM}
\tsc{QE}
%%%

\newtheorem{theorem}{Theorem}
\newtheorem{lemma}[theorem]{Lemma}
\newdefinition{rmk}{Remark}
\newproof{pf}{Proof}
\newproof{pot}{Proof of Theorem \ref{thm}}


\begin{document}
\let\WriteBookmarks\relax
\def\floatpagepagefraction{1}
\def\textpagefraction{.001}

% Short title
\shorttitle{}    

% Short author
\shortauthors{Christian Paredes}  

% Main title of the paper
\title[mode = title]{La cantidad de dinero y el nivel de precios. Un análisis de Wavelets}  

% Title footnote mark
\tnotemark[1]

% Title footnote 1.
\tnotetext[1]{Trabajo Fin de Máster}

% First author

\author[1]{Christian L. Paredes Aguilera}%[
	%type=editor,
	%style=Español,
	%auid=001,
	%bioid=1,
	%prefix=Sir,
	%twitter=www.twitter.com/juanfidelalvarez,
	%linkedin=www.linkedin.com/in/juanfidelalvarez,
	%gplus=001]
%]

% Corresponding author indication
%\cormark[1]

% Footnote of the first author
%\fnmark[]

% Email id of the first author
% \ead{soyfode@gmail.com}

% URL of the first author
% \ead[url]{www.christianparedes.com}

% Credit authorship
%\credit{Conceptualization of this study, Methodology, Software}

% Address/affiliation
\affiliation[2]{organization={Universidad de Vigo},
            addressline={R\'ua as Pedreiras, 2}, 
            city={Vigo},
%           citysep={}, % Uncomment if no comma needed between city and postcode
            postcode={36310}, 
            %state={},
            country={España}}

%\credit{Data curation, Writing - Original draft preparation}

% For a title note without a number/mark
%\nonumnote{}

% Here goes the abstract
\begin{abstract}
En construcción\\
En construcción\\
En construcción\\\\
\end{abstract}

% Use if graphical abstract is present
%\begin{graphicalabstract}
%\includegraphics{}
%\end{graphicalabstract}

% Research highlights
%\begin{highlights}
%\item 
%\item 
%\item 
%\end{highlights}

% Keywords
% Each keyword is seperated by \sep
\begin{keywords}
    Cantidad de Dinero \sep
    Nivel de Precios \sep
    Wavelets
\end{keywords}

\maketitle


% Main text
\section{Objetivo}
Estudiar la relación dinámica entre la cantidad de dinero y el nivel de precios, utilizando el análisis wavelets.


\section{Introducción}
Daremos un breve repaso de la teoría cuantitativa del dinero y la relación entre la cantidad de dinero y el nivel de precios. Además, se mencionará la importancia de estudiar esta relación. Se analizará las oportunidades de usar wavelets frente a técnica cómo la Transformada de Fourier. También mencionaremos algunos trabajos anteriores que aplican el análisis wavelet a la economía.


\section{Literatura relacionada}
Empezamos describiendo  la teoría cuantitativa del dinero, que se remonta a los escritos de David Hume y John Stuart Mill. Sin embargo, la versión moderna de la teoría cuantitativa del dinero se debe a Irving Fisher. La teoría cuantitativa del dinero establece que la cantidad de dinero en una economía es proporcional al nivel de precios.

Luego, mencionaremos algunos métodos clásicos para analizar la relación dinámica entre la cantidad de dinero y el nivel de precios. En particular, se mencionará la Transformada de Fourier, análisis de series de tiempo y el coeficiente de relación de Pearson.

\begin{itemize}
    \item Lucas, R. E. (1980). Two Illustrations of the Quantity Theory of Money. The American Economic Review, 70(5), 1005–1014. http://www.jstor.org/stable/1805778
    \item Dwyer, G. P., \& Fisher, M. E. (2009). Inflation and Monetary Regimes (Federal Reserve Bank of Atlanta Working Paper Series No. 2009-26). SSRN. https://ssrn.com/abstract=1480896
    \item Becker, R., Enders, W., \& Hurn, S. (2006). Modeling Inflation and Money Demand Using a Fourier-Series Approximation. En C. Milas, P. Rothman, \& D. van Dijk (Eds.), Nonlinear Time Series Analysis of Business Cycles (Contributions to Economic Analysis, Vol. 276, pp. 221-246). Elsevier. https://doi.org/10.1016/S0573-8555(05)76009-0
\end{itemize}

\section{Teoría wavelets y métodos}
Daremos una Introducción a la teoría wavelets y los métodos que se utilizarán en el trabajo. En particular, se mencionará la Transformada Wavelet Continua, la Coherencia Wavelet y la Diferencia de Fase.

    \begin{itemize}
	\item Aguiar-Conraria, L., Azevedo, N., Soares, M.J., 2008. Using wavelets to decompose the time–frequency effects of monetary policy. Physica A: Statistical Mechanics and its Applications 387, 2863–2878. URL: https://www.sciencedirect.com/science/article/pii/S037843710800040X,\\ doi:https://doi.org/10.1016/j.physa.2008.01.063.
    \end{itemize}

\subsection{Transformada wavelet continua}
En esta sección explicaremos la transformada wavelet continua. En particular la función wavelet de Morlet que utilizaremos para nuestros fines. 
	\begin{itemize}
	    \item Daubechies, I. (1988). Orthonormal bases of compactly supported wavelets. Communications on Pure and Applied Mathematics, 41(7), 909-996. https://doi.org/10.1002/cpa.3160410705
	    \item Morlet, J., Arens, G., Fourgeau, E., \& Glard, D. (1982). Wave propagation and sampling theory—Part I: Complex signal and scattering in multilayered media. Geophysics, 47(2), 203-221
	\end{itemize}
\subsection{Coherencia wavelet y diferencia de fase}
Por último explicaremos la coherencia wavelet y la diferencia de fase, que son los métodos que utilizaremos para analizar la relación dinámica entre la cantidad de dinero y el nivel de precios.

    \begin{itemize}
	\item Torrence, C., \& Compo, G. P. (1998). A Practical Guide to Wavelet Analysis. Bulletin of the American Meteorological Society, 79(1), 61-78. https://doi.org/10.1175/1520-0477(1998)079<0061:APGTWA>2.0.CO;2
    \end{itemize}

\section{Resultados}
\subsubsection{Descripción de los datos}
Inicialmente, mi objetivo es realizar un estudio exhaustivo de los diversos indicadores de la cantidad de dinero presentes en la economía. Este estudio se llevará a cabo para todos los países y abarcará todos los años para los que se disponga de registros. Los indicadores que se analizarán son los siguientes:
\begin{enumerate}
    \item Courrency-IFS
    \item RM-Courrency
    \item BM-IFS
    \item BM-Nac-IFS
    \item M1
    \item M2
    \item M3
    \item M4
    \item M5
    \item RR
\end{enumerate}
Cada uno de estos indicadores proporciona una visión única de la cantidad de dinero en circulación y será crucial para entender las dinámicas económicas a lo largo del tiempo y a través de diferentes países.

\subsubsection{resultados empíricos}
Se realizara un análisis con Python o Matlab.

\subsection{Conclusiones}

\begin{comment}
Según \cite{Ramsey2016}, las wavelets son una generalización del análisis de Fourier en el que la estacionariedad de la serie temporal ya no es crítica y se puede lograr la localización de una señal. En este sentido, tomando prestada la idea de Strang (véase \cite{Strang1996}, las wavelets son como una notación musical en la que cada nota se caracteriza por su frecuencia, su posición en el tiempo y su duración. Es decir, nos permitirá estudiar la evolución en tiempo y en espacio frecuencia (corto y largo plazo). 

Ahora, surge la siguiente duda: ¿De que manera podría ayudarnos las wavelets a analizar la relación entre la inflación y los regímenes monetarios?. Como menciona \cite{AguiarConraria2008} y \cite{Jiang2015}, podríamos utilizar herramientas como: la coherencia wavelet\footnote[1]{Analiza el coeficiente de correlación entre dos series de tiempo localizado en el espacio tiempo-frecuencia} y la diferencias de fase\footnote[2]{Proporciona información sobre el retardo, o sincronización, entre las oscilaciones de las dos series temporales.}, para encontrar vínculos fuertes, dinámicas transitorias y poder cuantificar el grado de relación no necesariamente lineal entre dos series temporales no estacionarias en tiempo-frecuencia.

En fin, las Wavelets y las herramientas mencionadas nos permitirán analizar la relación entre la cantidad de dinero y el nivel de precios en el tiempo y en el espacio frecuencia.
\end{comment}


% To print the credit authorship contribution details
%\printcredits

%\bibliographystyle{cas-model2-names}

% Loading bibliography database
%\bibliography{refs}

% Biography
%\bio{}
% Here goes the biography details.
%\endbio


\end{document}

