\chapter{Análisis de decisiones económicas y mercados}

\begin{multicols}{2}

\begin{itemize}
    \item Para criticar se debe conocer.
\end{itemize}

\section*{¿Qué preguntas puede abordar la teoría económica?.}

\begin{itemize}
    \item A la economía se le exige a más que otras ciencias. 
    \item ¿El objetivo de la ciencia en predecir?.
    \item El propósito de la ciencia es crear conocimiento donde se crea mundos abstractos.
\end{itemize}

    \subsection*{Objetivos de la economía}

    \begin{itemize}
	\item Entender el mundo real y,
	\item crear un lenguaje para poder discutir.
    \end{itemize}

    \subsection*{Métodos de investigación}
    \begin{itemize}
	\item Las ciencias físicas son mas fáciles que las ciencias económicas. Que sería si los electrones podrían pensar.
	\item Utilizamos el método inductivo.
    \end{itemize}

    \subsection*{¿Cual es el punto de partida para crear mundos abstractos?}
    Las llamamos las tres almas de Smith.
    \begin{enumerate}[1.]
	\item Estudiamos el comportamiento individual de los agentes.
	\item Un grupo es más que la suma de los individuos.
	\item Análisis de instituciones.
    \end{enumerate}
    \textbf{Se partirá del análisis individual}

    \subsection*{Los fundamentos de la escuela neoclásica}

    \begin{itemize}
	\item Se estudia ésta escuela porque es la mas exitosa hasta el momento.
	\item Se tiene tres características para poder ser identificado como neoclásico.
	\begin{enumerate}[1.]
	    \item Ser reduccionistas. Empezamos a analizar a los individuos.
	    \item ¿Que les motiva para tomar acciones?, los agentes deben ser racionales. 
	    \item Tener una noción de equilibro.
	\end{enumerate}
	\item No hay nada peor que estar adoctrinados y no saber lo que creemos.
	\item \textbf{Los economistas estudian decisiones, identificando factores que afectan esas decisiones y el resultado de la interacción de decisiones.}
    \end{itemize}




\end{multicols}
