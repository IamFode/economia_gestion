\textbf{Christian Limbert Paredes Aguilera}\\\\

\begin{center}
    \textbf{\Large Ejercicio Final}
\end{center}
\vspace{1cm}

\begin{enumerate}[\Large \bfseries 1.]

%----------1.
\item Explique a un familiar que no tiene ni idea de matemáticas la intuición económica de: (Nota. Intente ajustarse a la extensión máxima. Su familiar no espera un curso de economía, sino una respuesta breve.):

    \begin{enumerate}[\bfseries 1.]

	%----------1.
	\item ¿Cuál es el problema del consumidor Marshalliano?.\\\\
	    Respuesta.-\; Las cestas que voy a consumir no pueden superar la renta que son independientes de los precios. Pediremos en la cafetería de la Facultad un café y un croissant porque el gasto en café y el gasto en croissant no supera a la renta.\\\\

	%----------2.
	\item  ¿Por qué hay pubs que abren todos los días, mientras que otros solo abren de jueves a domingo?.\\\\
	    Respuesta.-\; Porque los costes (marginales) de estar abiertos un día adicional son superiores a los ingresos (marginal). Suponiendo que los precios de los pubs varían a lo largo de los días de la semana, los pubs estarán abiertos más días porque el precio será mucho mayor esto para que les compense los costes de seguir manteniendo días adicionales abiertas.\\\\

	%----------3.
	\item  ¿Qué es eficiencia en sentido de Pareto?.\\\\
	    Respuesta.-\; Es aquella en la que es imposible mejorar el bienestar de alguna persona sin empeorar el de ninguna otra.\\\\

    \end{enumerate}

%----------2.
\item 
    \begin{enumerate}[\bfseries a)]

	%----------a)
	\item Traslade los argumentos filosóficos de Buridán al lenguaje que que hemos construido en Tema 5 (¿Cómo decide un Consumidor?).
	%----------b)
	\item Indique qué elemento para entender la decisión del asno está planteado incorrectamente por Buridán, es decir, por qué el análisis de Buridán está incompleto desde el punto de vista de la Teoría del Consumidor.\\\\
	   Respuesta.-\; Hare una conjunción de los incisos a) y b) para dar una sola respuesta.\\\\ 
	    Según el supuesto de preferencias existe un axioma que denominaremos completas, donde se hace mención a la posibilidad de comprar dos cestas cualquiera. Es decir, dados cualesquiera cestas $X$ y $Y$ suponemos que $(x_1,x_2)\succeq (y_1,y_2)$ o $(y_1,y_2)\succeq (x_1,x_2)$ o las dos cosas, en cuyo caso, el consumidor es indiferente entre las dos cesas, en otras palabras decir que pueden compararse dos cestas cualesquiera es decir simplemente que el consumidor tiene la capacidad de elegir, discriminar como también tiene los conocimientos necesarios para evaluar las alternativas.\\\\
	    Sabiendo este supuesto, vemos que las alternativas de del burro son: 
	    \begin{enumerate}[1)]
		\item Comer la bala de heno A.
		\item Comer la bala B. 
		\item No comer ninguna bala y morir de hambre. 
	    \end{enumerate}
	    Dado que cabe suponer que el burro ordenará (3) por debajo de (1) y (2), no elegirá, como hizo Buridán, (3). Si lo hiciera, incurriría en una violación de nuestro supuesto fundamental que se acaba de detallar. Y por lo tanto vemos que el asno de Buridán no es racional: si lo fuera, habría elegido no morirse de hambre, suponiendo que prefiera comer a morirse de hambre\\\\

    \end{enumerate}

%----------3.
\item 
    \begin{enumerate}[\bfseries a)]

	%----------a)
	\item Los economistas, podrían estudiar el tráfico de coches y la congestión de las carreteras?.\\\\
	    Respuesta.-\; Si podrían estudiarlos, ya que podríamos estudiar las decisiones de los los agentes que compran coches o porque conducen el mismo en hora pico, podríamos estudiar los factores que influyen tales decisiones, como también estudiar los resultados de las interacciones de decisiones.\\\\

	%----------b)
	\item Argumente con la terminología utilizada en el curso de Decisions Económicas y Mercados ¿por qué $"$ las rotondas son liberales, y los semáforos son reglamentación estatal que coartan la libertad individual$"$? Indique el/los resultado/s teórico/s estudiado/s que está utilizando para su respuesta.\\\\
	    Respuesta.-\; Las rotondas son liberales debido a que son autoreguladores y los semáforos son de reglamentación estatal debido a que son reguladores. Es decir, las rotondas no tienen un ente regulador, el conductor del coche simplemente se $"$regula$"$ como también tiene la libertad de desplazarse cualquier momento, esto similar a la mano invisible de Adam Smith. Por otro lado el semáforo es un ente regulador ya que interviene y limita al conducto desplazarse cuando lo desee, esto similar a la teoría Keynesiana.\\\\

    \end{enumerate}

%----------4. 
\item Como economista (que cursó Decisions Económicas y Mercados), explique si tiene razón o no este tertuiliano.\\

    \begin{enumerate}[\bfseries i)] 	

	\item  ¿Puede explicar (de forma muy sencilla) por qué el precio de la electricidad es menor a las 3:00AM que a las 12:00AM?.\\\\
	    Respuesta.-\; El precio es menor debido a la baja demanda, que implica un nuevo punto de equilibro determinado por la disminución del coste marginal que ahora será determinado por la industria de energía nuclear.\\
	    A medida que la demanda incrementa hacemos uso, por así decirlo, de energías de ciclo combinado donde el precio tiene un realce debido a que los derechos de emisión de CO2 son más caros.\\\\

	\item (Muy importante) ¿Quién –qué agente/s económico/s determina el precio de la electricidad en cada una de estas franjas horarias? Identifique quién/es es/son esto/s agente/s.\\\\
	    Respuesta.-\; En el horario de las 3:00 AM el agente económico que determina el precio es las industria de energías nucleares. \\
	    En el horario de las 12:00 AM el agente económico que determina el precio es la industria de ciclo combinado-gas.\\\\

	\item Finalmente, explique de forma muy clara (como consecuencia de toda argumentación anterior): Las empresas eléctricas que están produciendo electricidad con molinos de viento –tecnología barata, ¿nos están robando a consumidores cuando les pagamos un precio en hora pico muy por encima del coste de producir dicha electricidad?.\\\\
	    Respuesta.-\; Según la teoría económica el coste de producir la última unidad será la que determina el precio, contraria al coste de la energía eólica que es muy barata, donde se tendrá una manipulación de precios en un mercado marginalista.

    \end{enumerate}


\end{enumerate}
