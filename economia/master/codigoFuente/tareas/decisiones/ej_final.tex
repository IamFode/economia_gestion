\begin{center}
    \textbf{\Large Ejercicio Final}
\end{center}
\vspace{1cm}

\begin{enumerate}[\Large \bfseries 1.]

%----------1.
\item Explique a un familiar que no tiene ni idea de matemáticas la intuición económica de: (Nota. Intente ajustarse a la extensión máxima. Su familiar no espera un curso de economía, sino una respuesta breve.):

    \begin{enumerate}[\bfseries 1.]

	%----------1.
	\item ¿Cuál es el problema del consumidor Marshalliano?.\\\\
	    Respuesta.-\; Las cestas que voy a consumir no pueden superar la renta que son independientes de los precios. Pediremos en la cafetería de la Facultad un café y un croissant porque el gasto en café y el gasto en croissant no supera a la renta.\\\\

	%----------2.
	\item  ¿Por qué hay pubs que abren todos los días, mientras que otros solo abren de jueves a domingo?.\\\\
	    Respuesta.-\; Porque los costes (marginales) de estar abiertos un día adicional son superiores a los ingresos (marginal). Suponiendo que los precios de los pubs varían a lo largo de los días de la semana, los pubs estarán abiertos más días porque el precio será mucho mayor esto para que les compense los costes de seguir manteniendo días adicionales abiertas.\\\\

	%----------3.
	\item  ¿Qué es eficiencia en sentido de Pareto?.\\\\
	    Respuesta.-\; Es aquella en la que es imposible mejorar el bienestar de alguna persona sin empeorar el de ninguna otra.\\\\

    \end{enumerate}

%----------2.
\item 
    \begin{enumerate}[\bfseries a)]

	%----------a)
	\item Traslade los argumentos filosóficos de Buridán al lenguaje que que hemos construido en Tema 5 (¿Cómo decide un Consumidor?).\\\\
	    Respuesta.-\;

	%----------b)
	\item Indique qué elemento para entender la decisión del asno está planteado incorrectamente por Buridán, es decir, por qué el análisis de Buridán está incompleto desde el punto de vista de la Teoría del Consumidor.\\\\
	    Respuesta.-\;

    \end{enumerate}

%----------3.
\item 
    \begin{enumerate}[\bfseries a)]

	%----------a)
	\item Los economistas, podrían estudiar el tráfico de coches y la congestión de las carreteras?.\\\\
	    Respuesta.-\;

	%----------b)
	\item Argumente con la terminología utilizada en el curso de Decisions Económicas y Mercados ´ ¿por qué $“$las rotondas son liberales, y los semáforos son reglamentación estatal que coartan la libertad individual$”$? Indique el/los resultado/s teórico/s estudiado/s que está utilizando para su respuesta.\\\\
	    Respuesta.-\;

    \end{enumerate}

%----------4. 
\item Como economista (que cursó Decisions Económicas y Mercados), explique si tiene razón o no este tertuiliano.\\\\
    Respuesta.-\;


\end{enumerate}
