\textbf{\bfseries Christian Limbert Paredes Aguilera}\\
\textbf{Análisis de decisiones económicas}\\\\
\subsection*{\center TAREA 1.A - Externalidades}
\vspace{1cm}
\textbf{Conflicto de regulación sobre alquileres a corto plazo}\\\\

\begin{enumerate}[\large\bfseries 1.]

    %-------------------- 1. -------------------------
    \item Se trata de una externalidad negativa en consumo ya que se impone un costo a un tercero. Es decir los afectados por la falta de regulación contra incendios serán las personas que ocupen el servicio de Airbnb. Por otro lado vemos que Airbnb tiene un poder de decisión más allá de su competencia directa con respecto a regulaciones y costes.\\\\

    %-------------------- 2. -------------------------
    \item Es del tipo de externalidad positiva a la producción, por el hecho de que genera un beneficio en un tercero, en este caso, genera ingresos a restaurantes, otros centros turísticos, etc. sin estar directamente relacionada con el alquiler que se presenta en este ejercicio.\\\\


    %-------------------- 3. -------------------------
    \item Sabiendo que las leyes vigentes son obsoletas, indefinidas e injustas se tendría que cambiar o modificar a un tipo de políticas y regulaciones estatales, que permita una buena administración y por ende coordinar el mercado evitando  externalidades negativas al turismo, como lo sería si se aplicase impuestos o subsidios.\\\\

    %-------------------- 4. -------------------------
    \item La externalidad de mayor magnitud es la de la pregunta 2 (Otra dimensión del mercado de alquiler a corto plazo debe ser la facturación de los grupos de la industria turística (por ejemplo, restaurantes), ya que el impacto que causa con la externalidad mencionada son a varias industrias inherentes al turismo contrario a la pregunta 1.\\\\

\end{enumerate}



\subsection*{\center TAREA 1.B - Externalidades}
\vspace{1cm}
\textbf{Coches eléctricos}\\\\
\begin{enumerate}[\large\bfseries 1.]

    %-------------------- 1. -------------------------
    \item Se pueden describir como externalidades positivas en el consumo y externalidades positiva en la producción. Por un lado y bien se sabe que un automóvil aporta beneficios de transporte al consumidor, aporta o coadyuva al cambio climático de forma positiva. Por otro lado la oferta y comercialización de autos eléctricos generará beneficios a la industria automovilística.\\
    Me gustaría abordar un tema que a largo plazo tal vez sería una externalidad negativa en el consumo, debido a que el los autos eléctricos en su mayoría necesitan baterías de litio. Si tomamos como referencia el impacto que se tuvo con del combustible fósil, podría haber un impacto negativo  referente al litio, por ejemplo: El cuerpo humano tiene alrededor de 7 miligramos de litio y a partir de 15 ya es altamente tóxico, por lo que si hay contaminación por litio en los mantos acuíferos o en los ríos, ese litio puede llegar a las comunidades y si llega vamos a tener problemas de toxicología y salud ambiental.\\\\

    %-------------------- 2. -------------------------
    \item El gráfico que representa a los coches eléctricos es la gráfica B. Ya que se tiene una externalidad positiva, es decir, el beneficio marginal social es mayor al beneficio marginal privado.\\

    %-------------------- 3. -------------------------
    \item	

    %-------------------- 4. -------------------------
    \item Me parece que una política basada en objetivos sería más eficiente que una política basada en coste ya que una política de objetivos se podría evaluar y controlar continuamente, y por lo tanto habría la oportunidad de corregir y/o modificar en función a los resultados observados.\\
    Una política de costes en mi experiencia, es más habitual que se implemente a posteriori de los hechos contraría a una política basada en objetivos.\\\\  

\end{enumerate}
