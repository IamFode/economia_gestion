\subsection*{\center TAREA 2}
\vspace{.4cm}
\subsection*{\center BIENES PÚBLICOS}
\vspace{1cm}

\begin{enumerate}[\large\bfseries A)]

    %-------------------- A
    \item 

    \begin{enumerate}[\bfseries a.]

	%----------a.
	\item \textbf{Uso de un espacio público como, por ejemplo un parque.-} Debido a que el deterioro de los parques depende del uso que se le de y por lo tanto es un bien rival, luego será imposible evitar que otros disfruten de los beneficios del parque, esto implica que sea un bien no excluible.\\\\

	%----------b.
	\item \textbf{Un bocadillo de queso.-} Por un lado supongamos que el queso es mio. A pesar de que podría invitar el queso a otras personas habría un momento que se me acabaría. Por otro lado supongamos que el queso está en el mercado entonces obtener el queso dependería de varios factores. Dicho esto concluimos que el bien es rival y excluible.\\\\

	%----------c.
	\item \textbf{Información de una página que está protegida por una contraseña.-} Ya que está protegida por una contraseña el bien es exclusive, luego por el hecho de que la información no disminuye si otros lo consumen entonces es un bien no rival.\\\\

	%----------d.
	\item \textbf{Información anunciada públicamente sobre la trayectoria del próximo huracán.-} Debido a que la información anunciada será pública por ende todos podrán ser informados, será un bien no rival y no excluible.\\\\

    \end{enumerate}

    %-------------------- B
    \item 
	\begin{enumerate}[\bfseries 1.]

	    %---------- pregunta 1
	    \item Sabiendo que el beneficio marginal social es la suma total de los beneficios de toda la sociedad, \\\\
		$$\begin{array}{r|cccc}
		    \hline\\
		    Nombres&1&2&3&4\\\\
		    \hline\\
			   Alice&60&110&150&180\\
			   Branda&80&120&140&150\\
			   Chip&120&200&270&330\\\\
			   \hline
			   \hline
		       B_{Marg}\; Social&260&170&130&100\\
		       \hline
		       \hline
		\end{array}$$

		\vspace{.4cm}
	Entonces se tiene que el nivel socialmente eficiente del bien será en 3 unidades, ya que el beneficio será mayor al coste, esto por la regla de decisión racional.\\\\

	    %---------- pregunta 2
	    \item Por el hecho de que el beneficio total menos el coste total para cada persona viene dado por:\\\\
		$$\begin{array}{r|ccccccc}
		    Nombres&1&2&3&4&Beneficio&Coste&Beneficio_{neto}\\\\
		    \hline\\
			   Alice&60&50&40&30&180&160&20\\
			   Brenda&80&40&20&10&150&160&-10\\
			   Chip&120&80&70&60&330&160&170\\
		\end{array}$$
		Entonces si los tres miembros tienen que compartir el coste del bien público por igual, Alice se quedará con las tres unidades del principio, Brenda por lo mucho querrá dos unidades, contrariamente a Chip que podría adquirir una cuarta unidad.\\\\

	\end{enumerate}

\end{enumerate}
