\subsection*{\center TAREA 3}
\vspace{.4cm}
\subsection*{\center INFORMACIÓN ASIMÉTRICA}
\vspace{1cm}

\begin{multicols}{2}

El artículo se refiere a la interacción de las diferencias de calidad y la incertidumbre puede explicar importantes instituciones del mercado laboral, para luego dar una estructura para determinar los costos económicos de la deshonestidad.\\ 
 Debido a la información asimétrica que existe en algunos mercado, se tiene incentivos para que los vendedores comercialicen mercancías de mala calidad, Como resultado, tiende a haber una reducción en la calidad promedio de los productos y también en el tamaño del mercado. También debe percibirse que en estos mercados los retornos sociales y privados difieren y, por lo tanto, en algunos casos, la intervención gubernamental puede aumentar el bienestar de todas las partes. \\\\
 Para poder entrar en el realismo del pensamiento anterior se hace referencia al mercado de automóviles. Por ejemplo se tiene autos buenos y autos malos ($"$limones$"$) que a la hora de comercializarlos, los vendedores, producen una información asimétrica ventajosa denominada selección adversa; esto por el manejo a priori del automóvil. En consecuencia los coches malos se venden al mismo precio que los buenos, ya que es imposible que un comprador pueda diferenciar entre un coche bueno y uno malo.\\\\
 Al igual del problema anterior pasa con el mercado de seguros para personas de tercera edad, donde la dificultad de conseguir un seguro es es menos probable. Para ello se hace la pregunta de: ¿Por qué el precio no sube para igualar el riesgo?. Por un lado el precio a pagar por un seguro no sería atractivo debido a los grandes costos que demanda un solicitante de tercera edad y por otro lado a pesar de la subida de precios habría una asimetría  es decir habría una selección adversa por el hecho de que la mayoría de las personas de tercera edad tiene un mayor riesgo a tener un accidente, cosa que la aseguradora solo infiere y supone. Por ello la aseguradoras trataran de maximizar beneficios evitando  y limitando a esta población dada la gran incertidumbre que generan.\\\\
 Algo similar puede ocurrir al mercado e trabajo donde los empleadores pueden negarse a contratar a personas que pertenecen a una minoría debido a factores como la raza, la calidad de la educación y las capacidades laborales generales.\\\\
El modelo de Lemons se puede utilizar para hacer algunos comentarios sobre los costos de la deshonestidad. Considere un mercado en el que los bienes se venden de forma honesta o deshonesta; El problema del comprador será identificar la calidad, donde será más propenso  que los tratos deshonestos tienden a sacar del mercado a los tratos honestos. Por lo tanto  el costo de la deshonestidad, no radica solo en la cantidad en que se estafa al comprador si no se debe incluir la pérdida incurrida por hacer desaparecer un negocio legítimo. Cabe mencionar que la deshonestidad en los negocios es un problema grave en los países subdesarrollados.\\ 
Existe evidencia considerable de que la variación de la calidad es mayor en las áreas subdesarrolladas que en las desarrolladas. Por ello la capacidad de un cliente de identificar productos malos es vital para regular la calidad de los mismo. En la producción, estas habilidades son igualmente necesarias, tanto para poder identificar la calidad de los insumos como para certificar la calidad de los productos. Y esta es una razón (añadida) por la que los comerciantes pueden, lógicamente, convertirse en los primeros emprendedores. \\\\
Por último el autor tomo como ejemplo el mercado de crédito en los países subdesarrollados relacionado con el principio del limón.  


 \end{multicols}
