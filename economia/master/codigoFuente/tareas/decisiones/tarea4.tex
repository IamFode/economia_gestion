\begin{enumerate}[\Large\bfseries 1.]

%--------------------1.--------------------
\item  \textbf{Construcción del agregado a partir de las decisiones individuales, y cálculo del equilibrio de mercado.} Se pide:

    \begin{enumerate}

	%----------1.a.
	\item[\bfseries 1.a]  \textbf{Construcción de la curvas de demanda de mercado.} Considere dos consumidores $i$ y $h$, donde sus respectivas curvas de demanda individuales de una mercancía son:


	donde sus respectivas curvas de demanda individuales de una mercancía son:
	$$q^i = d^i(p) = 21 - p$$ $$q^h = d^h(p) = 12-2p$$ 

	\begin{enumerate}[\bfseries i)]

		%----------i)
		\item Obtenga la \textbf{curva de demanda de mercado de un bien} $Q = D(p)$ como suma horizontal de las demandas individuales $(Q = D(p) \equiv d^i(p) + d^h(p))$\\\\
	    Respuesta.-\; $$Q = D(p) \equiv d^i(p) + d^h(p) = (21 - p) + (12 - 2p) = 33-3p$$\\

		%----------ii)
		\item Dibuje las curvas de demanda individuales, y la curva de demanda de mercado una al lado de la otra horizontalmente. En los tres gráficos indique las cantidades individuales y las de mercado que se consumen a un precio $p = 4$.

	    \end{enumerate}

	    \begin{multicols}{3}
		    \begin{center}
		    \begin{tikzpicture}[scale=.2]
			% abscisa y ordenada
			\tkzInit[xmax= 21,xmin=0,ymax=33,ymin=0]
			\tiny\tkzLabelXY[opacity=0.3,step=2, orig=false]
			% label x, f(x)
			\tkzDrawX[opacity= .6,label=p,right=0]
			\tkzDrawY[opacity= .6,label=$q^i$,below = -0.6]
			%dominio y función
			\draw [green,domain=0:21,thick,scale=1] plot(\x,{21-\x}); 
			\tkzText[red,above,opacity=1](10,32){\tiny $q^i = d^i(p)=21-p$}
			\draw[blue,dashed](4,0)--(4,17)--(0,17);
			\draw[blue](-2,17)node[left]{$q^i = 17$};
			\draw[blue](4,-2)node[below]{$p=4$};
		    \end{tikzpicture}
		    \end{center}

		    \begin{center}
		    \begin{tikzpicture}[scale=.2]
			% abscisa y ordenada
			\tkzInit[xmax=21,xmin=0,ymax=33,ymin=0]
			\tiny\tkzLabelXY[opacity=0.3,step=2, orig=false]
			% label x, f(x)
			\tkzDrawX[opacity= .6,label=p,right=0]
			\tkzDrawY[opacity= .6,label=$q^h$,below = -0.6]
			%dominio y función
			\draw [green,domain=0:6,thick,scale=1] plot(\x,{12-2*\x}); 
			\tkzText[red,above,opacity=1](10,32){\tiny $q^h = d^h(p) = 12-2p$}
			\draw[blue,dashed](4,0)--(4,4)--(0,4);
			\draw[blue](-1.5,4)node[left]{$q^i = 4$};
			\draw[blue](4,-2)node[below]{$p=4$};
		    \end{tikzpicture}
		    \end{center}

		    \begin{center}
		    \begin{tikzpicture}[scale=.2]
			% abscisa y ordenada
			\tkzInit[xmax= 21,xmin=0,ymax=33,ymin=0]
			\tiny\tkzLabelXY[opacity=0.3,step=2, orig=false]
			% label x, f(x)
			\tkzDrawX[opacity= .6,label=p,right=0]
			\tkzDrawY[opacity= .6,label=Q,below = -0.6]
			%dominio y función
			\draw [green,domain=0:11,thick,scale=1] plot(\x,{33-3*\x}); 
			\tkzText[red,above,opacity=1](10,32){\tiny $Q=D(p)=33-3p$}
			\draw[blue,dashed](4,0)--(4,21)--(0,21);
			\draw[blue](-2,21)node[left]{$q^i = 21$};
			\draw[blue](4,-2)node[below]{$p=4$};
		    \end{tikzpicture}
		    \end{center}
	    \end{multicols}
	    \vspace{.1cm}

	%----------1.b.
	\item[\bfseries 1.b.] \textbf{Construcción de la curvas de oferta de mercado.} Considere dos empresa, $j$ y $k$, donde sus respectivas ofertas individuales de una mercancía son:
	    $$q^j = o^j(p) = 5p-1$$ $$q^k = o^k(p) = p-2$$\\

	\begin{enumerate}[\bfseries i)]

		%----------i)
		\item Obtenga la \textbf{curva de oferta de mercado de un bien} $Q = D(p)$ como suma horizontal de las demandas individuales $(Q = O(p) \equiv o^j(p) + o^k(p))$\\\\
	    Respuesta.-\; $$Q = O(p) \equiv o^j(p) + o^k(p) = (5p-1) + (p-2) = 6p-3$$\\

		%----------ii)
		\item Dibuje las curvas de oferta individuales, y la curva de oferta de mercado una al lado de la otra horizontalmente. En los tres gráficos indique las cantidades individuales y las de mercado que se consumen a un precio $p = 4$.

	    \end{enumerate}

	    \begin{multicols}{3}
		    \begin{center}
		    \begin{tikzpicture}[scale=.2]
			% abscisa y ordenada
			\tkzInit[xmax= 19,xmin=0,ymax=33,ymin=0]
			\tiny\tkzLabelXY[opacity=0.3,step=2, orig=false]
			% label x, f(x)
			\tkzDrawX[opacity= .6,label=p,right=0]
			\tkzDrawY[opacity= .6,label=$q^i$,below = -0.6]
			%dominio y función
			\draw [green,domain=1/5:6,thick,scale=1] plot(\x,{5*\x-1}); 
			\tkzText[red,above,opacity=1](10,32){\tiny $q^j = o^j(p) = 5p-1$}
			\draw[blue,dashed](4,0)--(4,19)--(0,19);
			\draw[blue](-2,19)node[left]{$q^i = 19$};
			\draw[blue](4,-2)node[below]{$p=4$};
		    \end{tikzpicture}
		    \end{center}

		    \begin{center}
		    \begin{tikzpicture}[scale=.2]
			% abscisa y ordenada
			\tkzInit[xmax=19,xmin=0,ymax=33,ymin=0]
			\tiny\tkzLabelXY[opacity=0.3,step=2, orig=false]
			% label x, f(x)
			\tkzDrawX[opacity= .6,label=p,right=0]
			\tkzDrawY[opacity= .6,label=$q^h$,below = -0.6]
			%dominio y función
			\draw [green,domain=2:18,thick,scale=1] plot(\x,{\x-2}); 
			\tkzText[red,above,opacity=1](10,32){\tiny $q^k = o^k(p) = p-2$}
			\draw[blue,dashed](4,0)--(4,2)--(0,2);
			\draw[blue](-1.5,2)node[left]{$q^i = 2$};
			\draw[blue](4,-2)node[below]{$p=4$};
		    \end{tikzpicture}
		    \end{center}

		    \begin{center}
		    \begin{tikzpicture}[scale=.2]
			% abscisa y ordenada
			\tkzInit[xmax= 19,xmin=0,ymax=33,ymin=0]
			\tiny\tkzLabelXY[opacity=0.3,step=2, orig=false]
			% label x, f(x)
			\tkzDrawX[opacity= .6,label=p,right=0]
			\tkzDrawY[opacity= .6,label=Q,below = -0.6]
			%dominio y función
			\draw [green,domain=.5:5.5,thick,scale=1] plot(\x,{6*\x-3}); 
			\tkzText[red,above,opacity=1](10,32){\tiny $Q=O(p)=6p-3$}
			\draw[blue,dashed](4,0)--(4,21)--(0,21);
			\draw[blue](-2,21)node[left]{$q^i = 21$};
			\draw[blue](4,-2)node[below]{$p=4$};
		    \end{tikzpicture}
		    \end{center}
	    \end{multicols}
	    \vspace{.5cm}

	%----------1.c.
	\item[\bfseries 1.c.] Obtenga el equilibrio de mercado de la mercancía.\\\\
	    Respuesta.-\; Igualando la curva de oferta y demanda tenemos el precio de equilibrio,  $$33-3p=6p-3 \; \Longrightarrow \; p=4$$
	    luego obtenemos la cantidad de equilibro reemplazando $p$ en cualquiera de las curvas dadas, $$D(p)=33-4\cdot 4 \; \Longrightarrow \; D(p)=21$$\\ 
		    \begin{center}
		    \begin{tikzpicture}[scale=.2]
			% abscisa y ordenada
			\tkzInit[xmax= 19,xmin=0,ymax=33,ymin=0]
			\tiny\tkzLabelXY[opacity=0.3,step=2, orig=false]
			% label x, f(x)
			\tkzDrawX[opacity= .6,label=p,right=0]
			\tkzDrawY[opacity= .6,label=Q,below = -0.6]
			%dominio y función
			\draw [red,domain=.5:6,thick,scale=1] plot(\x,{6*\x-3}); 
			\tkzText[red,above,opacity=1](14,29){\tiny $Q=O(p)=6p-3$}
			\draw [green,domain=0:11,thick,scale=1] plot(\x,{33-3*\x}); 
			\tkzText[green,above,opacity=1](14,32){\tiny $Q=D(p)=33-3p$}
			\draw[blue,dashed](4,0)--(4,21)--(0,21);
			\draw[blue](-2,21)node[left]{$q^i = 21$};
			\draw[blue](4,-2)node[below]{$p=4$};
		    \end{tikzpicture}
		    \end{center}

    \end{enumerate}
\end{enumerate}
