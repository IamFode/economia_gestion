\subsection*{\center Ejercicio 5B. Teoría del consumidor}
\vspace{1cm}

\subsubsection*{\center I. Cálculo analítico de la función de demanda cuando la función de utilidad es diferenciable.}
\vspace{.5cm}


\begin{enumerate}[\large\bfseries 1.]

    %--------------------1.
    \item \textbf{¿Cómo calcular analíticamente la cesta óptima y la función de demanda cuando la función de utilidad es diferenciable.}\\\\
	Un consumidor, que posee una renta fija $\overline{M}[p.e., \overline{M} = 12]$, desea comprar dos mercancías que tienen un precio $(\overline{p_1} , \overline{p2})$. $[p.e., \overline{p} = (3, 1)]$. Las preferencias del consumidor sobre ambas mercancías están  representadas por la función de utilidad continua y diferenciable, $u(x1, x2) = x_1^\alpha x_2^\beta $ con $\alpha = \beta = 1$ (preferencias completas y monótonas).\\\\
Se pide:\\\\

    \begin{enumerate}[\bfseries a)]

	%----------a)
	\item \textbf{Función de demanda.} Suponga que el precio de las mercancías es $(p_1, p_2)$ y la renta del consumidor es $M$. Calcule analíticamente las funciones de demanda marshallianas;\\\\
	    Sea 

	%-----------b)
	\item \textbf{Elección óptima.} Calcule la cesta óptima para:\\\\ 

	    \begin{enumerate}[\bfseries b1)]

		%------b1
		\item Precios $\overline{p} = (\overline{p_1} , \overline{p_2} )=(3.00,1.00)$ y renta $\overline{M} =12$\\\\

		%------b2
		\item precios $\overline{p}^{'} = (\overline{p_1}^{'}, p_2 )=(12.00,1.00)$ y renta $\overline{M}=12$\\\

		%------b3
		\item precios $\overline{p}^{''} = (\overline{p_1}^{''} , p_2 )=(6.00,1.00)$ y renta $\overline{M} =12$.\\\\.

	    \end{enumerate}
	
	%----------c)
	\item Curvas de demanda. Represente gr´aficamente las curvas de demanda de los bienes 1 y
2 (respecto de sus precios), y sit´ue en las curvas de demanda las cestas ´optimas obtenidas
en b1)-b2)-b3) (Pista.- Recuerde la definici´on de curva de demanda, y tenga cuidado con los
desplazamientos de las curvas).\\\\
	
	%----------d)
	\item Función indirecta de utilidad. Calcule la función indirecta de utilidad.\\\\

    \end{enumerate}

\end{enumerate}
