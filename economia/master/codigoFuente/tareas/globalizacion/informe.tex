
\chapter{Indicadores económicos de Bolivia}

\section{Informe de situación 2015}

    \subsection{Indicadores generales}
    En 2015 la economía Boliviana alcanzó un producto interior bruto de 33.000 millones de dólares y una población de 10.725 millones. Bolivia representaba el 0,04 \% del PIB mundial y el 0,15 \%  de la población. De donde se tiene que el peso relativo de Bolivia en términos de PIB es de 0,27 veces el valor relativo de la población.\\
    El PIB per cápita fue de \$ 3.077,03, valor inferior al promedio mundial (0.3 veces menor), aunque 53.58 \% por encima del valor promedio de los países de ingreso medio bajos. En términos dólares por persona / día, un Boliviano disfruta de una cantidad diaria de bienes y servicios equivalente a 8.43 dólares al día, significativamente inferior a los 27,87 dólares correspondientes a la media mundial.\\
    En el caso de la economía Boliviana, prestamos especial atención a la tasa de paro y a la tasa de inflación. Por lo que en 2015 se registró una tasa de desempleo solo del 3.07 \%, inferior al promedio mundial (5.45\%). En cuanto a la inflación, los precios en Bolivia disminuyeron en 4.62 \%, La trayectoria descendente de la inflación estuvo determinada principalmente por el buen desempeño del sector agropecuario, bajas presiones inflacionarias externas y la estabilidad cambiaria, así como por el comportamiento estable de las tarifas de servicios y las expectativa inflacionarias moderadas \footnote{https://www.reuters.com/article/economia-bolivia-inflacion-idLTAKBN0UL12D20160107}, siendo una de las más bajas de América del Sur. \\

    \subsection{Oferta: Crecimiento}
    La economía Boliviana hacia 2015 creció un 4.86 \%, superando en casi dos veces a los países de referencia, que crecieron de media a un ritmo del 2.86 \%. A pesar de un significante crecimiento el PIB per capita se mantuvo igual en relación al año anterior.\\
    Suponiendo que el factor trabajo es homogéneo, y que se emplea la misma tecnología y una dotación fija de los restantes factores productivos Bolivia tuvo una productividad aparente total superior a los países de ingreso medios bajos, pero extremadamente inferior al promedio mundial, es decir, cada persona empleada tendría que haber recibido 4735 dolares al año muy por debajo de los 17318 dolares de la media mundial.\\
    Las diferencias de renta per cápita y de productividad se reducen cuando expresamos las unidades en paridad de poder adquisitivo. En Bolivia un consumidor puede adquirir una cesta de bienes y servicios más abundante. Con respecto a la Latino América, tenemos una productividad media inferior de 14877 dólares per cápita frente a 32444 dólares.


    \subsection{Oferta: Especialización productiva y eficiencia}
    En términos de producción y empleo sectorial, la economía Boliviana mantiene las características de un país en vías de desarrollo, ya que se tiene un bajo peso de la agricultura en términos de producción y empleo (más del 5 por ciento), peso decreciente de la industria pero cerca del 25 por ciento de referencia y mayor importancia absoluta en relación con el sector servicios (menos del 70 por ciento). Por lo que Bolivia encaja de manera casi ideal a este perfil.\\
    No está por demás señalar que la agricultura generó 337905 millones de dólares que equivale al 0.12 \% del promedio mundial. La industria genero 831685 millones de dolares y los servicios generaron 1517230 millones de dolares el cual equivale al 0.04\% y 0.03 \% respectivamente al promedio mundial.\\
    Lo dicho anteriormente va acompañada de una baja productividad en todos los sectores. Ya la comparación de los valores de Bolivia con respecto al mundo ya es muy ilustrativa, como veremos a continuación: Frente a una productividad agrícola en Bolivia de 1741 dólares, la media mundial es de 2.278 dólares. Frente a una productividad  de la industria de 5297 dolares, la media mundial es de 19700 dolares y frente a una productividad de servicios de 4387, la media mundial es de 22371 dolares. Está claro que nuestras metas de crecimiento deben basarse en el crecimiento de la productividad en todos los sectores en especial en de la industria y servicios.\\


    \subsection{Oferta: Especialización comercial y competitividad}
    El sector exterior es el mejor indicador para medir la competitividad de un país. En el mercado interno, el gobierno goza de cierta autonomía para establecer reglas que protejan a sus empresas de empresas extranjeras. Esta opción no encaja en los mercados extranjeros.\\
    Por lo dicho vemos que la mitad de las exportaciones en Bolivia vienen dadas por energías, más específicamente por el gas natural que se vende a Perú, Paraguay, Brasil y Argentina \footnote{$https://ibce.org.bo/images/ibcecifras_documentos/Cifras-902-Bolivia-Exportaciones-gas-natural.pdf$}, y por la exportación de minerales con un peso de 26.46 \%, es decir, el 76.46 \% de las exportaciones en Bolivia vienen son por la explotación de materias primas. Por lo que queda demostrado que Bolivia es un país extractivista. Contrario a esto Bolivia tiene una débil venta en exportaciones manufactureras. Comparando este último con países de ingresos altos, para el año 2015 se tuvo una diferencia de 67.15 \% puntos porcentuales.\\
    En 2015, la economía Bolivia mostró una tasa de apertura del 67,9 por ciento. Este peso del sector exterior sobre el PIB Boliviano fue un valor mayor en 10 puntos porcentuales a las medias mundiales, en 8\% a los países de ingresos altos y 24 \%  a América Latina y el Caribe. Por lo que podemos mencionar que el grado de integración de Bolivia y comercio exterior es alto. Para ese año, la tasa de cobertura, nos muestra que existió un déficit, ya que la exportaciones no pudieron cubrir los pagos requeridos por las importaciones. Ya que los ingresos de importaciones superaron a las  exportaciones en  16.72 \%. Cabe mencionar que este déficit representó el 7.8\%  por ciento del PIB.\\


    \subsection{Demanda: Sostenibilidad}
    Desde el punto de vista de la demanda agregada podemos analizar si la estructura de gasto de un país es sostenible en el tiempo. El principal objetivo de la actividad económica es satisfacer las necesidades actuales de las personas, por lo que es evidente que gran parte de la producción es demanda para el consumo, ya sea privado o público.\\
    Con un 17.5\% vemos que el porcentaje de consumo público para Bolivia es muy similar a los grupos de referencia. Y con un 68.4\% el consumo privado está 10 puntos porcentuales por encima de los grupos de referencias.\\
    El consumo futuro dependerá del esfuerzo que se haga en términos de inversión y de cómo se financie esta inversión. Para ello es importante conocer la capacidad de ahorro del país. Un país se puede salvar con recursos internos y también con recursos externos. Por lo que es fundamental conocer la relación entre tasa de inversión, tasa de ahorro y balanza comercial.\\
    En 2015, Bolivia tenía una tasa de inversión del 20,3 \%, cifra superior a la tasa de ahorro, que fue del 14.1\%. Por lo que parte del ahorro interno no fue suficiente para financiar necesidades de financiamiento y lograr cierta capacidad de financiamiento de otras economías. Por lo que que se recurrió a deuda externa pública por parte del gobierno Bolivia\footnote{$https://www.bcb.gob.bo/webdocs/informes_deudaexterna/Informe_Anual_2015.pdf$}.
    Una condición necesaria para el crecimiento económico es que un país adquiera deuda externa para cubrir las carencias del bajo ahorro interno. Es la forma de poder renovar el aparato productivo para ser más eficientes y competitivos en el futuro. Sin embargo, el país debe incrementar sus ahorros con el tiempo, incrementar sus exportaciones y así convertirse en acreedor neto. Este diagnóstico sólo lo podemos hacer a través del análisis de la evolución del país en un período más largo y acompañado de otros indicadores que nos muestren dónde se están utilizando los ahorros o los préstamos.\\
    Por otro lado dedicar recursos a la investigación es reducir el impacto de la actividad económica sobre el medio ambiente que no tendrán  repercusiones inmediatas, pero sus efectos acumulativos en el tiempo pueden tener un impacto significativo en el crecimiento futuro del país. Es probable que cuantos más investigadores contratemos, más gastemos en investigación y desarrollo, menos contaminemos y menos energía usemos hoy es decir el crecimiento tendrá un rumbo sostenible en el futuro. Lamentablemente estos datos no se encuentran disponibles para Bolivia, por el bajo incentivo hacia la investigación de parte de entidades públicas como privadas.\\

    \subsection{Demanda e ingreso: Capital}
    La acción del sector público y la distribución del ingreso permiten abordar la cuestión de cómo la actividad económica afecta otras variables sociales, especialmente la equidad entre las personas. El sector público tiene la capacidad a través del gasto público de cambiar la distribución del ingreso que se produce en el mercado. El gasto público en educación y el gasto público en salud juegan aquí un papel relevante. Si hay elementos se utilizan para este fin, el resultado se observará en los indicadores de distribución del ingreso.\\
    A medida que desagregamos los componentes del PIB, será más difícil encontrar información actualizada para todos los países. El número de países que presentan todos los datos ya es menor, falta homogeneidad temporal en muchos indicadores, por lo que los promedios por grupo ya no son representativos. En este caso, conviene tomar como referencia ciertos países que sirven como casos modelo. Por lo que tomaremos a España como punto de referencia.\\
    Primeramente, destacamos que Bolivia ha realizado un menor esfuerzo relativo en cuanto a recursos destinados a la sanidad respecto a su PIB ( 4.4 \%), muy por debajo de España con un 6.5 \%.\\
    En segundo lugar, Bolivia en términos de índice de Gini muestra un valor de 46.7, muy por encima de  España con un 36.2. El veinte por ciento de la población Boliviana con mayor renta per cápita (quintil superior) concentra el 51.1 \% de la renta, mientras que el quintil inferior recibe sólo el 3.9 \%. Frente a un ingreso promedio de \$ 7862 de los más ricos, (similar renta que reciben los más pobres en España \% 7479) y un ingreso promedio de los más pobres de \$ 600. En términos relativos, la tasa de desigualdad en Bolivia fue de 13.1. Esta ratio de desigualdad es mayor en  España (7.26). 
    
    \subsection{Resumen}
    Con los datos del año 2015 podemos deducir que Bolivia muestra características de un País en vías de desarrollo. Su ingreso per cápita está claramente dentro del grupo de países de ingresos mediano bajo. Mantiene también una estructura productiva y comercial similar a los países en vías de desarrollo, así como una equidad económica más cercana a este grupo de países. También notamos que tiene ciertas debilidades o aspectos a mejorar, como la baja utilización de mano de obra y la necesidad de dirigir la inversión a actividades con mayor potencial de crecimiento futuro. Determinar en qué medida estas características apuntan a un crecimiento equilibrado y sostenible requiere una evaluación de un período más amplio. Este es el objetivo de la segunda parte del informe.

\section{Informe de evolución 1990-2015}
A partir de los gráficos del documento proporcionado evaluaremos la economía española en el periodo 1990-2015. Como hemos dicho, se trata de determinar si la economía Boliviana crecerá económicamente durante ese período de forma equilibrada y sostenible.

    \subsection{Crecimiento}
    En el gráfico 1.1  En términos de población, Bolivia está ganando peso sobre el conjunto mundial. A principios de los 90 la población Boliviana suponía cerca del 0,13 \% de la población mundial, y para 2015 se tuvo casi un 0,15 \%, es decir, la tendencia es levemente positiva.\\
    La tendencia demográfica estuvo acompañada de una tendencia cíclica desde el punto de vista del peso del PIB. El período 1990-2004 experimentó una casi-estabilidad de peso relativa del 0.02 \% del PIB. Para luego tener un realce significativo de 0.25 puntos porcentuales para el año 2015. La estabilidad de las exportaciones tuvieron lugar en el intervalo de 1990 y 2004 con aproximadamente 0.02 \% y después se tuvo el boom de los precios de las materias primas para los años 2004 al 2015 con un realce en las exportaciones de 0.4 puntos porcentuales del PiB mundial.\\
    Es interesante notar cierta correlación entre el PIB y las exportaciones esto cuando el peso relativo de las exportaciones es superior al peso relativo del PIB (período 2004-2015). Para estos años se tuvo ingresos externos para incentivar una mayor producción interna.  Para los años a priori (1990-2004) se tuvo una estabilidad constante de las exportaciones y el PIB.\\

    El Gráfico 1.2 muestra que el PIB per cápita constante al año 2010 en Bolivia creció constantemente entre los años 1990 y 2105 con una tendencia positiva y constante, con un aumento significativo desde los 710 dolares hasta los 3077 dolares. Es decir, el ingreso per cápita aumento cada año en promedio un 4\%. \\
    Según la gráfica, Bolivia se encuentra ligeramente por encima de los países de ingresos medios bajos, y por debajo de los países de ingresos medios altos. Mientras estos últimos tuvieron un realce significativo entre los años 2003 y 2015, Bolivia siguió constante en su crecimiento junto a los países de ingresos medios bajos.
    A la par de los ingresos per cápita y explicando esta trayectoria de la economía Boliviana vemos que la evolución de la productividad y el empleo también per cápita creció constante con una ligera caída entre los años 1998 y 2002. A pesar de esta caída el empleo empezó un ascenso desde 2000 hasta 2011, esto con distintas políticas como el plan generación de empleo e inserción laboral, bolsas de empleo y capacitación laboral \footnote{$https://www.cepal.org/es/publicaciones/3715-politicas-la-insercion-laboral-mujeres-jovenes-estado-plurinacional-bolivia$}.

    \subsection{Especialización sectorial y productividad}
    La evolución Boliviana de la estructura sectorial de la producción muestra una clara tendencia a acercarse al perfil de los países en vías de desarrollado, pero sin algún avance significativo. El peso del sector de servicios tuvo un auge hacia 1998 con un 53\% para luego empezar un descenso hasta 2008 (40\%), y luego relativamente constante hasta 2014, este último año con una tendencia hacia la alza. La estructura sectorial de producción estuvo constante a lo largo de tiempo con un 30\%. Con respecto a la estructura sectorial agraria se tuvo un descenso de 6 puntos porcentuales a lo largo de los 25 años.\\

    Mientras la producción de servicios tuvo una bajada en los años 1998-2008 la evolución del empleo en términos de producción subió en casi 10\% puntos porcentuales y siguió su ascenso hasta el 2015. El sector industrial a lo largo de 25 años siguió constante principalmente en términos de empleo que parece mantener en entorno de del 20\%. Y en el caso del empleo del sector agrario y en concordancia con la evolución sectorial se ve una tendencia negativa para casi todo el periodo estudiado en 12 puntos porcentuales. Algo interesante que recalcar es el crecimiento del empleo en el sector agrícola y el decrecimiento en los sectores de servicios e industria para el año 2001 luego de la crisis económica que inició en 1997 y que afectó fuertemente la industria agraria.\\ 

    \subsection{Comercio exterior y productividad}
    Durante el período 1990-1998 el  grado de apertura de la economía de Bolivia, considerando su comercio exterior en relación con el conjunto de su actividad económica global mostró una tendencia mayoritariamente constante (50\%). Luego en 1999 hasta 2008 tuvo una tendencia hacia la alza en casi 20\% para luego volver a una tendencia relativamente constante. Por otro lado el porcentaje de las exportaciones que pueden pagarse con las importaciones registradas tuvo un realce significativo entre los periodos 1999 a 2004 esto por los tratados y acuerdos de libre comercio y políticas de comercio internacional que Bolivia acordó con los distintas regiones del continente.\\

    Con respecto al volumen de exportación de manufacturas y minería se tuvo una caída cíclica en el tiempo (1990-2015) de casi 20 puntos porcentuales, con algunos reactivaciones momentáneas en los años 199 y 2011 respectivamente. El volumen de exportaciones en temas de energía mas concretamente en la venta de gas en todos sus derivados representó desde el año 1999 al 2015 unos años de bonanza para Bolivia en casi 55\%, esto se debió más que todo al alza en los precios del petroleo y nuevos contratos con los países aledaños.\\
    Para el crecimiento económico, las exportaciones de bienes de alta tecnología juegan un papel relevante importante, un aspecto que debería tomar un papel primordial para mejorar la economía Boliviana. Lamentablemente no se tiene aún datos con lo que demuestra que se tiene mucho que hacer y trabajar.

    \subsection{Sustentabilidad}
    La evolución del ahorro y la inversión permite valorar si el modelo de crecimiento de la economía estuvo acompañado durante este periodo de criterios de sostenibilidad. 

    Observamos que durante el periodo de análisis, la economía Boliviana atravesó etapas de capacidad de financiación externa y etapas de necesidades de financiación, aunque predominaron las primeras, es decir, los años en los que la tasa de ahorro fue mayor que la de inversión. Básicamente, podemos ver que el crecimiento económico del PIB experimentado en los años 1990-2003 fue financiado en términos netos por financiamiento externo,  donde la tasa de inversión creció más rápido que la tasa de ahorro, lo que resultó en una tasa de financiación negativa. Una tasa de financiación negativa no es necesariamente indicativa de una mala estrategia de crecimiento. Es obvio que si un país quiere crecer necesita financiamiento e inversión extranjera que le permita innovar su proceso productivo para ser más competitivo en los mercados externos. La condición para la sostenibilidad financiera es que en los próximos años aumente el PIB, aumenten los ingresos y la economía comience a reducir sus déficits comerciales y pagar los préstamos. En este caso la economía en Bolivia aumento desde el año 2003 tanto en el ahorro El PIB y la tasa de inversión. \\
    Uno de los problemas latentes en Bolivia en la falta de inversión y gasto en I+D sobre el pib donde no se tiene muchos datos que puedan dar alguna directriz de la evolución del mismo.\\
    Los indicadores seleccionados en el Gráfico 4.3 de sostenibilidad ambiental también muestran que el modelo de crecimiento Boliviano fue más contaminante y más consumidor de energía en comparación con los países de renta media baja. Tanto la emisión CO2 y el consumo de energía crecieron de igual forma esto podría explicar la contaminación que emite la extracción de recursos energéticos relacionados con hidrocarburos. El crecimiento de la sustentabilidad medioambiental estuvo dado por más de 20 puntos porcentuales. Pensamos que contaminar y consumir más energía con otros países acaba dificultando competir a largo plazo en tecnologías o patrones de consumo más respetuosos con el medio ambiente.

    \subsection{Equidad}
    Podemos afirmar que, en general, 1990-2004, el consumo público aumentó a un ritmo mayor que el PIB, representando el primer indicador casi 2 puntos porcentuales más que el segundo. A pesar de la tendencia a la baja de los años 2001 a 2005, el porcentaje de consumo público se mantuvo por encima del inicio del período, para luego solo en 2015 tener un aumento muy significativo del 2\% similar al aumento en los periodos 1990-2004. Este mayor consumo público no fue del todo acompañado del aumento tanto en el gasto público en educación como en salud. Este último si tuvo un crecimiento entre los años 2000 y 2009 para luego una atenuación para los años 2009-2013.\\
    En el período 1990-1997 hubo aumentos en el índice de Gini, de 18 puntos porcentuales , provocando una mayor disparidad de casi 47 puntos entre el decil superior y el decil inferior. Ya para el periodo 1990-2015 el índice de Gini empezó una caída constante de más o menos 10 puntos porcentuales. La falta de información para todo el período nos obliga a estar muy atentos a la evolución del patrimonio. Para una visión más completa sería necesario realizar un análisis paralelo con datos del INE. Podemos comentar aquí que esta fuente confirma que la tendencia en los últimos años ha sido hacia una mayor concentración del ingreso, a pesar de la reducción de las desigualdades.

    \section{Conclusiones}	
    La economía Boliviana mantuvo una favorable y posiblemente constante crecimiento entre todo el periodo analizado, especialmente a partir de 2002, pero entró en una recesión a partir de 2014. Este crecimiento ha contribuido a que Bolivia consolide ciertas características de los países en vías de desarrollados, pero también aclare sus debilidades. Este informe muestra que hay una serie de indicadores que parecen más apropiados para evaluar si el proceso de crecimiento del PIB o PIB per cápita ha sido equilibrado y, sobre todo, sostenible en el largo plazo. \\
    En algunos de los indicadores estudiados, la economía Boliviana ha convergido considerablemente a la media de los países en vías de desarrollados de referencia, más específicamente en países de ingresos medios bajos. Bolivia tiene mucho que mejorar en temas ambientales, de desarrollo e investigación y en temas de tecnología, es decir, que a pesar del crecimiento constante que tuvo Bolivia que se debió a las crecientes subidas del precio del petroleo que se experimentos a partir de los años 2002 aproximadamente, se debe empezar a explorar nuevas fuentes de ingresos relacionados a la investigación y la tecnología.


\chapter{Objetivos del milenio (Bolivia)}

En cuanto al objetivo 1, erradicación de la pobreza y el hambre, se observan mejoras. El porcentaje de la población que vive con menos de \$ 1,90 al día se ha reducido del 18 \% en 1996 al 3 \% esto de 2000 al 2015 \textbf{(Objetivo 1.1)}. También hay una reducción significativa en la línea de pobreza nacional \textbf{(Objetivo 1.2)}. Para  \text{Objetivo 1.3b}  con la participación indicada del ingreso del grupo más pobre en el ingreso total, se tiene una tendencia positiva de 1\% a 4\% por lo tanto  hubo una mejora. En cuanto al \textbf{(Objetivo 1.8)}, la existencia de bajo peso al nacer en los niños, la proporción bajó del 11 por ciento al 4 por ciento.\\
En cuanto a la educación, hay una menor tasa de matrícula en la educación primaria \textbf{(Objetivo 2.1)}. De valores iniciales del 95  \%, en años más recientes a 89 \%. La tasa de alumnos que terminan la escuela primaria decendió desde 2010 al 2015 en 9 punto porcentuales \textbf{(Objetivo 2.2)}, aunque hubo un aumento de la tasa de alfabetización entre los 15 y los 24 años \textbf{(Objetivo 2.3).}\\
En cuanto a la igualdad de género, la proporción de mujeres en el parlamento tuvo un  ascenso de 9\% a 18\& \textbf{(Objetivo 3.3)}. Con respecto al indicador de mujeres que trabajan fuera del sector agrícola entre 1990 y 2007 tuvo un realce de 35\% a 42\% \textbf{(Objetivo 3.2)}.\\
Tanto la mortalidad infantil como la de menores de 5 años se han reducido notablemente. La primera de 120 a 40 muertes por cada mil nacimientos \textbf{(Objetivo 4.1)}, y el el segundo de 83 a 30 por mil nacimientos \textbf{(Objetivo 4.2)}. El aumento de la inmunización contra el sarampión está contribuyendo a esta reducción de la mortalidad en este caso de 47\% a 58\% de inmunidad \textbf{(Objetivo 4.3).}\\
La tasa de mortalidad materna se destaca a la baja de 4250 mujeres a 2000 mujeres por cada 100000 \textbf{Objetivo 5.1}. Similar tendencia tiene la tasa de fecundidad con valores iniciales de más de 90 nacimientos por cada 1.000 mujeres de 15 a 19 años, a 70 por 1000. \textbf{(Objetivo 5.4)}. Sin embargo, sigue siendo alto un gran número de partos que continúan realizándose sin la atención de un médico  especialista de 40\% a más de 80\% \textbf{(Objetivo 5.2)}.\\
El objetivo 6 relacionado con la enfermedad muestra una evolución  con la falta de información. Existe  más personas VIH positivas tratadas con medicamentos antirretrovirales (casi el 30 por ciento), con un crecimiento  exponencial \textbf{(Objetivo 6.5)}. Seguido a que existe una disminución de muertes por tuberculosis de 18 a 10 personas por cada cien mil \textbf{(Objetivo 6.9)}. Por último los casos de malaria reportados ascendieron en 1997 a casi 80000 personas para luego hasta 2015 tener una disminución significativa con menos de 10000 casos.\\
Con respecto al medio ambiente \textbf{(Objetivo 7)}, los indicadores muestran tendencias positivas y negativas. En el lado positivo, las emisiones de CO2 por unidad de PIB se redujeron cíclicamente de 0,34 a 0,29 kg por dólar de PIB en paridad de poder adquisitivo \textbf{(Objetivo 7.2)} y la población con acceso a saneamiento aumentó de 30 \% a 50 \% \textbf{(Objetivo 7.8)}. Otro punto positivo es la proporción de la población que vive en barrios marginales que disminuyo del 60 al 42 por ciento.\\
Finalmente, con respecto a la objetivo 8 Bolivia es un país que recibe en promedio  \$ 80 per cápita en Ayuda Oficial al Desarrollo \textbf{(Objetivo 8.5)}. El servicio de la deuda externa cayó del 35 \% en 1993 al 3 \% en 2014 \textbf{(Objetivo 8.12)}. En cuanto a las comunicaciones, se ha incrementado significativamente el número de teléfonos móviles hasta llegar casi al 100 \% \textbf{(Objetivo 8.15)} similar crecimiento se tuvo con la población con acceso a internet pero con cifras de 0 al 35\% \textbf{(Objetivo 8.16)}. 


\section{Resumen}
El crecimiento económico observado para Bolivia durante el período que se examina estuvo acompañado de cambios en su gran mayoría positivos. En resumen, pareció tener más efectos sobre la mayoría de las variables relacionadas con la salud, y acceso a telefonía e internet.  Poca mejora se  observan  en las variables educativas.  y en la sostenibilidad ambiental.




