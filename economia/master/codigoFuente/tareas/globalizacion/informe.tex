


\section{Informe de situación 2015}

    \subsection{Indicadores generales}
    En 2015 la economía Boliviana alcanzó un producto interior bruto de 33.000 millones de dólares y una población de 10.725 millones. Bolivia representaba el 0,04 \% del PIB mundial y el 0,15 \%  de la población. De donde se tiene que el peso relativo de Bolivia en términos de PIB es de 0,27 veces el valor relativo de la población.\\
    El PIB per cápita fue de \$ 3.077,03, valor inferior al promedio mundial (0.3 veces menor), aunque 53.58 \% por encima del valor promedio de los países de ingreso medio bajos. En términos dólares por persona / día, un Boliviano disfruta de una cantidad diaria de bienes y servicios equivalente a 8.43 dólares al día, significativamente inferior a los 27,87 dólares correspondientes a la media mundial.\\
    En el caso de la economía Boliviana, prestamos especial atención a la tasa de paro y a la tasa de inflación. Por lo que en 2015 se registró una tasa de desempleo solo del 3.07 \%, inferior al promedio mundial (5.45\%). En cuanto a la inflación, los precios en Bolivia disminuyeron en 4.62 \%, La trayectoria descendente de la inflación estuvo determinada principalmente por el buen desempeño del sector agropecuario, bajas presiones inflacionarias externas y la estabilidad cambiaria, así como por el comportamiento estable de las tarifas de servicios y las expectativa inflacionarias moderadas \footnote{https://www.reuters.com/article/economia-bolivia-inflacion-idLTAKBN0UL12D20160107}, siendo una de las más bajas de América del Sur. \\

    \subsection{Oferta: Crecimiento}
    La economía Boliviana hacia 2015 creció un 4.86 \%, superando en casi dos veces a los países de referencia, que crecieron de media a un ritmo del 2.86 \%. A pesar de un significante crecimiento el PIB per capita se mantuvo igual en relación al año anterior.\\
    Suponiendo que el factor trabajo es homogéneo, y que se emplea la misma tecnología y una dotación fija de los restantes factores productivos Bolivia tuvo una productividad aparente total superior a los países de ingreso medios bajos, pero extremadamente inferior al promedio mundial, es decir, cada persona empleada tendría que haber recibido 4735 dolares al año muy por debajo de los 17318 dolares de la media mundial.\\
    Las diferencias de renta per cápita y de productividad se reducen cuando expresamos las unidades en paridad de poder adquisitivo. En Bolivia un consumidor puede adquirir una cesta de bienes y servicios más abundante. Con respecto a la Latino América, tenemos una productividad media inferior de 14877 dólares per cápita frente a 32444 dólares.


    \subsection{Oferta: Especialización productiva y eficiencia}
    En términos de producción y empleo sectorial, la economía Boliviana mantiene las características de un país en vías de desarrollo, ya que se tiene un bajo peso de la agricultura en términos de producción y empleo (más del 5 por ciento), peso decreciente de la industria pero cerca del 25 por ciento de referencia y mayor importancia absoluta en relación con el sector servicios (menos del 70 por ciento). Por lo que Bolivia encaja de manera casi ideal a este perfil.\\
    No está por demás señalar que la agricultura generó 337905 millones de dólares que equivale al 0.12 \% del promedio mundial. La industria genero 831685 millones de dolares y los servicios generaron 1517230 millones de dolares el cual equivale al 0.04\% y 0.03 \% respectivamente al promedio mundial.\\
    Lo dicho anteriormente va acompañada de una baja productividad en todos los sectores. Ya la comparación de los valores de Bolivia con respecto al mundo ya es muy ilustrativa, como veremos a continuación: Frente a una productividad agrícola en Bolivia de 1741 dólares, la media mundial es de 2.278 dólares. Frente a una productividad  de la industria de 5297 dolares, la media mundial es de 19700 dolares y frente a una productividad de servicios de 4387, la media mundial es de 22371 dolares. Está claro que nuestras metas de crecimiento deben basarse en el crecimiento de la productividad en todos los sectores en especial en de la industria y servicios.\\


    \subsection{Oferta: Especialización comercial y competitividad}
    El sector exterior es el mejor indicador para medir la competitividad de un país. En el mercado interno, el gobierno goza de cierta autonomía para establecer reglas que protejan a sus empresas de empresas extranjeras. Esta opción no encaja en los mercados extranjeros.\\
    Por lo dicho vemos que la mitad de las exportaciones en Bolivia vienen dadas por energías, más específicamente por el gas natural que se vende a Perú, Paraguay, Brasil y Argentina \footnote{$https://ibce.org.bo/images/ibcecifras_documentos/Cifras-902-Bolivia-Exportaciones-gas-natural.pdf$}, y por la exportación de minerales con un peso de 26.46 \%, es decir, el 76.46 \% de las exportaciones en Bolivia vienen son por la explotación de materias primas. Por lo que queda demostrado que Bolivia es un país extractivista. Contrario a esto Bolivia tiene una débil venta en exportaciones manufactureras. Comparando este último con países de ingresos altos, para el año 2015 se tuvo una diferencia de 67.15 \% puntos porcentuales.\\
    En 2015, la economía Bolivia mostró una tasa de apertura del 67,9 por ciento. Este peso del sector exterior sobre el PIB Boliviano fue un valor mayor en 10 puntos porcentuales a las medias mundiales, en 8\% a los países de ingresos altos y 24 \%  a América Latina y el Caribe. Por lo que podemos mencionar que el grado de integración de Bolivia y comercio exterior es alto. Para ese año, la tasa de cobertura, nos muestra que existió un déficit, ya que la exportaciones no pudieron cubrir los pagos requeridos por las importaciones. Ya que los ingresos de importaciones superaron a las  exportaciones en  16.72 \%. Cabe mencionar que este déficit representó el 7.8\%  por ciento del PIB.\\


    \subsection{Demanda: Sostenibilidad}
    Desde el punto de vista de la demanda agregada podemos analizar si la estructura de gasto de un país es sostenible en el tiempo. El principal objetivo de la actividad económica es satisfacer las necesidades actuales de las personas, por lo que es evidente que gran parte de la producción es demanda para el consumo, ya sea privado o público.\\
    Con un 17.5\% vemos que el porcentaje de consumo público para Bolivia es muy similar a los grupos de referencia. Y con un 68.4\% el consumo privado está 10 puntos porcentuales por encima de los grupos de referencias.\\
    El consumo futuro dependerá del esfuerzo que se haga en términos de inversión y de cómo se financie esta inversión. Para ello es importante conocer la capacidad de ahorro del país. Un país se puede salvar con recursos internos y también con recursos externos. Por lo que es fundamental conocer la relación entre tasa de inversión, tasa de ahorro y balanza comercial.\\
    En 2015, Bolivia tenía una tasa de inversión del 20,3 \%, cifra superior a la tasa de ahorro, que fue del 14.1\%. Por lo que parte del ahorro interno no fue suficiente para financiar necesidades de financiamiento y lograr cierta capacidad de financiamiento de otras economías. Por lo que que se recurrió a deuda externa pública por parte del gobierno Bolivia\footnote{$https://www.bcb.gob.bo/webdocs/informes_deudaexterna/Informe_Anual_2015.pdf$}.
    Una condición necesaria para el crecimiento económico es que un país adquiera deuda externa para cubrir las carencias del bajo ahorro interno. Es la forma de poder renovar el aparato productivo para ser más eficientes y competitivos en el futuro. Sin embargo, el país debe incrementar sus ahorros con el tiempo, incrementar sus exportaciones y así convertirse en acreedor neto. Este diagnóstico sólo lo podemos hacer a través del análisis de la evolución del país en un período más largo y acompañado de otros indicadores que nos muestren dónde se están utilizando los ahorros o los préstamos.\\
    Por otro lado dedicar recursos a la investigación es reducir el impacto de la actividad económica sobre el medio ambiente que no tendrán  repercusiones inmediatas, pero sus efectos acumulativos en el tiempo pueden tener un impacto significativo en el crecimiento futuro del país. Es probable que cuantos más investigadores contratemos, más gastemos en investigación y desarrollo, menos contaminemos y menos energía usemos hoy es decir el crecimiento tendrá un rumbo sostenible en el futuro. Lamentablemente estos datos no se encuentran disponibles para Bolivia, por el bajo incentivo hacia la investigación de parte de entidades públicas como privadas.\\

    \subsection{Demanda e ingreso: Capital}
    La acción del sector público y la distribución del ingreso permiten abordar la cuestión de cómo la actividad económica afecta otras variables sociales, especialmente la equidad entre las personas. El sector público tiene la capacidad a través del gasto público de cambiar la distribución del ingreso que se produce en el mercado. El gasto público en educación y el gasto público en salud juegan aquí un papel relevante. Si hay elementos se utilizan para este fin, el resultado se observará en los indicadores de distribución del ingreso.\\
    A medida que desagregamos los componentes del PIB, será más difícil encontrar información actualizada para todos los países. El número de países que presentan todos los datos ya es menor, falta homogeneidad temporal en muchos indicadores, por lo que los promedios por grupo ya no son representativos. En este caso, conviene tomar como referencia ciertos países que sirven como casos modelo. Por lo que tomaremos a España como punto de referencia.\\
    Primeramente, destacamos que Bolivia ha realizado un menor esfuerzo relativo en cuanto a recursos destinados a la sanidad respecto a su PIB ( 4.4 \%), muy por debajo de España con un 6.5 \%.\\
    En segundo lugar, Bolivia en términos de índice de Gini muestra un valor de 46.7, muy por encima de  España con un 36.2. El veinte por ciento de la población Boliviana con mayor renta per cápita (quintil superior) concentra el 51.1 \% de la renta, mientras que el quintil inferior recibe sólo el 3.9 \%. Frente a un ingreso promedio de \$ 7862 de los más ricos, (similar renta que reciben los más pobres en España \% 7479) y un ingreso promedio de los más pobres de \$ 600. En términos relativos, la tasa de desigualdad en Bolivia fue de 13.1. Esta ratio de desigualdad es mayor en  España (7.26). 
    
    \subsection{Resumen}
    Con los datos del año 2015 podemos deducir que Bolivia muestra características de un País en vías de desarrollo. Su ingreso per cápita está claramente dentro del grupo de países de ingresos mediano bajo. Mantiene también una estructura productiva y comercial similar a los países en vías de desarrollo, así como una equidad económica más cercana a este grupo de países. También notamos que tiene ciertas debilidades o aspectos a mejorar, como la baja utilización de mano de obra y la necesidad de dirigir la inversión a actividades con mayor potencial de crecimiento futuro. Determinar en qué medida estas características apuntan a un crecimiento equilibrado y sostenible requiere una evaluación de un período más amplio. Este es el objetivo de la segunda parte del informe.

\section{Informe de evolución 1990-2015}
A partir de los gráficos del documento proporcionado evaluaremos la economía española en el periodo 1990-2015. Como hemos dicho, se trata de determinar si la economía Boliviana crecerá económicamente durante ese período de forma equilibrada y sostenible.

    \subsection{Crecimiento}
    En el gráfico 1.1  En términos de población, Bolivia está ganando peso sobre el conjunto mundial. A principios de los 90 la población española suponía cerca del 0,13 \% de la población mundial, y para 2015 se tuvo el 0,14.5 \%, es decir, la tendencia es levemente positiva.\\
    La tendencia demográfica estuvo acompañada de una tendencia cíclica desde el punto de vista del peso del PIB. El período 1990-2004 experimentó una casi-estabilidad de peso relativa del 0.02 \% del PIB. Para luego tener un realce significativo de 0.25 puntos porcentuales para el año 2015. La estabilidad de las exportaciones tuvieron lugar en el intervalo de 1990 y 2004 con aproximadamente 0.02 \% y después se tuvo el boom de los precios de las materias primas para los años 2004 al 2015 con un realce en las exportaciones de 0.4 puntos porcentuales del PiB mundial.\\
    Es interesante notar cierta correlación entre el PIB y las exportaciones esto cuando el peso relativo de las exportaciones es superior al peso relativo del PIB (período 2004-2015). Para estos años se tuvo ingresos externos para incentivar una mayor producción interna.  Para los años a priori (1990-2004) se tuvo una estabilidad constante de las exportaciones y el PIB.\\

    El Gráfico 1.2 muestra que el PIB per cápita constante al año 2010 en Bolivia creció constantemente entre los años 1990 y 2008. Con un aumento significativo desde los 23000 dolares hasta los casi 33000 dolares.  Si estaba por debajo de los \$ 25.000 constantes a principios de la década de 1990, estaba aumentando constantemente hasta un máximo de \$ 32.303 por persona en 2008. Prácticamente, el ingreso per cápita se multiplicó por 1,4 entre 1990 y 2008. El gráfico muestra que una brecha constante de ocho mil dólares se mantiene con respecto a los países de ingresos altos, la diferencia relativa se ha reducido en 3,78 puntos (en números índice). Esta tendencia estalló en 2008 como resultado de la recesión económica mundial que comenzó ese año. En el caso de España, afecta con más intensidad y provoca un descenso del PIB per cápita. En 2013 se tiene un PIB per cápita de \$ 30.532, muy similar a los de doce años antes.\\
    Una explicación de esta trayectoria de la economía española se resume en el Gráfico 1.3. Este gráfico muestra cómo ha crecido el PIB per cápita en comparación con la evolución de la productividad y el empleo per cápita, las dos variables que ayudan a identificar qué estrategia de crecimiento sigue un país. Durante este periodo hemos observado que, salvo algunos años, la estrategia de crecimiento de España siempre ha estado claramente basada en aumentar la producción aumentando el empleo. Entre 1995 y 2008, la productividad apenas aumentó 12. El crecimiento del PIB per cápita observado más recientemente sigue basándose en esta estrategia. Por tanto, en términos de crecimiento podemos afirmar que no estamos siguiendo la estrategia más adecuada.\\

    \subsection{Especialización sectorial y productividad}
    La evolución española de la estructura sectorial de la producción muestra una clara tendencia a acercarse al perfil de los países desarrollados. El peso del sector servicios sigue aumentando, tanto en la producción como en el empleo, en ambos casos ya por encima del 70 por ciento del total del sector. El sector industrial empieza a situarse en torno al 20 por ciento, perdiendo peso relativo principalmente en términos de empleo. En el caso de la agricultura, el valor agregado parece mantenerse en torno al valor de 2,5 por ciento, pero el empleo mantiene una tendencia a la baja.\\
    Los cambios anteriores están íntimamente relacionados con la evolución de la productividad. Si bien la agricultura reduce su fuerza de trabajo, el gran aumento de la productividad (duplicada en el período 1994-2016) le permite mantener su contribución a la producción total. En el caso de los servicios, la productividad no experimentó crecimiento durante el período de análisis. Como sabemos, este es un sector que, por su propia naturaleza, muestra dificultades para mejorar su productividad. En la mayoría de las ramas del sector servicios, aumentar la producción requiere aumentar el número de empleados (hostelería, restauración, educación, sanidad, etc.).\\
    Respecto a la industria, la evolución de la productividad muestra un claro cambio de tendencia. Entre 1994 y 2008, la productividad industrial disminuyó levemente. Como decíamos antes, sucedió que fue un periodo en el que España recurrió al aumento del empleo para aumentar la producción, sin que se produjeran en promedio aumentos de productividad. A partir del año 2008 observamos un incremento muy significativo en la productividad. Sin embargo, debemos ser escépticos sobre los fundamentos reales de tal fenómeno. Existen otros factores que pueden estar provocando un aumento ficticio de la productividad, como son: el despido de los trabajadores menos productivos, la reducción de los costes laborales y los efectos de la evolución del tipo de cambio euro-dólar. Durante este período se produce una devaluación del euro que provoca un aumento de la producción calculada en dólares. Se necesita un análisis más profundo de estos factores para concluir con mayor precisión la evolución de la productividad.\\

    \subsection{Comercio exterior y productividad}
    Durante el período 1990-2016 el coeficiente de apertura español mostró una tendencia mayoritariamente creciente. A principios de la década de 1990, el volumen de exportaciones e importaciones representaba menos del 40 por ciento del PIB y, al final del período, la cifra superaba el 60 por ciento. En 2009 se produjo una fuerte caída, de casi diez puntos porcentuales. Aún así, el coeficiente fue del 46,5 por ciento, por encima de los valores iniciales del período.\\
    La evolución de la tasa de cobertura es muy diferente, ya la vez muy ilustrativa de las peculiaridades del sector exterior español. Los ciclos fuertemente deficitarios (tasa de cobertura en torno al 80 por ciento) se combinan con ciclos de balance moderados. Se destacan los más destacados en los últimos años 2012-2016, logrando superávits de 10 puntos porcentuales. Parece que el crecimiento económico español depende en gran medida de las exportaciones, pero una vez que la tasa de variación del PIB es alta, la propensión a importar crece en mayor proporción.\\
    Gran parte de este efecto cíclico está relacionado con la fabricación, ya que suele representar entre el setenta y el ochenta por ciento de los ingresos. Observamos que durante los primeros diez años del período las exportaciones de manufacturas aumentaron su peso relativo, acercándose al ochenta por ciento del total. Sin embargo, desde 2001 se ha iniciado la pérdida relativa de peso, cayendo al 67,79 por ciento en 2012.\\
    ara el crecimiento económico, las exportaciones de bienes de alta tecnología juegan un papel relevante. He aquí un aspecto que debería mejorar la economía española. Su volumen de exportaciones de esta clase está entre el 0,72 por ciento y el 1,26 por ciento de las exportaciones totales de alta tecnología del grupo de países de ingresos altos. Este valor es inferior al peso de la economía española en el PIB total (entre el 2,5 y el 3,5 por ciento) y en las exportaciones totales (entre el 2 y el 3 por ciento) tomando como referencia los países de renta alta.\\

    \subsection{Sustentabilidad}
    La evolución del ahorro y la inversión permite valorar si el modelo de crecimiento de la economía española estuvo acompañado durante este periodo de criterios de sostenibilidad. Podemos observar en el Gráfico 4.1 que durante el periodo de análisis la economía española 12 atravesó etapas de capacidad de financiación externa y etapas de necesidades de financiación, aunque predominaron estas últimas, es decir, los años en los que la tasa de ahorro fue menor que la de inversión. . Básicamente, podemos ver que el crecimiento económico del PIB experimentado en los años 1995-2007 fue financiado en términos netos por financiamiento externo. Con una tasa de financiación equilibrada en 1996-1997, la tasa de inversión creció más rápido que la tasa de ahorro, lo que resultó en una tasa de financiación negativa de casi seis puntos porcentuales en 2006.\\
    Una tasa de financiación negativa no es necesariamente indicativa de una mala estrategia de crecimiento. Es obvio que si un país quiere crecer necesita financiamiento e inversión extranjera que le permita innovar su proceso productivo para ser más competitivo en los mercados externos. La condición para la sostenibilidad financiera es que en los próximos años aumente el PIB, aumenten los ingresos y la economía comience a reducir sus déficits comerciales y pagar los préstamos. En el caso de la economía española podemos decir que durante ese periodo el PIB y la renta aumentaron. Sin embargo, no había indicaciones claras de que se cumplieran otras condiciones. Ya sabemos que hasta 2007 el déficit comercial iba en aumento. Observamos que durante este período el esfuerzo en innovación, promediado por el gasto en I+D sobre el PIB (Gráfico 4.2), fue aumentando del 0,78 por ciento en 1996 al 1,35 por ciento en 2010. Sin embargo, aún existe una brecha importante con respecto al promedio de los países de altos ingresos, que en ese período aumentó de 2,14 por ciento a 2,37 por ciento. Obviamente, gran parte del gasto en I+D no se hará efectivo hasta más adelante, pero la disminución del esfuerzo de investigación entre 2010 y 2014 sugiere que la brecha se ampliará.\\
    Los indicadores seleccionados en el Gráfico 4.3 de sostenibilidad ambiental también muestran que el modelo de crecimiento español fue más contaminante y más consumidor de energía en comparación con los países de renta alta. Durante los años 1990-2007 hubo una clara tendencia a emitir relativamente más toneladas de CO2 al medio ambiente y un mayor consumo de energía per cápita. En ambos se experimentó una variación de casi 20 puntos porcentuales. Pensamos que contaminar y consumir más energía con otros países acaba dificultando competir a largo plazo en tecnologías o patrones de consumo más respetuosos con el medio ambiente.\\

    \subsection{Equidad}
    Podemos afirmar que, en general, durante el período 1990-2016, el consumo público aumentó a un ritmo mayor que el PIB (Gráfico 5.1), representando el primer indicador casi cinco puntos porcentuales más que el segundo. A pesar de la tendencia a la baja de los últimos años, el porcentaje de consumo público se ha mantenido por encima del inicio del período. Este mayor consumo público estuvo acompañado de un aumento tanto del gasto público en educación como en salud. Entre las dos partidas de gasto hay un incremento de un punto porcentual sobre el PIB para el período 2004-2014. Sin embargo, este aumento no tiene efectos similares en la distribución del ingreso (Gráfico 5.2). Todo lo contrario. En el período 2004-2016 hubo aumentos en el índice de Gini, de 31,8 a 36,2, provocando una mayor disparidad de casi cuatro puntos entre el decil superior y el decil inferior.\\
    La falta de información para todo el período nos obliga a estar muy atentos a la evolución del patrimonio. Para una visión más completa sería necesario realizar un análisis paralelo con datos del INE o con datos de Eurostat. Podemos comentar aquí que estas fuentes confirman que la tendencia en los últimos años ha sido hacia una mayor concentración del ingreso, a pesar de la reducción de las desigualdades operada en el período 1995-2004. Es posible que la intervención del Estado no esté contrarrestando esta mayor desigualdad porque todavía existe una brecha de alrededor de cinco puntos porcentuales entre la Unión Europea y España en términos de gasto público sobre PIB.

    \section{Conclusiones}	
    La economía española mantuvo una favorable senda de crecimiento entre el período 1990-2007, especialmente a partir de 1995, pero entró en crisis a partir de 2008. Este crecimiento ha contribuido a que España consolide ciertas características de los países desarrollados, pero también aclare sus debilidades. Este informe muestra que hay una serie de indicadores que parecen más apropiados para evaluar si el proceso de crecimiento del PIB o PIB per cápita ha sido equilibrado y, sobre todo, sostenible en el largo plazo. Estos indicadores incluyen lo siguiente:

    \begin{itemize}
	\item Productividad 
	\item Empleo per cápita 
	\item VEB Agrícola  
	\item Empleo agrícola 
	\item Productividad sectorial 
	\item Exportaciones de bienes y servicios  
	\item Exportaciones de alimentos 
	\item Exportaciones de Manufacturas de Alta Tecnología. 
	\item Tasa de ahorro  
	\item Tasa de inversión  
	\item Inversión en I+D  
	\item Gasto público sobre PIB. 
	\item Índice de Gini o Indicador de Desigualdad
    \end{itemize}

    En algunos de estos indicadores, la economía española ha convergido considerablemente a la media de los países desarrollados desarrollados de referencia. En otros indicadores persisten divergencias que ayudan a comprender las dificultades del modelo de crecimiento económico español para competir eficientemente en el contexto de la globalización sin perder de vista los objetivos sociales. Por ambas razones, la experiencia española puede servir de referencia para otros países más alejados de los niveles de desarrollo económico y social de la mayoría de los países de renta alta que consideramos economías desarrolladas.





