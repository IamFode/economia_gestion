
\section{Indicadores económicos}

    \subsection{Situación económica reciente}
    Mozambique es un país de 28 millones de habitantes con una renta per cápita de 528,31 dólares en 2015. Está dentro del grupo de países de renta baja, y por debajo de la media. Cabe señalar que en el período 1990-2015 el PIB creció a una tasa de 7,6 por ciento y la población a 3,04 por ciento. Como resultado, el PIB per cápita aumentó a una tasa promedio acumulada de 4,43 por ciento. Ha ganado así un peso relativo a nivel mundial y especialmente dentro del grupo de países de bajos ingresos.\\
    El desglose del PIB per cápita en productividad y empleo per cápita revela que es un país con baja productividad en comparación con los países del grupo de ingresos (80,27 por ciento). Hay una alta utilización del empleo, 59,01 por ciento de la población total, aunque es baja en comparación con su grupo de referencia (68,28 por ciento).\\
    Este es un país con un alto peso del sector primario. Esto representa el 22,94 por ciento de la producción y el 73,56 por ciento del empleo. Llama la atención la desproporción en el sector industrial entre el peso del valor agregado (19,65 por ciento) y el empleo (4,28 por ciento). Parte de esta actividad industrial está relacionada con la explotación de minerales (carbón, sal, bauxita, oro, grafito, piedras preciosas, gas natural y mármol). Gran parte de las exportaciones de productos básicos son minerales (38,46 por ciento) y energía (30,38 por ciento). Dado que es un país costero, el turismo proporciona una gran cantidad de ingresos para los turistas que visitan sus playas de clima tropical en el Océano Índico.\\
    Sectorialmente, destaca también la productividad en el sector industrial. Es 50 puntos más alto con el grupo de bajos ingresos. En cambio, en el sector agropecuario la productividad es 35 puntos superior y en el sector servicios es el promedio de los países de bajos ingresos.\\

    \subsection{Evolución económica}
    Durante el período 1990-95 Mozambique aumentó su peso en términos de población mundial: pasó del 0,25 por ciento al 0,38 por ciento. Aunque el peso es menor en términos de PIB y participación de las exportaciones (menos del 0,03 por ciento), también experimentó una ligera tendencia al alza.\\
    Según datos del Banco Mundial, el aumento del PIB se debió principalmente a un aumento de la productividad. También se ha basado en el crecimiento del empleo per cápita, pero esto se considera una cifra negativa, ya que la mano de obra se utiliza de forma intensiva.\\
    La evolución muestra cómo el sector agrícola ha perdido peso relativo. A principios de la década de 1990, la agricultura representaba más del 30 por ciento del PIB total, cuando hoy roza el veinte por ciento. No hay una tendencia clara entre los otros dos sectores. La pérdida de peso de la agricultura es ostensible en lo que a empleo se refiere. A principios de la década de 1990 se situó en el 85,4 por ciento, casi doce puntos por encima del valor más reciente. Ese empleo se concentra en el sector servicios, que duplicó su peso del 11,7 por ciento en 1990 al 22,2 por ciento en 2015.\\
    En concordancia con lo anterior, en los tres sectores hubo un aumento de la productividad. No hay aumento del empleo en la industria porque es el sector con mayor tasa de crecimiento de la productividad. Aumentó de alrededor de \$ 2,000 a \$ 6,000. También se destacó el aumento de la productividad en el sector servicios, de \$ 2.000 a \$ 3.800. La agricultura ha duplicado su productividad, pero sigue siendo baja (en el período pasó de \$ 200 a \$ 400).

    \subsection{Sustentabilidad}
    Durante el período de estudio, Mozambique estaba abriendo su economía al exterior. En la década de 1990, el coeficiente de apertura estaba entre 40 y 60 puntos. En los últimos diez años los valores han crecido hasta los 120 puntos, por lo que el comercio exterior supera al tamaño de la propia economía. Es una economía más exportadora que importadora. La tasa de cobertura siempre ha estado en valores por debajo de los 60 puntos, y más recientemente en los 40 puntos. Este superávit se debe principalmente a las exportaciones de minerales (entre 40 y 60 por ciento) y energía (entre 15 y 30 por ciento). Las cifras de exportación muestran fuertes fluctuaciones, posiblemente debido a las fluctuaciones de los precios. Hubo una fuerte caída en las exportaciones de alimentos, que en la primera parte del período llegó a más del 70 por ciento. Las exportaciones de productos de alta tecnología son insignificantes.\\
    l crecimiento económico anterior ha estado acompañado de una tasa de inversión siempre por encima del 20 por ciento, y en los últimos años muy alta, casi cercana al 60 por ciento. Sin embargo, sorprende que en este país la tasa de ahorro sea prácticamente nula. Se volvió incluso negativo en -64 por ciento en 1990.\\
    Apenas hay gasto en I+D y las cifras de emisiones de CO2 y consumo energético son bastante bajas. Sin embargo, ha habido un fuerte aumento en el consumo de energía en los últimos años.\\
    El consumo público representó entre el 13 y el 25 por ciento del PIB. En el período se destaca la pérdida de peso relativo del gasto militar y el aumento del gasto en educación. La errática evolución del gasto público puede estar explicando que no haya cambios en la distribución del ingreso. El índice de Gini ronda el valor de 50, de modo que el grupo de población más rico concentra más del 40 por ciento de los ingresos, frente al grupo más pobre que se encuentra por debajo del 2 por ciento.\\
    En resumen, a pesar de seguir siendo una economía de bajos ingresos, Mozambique ha crecido económicamente.

\section{Objetivos del milenio}
En cuanto al objetivo 1, erradicación de la pobreza y el hambre, se observan mejoras. El porcentaje de la población que vive con menos de \$ 1,90 al día se ha reducido del 50 por ciento en 1996 al 30 por ciento en 2014 (Meta 1.1). También hay una reducción significativa en la línea de pobreza nacional (Objetivo 1.2). Es difícil interpretar el Objetivo 1.3 con la participación indicada del ingreso del grupo más pobre en el ingreso total. Tampoco hay una tendencia clara. El porcentaje fluctuó sin una tendencia clara entre el 4 y el 5,4 por ciento. Lo más probable es que hubo una mejora, dado que en cuanto a la Meta 1.8, la existencia de bajo peso al nacer en los niños, la proporción bajó del 26 por ciento al 15 por ciento.\\
En cuanto a la educación, hay una mayor tasa de matrícula en la educación primaria (Objetivo 2.1). De valores iniciales del 40 por ciento, en años más recientes ya se ha acercado a valores del 90 por ciento. La tasa de alumnos que terminan la escuela primaria sigue siendo baja (alrededor del 50\%) (Objetivo 2.2), aunque observa un aumento de la tasa de alfabetización los jueves entre los 15 y los 24 años (Objetivo 2.3).\\
En cuanto a la igualdad de género, la proporción de mujeres en la escuela primaria ha aumentado pero sigue siendo baja (7,3 de cada 10 hombres) (Objetivo 3.1), al igual que la proporción de mujeres en el parlamento (1 de cada 10 hombres) (Objetivo 3.3) ,. No hay información sobre el indicador de mujeres que trabajan fuera del sector agrícola (Objetivo 3.2).\\
El progreso más claro se está logrando en el Objetivo 4 de reducir la mortalidad. Tanto la mortalidad infantil como la de menores de 5 años se han reducido notablemente. La primera de 250 a 75 muertes por cada mil nacimientos (Meta 4.1); y el segundo de 160 a 60 por mil nacimientos (Objetivo 4.2). El aumento de la inmunización contra el sarampión está contribuyendo a esta reducción de la mortalidad (Objetivo 4.3).\\
Hay muy poca información sobre el objetivo 5, mejorar la salud materna. Se destaca la tendencia a la baja de la tasa de mortalidad materna (Objetivo 5.1) y de la tasa de fecundidad adolescente (Objetivo 5.4). La primera tasa se ha reducido de 1.300 muertes por cada 100.000 nacimientos a 500. La segunda, con valores iniciales de más de 180 nacimientos por cada 1.000 mujeres de 15 a 19 años, ha bajado a 140 por mil. Sin embargo, sigue siendo alto. Un gran número de partos (más del 60 por ciento) continúan realizándose sin la atención de un médico por médico especialista (Objetivo 5.2).\\
El objetivo relacionado con la enfermedad (Objetivo 6) muestra una evolución muy dispar, que se combina con la falta de información. En el lado positivo, hay más personas VIH positivas tratadas con medicamentos antirretrovirales (casi el 40 por ciento), con un crecimiento casi exponencial (Meta 6.5). En el lado negativo, no hay estabilidad en la disminución de las tasas de mortalidad por tuberculosis (Objetivo 6.9) o en el número de personas afectadas por la malaria (Objetivo 6.6).\\
Con respecto al medio ambiente (Objetivo 7), los indicadores muestran tendencias positivas y negativas. En el lado positivo, las emisiones de CO2 por unidad de PIB se redujeron de 0,40 a 0,20 kg por dólar de PIB en paridad de poder adquisitivo (Objetivo 7.2) y la población con acceso a saneamiento aumentó de 35 por ciento a 51 por ciento (Objetivo 7.8). Sin embargo, la proporción de la población que vive en barrios marginales ha aumentado del 77 al 80 por ciento.\\
Finalmente, con respecto a la meta, 8 Mozambique es un país que recibe casi \$ 100 per cápita en Ayuda Oficial al Desarrollo (Objetivo 8.5). El servicio de la deuda externa cayó del 35 \% en 1995 al 10 \% en 2015, alcanzando el 5 \% durante varios años (Objetivo 8.12). En cuanto a las comunicaciones, se ha incrementado significativamente el número de teléfonos móviles (el 60 \% de la población tiene teléfono móvil) (Meta 8.15), aunque la población con acceso a internet se ha reducido al 15\% (Meta 8.16). Apenas ha habido mejoras en la disponibilidad de telefonía fija, que no llega al uno por ciento de toda la población.\\

\section{Objetivos de desarrollo sostenible}

\section{Resumen}
El crecimiento económico observado por Mozambique durante el período que se examina estuvo acompañado de cambios positivos en algunos indicadores del milenio, pero no en todos. En resumen, pareció tener más efectos sobre la mayoría de las variables relacionadas con la salud. También se observan mejoras en las variables educativas. Sin embargo, no se observan efectos positivos sobre la distribución del ingreso. En los demás grupos de indicadores no hay una tendencia clara.





