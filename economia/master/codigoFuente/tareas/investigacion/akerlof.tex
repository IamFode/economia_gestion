\begin{center}
    \large
    \textbf{EL MERCADO DE LIMONES :\\
    LA INCERTIDUMBRE DE LA CALIDAD Y EL \\
    MECANISMO DEL MERCADO}\\
    \vspace{.5cm}
    \normalsize
    \textbf{George A. Akerlof}
\end{center}

\vspace{2cm}

\begin{enumerate}[\bfseries 1.]

    %1.
    \item \textbf{Qué te ha llamado más la atención del contenido de este post?.}\\\\

	Me llamaron la atención dos puntos. El primero: Como una teoría macroeconómica. Es decir, la teoría del crecimiento y su aplicación a la educación, podrían llevar a Akerlof a escribir sobre un tema microeconómico, para luego ganar el premio Nobel. Trato de decir que podemos iniciar una investigación económica de varias perspectivas. \\
	 Akerlof estaba claro de cada argumento que escribió. Por lo que el segundo punto que me llamo la atención, es la constancia y perseverancia que tuvo para que alguna revista pueda publicar su hallazgo. \\\\

    %2.
    \item \textbf{Lee el artículo “The Market for Lemons” (QJE, 1970) y resume, con tus propias palabras las siguientes partes: la tesis (pregunta), métodos (el cómo testar la pregunta empírica o resolverla) y los resultados más relevantes.}

\end{enumerate}
