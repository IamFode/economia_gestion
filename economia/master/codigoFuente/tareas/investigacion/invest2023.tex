\chapter{Nueva investigación}

\section*{Desarrollo de la pregunta de investigación}
\begin{enumerate}[1.]
    \item ¿Cuál es el tema general de investigación?
    \item ¿Cuál es la pregunta de investigación?
    \item ¿Cuál es la hipótesis de investigación?
\end{enumerate}

Empezamos a revisar la bibliografía, observando que preguntas pendientes existen o que brechas (baches) existen en la literatura. Para ello, necesito realizar búsquedas finas leyendo la introducción y las conclusiones de cada articulo de investigación.\\


\section*{Organización del trabajo final y su redacción}

\begin{enumerate}[1.]
    \item Introducción
    \item Revisión de la literatura
    \item Marco teórico .- Al realizar el marco teórico se debe conceptualizar.
    \item Análisis empírico .- Modelización econométrica, estadística, etc.
    \item Conclusiones .- Lista de que se hizo, que no se hizo, que se puede hacer en el futuro, cual resultado fue mejor, que resultados obtuvimos, las debilidades del estudio, etc.
    \item Apéndices (si las hay)
    \item Referencias
\end{enumerate}

\section*{Qué podemos hacer a la hora de investigación}

\begin{itemize}
    \item Se podría proponer una nueva idea.
    \item Una nueva metodología.
    \item Una nueva hipótesis.
    \item Proponer políticas nuevas con datos antiguos.
    \item Se pueden predecir el futuro con el pasado.
\end{itemize}


\section*{TIPS de escritura}

Los buenos escritos de economía tomarán como punto de partida el supuesto de un comportamiento racional. Un análisis exhaustivo de cualquier comportamiento y una descripción bien escrita del mismo deben tener en cuenta los efectos de incentivo.
