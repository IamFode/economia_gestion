\documentclass[1p,preprint]{elsarticle}


\usepackage[figuresright]{rotating}
\usepackage{hyperref}
\hypersetup{
    colorlinks=true,    % Activa los enlaces en color
    linkcolor=blue,     % Color de los enlaces internos (por ejemplo, índice)
    citecolor=blue,     % Color de los enlaces de citas
    urlcolor=blue       % Color de los enlaces URL
}
\usepackage[utf8]{inputenc}

\biboptions{authoryear}

\begin{document}

\begin{frontmatter}

\title{Análisis del desempeño histórico de las carteras recomendadas por varios bancos de inversión en función del perfil de riesgo de los clientes\tnoteref{label1}}
\tnotetext[label1]{Esta investigación es parte de la asignatura de técnicas de investigación del máster en economía de las universidad de Vigo, Santiago de Compostela y La Coruña.}

\author{Juan Fidel Álvarez Escontrela \fnref{fn1}}
\ead{fidel220492@gmail.com}
\affiliation[1]{organization={Universidad de Santiago de Compostela},
addressline={Campus Norte, Av. do Burgo das Nacións, s/n},
postcode={15782},
city={Santiago de Compostela},
country={España}}
\fntext[fn1]{Alumno del máster en economía de la universidad Santiago de Compostela}

\author{Christian L. Paredes Aguilera.\fnref{fn1}}
\ead{soyfode@gmail.com}
\affiliation[2]{organization={Universidad de Vigo},
addressline={Rúa as Pedreiras, 2},
postcode={36310},
city={Vigo},
country={España}}



\begin{abstract}
    Este estudio se centra en analizar el desempeño histórico de las carteras recomendadas por los principales bancos de inversión a nivel internacional, considerando los perfiles de riesgo de sus clientes y la estrategia de inversión utilizada (activa o pasiva). El objetivo es determinar si las carteras recomendadas han sido congruentes con los perfiles de riesgo y si han logrado un rendimiento acorde. Para ello, se utilizarán datos de rendimiento y riesgo de índices de acciones y bonos, así como la composición de las carteras recomendadas por los bancos.

El análisis se llevará a cabo a lo largo de períodos de 3, 5 y 10 años, evaluando si las carteras recomendadas se encuentran en la frontera eficiente y si el riesgo asumido se ajusta a los rendimientos obtenidos. Además, se compararán las carteras recomendadas con carteras de activos equivalentes, como fondos mutuales y ETFs, para determinar cuál estrategia resulta más beneficiosa según el perfil de riesgo y el horizonte de inversión de los clientes.

Se anticipa que este estudio arrojará luz sobre la efectividad de las carteras recomendadas por los bancos de inversión y si las estrategias activas o pasivas han tenido un mejor desempeño histórico. Los resultados proporcionarán información valiosa para los inversores y las instituciones financieras en la toma de decisiones de inversión.
\end{abstract}

\begin{keyword}
    Finanzas \sep Cartera \sep Bancos \sep Riesgo.
\end{keyword}

\newpageafter{abstract}

\end{frontmatter}



\section{Objetivo(s)}
Hallar las proporciones representadas por diversos instrumentos financieros en las
carteras recomendadas por los principales bancos de inversión a nivel internacional a sus clientes (en
función de su perfil de riesgo), y determinar cuál ha sido el desempeño de esos portafolios en los
últimos 3, 5 y 10 años. Además, se analizará si el desempeño de las carteras muestra divergencias
considerando el tipo de estrategia seguida (activa o pasiva).

\section{Problema o problema central} 
¿Cuál ha sido el desempeño histórico de las carteras recomendadas
por los principales bancos de inversión, ajustado por criterios de riesgo y tipo de estrategia (activa o
pasiva)? ¿Ha sido este desempeño congruente con el estilo y riesgo asumidos?

\section{Bases teóricas} 
\cite{fama1970}, establece una teoría en la que describe que el precio actual de
un activo refleja toda la información disponible que exista acerca del mismo, sea la información:
histórica, pública y/o privada. En base a ello, y considerando la eficiencia de los mercados
desarrollados, los instrumentos de inversión pasiva han mostrado un crecimiento en detrimento de
otros instrumentos activamente gerenciados, tales como fondos mutuales, dado el mejor desempeño
histórico que ha exhibido las estrategias pasivas.

Así, por ejemplo, \cite{Chan1999} consiguen que solo una baja proporción de los
fondos activamente gerenciados logra superar, de forma persistente, al desempeño de los índices de
referencia pasivos del mercado. \cite{Pastor2020} hallan que durante la crisis de la COVID-19 el
flujo de fondos hacia vehículos de inversión activa ha disminuido, aunado a un desempeño inferior en
relación a los índices de referencia y fondos de inversión pasiva. Todo ello va de la mano con la teoría
moderna de portafolios, desarrollada por \cite{Markowitz1952}, la cual establece un modelo que consiste
en ubicar una cartera de inversión óptima, ajustada por el riesgo, mediante la elección de los activos
adecuados que compongan dicha cartera.

\section{Hipótesis} 
La inversión ha sido una práctica que con los años se ha vuelto cada vez más popular. En
sus comienzos era solo aplicada en países muy específicos, pero se ha globalizado y hoy en día es
conocida y desarrollada en una gran variedad de países. 

En el pasado existían muchas limitantes a la hora de invertir en el mercado de valores. Era necesario
tener acceso a una cuenta de jubilación proporcionada por un empleador de una compañía, o poseer
los fondos y el conocimiento suficiente para poder abrir una cuenta de inversión no patrocinada por
un empleador.

Con el pasar de los años estas limitantes se han ido mitigando y el tener acceso a una cuenta de
inversión se ha vuelto cada vez más sencillo. Existen múltiples herramientas financieras que le
permiten a los inversionistas adquirir participación en una empresa sin hacer uso de una gran cantidad
de fondos. Además, existen muchas compañías que permiten abrir cuentas de inversión a nivel
mundial desde plataformas virtuales, múltiples cursos de inversión e información de fácil acceso a
través del Internet.

Sin embargo, a pesar de la facilidad que existe para adquirir información hoy en día, muchos de los
nuevos inversionistas no cuentan con los conocimientos necesarios para poder manejar sus
inversiones de la forma más eficiente posible o guiarlas de acuerdo con sus necesidades. Lo anterior
se debe a que, en su mayoría, estas personas desconocen cuáles son sus metas financieras y sus
limitantes a la hora de invertir, por lo cual no utilizan estrategias de inversión que vayan de la mano
con sus expectativas.

Por esto último, la industria financiera, mediante bancos de inversión, ofrece asesorías de inversión
hacia clientes, especialmente con ciertos requerimientos mínimos de patrimonio. De forma usual,
ante la mayor demanda de este tipo de servicios, muchos bancos optan por ofrecer portafolios
estandarizados bajo ciertos perfiles de riesgo.

En base a lo anterior, las instituciones financieras recomiendan sus carteras a los inversores que
encajen en cada perfil ofrecido. Si bien esta práctica responde a una necesidad, podría estar dejando
de lado ciertas oportunidades de mejora. Esta investigación se enfoca en indagar sobre este tópico y
poder así determinar si los portafolios ofrecidos a los inversionistas se ajustan a los perfiles de riesgo
establecidos, y si éstos pueden ser considerados óptimos, o presentan un espacio de mejora en el
desempeño. De igual forma, se busca establecer si la estrategia ofrecida resulta ser la más beneficiosa
por perfil y horizonte de tiempo,

\section{Método o plan de investigación} 
Se utilizará la base de datos de Bloomberg para obtener el
rendimiento y el riesgo de los índices de precios de acciones y bonos (por ejemplo, S\&P 500 e índices
de bonos de Barclays), así como también las composiciones de los portafolios estandarizados ofrecidos
por los principales bancos de inversión a nivel mundial (tales como Blackrock, Vanguard, Merril Lynch
y Charles Schwab) a sus clientes en función del perfil de riesgo.

Se hallará la frontera eficiente para cada período de tiempo considerado varios supuestos (ventas en
corto no permitidas y existencia de una tasa de interés libre de riesgo), y utilizando activos
tradicionales (acciones y bonos).

A continuación, se examinará si los portafolios recomendados por los bancos de inversión se ubican
en la frontera eficiente (se analizará si el rendimiento de la cartera de la frontera eficiente que tenga
la misma desviación estándar que la cartera recomendada por el banco de inversión respectivo es 
estadísticamente igual -usando varios niveles de significación estadística, - similar al análisis
desarrollado por \cite{Garay2005}. Lo anterior se hará para períodos de referencia de 3,
5, 10 años (2012-2022), evaluando si el riesgo evidenciado por las decisiones de inversión se ajustó a
sus rendimientos, mediante el uso del ratio de Sharpe.

Para establecer una comparación aproximada del rendimiento de un portafolio recomendado, por
perfil de riesgo, compuesto por un solo tipo de activos con respecto a un mismo portafolio de un tipo
de activo similar, se considerará identificar el activo equivalente al recomendado, utilizando el índice
replicado como referencia. Esto es, si la recomendación incluye únicamente fondos mutuales, se
determinará que fondos cotizados (ETFs) son equivalentes a esos fondos mutuales, y viceversa, a
modo de determinar qué tipo de estrategias resulta más atractiva para cada inversor, de acuerdo con
su perfil de riesgo y horizonte de inversión.

\section{Conclusiones y resultados que pueden preverse} 
Se espera hallar una congruencia entre el nivel de
rendimiento exhibido por las carteras de los bancos de inversión y la correspondiente a las carteras
ubicadas en la frontera eficiente obtenida (con el mismo riesgo). Además, se espera que las carteras
de inversión activas recomendadas por los bancos exhiban un desempeño inferior al de las carteras
pasivas, ajustadas a su riesgo.


\bibliographystyle{elsarticle-num-names}
\bibliography{bibliography}


\end{document}

