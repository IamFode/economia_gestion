\documentclass{class/cas-sc}


\usepackage[figuresright]{rotating}
\usepackage{hyperref}
\hypersetup{
    colorlinks=true,    % Activa los enlaces en color
    linkcolor=blue,     % Color de los enlaces internos (por ejemplo, índice)
    citecolor=blue,     % Color de los enlaces de citas
    urlcolor=blue       % Color de los enlaces URL
}
\usepackage[utf8]{inputenc}

%\biboptions{authoryear}

\begin{document}

\begin{frontmatter}

\title{Análisis del desempeño histórico de las carteras recomendadas por varios bancos de inversión en función del perfil de riesgo de los clientes\tnoteref{label1}}
\tnotetext[label1]{Esta investigación es parte de la asignatura de técnicas de investigación del máster en economía de las universidad de Vigo, Santiago de Compostela y La Coruña.}

\author{Juan Fidel Álvarez Escontrela \fnref{fn1}}
\ead{fidel220492@gmail.com}
\affiliation[1]{organization={Universidad de Santiago de Compostela},
addressline={Campus Norte, Av. do Burgo das Nacións, s/n},
postcode={15782},
city={Santiago de Compostela},
country={España}}
\fntext[fn1]{Alumno del máster en economía de la universidad Santiago de Compostela}

\author{Christian L. Paredes Aguilera.\fnref{fn1}}
\ead{soyfode@gmail.com}
\affiliation[2]{organization={Universidad de Vigo},
addressline={Rúa as Pedreiras, 2},
postcode={36310},
city={Vigo},
country={España}}



\begin{abstract}
    Este estudio se centra en analizar el desempeño histórico de las carteras recomendadas por los principales bancos de inversión a nivel internacional, considerando los perfiles de riesgo de sus clientes y la estrategia de inversión utilizada (activa o pasiva). El objetivo es determinar si las carteras recomendadas han sido congruentes con los perfiles de riesgo y si han logrado un rendimiento acorde. Para ello, se utilizarán datos de rendimiento y riesgo de índices de acciones y bonos, así como la composición de las carteras recomendadas por los bancos.

El análisis se llevará a cabo a lo largo de períodos de 3, 5 y 10 años, evaluando si las carteras recomendadas se encuentran en la frontera eficiente y si el riesgo asumido se ajusta a los rendimientos obtenidos. Además, se compararán las carteras recomendadas con carteras de activos equivalentes, como fondos mutuales y ETFs, para determinar cuál estrategia resulta más beneficiosa según el perfil de riesgo y el horizonte de inversión de los clientes.

Se anticipa que este estudio arrojará luz sobre la efectividad de las carteras recomendadas por los bancos de inversión y si las estrategias activas o pasivas han tenido un mejor desempeño histórico. Los resultados proporcionarán información valiosa para los inversores y las instituciones financieras en la toma de decisiones de inversión.
\end{abstract}

\begin{keyword}
    Finanzas \sep Cartera \sep Bancos \sep Riesgo.
\end{keyword}

\newpageafter{abstract}

\end{frontmatter}





\bibliographystyle{bibstyle/cas-model2-names}
\bibliography{bibliography}


\end{document}
\end{comment}
