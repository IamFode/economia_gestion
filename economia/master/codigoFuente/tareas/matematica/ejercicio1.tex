\section*{\center Entrega 1}
\vspace*{1cm}

\begin{enumerate}

    \item[\bfseries Problema 1.] \textbf{\boldmath Calcule los autovalores y los autovectores de la matriz $A$ y halle $A^n$, con $n\in \mathbb{N}:$}\\

    $$A=\begin{pmatrix}
	0 & -1 & -1 \\
	-1 & -2 & -1 \\
	1 & 3 & 2 
    \end{pmatrix}$$\\\\

    \textbf{Respuesta.-}\;  Por definición, y la regla de Sarrus el polinomio característica de $A$ es:
    
    $$ \begin{array}{rcl} 
	\det \left[
    \begin{pmatrix}
	0 & -1 & -1 \\
	-1 & -2 & -1 \\
	1 & 3 & 2 
    \end{pmatrix} - \alpha 
    \begin{pmatrix}
	1 & 0 & 0 \\
	0 & 1 & 0 \\
	0 & 0 & 1 
\end{pmatrix} \right] & = & 
    \begin{vmatrix}
	-\alpha & -1 & -1 \\
	-1 & -2-\alpha & -1 \\
	1 & 3 & 2-\alpha 
	\end{vmatrix} \\\\ 
	&=&  
	\begin{vmatrix}
	    -\alpha & -1 & -1 \\
	    -1 & -2-\alpha & -1 \\
	    1 & 3 & 2-\alpha 
	\end{vmatrix} \hspace{.4cm} 
	\begin{matrix}
	     -\alpha & -1 \\
	     -1 & -2-\alpha \\
	     1 & 3 \\ 
	\end{matrix}\\\\
	&=&\left[-\alpha (-2-\alpha)(2-\alpha)\right] + \left[(-1)(-1)1\right] + \left[(-1)(-1)3\right]\\\\
	&-&\left[1(-2-\alpha)(-1)\right]-\left[3(-1)(-\alpha)\right]-\left[(2-\alpha)(-1)(-1)\right]\\\\
	&=&4\alpha-\alpha^3 + 1 + 3 - 2 - \alpha - 3\alpha - 2+\alpha\\\\
	&=& -\alpha(\alpha^2-1)\\
    \end{array}$$\\

    Luego igualamos el último resultado a $0$, de donde obtenemos,\\
    $$-\alpha(\alpha^2-1)=0$$ 
    así los autovalores estarán dados por,
    $$\alpha=0 \;\; \lor\;\; \alpha=1 \;\;\lor\;\; \alpha=-1$$\\

    Ahora calculemos los autovectores.

    \begin{itemize}

	\item Para $\alpha=0$
	    $$
	    \begin{pmatrix}
		0 & -1 & -1 \\
		-1 & -2-0 & -1 \\
		1 & 3 & 2-0
	    \end{pmatrix}  
	    \begin{pmatrix}
		v_1 \\
		v_2 \\
		v_3
	    \end{pmatrix} = 
	    \begin{pmatrix}
		0 & -1 & -1 \\
		-1 & -2 & -1 \\
		1 & 3 & 2
	    \end{pmatrix}  
	    \begin{pmatrix}
		v_1 \\
		v_2 \\
		v_3
	    \end{pmatrix} = 
	    \begin{pmatrix}
		0 \\
		0 \\
		0	
	    \end{pmatrix} 
	    $$

	    $$\left.\begin{array}{rcl}
		    0\cdot v_1 -1v_2 -1v_3&=&0\\
			    -1v_1-2v_2-1v_3&=&0\\
			    1v_1+3v_2+2v_3&=&0
		\end{array}\right\} \Longleftrightarrow v_2=-v_3 \; \land \; v_1=v_3 \;\land \; v_3=v_3  \Longleftrightarrow  
		\begin{pmatrix}
		    x_3 \\
		    -x_3 \\
		    x_3
		\end{pmatrix}=v_3
		\begin{pmatrix}
		    1 \\
		    -1 \\
		    1 
		\end{pmatrix}
		$$

		Sea $v_3=1$ entonces el autovector para $\alpha=0$ es
		$\begin{pmatrix}
		    1 \\
		    -1 \\
		    1 
		\end{pmatrix}$\\\\

	\item Para $\alpha=1$
	    $$
	    \begin{pmatrix}
		-1 & -1 & -1 \\
		-1 & -2-1 & -1 \\
		1 & 3 & 2-1
	    \end{pmatrix}  
	    \begin{pmatrix}
		v_1 \\
		v_2 \\
		v_3
	    \end{pmatrix} = 
	    \begin{pmatrix}
		-1 & -1 & -1 \\
		-1 & -3 & -1 \\
		1 & 3 & 1 
	    \end{pmatrix}  
	    \begin{pmatrix}
		v_1 \\
		v_2 \\
		v_3
	    \end{pmatrix} = 
	    \begin{pmatrix}
		0 \\
		0 \\
		0	
	    \end{pmatrix} 
	    $$

	    $$\left.\begin{array}{rcl}
		    -1 v_1 -1v_2 -1v_3&=&0\\
			    -1v_1-3v_2-1v_3&=&0\\
			    1v_1+3v_2+1v_3&=&0
		\end{array}\right\} \Longleftrightarrow v_2=0 \; \land \; v_1=-v_3 \;\land \; v_3=v_3  \Longleftrightarrow  
		\begin{pmatrix}
		    -x_3 \\
		    0 \\
		    x_3
		\end{pmatrix}=v_3
		\begin{pmatrix}
		    -1 \\
		    0 \\
		    1 
		\end{pmatrix}
		$$

		Sea $v_3=1$ entonces el autovector para $\alpha=1$ es
		$\begin{pmatrix}
		    -1 \\
		    0 \\
		    1 
		\end{pmatrix}$\\\\

	\item Para $\alpha=-1$
	    $$
	    \begin{pmatrix}
		1 & -1 & -1 \\
		-1 & -2+1 & -1 \\
		1 & 3 & 2+1
	    \end{pmatrix}  
	    \begin{pmatrix}
		v_1 \\
		v_2 \\
		v_3
	    \end{pmatrix} = 
	    \begin{pmatrix}
		1 & -1 & -1 \\
		-1 & -1 & -1 \\
		1 & 3 & 3 
	    \end{pmatrix}  
	    \begin{pmatrix}
		v_1 \\
		v_2 \\
		v_3
	    \end{pmatrix} = 
	    \begin{pmatrix}
		0 \\
		0 \\
		0	
	    \end{pmatrix} 
	    $$

	    $$\left.\begin{array}{rcl}
		     v_1 -v_2 -v_3&=&0\\
			    -v_1-v_2-v_3&=&0\\
			    v_1+3v_2+3v_3&=&0
		\end{array}\right\} \Longleftrightarrow v_2=-v_3 \; \land \; v_1=0 \;\land \; v_3=v_3  \Longleftrightarrow  
		\begin{pmatrix}
		    0 \\
		    -x_3 \\
		    x_3
		\end{pmatrix}=v_3
		\begin{pmatrix}
		    0 \\
		    -1 \\
		    1 
		\end{pmatrix}
		$$

		Sea $v_3=1$ entonces el autovector para $\alpha=-1$ es
		$\begin{pmatrix}
		    0 \\
		    -1 \\
		    1 
		\end{pmatrix}$\\\\
	
    \end{itemize}

    Por último hallemos $A^n$\\

    Sea $D=\mbox{diag}(0,1,-1)=
    \begin{pmatrix}
	0 & 0 & 0 \\
	0 & 1 & 0 \\
	0 & 0 & -1 
    \end{pmatrix}, \qquad 
    V = \begin{pmatrix}
	1 & -1 & 0 \\
	-1 & 0 & -1 \\
	1 & 1 & 1
    \end{pmatrix}
    $

    Necesitaremos también encontrar la inversa de $V$ por lo que se podrá calcular aplicando,

    $$V^{-1}=\dfrac{\left(\mbox{adj}(A)\right)^T}{\det(V)}\cdot $$

    Por lo que el resultado será:

    $$V^{-1} = \begin{pmatrix}
	1 & 1 & 1 \\
	0 & 1 & 1 \\
	-1 & -2 & -1 
	\end{pmatrix} $$

    Dado que $A^n = VD^n V^{-1}$, y aplicando la multiplicación de matrices obtendremos, 

    $$A^n = \begin{pmatrix}
	1 & -1 & 0 \\
	-1 & 0 & -1 \\
	1 & 1 & 1
    \end{pmatrix} \times \begin{pmatrix}
	0 & 0 & 0 \\
	0 & 1 & 0 \\
	0 & 0 & -1)^n 
    \end{pmatrix} \times \begin{pmatrix}
	1 & 1 & 1 \\
	0 & 1 & 1 \\
	-1 & -2 & -1 
	\end{pmatrix} = \begin{pmatrix}
	0 & -1 & -1 \\
	(-1)^n & 2(-1)^n & (-1)^n\\
	-(-1)^n & 1-2(-1)^n & -1(-1)^n
    \end{pmatrix}$$\\\\






    \item [\bfseries Problema 2.] \textbf{Resuelva los siguientes problemas de optimización:}\\\\
	Sea $f(x,y,z)=(x-z-1)^2+(y-x)^2 + z^2.$\\\\

	\begin{enumerate}[\bfseries a)]

	    %-------------------- a) --------------------
	    \item $\min f(x,y,z)$.\\\\
		\textbf{Respuesta.-}\; Vemos que el dominio de definición es $\mathbb{R}^3$, es abierto y $f$ es de clase $1$  en $\mathbb{R}^3$.\\\\

		Ahora calculamos las derivadas parciales de cada variable.\\

		$$\begin{array}{rcl}
		    \dfrac{\partial f}{\partial x} \left[(x-z-1)^2+(y-x)^2 + z^2 \right]&=&\dfrac{\partial f}{\partial x}\left(2x^2-2xz-2xy-2x+2z^2+2z+y^2+1\right)\\\\
											&=&4x-2z-2y-2=0\\\\
											&&\\
		    \dfrac{\partial f}{\partial y}\left(2x^2-2xz-2xy-2x+2z^2+2z+y^2+1\right)&=&2y-2x=0\\\\
											&&\\
		    \dfrac{\partial f}{\partial y}\left(2x^2-2xz-2xy-2x+2z^2+2z+y^2+1\right)&=&4z+2-2x=0\\\\
											    &&\\
		\end{array}$$

		de donde $$y=x,\qquad z=\dfrac{x-1}{2}$$
		Así
		$$x=1,\qquad y=1,\qquad z=0$$\\
		Por lo tanto los valores que mínimizan a la función son:
		$$(x,y,z)=(1,1,0)$$\\

		$f$ es de clase $2$ en $\mathbb{R}^3$ dado que sus derivadas parciales en segundo orden existen y son continuas en $\mathbb{R}^3$ (abierto y convexo).\\

		$$Hf(x,y,z) = \begin{pmatrix}
		    4&-2&-2\\
		     -2&2&0\\
		       -2&0&4
		\end{pmatrix}$$\\

		Sabemos que la función será convaca si y sólo si la hessiana de $f$ definida negativa y será convexa si será positiva.\\

		Calculando los autovalores de $Hf(x,y,z):$


		$$\det\begin{pmatrix}
		    4-\alpha&-2&-2\\
		     -2&2-\alpha&0\\
		       -2&0&4-\alpha
		\end{pmatrix} = -\alpha^3 + 10\alpha^2 - 24\alpha + 8=0$$\\

	    %-------------------- b) -------------------
	    \item $\min\lbrace f(x,y,z) \, : \, -x+y+z+4=0,\; x+y-z-4=0\rbrace$\\\\
		\textbf{Respuesta.-}\; $(x,y,z)=(2,0,-2)$

	    %-------------------- c) -------------------
	    \item $\min \lbrace f(x,y,z) \, : \, -x+y+z+4\leq 0,\; x+y-z-4\leq 0 \rbrace$\\\\
		\textbf{Respuesta.-}\; $(x,y,z) = (-1,-3,-2)$


	\end{enumerate}

\end{enumerate}




