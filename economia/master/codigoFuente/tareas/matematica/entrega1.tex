\section*{\center \large Primera Entrega}
\begin{center}
    \textbf{Christian Limbert Paredes Aguilera}.
\end{center}
\vspace{1cm}

\begin{enumerate}

    \item[\bfseries Problema 1.] \textbf{\boldmath Calcule los autovalores y autovectores de la matriz $A$. Cuando sea posible, diagonalice y obtenga una expresión en función de la diagonal para $A^n$.}

    $$A=\left(\begin{array}{*{3}{r}}
	1 & 0 & 0 \\
	2 & -1 & 2 \\
	0 & 0 & 1 
    \end{array}\right)$$\\

    \textbf{Respuesta.-}\;  Por definición, y la regla de Sarrus el polinomio característica de $A$ es:
    
    $$ \begin{array}{rcl} 
	\det \left[
    \left(\begin{array}{*{3}{r}}
	1 & 0 & 0 \\
	2 & -1 & 2 \\
	0 & 0 & 1 
    \end{array}\right) - \alpha 
    \left(\begin{array}{*{3}{r}}
	1 & 0 & 0 \\
	0 & 1 & 0 \\
	0 & 0 & 1
\end{array}\right)\right] & = & 
    \left|\begin{array}{*{3}{c}}
	1-\alpha & 0 & 0 \\
	2 & -1-\alpha & 2 \\
	0 & 0 & 1-\alpha
	\end{array}\right| \\\\ 
	&=&  
	\left|\begin{array}{*{3}{c}}
	1-\alpha & 0 & 0 \\
	2 & -1-\alpha & 2 \\
	0 & 0 & 1-\alpha
	\end{array}\right|
	\left.\begin{array}{*{3}{c}}
	1-\alpha & 0  \\
	2 & -1-\alpha  \\
	0 & 0
	\end{array}\right|\\\\
	&=&(1-\alpha)(-1-\alpha)(1-\alpha)+0\cdot 2 \cdot 0 + 0\cdot 2 \cdot 0\\\\
	&=&-(1+\alpha)\left(1-\alpha\right)^2.
    \end{array}$$

    Luego igualamos el último resultado a $0$, de donde obtenemos,\\
    $$-(1+\alpha)\left(1-\alpha\right)^2=0\quad \Rightarrow \quad (1+\alpha)\left(1-\alpha\right)^2=0.$$\\
    Así, los autovalores estarán dados por:
    \begin{tcolorbox}
	$$\alpha=1\; (\mbox{doble})  \;\;\lor\;\; \alpha=-1$$
    \end{tcolorbox}
    \vspace*{.2cm}

    Ahora calculemos los autovectores.

    \begin{itemize}

	\item Para $\alpha=1$
	    $$
	    \left(\begin{array}{*{3}{c}}
		1-1 & 0 & 0\\
		2 & -1-1 & 2 \\
		0 & 0 & 1-1
	    \end{array}\right) 
	    \left(\begin{array}{*{3}{c}}
		v_1 \\
		v_2 \\
		v_3
	    \end{array}\right) = 
	    \left(\begin{array}{*{3}{c}}
		0 & 0 & 0 \\
		2 & -2 & 2 \\
		0 & 0 & 0
	    \end{array}\right)
	    \left(\begin{array}{*{3}{c}}
		v_1 \\
		v_2 \\
		v_3
	    \end{array}\right) = 
	    \left(\begin{array}{*{3}{c}}
		0 \\
		0 \\
		0	
	    \end{array}\right) 
	    $$

	    Entonces, por el método de eliminación Gauss-Jordan

	    $$\left\{\begin{array}{rcl}
		v_1-v_2+v_3 &=& 0\\
		0v_1+0v_2+0v_3 &=& 0\\
		0v_1+0v_2+0v_3 &=& 0
	    \end{array}\right.
	    \quad \Rightarrow \quad
	    \left\{\begin{array}{rcl}
		v_1 &=& v_2-v_3\\
		v_2 &=& v_2\\
		v_3 &=& v_3
	    \end{array}\right.$$

	    De donde, el autovector asociado a $\alpha=1$ será:\\
	    \begin{tcolorbox}
	    $$v=(v_2-v_3,s,t) \quad s,t\in \textbf{F}.$$
	    \end{tcolorbox}

	    Por ejemplo, sea $s=1$ y $t=0$, entonces
	    \begin{tcolorbox}
		\begin{equation}
		    v=(1,0,0)
		\end{equation}
	    \end{tcolorbox}

	    Como existe la dualidad podemos tomar $s=0$ y $t=1$. Por lo tanto,
	    \begin{tcolorbox}
		\begin{equation}
		    v=(-1,0,1)
		\end{equation}
	    \end{tcolorbox}
	    \vspace{.6cm}

	\item Para $\alpha=-1$
	    $$
	    \left(\begin{array}{*{3}{c}}
		1+1 & 0 & 0\\
		2 & -1+1 & 2 \\
		0 & 0 & 1+1
	    \end{array}\right)  
	    \left(\begin{array}{c}
		v_1 \\
		v_2 \\
		v_3
	    \end{array}\right) = 
	    \left(\begin{array}{*{3}{c}}
		2 & 0 & 0 \\
		2 & 0 & 2 \\
		0 & 0 & 2
	    \end{array}\right)
	    \left(\begin{array}{c}
		v_1 \\
		v_2 \\
		v_3
	    \end{array}\right) = 
	    \left(\begin{array}{c}
		0 \\
		0 \\
		0	
	    \end{array}\right)
	    $$

	    Entonces, por el método de eliminación Gauss-Jordan

	    $$\begin{array}{ccccc}
		\left(\begin{array}{*{3}{r}}
		    2 & 0 & 0 \\
		    2 & 0 & 2 \\
		    0 & 0 & 2
		\end{array}\right)
		&
		\dfrac{1}{2}R_1\to R_1
		&
		\left(\begin{array}{*{3}{r}}
		    1 & 0 & 0 \\
		    2 & 0 & 2 \\
		    0 & 0 & 2
		\end{array}\right)
		&
		-2R_1+R_2\to R_2
		&
		\left(\begin{array}{*{3}{r}}
		    1 & 0 & 0 \\
		    0 & 0 & 2 \\
		    0 & 0 & 2
		\end{array}\right)\\\\
		&
		-R_2+R_3\to R_3
		&
		\left(\begin{array}{*{3}{r}}
		    1 & 0 & 0 \\
		    0 & 0 & 2 \\
		    0 & 0 & 0
		\end{array}\right)
		&
		\dfrac{1}{2}R_2\to R_2
		&
		\left(\begin{array}{*{3}{r}}
		    1 & 0 & 0 \\
		    0 & 0 & 1 \\
		    0 & 0 & 0
		\end{array}\right)
	    \end{array}$$

	    Así, el autovector asociado a $\alpha=-1$ será:\\

	    \begin{tcolorbox}
		$$v=(0,t,0) \quad t\in \textbf{F}.$$
	    \end{tcolorbox}

	Sea $t=1$, entonces
	\begin{tcolorbox}
	    \begin{equation}
		v=(0,1,0)
	    \end{equation}
	\end{tcolorbox}

    \end{itemize}
    \vspace{.6cm}

    Ahora, hallemos $A^n$. Sabemos que existe $V$ inversible cuyas columnas son autovectores de $A$. Por tanto, $A$ es diagonalizable y $A=VDV^{-1}$, con
	\begin{tcolorbox}
    $$D=\mbox{diag}(1,1,-1)=
    \left(\begin{array}{*{3}{r}}
	1 & 0 & 0 \\
	0 & 1 & 0 \\
	0 & 0 & -1
    \end{array}\right)$$ 
    	\end{tcolorbox}

    De (1), (2) y (3) se sigue que

    \begin{tcolorbox}
    $$V = \left(\begin{array}{*{3}{r}}
	1 & -1 & 0 \\
	0 & 0 & 1 \\
	0 & 1 & 0 
    \end{array}\right)$$
    \end{tcolorbox}

    Luego, saquemos $V^{-1}$ de la siguiente manera:

    $$\begin{array}{ccc}
	V^{-1}=\left(\begin{array}{*{6}{rrr|rrr}}
	    1 & -1 & 0 & 1 & 0 & 0 \\
	    0 & 0 & 1 & 0 & 1 & 0 \\
	    0 & 1 & 0 & 0 & 0 & 1 
	\end{array}\right)
	&
	R_3\leftrightarrow R_2
	&
	\left(\begin{array}{*{6}{rrr|rrr}}
	    1 & -1 & 0 & 1 & 0 & 0 \\
	    0 & 1 & 0 & 0 & 0 & 1  \\
	    0 & 0 & 1 & 0 & 1 & 0
	\end{array}\right)\\\\
	&
	R_1+R_2 \to R_1
	&
	\left(\begin{array}{*{6}{rrr|rrr}}
	    1 & 0 & 0 & 1 & 0 & 1 \\
	    0 & 1 & 0 & 0 & 0 & 1  \\
	    0 & 0 & 1 & 0 & 1 & 0
	\end{array}\right)\\\\
    \end{array}$$
    Por lo tanto,
    \begin{tcolorbox}
    $$V^{-1}=\left(\begin{array}{*{3}{c}}
	     1 & 0 & 1 \\
	     0 & 0 & 1  \\
	     0 & 1 & 0
    \end{array}\right)$$
    \end{tcolorbox}


    Dado que $A^n = VD^n V^{-1}$, se sigue

    $$\begin{array}{rcl}
	 A^n &=& \left[\left(\begin{array}{*{3}{r}}
	    1 & -1 & 0 \\
	    0 & 0 & 1 \\
	    0 & 1 & 0 
	\end{array}\right) 
	\times 
	\left(\begin{array}{*{3}{llc}}
	    1^n & 0 & 0 \\
	    0 & 1^n & 0 \\
	    0 & 0 & (-1)^n
	\end{array}\right)\right] 
	\times 
	\left(\begin{array}{*{3}{r}}
	    1 & 0 & 1 \\
	    0 & 0 & 1  \\
	    0 & 1 & 0
	\end{array}\right)\\\\
	&=&
	\left(\begin{array}{rcl}
	    \gamma_1 &=& 1\cdot(1,0,0)+(-1)\cdot(0,1,0)+0\cdot\left[0,0,(-1)^n\right]\\
		     &=&(1,-1,-1)\\\\
	    \gamma_2 &=& 0\cdot(1,0,0)+0\cdot(0,1,0)+1\cdot\left[0,0,(-1)^n\right]\\
		     &=&\left[0,0,(-1)^n\right]\\\\
	    \gamma_3 &=& 0\cdot(1,0,0)+1\cdot(0,1,0)+0\cdot\left[0,0,(-1)^n\right]\\
		     &=&(0,1,0)
	\end{array}\right)
	\times 
	\left(\begin{array}{*{3}{r}}
	    1 & 0 & 1 \\
	    0 & 0 & 1  \\
	    0 & 1 & 0
	\end{array}\right)\\\\
	&=&
	\left(\begin{array}{*{3}{r}}
	    1 & -1 & 0 \\
	    0 & 0 & (-1)^n \\
	    0 & 1 & 0
	\end{array}\right)
	\times 
	\left(\begin{array}{*{3}{r}}
	    1 & 0 & 1 \\
	    0 & 0 & 1  \\
	    0 & 1 & 0
	\end{array}\right)\\\\
	&=&
	\left(\begin{array}{rcl}
	    \gamma_1 &=& 1\cdot(1,0,1)+(-1)\cdot(0,0,1)+0\cdot(0,1,0)\\
		     &=&(1,0,0)\\\\
	    \gamma_2 &=& 0\cdot(1,0,1)+0\cdot(0,0,1)+(-1)^n\cdot(0,1,0)\\
		     &=&\left[0,(-1)^n,0\right]\\\\
	    \gamma_3 &=& 0\cdot(1,0,1)+1\cdot(0,0,1)+0\cdot(0,1,0)\\
		     &=&(0,0,1)
	\end{array}\right)\\\\
    \end{array}$$
    
    \begin{tcolorbox}
	$$A^n = \left(\begin{array}{*{3}{r}}
	    1 & 0 & 0 \\
	    0 & (-1)^n & 0 \\
	    0 & 0 & 1
	\end{array}\right)$$
    \end{tcolorbox}
    \vspace{1cm}


    %------------------------------EJERCICIO 2..................................

    \item [\bfseries Problema 2.] \textbf{\boldmath Resuelva analítica y gráficamente el problema de optimización que le corresponda por número.
    $$\min(\max):x^2+y^2\mbox{ sujeto a: } x+y\leq 2; (x-1)^2+y^2\leq 1.$$\\
	Respuesta.-\;} Tenemos que optimizar:
	$$\left\{\begin{array}{llll}
		Opt.: & x^2+y^2&=&f\\
		s.a. & x+y-2\leq 0&=&g_1\\
		     & x^2-2x+y^2\leq 0&=&g_2.
	\end{array}\right\}$$
	Veamos que el dominio esta en $\mathbb{R}^2$. Luego, escribimos el conjunto de las soluciones factibles:
	\begin{tcolorbox}
	$$C=\left\{(x,y)\in \mathbb{R}^2:x+y-2\leq 0, \; x^2-2x+y^2\leq 0\right\}$$
	\end{tcolorbox}

	Ya que, 
	$$0+1\leq 2 \quad \land (1-1)^2+0^2\leq 1,$$
	la región factible es:
	\begin{center}
	    \begin{tikzpicture}
		\begin{axis}[scale=.5,draw opacity =.5,samples=100,smooth, 
		  axis x line=center, 
		  axis y line=center,
		  ylabel = {$f(x)$},
		  xlabel = {$x$},
		  xlabel style={below right},
		  ylabel style={above left},
		  label style={font=\tiny},
		  tick label style={font=\tiny},
		  enlargelimits=upper] 
		  \addplot[name path=f,domain=0:3.2,blue] {-x+2};
                  \path[name path=axis] (axis cs:0,-1.5) -- (axis cs:-1.5,0);
		    \addplot [
			thick,
			color=blue,
			fill=blue, 
			fill opacity=0.05
		    ]
		    fill between[
			of=f and axis,
			split
			every segment no 0/.style={
			    %fill=none,
			    yellow,
			},
		    ];
		  \addplot[name path=f,domain=0:2,red] {(-(x-1)^2+1)^(1/2)};
                  \path[name path=axis] (axis cs:0,0) -- (axis cs:0,0);
		    \addplot [
			thick,
			color=red,
			fill=red, 
			fill opacity=0.04
		    ]
		    fill between[
			of=f and axis,
			split
			every segment no 0/.style={
			    %fill=none,
			    yellow,
			},
		    ];
		  \addplot[name path=f,domain=0:2,red] {-(-(x-1)^2+1)^(1/2)};
                  \path[name path=axis] (axis cs:0,0) -- (axis cs:0,0);
		    \addplot [
			thick,
			color=red,
			fill=red, 
			fill opacity=0.04
		    ]
		    fill between[
			of=f and axis,
			split
			every segment no 0/.style={
			    %fill=none,
			    yellow,
			},
		    ];

		\end{axis}
	    \end{tikzpicture}
	\end{center}

	Donde, se ve claramente que es compacto y convexo.\\

	Ahora construimos la Lagrangiana, como sigue

	\begin{tcolorbox}
	    $$\mathcal{L}(x,y,\gamma,\mu)=x^2+y^2-\gamma(x+y-2)-\mu(x^2-2x+y^2).$$
	\end{tcolorbox}

	Los puntos de Kun Tuker deben cumplir que las derivadas parciales de la Lagranjeana respecto de las variables de función origintal se anulen. Es decir,

	\begin{enumerate}[i)]
	    \item
		$$\left\{\begin{array}{rcl}
		    2x-\gamma - 2\mu x+2\mu&=&0\\\\
		    2y-\gamma -2\mu y&=&0
		\end{array}\right.$$

	    \item $$
		    \left\{
			\begin{array}{rcl}
			    \gamma(x+y-2)& = &0\\\\
			    \mu(x^2-2x+y^2)& = &0.
			\end{array}
		    \right.
		  $$ 

	    \item
		$$
		    \left\{
			\begin{array}{ll}
			     \max & \gamma\geq 0,\; \; \mu\geq 0\\\\
			     \min & \gamma\leq 0,\; \; \mu\leq 0
			\end{array}
		    \right.
		$$
	    \item $$
		    \left\{
			\begin{array}{rcl}
			    x+y&\leq& 2\\\\
			    (x-1)^2+y^2&\leq &1
			\end{array}
		    \right.
		    $$
	\end{enumerate}
	\vspace{.5cm}

	Después, resolviendo para ver quienes son puntos de Kuhn Tucker:\\

	\begin{itemize}
	    \item Si $\gamma = \mu = 0$. Entonces,
		$$\left\{\begin{array}{rcl}
		    2x=0&\Rightarrow &x=0\\\\
		    2y=0&\Rightarrow &y=0.
		\end{array}\right.$$

		Así, los puntos de Kuhn Tucker serán: 
		\begin{tcolorbox}
		    $$x=0,\quad  y=0.$$
		\end{tcolorbox}

	    \item Si $\lambda=0$ y $\mu\neq 0$. Entonces,
		$$\left\{\begin{array}{rcl}
		    2x -2\mu x + 2\mu=0&\Rightarrow &\mu=\dfrac{x}{x+1}\\\\
		    2y-2\mu y=0&\Rightarrow &y=0.
		\end{array}\right.$$

		Luego, si $\mu\neq 0$, obligatoriamente, se tiene que
		$$\mu(x^2-2x+y^2)=0\quad \Rightarrow \quad x^2-2x+y^2=0.$$
		Pero como $y=0$. Entonces,
		$$x(x-2)=0 \quad \Rightarrow \quad x=0,\qquad  x=2.$$
		Si $x=0$, entonces $\mu=\frac{0}{0+1}=0$. Lo que se contradice con la condición dada. Por lo que, para $\lambda=0$ y $\mu\neq 0$, no existe puntos de Kuhn Tucker.\\\\

	    \item Si $\lambda \neq 0$ y $\mu=0$. entonces,
		$$\begin{array}{rcl}
		    2x-\lambda = 0 &\Rightarrow & \lambda = -2x\\\\
		    2y-2\mu y = 0 &\Rightarrow& y=0.
		\end{array}$$

		Luego, si $\lambda \neq 0$, obligatoriamente, tenemos
		$$x+y-2=0.$$
		Dado que, $y=0$. Entonces,
		$$x-2=0\quad \Rightarrow \quad x=2.$$
		Por lo tanto, los puntos de Kuhn Tucker estarán dados por:
		\begin{tcolorbox}
		    $$y=0,\quad x=2,\quad \lambda = -4.$$
		\end{tcolorbox}

	    \item Si $\lambda \neq 0$ y $\mu \neq 0$, entonces
		$$\left\{\begin{array}{rcl}
		    x+y-2=0 &\Rightarrow & x=2-y\\\\
		    x^2-2x+y^2=0&&
		\end{array}\right.$$

		Reemplazando $x$ en la segunda ecuación, tenemos
		$$\begin{array}{rcl}
		    (2-y)^2-2(2-y)+y^2=0 &\Rightarrow & 2y^2-2y=0\\\\
					 &\Rightarrow & y_1=0 \; \land \; y_2=1.
		\end{array}$$
		De donde,
		$$x_1=2 \quad \land \quad x_2=1.$$
		Luego \\
		\begin{itemize}
		    \item Para $x_1=2$ e $y_1=0$
		    $$\begin{array}{rcl}
			4-\lambda-4\mu_1+2\mu_1 = 0 &\Rightarrow & \mu_1 = \dfrac{4-\lambda}{2}\\\\
			2\cdot 0-\lambda_1 - 2\mu_1 \cdot 0 = 0 &\Rightarrow & \lambda_1 = 0
		    \end{array}$$
		    Por lo tanto,
		    \begin{tcolorbox}
			$$x_1=2,\quad y_1=0, \quad \lambda_1 = 0, \quad \mu_1=2$$
		    \end{tcolorbox}
		    \item Para $x_2=1$ e $y_2=1$
		    $$\begin{array}{rcl}
			2-\lambda_2 -2\mu_2 + 2\mu_2 = 0 &\Rightarrow&\lambda_2 = 2\\\\
			2-2 -2\mu_2  = 0 &\Rightarrow&\mu_2 = 0
		    \end{array}$$
		    Por lo tanto,
		    \begin{tcolorbox}
			$$x_2=1,\quad y_2=1, \quad \lambda_2 = 2, \quad \mu_2=0$$
		    \end{tcolorbox}

		\end{itemize}

	\end{itemize}

	Puntos de Kuhn Tucker:

	$$
	\begin{array}{*{5}{crrcc}}
	    (x,y) & \lambda & \mu & fact & candidato\\
	    \hline\\
	    (0,0)&0&0&si&\min\\\\
	    (2,0)&-4&0&si&\min\\\\
	    (2,0)&0&2&si&\max\\\\
	    (1,1)&2&0&si&\max\\\\
	\end{array}
	$$

	Las curvas de nivel estaran dadas por el conjunto:
	$$C_\alpha (f) = \left\{ (x,y) \in \mathbb{R} \; | \; x^2+y^2=\alpha \right\}$$
	De donde,
	$$y=\pm\sqrt{\alpha -x^2}$$

	\begin{center}
	    \begin{tikzpicture}
		\begin{axis}[scale=.7,draw opacity =.5,samples=100,smooth, 
		  axis x line=center, 
		  axis y line=center,
		  ylabel = {$f(x)$},
		  xlabel = {$x$},
		  xlabel style={below right},
		  ylabel style={above left},
		  label style={font=\tiny},
		  tick label style={font=\tiny},
		  enlargelimits=upper] 
		  \addplot[name path=f,domain=0:3.2,blue] {-x+2};
                  \path[name path=axis] (axis cs:0,-1.5) -- (axis cs:-1.5,0);
		    \addplot [
			thick,
			color=blue,
			fill=blue, 
			fill opacity=0.05
		    ]
		    fill between[
			of=f and axis,
			split
			every segment no 0/.style={
			    %fill=none,
			    yellow,
			},
		    ];
		  \addplot[name path=f,domain=0:2,red] {(-(x-1)^2+1)^(1/2)};
                  \path[name path=axis] (axis cs:0,0) -- (axis cs:0,0);
		    \addplot [
			thick,
			color=red,
			fill=red, 
			fill opacity=0.04
		    ]
		    fill between[
			of=f and axis,
			split
			every segment no 0/.style={
			    %fill=none,
			    yellow,
			},
		    ];
		  \addplot[name path=f,domain=0:2,red] {-(-(x-1)^2+1)^(1/2)};
                  \path[name path=axis] (axis cs:0,0) -- (axis cs:0,0);
		    \addplot [
			thick,
			color=red,
			fill=red, 
			fill opacity=0.04
		    ]
		    fill between[
			of=f and axis,
			split
			every segment no 0/.style={
			    %fill=none,
			    yellow,
			},
		    ];
		  \addplot[domain=-.8:.8,black,opacity=.15]{sqrt(.1-x^2)};
		  \addplot[domain=-.8:.8,black,opacity=.15]{-sqrt(.1-x^2)};
		  \addplot[domain=-.5:.5,black,opacity=.15]{(.25-x^2)^(1/2)};
		  \addplot[domain=-.5:.5,black,opacity=.15]{-(.25-x^2)^(1/2)};
		  \addplot[domain=-1:1,black,opacity=.15]{sqrt(.5-x^2)};
		  \addplot[domain=-1:1,black,opacity=.15]{-sqrt(.5-x^2)};
		  \addplot[domain=-1:1,black,opacity=.15]{(1-x^2)^(1/2)};
		  \addplot[domain=-1:1,black,opacity=.15]{-(1-x^2)^(1/2)};
		  \addplot[domain=-1.3:1.3,black,opacity=.15]{(1.5-x^2)^(1/2)};
		  \addplot[domain=-1.3:1.3,black,opacity=.15]{-(1.5-x^2)^(1/2)};
		  \addplot[domain=-2:2,black,opacity=.15]{(2.2-x^2)^(1/2)};
		  \addplot[domain=-2:2,black,opacity=.15]{-(2.2-x^2)^(1/2)};
		  \addplot[domain=-1.8:1.8,black,opacity=.15]{(3-x^2)^(1/2)};
		  \addplot[domain=-1.8:1.8,black,opacity=.15]{-(3-x^2)^(1/2)};
		  \addplot[domain=-2:2.2,black,opacity=.15]{(4-x^2)^(1/2)};
		  \addplot[domain=-2:2.2,black,opacity=.15]{-(4-x^2)^(1/2)};

		\end{axis}
	    \end{tikzpicture}
	\end{center}

	Sólo $\exists C_\alpha(f)$ si $\alpha\geq 0$. Luego, si $\alpha=0$, entonces $C_{\alpha}(f)=l\left\{ (0,0) \right\}$. Después, si $\alpha\geq 0$, entonces la curva de nivel tiene una circunferencia $\left[(0,0),\sqrt{\alpha}\right]$. Por lo tanto, el punto $(0,0)$ es un mínimo global estricto.\\
	Ahora, el punto máximo será la última circunferencia que corte, a la reta y circunferencia dada. Para ello, primero encontramos los puntos en los que se intersecan $x+y-2=0$ y $x^2-2x+y^2=0$, de la siguiente manera:
	$$y=2-x \quad\mbox{de donde}\quad x^2-2x+(2-x)^2=0\quad \Rightarrow \quad x^2-3x+2=0.$$
	Así,
	    $$x_1=2\quad \lor \quad x_2=1$$
	Después, reemplazamos en $y=2-x$, para encontrar los puntos de intersección:
	\begin{tcolorbox}
	    $$(2,0)\quad \lor \quad (1,1)$$
	\end{tcolorbox}

	Ya que $\alpha$ es el radio de la circunferencia a optimizar, entonces reemplazamos estos últimos resultados:
	$$
	\begin{array}{rcl}
	    (2,0) & \Rightarrow & \alpha=2^2+0^2=4\\\\
	    (1,1) & \Rightarrow & \alpha=1^2+1^2=2
	\end{array}
	$$
	De donde, $\alpha=4$ será el radio máximo donde se intersecan la recta y la circunferencia dada. Así, el máximo local absoluto será $(2,0)$. Por lo tanto, 
	\begin{tcolorbox}
	    $$
	    \begin{array}{rcl}
		\min &=& (0,0)\\\\
		\max &=& (2,0).
	    \end{array}
	    $$
	\end{tcolorbox}

\end{enumerate}



