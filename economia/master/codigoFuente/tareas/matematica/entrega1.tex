\section*{\center \large Primera entrega}
\vspace*{1cm}

\begin{enumerate}

    \item[\bfseries Problema 1.] \textbf{\boldmath Calcule los autovalores y autovectores de la matriz $A$. Cuando sea posible, diagonalice y obtenga una expresión en función de la diagonal para $A^n$.}

    $$A=\left(\begin{array}{*{3}{r}}
	1 & 0 & 0 \\
	2 & -1 & 2 \\
	0 & 0 & 1 
    \end{array}\right)$$\\

    \textbf{Respuesta.-}\;  Por definición, y la regla de Sarrus el polinomio característica de $A$ es:
    
    $$ \begin{array}{rcl} 
	\det \left[
    \left(\begin{array}{*{3}{r}}
	1 & 0 & 0 \\
	2 & -1 & 2 \\
	0 & 0 & 1 
    \end{array}\right) - \alpha 
    \left(\begin{array}{*{3}{r}}
	1 & 0 & 0 \\
	0 & 1 & 0 \\
	0 & 0 & 1
\end{array}\right)\right] & = & 
    \left|\begin{array}{*{3}{c}}
	1-\alpha & 0 & 0 \\
	2 & -1-\alpha & 2 \\
	0 & 0 & 1-\alpha
	\end{array}\right| \\\\ 
	&=&  
	\left|\begin{array}{*{3}{c}}
	1-\alpha & 0 & 0 \\
	2 & -1-\alpha & 2 \\
	0 & 0 & 1-\alpha
	\end{array}\right|
	\left.\begin{array}{*{3}{c}}
	1-\alpha & 0  \\
	2 & -1-\alpha  \\
	0 & 0
	\end{array}\right|\\\\
	&=&(1-\alpha)(-1-\alpha)(1-\alpha)+0\cdot 2 \cdot 0 + 0\cdot 2 \cdot 0\\\\
	&=&-(1+\alpha)\left(1-\alpha\right)^2.
    \end{array}$$

    Luego igualamos el último resultado a $0$, de donde obtenemos,\\
    $$-(1+\alpha)\left(1-\alpha\right)^2=0\quad \Rightarrow \quad (1+\alpha)\left(1-\alpha\right)^2=0.$$\\
    Así, los autovalores estarán dados por:
    \begin{tcolorbox}
	$$\alpha=1\; (\mbox{doble})  \;\;\lor\;\; \alpha=-1$$
    \end{tcolorbox}
    \vspace*{.2cm}

    Ahora calculemos los autovectores.

    \begin{itemize}

	\item Para $\alpha=1$
	    $$
	    \left(\begin{array}{*{3}{c}}
		1-1 & 0 & 0\\
		2 & -1-1 & 2 \\
		0 & 0 & 1-1
	    \end{array}\right) 
	    \left(\begin{array}{*{3}{c}}
		v_1 \\
		v_2 \\
		v_3
	    \end{array}\right) = 
	    \left(\begin{array}{*{3}{c}}
		0 & 0 & 0 \\
		2 & -2 & 2 \\
		0 & 0 & 0
	    \end{array}\right)
	    \left(\begin{array}{*{3}{c}}
		v_1 \\
		v_2 \\
		v_3
	    \end{array}\right) = 
	    \left(\begin{array}{*{3}{c}}
		0 \\
		0 \\
		0	
	    \end{array}\right) 
	    $$

	    Entonces, por el método de eliminación Gauss-Jordan

	    $$\left\{\begin{array}{rcl}
		v_1-v_2+v_3 &=& 0\\
		0v_1+0v_2+0v_3 &=& 0\\
		0v_1+0v_2+0v_3 &=& 0
	    \end{array}\right.
	    \quad \Rightarrow \quad
	    \left\{\begin{array}{rcl}
		v_1 &=& v_2-v_3\\
		v_2 &=& v_2\\
		v_3 &=& v_3
	    \end{array}\right.$$

	    De donde, el autovector asociado a $\alpha=1$ será:\\
	    \begin{tcolorbox}
	    $$v=(v_2-v_3,s,t) \quad s,t\in \textbf{F}.$$
	    \end{tcolorbox}

	    Por ejemplo, sea $s=1$ y $t=0$, entonces
	    \begin{tcolorbox}
		\begin{equation}
		    v=(1,0,0)
		\end{equation}
	    \end{tcolorbox}

	    Como existe la dualidad podemos tomar $s=0$ y $t=1$. Por lo tanto,
	    \begin{tcolorbox}
		\begin{equation}
		    v=(-1,0,1)
		\end{equation}
	    \end{tcolorbox}
	    \vspace{.6cm}

	\item Para $\alpha=-1$
	    $$
	    \left(\begin{array}{*{3}{c}}
		1+1 & 0 & 0\\
		2 & -1+1 & 2 \\
		0 & 0 & 1+1
	    \end{array}\right)  
	    \left(\begin{array}{c}
		v_1 \\
		v_2 \\
		v_3
	    \end{array}\right) = 
	    \left(\begin{array}{*{3}{c}}
		2 & 0 & 0 \\
		2 & 0 & 2 \\
		0 & 0 & 2
	    \end{array}\right)
	    \left(\begin{array}{c}
		v_1 \\
		v_2 \\
		v_3
	    \end{array}\right) = 
	    \left(\begin{array}{c}
		0 \\
		0 \\
		0	
	    \end{array}\right)
	    $$

	    Entonces, por el método de eliminación Gauss-Jordan

	    $$\begin{array}{ccccc}
		\left(\begin{array}{*{3}{r}}
		    2 & 0 & 0 \\
		    2 & 0 & 2 \\
		    0 & 0 & 2
		\end{array}\right)
		&
		\dfrac{1}{2}R_1\to R_1
		&
		\left(\begin{array}{*{3}{r}}
		    1 & 0 & 0 \\
		    2 & 0 & 2 \\
		    0 & 0 & 2
		\end{array}\right)
		&
		-2R_1+R_2\to R_2
		&
		\left(\begin{array}{*{3}{r}}
		    1 & 0 & 0 \\
		    0 & 0 & 2 \\
		    0 & 0 & 2
		\end{array}\right)\\\\
		&
		-R_2+R_3\to R_3
		&
		\left(\begin{array}{*{3}{r}}
		    1 & 0 & 0 \\
		    0 & 0 & 2 \\
		    0 & 0 & 0
		\end{array}\right)
		&
		\dfrac{1}{2}R_2\to R_2
		&
		\left(\begin{array}{*{3}{r}}
		    1 & 0 & 0 \\
		    0 & 0 & 1 \\
		    0 & 0 & 0
		\end{array}\right)
	    \end{array}$$

	    Así, el autovector asociado a $\alpha=-1$ será:\\

	    \begin{tcolorbox}
		$$v=(0,t,0) \quad t\in \textbf{F}.$$
	    \end{tcolorbox}

	Sea $t=1$, entonces
	\begin{tcolorbox}
	    \begin{equation}
		v=(0,1,0)
	    \end{equation}
	\end{tcolorbox}

    \end{itemize}
    \vspace{.6cm}

    Ahora, hallemos $A^n$. Sabemos que existe $V$ inversible cuyas columnas son autovectores de $A$. Por tanto, $A$ es diagonalizable y $A=VDV^{-1}$, con
	\begin{tcolorbox}
    $$D=\mbox{diag}(1,1,-1)=
    \left(\begin{array}{*{3}{r}}
	1 & 0 & 0 \\
	0 & 1 & 0 \\
	0 & 0 & -1
    \end{array}\right)$$ 
    	\end{tcolorbox}

    De (1), (2) y (3) se sigue que

    \begin{tcolorbox}
    $$V = \left(\begin{array}{*{3}{r}}
	1 & -1 & 0 \\
	0 & 0 & 1 \\
	0 & 1 & 0 
    \end{array}\right)$$
    \end{tcolorbox}

    Luego, saquemos $V^{-1}$ de la siguiente manera:

    $$\begin{array}{ccc}
	V^{-1}=\left(\begin{array}{*{6}{rrr|rrr}}
	    1 & -1 & 0 & 1 & 0 & 0 \\
	    0 & 0 & 1 & 0 & 1 & 0 \\
	    0 & 1 & 0 & 0 & 0 & 1 
	\end{array}\right)
	&
	R_3\leftrightarrow R_2
	&
	\left(\begin{array}{*{6}{rrr|rrr}}
	    1 & -1 & 0 & 1 & 0 & 0 \\
	    0 & 1 & 0 & 0 & 0 & 1  \\
	    0 & 0 & 1 & 0 & 1 & 0
	\end{array}\right)\\\\
	&
	R_1+R_2 \to R_1
	&
	\left(\begin{array}{*{6}{rrr|rrr}}
	    1 & 0 & 0 & 1 & 0 & 1 \\
	    0 & 1 & 0 & 0 & 0 & 1  \\
	    0 & 0 & 1 & 0 & 1 & 0
	\end{array}\right)\\\\
    \end{array}$$
    Por lo tanto,
    \begin{tcolorbox}
    $$V^{-1}=\left(\begin{array}{*{3}{c}}
	     1 & 0 & 1 \\
	     0 & 0 & 1  \\
	     0 & 1 & 0
    \end{array}\right)$$
    \end{tcolorbox}


    Dado que $A^n = VD^n V^{-1}$, se sigue

    $$\begin{array}{rcl}
	 A^n &=& \left[\left(\begin{array}{*{3}{r}}
	    1 & -1 & 0 \\
	    0 & 0 & 1 \\
	    0 & 1 & 0 
	\end{array}\right) 
	\times 
	\left(\begin{array}{*{3}{llc}}
	    1^n & 0 & 0 \\
	    0 & 1^n & 0 \\
	    0 & 0 & (-1)^n
	\end{array}\right)\right] 
	\times 
	\left(\begin{array}{*{3}{r}}
	    1 & 0 & 1 \\
	    0 & 0 & 1  \\
	    0 & 1 & 0
	\end{array}\right)\\\\
	&=&
	\left(\begin{array}{rcl}
	    \gamma_1 &=& 1\cdot(1,0,0)+(-1)\cdot(0,1,0)+0\cdot\left[0,0,(-1)^n\right]\\
		     &=&(1,-1,-1)\\\\
	    \gamma_2 &=& 0\cdot(1,0,0)+0\cdot(0,1,0)+1\cdot\left[0,0,(-1)^n\right]\\
		     &=&\left[0,0,(-1)^n\right]\\\\
	    \gamma_3 &=& 0\cdot(1,0,0)+1\cdot(0,1,0)+0\cdot\left[0,0,(-1)^n\right]\\
		     &=&(0,1,0)
	\end{array}\right)
	\times 
	\left(\begin{array}{*{3}{r}}
	    1 & 0 & 1 \\
	    0 & 0 & 1  \\
	    0 & 1 & 0
	\end{array}\right)\\\\
	&=&
	\left(\begin{array}{*{3}{r}}
	    1 & -1 & 0 \\
	    0 & 0 & (-1)^n \\
	    0 & 1 & 0
	\end{array}\right)
	\times 
	\left(\begin{array}{*{3}{r}}
	    1 & 0 & 1 \\
	    0 & 0 & 1  \\
	    0 & 1 & 0
	\end{array}\right)\\\\
	&=&
	\left(\begin{array}{rcl}
	    \gamma_1 &=& 1\cdot(1,0,1)+(-1)\cdot(0,0,1)+0\cdot(0,1,0)\\
		     &=&(1,0,0)\\\\
	    \gamma_2 &=& 0\cdot(1,0,1)+0\cdot(0,0,1)+(-1)^n\cdot(0,1,0)\\
		     &=&\left[0,(-1)^n,0\right]\\\\
	    \gamma_3 &=& 0\cdot(1,0,1)+1\cdot(0,0,1)+0\cdot(0,1,0)\\
		     &=&(0,0,1)
	\end{array}\right)\\\\
    \end{array}$$
    
    \begin{tcolorbox}
	$$A^n = \left(\begin{array}{*{3}{r}}
	    1 & 0 & 0 \\
	    0 & (-1)^n & 0 \\
	    0 & 0 & 1
	\end{array}\right)$$
    \end{tcolorbox}
    \vspace{1cm}


    %------------------------------EJERCICIO 2..................................

    \item [\bfseries Problema 2.] \textbf{\boldmath Resuelva analítica y gráficamente el problema de optimización que le corresponda por número.
    $$\min(\max):x^2+y^2\mbox{ sujeto a: } x+y\leq 2; (x-1)^2+y^2\leq 1.$$\\
	Respuesta.-\;} Tenemos que optimizar:
	$$\left\{\begin{array}{llll}
		Opt.: & x^2+y^2&=&f\\
		s.a. & x+y-2\leq 0&=&g_1\\
		     & x^2-2x+y^2\leq 0&=&g_2.
	\end{array}\right\}$$
	Veamos que el dominio esta en $\mathbb{R}^2$. Luego, escribimos el conjunto de las soluciones factibles:
	$$C=\left\{(x,y)\in \mathbb{R}^2:x+y-2\leq 0, x^2-2x+y^2\leq 0\right\}$$



	Sabemos que es compacto y convexo. 

	$$\mathcal{L}(x,y,\gamma,\mu)=x^2+y^2-\gamma(x+y-2)-\mu(x^2-2x+y^2).$$

	Los puntos de Kun Tuker deben cumplir que las derivadas parciales de la Lagranjeana respecto de las variables de función origintal se anulen. Es decir,

	\begin{enumerate}[i)]
	    \item
		$$\begin{array}{rcl}
		    2x-\gamma - 2\mu x+2\mu&=&0\\
		    2y-\gamma -2\mu y&=&0
		\end{array}$$

	    \item $$\gamma(x+y-2)\geq 0,\; \; \mu(x^2-2x+y^2)\geq 0.$$ 
	    \item $$\gamma\geq 0,\; \; \mu\geq 0, \; \; \gamma\leq 0,\; \; \mu\leq 0.$$
	    \item $$x+y\leq 2,\; (x-1)^2+y^2\leq 1$$
	\end{enumerate}

	Resolviendo tenemos,

	\begin{itemize}
	    \item Si $\gamma =\mu =0$. Entonces,
		$$\begin{array}{rcl}
		    2x-0 - 2\cdot 0 x+2\cdot 0&=&0\\
		    2y-0 -2\cdot 0 y&=&0
		\end{array}$$
	\end{itemize}

	\textbf{Solución gráfica:}
	Dado que el problema a optimizar está en $\mathbb{R}^2$. Podemos representarla gráficamente. Para ello,
	\begin{enumerate}
	    \item Representamos en $\$
	\end{enumerate}
	


\end{enumerate}




