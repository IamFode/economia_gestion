\section*{\center \large Segunda entrega}
\begin{center}
    \textbf{Christian Limbert Paredes Aguilera}.
\end{center}
\vspace{1cm}

\begin{enumerate}[\textbf{Ejercicio} \bfseries 1.]

    %-------------------- 1.
    \item \textbf{\boldmath Calcula las soluciones generales de las siguientes ecuaciones diferenciales, así como las soluciones que verifican los datos iniciales dados:
	$$e^{2t}x'-x^2-2x-1=0,\quad (0,0).$$}
	\setlength{\columnseprule}{.1pt}
	\begin{multicols}{2}
	\textbf{Respuesta.-}\; Resolveremos, utilizando el método de primer orden de variables separadas. Primero reescribimos la ecuación, de la siguiente forma:
	$$
	\begin{array}{rcl}
	    e^{2t}x'-x^2-2x-1&=&0 \\\\
	    e^{2t}x'&=&x^2+2x+1\\\\
	    \dfrac{e^{2t}}{2^{2t}}&=&\dfrac{x^2+2x+1}{e^{2t}}\\\\
	    x'&=&\dfrac{x^2+2x+1}{e^{2t}}\\\\
	    \dfrac{x'}{x^2+2x+1}&=&\dfrac{\dfrac{x^2+2x+1}{e^{2t}}}{x^2+2x+1}\\\\
	    \dfrac{1}{x^2+2x+1}x'&=&\dfrac{1}{e^{2t}}.
	\end{array}
	$$
	Luego, integrando la última ecuación, tenemos\\
	$$
	\begin{array}{rcl}
	    \displaystyle\int \dfrac{1}{x^2+2x+1}\; dx &=&\displaystyle\int \dfrac{1}{e^{2t}}\; dt\\\\
	\end{array}
	$$
	Resolvamos la integral de la izquierda de la ecuación. Sea, 
	$$u=x+1 \quad \Rightarrow \quad  \dfrac{du}{dx}=1 \quad \Rightarrow \quad dx=du.$$ 
	Entonces,
	$$
	\begin{array}{rcl}
	    \displaystyle\int \dfrac{1}{x^2+2x+1}\; dx &=& \displaystyle\int \dfrac{1}{(x+1)^2}\; dx\\\\
						       &=& \displaystyle\int \dfrac{1}{u^2}\; du\\\\
						       &=& \displaystyle\int u^{-2}dx\\\\
						       &=& -u^{-1}+c\\\\
						       &=& -\dfrac{1}{x+1}+c_1\\\\
	\end{array}
	$$
	Por otro lado, resolvamos la integral de la derecha de la ecuación. Sea,
	$$u=-2t\quad \Rightarrow \quad \dfrac{du}{dt}=-2 \quad \Rightarrow \quad dt=-\dfrac{du}{2}$$
	De donde,
	$$
	\begin{array}{rcl}
	    \displaystyle\int \dfrac{1}{e^{2t}}\; dt &=& \displaystyle\int e^{-2t}\; dt\\\\
						     &=& -\dfrac{1}{2}\displaystyle\int e^{u}\; du\\\\
						     &=& -\dfrac{1}{2}e^{u}+c_2\\\\
						     &=& -\dfrac{1}{2}e^{-2t}+c_2.
	\end{array}
	$$
	Por lo tanto, 
	\begin{tcolorbox}
	    $$-\dfrac{1}{x+1}+c_1=-\dfrac{1}{2}e^{-2t}+c_1.$$
	\end{tcolorbox}

	Sea $c=c_2-c_1$, entonces despejando $x$ en función de $t$ se tiene:

	$$
	\begin{array}{rcl}
	    -\dfrac{1}{x+1}&=&-\dfrac{1}{2}e^{-2t}+c\\\\
	    \dfrac{1}{x+1}&=&\dfrac{1}{2}e^{-2t}-c\\\\
	    \dfrac{1}{x+1}&=&\dfrac{e^{-2t}-2c}{2}\\\\
	    x+1 &=& \dfrac{2}{e^{-2t}-2c}\\\\
	    x &=& \dfrac{2}{e^{-2t}-2c}-1.
	\end{array}
	$$
	Así, la solución general de la ecuación diferencial será:
	\begin{tcolorbox}
	    $$x(t) = \dfrac{2}{e^{-2t}-2c}-1$$
	\end{tcolorbox}

	Por último, reemplazamos $(0,0)$ en la ecuación general,
	$$
	\begin{array}{rcl}
	    0&=&\dfrac{2}{e^{-2(0)}-2c}-1\\\\
	    0&=&\dfrac{2-(1-2c)}{1-2c}\\\\
	    c&=&\dfrac{1}{2}.
    \end{array}
	$$
	Así, la solución inicial de la ecuación diferencial estará dada por:
	\begin{tcolorbox}
	    $$x(t) = \dfrac{2}{e^{-2t}-1}-1$$
	\end{tcolorbox}
	$\hfill\blacksquare$
	\end{multicols}
	\vspace{.5cm}

    %-------------------- 2.
    \item \textbf{\boldmath Resuelve las siguiente ecuaciones diferenciales:
	$$x''+2x'+x=t^2,\quad x(0)=0,\quad x'(0)=1.$$}\\
    \begin{multicols}{2}
    	\textbf{Respuesta.-}\; La solución general se puede escribir de la siguiente manera:
	$$x_g=x_h+x_p.$$
	Donde $x_h$ es la solución para la ecuación diferencial homogénea 
	$$a(x)x''+b(x)x'+c(x)x=0.$$
	En particular resolvamos para 
	$$x''(t)-2x'+x=0.$$
	Sea $x=e^{\lambda t}$, entonces
	$$\left(e^{\lambda t}\right)''+2\left(e^{\lambda t}\right)'+e^{\lambda t}=0.$$
	en el que,
	$$
	\begin{array}{rcl}
	    \left(e^{\lambda y}\right)' &=& \lambda e^{\lambda t}\\\\
	    \left(e^{\lambda t}\right)'' &=& \lambda^2e^{\lambda t}\\\\
	\end{array}
	$$
	en consecuencia,
	$$
	\begin{array}{rcl}
	    \lambda^2e^{\lambda t}+2e^{\lambda t}y+e^{\lambda t}&=&0\\\\
	    e^{\lambda t}\left(\lambda^2+2\lambda +1\right)&=&0\\\\
	\end{array}
	$$
	Por el hecho de que $e^{\lambda t}\neq 0$, entonces 
	$$\lambda^2+2\lambda +1=0$$.
	De donde, 
	\begin{tcolorbox}
	    $$\lambda=-1 \; (\mbox{mult. doble}).$$
	\end{tcolorbox}
	Dado, que las raíces son dobles, entonces
	\begin{tcolorbox}
	    $$x_g(t)=c_1e^{-t}+c_2te^{-t}.$$
	\end{tcolorbox}

	Después, resolvamos la ecuación diferencial particular de la siguiente manera: Sea la solución de la forma 

	$$x_p(t)=At^2+Bt+C.$$
	
	Calculamos la primera y segunda derivada,

	$$
	\begin{array}{rcl}
	    x_p'(t)&=&2At+B\\\\
	    x_p''(t)&=&2A.
	\end{array}
	$$

	Luego, reemplazamos en la ecuación diferencial:

	$$\left(2A\right)+2(2At+B)+\left(At^2+Bt+C\right)=t^2.$$

	Resolviendo, tenemos

	$$At^2+(4A+B)t+(2A+2B+C)=t^2.$$

	Igualando coeficiente de igual potencia en $t$, se tiene el sistema:
	$$
	\begin{array}{rcl}
	    A&=&1\\\\
	    4A+B&=&0\\\\
	    2A+2B+C&=&0.
	\end{array}
	$$
	Resolviendo el sistema, tenemos:
	\begin{tcolorbox}
	    $$A=1,\quad B=-4,\quad C=6.$$
	\end{tcolorbox}

	Ahora, sustituimos estos parámetros en $x_p(t)$
	\begin{tcolorbox}
	    $$x_p(t)=t^2-4t+6.$$
	\end{tcolorbox}

	Por lo tanto, la solución general será:

	\begin{tcolorbox}
	    $$x_g(t)=c_1e^{-t}+c_2te^{-t}+ t^2-4t+6$$\\
	\end{tcolorbox}

	Por último, resolvamos para: 
	\begin{itemize}
	    \item $x(0)=0$.
		$$
		\begin{array}{rcr}
		    c_1e^{0}+c_2e^{0}+6&=&0\\\\
		    c_1+c_2+6&=&0\\\\
		    c_1+c_2&=&-6.
		\end{array}
		$$

	    \item $x'(0)=1$.
		Derivando $x_g(t)$, tenemos:
		$$x_g'(t)=-c_1e^{-t}+c_2\left(e^{-t}-te^{-t}\right)+2t-4$$

		Luego, 
		$$
		\begin{array}{rcl}
		    -c_1e^{0}+c_2\left(e^{0}-0e^{0}\right)+2(0)-4&=&1\\\\
		    -c_1+c_2-4&=&1\\\\
		    c_2-c_1&=&5.
		\end{array}
		$$

		Resolviendo las dos ecuaciones:
		$$
		\left\{
		    \begin{array}{rcr}
			c_1+c_2&=&-6\\\\
			c_2-c_1&=&5
		    \end{array}
		\right. 
		$$
		
		Por lo tanto,

		\begin{tcolorbox}
		    $$c_1=-\dfrac{11}{2},\quad c_2=-\dfrac{1}{2}.$$
		\end{tcolorbox}
		Así, 

		\begin{tcolorbox}
		    $$x_g(t)=-\dfrac{11}{2}e^{-t}-\dfrac{1}{2}te^{-t}+ t^2-4t+6.$$
		\end{tcolorbox}

	\end{itemize}
	$\hfill\blacksquare$
	\vspace{.5cm}
    \end{multicols}


    %-------------------- 3.
    \item \textbf{\boldmath Resuelve el sistema con las condiciones iniciales que se indican:
    $$
    \left.
	\begin{array}{rcl}
	    x'&=&a(x+y)\\
	    y'&=&b(x+y)
	\end{array}
    \right\} \quad\textbf{con}\quad
    \begin{array}{rcl}
	x(0)&=&a\\
	y(0)&=&b.
    \end{array}
    $$
    \vspace{0.4cm}\\
	Respuesta.-}\; 

    %-------------------- 4.
    \item \textbf{\boldmath Dibuja el diagrama de fase y esboza la solución de las ecuaciones diferenciales:
    $$x'=x(x-3)(x+2).$$\\
	Respuesta.-}\; 

\end{enumerate}
