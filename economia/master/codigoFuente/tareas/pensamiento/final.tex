\textbf{Christian Limbert Paredes Aguilera}\
\textbf{Pensamiento Económico}\\\\
\begin{center}
\textbf{\Large Crecimiento y libertad económica: Un análisis cuantitativo}\\
\end{center}
\vspace{1cm}

    \subsection*{Introducción}
    Se realizará un análisis cuantitativo entre el crecimiento económico y la libertad económica.
    \subsection*{Análisis descriptivo del índice de libertad económica y sus componentes}
    \subsection*{La libertad económica en contexto}
    \subsection*{La libertad económica y el crecimiento económico}

    Los beneficios de la libertad económica.
    La ausencia de crecimiento económico implica la existencia continua de pobreza y penuria\\
    Una nueva línea de investigación sobre la libertad económica responde como lo hizo Adam Smith hace mucho tiempo. “Libertad económica” significa el grado en que existe una economía de mercado, donde los componentes centrales son el intercambio voluntario, la libre competencia y la protección de las personas y la propiedad. El objetivo es caracterizar la estructura institucional y las partes centrales de la política económica. La libertad económica puede constituir un factor explicativo del crecimiento y la distribución del ingreso. En el análisis econométrico, la libertad económica es, por tanto, una variable independiente. Sin embargo, la libertad económica también puede verse afectada por otras variables y, por lo tanto, constituir una variable dependiente, posiblemente influenciada por factores como la libertad política, la riqueza o la democracia. El intento más ambicioso de cuantificar la libertad económica es el Índice de Libertad Económica (EFI). \\

    El concepto de libertad económica\\
    La libertad económica es un compuesto que intenta caracterizar el grado en que una economía es una economía de mercado, es decir, el grado en el que implica la posibilidad de celebrar contratos voluntarios dentro del marco de un estado de derecho estable y predecible que respeta los contratos. Y protege la propiedad privada, con un grado limitado de intervencionismo en forma de propiedad del gobierno, regulaciones e impuestos. La libertad económica es distinta de la libertad política y  libertad civil.  El EFI es un medio de medir el grado de libertad económica al incluir treinta y siete componentes divididos en cinco grupos en un índice para los años 1970 (54 países), 1975 (83 países), 1980 (105 países), 1985 (111 países), 1990 (113 países), 1995 (123 países) y 2000 (123 países). Los cinco grupos son: (1) tamaño del gobierno: gastos, impuestos y empresas; (2) estructura legal y seguridad de los derechos de propiedad; (3) acceso a moneda sólida; (4) libertad para intercambiar con extranjeros; y (5) regulación del crédito, el trabajo y las empresas. Cada componente se mide de 0 (“sin libertad económica”) a 10 (“plena libertad económica”). El índice se calcula utilizando promedios aritméticos. Cabe señalar que los componentes del EFI, así como los esquemas de ponderación, han cambiado en las distintas ediciones que ha sido publicado. Por lo tanto, al comparar estudios, es necesario tener cuidado de aclarar qué ediciones se utilizan.\\

    La importancia de la libertad económica\\
    Es probable que la libertad económica sea un factor importante que explica el crecimiento económico por motivos puramente teóricos. Los incentivos que enfrentan los actores económicos (emprendedores, innovadores, financieros, industriales y otros) están determinados en gran parte por las instituciones existentes, las cuales, como señala Douglass C. North (1990), pueden ser ineficientes o eficientes. En la medida en que las instituciones estimulan acciones que contribuyen a la producción de un producto más valioso, contribuyen al crecimiento económico. Las instituciones que garantizan la libertad económica tienen la capacidad de proporcionar el tipo de incentivos que mejoran el crecimiento, por varias razones:\\

    Promueven un alto rendimiento de los esfuerzos productivos a través de bajos impuestos, un sistema legal independiente y la protección de la propiedad privada; permiten que el talento se asigne a donde genera el valor más alto; fomentan una economía dinámica, organizada experimentalmente en la que puede tener lugar una gran cantidad de pruebas y errores comerciales y en la que la competencia entre diferentes actores se produce porque las regulaciones y las empresas gubernamentales son pocas; facilitan la toma de decisiones predecible y racional a través de una tasa de inflación baja y estable; y promueven el flujo del comercio y la inversión de capital donde la satisfacción de preferencias y los rendimientos son los más altos. Aunque se puede esperar que ciertos tipos de cambio institucional tengan efectos de crecimiento claramente positivos al introducir el tipo de incentivos que acabamos de mencionar, las instituciones per se, en su lugar a lo largo del tiempo, pueden ejercer una influencia no solo en el nivel de riqueza sino también en las tasas de crecimiento. En igualdad de condiciones. En un período determinado, las instituciones establecidas establecen los incentivos económicos e influyen en lo que hacen los actores económicos. Suponemos que una libertad económica muy alta y estable permite que una economía dinámica funcione y crezca, aunque un aumento de la libertad económica desde un nivel bajo podría ejercer una influencia mucho más clara sobre la tasa de crecimiento durante un período determinado. Además, las altas tasas de crecimiento sostenidas implican en última instancia una gran riqueza, por lo que, a largo plazo, también se puede esperar que la libertad económica que aumenta el crecimiento aumente la riqueza acumulada. \\
    Si tenemos razones teóricas para esperar una relación positiva entre la libertad económica y el crecimiento económico, ¿la evidencia empírica confirma este efecto? Pues parece que si: no es difícil afirmar que es probable que la libertad económica tenga un efecto favorable sobre la prosperidad económica, por la sencilla razón de que los últimos cincuenta años de experiencia internacional confirman más o menos el hecho de que dondequiera que los gobiernos utilizaran más los mercados y comprometidos con políticas más abiertas en el comercio exterior y la inversión, de hecho con más libertad económica de diferentes tipos, sus países han tendido a prosperar. Por el contrario, aquellos países que se volvieron hacia adentro y tenían regulaciones extensas de todo tipo sobre la toma de decisiones económicas internas en materia de producción, inversión e innovación, son los países que realmente no lo han hecho demasiado bien.\\
    Un simple mapeo de Gwartney y Lawson (2002) apoya en gran medida esta posición. La quinta parte de los países que han tenido la libertad económica más alta han crecido considerablemente más rápido que otros países, mientras que la quinta parte de los países con la libertad económica más baja han tenido, de hecho, un crecimiento negativo. \\
    Varios estudios econométricos corroboran esta conclusión, con diferentes puntos fuertes y de diferentes formas. Los resultados deben interpretarse con el cuidado habitual. \\
    Los hallazgos de un efecto positivo del nivel inicial de libertad económica son generalmente más débiles que los que indican un efecto positivo de los aumentos en la libertad económica y, en varios casos, el efecto de nivel parece estadísticamente significativo solo si el cambio en la libertad económica también se incluye como variable. Algunas partes de la EFI pueden promover el crecimiento más que otras. \\
    Carlsson y Lundström (2002) establecen que de los siete grupos EFI (en la versión publicada en 2000), cuatro están relacionados de manera positiva y estadísticamente significativa con el crecimiento (estructura económica y uso de mercados, libertad de uso de monedas alternativas, estructura legal y seguridad de la economía). Propiedad y libertad de intercambio en los mercados de capitales), dos están relacionados de manera negativa y estadísticamente significativa con el crecimiento (el tamaño del gobierno y el intercambio internacional / libertad para comerciar con extranjeros), y uno no está relacionado de manera estadísticamente significativa con el crecimiento (política monetaria y precios estabilidad).\\
    Lo más sorprendente de estos resultados, tanto desde una perspectiva teórica como en comparación con otros resultados empíricos,  son las dos relaciones negativas detectadas. Implican que cuanto menor sea el tamaño del gobierno y mayor libertad para comerciar con extranjeros, más lenta será la tasa de crecimiento. \\
    Una dificultad con las mediciones agregadas de este tipo es que algunas empresas públicas pueden tener efectos positivos en el crecimiento, mientras que otras tienen efectos negativos. Por lo tanto, parece necesario realizar más estudios que consideren los componentes individuales antes de extraer conclusiones políticas detalladas, especialmente cuando se presentan conclusiones que están en desacuerdo con muchos estudios anteriores. Además, las empresas públicas por debajo y por encima de un cierto nivel pueden obstaculizar el crecimiento, aunque potencian el crecimiento en ese nivel medio. Es decir, la relación puede ser no lineal.\\
    Otros estudios analizan el crecimiento o el producto interno bruto (PIB) per cápita en función de la libertad económica o sus componentes. En general, los resultados son compatibles con los mencionados anteriormente, y aquí se presenta una selección de estos estudios. Hanke y Walters (1997) estudian la relación entre la libertad económica y el PIB per cápita y la encuentran significativa y positiva. Leschke (2000) muestra que, en particular, el marco dentro del cual funciona la economía de mercado y el grado de intervencionismo en el proceso político son de gran importancia para la riqueza de las naciones. De Haan y Siermann (1996, 1998) aclaran que el índice de libertad construido por Scully y Slottje (1991) está relacionado con el crecimiento, pero solo en algunos de los nueve esquemas de ponderación desarrollados. Claramente, la construcción de un índice necesita ser analizada. Goldsmith (1997) utiliza el EFI y muestra que los países en desarrollo que proteger mejor los derechos económicos tienden a crecer más rápido, tener un ingreso nacional promedio más alto y tener un mayor grado de bienestar humano. Wu y Davis (1999) investigan la relación entre la libertad y el crecimiento económico y político. Encuentran que la libertad económica es importante para el crecimiento y que un alto nivel de ingresos es importante para la libertad política. De Vanssay y Spindler (1994) utilizan una versión del índice de libertad económica de Scully-Slottje, que se incluye en un modelo de crecimiento de Soloviano, y encuentran una relación positiva entre este y el crecimiento económico. Está demostrado que los derechos positivos obstaculizan el crecimiento y que los derechos negativos lo potencian. De Vanssay y Spindler (1996) estudian cómo los diferentes factores constitucionales y la libertad económica (en la forma del índice de Scully-Slotje) afectan la convergencia económica, y encuentran que la libertad económica, de todas las variables que estudiaron, tiene el efecto más fuerte.\\
    Antes, es necesario tener cuidado al interpretar estudios empíricos, especialmente cuando faltan análisis de sensibilidad y cuando no se utilizan datos de panel. La relación causal entre las variables puede no estar claras. Por ejemplo, si se puede establecer una correlación entre la libertad económica y el crecimiento, ¿implica esto que la libertad económica provoca el crecimiento o es al revés? Sobre este tema, Gwartney, Lawson y Holcombe (1999) encuentran que el crecimiento económico no es capaz de predecir aumentos futuros de la libertad económica de manera significativa. Wu y Davis (1999) y Heckelman (2000) llegan a un resultado de causalidad similar. El último estudio utiliza el índice de libertad económica de Heritage Foundation / Wall Street Journal y encuentra que el nivel promedio de libertad económica precede al crecimiento. Farr, Lord y Wolfenbarger (1998) identifican la causalidad conjunta de la libertad económica y la riqueza económica, pero no analizan la relación causal entre la libertad económica y el crecimiento. La prueba más extensa de la relación causal entre la libertad económica y el crecimiento se encuentra en el trabajo más reciente de Dawson (de próxima publicación).\\
    Entre otras cosas, Dawson afirma que los estudios existentes son capaces de establecer una correlación entre la libertad económica y el crecimiento, pero no son capaces de establecer una causalidad. Usando una técnica de causalidad de Granger, encuentra que el nivel de libertad económica parece afectar el crecimiento, mientras que los aumentos en la libertad económica están determinados conjuntamente con el crecimiento. La complejidad de la relación se aclara en el estudio, con algunos componentes de EFI que causan crecimiento (en particular, el uso de mercados y derechos de propiedad), algunos componentes de EFI son causados por el crecimiento y algunos componentes de EFI se determinan conjuntamente con el crecimiento. \\
    Los resultados más importantes se resumen en el cuadro 3. No se han reportado resultados que muestren que la libertad económica obstaculice el crecimiento o que esté asociada con un PIB per cápita más bajo. Por el contrario, los resultados en general muestran que un aumento de la libertad económica ejerce una influencia positiva en el desarrollo de la riqueza económica.\\

    Igualdad de ingresos\\
    Incluso si se puede demostrar que la libertad económica contribuye al crecimiento económico, algunas personas pueden resistirse a los cambios de política que aumentan este tipo de libertad porque temen que tales cambios conlleven mayores diferencias de ingresos. Teóricamente, es una pregunta abierta cómo los ingresos disponibles de diferentes individuos y grupos se ven afectados por un aumento de la libertad económica. Por un lado, la libertad económica está relacionada negativamente con la igualdad de ingresos, en un sentido estático (es decir, si se observa el efecto parcial e inmediato de un cambio de política) y si la medida del ingreso son los ingresos disponibles (porque los impuestos y Se puede esperar que los gastos de bienestar generalmente asociados con una mayor libertad económica reduzcan la posición relativa de las personas de bajos ingresos). \\
    Por otro lado, los aumentos en la libertad económica afectan positivamente el crecimiento de los ingresos brutos, y si los grupos de bajos ingresos tienen una tasa de crecimiento más alta que otros como resultado de una mayor libertad económica, la distribución del ingreso puede hacerse más equitativa. Un simple mapeo de Gwartney y Lawson (2002) muestra que no parece existir una relación clara entre la libertad económica y la situación relativa de los más pobres. Tres estudios empíricos implican que bajo ciertas condiciones la relación es en realidad estadísticamente positiva. Berggren (1999) encuentra que cuanto más aumentó la libertad económica en un país entre 1975 y 1985, mayor fue el grado de igualdad de ingresos de ese país alrededor de 1985. En particular, este resultado es válido para los países en desarrollo y para el momento en que se produjeron los cambios de política. \\
    La igualdad se mide como coeficientes de Gini y como comparaciones entre la participación en el ingreso o el consumo de las personas con ingresos bajos y altos. Al mismo tiempo, el nivel de libertad económica en 1985 parece estar relacionado negativamente con la igualdad de ingresos, lo que probablemente sea un efecto de la redistribución reducida. \\
    Grubel (1998) da la vuelta al tema y estudia cómo la igualdad de ingresos afecta el PIB per cápita, el crecimiento económico y la libertad económica en diecisiete años países con un PIB per cápita superior a $\$$ 17.000. Los resultados sugieren que una mayor igualdad de ingresos está relacionada con un menor PIB per cápita, un menor crecimiento y una menor libertad económica. Scully (2002) estima un modelo estructural y modelos de forma reducida y muestra que la libertad económica es beneficiosa tanto para el crecimiento económico como para la igualdad porque tiene un efecto negativo significativo sobre los coeficientes de Gini. Además, una mayor igualdad reduce el crecimiento, pero solo en una pequeña cantidad.


\begin{enumerate}

    \item[Área 1.] \textbf{Tamaño del gobierno}.- A medida que aumentan el gasto público, los impuestos y el tamaño de las empresas controladas por el gobierno, la toma de decisiones del gobierno sustituye a la elección individual y se reduce la libertad económica.

    \item[Área 2.] \textbf{Sistema legal y derechos de propiedad}.- La protección de las personas y su propiedad legítimamente adquirida es un elemento central tanto de la libertad económica como de la sociedad civil. De hecho, es la función más importante del gobierno.

    \item[Área 3.] \textbf{La inflación monetaria sólida}.- erosiona el valor de los salarios y ahorros ganados legítimamente. Por lo tanto, el dinero sólido es esencial para proteger los derechos de propiedad. Cuando la inflación no solo es alta sino también volátil, se vuelve difícil para las personas planificar el futuro y, por lo tanto, utilizar la libertad económica de manera eficaz.

    \item[Área 4.] \textbf{Libertad para comerciar internacionalmente}.- La libertad para intercambiar —en su sentido más amplio, comprar, vender, hacer contratos, etc.— es esencial para la libertad económica, que se reduce cuando la libertad de intercambiar no incluye empresas e individuos de otras naciones.

    \item[Área 5.] \textbf{Regulación}.- Los gobiernos no solo usan una serie de herramientas para limitar el derecho a intercambiar internacionalmente, sino que también pueden imponer regulaciones onerosas que limitan el derecho a intercambiar, obtener crédito, contratar o trabajar para quien usted desee u operar libremente su negocio.

\end{enumerate}
