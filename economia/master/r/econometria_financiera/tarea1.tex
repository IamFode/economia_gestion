% Options for packages loaded elsewhere
\PassOptionsToPackage{unicode}{hyperref}
\PassOptionsToPackage{hyphens}{url}
%
\documentclass[
]{article}
\usepackage{amsmath,amssymb}
\usepackage{iftex}
\ifPDFTeX
  \usepackage[T1]{fontenc}
  \usepackage[utf8]{inputenc}
  \usepackage{textcomp} % provide euro and other symbols
\else % if luatex or xetex
  \usepackage{unicode-math} % this also loads fontspec
  \defaultfontfeatures{Scale=MatchLowercase}
  \defaultfontfeatures[\rmfamily]{Ligatures=TeX,Scale=1}
\fi
\usepackage{lmodern}
\ifPDFTeX\else
  % xetex/luatex font selection
\fi
% Use upquote if available, for straight quotes in verbatim environments
\IfFileExists{upquote.sty}{\usepackage{upquote}}{}
\IfFileExists{microtype.sty}{% use microtype if available
  \usepackage[]{microtype}
  \UseMicrotypeSet[protrusion]{basicmath} % disable protrusion for tt fonts
}{}
\makeatletter
\@ifundefined{KOMAClassName}{% if non-KOMA class
  \IfFileExists{parskip.sty}{%
    \usepackage{parskip}
  }{% else
    \setlength{\parindent}{0pt}
    \setlength{\parskip}{6pt plus 2pt minus 1pt}}
}{% if KOMA class
  \KOMAoptions{parskip=half}}
\makeatother
\usepackage{xcolor}
\usepackage[margin=1in]{geometry}
\usepackage{graphicx}
\makeatletter
\def\maxwidth{\ifdim\Gin@nat@width>\linewidth\linewidth\else\Gin@nat@width\fi}
\def\maxheight{\ifdim\Gin@nat@height>\textheight\textheight\else\Gin@nat@height\fi}
\makeatother
% Scale images if necessary, so that they will not overflow the page
% margins by default, and it is still possible to overwrite the defaults
% using explicit options in \includegraphics[width, height, ...]{}
\setkeys{Gin}{width=\maxwidth,height=\maxheight,keepaspectratio}
% Set default figure placement to htbp
\makeatletter
\def\fps@figure{htbp}
\makeatother
\setlength{\emergencystretch}{3em} % prevent overfull lines
\providecommand{\tightlist}{%
  \setlength{\itemsep}{0pt}\setlength{\parskip}{0pt}}
\setcounter{secnumdepth}{-\maxdimen} % remove section numbering
\ifLuaTeX
  \usepackage{selnolig}  % disable illegal ligatures
\fi
\IfFileExists{bookmark.sty}{\usepackage{bookmark}}{\usepackage{hyperref}}
\IfFileExists{xurl.sty}{\usepackage{xurl}}{} % add URL line breaks if available
\urlstyle{same}
\hypersetup{
  pdftitle={Tarea 1},
  pdfauthor={Christian Limbert Paredes Aguilera},
  hidelinks,
  pdfcreator={LaTeX via pandoc}}

\title{Tarea 1}
\author{Christian Limbert Paredes Aguilera}
\date{2023-05-26}

\begin{document}
\maketitle

𝑝𝑝
\[\qquad w_{3t}^* = \dfrac{\mu_{3,t+1/t}/\sigma_{3,t+1/t}^2}{\dfrac{\mu_{1,t+1/t}^2}{\sigma_{1,t+1/t}}+\dfrac{\mu_{2,t+1/t}^2}{\sigma^2_{2,t+1/t}}+\dfrac{\mu_{3,t+1/t}^2}{\sigma^2_{3,t+1/t}}}\overline{\mu}_p,\]

\[\qquad w_{2t}^* = \dfrac{\mu_{2,t+1/t}/\sigma_{2,t+1/t}^2}{\dfrac{\mu_{1,t+1/t}^2}{\sigma_{1,t+1/t}}+\dfrac{\mu_{2,t+1/t}^2}{\sigma^2_{2,t+1/t}}+\dfrac{\mu_{3,t+1/t}^2}{\sigma^2_{3,t+1/t}}}\overline{\mu}_p,\qquad w_{3t}^* = \dfrac{\mu_{3,t+1/t}/\sigma_{3,t+1/t}^2}{\dfrac{\mu_{1,t+1/t}^2}{\sigma_{1,t+1/t}}+\dfrac{\mu_{2,t+1/t}^2}{\sigma^2_{2,t+1/t}}+\dfrac{\mu_{3,t+1/t}^2}{\sigma^2_{3,t+1/t}}}\overline{\mu}_p,\]

Elegimos 3 stock, Apple (AAPL) del sector tecnológico, NLY del sector
financiero(FCX) y Exxon Mobil Corporation XOM del sector del petróleo
para formar una cartera diversificada. La muestra consiste en precios
diarios desde el 2 de enero de 2008 hasta el 31 de mayo de 2013, para un
total de 1363 observaciones. Nos gustaría asignar el capital óptimamente
de tal modo que minimicemos el riesgo y obtengamos un rendimiento diario
medio de \(\overline{\mu}_p=0.15\) por ciento. Asumiendo que los stocks
en la cartera están incorrelacionados (sino a la varianza de la cartera
debería incorporarse los términos de covarianza), obtener las
ponderaciones óptimas siguiendo el problema de optimización planteado
anteriormente.

\begin{enumerate}
\def\labelenumi{\arabic{enumi}.}
\item
  Para ello en primer lugar, encontrar el mejor modelo para las series
  temporales de rendimientos de los 3 stocks tanto para la media
  condicional como para su varianza condicional.
\item
  Basándose en esos modelos obtener la media condicional y la varianza
  condicional y a partir de ellas calcular las ponderaciones óptimas
  variando en el tiempo.
\item
  Hacer gráficos de evolución temporal de esas ponderaciones e
  interpretarlos.
\item
  Realizar una tabla de los estadísticos descriptivos de las
  ponderaciones óptimas y comentarlos.
\end{enumerate}

\end{document}
