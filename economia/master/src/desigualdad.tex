\chapter{Instrumentos para medir la distribución y desigualdad de la renta}

\section{Introducción}
El bienestar económico y social están directamente relacionados con la cantidad de bienes y servicios generados por un país y que hacen mejorar su calidad de vida. Por tanto, la información que ofrecen distintas variables macroeconómicas, como el PIB o la RN, es necesario complementarla con diversos criterios de interrelación con otras variables.\\
Un primer indicador es el usual ratio de RN/población, es decir, el concepto de media o renta por habitante. En general, la renta per cápita está positivamente correlacionada con la calidad de vida de los habitantes de un país. No obstante, en la práctica esta variable presenta serias deficiencias, pues lleva implícita una distribución equitativa o media de la RN (u otras macromagnitudes) entre los habitantes de una región o país. Por tanto, no ofrece información sobre el modo en que dicha Renta Nacional (o PIB) se reparte entre los diferentes grupos de población, fundamentalmente en el caso de la distribución personal; es decir, no es un buen indicador de bienestar social. - Por ello, es necesario utilizar diversos criterios de distribución de la renta. La distribución de la renta recoge el reparto de ingresos entre distintos tramos de población, ya sean estos grupos clasificados por sus niveles de ingresos, regiones, sectores económicos o factores de producción.\\
En función de estos grupos, se diferencias los siguientes criterios:

\begin{itemize}
    \item Distribución personal (grupos de ingresos).
    \item Distribución espacial (regiones).
    \item Distribución sectorial (ranas del PIB).
    \item Distribución funciones (factores productivos).
\end{itemize}
Estos criterios recogen aspectos tan fundamentales como: el grado de concentración de la renta en diferentes grupos sociales, en ámbitos espaciales o regionales, la sectorización del origen del producto nacional y su distribución entre los principales factores que intervienen en la producción (trabajo y capital).\\
Nos centraremos en los criterios de distribución personal (grupos de ingresos), distribución espacial y de distribución funcional (factores de producción) de la renta.\\
Cuando la distribución de renta presenta un relativo alto grado de concentración en pocos grupos se habla de desigualdad o inequidad económica. La desigualdad económica mide, por tanto, las diferencias de renta entre los habitantes de un país, es decir, las disparidades de ingresos entre los distintos grupos de población o factores productivos. Para valorar el grado de concentración/distribución de la renta a nivel personal, resulta necesaria la utilización de los denominados índices de concentración y de desigualdad. De los cuales, los más utilizados son: 
\begin{itemize}
    \item Indice de Gini. 
    \item Representación gráfica de Lorenz. 
    \item Ratio S80/S20.
\end{itemize}
Por último, existen otros factores que influyen en la calidad de vida de una sociedad y, por tanto, en su bienestar: nivel de empleo, índice de pobreza y crecimiento económico sostenible desde un punto de vista medioambiental.

\section{Indice de Gini}
El índice de concentración de Gini (IG) recoge el grado de concentración o de distribución de la renta entre los distintos grupos de ingresos en los que se clasifica una sociedad (por decilas – percentiles o ventilas – quintiles - de ingresos). El IG viene determinado por la siguiente expresión:
\begin{equation}
    IG = \bigg\|\dfrac{\sum\limits_{i=1}^{n-1}\left(P_i-Q_i\right)}{\sum\limits_{i=1}^{n-1}P_i}\bigg\| 
\end{equation}

siendo $n$ el número de tramos o número de grupos de ingresos, $Q_i$ los porcentajes acumulados de renta, ordenados de menor a mayor nivel de renta, y $P_i$ los porcentajes acumulados de población para cada uno de los grupos o tramos de $Q_i$ definidos. En general, el valor del IG oscila entre el valor mínimo de 0 y el máximo de 1. Nunca alcanzará estos dos valores extremos, pero en la medida en que se aproxime a la unidad nos indicará una situación de elevada concentración de la renta en poca población, mientras que, si los resultados se acercan al valor nulo, ello será reflejo de una superior equidistribución de la misma. En el caso de los PD, en los cuales suelen existir políticas de distribución de la renta (políticas fiscales y políticas sociales) en mayor o menor grado, se suele considerar un valor de referencia de $0.3$. Valores superiores indican concentración de la renta en pocos grupos de ingresos, mientras que valores iguales o inferiores a $0.3$ indican una renta más distribuida por grupos de ingresos.
