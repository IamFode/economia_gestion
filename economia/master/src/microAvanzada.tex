\chapter{Teoría del consumidor: Preferencias y dualidad}

\section{Problema del consumidor}
El problema del consumidor tiene tres ingredientes:

\begin{enumerate}
    \item Que cestas existen?
    \item Qué cestas puede adquirir un consumidor?
    \item Cuáles son los gustos del consumidor?
\end{enumerate}

Decimos que hay $i=1,2,\ldots,l$ consumidores, e $I=1,2,\ldots,m$ mercancías.\\

\begin{center}
    \textit{La solución del problema de elección del consumidor consiste en elegir de entre todas las cestas que existen y que podemos adquirir aquella cesta que le aporte más bienestar.}
\end{center}

La solución estará representada por una función de demanda, que nos dirá cuanto de esos bienes que consumiremos para \textbf{maximizar nuestro bienestar}.\\



\section{Dos supuestos implícitos de la teoría del consumidor}

\begin{enumerate}
    \item Cada \textbf{agente es aislado o independiente} de los demás. No hay interacción entre los agentes.
    \item El objetivo de esta teoría no es descubrir precios. Si no cual es el comportamiento del consumidor con \textbf{precios dados o exógenos que los aceptamos}.
\end{enumerate}

\section{Conjunto de consumo}

\paragraph{Mercancía} Es un bien o servicio definido por sus características físicas, su localización o estado de naturaleza.\\
\paragraph{Conjunto de consumo} Es un conjunto de planes de consumo que existen para un consumidor $i$, representado por el conjunto $\mathcal{X}^i$.
\paragraph{Plan de consumo} $i:\textbf{X}^i=\left(x_1^i,\ldots,X_L^i\right)\in \mathcal{X}^i$ donde $x_l^i$ es la cantidad de mercadería $l$ consumida por el consumidor $i$.

\subsection{Propiedades del conjunto de consumo \boldmath$\mathcal{X}^i$ (Que cestas existen?))}

\begin{itemize}
    \item \textbf{No vacío}: $\mathcal{X}^i\neq \emptyset$;
    \item Los bienes son una cantidad no negativa: $x_l^i\geq 0$;
    \item No consumir nada es una opción de consumo: $\textbf{0}^i\in \mathcal{X}^i$;
    \item Es un conjunto cerrado;
    \item Es un conjunto convexo (Cualquier cesta de bienes pertenece al conjunto de consumo).
\end{itemize}

\subsection{Conjunto presupuestario (Qué cestas puede adquirir un consumidor?)}
\paragraph{Dotaciones iniciales} Es un vector $\overline{\omega}^i = \left(\overline{\omega}_1^i,\ldots,\overline{\omega}_L^i\right)\in \mathbb{R}^L_+$ donde $\overline{\omega}^i_l$ es la dotación del bien $l$ adquirida por el consumidor $i$..
\paragraph{Precios dados} $\overline{p}= \left(\overline{p}_1,\ldots,\overline{p}_L\right)\in \mathbb{R}^L_+$ donde $\overline{p}_l$ es el precio del bien $l$.
\paragraph{Riqueza o renta} $M^i\left(\overline{\textbf{p}}\right)=\left[\overline{\textbf{p}} \overline{\omega}^i=\displaystyle\sum_{l=1}^L \overline{\textbf{p}_l} \overline{\omega}_l^i\right]$
\paragraph{Gasto del plan de consumo \boldmath$\textbf{x}^i$} $\overline{\textbf{p}}\textbf{x}^i=\displaystyle\sum_{l=1}^L \overline{\textbf{p}_l} x_l^i$
\paragraph{Conjunto presupuestario \boldmath $\beta^i$} De aquellos de los que existen y que podemos adquirir, estarán representados por:
$$\beta^i\left(\overline{\textbf{p}}\right)=\hat{\beta}^i\left[\overline{\textbf{p}},M^i\left(\overline{\textbf{p}}\right)\right]=\left\{\textbf{x}^i\in \mathcal{X}^i:\overline{\textbf{p}}\textbf{x}^i\leq M^i\left(\overline{\textbf{p}}\right)\left[=\overline{\textbf{p}}\overline{\omega}^i\right]\right\}$$
Es decir, de todas cestas que están en el conjunto de consumo, aquellas en las que su precio valga por lo mucho mi renta.
\paragraph{NOTA: Relación marginal de sustitución} $\text{RMI}(\textbf{x})=-\dfrac{\triangle x_2}{\triangle x_1}=\dfrac{\overline{p}_1}{\overline{p}_2}$. Es lo que nos estamos perdiendo de un bien para incrementar en una unidad del otro bien.

\subsection{Preferencias del consumidor (Cuáles son los gustos del consumidor?)}
\begin{center}
    \textit{Yo prefiero esto antes que aquello.}
\end{center}
Formalmente es una relación binaria, con el criterio de \textbf{ser como mínimo tan preferido como}: $\succeq^i$. 

\begin{axioma}[Completitud]
    Para todo $\textbf{x}^i,\textbf{y}^i\in \mathcal{X}^i$ se cumple que $\textbf{x}^i\succeq^i \textbf{y}^i$ o bien $\textbf{y}^i\succeq^i \textbf{x}^i$
\end{axioma}

\begin{axioma}[Reflexividad]
    Par todo $\textbf{x}^i\in \mathcal{X}^i$ se cumple que $\textbf{x}^i\succeq^i \textbf{x}^i$. Por lo tanto $\textbf{x}^i\sim^i \textbf{x}^i$.
\end{axioma}

\begin{axioma}[Transitividad]
    Para todo $\textbf{x}^i,\textbf{y}^i,\textbf{z}^i\in \mathcal{X}^i$ se cumple que si $\textbf{x}^i\succeq^i \textbf{y}^i$ e $\textbf{y}^i\succeq^i \textbf{z}^i$ entonces $\textbf{x}^i\succeq^i \textbf{z}^i$.
\end{axioma}

Estos axiomas se llaman \textbf{preferencias racionales}.

\subsection{Conjunto de planes de consumo}

\begin{itemize}
    \item Por lo menos mejores que ese: $\mathcal{MI}^i\left(\textbf{x}^i_0\right) \equiv \left\{\textbf{x}^i\in \mathcal{X}^i:\textbf{x}^i\succeq^i \textbf{x}^i_0\right\}$;
    \item Peores o iguales que ese: $\mathcal{PI}^i\left(\textbf{x}^i_0\right) \equiv \left\{\textbf{x}^i\in \mathcal{X}^i:\textbf{x}_0^i\succeq^i \textbf{x}^i\right\}$;
    \item Estrictamente mejores: $\mathcal{M}^i\left(\textbf{x}^i_0\right) \equiv \left\{\textbf{x}^i\in \mathcal{X}^i:\textbf{x}^i\succ^i \textbf{x}^i_0\right\}$;
    \item Estrictamente peores: $\mathcal{P}^i\left(\textbf{x}^i_0\right) \equiv \left\{\textbf{x}^i\in \mathcal{X}^i:\textbf{x}_0^i\succ^i \textbf{x}^i\right\}$;
    \item Indiferentes: $\mathcal{I}\left(\textbf{x}_0^i\right)\equiv \left\{\textbf{x}^i\in \mathcal{X}^i:\textbf{x}_0^i\sim^i \textbf{x}^i\right\}$.
\end{itemize}

Los conjuntos $\mathcal{MI}^i\left(\textbf{x}^i\right)$ y $\mathcal{PI}^i\left(\textbf{x}^i\right)$ son conjuntos cerrados. Y $\mathcal{M}^i\left(\textbf{x}^i\right)$ y $\mathcal{P}^i\left(\textbf{x}^i\right)$ son conjuntos abiertos.

\begin{axioma}[No saciabilidad local]
    Para todo $\textbf{x}_0^i\in \mathcal{X}^i$, y para todo número real $\epsilon>0$. Existe una cesta $\textbf{x}^i\in B\left(\textbf{x}^i_0,\epsilon\right)$ tal que $\textbf{x}^i\succ^i \textbf{x}^i_0$.
\end{axioma}

\begin{axioma}[Monotonicidad]
    Para todo par de planes de consumo $\textbf{x}^i,\textbf{y}^i\in \mathcal{X}^i$, tal que $x^i\geq y^i$. Entonces, $\textbf{x}^i\succeq^i \textbf{y}^i$ (Siempre vamos a querer más que menos).
\end{axioma}

\begin{axioma}[Convexidad]
    Para todo par de planes de consumo $\textbf{x}^i,\textbf{y}^i\in \mathcal{X}^i$, y $\lambda\in [0,1]$ se cumple que $\textbf{z}^i_\lambda = \lambda\textbf{x}^i+(1-\lambda)\textbf{y}^i$. Entonces, $\textbf{z}_\lambda^i\succeq^i \textbf{x}^i$ y $\textbf{z}^i \succeq^i \textbf{y}^i$. 
\end{axioma}

\begin{axioma}[Convexidad estricta]
    Para todo par de planes de consumo $\textbf{x}^i,\textbf{y}^i \textbf{z}^i \in \mathcal{X}^i$, con $\textbf{z}^i\neq \textbf{y}^i$, tal que $\textbf{z}^i\succeq^i \textbf{x}^i$ e $\textbf{y}^i \succeq^i \textbf{x}^i$. Entonces,  $\lambda\textbf{z}^i+(1-\lambda)\textbf{y}^i \succeq^i \textbf{x}^i$ para todo $\lambda\in [0,1]$.
\end{axioma}

$\mathcal{M}^i\left(\textbf{x}^i\right)$ y $\mathcal{MI}^i\left(\textbf{x}^i\right)$ son convexos.


\section{Función de utilidad}
Para representar los gustos debemos crear una función de utilidad con valores numéricos que solo nos indicarán si prefiero una cesta a otra, no cuanto la prefiero.

Sea $u^i:\mathcal{X}^i \equiv \mathbb{R}^L_+ \to \mathbb{R}$ la función de utilidad de las preferencias del consumidor $i$. Entonces, para todo $\textbf{x}^i,\textbf{y}^i\in \mathcal{X}^i$ se verifica que 
\begin{enumerate}[\bfseries i)]
    \item $\textbf{x}^i\succ^i\textbf{y}^i \iff u^i\left(\textbf{x}^i\right)>u^i\left(\textbf{y}^i\right)$;
    \item $\textbf{x}^i\sim^i\textbf{y}^i \iff u^i\left(\textbf{x}^i\right)=u^i\left(\textbf{y}^i\right)$.
 \end{enumerate}


\begin{teo}
    Si la relación binaria $\succeq$ es completa, transitiva, continua y estrictamente monótona. Existe, una función real y continua, $u:\mathbb{R}^n_+\to \mathbb{R}$ el cual representa a $\succeq$.\\

	Demostración.-\; Dado que sólo menciona la existencia podemos afirmar que podría haber por lo menos una función que represente la relación de preferencia. Por lo que si podemos imaginar una sola función que sea continua y que represente las preferencias dadas, habremos demostrado el teorema.\\

	Sea un vector de unos $\textbf{e}\equiv(1,\ldots,1)\in \mathbb{R}^n_+$ y sea $u:\mathbb{R}^n_+\to \mathbb{R}$ tal que,
	\begin{equation}
	    u(\textbf{x})\textbf{e}\sim \textbf{x}.
	\end{equation}
	Es decir, tomamos cualquier $\textbf{x}$ en el dominio de $\mathbb{R}^n_+$ y asignamos el número $u(\textbf{x})$ de manera que el conjunto $u(\textbf{x})e$, con $u(\textbf{x})$ unidades de cada bien sea indiferente a $\textbf{x}$.\\

	Ahora, nos preguntamos ¿si existe siempre un número $u(\textbf{x})$ que satisfaga 1.1? y ¿está determinada de forma única, de modo que $u(x)$ sea una función bien definida?. \\

	Para ver la primera cuestión; sea $\textbf{x}\in \mathbb{R}^n_+$.  Consideremos los dos conjuntos:
	$$
	\begin{array}{rcl}
	    A &\equiv& \left\{t\geq 0: t\textbf{e}\succeq \textbf{x}\right\}\\\\
	    B &\equiv& \left\{t\geq 0: t\textbf{e}\preceq \textbf{x}\right\}
	\end{array}
	$$
	Notemos que si $t^*\in A\cap B$. Entonces, $t^*\textbf{e}\sim \textbf{x}$. Por lo que, $u(\textbf{x})=t^*$ podría satisfacer 1.1. Para ello, demostraremos que $A\cap B$ es no vacío. \\

	Sabemos que la continuidad de $\succeq$ para $A$ y $B$ es cerrada en $\mathbb{R}_+$. También por la Monotonicidad estricta, $t\in A$ implica que $t'\in A$ para todo $t'\geq t$. Como consecuencia, $A$ es un intervalo $[\underline{t},\infty)$. Lo propio para $B\in \mathbb{R}_+$ que es un intervalo con dominio en $[0,\overline{t}]$.\\

	Luego, para cualquier $t\geq 0$, $\succeq$ implica que $t\textbf{e}\succeq \textbf{x}$ o $t\textbf{e}\preceq \textbf{x}$. Es decir, $t\in A\cup B$. Esto significa que $\mathbb{R}_+=A\cup B=[0,\overline{t}]\cup [\underline{t},\infty]$. Concluimos que $\underline{t}\leq \overline{t}$, de modo que $A\cap B\neq \emptyset$.\\

	Respondamos la segunda cuestión. Debemos demostrar que sólo hay un número $t\geq 0$ tal que $t\textbf{e}\sim \textbf{x}$. Supongamos que existe $t_1\textbf{e}\sim \textbf{x}$ y $t_2\textbf{e}\sim\textbf{x}$, por lo transitividad de $\sim$ se tiene $t_1\textbf{e}\sim t_2\textbf{e}$. Así, por la Monotonicidad estricta $t_1=t_2$.\\

	Concluimos que para cada $\textbf{x}\in \mathbb{R}_+^n$ hay exactamente un número $u(\textbf{x})$, tal que se satisface 1.1. Ahora, demostraremos que esta función representa las preferencias.\\

	Consideremos dos cestas $\textbf{x}^1$ y $\textbf{x}^2$. Consideremos también que sus números de utilidad asociados a $u(\textbf{x}^1)$ y $u(\textbf{x}^2)$. Lo que por definición satisface $u(\textbf{x}^1)\textbf{e}\sim\textbf{x}^1$ y $u(\textbf{x}^2)\textbf{e}\sim\textbf{x}^2$. 
	$$
	\begin{array}{rcl}
	    \textbf{x}^1 \succeq \textbf{x}^2 &\iff& u(\textbf{x}^1)\textbf{e}\sim \textbf{x}^1\succeq \textbf{x}^2 \sim u(\textbf{x}^2)\textbf{e}\\\\
					      &\iff& u(\textbf{x}^1)\succeq u(\textbf{x}^2)\\\\
					      &\iff& u(\textbf{x}^1)\geq u(\textbf{x}^2).
	\end{array}
	$$

	Sólo queda demostrar que la función de utilidad $u:\mathbb{R}_+^n\to \mathbb{R}$ es continua. Basta demostrar que la imagen inversa sobre $u$ de cada bola abierta en $\mathbb{R}$ es abierta en $\mathbb{R}_+^n$. Ya que, la bola abierta en $\mathbb{R}$, son intervalos abiertos. Entonces, equivale a demostrar que $u^{-1}\left[(a,b)\right]$ es abierto en $\mathbb{R}_+^n$ para todo $a<b$. Por lo que,
	$$
	\begin{array}{rcl}
	    u^{-1}\left[(a,b)\right] &=& \left\{x\in \mathbb{R}_+^n | a<u(\textbf{x}) < b\right\}\\\\
				     &=& \left\{x\in \mathbb{R}_+^n | a\textbf{e}\prec u(\textbf{x})\textbf{e} \prec b\textbf{e}\right\}\\\\
				     &=& \left\{x\in \mathbb{R}_+^n | a\textbf{e}\prec \textbf{x} \prec b\textbf{e}\right\}.
	\end{array}
	$$
	Es decir,
	\begin{equation}
	    e^{-1}\left[(a,b)\right]=\;\; \succ (a\textbf{e}) \bigcap \prec (b\textbf{e}).
	\end{equation}
	Sabemos que por la continuidad de $\succeq$, los conjuntos $\succeq (a\textbf{e})$ y $\prec (b\textbf{e})$ son cerrados en $X=\mathbb{R}_+^n$. En consecuencia, los conjuntos de 1.2 al ser complementos de estos conjuntos cerrados, son abiertos en $\mathbb{R}_+^n$. Por lo tanto, $u^{-1}\left[(a,b)\right]$ siendo la intersección de dos conjuntos abiertos en $\mathbb{R}^n_+$ es abierto en $\mathbb{R}_+^n$.
\end{teo}


\chapter{Teoría del consumidor: Decisiones y dualidad}

La solución al problema del consumidor viene dado por:

$$
\left[P^i\right]
\left\{
    \begin{array}{rcl}
	\max_{x_1^i,x_2^i,\ldots,x_L^i}\in \mathbb{R}^L & u\left(x_1^i,x_2^i,\ldots,x_L^i\right) & :\textbf{S3}\\\\
	\text{s.a.} & \textbf{x}^i \in \beta^i(\textbf{p}\left[=\hat{\beta}^i \left(\textbf{p},M^i(\textbf{p})\right)\right] & :\textbf{S1,S2}\\\\
	\text{dados} & \left(\overline{p}_1,\overline{p}_2,\ldots,\overline{p}_L\right), \overline{M}^i \left(=M^i(\overline{p})\right) & :\textbf{S2}
    \end{array}
\right.
$$

Nos preguntamos:

\begin{enumerate}
    \item ¿Por qué hay solución?
    \item ¿Por qué sólo hay una?
    \item ¿por qué gastamos toda la renta?
\end{enumerate}

\subsection{Existencia de solución}

\begin{itemize}
    \item Porque el conjunto de consumo $\beta^i\left(\overline{p},\overline{M}^i\right)$ es compacto (cerrado y acotado), y
    \item porque las preferencias son continuas.
\end{itemize}

Esto se verifica con

\begin{teo}[Teorema de Weierstrass] Si una función continua es un compacto. Entonces, dentro de ese compacto alcanzará al máximo y al mínimo.
\end{teo}

\subsection{Unicidad de la solución}
\begin{itemize}
    \item Porque un conjunto de consumo $\beta^i\left(\overline{p},\overline{M}^i\right)$ es compacto, no vacío y convexo, y
    \item porque las preferencias son continuas y estrictamente convexas.
\end{itemize}

Es decir, si existieran dos soluciones igual de buenas. Entonces, la combinación convexa de ambas sería mejor que cada una de ellas. Lo que contradice la hipótesis de que ambas son soluciones en un conjunto convexo.

\subsection{Gasto total de la renta}
\begin{itemize}
    \item Porque los precios son estrictamente positivos, y
    \item porque la no saciabilidad local, y monotonicidad de las preferencias. 
\end{itemize}

Es decir,
$$\textbf{x}^{i*}\in \mathcal{F}r \beta^i\left(\overline{p},\overline{M}^i\right) \textbf{ y }  \overline{px}^* = \overline{M}^i.$$
En otras palabras,
El problema del consumidor $\textbf{x}^{i*}$ cae en la frontera del conjunto presupuestario $\mathcal{F}r$ (la solución), y lo que vale esa solución $\overline{px}^*$ es toda la renta $\overline{M}^i$. \\

Cómo más es mejor (monotonía), y no hay saciedad local. Entonces, la solución debe estar en la frontera del conjunto presupuestario. 

\paragraph{Observaciones:} 

\begin{itemize}
    \item La función de utilidad es diferenciable.
    \item La solución al problema $\left[P^i\right]$,
	$$
	\begin{array}{rrcl}
	    d^i: & \mathbb{R}_{++}^L & \to & \mathcal{X}^i\equiv\mathbb{R}^L_+\\\\
		 & \left(\textbf{p},M^i\right) & \mapsto & d^i\left(\textbf{p},M^i\right) = \textbf{x}^{i*}.
	\end{array}
	$$
    se denomina \textbf{función de demanda normal} o función de demanda marshalliana.
    \item Una función de demanda es continua.
    \item Una función de demanda es homogénea de grado cero.
	Sea $f(x,y)$ una función homogénea de grado $k$. Si $k=0$.  Entonces,
	$$f(\lambda x, \lambda y) = \lambda^k f(x,y) = \lambda^0 f(x,y)= f(x,y).$$
\end{itemize}

\section{Curva de demanda y curva de Engel}
La \textbf{curva de demanda} son las cantidad de un bien que estamos dispuestos a comprar para cada nivel de precios, manteniendo constante las demás variables.
$$x_1^* = D(p_1)=d_1(\overline{p}_1,\overline{p}_2,\ldots,\overline{p}_L,\overline{M}).$$

La \textbf{curva de Engel} son las cantidades de un bien que estamos dispuestos a comprar para cada nivel de renta, manteniendo constante las demás variables.
$$x_1^* = E(M)=d_1(\overline{p}_1,\overline{p}_2,\ldots,\overline{p}_L,M).$$

\section{Tipos de bienes}

Si hacemos variar el precio $p_1$,
$$x_1^* = D(p_1)=d_1(p_1,\overline{p}_2,\ldots,\overline{p}_L,\overline{M}).$$
Entonces,

$$\dfrac{\triangle \hat{d}_1^i\left(\textbf{p},M^i\right)}{\triangle p_1} < 0 \qquad \left(\dfrac{\partial d_1^i\left(\textbf{p},M^i\right)}{\partial p_1}<0\right)$$

Se llamará un \textbf{bien ordinario}. Es decir, si sube el precio del bien 1, la cantidad demandada del bien 1 disminuye.\\

Y si,

$$\dfrac{\triangle \hat{d}_1^i\left(\textbf{p},M^i\right)}{\triangle p_1} > 0 \qquad \left(\dfrac{\partial d_1^i\left(\textbf{p},M^i\right)}{\partial p_1}>0\right)$$

Se llamará un \textbf{bien Giffen}. Es decir, si sube el precio del bien 1, la cantidad demandada del bien 1 aumenta (compro más).\\

\subsection{Tipo de bienes según la curva de Engel}
Si hacemos variar la renta $M$,
$$x_1^* = E(M)=d_1(\overline{p}_1,\overline{p}_2,\ldots,\overline{p}_L,M).$$
Entonces, 
$$\dfrac{\triangle E_1\left(\textbf{p},M^i\right)}{\triangle M} > 0 \qquad \left(\dfrac{\partial E_1\left(\textbf{p},M^i\right)}{\partial M}>0\right)$$
Se llamará un \textbf{bien normal}. Es decir, si sube la renta, la cantidad demandada del bien 1 aumenta.\\
Y si,
$$\dfrac{\triangle E_1\left(\textbf{p},M^i\right)}{\triangle M} < 0 \qquad \left(\dfrac{\partial E_1\left(\textbf{p},M^i\right)}{\partial M}<0\right)$$
Se llamará un \textbf{bien inferior}. Es decir, si sube la renta, la cantidad demandada del bien 1 disminuye (compro menos).\\

Los \textbf{bienes Veblen} son los que si sube la renta aumenta la cantidad demandada. Es decir, son bienes normales con pendiente positiva.\\

\section{Función indirecta de utilidad}

La \textbf{función indirecta de utilidad} es la función que nos permite calcular el nivel de utilidad que alcanza el consumidor para cada nivel de precios y renta. Es decir, es la función que nos permite calcular el valor de la función de utilidad para cada nivel de precios y renta (Ahora depende de los precios). \\
$$
\begin{array}{rrcl}
    v^i: & \mathbb{R}_{++}^L \times \mathbb{R}_+ & \to &\mathbb{R}\\\\
	 & \left(\textbf{p},M^i\right) & \mapsto & v^i\left(\textbf{p},M^i\right) = u^i\left(\textbf{x}^{i*}\right)=u^i\left(d^i\left(d^i(\textbf{p},M^i)\right)\right).
\end{array}
$$

Si los precios fueran estos tu utilidad fueran estos otros. 

\paragraph{Propiedades de la función indirecta de utilidad:}

\begin{itemize}
    \item Continua,
    \item Homogénea de grado cero en $\left(\textbf{p},M^i\right)$, y
    \item cuasi-convexa en $\textbf{p}$.
\end{itemize}

Podemos pasar de la función indirecta de utilidad a la demanda marshalliana. Se le llama \textbf{identidad de Roy}, que nos da la cantidad de cada bien demandada en el óptimo.

$$x_k^{i*} = d_k^i\left(\textbf{p},M^i\right) = -\dfrac{\dfrac{\partial v^i\left(\textbf{p},M^i\right)}{\partial p_k}}{\dfrac{\partial v^i\left(\textbf{p},M^i\right)}{\partial M}}, \qquad k=1,2,\ldots, l$$


\section{Función de gasto}
¿Cuál sería el gasto mínimo que debemos de hacer para llegar a un nivel de utilidad?.

$$
\begin{array}{rrclcll}
    e^i: & \mathbb{R}_{++}^L \times \mathbb{R} & \to &\mathbb{R}&&&\\\\
	 & \left(\textbf{p},u^i\right) & \mapsto & e^i\left(\textbf{p},u^i\right) &=& \min_{\textbf{x} \in \mathcal{X}^i} & \textbf{px}\\\\
	 & & & &&\text{s.a. } & u^i(\textbf{x}) \geq u^0.
\end{array}
$$

\paragraph{Propiedades de la función de gasto:}

\begin{itemize}
    \item Continua,
    \item Homogénea de grado uno en $\textbf{p}$, y
    \item cóncava en $\textbf{p}$.
\end{itemize}

Lo que nos da el problema dual del consumidor $i$.
$$
\left[P^i\right]
\left\{
    \begin{array}{rl}
	\min_{\left(x_1,x_2,\ldots,x_L^i\right) \in \mathbb{R}^L} & \overline{p}_1x_1^i+\overline{p}_2x_2^i+\ldots+\overline{p}_Lx_L^i\\\\
	\text{s.a. } & u^i\left(x_1^i,x_2^i,\ldots,x_L^i\right) \geq u^0\\\\
    \end{array}
\right.
$$

La solución nos dará una \textbf{función de demanda compensada o función de demanda hicksiana}.
$$
\begin{array}{rrcl}
    h^i: & \mathbb{R}_{++}^L \times \mathbb{R} & \to & \mathcal{X}^i\equiv \mathbb{R}_{+}^L\\\\
	 & \left(\textbf{p},u^i\right) & \mapsto & h^i\left(\textbf{p},u^i\right) = \textbf{x}^{i*}= \left(x_1^{i*},x_2^{i*},\ldots,x_L^{i*}\right)=h\left(p_1,p_2,\ldots,p_L,u\right).
\end{array}
$$
Llevando $\textbf{x}^{i*}$ a la función de gasto,
$$e^i\left(\textbf{p},u^i\right) = \textbf{px}^{i*} = \textbf{p}h^i\left(p_1,p_2,\ldots,p_L,u\right).$$

Ahora, podemos pasar de la función de gasto a la demanda compensada hicksiana. Sea $e\left(P_1,P_2,\ldots,P_L,u\right)$ la función de gasto de un consumidor, obtenemos por el \textbf{lema de Shephard} la demanda hicksiana de la siguiente forma:
$$h_k^i \left(\textbf{p},u\right) = \dfrac{\partial e^i\left(\textbf{p},u\right)}{\partial p_k}, \qquad k=1,2,\ldots,L.$$



\chapter{Núcleo y equilibrio}

\section{Economía de intercambio}

Una economía $\mathcal{E}$ es definida por: $n$ número de consumidores que intercambian $l$ mercancías. Cada consumidor $i$ es un conjunto de consumo $X_i\subset \mathbb{R}^l_+$, con un consumo inicial y una relación de preferencias representados por la función de utilidad $U_i$ que ordenan los distintos vectores $\omega_i\in X_i$.\\

Si el conjunto de consumo es positivo y continuo, siempre existe una función de utilidad que las representa. Es decir, La función de utilidad $U_i:\mathbb{R}^l_+\to \mathbb{R}$, representa las preferencias de un agente $i$ sobre el consumo. $U_i(x)\geq U_i(y)$ si y sólo si $x\succeq_i y$.\\
$$\mathcal{E}=\left\{X_i,\omega_i,U_i,i=1,\ldots,n\right\}.$$


\section{Distribución de los recursos}

Sean, $\displaystyle \omega^h=\sum_{i=1}^n w_i^h,\; h\in \left\{1,\ldots,l\right\}$ y  $\displaystyle X=\prod_{i=1}^n X_i$. Una asignación $x=\left(x_1,x_2,\ldots,x_n\right)\in X$ es factible (si lo que le corresponderá a cada uno es lo que hay) $\displaystyle\sum_{i=1}^n x_i\leq \sum_{i=1}^n \omega_i$. Esto es, $\displaystyle\sum_{i=1}^n x_i^h \leq \sum_{i=1}^n w_i^h = w^h$ para todo $h\in\left\{1,\ldots,l\right\}$.

Nos preguntamos: ¿Que propiedades serían deseadas?

\section{Eficiencia o optimalidad de Pareto}
Un reparto es optimo de Pareto, si es factible (que no se reparte más de lo que hay).\\

Sea $x=\left(x_1,\ldots,x_n\right)$ una asignación factible. 
\begin{itemize}
    \item $x$ es débilmente eficiente si no existe una asignación factible $y$ tal que: $x_i\succeq_i y_i$ para cada individuo $i$.
    \item $x$ es fuertemente eficiente si no existe una asignación factible $y$ tal que: 
	\begin{enumerate}
	    \item $y_i\succeq_i x_i$ para cada individuo $i$.
	    \item $y_k\succeq_k x_k$ para cada consumidor $j$.
	\end{enumerate}
\end{itemize}

Es eficiente si no hay otro mejor. Por ejemplo, todo para mi es eficiente porque no hay otro reparto donde yo esté mejor.\\


\section{El núcleo}
Una coalición $S$ de consumidores, bloquea, veta, impide un reparto $x$ si la coalición no puede consumir más de lo que tiene disponible y la coalición puede encontrar una alternativa que haga que todos sus miembros estén mejor. Es decir, una coalición $S$ de consumidores, bloquea $x$ si existe una asignación $y$ tal que:
\begin{enumerate}[i)]
    \item $\sum_{i\in S} y_i\leq \sum_{i\in S} \omega_i$.
    \item $y_i\succeq_i x_i$ para cada miembro en la coalición $S$.
\end{enumerate}

Somos todos agentes de la economía y nos proponen un reparto, pero tu y yo nos repartimos lo que tenemos y estamos bien.\\

El núcleo esta formado por las asignaciones que no están vetadas por ninguna coalición de agentes que son óptimos de Pareto de cualquier subeconomía que se puede forma una economía.\\

La coalición es una alianza temporal de agentes que se unen para vetar un reparto.

El núcleo da fundamentos para argumentar el equilibro Walrasiano, porque Edgeworth conjeturo que si la economía crecía, entonces las relaciones de equilibro iban a coincidir. 


\section{Asignaciones JUSTAS}
Es libre de envidia. Es decir, si lo que me toca a mi es por lo menos tan bueno como lo que le toca a ti. Si todos tenemos derecho al pastel, por que a ti te tocará más de lo que me toca.


Un juego $\mathcal{G}$ en su forma normal se define por:
\begin{itemize}
	\item Un conjunto de jugadores $J=\left\{1,2,\ldots,n\right\}$.
	\item Para cada jugador $i$, un conjunto de estrategias $S_i$.
	\item $\begin{array}{rrcl}
		\pi_i:&S\times S_i &\to& \mathbb{R}\\
		      &(s,z) &\mapsto& \pi_i(s,z)
	\end{array}$ denotado la función de pago para el jugador $i$. $z$ es el pago que recibe el jugador $i$, si cambia de $s_i$ a $z$.
\end{itemize}
Dado un perfil de estrategias $s$, denotemos con $s_{-i}$ las estrategias de todos los jugadores excepto $i$. Por tanto, escribimos $s=(s_i,s_{-i})$.\\

Un equilibro de Nash es un perfil de estrategias $s^*\in S$, tal que para cada jugador $i\in J$ 
$$\pi_i(s^*)\geq \pi_i(s_{-i}^*,s_i),\; \forall s_i\in S_i.$$
El pago que recibe el jugador $i$ es mayor o igual al pago que recibe si cambia de estrategia.\\

El equilibro de Nash no es eficiente, pero es estable.


\section{Precios}
El sistema de precios es un vector $p\in\mathbb{R}^l_+$. El conjunto presupuestario es:
$$B_i(p)=\left\{x\in X_i| p\cdot x\leq p\cdot \omega_i\right\}.$$
Podemos normalizar los precios como:
$$\triangle = \left\{p\in\mathbb{R}^l_+, \sum_{h=1}^l p_h=1\right\}.$$


\section{Equilibrio competitivo (Walrasiano)}
El equilibro Walrasiano para $\mathcal{E}$ es un par $(p,x$, donde $p$ es un sistema de precios y $x$ es un reparto factible, tal que para cada agente $i$, el paquete $x$ maximiza la función de utilidad $U_i$ en el conjunto presupuestario:
$$B_i(p)=\left\{x\in X_i| p\cdot x\leq p\cdot \omega_i\right\}.$$
Tiene un reparto descentralizado, nadie tiene que decirle a nadie que hacer.\\
¿Que tiene de interese esta definición?

\section{Ley de Walras}
El valor de exceso de demanda agregada es cero
$$p\cdot Z(p)=p\cdot \left(\sum_{i=1}^n x_i(p)-\sum_{i=1}^n \omega_i\right)=0.$$\\

¿Que supuestos sobre las características de los individuos nos aseguran que existe un equilibro Walrasiano?, ¿El equilibro Walrasiano, cuando existe,es único?. ¿Que propiedades bienestar cumplen las distribuciones de recursos inducidas por un equilibro Walrasiano?. ¿Es el equilibro Walrasiano socialmente estable?.

\chapter{Intercambio de economías puras: Teoremas Welfare}

\section{Intercambios y mercados}

\begin{itemize}
    \item Mecanismos de intercambio (libre de precios):
	Asignaciones eficientes y el núcleo.
    \item Mercados (precios):
	Equilbrio competitivo (Walrasiano).
\end{itemize}

Tenemos dos preguntas

\begin{itemize}
    \item Dada una asignación walrasiana, ¿es eficiente? ¿Está en el núcleo?. El primer teorema es: Toda asignación walrasiana es eficiente.
    \item Dada una asignación eficiente, ¿se puede descentralizar mediante un sistema de precios?.
\end{itemize}

\section{Primer teorema del bienestar}
\begin{teo}
    Sea $x$ una asignación Walrasiana. Existe un sistema de precios $p$ tal que $(x, p)$ es equilibrio.\\

	Demostración.-\; Supongamos que $x$ no está en el núcleo. Hay una coalición $S$ veta $x$. $\sum_{i\in S} y_i\leq \sum_{i\in S} \omega_i$ y $U_i(y_i)> U_i(x_i)$ para todo $i\in S$.
	$$p\cdot y_i > p\cdot \omega_i, \; i\in S.$$
	$$p\cdot \sum_{i\in S} y_i > p\cdot \sum_{i\in S} \omega_i.$$
\end{teo}

\subsection{Eficiencia y descentralización}
$X:I=[0,1] \to \mathbb{R}^n_+$, $X$ es factible si y sólo si
$$\int_I x(t)\; dt\leq \int_I \omega(t)\; dt.$$
Ahora, $X$ está vetada por $S\subset [0,1]$ si y sólo si, existe $I: S\to \mathbb{R}^n_1$ tal que
\begin{enumerate}[i)]
    \item $\displaystyle\int_S y(t)\; dt\leq \int_S \omega(t)\; dt$.
    \item $u_i(y(t))>u_i(x(t))$ para todo $i\in S$.
\end{enumerate}

Aumann considere que la economía tiene infinitos agentes. Dado $\epsilon$ continua. Entonces,
$$W(\epsilon)=C(\epsilon).$$

Cuando se tiene equilibro finito. Entonces,

\section{Segundo teorema del bienestar}
\begin{teo}
    Supongamos preferencias son continuas monótonas y convexas para cada consumidor $i$. Además supongamos $w_i\geq 0$ para cada $i$.\\

    Si $x$ es una asignación eficiente (óptima de Pareto), entonces existen precios $p$ tales que $(p, x)$ es un equilibrio walrasiano para la economía con dotaciones dadas por $x$.\\

	Demostración.-\; La idea de nuestra prueba es la siguiente: Considere $\omega$, las dotaciones totales de la economía. Sea $A$ el conjunto de dotaciones totales que podrían distribuirse de manera que cada agente esté estrictamente en mejor situación que con la asignación $x$. $A$ es convexo y $\omega$ no pertenece a $A$. Existe $p$ que separa $A$ de $\omega$. No es difícil demostrar que $(p, x)$ es un equilibrio walrasiano si las dotaciones son $x$.
\end{teo}


\chapter{Existencia del equilibro de Nash}
\begin{teo} Para cada jugador $i$, sea $S_i$ no vacío, compacto y convexo, y $\pi_i$ continuo en los perfiles de estrategia y estrictamente cuasi cóncavo en la estrategia del jugador $i$. Entonces, hay un equilibrio de Nash.
\end{teo}


\chapter{Decisión colectiva: Teorema de Imposibilidad de Arrow}

\section{Teoría de elección social}
Estudia la toma de decisiones colectivas a partir de las preferencias de los individuos de una sociedad. La idea principal es:
\begin{center}
    Pasar de las preferencias individuales a las preferencias sociales.
\end{center}

Las \textbf{preferencias sociales} son una relación de Orden social coherente que permita una ordenación de alternativas desde el punto de vista social $R=f(R^1,\ldots,R^n)$.

\subsection{Construcción de una relación de orden social (R)}
\begin{enumerate}
    \item $\textbf{N}=\left\{1,2,\ldots,n\right\}$ es el conjunto de individuos de la sociedad $(N\geq 2)$.
    \item $\textbf{X}:$ conjunto no vacío de estados o alternativas sociales.
	\begin{itemize}
	    \item Candidatos a un puesto (delegado, decano, presidente, etc.).
	    \item Colección de planes de obras públicas (carril bici, parque infantil, etc.).
	\end{itemize}
    \item $R^i$ orden de preferencias del individuo $i$-ésimo sobre alternativas de $X$ ($P^i$ preferencias estricta: $I^i$ indiferencia).
    \item $R^i$ es completa y transitiva (racional) con objeto de que los individuos puedan realizar comparaciones binarias sobre cualesquiera de los elementos de $X$.
\end{enumerate}

\subsection{Relación de orden social (R)}

\begin{def.} Una relación de orden social, $\textbf{R}$, es una relación binaria sobre el conjunto de estados sociales $\textbf{X}$ completa y transitiva, que depende de alguna forma de los órdenes de preferencias individuales.
    $$R=f(R^1,\ldots,R^n).$$
    siendo $(R^1,\ldots,R^n)$ el perfil de preferencias de los $N$ individuos.
\end{def.}

Para cualesquiera $(x,y)\in \textbf{X}$ podemos decir que $xRy$, $x$ es socialmente al menos tan bueno como $y$ (con $P$ e $I$ suendo las relaciones asociadas de estricta preferencia e indiferencia asociadas de estricta preferencia e indiferencia social, respectivamente).

\begin{ejem} Supongamos que los agentes tienen preferencias sobre alternativas $(x,y)$ representadas por la función:
    $$
    R^i(x,y) = 
	\left\{
	    \begin{array}{rcl}
		1 & \mbox{si} & xP^iy\\\\
		0 & \mbox{si} & xI^iy\\\\
		-1 & \mbox{si} & yP^ix
	    \end{array}
	\right.
    $$
    Un ejemplo de relación de orden social que representa las preferencias sociales pueden ser:
    $$ R = f(R^1,\ldots,R^n) = \text{signo} \sum_{i=1}^N \beta^i R^i.$$
    donde $\beta^i$ son los pesos que se asignan a las preferencias de los individuos.
\end{ejem}

\begin{ejem}[Votación por mayoría] donde $\beta=1, \; \forall i = 1,\ldots,N$, donde todos tenemos la misma oportunidad.
    $$R=f(R^1,\ldots,R^n)=\text{signo} \sum_{i=1}^N R^i, \quad 
    R^i \to
	\left\{
	    \begin{array}{rcl}
		>0 & \mbox{si} & xPy\\
		=0 & \mbox{si} & xIy\\
		<0 & \mbox{si} & yPx
	    \end{array}
	\right.
    $$
\end{ejem}

\begin{ejem}[Dictadura] $\beta^j=1$ y $\beta^i=0$ para $\forall i\neq j$, $j$ (dictador).
	$$R=f(R^1,\ldots,R^n) =
	R^j  = 
	\left\{
	    \begin{array}{rcl}
		1 & \mbox{si} & xPy\\
		0 & \mbox{si} & xIy\\
		-1 & \mbox{si} & yPx
	    \end{array}
	\right.
	$$
\end{ejem}

\begin{ejem}[Votación con 2 alternativas] Sean $3$ agentes $(1,2,3$ con el siguiente perfil de preferencias sobre 2 alternativas $(x,y)$.
    $$
    \begin{matrix}
	& \textbf{1} & \textbf{2} & \textbf{3}\\
	1^o & x & y & y\\
	2^o & y & x & x\\
    \end{matrix}
    $$
    De donde, $R^1(x,y)=1$, $R^2(x,y)=-1$, $R^3(x,y)=-1$. Entonces, 
    $$R(x,y)=\text{signo} \sum_{i=1}^3 R^i(x,y)=\text{signo}(1-1-1)=-1<0$$
    Por lo tanto, la alternativa $y$ es la socialmente preferida: $yPx$.
\end{ejem}

\begin{ejem}[Con tres alternativas]
    $$
    \begin{matrix}
	& \textbf{1} & \textbf{2} & \textbf{3}\\
	1^o & x & y & z\\
	2^o & y & z & x\\
	3^o & z & x & y\\
    \end{matrix}
    $$
    Encontramos que existe un triple empate.
\end{ejem}

\textit{\textbf{¿Existe algún otro criterio (distinto del dictatorial) que permita llegar a una preferencia social (algún ganador)?}}


\subsection{Normas para elegir la propuesta}
\begin{enumerate}[\bfseries 1.]
    \item  Regla de pluralidad (mayoría relativa): cada uno vota por la alternativa preferida y sale elegida la que obtenga más votos: gana A. 
    \item Regla de Borda: cada votante asigna 4, 3, 2, 1 y 0 votos de la alternativa más preferida a la menos preferida. Gana la que más puntos obtenga: gana B. 
    \item Método de Condorcet: comparamos propuesta con cada una de las otras por separado (no siempre hay ganador de Condorcet): gana C. 
    \item Eliminación secuencial: se vota en diferentes rondas eliminando la alternativa menos votada en cada ronda. Gana la que se mantiene:gana D. 
    \item Doble vuelta: Las dos alternativas con más votos en una primera ronda se enfrentan en una segunda ronda. Gana el que tenga más votos en la 2ª ronda: gana E.
\end{enumerate}

\section{Teorema de Arrow}
Arrow pensó en incorporar un conjunto de propiedades que debería cumplir una relación de orden social para que fuera considerada como una relación de orden social justa. \\

Ahora, para diseñar una \textbf{regla de valoración social} requiere dar respuesta a las siguientes cuestiones 

\begin{enumerate}[a)]
    \item Argumentos a utilizar, solo ordenación de alternativas, o también intensidad de preferencias.
    \item Aplicabilidad: a todas las configuraciones posibles de valoraciones individuales (perfiles) o sólo a algunas.
    \item Coherence: presente alguna racionalidad (ej. aplicada en dos circunstancias idénticas de el mismo resultado)
    \item Operatividad: elija alguna alternativa.
    \item Comportamiento: como responde la regla a las valoraciones individuales (ej. si todos prefieren $x$ a $y$, la regla debería valorar más a $x$ que a $y$).
\end{enumerate}

Argumentos de \textbf{enfoque de Arrow} a utilizar: ordenaciones individuales (no las intensidades de preferencia).
\begin{itemize}
    \item  Aplicabilidad: universal, es decir, pueda ser aplicada a cualesquiera que sean las configuraciones de pref. indiv.
    \item  Coherencia/operatividad: la regla de evaluación debe ser una relación de orden completa y transitiva.
    \item  Comportamiento: unanimidad, mínimamente democrática (no dictatorial), e independiente de alternativas irrelevantes (tenga sólo en cuenta las preferencias individuales sobre las alternativas que se están valorando.
\end{itemize}

Todo esto da lugar a los \textbf{Aximoas de Arrow}:

\begin{enumerate}[\bfseries 1.]
    \item Dominio Universal o no restringido (U). La relación de orden social debe estar definida sobre cualquier perfil de preferencias individuales.
    \item Principio de Pareto Débil (P). Para cualesquiera par de alternativas sociales $x,y$ en $X$, si $x$ es preferido a y para todo $i$ $(xP^iy,\forall i)$, entonces $x$ será socialmente preferido a y $(xPy)$.
    \item Independencia de alternativas irrelevantes (IAI). El orden social entre dos alternativas $x,y$ sólo depende del orden de estas dos alternativas. Formalmente, sea $Y\subset \textbf{X}$, y sea $R(\textbf{X}/Y)$ el orden social de $X$ restringido al conjunto $Y$, $R(Y)$ el orden social del conjunto $Y$. Entonces, $R(\textbf{X}/Y) = R(Y)$.
    \item No dictatorial (ND): no existe un individuo cuyas preferencias determinan el orden social.
\end{enumerate}

\subsection{Teorema de imposibilidad de Arrow}
Si el número de agentes es $N\geq 2$, y el número de alternativas (estados sociales) es mayor o igual que tres, no existe ninguna Relación de Orden Social que satisfaga simultáneamente los axiomas DU, P, IAI y ND.

La única ROS que satisface los axiomas DU, P, IAI es la dictatorial. Es decir, existe algún $j\in N$ tal que, para cualquier perfil de preferencias $(R_1,\ldots,R^N)$, y para todo par de alternativas $(x,y)\in \textbf{X}$, se verifica:
$$xP^jy \Rightarrow x P y.$$

Existen algunos problemas con el teorema de Arrow:

\begin{itemize}
    \item No existe ningún mecanismo “perfecto” (que cumpla las condiciones de Arrow) que permita agregar las preferencias individuales para obtener una preferencia social.
    \item  Sen (1970, pág. 38): cada una de las condiciones de Arrow resulta bastante inocua, pero conjuntamente “parecen producir un monstruo capaz de devorar todas las pequeñas funciones de bienestar social del mundo”.
    \item  Desde entonces, gran parte del programa de investigación se ha dedicado a la búsqueda de vías de escape a este resultado.
\end{itemize}


Ahora necesitamos buscar un \textbf{Funcional de bienestar social} 

