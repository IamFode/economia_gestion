\documentclass[10pt]{book}

% Configuración de página
\usepackage[text=17cm,
	    left=2.54cm,
	    right=2.54cm, 
	    headsep=20pt, 
	    top=2.54cm, 
	    bottom = 2.54cm,
	    letterpaper,
	    showframe = false]{geometry}
%Paquetes
\usepackage{multicol}
\usepackage{titlesec} %formato de títulos
\usepackage{titling}
\usepackage[spanish,english]{babel}
\usepackage{enumerate}
\usepackage{marginnote}
\usepackage[Bjornstrup]{fncychap}
\usepackage{graphicx}
\usepackage{natbib}
\usepackage[backref=page]{hyperref}
\usepackage{bibentry}
\usepackage{booktabs}
\usepackage{lscape}
\usepackage{forest}
\usepackage{fancyhdr}
\usepackage{tcolorbox}
\usepackage{truncate}
\usepackage{amsmath}
\usepackage{amssymb}
\usepackage{pgfplots}
\usetikzlibrary{intersections}
\usepgfplotslibrary{fillbetween}

%configuración tcolorbox
\tcbset{colback=black!2,colframe=white}

%configuración de separación de sección
\titlespacing*{\section}{0pt}{1.7cm}{.3cm}

%configuración de separación de sección
\titleformat*{\section}{\bfseries}
\titleformat*{\subsection}{\bfseries}
\titleformat*{\subsubsection}{\bfseries}
\titleformat*{\paragraph}{\bfseries}
\titleformat*{\subparagraph}{\bfseries}

%Definiciones
\newtheorem{axioma}{Axioma}
\newtheorem{teo}{Teorema}
\newtheorem{def.}{Definición}
\newtheorem{ejem}{Ejemplo}

%Formato de título de capítulos
\titleformat{\chapter}[display]
{\vspace{4ex}\bfseries\huge}
{\filleft\Huge\thechapter}
{2ex}
{\filleft}
