
\section{Interacción económica}

\begin{itemize}
    \item Estructura de la interacción entre diferentes centros económicos.
    \begin{itemize}
	\item Comercio.
	\item Flujos de capital y mano de obra.
	\item Medios de comunicación modernos (intercambio de ideas, exposición a otras influencias culturales)
    \end{itemize}
\end{itemize}

\begin{itemize}
    \item Las exportaciones de bienes y servicios de un país a otro implican tiempo y esfuerzo y, por lo tanto, costos.
    \begin{itemize}
	\item Los bienes deben cargarse, descargarse, transportarse por vía, tren, barco, avión y empaquetarse, asegurarse y rastrearse antes de su destino
	\item Desempacarse, verificarse, ensamblarse y exhibidos en el destino (antes de las ventas)
	\item Red de distribución y mantenimiento.
	\item El exportador debe estar familiarizado con las reglas y procedimientos (legales) 
	\item Otros idiomas, cultura diferente. 
    \end{itemize}
    \item Estos costos aumentan con la distancia
    \begin{itemize}
	\item Distancia física.
	\item Distancia política, cultural o social      
    \end{itemize}
    \item Ecuación de la gravedad: 149 socios comerciales alemanes
\end{itemize}

\begin{itemize}
    \item Primera confirmación de la relación negativa entre distancia y flujos comerciales
    \item La distancia no es el único determinante de los flujos comerciales
    \item La demanda potencial de productos alemanes es, en igualdad de condiciones, mayor en Italia que en los Países Bajos.
    \item Corrección de los flujos de exportación alemanes por este efecto de la demanda, dividiendo las exportaciones por el PIB de un país
    \item La relación es claramente más estrecha
    \item Varios países de ejemplo (cercanos o con un gran PIB) identificados en la primera figura (que tienden a estar por encima de la línea de regresión) están todos cerca de la línea de regresión en la segunda figura
    \item Una regresión simple que incluye tanto el impacto del ingreso como de la distancia en los flujos de exportación da:
    \item $ln(exportaciones) = 0.98*ln(PIB) - 1.02*ln(distancia); R2=0.915$
\end{itemize}

\section{Cambio rápido en la distribución de la producción y la población.}
\begin{itemize}
    \item Ejemplos de la distribución desigual en el espacio de la población y la actividad económica
    \item Regularidad en esta distribución.
    \item Interacción entre centros económicos.
    \item Característica común: ejemplos recientes ¿Siempre estuvo la actividad económica distribuida de manera desigual?
    \item La respuesta es sí basada en:
	\begin{itemize}
	    \item Las ciudades, ya comenzaban a surgir después de la revolución neolítica (aumento del excedente agrícola) 
	    \item Las ciudades como densa concentración de personas siempre han representado centros de poder económico, político, cultural, sociológico, militar y científico 
	    \item El fenómeno de la ciudad como representación de la desigual distribución de la actividad, ya nos acompaña desde hace mucho tiempo.
	\end{itemize}
\end{itemize}

\begin{itemize}
    \item La respuesta es no basado en:
    \begin{itemize}
	\item La asimetría o desigualdad de la distribución de la actividad económica ha cambiado con el tiempo
	\item Cambio en el grado de urbanización en Europa cuando se pone en perspectiva histórica
	\item Los aumentos en la participación de la producción mundial en los milagros asiáticos muestran un aumento mucho más rápido a nivel de país
    \end{itemize}
    \item Desafíos para un modelador económico geográfico (nivel de ciudad, nivel global, cambios rápidos)
\end{itemize}

\section{}
\begin{itemize}
	\item Proceso de acumulación económica,  producción de alimentos,  domesticación de plantas y animales  
	\begin{itemize}
	    \item Acumulación de bienes no muebles 
	    \item Mayores densidades de población 
	    \item Desarrollo de ciudades con especialistas no productores de alimentos \\
	\end{itemize}
\end{itemize}

\subsection{Ventajas del continente euroasiático}
		
\begin{enumerate}
    \item Domesticación temprana de plantas y animales.
	\begin{enumerate}
	    \item Plantas: de 56 especies de pastos silvestres candidatas potencialmente adecuadas para la domesticación.
		\begin{itemize}
		    \item Treinta y tres se pueden encontrar en el oeste de Asia, Europa y el norte de África (continente de Eurasia) 
		    \item Cuatro en el África subsahariana, cuatro en América del Norte, cinco en Mesoamérica y dos en América del Sur.
		\end{itemize}
	    \item Animales: grandes animales mamíferos
		\begin{itemize}
		    \item Continente euroasiático capaz de domesticar trece especies
		    \item  Américas solo una
		\end{itemize}
	\end{enumerate}
    \item Orientación de oeste a este (áreas con la misma latitud)
	\begin{itemize}
	    \item Compartir la misma duración del día, variación estacional, enfermedades, temperaturas.
	    \item Mayor capacidad para domesticar plantas y animales 
	    \item Facilitó la difusión del conocimiento sobre la producción de alimentos de oeste a este o viceversa 
	    \item El continente euroasiático pudo disfrutar de los beneficios de la domesticación plantas y animales por asimilación más que por invención miles de años antes de su invención en las Américas
	\end{itemize}

    \item Acumulación de los beneficios de la interacción económica y la difusión del conocimiento a lo largo de la historia
	\begin{itemize}
	    \item Posición dominante alrededor de 1500 
	    \item Expansión posterior de los pueblos de ese continente al resto del mundo
	\end{itemize}
\end{enumerate}

\section{Conclusiones}

\begin{itemize}
    \item Gran variación en la densidad de población y actividad económica en:
	\begin{itemize}
	    \item Nivel global 
	    \item Nivel continental 
	    \item Nivel nacional
	\end{itemize}
    \item Dimensión fractal
	\begin{itemize}
	    \item La distribución altamente desigual de la actividad económica se repite en diferentes niveles de agregación 
	    \item Mecanismos de agrupamiento en acción
	\end{itemize}
    \item Regularidad de la distribución de la actividad económica
	\begin{itemize}
	    \item Patrón bien ordenado subyacente a la distribución de la actividad económica 
	    \item Distribución del tamaño de la ciudad, potencial de mercado, ecuación de gravedad.
	\end{itemize}
    \item Cambios rápidos
    \item Geografía Física
\end{itemize}

\chapter{Aglomeración y teoría económica}

\section{Introducción}

\begin{enumerate}
    \item Mensaje central de la primera conferencia
	\begin{itemize}
	    \item La geografía es importante
	    \item La actividad económica no está distribuida uniformemente en el espacio (gran variedad de aglomeraciones económicas)
	    \begin{itemize}
		\item Nivel global: dualismo Norte-Sur 
		\item Grandes áreas económicas: UE, Este-Asia 
		\item Nivel nacional: India, España
	    \end{itemize}
	\end{itemize}
    \item ¿Por qué es importante la ubicación para las actividades económicas?
	\begin{itemize}
	    \item Necesidad de un marco analítico en el que la geografía desempeñe un papel de una forma u otra 
	    \item Las decisiones de los agentes económicos están parcialmente determinadas por la geografía 
	    \item La geografía de la economía misma puede derivarse del comportamiento de los agentes económicos
	\end{itemize}
    \item Mecanismos que dan lugar a la aglomeración (teorías de la aglomeración)
	\begin{itemize}
	    \item Rendimientos crecientes a escala en un proceso de producción
	    \item Externalidades
	    \item Mercados imperfectamente competitivos
	    \item Competencia espacial
	    \item Espacio no homogéneo/espacio no neutral
		\begin{itemize}
		    \item Racionalizar la aglomeración sin ninguna forma de rendimientos crecientes a escala
		\end{itemize}
	\end{itemize}
    \item No existe un modelo único para todos 
    \item Nuestro enfoque: economía geográfica/nueva geografía económica
	\begin{itemize}
	    \item Elimina la competencia espacial que da lugar a que el comportamiento estratégico no se tiene en cuenta y las empresas toman como dado el comportamiento de las demás (fijación de precios) Elimina el espacio no neutral.
	\end{itemize}
	    
\end{enumerate}

\section{El espacio y el paradigma competitivo}

\subsection{Teoría económica neoclásica}

\begin{itemize}
    \item Suposiciones 
	\begin{enumerate}
	    \item Tecnologías sujetas a rendimientos constantes a escala (CRS) 
	    \item Mercados perfectamente competitivos (los agentes económicos son tomadores de precios)
	\end{enumerate}
    \item El marco estándar de equilibrio general (retornos constantes-paradigma de competencia perfecta) no funcionará si queremos que la geografía o el espacio importen.
    \item El teorema de la imposibilidad espacial (Starret, 1978)
\end{itemize}

\subsubsection{El teorema de la imposibilidad espacial}
\begin{itemize}
    \item Resultado: Introducir un espacio homogéneo en el modelo de Arrow-Debreu 
	\begin{itemize}
	    \item Los costos totales de transporte en la economía deben ser cero en cualquier equilibrio competitivo espacial 
	\end{itemize}
    \item La especialización regional, las ciudades y el comercio no pueden ser resultados de equilibrio 
    \item El modelo competitivo per se no puede usarse como la base para el estudio de una economía espacial
    \item El modelo competitivo no puede generar aglomeraciones económicas 
    \item Recuerde: estamos interesados en identificar mecanismos puramente económicos que lleven a los agentes a aglomerarse incluso en un espacio sin rasgos distintivos
    \item Intuición: 
	\begin{enumerate}
	    \item Espacio homogéneo (distribución uniforme de los recursos naturales)
	    \begin{itemize}
		\item Los agentes económicos utilizan la tierra,  implica que no pueden estar todos juntos en el mismo lugar 
		\item Equilibrio compatible con el entorno competitivo y un espacio homogéneo, implica  la colección de autarquías locales 
	    \end{itemize}
	    \item  Tecnologías convexas 
		\begin{itemize}
		    \item La producción no muestra rendimientos crecientes a escala, cualquiera que sea su escala. 
		    \item Permitir que cada actividad productiva se lleve a cabo a un nivel arbitrario sin pérdida de eficiencia. 
		    \item Fragmentar las operaciones de una empresa en unidades más pequeñas en diferentes ubicaciones no reduce la producción total disponible de la misma dados los insumos mientras que los costos de transporte disminuyen
		\end{itemize}
	\end{enumerate}
    \item Si 1 y 2 se cumplen: 
	\begin{itemize}
	    \item Todos los insumos y productos necesarios para satisfacer las demandas de los consumidores se pueden ubicar en un área pequeña cerca de donde viven los consumidores 
	    \item La economía es tal que cada persona produce para su propio consumo ("Autarquía" o "Patio trasero"). capitalismo”) 
	    \item Se puede evitar el transporte de personas y mercancías 
	\end{itemize}
    \item Los rendimientos crecientes a escala son fundamentales para explicar la distribución geográfica de las actividades productivas (teorema popular de la economía espacial)
\end{itemize}

\subsubsection{Quitarse y problemas}

\begin{enumerate}
    \item El análisis de equilibrio general estándar se abstiene de la consideración de indivisibilidades o rendimientos crecientes a escala 
	\begin{itemize}
	    \item Cada lugar preferirá la autarquía para ahorrar en costos de transporte 
	    \item No captura el impacto esencial del transporte y la tierra cuando se trata de estudiar la distribución espacial de las actividades económicas 
	    \item Perfectamente El mecanismo de precios competitivos por sí solo no puede explicar la formación endógena de aglomeraciones económicas (Starret, 1978)
	\end{itemize}

		Koopmans (1957) Sin reconocer las indivisibilidades – en la persona humana, en las residencias, plantas, equipos y en el transporte – los patrones de ubicación, hasta los de la aldea más pequeña, no se pueden entender Duranton, (2008) Si uno toma los costos de transporte como un hecho inevitable de la vida, debe asumir alguna falta de homogeneidad en el espacio o alguna falta de convexidad en la producción conjuntos.\\
    \item La ubicación de las actividades económicas se vuelve importante solo si hay indivisibilidades o costos adicionales involucrados
    \item La modelización de la economía espacial implica que la teoría debe partir del análisis competitivo general
	\begin{itemize}
	    \item No convexidad de los conjuntos de producción (Economía Urbana y Regional)
	    \item No homogeneidad del espacio (Teoría del Comercio Internacional -neoclásica) 
	    \item No convexidad de los conjuntos de producción + costos de transporte (Economía Geográfica) 
		\begin{itemize}
		    \item Con rendimientos crecientes pero sin costos de transporte implica que las empresas producen en una sola planta, pero no les importa la ubicación de esta planta 
		    \item En ausencia de rendimientos crecientes (o espacio no homogéneo) pero con costos de transporte implica la situación de capitalismo de traspatio-
		\end{itemize}
	\end{itemize}
    \item Dificultad para combinar costos de transporte, rendimientos crecientes a escala y competencia imperfecta en un marco de equilibrio general

\end{enumerate}

\section{Por qué observamos conglomeraciones}
\begin{enumerate}
    \item La evidencia empírica sugiere que el costo de vida en las grandes áreas metropolitanas suele ser más alto que en las áreas urbanas más pequeñas 
    \item Una mejor infraestructura y el rápido desarrollo de las TIC podrían sugerir la muerte de la distancia  implica que la diferencia de ubicación se desvanecería gradualmente (las economías de aglomeración desaparecerían) 
    \item Las cosas no son tan simples!! 
	\begin{itemize}
	    \item La relación entre la disminución de los costos de transporte y el grado de aglomeración de las actividades económicas no es la esperada por muchos analistas 
	    \item Por debajo de ciertos valores críticos, nuevas disminuciones pueden generar dispersión (diferencias de precios de los factores) 
	    \item El progreso tecnológico genera nuevas actividades de innovación que más se benefician ser aglomerado 
	\end{itemize}
    \item La configuración espacial de las actividades económicas es el resultado: 
	\begin{itemize}
	    \item Fuerzas de aglomeración (o centrípetas) 
	    \item Fuerzas de dispersión (o centrífugas)
	\end{itemize}
\end{enumerate}

\subsection{Aglomeración y retornos crecientes}
\begin{itemize}
    \item Los RI en las actividades de producción son necesarios para explicar las aglomeraciones económicas (dejando de lado la geografía física) 
	\begin{itemize}
	    \item Disminución de los costos promedio a medida que se produce más producción dentro de un área geográfica específica 
	\end{itemize}
    \item Preguntas: 
	\begin{itemize}
	    \item ¿Cómo se produce la disminución de los costos promedio de una empresa? (el tipo de RI importa) 
	    \item ¿Cuáles son las fuentes de los RI? (desbordamiento de conocimientos, habilidades especializadas, vínculos ByF) 
	    \item ¿Cómo modelamos las fuentes de los RI? (aprender cómo y cuándo cambian los RI y explorar su impacto en el comportamiento de la economía)
\end{itemize}
\end{itemize}

\subsection{Significado de IRs/Economías de Escala}
\begin{itemize}
    \item Economías de escala / rendimientos crecientes a escala
	\begin{itemize}
	    \item Aumento en el nivel de producto producido implica disminución en los costos promedio por unidad de producto para la empresa 
	    \item Curva de costo promedio con pendiente negativa
	\end{itemize}
\end{itemize}

