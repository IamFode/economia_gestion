ALUMNO: Christian L. Paredes Aguilera\\

\vspace{1cm}

\begin{center}
    \large
    \textbf{EL MERCADO DE LIMONES :\\
    LA INCERTIDUMBRE DE LA CALIDAD Y EL \\
    MECANISMO DEL MERCADO}\\
    \vspace{.5cm}
    \normalsize
    \textbf{George A. Akerlof}
\end{center}

\vspace{2cm}

\begin{enumerate}[\bfseries 1.]

    %1.
    \item \textbf{Qué te ha llamado más la atención del contenido de este post?.}\\\\

	Me llamaron la atención dos puntos. El primero: Como una teoría macroeconómica. Es decir, la teoría del crecimiento y su aplicación a la educación, podrían llevar a Akerlof a escribir sobre un tema microeconómico, para luego ganar el premio Nobel. Trato de decir que podemos iniciar una investigación económica de varias perspectivas. \\
	 Akerlof estaba claro de cada argumento que escribió. Por lo que el segundo punto que me llamo la atención, es la constancia y perseverancia que tuvo para que alguna revista pueda publicar su hallazgo. \\\\

    %2.
    \item \textbf{Lee el artículo “The Market for Lemons” (QJE, 1970) y resume, con tus propias palabras las siguientes partes: la tesis (pregunta), métodos (el cómo testar la pregunta empírica o resolverla) y los resultados más relevantes.}\\\\

    El artículo "The Market for Lemons" escrito por George Akerlof y publicado en el Quarterly Journal of Economics en 1970, se centra en analizar cómo la asimetría de la información puede afectar al mercado. En particular, la tesis principal del artículo es que la asimetría de la información puede llevar a una selección adversa en el mercado, donde los bienes de baja calidad expulsan del mercado a los de alta calidad.\\

    Akerlof utiliza un modelo de mercado para ilustrar cómo funciona la selección adversa. El modelo describe un mercado en el que los vendedores tienen información completa sobre la calidad de sus bienes, mientras que los compradores solo tienen información parcial. En este contexto, los compradores estarán dispuestos a pagar un precio promedio, que refleja la calidad promedio de los bienes. Sin embargo, los vendedores de alta calidad no estarán dispuestos a vender a este precio, ya que obtendrían un precio inferior a su calidad. En cambio, solo los vendedores de baja calidad estarán dispuestos a vender a ese precio, y como resultado, el mercado se llenará de bienes de baja calidad.\\

    El resultado más relevante del artículo es que la selección adversa puede llevar a la desaparición del mercado. Cuando el mercado se llena de bienes de baja calidad, los compradores comienzan a dudar de la calidad de todos los bienes, lo que hace que disminuya la demanda. Esto a su vez hace que los vendedores de alta calidad se retiren del mercado, lo que agrava aún más la selección adversa. Por lo tanto, el artículo concluye que en los mercados donde hay asimetría de información, como el mercado de autos usados que se utiliza como ejemplo en el artículo, es necesario encontrar formas de reducir la asimetría de información para evitar la selección adversa y el colapso del mercado.

\end{enumerate}
