\chapter*{El reto de construir un contrato social progresista y sostenible}

\begin{multicols}{2}

\subsection*{Introducción}

Una de las implicaciones de la primera guerra mundial es la desigualdad. Charles Dickens en su novela $"$historia de dos ciudades$"$, hace mención a la preocupación por la desigualdad en si, no sólo par medirlo si no para estudiar los efectos y consecuencias que lo acarrean.
\begin{center} $"$Era el mejor de los mundos, era el peor de los mundos$"$ \end{center}
Yo me pregunto, realmente acabará en algún momento la desigualdad o por lo menos tendrá una reducción significativa?. A continuación veamos un instrumento que puede ser de utilizar para tal efecto.

\subsection*{Problema}

\begin{center}¿Que es lo que una democracia pluralísta con una economía de mercado sea sostenible, que funcione armoniosamente y prevenga la barbarie política de los años 30?, ¿cuales son las causas del actual y del populísmo?\end{center}

Si se tuviese una respuesta para esta pregunta, seriamos muy útiles de sugerir normas inherentes al progreso social, económico y politico. \\
Una democracia liberal y democrático, puede conciliar progreso social y democracia siempre que haya un contrato social, no un pacto, que retóricamente tiene que ser un compromiso entre aquellos a los que le va bien con el sistema de democracia liberal y aquellos que tiene el riesgo con el disfrute de quedarse atrás en la falta de ingresos y oportunidades. Este compromiso debe ser apoyado por instituciones, de donde la igualdad de oportunidades no se puede garantizar sin una buena educación. \\\\
Parte del contrato social de la izquierda de la época en los años $50's$, se basó en acuerdos de legitimización de mercados capitalistas, y por lo tanto estos años fueron gratificantes. \\
Menciona John Rwals, que los científicos y filósofos morales y politicos se preocupan por conocer y aplicar contratos sociales, todo lo contrario a los economistas; esto, tal vez, por hecho de que se cree que los mercados libres funciona armoniosamente, que por desgracia tuvo un gran éxito.\\
De todas formas y gracias a algunos indicadores se menciona que la productividad esta decreciendo por causas del contrato social. De lo que se puede inferir que el crecimiento vendrá siempre y cuando se tenga un detallado e inteligente contrato social. En la percepción de Antón, será un crecimiento dual.\\\\
Algo curioso que pasa estos últimos años es que la pandemia reciente esta trayendo una rehabilitación del mencionado contrato social. Y no así la crisis del $2008$ donde el contrato social se destruye con recortes y política fiscal contractiva contrario a lo que se ve ahora con políticas expansivas y dirigida a la educación.\\\\
Así y no menor es el hecho de que no debemos fijarlos en la prosperidad perdida, si no al miedo de perdidas futuras en temas de prosperidad siempre tratando de evitar el malestar socio-económico que desencadenaría en la segunda derivada de manifestaciones culturales.\\\\
A continuación se toca algunos puntos importantes referentes al problema.

\subsubsection*{La sostenibilidad social y democracia}
Keynes menciona que son las ideas y no los intereses son lo que al final marcan el rumbo de la sociedad,  por lo tanto debemos estar consientes de que algunas ideas de Adam Smith son fundamentales en la modelización óptima en la sociedad, más concretamente, la idea de sostenibilidad en el siglo $21$ moldeará la economía durante décadas. Por ejemplo la sostenibilidad medioambiental ya es parte del ser humano en si, pero aún no somos responsables de aquello y de lo que implica y el impacto positivo de tenerun contrato social democrático.\\
Si no somos capaces de garantizar el pleno empleo, es decir la sostenibilidad social, entonces se arriesga los temas inherentes al medio ambiente y  democracia. Debemos tomar en cuenta que todos los efectos que estimamos funcionará a través de la capacidad que tengan las democracias. La democracia es como un árbol que hecha raíces donde la raíces son la estabilidad social.

\subsubsection*{Laberintos de la prosperidad}
Cuando una sociedad percibe que no tiene oportunidades, entonces se debe reducir la desigualdad. Pero como reducir la desigualdad? Cuales son las fuentes de la desigualdad?.\\
Esto podría ser por el mal funcionamiento de la redistribución que tiene lugar desde los años $30$ y/o de la poca capacidad de las personas que están en el mercado laboral.\\\\
Una tesis relacionada está en la mesa, las fuentes mas importantes de la desigualdad son el desempleo y los salarios.  Como buscamos la prosperidad.? (Se indica un cuadro con un laberinto que nos podría llevar a esa
prosperidad deseada) Se propone una taxonomía para poder tener políticas económicas más útiles.

\subsubsection*{Una nueva epifanía económica}
Mencionemos un dilema, la ley de Ockham, que entre la busque de la eficiencia y la equidad el autor se
preocupado por la desigualdad. Pero con el único detalle de que al querer relacionarlo obtiene una función negativa, que es contrario a lo que se sabe hoy en día. A ésto se tiene una mejora razonable de la justicia social que no perjudica al estado; es más sana y sostenible en el tiempo, como una epifanía. Por lo tanto se necesita apóstoles para traer una economía más sostenible.

\subsection*{Conclusiones}
En toda medida política y económica se tienen ganadores y perdedores que se debe identificar, de todas maneras es casi un hecho que debemos investigar este tiempo en ese sentido, donde combinemos objetivos deseables para la sociedad y propongamos soluciones que la sociedad compre, en otras palabras los economistas tenemos que dirigirnos no solo a los políticos si no también a la sociedad en si. \\
Creemos en una democracia pluralista, donde  aprendamos a vivir en sociedad.

\end{multicols}
