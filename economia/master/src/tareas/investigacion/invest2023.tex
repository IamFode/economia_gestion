\chapter{Nueva investigación}

\section{Desarrollo de la pregunta de investigación}

\begin{enumerate}[1.]
    \item ¿Cuál es el tema general de investigación?
    \item ¿Cuál es la pregunta de investigación?
    \item ¿Cuál es la hipótesis de investigación?
\end{enumerate}

Empezamos a revisar la bibliografía, observando que preguntas pendientes existen o que brechas (baches) existen en la literatura. Para ello, necesito realizar búsquedas finas leyendo la introducción y las conclusiones de cada articulo de investigación.\\


\section{Organización del trabajo final y su redacción}

\begin{enumerate}[1.]
    \item Introducción
    \item Revisión de la literatura
    \item Marco teórico .- Al realizar el marco teórico se debe conceptualizar.
    \item Análisis empírico .- Modelización econométrica, estadística, etc.
    \item Conclusiones .- Lista de que se hizo, que no se hizo, que se puede hacer en el futuro, cual resultado fue mejor, que resultados obtuvimos, las debilidades del estudio, etc.
    \item Apéndices (si las hay)
    \item Referencias
\end{enumerate}

\section{Qué podemos hacer a la hora de investigación}

\begin{itemize}
    \item Se podría proponer una nueva idea.
    \item Una nueva metodología.
    \item Una nueva hipótesis.
    \item Proponer políticas nuevas con datos antiguos.
    \item Se pueden predecir el futuro con el pasado.
\end{itemize}

\section{TIPS de escritura}

Los buenos escritos de economía tomarán como punto de partida el supuesto de un comportamiento racional. Un análisis exhaustivo de cualquier comportamiento y una descripción bien escrita del mismo deben tener en cuenta los efectos de incentivo.


\section{Plan}

\begin{enumerate}[1.]
    \item Existe una investigación cuando somos capaces de identificar un tema interesante, nueva que no fue tratada o
    \item cuando existe un brecha donde no se han puesto de acuerdo los investigadores.
\end{enumerate}

Para comenzar debemos preguntarnos \textbf{¿Qué se ha hecho al respecto?}. Para ello se debe revisar la literatura.


\subsection{¿Qué es y que no es una revisión bibliográfica?}

Es una sección donde \textbf{organizas, sintetizas y explicar} la bibliografía. Se debe filtrar la información para responder mi pregunta de investigación.\\

\subsection{Efectos de una revisión de la literatura}
Debemos identificar las obras más importantes, los puntos en acuerdo y desacuerdo. La mejor manera es aplicar un \textbf{enfoque embudo}; es decir, de lo general a lo particular. También se habla del bosque y el árbol.


\subsection*{Tipos de literatura críticas en las ciencias sociales}
Llamamos a la literatura BRIX es literatura que no fue publicada en revistas científicas. \\
Revisión por pares es la revisión antes de ser publicada.\\

En concreto Journal of Economic Literature y Journal of Economic Perspectives son puntos de partida ya que se publican artículos muy maduros, este sería el punto de partida.

\subsection{A partir de una revisión de la literatura}
Podemos ir viendo la tendencia de la investigación a través de los años.

\subsection{Antes de organizar la escritura una revisión de literatura}
Debemos filtrar a través de resúmenes: Leyendo el abstract, introducción y conclusiones.\\

Las preguntas clave para la toma de notas:
\begin{enumerate}[1.]
    \item RESUMEN: ¿Cuál es el tema de la fuente, las preguntas de investigación, metodología y resultados centrales?.
    \item SÍNTESIS Y ORGANIZACIÓN: ¿Cómo esta la fuente relacionado con mis preguntas temas, tesis, y de investigación?¿Tampoco apoya ni contradice mi tesis?.
    \item EVALUACIÓN: ¿Cuáles son las fortalezas y debilidades de la investigación en la fuente? ¿Qué tan importante o influyente es esta fuente?.
    \item ¿Cómo está la fuente relacionada con otras investigaciones sobre el mismo tema? ¿Se emplea una metodología diferente? ¿Pertenecen a una población diferente, región, lapso de tiempo? ¿Funciona con un conjunto de datos diferente?
    \item ¿Cuáles son los puntos de acuerdo o desacuerdo entre la fuente y otras investigaciones sobre el mismo tema?
\end{enumerate}


\section{Escritura de una revisión de la literatura}

\begin{itemize}
    \item Debe crear claro, el \textbf{que, por que y como}. 
    \item Párrafos cortos podría ayudar. 
    \item Se puede realizar movimientos retóricos, de similitud y contradicción.
    \item Escribir de manera impersonal o en primera persona.
\end{itemize}


\section{Plan de trabajo}
\begin{itemize}
    \item ¿Cómo manejamos nuestros sentimientos?.- Debemos presentar nuestros argumentos de manera objetiva, a través de la evidencia apoyada en la literatura.
\end{itemize}

\subsection{Estrategias de Organización}
\begin{itemize}
    \item Un párrafo una idea.
    \item Cohesión, se consigue uniendo ideas relevantes y concretas.
    \item Se tiene que contar una historia.
    \item Minimizar de repeticiones.
    \item Cambiar verbos por sustantivos.
\end{itemize}



\chapter{Unas notas sobre cómo estructurar tu mensaje}

\section{¿Qué es un ensayo analítico?}
\begin{itemize}
    \item Análisis.
    \item Interpretación
    \item Persuasión.
    \item Argumentación lógica.
\end{itemize}
\begin{center}
    Afirmación central/Tesis $\rightarrow$ Razón $\rightarrow$ Evidencia.
\end{center}

\section{Estructura de un ensayo analítico}

\subsection{Trabajos empíricos}

\begin{multicols}{2}
    \begin{enumerate}[1.]
	\item Introducción.
	\item Metodología.
	\item Datos.
	\item Resultados.
	\item Discusión.
	\item Conclusión.
    \end{enumerate}

    \begin{enumerate}[1.]
	\item Introducción al tema, motivación, la tesis.
	\item contexto de la investigación.
	\item[3-4.] Modelo a estimar, las variables en la regresión, fuentes de información.
	\item[5-6.] Resumen de los estadísticos, los parámetros estimados, pruebas de hipótesis, la Interpretación.
	\item[7.] Volver a la tesis.
    \end{enumerate}
\end{multicols}

\subsection{Trabajos teóricos}

\begin{multicols}{2}
    \begin{enumerate}[1.]
	\item Introducción.
	\item Revisión de literatura.
	\item Configuración del modelo.
	\item Los resultados del modelo/análisis.	
	\item Extensiones del modelo.
	\item Conclusión.
    \end{enumerate}

    \begin{enumerate}[1.]
	\item Introducir el tema, la motivación, la tesis.
	\item Contexto de la investigación.
	\item Definiciones y supuestos.
	\item La manipulación del modelo y la Interpretación de sus implicaciones.
	\item Supuestos a extender y derivando nuevos resultados.
	\item Volver a la tesis.
    \end{enumerate}
\end{multicols}

Los trabajos empíricos son los más populares y comunes. 

\section{Resultados de la investigación}
\begin{itemize}
    \item Utilizar gráficos relevantes. 
    \item Utilizar numeración con título y fuente.
    \item El título debe ser preciso. 
    \item especificar las unidades de medida entre paréntesis.
    \item Debemos vincularla al texto.
    \item Que haya coherencia entre el texto y la gráfica.
    \item Presentar regresiones con datos completos.
    \item Debemos numerar las ecuaciones.
    \item A través de las ecuaciones podemos presentar los resultados.
\end{itemize}

En general utilizamos el formato APA.


\chapter{tratamiento de meta-análisis}
Es el estudio de múltiples estudios. Lo que realizamos es sintetizar los resultados.\\

Existe dos tipos de análisis:
\begin{itemize}
    \item Cuantitativo y
    \item cualitativo.
\end{itemize}

\section{Análisis cuantitativo}

\begin{itemize}
    \item La identificación de los estudios es crucial para saber que es lo que se hizo o lo que no. Esto evita los sesgos cómo investigador. Debemos de rechazar lo que no nos aportará algo.
    \item ¿Cómo incluyo o excluyo?. Debemos leer el abstract, introducción y conclusiones.
\end{itemize}

Un ejemplo es un estudio de clase que se hizo donde donde relacionaba las notas con la cantidad de alumnos.\\

Los pasos para realizar un meta-análisis son:
\begin{enumerate}[1.]
    \item Definir el objetivo del estudio.
    \item buscar los estudios realizados más relevantes.
    \item Definir las variables más cruciales para el estudio.
    \item Calcular estadísticas descriptivas.
    \item Correr modelos de meta-regresiones.
    \item Correr diagnósticos.
    \item Reportar resultados.
\end{enumerate}

Se debe ser lo mas claro posible.

\subsection{Búsqueda de estudios}
Tendremos sesgo al buscar estudios realizados, ya que:
\begin{itemize}
    \item El idioma será un problema.
    \item Números de citas en cada estudio.
    \item Efectos diferentes por el lugar donde se realizó el estudio.
\end{itemize}

\subsection{Variables de interés}
Por ejemplo, debemos saber cómo medir los datos de interés.

\subsubsection{Sesgos de publicación}
La simetría de los resultados será muy importante.\\

Cuando realizamos estudios sacamos variables de varios artículos y al final a través de estas variables poder correr regresiones.


\section{Análisis cualitativo}
Tecnología de conteo. Se realiza mediante el conteo se utilizo una palabra en los estudios 


\chapter{Técnicas de investigación: Encuesta}
Se realizará la metodología de recolección de datos propios (Datos que no sean públicos o no existan).\\

Se recopilará datos subjetivos.

\section{Características de una encuesta}
\begin{itemize}
    \item Tenemos que obtener una muestra lo más semejante a la población, esto podemos encontrarlo en un censo.
    \item Si no es representativa habrá sesgo y por lo tanto no se podrá inferir.
\end{itemize}

El objetivo será obtener la información que queremos.\\

Debemos considerar los siguientes parámetros:
\begin{itemize}
    \item Tipo de universo.
    \item Unidad de muestreo.
    \item Lista de fuentes (Información básica primaria).
    \item Tamaño de la muestra.
    \item Parámetros de interés (Media, mediana, etc).
    \item Restricción presupuestaria.
    \item Procedimiento de muestreo (Internet, personal, por correo).
\end{itemize}

\section{Calidad de la muestra}

\begin{itemize}
    \item Si la muestra no es aleatoria, los datos estarán correlacionados. IMPORTANTE.
    \item Si es aleatoria podemos jugar con los pesos.
    \item Debemos hacer que la muestra sea representativa.
    \item Se utiliza técnicas de cuotas ex-ante y ex-post.
    \item Debemos realizar preguntas trampas para ver quienes nos mintieron.
    \item En linea se puede saber, cuanto tiempo le costo responder.
    \item El objetivo es minimizar el error muestral.
\end{itemize}

\section{Errores en la medición}

\begin{itemize}
    \item Respuestas del encuestado: puede ser reacio a responder.
    \item Situación en la que se desarrolla la encuesta.
    \item Medición: se puede distorsionar el resultado al reordenar o reescribir las preguntas.
    \item Instrumento: el tipo de encuesta llevada a cabo puede distorsionar el resultado.
    \item Muestreo: una mala elección de la muestra.
    \item Debemos ajustarla con pesos.
    \item El método Dillman menciona que los sujetos estén involucrados en la encuentra.
\end{itemize}


\section{Método Dillman}

\begin{itemize}
    \item Realizar grupos de discusión.- Grupos reducidos (8-10).
    \begin{itemize}
	\item Debate sobre el interés del tema planteado.
	\item Comentarle que será una encuesta.
    \end{itemize}
    \item Diseño del cuestionario.
    \begin{itemize}
	\item Consentimiento del entrevistado.
	\item Preguntas generales o de calentamiento.
	\item Preguntas más difíciles.
	\item Preguntas concretas.
	\item En que año naciste en vez de decir la edad.
	\item Trabajas en ONG o quisieras trabajar, esto para no preguntarle si ayudar es bueno.
	\item No pongas desempleo, si no en búsqueda de empleo.
    \end{itemize}
\end{itemize}

\subsection{Pasos a seguir en un estudio con encuesta}
\begin{itemize}
    \item Presentar el problema que trata de estudiar en un grupo de discusión.
    \item Discutir y diseñar la encuesta.
    \item Presentar la encuesta en grupos de discusión.
    \item Realizar la encuesta.
    \item Obtener e interpretar los resultados.
    \item Validar tus resultados.
\end{itemize}


\subsection{Modo de administración.}

\begin{itemize}
    \item Cara a cara.
    \item Por correo.
    \item Por teléfono.
    \item A través de internet.
\end{itemize}


\section{Encuesta}

Tema: \textbf{Incidencia del covid en la sociedad.}

\begin{itemize}
    \item Calidad de datos.
    \item Debe ser una encuesta mayor a 18 años.
    \item Tener cuidado con la forma de recoger datos.
    \item Genero inclusivo.
    \item Religión.
    \item Asegurar que todos se sientan cómodos.
    \item Generar un informe.
\end{itemize}


\section{Técnicas cualitativas}

\begin{itemize}
    \item Están en auge gracias a la inteligencia artificial. Por ejemplo, el análisis de texto.
    \item También existe los métodos híbridos. Donde la maquina y el humano trabajan juntos.
\end{itemize}

\subsection{¿Por qué una investigación cualitativa?}

\begin{itemize}
    \item Para entender el comportamiento humano.
    \item Se tiene una perspectiva mas informada.
\end{itemize}


\subsection{Instrumentos para investigación cualitativa}

Tipos de entrevistas

\begin{itemize}
    \item Estructuradas.
    \item Semi-estructuradas.
    \item No estructuradas.
\end{itemize}

\subsubsection{Entrevistas en profundidad}

\begin{itemize}
    \item Se graba la entrevista.
    \item Preparar el sketch.
    \item Se debe tener un guion.
    \item Preparados para cualquier respuesta.
    \item Consentimiento de la persona.
\end{itemize}


\paragraph{Entrevista semi-estructurada}

\begin{itemize}
    \item Se tiene información gestual y verbal.
    \item Abanico de cuestiones.
\end{itemize}

\subsubsection{Reunion de grupo}
\begin{itemize}
    \item Sacamos temas de forma estructurada.
    \item Grabación con previo consentimiento.
    \item Confianza.
    \item Empezamos con una presentación de nosotros y explicando el objetivo.
    \item Preguntas sencillas y genéricas del tema.
    \item Dar las gracias.
    \item Debemos mezclar hombres y mujeres.
    \item Aseguramos con todos estén cómodos.
    \item Incorporamos a todos a responder para tener una perspectiva conjunta.
\end{itemize}

El orden de las preguntas es MUY IMPORTANTE. Por ejemplo, se podría preguntar cuanto esta a gusto con su matrimonio, luego preguntamos cuan a gusto esta con su vida en general. En encuestado podría responder en función a la primera pregunta.


\subsection{Caso de estudio}

\begin{itemize}
    \item Relatamos una historia, describiendo y evaluando.
    \item Debemos verificar y resaltar si los resultados obtenidos son verdaderos comparando con otras encuestas y estudios.
    \item Se articulan de distintas maneras.
    \item Debe ser una encuesta que se debe reproducir en cualquier parte del mundo.
\end{itemize}


\begin{itemize}
    \item Los universos deben ser aleatorios.
    \item Hipótesis de la encuesta.
    \item Cómo lo motivamos.
\end{itemize}


\chapter{Cómo preparar y preparar una presentación}

\subsubsection{El problema}
\begin{itemize}
    \item Poner atención a la audiencia.
\end{itemize}

\subsubsection{Comunicación efectiva}
\begin{itemize}
    \item Mantener la atención entreteniendo de la mejor manera al público.
\end{itemize}

\subsubsection{Reforzar la visualización}
\begin{itemize}
    \item Asegurar que el público entienda el mensaje expresándote de la mejor manera.
\end{itemize}

\subsubsection{¿Cómo lo hacemos?}
\begin{itemize}
    \item Preparandote exahustivamente.
    \item Con una buena estructura en la presentación.
    \item Practicar y ensayar delante de un  espejo, para ver tus gestos y movimientos.
\end{itemize}

\subsubsection{Planificando}
\begin{itemize}
    \item Conocer bien el trabajo que se va a tratar.
    \item Contar cosas útiles.
    \item Ilustrarlo con un slides por minuto.
    \item Las conclusiones nunca mas de 4 mensajes.
\end{itemize}

\subsubsection{Estructura de la presentación}

\begin{enumerate}[\bfseries 1)]
    \item Apertura.
	\begin{itemize}
	    \item Empieza fuerte y termina fuerte.
	    \item Esto es lo interesante.
	    \item Convencer que sabemos del tema.
	    \item El problema a tratar.
	    \item Que se sabe.
	    \item La estructura fundamental del examen, el objetivo.
	    \item Pequeño resumen de lo que se presentará.
	\end{itemize}
    \item Cuerpo.
	\begin{itemize}
	    \item Habla fácil.
	\end{itemize}
    \item Resumen.
\end{enumerate}


\subsubsection{Slides}
\begin{itemize}
    \item Slides con imágenes.
    \item Usar gama de colores consistente.
    \item Trata de evitar demasiadas ecuaciones.
    \item Práctica los tiempos.
    \item Habla pero no leas.
    \item Ser natural.
\end{itemize}


\subsubsection{¿Cómo perdemos a la audencia?}
\begin{itemize}
    \item Trata de resumir.
    \item De ser conciso.
    \item Hablar espaciado ni muy lento ni rápido.
\end{itemize}

\subsubsection{Lidiar con el nerviosismo}
\begin{itemize}
    \item prepararse.
    \item Nunca pasarse del tiempo.
\end{itemize}


\subsubsection{Ronda de preguntas}
\begin{itemize}
    \item Articular bien las respuestas.
    \item Presentación de 10 minutos, 5 minutos de preguntas.
    \item Si no se sabe una pregunta, se dice que no se sabe que se investigará más tarde y se agradecerá por la pregunta.
\end{itemize}




