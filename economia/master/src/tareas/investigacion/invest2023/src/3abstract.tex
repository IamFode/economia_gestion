% Here goes the abstract
\begin{abstract}[SUMMARY]
Este estudio se centra en analizar el desempeño histórico de las carteras recomendadas por los principales bancos de inversión a nivel internacional, considerando los perfiles de riesgo de sus clientes y la estrategia de inversión utilizada (activa o pasiva). El objetivo es determinar si las carteras recomendadas han sido congruentes con los perfiles de riesgo y si han logrado un rendimiento acorde. Para ello, se utilizarán datos de rendimiento y riesgo de índices de acciones y bonos, así como la composición de las carteras recomendadas por los bancos.

El análisis se llevará a cabo a lo largo de períodos de 3, 5 y 10 años, evaluando si las carteras recomendadas se encuentran en la frontera eficiente y si el riesgo asumido se ajusta a los rendimientos obtenidos. Además, se compararán las carteras recomendadas con carteras de activos equivalentes, como fondos mutuales y ETFs, para determinar cuál estrategia resulta más beneficiosa según el perfil de riesgo y el horizonte de inversión de los clientes.

Se anticipa que este estudio arrojará luz sobre la efectividad de las carteras recomendadas por los bancos de inversión y si las estrategias activas o pasivas han tenido un mejor desempeño histórico. Los resultados proporcionarán información valiosa para los inversores y las instituciones financieras en la toma de decisiones de inversión.
\end{abstract}

\begin{keywords}
    Finanzas \sep Cartera \sep Bancos \sep Riesgo.
\end{keywords}

\maketitle
