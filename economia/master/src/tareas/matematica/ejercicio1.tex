\section*{\center Entrega 1}
\vspace*{1cm}

\begin{enumerate}

    \item[\bfseries Problema 1.] \textbf{\boldmath Calcule los autovalores y los autovectores de la matriz $A$ y halle $A^n$, con $n\in \mathbb{N}:$}\\

    $$A=\begin{pmatrix}
	0 & -1 & -1 \\
	-1 & -2 & -1 \\
	1 & 3 & 2 
    \end{pmatrix}$$\\\\

    \textbf{Respuesta.-}\;  Por definición, y la regla de Sarrus el polinomio característica de $A$ es:
    
    $$ \begin{array}{rcl} 
	\det \left[
    \begin{pmatrix}
	0 & -1 & -1 \\
	-1 & -2 & -1 \\
	1 & 3 & 2 
    \end{pmatrix} - \alpha 
    \begin{pmatrix}
	1 & 0 & 0 \\
	0 & 1 & 0 \\
	0 & 0 & 1 
\end{pmatrix} \right] & = & 
    \begin{vmatrix}
	-\alpha & -1 & -1 \\
	-1 & -2-\alpha & -1 \\
	1 & 3 & 2-\alpha 
	\end{vmatrix} \\\\ 
	&=&  
	\begin{vmatrix}
	    -\alpha & -1 & -1 \\
	    -1 & -2-\alpha & -1 \\
	    1 & 3 & 2-\alpha 
	\end{vmatrix} \hspace{.4cm} 
	\begin{matrix}
	     -\alpha & -1 \\
	     -1 & -2-\alpha \\
	     1 & 3 \\ 
	\end{matrix}\\\\
	&=&\left[-\alpha (-2-\alpha)(2-\alpha)\right] + \left[(-1)(-1)1\right] + \left[(-1)(-1)3\right]\\\\
	&-&\left[1(-2-\alpha)(-1)\right]-\left[3(-1)(-\alpha)\right]-\left[(2-\alpha)(-1)(-1)\right]\\\\
	&=&4\alpha-\alpha^3 + 1 + 3 - 2 - \alpha - 3\alpha - 2+\alpha\\\\
	&=& -\alpha(\alpha^2-1)\\
    \end{array}$$\\

    Luego igualamos el último resultado a $0$, de donde obtenemos,\\
    $$-\alpha(\alpha^2-1)=0$$ 
    así los autovalores estarán dados por,
    $$\alpha=0 \;\; \lor\;\; \alpha=1 \;\;\lor\;\; \alpha=-1$$\\

    Ahora calculemos los autovectores.

    \begin{itemize}

	\item Para $\alpha=0$
	    $$
	    \begin{pmatrix}
		0 & -1 & -1 \\
		-1 & -2-0 & -1 \\
		1 & 3 & 2-0
	    \end{pmatrix}  
	    \begin{pmatrix}
		v_1 \\
		v_2 \\
		v_3
	    \end{pmatrix} = 
	    \begin{pmatrix}
		0 & -1 & -1 \\
		-1 & -2 & -1 \\
		1 & 3 & 2
	    \end{pmatrix}  
	    \begin{pmatrix}
		v_1 \\
		v_2 \\
		v_3
	    \end{pmatrix} = 
	    \begin{pmatrix}
		0 \\
		0 \\
		0	
	    \end{pmatrix} 
	    $$

	    $$\left.\begin{array}{rcl}
		    0\cdot v_1 -1v_2 -1v_3&=&0\\
			    -1v_1-2v_2-1v_3&=&0\\
			    1v_1+3v_2+2v_3&=&0
		\end{array}\right\} \Longleftrightarrow v_2=-v_3 \; \land \; v_1=v_3 \;\land \; v_3=v_3  \Longleftrightarrow  
		\begin{pmatrix}
		    x_3 \\
		    -x_3 \\
		    x_3
		\end{pmatrix}=v_3
		\begin{pmatrix}
		    1 \\
		    -1 \\
		    1 
		\end{pmatrix}
		$$

		Sea $v_3=1$ entonces el autovector para $\alpha=0$ es
		$\begin{pmatrix}
		    1 \\
		    -1 \\
		    1 
		\end{pmatrix}$\\\\

	\item Para $\alpha=1$
	    $$
	    \begin{pmatrix}
		-1 & -1 & -1 \\
		-1 & -2-1 & -1 \\
		1 & 3 & 2-1
	    \end{pmatrix}  
	    \begin{pmatrix}
		v_1 \\
		v_2 \\
		v_3
	    \end{pmatrix} = 
	    \begin{pmatrix}
		-1 & -1 & -1 \\
		-1 & -3 & -1 \\
		1 & 3 & 1 
	    \end{pmatrix}  
	    \begin{pmatrix}
		v_1 \\
		v_2 \\
		v_3
	    \end{pmatrix} = 
	    \begin{pmatrix}
		0 \\
		0 \\
		0	
	    \end{pmatrix} 
	    $$

	    $$\left.\begin{array}{rcl}
		    -1 v_1 -1v_2 -1v_3&=&0\\
			    -1v_1-3v_2-1v_3&=&0\\
			    1v_1+3v_2+1v_3&=&0
		\end{array}\right\} \Longleftrightarrow v_2=0 \; \land \; v_1=-v_3 \;\land \; v_3=v_3  \Longleftrightarrow  
		\begin{pmatrix}
		    -x_3 \\
		    0 \\
		    x_3
		\end{pmatrix}=v_3
		\begin{pmatrix}
		    -1 \\
		    0 \\
		    1 
		\end{pmatrix}
		$$

		Sea $v_3=1$ entonces el autovector para $\alpha=1$ es
		$\begin{pmatrix}
		    -1 \\
		    0 \\
		    1 
		\end{pmatrix}$\\\\

	\item Para $\alpha=-1$
	    $$
	    \begin{pmatrix}
		1 & -1 & -1 \\
		-1 & -2+1 & -1 \\
		1 & 3 & 2+1
	    \end{pmatrix}  
	    \begin{pmatrix}
		v_1 \\
		v_2 \\
		v_3
	    \end{pmatrix} = 
	    \begin{pmatrix}
		1 & -1 & -1 \\
		-1 & -1 & -1 \\
		1 & 3 & 3 
	    \end{pmatrix}  
	    \begin{pmatrix}
		v_1 \\
		v_2 \\
		v_3
	    \end{pmatrix} = 
	    \begin{pmatrix}
		0 \\
		0 \\
		0	
	    \end{pmatrix} 
	    $$

	    $$\left.\begin{array}{rcl}
		     v_1 -v_2 -v_3&=&0\\
			    -v_1-v_2-v_3&=&0\\
			    v_1+3v_2+3v_3&=&0
		\end{array}\right\} \Longleftrightarrow v_2=-v_3 \; \land \; v_1=0 \;\land \; v_3=v_3  \Longleftrightarrow  
		\begin{pmatrix}
		    0 \\
		    -x_3 \\
		    x_3
		\end{pmatrix}=v_3
		\begin{pmatrix}
		    0 \\
		    -1 \\
		    1 
		\end{pmatrix}
		$$

		Sea $v_3=1$ entonces el autovector para $\alpha=-1$ es
		$\begin{pmatrix}
		    0 \\
		    -1 \\
		    1 
		\end{pmatrix}$\\\\
	
    \end{itemize}

    Por último hallemos $A^n$\\

    Sea $D=\mbox{diag}(0,1,-1)=
    \begin{pmatrix}
	0 & 0 & 0 \\
	0 & 1 & 0 \\
	0 & 0 & -1 
    \end{pmatrix}, \qquad 
    V = \begin{pmatrix}
	1 & -1 & 0 \\
	-1 & 0 & -1 \\
	1 & 1 & 1
    \end{pmatrix}
    $

    Necesitaremos también encontrar la inversa de $V$ por lo que se podrá calcular aplicando,

    $$V^{-1}=\dfrac{\left(\mbox{adj}(A)\right)^T}{\det(V)}\cdot $$

    Por lo que el resultado será:

    $$V^{-1} = \begin{pmatrix}
	1 & 1 & 1 \\
	0 & 1 & 1 \\
	-1 & -2 & -1 
	\end{pmatrix}$$

    Dado que $A^n = VD^n V^{-1}$, y aplicando la multiplicación de matrices obtendremos, 

    $$A^n = \begin{pmatrix}
	1 & -1 & 0 \\
	-1 & 0 & -1 \\
	1 & 1 & 1
    \end{pmatrix} \times \begin{pmatrix}
	0 & 0 & 0 \\
	0 & 1 & 0 \\
	0 & 0 & -1)^n 
    \end{pmatrix} \times \begin{pmatrix}
	1 & 1 & 1 \\
	0 & 1 & 1 \\
	-1 & -2 & -1 
	\end{pmatrix} = \begin{pmatrix}
	0 & -1 & -1 \\
	(-1)^n & 2(-1)^n & (-1)^n\\
	-(-1)^n & 1-2(-1)^n & -1(-1)^n
    \end{pmatrix}$$\\\\




    \item [\bfseries Problema 2.] \textbf{Resuelva los siguientes problemas de optimización:}\\\\
	Sea $f(x,y,z)=(x-z-1)^2+(y-x)^2 + z^2.$\\\\

	\begin{enumerate}[\bfseries a)]

	    %-------------------- a) --------------------
	    \item $\min f(x,y,z)$.\\\\
		\textbf{Respuesta.-}\; Vemos que el dominio de definición es $\mathbb{R}^3$, es abierto y $f$ es de clase $1$  en $\mathbb{R}^3$.\\\\

		Ahora calculamos las derivadas parciales de cada variable.\\

		$$\begin{array}{rcl}
		    \dfrac{\partial f}{\partial x} \left[(x-z-1)^2+(y-x)^2 + z^2 \right]&=&\dfrac{\partial f}{\partial x}\left(2x^2-2xz-2xy-2x+2z^2+2z+y^2+1\right)\\\\
											&=&4x-2z-2y-2=0\\\\
											&&\\
		    \dfrac{\partial f}{\partial y}\left(2x^2-2xz-2xy-2x+2z^2+2z+y^2+1\right)&=&2y-2x=0\\\\
											&&\\
		    \dfrac{\partial f}{\partial y}\left(2x^2-2xz-2xy-2x+2z^2+2z+y^2+1\right)&=&4z+2-2x=0\\\\
											    &&\\
		\end{array}$$

		de donde $$y=x,\qquad z=\dfrac{x-1}{2}$$
		Así
		$$x=1,\qquad y=1,\qquad z=0$$\\
		Por lo tanto los valores que mínimizan a la función son:
		$$(x,y,z)=(1,1,0)$$\\

		$f$ es de clase $2$ en $\mathbb{R}^3$ dado que sus derivadas parciales en segundo orden existen y son continuas en $\mathbb{R}^3$ (abierto y convexo).\\

		$$Hf(x,y,z) = \begin{pmatrix}
		    4&-2&-2\\
		     -2&2&0\\
		       -2&0&4
		\end{pmatrix}$$\\

		Sabemos que la función será convaca si y sólo si la hessiana de $f$ definida es negativa y será convexa si será positiva.\\

		Calculando los autovalores de $Hf(x,y,z):$


		$$\det\begin{pmatrix}
		    4-\alpha&-2&-2\\
		     -2&2-\alpha&0\\
		       -2&0&4-\alpha
		\end{pmatrix} = -\alpha^3 + 10\alpha^2 - 24\alpha + 8=0$$\\

		por lo que las soluciones estarán dadas por:

		$$\begin{array}{rcl}
		    \alpha&=&\dfrac{2\left[-\sqrt{7}\cos\left(\dfrac{\mbox{arccos}\left(\frac{\sqrt{7}}{14}\right)}{3}\right)-\sqrt{21}\sen\left(\dfrac{\mbox{arcos}\left(\frac{\sqrt{7}}{14}\right)}{3}\right)+5\right]}{3} = 0.3912\\\\
		    \alpha&=&\dfrac{2\left[2\sqrt{7}\cos\left(\dfrac{\arccos\left(\frac{\sqrt{7}}{3}\right)}{3}\right)+5\right]}{3}=6.49396\\\\
		    \alpha&=&\dfrac{2\left[\sqrt{21}\sen\left(\dfrac{\mbox{arccos}\left(\frac{\sqrt{7}}{14}\right)}{3}\right)-\sqrt{7}\cos\left(\dfrac{\mbox{arcos}\left(\frac{\sqrt{7}}{14}\right)}{3}\right)+5\right]}{3} = 3.10992\\\\
		\end{array}$$

		Viendo que las 3 raíces son positivas entonces la $Hf(x,y,z)$ es definida positiva para todo $(x,y,z)\in \mathbb{R}^3$ por lo que llegamos a la conclusión que es estrictamente convexa en $\mathbb{R}^3$. Así aplicando la condición suficiente de extremo global, $f$ alcanza en $(x,y,z)=(1,1,0)$ un mínimo global estricto. Y el $\min f(x,y,z) = f(1,1,0)=0.$\\\\

	    %-------------------- b) -------------------
	    \item $\min\lbrace f(x,y,z) \, : \, x-y-z-2=0,\; x+y-2z-1=0\rbrace$\\\\
		\textbf{Respuesta.-}\; Sean $g(x,y,z)=x-y-z-4=0$ y $h(x,y,z)=x+y-2z-1=0$. El conjunto de soluciones factibles es:
		$$S=\lbrace(x,y,z)\in \mathbb{R}^3: (x-z-1)^2+(y-x)^2 + z^2\rbrace$$

		El rango de la jacobiana de las funciones estará dada por,
		$$rango\; J(g,h)(x,y,z)=rango\;\begin{pmatrix}
		    1&-1&-1\\
		    1&1&-2\\
		\end{pmatrix}$$

		Si escogemos 

		$$\det \begin{pmatrix}
		    1&-1\\
		     1&1\\
		 \end{pmatrix} = 2 \neq 0\; \Longrightarrow rango\; J(g,h)(x,y,z)=2\; \forall(x,y,z)\in \mathbb{R}^3 \Longrightarrow \forall(x,y,z)\in S.$$
		  Lo que implica que todo punto factible es regular.\\\\

		Sabemos que el dominio de definición de $f,g,h$ es $\mathbb{R}^3$, el cual es abierto. Estas funciones son del clase 1 en $\mathbb{R}^3$. Entonces la función lagrangiana es:\\
		$$L(x,y,z,\lambda,\mu) = (x-z-1)^2+(y-x)^2 + z^2 + \lambda(x+y+z+4)+\mu(x+y-z-4)$$\\
		
		Ahora calculamos los puntos críticos derivando la función lagrangiana.\\

		$$\begin{array}{rcl}
		    \dfrac{\partial L}{\partial x}&=&2x-2z+\lambda + \mu -2 = 0\\\\
		    \dfrac{\partial L}{\partial y}&=&2y-2z+\lambda + \mu = 0\\\\
		    \dfrac{\partial L}{\partial z}&=&2-\mu + \lambda +6z-2y-2x=0\\\\
		    \dfrac{\partial L}{\partial x}&=&x-y-z-4=0\\\\
		    \dfrac{\partial L}{\partial x}&=&x+y-2z-1=0\\\\
		\end{array}$$

		Resolviendo el sistema se tiene:

		$$x=-2,\qquad y=-3,\qquad z=-3,\qquad \lambda=3,\qquad \mu = -3$$\\

		Por lo tanto, el único punto crítico es $\left(-2,-3,-3\right)$ con multiplicadores asociados $\lambda=3$ y $\mu=-3$.\\\\

		Aplicando el teorema del caso convexo, que sería la condición de extremo global. $f,g$ y $h$ son de clase 1 en $\mathbb{R}^3$ abierto y convexo. Luego $g$ y $h$ son lineales y $f$ es de clase 2 en $\mathbb{R}^3$. Así la hessiana de $f$ es:
		$$Hf(x,y,z)=\begin{pmatrix}
		    2&0&-2\\
		    0&2&-2\\
		    -2&-2&6\\
		\end{pmatrix}$$

		Ahora la clasificamos mediante los menores principales:

		$$\triangle_1 = 2>0,\qquad \triangle_2 = \det \begin{pmatrix}2&0\\0&2\end{pmatrix} = 4>0,\qquad \triangle_3=\det(Hf(x,y,z))=8>0, \quad \forall (x,y,z)\in \mathbb{R}^3$$

		Por lo que es definida positiva $\forall(x,y,z)\in \mathbb{R}^3$ por lo que es estrictamente convexa en $\mathbb{R}^3$. En consecuencia, $f$ alcanza en $\left(-2,-3,-3\right)$ un mínimo global estricto sobre $S$ y $\min(S)=f\left(-2,-3,-3\right)=10$\\\\


	    %-------------------- c) -------------------
	    \item $\min \lbrace f(x,y,z) \, : \, x-y-z-2\leq 0,\; x+y-2z-1\leq 0 \rbrace$\\\\	
		\textbf{Respuesta.-}\; $f,g$ y $h$ son funciones de clase 1 en $\mathbb{R}^3$. Luego dado que las restricciones son lineales, se verifica la cualificación de Karlin, por lo tanto, todo punto factible es regular.\\

		La lagrangiana será la misma que en el inciso (b).
		$$L(x,y,z,\lambda,\mu) = (x-z-1)^2+(y-x)^2 + z^2 + \lambda(x-y-z-2)+\mu(x+y-2z-1)$$\\

		Las condiciones de Kuhn-Tucker son:

		$$\begin{array}{rcl}
		    \dfrac{\partial L}{\partial x}&=&2x-2z+\lambda + \mu -2 = 0\\\\
		    \dfrac{\partial L}{\partial y}&=&2y-2z+\lambda + \mu = 0\\\\
		    \dfrac{\partial L}{\partial z}&=&2-\mu + \lambda +6z-2y-2x=0\\\\
						  &&x-y-z-2\leq0,\quad \lambda\geq 0,\quad \lambda(x-y-z-2)=0\\\\
						  &&x+y-2z-1\leq 0, \quad \mu\geq 0,\quad \mu(x+y-2z-1)=0\\\\
		\end{array}$$

		Ahora resolvemos el sistema por casos.\\


		\begin{enumerate}[1.]

		    \item $\lambda = \mu = 0$\\
			$$\begin{array}{rcl}
			    \dfrac{\partial L}{\partial x}&=&2x-2z+\lambda + \mu -2 = 0\\\\
			    \dfrac{\partial L}{\partial y}&=&2y-2z+\lambda + \mu = 0\\\\
			    \dfrac{\partial L}{\partial z}&=&2-\mu + \lambda +6z-2y-2x=0\\\\
							  &&\lambda= 0\\\\
							  &&\mu = 0\\\\
			\end{array}$$


			de donde la solución del sistema estará dado por:

			$$x=1,\quad y=0,\quad z=0,\quad \lambda=0,\quad \mu=0$$
		    
			Ahora veamos las demás restricciones reemplazando en las funciones $g$ y $h$. 

			$$g(1,0,0)=1-0-0-4\leq 0, \qquad h(1,0,0)=1+0-2\cdot 0 - 1 \leq 0$$
			
			Por lo tanto $(1,0,0)$ es un punto critico (para mínimo) ya que  verifica las condiciones de KKT. \\\\


		    \item $\lambda = 0,\; \mu = x+y-2z-1=0$\\

			$$\begin{array}{rcl}
			    \dfrac{\partial L}{\partial x}&=&2x-2z+\lambda + \mu -2 = 0\\\\
			    \dfrac{\partial L}{\partial y}&=&2y-2z+\lambda + \mu = 0\\\\
			    \dfrac{\partial L}{\partial z}&=&2-\mu + \lambda +6z-2y-2x=0\\\\
							  &&\lambda= 0\\\\
							  &&x+y-2z-1 = 0\\\\
			\end{array}$$

			Resolviendo el sistema, se obtiene:

			$$x=1,\quad y=0,\quad z=0,\quad \lambda=0,\quad \mu=0$$

			De la misma forma que el caso 1 se verifica que $(1,0,0)$ es un punto critico (para mínimo) ya que verifica las condiciones de KKT. \\\\


		    \item $\lambda = x-y-z=0,\; \mu = 0$\\

			$$\begin{array}{rcl}
			    \dfrac{\partial L}{\partial x}&=&2x-2z+\lambda + \mu -2 = 0\\\\
			    \dfrac{\partial L}{\partial y}&=&2y-2z+\lambda + \mu = 0\\\\
			    \dfrac{\partial L}{\partial z}&=&2-\mu + \lambda +6z-2y-2x=0\\\\
							  &&x-y-z-2= 0\\\\
							  &&\mu= 0\\\\
			\end{array}$$

			De donde la solución del sistema estará dado por:
			$$x=-\dfrac{1}{3},\qquad y=-\dfrac{4}{3},\qquad z=-1,\qquad \lambda = \dfrac{2}{3},\qquad \mu=0$$

			Ahora verificamos que se cumpla las condiciones de Kuhn-Tucker en las funciones $g$ y $h$.

			$$g\left(-\dfrac{1}{3},-\dfrac{4}{3},-1\right)=0,\qquad h\left(-\dfrac{1}{3},-\dfrac{4}{3},-1\right)=-0.666$$

			Por lo tanto $(-\dfrac{1}{3},-\dfrac{4}{3},-1)$ es un punto critico (para mínimo) ya que verifica las condiciones de KKT.\\\\


		    \item $\lambda = x-y-z=0,\; \mu = x+y-2z-1=0$\\

			$$\begin{array}{rcl}
			    \dfrac{\partial L}{\partial x}&=&2x-2z+\lambda + \mu -2 = 0\\\\
			    \dfrac{\partial L}{\partial y}&=&2y-2z+\lambda + \mu = 0\\\\
			    \dfrac{\partial L}{\partial z}&=&2-\mu + \lambda +6z-2y-2x=0\\\\
							  &&x-y-z-2= 0\\\\
							  &&x+y-2z-1 = 0\\\\
			\end{array}$$

			La solución del sistema es:
			$$x=0,\qquad y=-1,\qquad z=-1,\qquad \lambda=1,\qquad \mu=-1$$

			Luego, ya que $\mu$ no cumple con la condicion de Kuhn-Tucker, entonces no es candidato para mínimo ni máximo.

		\end{enumerate}

	\end{enumerate}

\end{enumerate}




