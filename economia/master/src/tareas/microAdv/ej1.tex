\begin{center}
    \textbf{Teoría Clásica del Consumidor. Fundamentos.}
\end{center}

\begin{center}
\textbf{Prof. Javier Hervés Estévez}
\end{center}

\begin{center}
\textbf{Alumno: Christian Limbert Paredes Aguilera}
\end{center}

\begin{center}
	\rule{0.5\textwidth}{0.4pt}
\end{center}
\vspace{1cm}

\begin{enumerate}[\bfseries I.]

    %-------------------- I.
    \item \textbf{Conjunto de consumo.}\;\\

	\begin{enumerate}[\bfseries 1.]

	    %---------- 1.
	    \item \textbf{\boldmath Demuestre que $X=\left\{(x,y)\in \mathbb{R}^2_+|x\geq y\right\}$ cumple todas las propiedades que se le requieren a un conjunto de consumo.\\\\
		Demostración.-}\; Por el supuesto 1.1 (las propiedades del conjunto de consumo, Jehle and Reny), debemos demostrar que:
		$$
		\begin{array}{cl}
		    (a) & X \subseteq \mathbb{R}^2_+.\\
		    (b) & X \mbox{ es cerrado.}\\
		    (c) & X \mbox{ es convexo.}\\
		    (d) & 0 \in X
		\end{array}
		$$

		Por definición de inclusión, tenemos

		$$(a,b)\in X \quad \Rightarrow \quad (a,b)\in \mathbb{R}^2_+.$$

		Tal que, 

		$$a\geq b \quad \Rightarrow \quad a-b\geq 0.$$\\
		Con lo que cubrimos las demostraciones (a), (b) y (c). Es decir, si tomamos un par ordenado cualquiera $(a,b)$ de $X$, entonces se garantiza que el conjunto es no vacío y cerrado, esto ya que $(a,b)\in\mathbb{R}_+^2\, |\, a-b\geq 0$. Este último también nos menciona que $0\in X$.\\

		Ahora, veamos la demostración de (c). Por definición de convexidad sean $(a,b)\in X$, y $\lambda$ tal que $0<\lambda<1$, entonces
		$$(1-\lambda)a+\lambda b$$

	\end{enumerate}

    %-------------------- II.
    \item \textbf{Conjunto presupuestario.}\;\\

	\begin{enumerate}[\bfseries 1.]

	    %---------- 1.
	    \item \textbf{Defina conjunto presupuestario en función de los recursos iniciales (la renta es el alor de dichos recursos iniciales). Compruebe que el conjunto presupuestario no cambia si los precios se duplican.\\\\
		Demostración.-}\; Supongamos que nuestra recta presupuestaria original es
		$$p_1x_1+p_2x_2=m.$$
		Supongamos que ambos precios se multiplican por $t$:
		$$tp_1 x_1+tp_2 x_2=m.$$
		Pero esta ecuación es igual que
		$$p_1x_1+p_1x_2=\dfrac{m}{t}.$$
		Por lo tanto, multiplicar ambos precios por una cantidad constante $t$ es exactamente lo mismo que dividir la renta por la misma constante, de lo que se deduce que si multiplicamos por $t$ tanto los precios como la renta, la recta presupuestaria no varía en absoluto.\\\\

	    %---------- 2.
	    \item \textbf{\boldmath Suponga que dos agentes disponen de renta $m_A=10$ y $m_B=50$ respectivamente y que ambos se enfrentan a precios $(p_1,p_2)=(1,2)$. Considere dos esquemas fiscales alternativos:} 
		
		\begin{enumerate}[\bfseries a)]

		    \item \textbf{\boldmath Escriba la restricción y dibuje la recta presupuestaria para ambos agentes antes y después de un impuesto sobre la renta del $40\%$.\\\\
			Respuesta.-}\;

		    \item \textbf{\boldmath Realice el mismo ejercicio pero con un impuesto $"$de suma fija$"$ de $T=8$ para ambos agentes.\\\\
			Respuesta.-}\;

		    \item \textbf{\boldmath Piense en cómo afectan ambos impuestos a la capacidad adquisitiva de los agentes. Calcule qué porcentaje representa la renta del agente $A$ respecto de la del agente $B$ antes y después de ambos impuestos.\\\\
			Respuesta.-}\;

		\end{enumerate}

	\end{enumerate}

\end{enumerate}


