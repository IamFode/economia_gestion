\begin{center}
    \textbf{Equilibro general}
\end{center}

\begin{center}
\textbf{Prof. Carlos Hervés Beloso}
\end{center}

\begin{center}
\textbf{Alumno: Christian Limbert Paredes Aguilera}
\end{center}

\begin{center}
	\rule{0.5\textwidth}{0.4pt}
\end{center}
\vspace{1cm}

\begin{enumerate}[\textbf{Ejercicio} \bfseries 1.]

    %-------------------- 1.
    \item \textbf{Explica por qué crees que el concepto de equilibrio de Walras es relevante en la teoría económica. Con este ejercicio se pretende que el alumno destaque las propiedades del equilibrio en tres líneas usando en cada de ellas una de estas palabras: eficiencia, bienestar y descentralizar.}\\\\
	\textbf{Respuesta:} El equilibrio de Walras promueve la \textbf{eficiencia} en la asignación de recursos, ya que no se puede mejorar el bienestar de un individuo sin empeorar el de otro. Además, este equilibrio maximiza el \textbf{bienestar} social al despejar los mercados al precio que iguala la oferta y la demanda, reflejando así las preferencias de los consumidores. Finalmente, permite la \textbf{descentralización} de las decisiones económicas, ya que cada agente actúa de manera independiente, guiado por los precios del mercado.\\\\

    %-------------------- 2.
    \item \textbf{Demuestra que si las preferencias son continuas y monótonas (más es mejor), los conceptos de veto fuerte y débil son equivalentes.}\\\\
	\textbf{Respuesta:} Para demostrar la equivalencia entre los conceptos de veto fuerte y débil, primero asumimos que las preferencias son continuas y monótonas. Luego, procedemos a demostrar la equivalencia de ambas definiciones:\\

	Supongamos que un agente tiene un veto débil sobre un bien. Esto significa que existe una cantidad positiva del bien tal que el agente la prefiere a ninguna cantidad del mismo. Formalmente, sea \( x^* \) la cantidad mínima del bien tal que $x^* \succ 0$, donde $\succ$ denota la relación de preferencia estricta. Dado que las preferencias son continuas, cualquier cantidad mayor que $x^*$ también será preferida a ninguna cantidad del bien. Por lo tanto, el agente tiene un veto fuerte sobre el bien.\\

	Por otro lado, si un agente tiene un veto fuerte sobre un bien, entonces por definición, prefiere cualquier cantidad positiva del bien a ninguna cantidad del mismo. Formalmente, si $x' \succ 0$, donde $x'$ es cualquier cantidad positiva del bien, entonces el agente tiene un veto débil sobre el bien.\\

	Dado que hemos demostrado que un agente que tiene un veto débil también tiene un veto fuerte, y viceversa, concluimos que los conceptos de veto fuerte y débil son equivalentes cuando las preferencias son continuas y monótonas.
\end{enumerate}
