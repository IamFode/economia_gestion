\documentclass[10pt]{article}
\usepackage[text=17cm,left=2.5cm,right=2.5cm, headsep=20pt, top=2.5cm, bottom = 2cm,letterpaper,showframe = false]{geometry} 	
\usepackage{latexsym,amsmath,amssymb,amsfonts}	%(símbolos de la AMS).7
\parindent = 0cm 								%sangria
\usepackage{lmodern}							% tipos de letras
\usepackage[T1]{fontenc}						%acentos en español
\usepackage[spanish]{babel}
\usepackage{titlesec} %formato de títulos
\pagestyle{empty}								%elimina numeración de página
\usepackage{multicol}
\usepackage{xcolor}



\begin{document}
\begin{center}
\huge Microeconomía intermedia\\
\vspace*{0.5cm}
\large Hal R. Varian\\
\vspace{1cm}
\Large Apuntes por Fode.
\vspace{1.5cm}
\end{center}
\part*{\center El marcado}
\section*{Cómo se construye un modelo}
La economía se basa en la construcción de modelos de los fenómenos sociales. Entendemos por modelo una representación simplificada de la realidad.\\
En general, lo mejor es adoptar el modelo más sencillo capaz de describir la situación económica que estemos examinando. 
\subsection*{Optimización y equilibro}
Siempre que tratamos de explicar la conducta de los seres humanos, necesitamos tener un modelo en el que basar el análisis. En economía se utiliza casi siempre un modelo basado en los dos principios siguientes.
\begin{itemize}
\item \textbf{El principio de la optimización:} los individuos tratan de elegir las mejores pautas de consumo que están a su alcance.
\item \textbf{El principio del equilibrio:} los precios se ajustan hasta que la cantidad que demandan los individuos de una cosa es igual a la que se ofrece.
\end{itemize}
 Si los individuos pueden decidir libremente sus actos, es razonable suponer que tratan de elegir las cosas que desean y no las que no desean. 
\section*{La curva de demanda}
La cantidad máxima que una determinada persona está dispuesta a pagar suele denominarse precio de reserva. En otras palabras, el precio de reserva de una persona es aquel al que le da exactamente igual comprar una cosa que no comprarla.\\
representa una curva de demanda, que relaciona la cantidad demandada y el precio de mercado.\\
\textbf{ La curva de demanda describe la cantidad demandada a
cada uno de los posibles precios.} La curva de demanda de apartamentos tiene pendiente negativa: los individuos están más dispuestos a alquilar apartamentos a medida que baja su precio.
\section*{La curva de oferta}
La curva de oferta a corto plazo es fija.
\section*{El equilibro de mercado}
El precio de equilibrio, $p$, se encuentra en la intersección de las curvas de oferta y de demanda.
\section*{Estática comparativa}
consiste en comparar dos equilibrios “estáticos”, sin preocuparse especialmente por la forma en que el mercado pasa de uno a otro.\\
El análisis de estática comparativa consiste solamente en comparar equilibrios, lo que ya plantea por el momento suficientes interrogantes que deben resolverse en este modelo.
\section*{Otras formas de asignar los apartamentos}
\subsection*{El monopolista discriminador}
 El caso en el que el mercado de un producto está dominado por un único vendedor se denomina monopolio.\\
\subsection*{Monopolista ordinario}
Se podrá restringir la producción a fin de maximizar el beneficio.
\subsection*{El control de los alquileres}
Se fija un precio máximo que se pueden cobrar.
\section*{La eficiencia en el sentido de Pareto}
 Un criterio útil para comparar los resultados de diferentes instituciones económicas es un concepto conocido con el nombre de eficiencia en el sentido de Pareto o eficiencia económica.\\
 Comenzamos con la siguiente definición: si podemos encontrar una forma de mejorar el bienestar de alguna persona sin empeorar el de ninguna otra, tenemos una mejora en el sentido de Pareto. Si una asignación puede ser mejorable en el sentido de Pareto, esta asignación se denomina ineficiente en el sentido de Pareto; si no puede ser mejorable en el sentido de Pareto, esta asignación se denomina eficiente en el sentido de Pareto.\\
 \textbf{Una situación económica es eficiente en el sentido de Pareto si no existe ninguna forma de mejorar el bienestar de un grupo de personas sin empeorar el de algún otro. El concepto de eficiencia en el sentido de Pareto puede utilizarse para evaluar las diferentes formas de asignar los recursos.}
\part*{\center La restricción presupuestaria}
La teoría económica del consumidor es muy sencilla: los economistas suponen que los consumidores eligen la mejor cesta de bienes que pueden adquirir.\\
\section*{La restricción presupuestaria}
La restricción presupuestaria es escrita algebraicamente como sigue:
$$p_1 x_1 + p_2 x_2 \leq m$$
\section*{Dos bienes suelen ser suficientes}
El supuesto de los dos bienes es más general de lo que parece a primera vista, ya que normalmente podemos considerar que uno de ellos representa todo lo demás que al individuo le gustaría consumir.\\
decimos que el bien 2 es un \textbf{bien compuesto} porque representa todo lo demás que podría consumir el individuo, aparte del bien 1. 
\section*{Propiedades del conjunto presupuestario}
La recta presupuestaria es el conjunto de cestas que cuestan exactamente $m$: $$p_1 x_1 + p_2 x_2 = m$$
Éstas son las cestas de bienes que agotan exactamente la renta del consumidor.\\
La restricción presupuestaria de la ecuación [2.3] también puede expresarse de la forma siguiente:
$$x_2=\dfrac{m}{p_2}-\dfrac{p_1}{p_2} x_1$$
Indica cuántas unidades del bien 2 necesita consumir el individuo para satisfacer exactamente la restricción presupuestaria si está consumiendo $x_1$ unidades del bien 1.\\
Debe preguntarse qué cantidad del bien 1 podría comprar si gastara todo el dinero en dicho bien. La respuesta es $m/p1$. Por lo tanto, las coordenadas en el origen miden la cantidad que podría comprar el consumidor si gastara todo el dinero en los bienes 1 y 2, respectivamente.\\
La pendiente de la recta presupuestaria tiene una bonita interpretación económica. Mide la relación en la que el mercado está dispuesto a sustituir el bien 2 por el 1.\\
Algunas veces los economistas dicen que la pendiente de la recta presupuestaria mide el\textbf{ coste de oportunidad} de consumir el bien 1. Para consumir una mayor cantidad de dicho bien hay que renunciar a alguna cantidad del 2.La renuncia a la oportunidad de consumir el bien 2 es el verdadero coste económico de consumir una mayor cantidad del 1, y ese coste está representado por la pendiente de la recta presupuestaria.
\section*{Cómo varía la recta presupuestaria}
Un incremento de la renta da lugar a un desplazamiento paralelo hacia fuera de la recta presupuestaria. En cambio, una reducción de la renta provoca un desplazamiento paralelo hacia dentro.\\

\section*{El numerario}
En la definición de la recta presupuestaria se utilizan dos precios y una renta, pero una de estas variables es redundante. Podríamos mantener fijo uno de los precios o la renta y ajustar la otra variable para que describiera exactamente el mismo conjunto presupuestario.\\
Cuando supongamos que uno de los precios es $1$, como hemos hecho antes, a menudo decimos que éste es el precio del numerario: el precio en relación con el cual medimos el otro precio y la renta.\\
A veces resulta útil considerar que uno de los bienes es un bien numerario, ya que de esa forma hay un precio menos del que preocuparse.

\part*{\center Las preferencias}
El modelo económico de la conducta del consumidor es muy sensillo: Afirma que los individuos eligen las mejores cosas que están a su alcance. En él tratamos de aclarar el significado de $"$están a su alcance$"$ y en éste trataremos de aclarar el concepto económico de $"$mejores cosas$"$.\\

\section*{Las preferencias del consumidor}
Utilizaremos el símbolo $\succ$ para indicar que una cesta se prefiere estrictamente a otra, por lo que debe interpretarse que $(x_1,x_2)\succ (y_1,y_2)$ significa que el consumidor prefiere estrictamente $(x_1,x_2)$ a $(y_1, y_2)$\\
Si al consumidor le resulta indiferente elegir una u otra de las dos cesas de bienes, utilizamos el símbolo $\sim$ y escribimos $(x_1,x_2) \sim (y_1,y_2)$. Esto significa que, de acuerdo con sus propias preferencias, cualquiera de las dos cestas satisfaría igualmente al consumidor.\\
Si el individuo prefiere una de las dos cestas o es indiferentes entre ellas, decimos que prefiere débilmente la $(x_1,x_2)$ a la $(y_1,y_2)$ y escribimos $(x_1,x_2)\succeq (y_1,y_2)$\\

\section*{Supuestos sobre las preferencias}
Normalmente los economistas parten de una serie de supuestos sobre las relaciones de preferencia. Algunos son tan importantes que podemos llamarlos Axiomas de la teoría del consumidor. He aquí tres de ellos. Decimos que las preferencias son:
\begin{itemize}
    \item \textbf{Completas.} Suponemos que es posible dos cestas cualesquiera. Es decir, dada cualquier cesta $X$ y cualquier cesta $Y$, suponemos que $(x_1,x_2)\succeq (y_1,y_2)$ o $(y_1,y_2)\succeq (x_1,x_2)$ o las dos cosas, en cuyo caso, el consumidor es indiferente entre las dos cestas.
    \item \textbf{Reflexivas.} Suponesmo que cualquier cesta es al menos tan buena como ella misma: $(x_1,x_2) \succeq (x_1,x_2)$
    \item \textbf{Transitivas.} Si $(x_1,x_2) \succeq (y_1,y_2)$ y $(y_1,y_2) \succeq (z_1,z_2),$ suponemos que $(x_1,x_2) \succeq (z_1,z_2)$. En otras palabras, si el consumidor piensa que la cesta $X$ es al menos tan buena como la $Y$ y que la $Y$ es al menos tan buena como la $Z$, piensa que la $X$ es al menos tan buena como la $Z$. 
\end{itemize}
\section*{Las curvas de indiferencias}
Curva de indiferencias son las cestas indiferentes a $(x_1,x_2)$\\
Las curvas de indiferencia que representan distintos niveles de preferencias no pueden cortarse.

\subsection*{Sustitutivos perfectos}
Dos bienes son sustitutivos perfectos si el consumidor está dispuesto a sustituir uno por otro  a una tasa constante. La característica más importante de los sustitutivos perfectos reside en que las curvas de indiferencia tienen una pendiente constante.

\subsubsection*{Complementarios perfectos}
Los complementarios perfectos son bienes que siempre se consumen juntos en proporciones fijas. La característica más importante de los complementarios perfectos radica en que el consumidor prefiere consumir los bienes en proporciones fijas y no necesariamente en que la proporción sea de 1 a 1.

\subsection*{Males}
Un mal es una mercancía que no gusta al consumidor.

\subsection*{Neutrales}
Un bien es neutral si al consumidor le da igual.

\subsection*{Saciedad}
A veces interesa considerar una situación de saciedad, en la que hay una cesta global mejor para el consumidor y cuanto mas cerca se encuentre de esa cesta, mejor; mayor será su bienestar, en función de sus propias preferencias. 

\subsection*{Bienes discreto}
Algunas veces analizaremos examinaremos las preferencias que se encuentran en unidades discretas.

\section*{Las preferencias regulares}
Acá conoceremos las preferencias monótonas, donde para el consumidor es mejor la cesta que contiene una mayor cantidad de ambos bienes y peor la que contiene una cantidad menor. Su característica es que contiene pendiente negativa. En segundo lugar suponemos que que se prefiere las medias que los extremos.

\section*{La relación marginal de sustitución}
Muchas veces es útilo referirse a la pendiente de las curvas de indiferencias en un determinado punto, tanto es así que recibe incluso un nombre: se llama \textbf{Relación marginal de sustitución (RMS)}
$$\triangle x_2 / \triangle x_1$$
es la relación en que el consumidor está dispuesto a sustituir el bien $1$ por el $2$.\\
Imaginemos que $\triangle x_1$ es una variación muy pequeña, es decir, una variación marginal. En ese caso, el cociente $\triangle x_2 / \triangle x_1$ mide la relación marginal de sustitución del bien 1 por el 2. A medida que disminuye $\triangle x_1, \triangle x_2 / \triangle x_1$ se aproxima a la pendiente de la curva de indiferencia. En otras palabras La relación marginal de sustitución mide la pendiente de la curva de indiferencia.

\part*{\center La utilidad}
Una función de utilidad es un instrumento para asignar un número a todas las cestas de consumo posibles de tal forma que las que se prefieren tengan un número más alto que las que no se prefieren. Es decir, la cesta $(x_1,x_2)$ se prefuere a la $(y_1,y_2)$ si y sólo si la utilidad de la primera es mayor que la utilidad de la segunda; en símbolos, $(x_1,x_2) \succ (y_1,y_2)$ si y sólo si $u(x_1,x_2) > u(y_1,y_2)$.\\
La tasa de variación de $f(u)$ provocada por una variación de $u$ puede medirse observando la variación que experimenta $f$ entre dos valores de $u$, dividida por la variación de $u$: $$\dfrac{\triangle f}{\triangle u}=\dfrac{f(u_2)-f(u_1)}{u_2-u_1}$$ En el caso de una transformación monótona, $f(u_2)-f(u_1)$ siempre tiene el mismo signo que $u_2 - u_1$. Por lo tanto, una función monótona siempre tiene una tasa de variación positiva, lo que significa que el gráfico de una función de este timpo siempre tiene pendiente positiva.\\
Si $f(u)$ es una transformación monótona cualquiera de una función de utilidad que representa las preferencias $\succeq, f(u(x_1,x_2))$ también es una función de utilidad que representa esas mismas preferencias. Ya que:
\begin{enumerate}
    \item Decir que $u(x_1,x_2)$ representa las preferencias $\succeq$ significa que $u(x_1,x_2) > u(y_1,y_2)$ si y sólo si $(x_1,x_2)\succ (y_1,y_2)$
    \item Pero si $f(u)$ es una transformación monótona, $u(x_1,x_2)>u(y_1,y_2)$ si y sólo si $f(u(x_1,x_2))>f(u(y_1,y_2))$ 
    \item Por lo tanto, $f(u(x_1,x_2))>f(u(y_1,y_2))$ si y sólo si $(x_1,x_2)\succ (y_1,y_2)$ por lo tanto la función $f(u)$ representa las preferencias $\succeq$ de la misma forma que la función de utilidad original $u(x_1,x_2)$
\end{enumerate}
Resumimos este análisis formulando el siguiente principio: Una transformación monótona de un función de utilidad es una función de utilidad que representa las mismas preferencias que la función de utilidad marginal.
Desde el punto de vista geométrico, una función de utilidad es una forma de denominar las curvas de indiferencia. Dado que todas las cestas de una curva de indiferencia deben tener la misma utilidad, una función de utilidad es un instrumento para asignar números a las distintas curvas de indiferencia de tal manera que las más altas reciban números más altos. Desde este punto de vista una transformación monótona equivale exactamente a denominar de nuevo las curvas de indiferencia. Mientras las curvas de indiferencia que contengan las cestas que se prefieren reciban un número más alto que las que contienen las cestas que no se prefieren, las denominaciones representarán las mismas preferencias.

\section*{Algunos ejemplos de funciones de utilidad}
Supongamos que la función de utilidad es $u(x_1,x_2)=x_1 x_2$ ¿Como son las curvas de indiferencia?\\
Sabemos que una curva de indiferencia tipo es el conjunto de todas las $x_1$ y las $x_2$ tal que $k=x_1 x_2$. Y por lo tanto una curva de indiferencia tipo tiene la fórmula: $$x_2=\dfrac{k}{x_1}$$.\\
Supongamos por otro lado que nos dan la función de utilidad $u(x_1,x_2)=x_1^2 x_2^2$ y en consecuencia $$u(x_1,x_2)=x_1^2 x_2^2 = (x_1,x_2)^2 =u(x_1,x_2)^2$$

\subsection*{Sustitutos perfectos}
Esta dada por la función $$u(x_1,x_2)=ax_1 + bx_2$$

\subsection*{Complementarios perfectos}
Esta dada por la función $$u(x_1,x_2)=min\lbrace ax_1, bx_2\rbrace$$

\subsection*{Preferencias cuasilineales}
Esta dada por la función $$u(x_1,x_2)=k=v(x_1)+x_2$$

\subsection*{Preferencias Cobb-Douglas}


\end{document}
