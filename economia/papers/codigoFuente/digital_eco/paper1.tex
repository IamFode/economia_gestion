\chapter*{Cambio estructural en el mercado laboral influenciado por la inteligencia artificial: análisis teórico y empírico}

\begin{center}
    \textbf{Daryna Rozum, Nadiya Grazhe y Volodymyr Birchenko.}
\end{center}

El objetivo del artículo es estudiar los efectos de las tecnologías de inteligencia artificial en el mercado laboral en el entorno económico global. 

\section*{\center I. INTRODUCCIÓN}
La Cuarta Revolución Industrial, iniciada en el siglo $XXI$, se diferencia de las tres anteriores en que fusiona la tecnología de fabricación, la fabricación automatizada y el intercambio de datos en un sistema autorregulado, acompañado de la reducción de la intervención humana en el proceso de producción, provoca cambios estructurales significativos en el mercado laboral que afectan el desarrollo de la economía global en su conjunto. Según los organizadores del Foro Económico Mundial, el $42 \%$ de las habilidades básicas que ahora tienen los trabajadores se están transformando significativamente [1].\\

Reemplazar puestos de trabajo con las tecnologías de la Cuarta Revolución Industrial requerirá una mayor capacitación de más de mil millones de personas para 2030. Si no se mantiene, la crisis de habilidades ampliará la brecha cada vez mayor entre ricos y pobres, lo que hará imposible lograr la paz, el medio ambiente. sostenibilidad y prosperidad [1]. En el contexto del problema que se analiza, se presta especial atención a la inteligencia artificial, que está a la vanguardia de las últimas tecnologías de vanguardia.

\section*{\center II. RESULTADOS DE LA INVESTIGACIÓN}

\subsection*{A. Antecedentes teóricos de la investigación.}
Las hipótesis sobre la posibilidad de la existencia de inteligencia artificial fueron presentadas por expertos en varios campos de la ciencia a mediados del siglo XX. Un paso importante en la comprensión teórica de este tema lo dio el famoso científico británico A. Turing en su artículo “Computing Machinery and Intelligence”, publicado en 1950, donde el autor cuestionaba la capacidad de pensar de la máquina. Al interpretar el pensamiento como la capacidad de realizar acciones significativas, el científico propuso la llamada prueba de Turing, cuya esencia radica en la incapacidad del experimentador para distinguir a un humano de una máquina, mientras se comunica con los sujetos por medio de un teletipo. Según A. Turing, sólo cuando el experimentador percibe erróneamente la máquina por un humano, podemos decir que piensa [2].\\

La introducción de la inteligencia artificial, por un lado, ayuda a reducir los costos de producción y aumentar la productividad y, por otro lado, conduce a cambios sin precedentes en el mercado laboral, incluidos los recortes de empleo a nivel mundial y el aumento de las tasas de desempleo.\\

Cabe señalar que existen puntos de vista contradictorios sobre este problema en la literatura científica. En particular, muchos estudios predicen posibles cambios positivos en el mercado laboral en la era de la inteligencia artificial, asumiendo que las nuevas tecnologías pueden crear más empleos que eliminar [4].\\

Asimismo, cabe hacer una mención especial a la relación directa del fenómeno de la inteligencia artificial con la Cuarta Revolución Industrial (Industria 4.0). Se sabe que este término fue introducido por K. Schwab en 2017 [5]. 
Al analizar el impacto de la Cuarta Revolución Industrial en el empleo, el científico llama la atención sobre la existencia de dos enfoques contradictorios sobre este tema, a saber: (1) optimista, cuyos partidarios creen que los trabajadores reemplazados por las tecnologías de Industria 4.0 eventualmente encontrarán un nuevo trabajo y así comienza una nueva era de prosperidad; (2) pesimista, que difunde la idea de que las nuevas tecnologías provocan un Armagedón social y político, creando desempleo tecnológico global. Según el investigador, la experiencia histórica muestra que el resultado real de estos cambios está en algún punto intermedio [5]. Por lo tanto, es necesario definir qué se debe hacer para garantizar un resultado positivo y ayudar a quienes sufren los cambios.\\

Por ejemplo, según los expertos del Foro Económico Mundial, se crearán 133 millones de nuevos puestos de trabajo en los principales países para 2022 para satisfacer las necesidades de la Cuarta Revolución Industrial. Se garantizará un importante crecimiento del empleo en siete áreas profesionales, a saber, Care; Ingeniería y Computación en la Nube; Marketing; Datos e Inteligencia Artificial; Empleos Verdes; Pueblo y Cultura; y Gestión de Proyectos Especializados [1].\\

Al mismo tiempo, C. Frey, M. Osbourne [6], J. Furman, R. Seamans [7] etc. prestan atención a los efectos negativos de la automatización en el mercado laboral y clasifican los trabajos en peligro, con mano de obra intensiva y ocupaciones que consumen mucho tiempo en la parte superior de la lista. Estudios de D. Acemoglu y P. Restrepo [8] V. Dignum [9], L. Spector [3], Smith et al. [2] se dedican al análisis de tendencias amenazantes de despidos bajo la influencia de la automatización de la producción; además, se da la justificación de las medidas utilizadas para minimizar los riesgos para el mercado laboral.\\

Según N. Dyer-Witheford, autor de “Cyber Proletariat: Global Labor in the Digital Vortex” (2015), las nuevas tecnologías digitales han cambiado paradójicamente la estructura orgánica del capital. Al incluir cada vez más recursos humanos en ella, la fuerza de trabajo se vuelve innecesaria como clase de la sociedad capitalista. Además, el autor presenta el tema de la segmentación de la clase obrera según la jerarquía de zonas con diferentes salarios, que se extiende desde Bangladesh hasta Baltimore. Al mismo tiempo, la nueva clase obrera, organizada en red, combinada (aunque con diferente intensidad) con capital y bienes a través de 2 mil millones de conexiones a Internet y 7 mil millones de teléfonos móviles, emerge como ciberproletariado. El científico argumenta que el trabajo informal, el nuevo trabajo industrial y los servicios cibernéticos son los segmentos de la proletarización clásica incrustados en la matriz cibernética. En lugar del trabajo basado en el conocimiento, prometido por el surgimiento de la cibernética, que ofrece control sobre las condiciones de trabajo, los trabajadores tienen que proporcionar mano de obra independientemente de qué y cómo produzcan bajo una competencia laboral despiadada. En este contexto, merece atención la conclusión de N. Dyer-Witheford sobre la disminución gradual del empleo en el futuro cercano, ya que los empleadores contratarán trabajadores que posean talentos únicos y específicos para realizar una determinada cantidad de tareas. Por lo tanto, se observará un nuevo fenómeno de precariedad que incluye rotación masiva de personal, pérdida de confianza en el empleo estable y salarios [10].\\

Los estudios empíricos en este campo confirman que las tecnologías de la Industria 4.0 generan amenazas y riesgos sin precedentes para el empleo. Según las estimaciones de C. Frey y M. Osbourne [6], el $47\%$ de los puestos de trabajo en la economía de los EE. UU. están en riesgo debido al uso de robots que pueden reemplazar rápidamente al personal regular en las próximas décadas. El otro $33\%$ de los puestos (p. ej., médicos, abogados, ingenieros, profesores, etc.) entran en la categoría de bajo riesgo, que es una clasificación un tanto relativa, ya que estos puestos también pueden sufrir la presión de la automatización [6].\\

Un claro ejemplo de la amenaza de la automatización laboral masiva es la sustitución de 34 empleados de la compañía de seguros japonesa Fukoku Mutual por el sistema de inteligencia artificial desarrollado por IBM. Este sistema es capaz de analizar e interpretar los datos médicos de los pacientes (texto no estructurado, audio, video e imágenes gráficas) y calcular los pagos de seguros correspondientes. Como resultado, la gerencia de la empresa planea aumentar la productividad en un $30\%$ al tiempo que reduce los costos de producción debido al ahorro en los salarios de los empleados de aproximadamente $\$ 1.2$ millones al año [11].\\

Sin embargo, muchos estudios presentan la posibilidad de cambios positivos en el mercado laboral en la era de la inteligencia artificial, basados en el supuesto de que las nuevas tecnologías pueden crear más puestos de trabajo que eliminar [12]. Según las previsiones del WEF, alrededor de 5 millones de personas que trabajan en las 15 principales economías, donde el mercado laboral agregado representa el $65\%$ de la fuerza laboral total del mundo, pueden perder sus empleos en los próximos 5 años. Sin embargo, los mayores cambios afectarán a los trabajos de oficina y administrativos, así como a algunas profesiones del ámbito social. Al mismo tiempo, aumentará la demanda de analistas de big data, arquitectos y representantes comerciales [12].\\

De esta forma, cabe señalar que en 2018, los países europeos tecnológicamente avanzados (principalmente Alemania y Francia) invirtieron 700 millones de euros en reciclaje profesional. El objetivo principal de esta medida era garantizar un empleo estable y pasarlo a la digitalización [13]. La protección social seguramente se convertirá en la prioridad de la política de empleo de los países de la UE tecnológicamente avanzados, incorporada en la gestión regulatoria de la interacción hombre-máquina y el seguro social (por ejemplo, el proyecto "RoboLaw"). Para asegurar esta declaración, podemos dar un ejemplo de un conjunto detallado de reglas desarrolladas por el Parlamento Europeo, que controlan la interacción y el trabajo hombre-máquina. En particular, se ofrece un seguro obligatorio para cubrir los daños que puedan causar los robots [13]. Las iniciativas de responsabilidad social corporativa destinadas a mejorar las habilidades y el reciclaje del personal también son soluciones importantes para los problemas mencionados anteriormente.\\

Por tanto, la introducción de la inteligencia artificial refuerza el papel y la importancia de la política de empleo, que debe transformarse a la luz de las oportunidades y desafíos de la Cuarta Revolución Industrial. En este contexto, se destacan las propuestas de K. Schwab sobre la necesidad de reformas educativas y programas de reciclaje para desempleados, capaces de mitigar los efectos destructivos de los sistemas de inteligencia artificial [5].\\

\subsection*{B. Metodología de investigación}
La investigación empírica en esta área confirma que las tecnologías de Industria 4.0 con inteligencia artificial a la vanguardia generan amenazas laborales sin precedentes. Accenture estima que la incapacidad de satisfacer la necesidad de mano de obra de la nueva era tecnológica en los países del G20 puede poner en riesgo $\$ 11.5$ billones de PIB en la próxima década [14]. En nuestra opinión, el cambio estructural en el mercado laboral de Alemania, que se relaciona con el aumento gradual del empleo en los servicios combinado con la reducción del empleo en la industria y la agricultura, como resultado del progreso científico, los cambios tecnológicos y la difusión de las tecnologías de inteligencia artificial en todos los campos de las relaciones socioeconómicas. (Figura 1.).\\

Para evaluar el impacto del progreso científico y los cambios tecnológicos en el empleo en el sector industrial, los autores aplican análisis cuantitativos de estadísticas internacionales y modelos econométricos. Los métodos de investigación mencionados permiten formalizar la interrelación entre el cambio estructural en el empleo y el sistema de indicadores que describen indirectamente el estado y las tendencias en el desarrollo de la inteligencia artificial y los cambios tecnológicos, y proporcionan una base empírica para las conclusiones teóricas sobre el papel de la Industria 4.0 en la estimulación. de cambios profundos en la estructura del mercado laboral moderno.\\

El análisis se basa en los datos proporcionados por el Banco Mundial. Utilizando el análisis de regresión, los autores han desarrollado un modelo econométrico, que mide la influencia del desarrollo de la inteligencia artificial, los cambios tecnológicos y el progreso científico en el empleo en el sector industrial que se utilizó como variable dependiente o de salida. Se eligieron los siguientes regresores o variables independientes: exportaciones de alta tecnología, ingresos por transferencia internacional de derechos de propiedad intelectual, gasto público en educación, número de investigadores en I+D, gasto en investigación y desarrollo, empleo en servicios. Teniendo en cuenta que las diferentes variables en el conjunto de datos tienen valores en diferentes rangos, los autores aplican la transformación logarítmica para proporcionar la no dimensionalización y la normalización de las variables independientes.\\

Para estimar el parámetro del modelo, los autores aplican el método de mínimos cuadrados con errores estándar consistentes de heterocedasticidad de Newey-West y estimador de covarianza. El análisis de regresión se ha realizado en el software IHS EViews. La muestra de variables antes mencionadas de una muestra transversal de 100 países se utiliza como entrada del modelo. La muestra incluye datos proporcionados por el Banco Mundial a partir de enero de 2019. La muestra seleccionada utilizada en la ecuación de regresión incluye dos grupos de países: economías de ingresos altos (estados de la UE, EE. UU., Japón, etc.), economías de ingresos medios altos (Argentina, México , Turquía, etc.) y economías seleccionadas de ingresos medianos bajos (India, Rusia, Ucrania) que ya han completado o están experimentando transformaciones tecnológicas posindustriales.\\

\subsection*{C. Resultados del análisis de regresión}
El análisis de regresión realizado permite identificar y medir el impacto del desarrollo de tecnologías de la Industria 4.0 y la difusión de la inteligencia artificial en la estructura del mercado laboral moderno. El modelo econométrico se ha desarrollado en forma de regresión lineal múltiple de acuerdo con la metodología considerada anteriormente utilizando datos de muestra de una muestra representativa de 100 países. La regresión lineal múltiple demuestra la relación entre el empleo en la industria (participación del empleo total) como variable de respuesta y seis variables explicativas consideradas anteriormente.\\

Los autores sugieren que los países que adoptan activamente tecnologías de inteligencia artificial están a la vanguardia del progreso científico y tecnológico. Por lo tanto, a falta de datos representativos cuantitativos de varios países sobre el desarrollo de la inteligencia artificial, los autores utilizan las variables explicativas antes mencionadas, como las exportaciones de alta tecnología o el gasto en investigación y desarrollo, que representan indirectamente el desarrollo de las tecnologías de inteligencia artificial y su impacto en el economía.\\

Después de la adimensionalización y la normalización de las variables independientes, la estimación de los parámetros de los modelos y las pruebas de las características de calidad del modelo, ha tomado la siguiente forma definida matemáticamente:

$$EI = $$

38,15 – constante, que representa el valor medio esperado de la variable de respuesta cuando todas las variables explicativas son iguales a cero;

\begin{itemize}
    \item EI.- Empleo en la industria como $\%$ del empleo total; 
    \item HTE.-  Exportaciones de alta tecnología en dólares estadounidenses; 
    \item IPR.- Recibo de transferencia internacional de derechos de propiedad intelectual en dólares estadounidenses; 
    \item GEE.- Gasto público en educación expresado como porcentaje del PIB; 
    \item RDP.- Número de investigadores en I+D por millón de habitantes; 
    \item RDE.- Gasto en investigación y desarrollo expresado como porcentaje del PIB; 
    \item ES.- Empleo en servicios como proporción del empleo total. 
\end{itemize}
	El análisis de regresión realizado ha demostrado la adecuación y las características de calidad del modelo (tabla 1).\\

La aplicación del estimador de Newey-West ha permitido corregir la heterocedasticidad de los residuos y evitar sesgos en los errores estándar. Los errores estándar del modelo son bastante bajos, las estadísticas F y todas las estadísticas t son significativas, las estadísticas de Durbin-Watson muestran un nivel satisfactorio, los criterios de Akaike, Schwarz y Hannan-Quinn se minimizaron. El análisis de la matriz de correlación no muestra multicolinealidad entre las variables explicativas independientes.\\

\subsection*{\center III. CONCLUSIONES}
La interpretación de los resultados del análisis de regresión demuestra un efecto múltiple negativo considerable del progreso científico, los cambios tecnológicos y el desarrollo de la inteligencia artificial en el empleo en el sector industrial. En particular, el aumento de los ingresos por la transferencia internacional de derechos de propiedad intelectual provoca la disminución del empleo en la industria; el aumento del gasto público en educación y el número de investigadores dedicados a la investigación y el desarrollo también conduce a la disminución del empleo en el sector industrial. Además, la reducción del empleo en la industria es seguida por el aumento del empleo en los servicios como resultado del cambio tecnológico.\\

Cabe señalar que, además del impacto indudablemente negativo en el empleo en la industria, el progreso científico, los cambios tecnológicos y el desarrollo activo de la inteligencia artificial tienen una fuerte influencia positiva en la formación efectiva de la estructura de producción, estimulando el desarrollo innovador y asegurando la competitividad global de las economías nacionales. Al estar a la vanguardia del progreso científico y técnico, los países en desarrollo aceleran la transferencia transfronteriza de tecnologías y fortalecen su propio potencial de innovación en la economía posindustrial moderna utilizando el mercado tecnológico global. La adopción de tecnologías de inteligencia artificial en los negocios proporciona reducción de costos y genera nuevas fuentes de ingresos para las empresas.\\

Las principales consecuencias socioeconómicas de la difusión activa de tecnologías de inteligencia artificial en todos los campos de la actividad económica son las siguientes: profundizar la estratificación de la sociedad en productores de innovación y empleados en actividades no relacionadas con TI; aumento del desempleo debido a la automatización del trabajo; la rotación del personal y la amenaza de una brecha educativa cuando el personal requerido por la Industria 4.0 aún no ha recibido la capacitación necesaria; el fenómeno de la “precariedad”, que encarna la inestabilidad laboral y la falta de seguridad laboral al tiempo que estimula la ansiedad y la tensión social; transformación de empresas y gremios de empleados en el principal proveedor de protección social; reducir el papel del Estado en este ámbito; aumento de la demanda de trabajadores de tecnología de TI; convertir talentos únicos y altamente específicos en el capital central de empresas altamente competitivas, etc. [16].\\

En este contexto, existe una necesidad urgente de racionalización institucional y organizativa del desarrollo de tecnologías de inteligencia artificial y su implementación en la producción, y la regulación de las consecuencias correspondientes para la economía y la sociedad. Debe prestarse especial atención a la regulación de los derechos de propiedad intelectual de los objetos creados con inteligencia artificial, ya que en la era de la Industria 4.0, los trabajadores generadores de innovación son el motor de la economía y la transformación del mercado laboral. El uso efectivo de su potencial creativo no es posible sin la racionalización institucional y organizativa de las relaciones de propiedad intelectual, en particular, sin una clara especificación y protección de estos derechos.
