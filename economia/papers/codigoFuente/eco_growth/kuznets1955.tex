Economic Growth and Income Inequality\\
Author(s): Simon Kuznets\\
Source: The American Economic Review , Mar., 1955, Vol. 45, No. 1 (Mar., 1955), pp. 1-28\\
Published by: American Economic Association\\
Stable URL: \url{https://www.jstor.org/stable/1811581}\\\\
REFERENCES\\
Linked references are available on JSTOR for this article:\\
\url{https://www.jstor.org/stable/1811581?seq=1&cid=pdfreference#references_tab_contents}\\
You may need to log in to JSTOR to access the linked references.
\let\cleardoublepage\clearpage

\chapter*{Crecimiento Económico y desigualdad del ingreso} 
El tema central de este artículo es el carácter y las causas de los cambios a largo plazo en la distribución personal del ingreso. ¿La desigualdad en la distribución del ingreso aumenta o disminuye en el curso del crecimiento económico de un país?, ¿Qué factores determinan el nivel secular y las tendencias de las desigualdades de ingresos?. \\
Puede ser útil especificar las características de las distribuciones del tamaño de la renta \footnote{La distribución de la renta es la manera en que se recogen los recursos materiales fruto de la actividad económica en los distintos estratos socio-económicos} que queremos examinar y cuyos movimientos queremos explicar. Para ello podríamos ver cinco características. 
\begin{enumerate}[1.]
    \item Las unidades que se registran y se agrupan deben ser unidades de gasto familiar, ajustadas por el número de personas que las componen.
    \item Las distribuciones podrían ser completas, es decir, debe abarcar todas las unidades de un país y no un segmento en la  cola superior o inferior. 
    \item Si es posible, hay que segregar las unidades cuyos principales ingresos están todavía en fase de aprendizaje o ya en la fase de retiro de su ciclo de vida, para no complicar el panorama incluyendo los ingresos que no están asociados a la participación plena y a tiempo completo en la actividad económica. 
    \item El ingreso debe definirse como lo es ahora para el ingreso nacional en este país, es decir, recibido por individuos, incluidos los ingresos en especie, antes y después de los impuestos directos, excluyendo las ganancias de capital. 
    \item Las unidades deben agruparse por niveles seculares \footnote{Cuando las ganancias de la empresa asociada permanecen constantes independientemente de otras tendencias que ocurran dentro del mercado. La tendencia secular es un comportamiento persistente en el horizonte de largo plazo. Es decir, la variable analizada recibe presiones a la baja o al alza constantemente durante varios años, o incluso décadas.} de ingreso, libres de perturbaciones cíclicas y otras transitorias.\\
\end{enumerate}
Que las clases de renta $"$baja$"$ ganaron o perdieron durante los últimos veinte años en el sentido de que su participación en la renta total aumentó o disminuyó, sólo tiene sentido si las unidades han sido clasificadas como miembros de las clases $"$bajas$"$ a lo largo de esos $20$ años, y para los que han entrado o salido de esas clases recientemente esa afirmación no tiene ningún significado. Además deberíamos ser capaces de rastrear los niveles de ingresos seculares no sólo a través de una sola generación, sino al menos a través de dos, conectando los ingresos de una generación determinada con los de sus descendientes inmediatos. Podríamos entonces distinguir las unidades que, a lo largo de una determinada generación, permanecen dentro de un grupo ordinal y cuyos hijos -a lo largo de su generación- también están dentro de ese grupo, de las unidades que permanecen dentro de un grupo a lo largo de su generación pero cuyos hijos ascienden o descienden en la escala económica relativa en su momento. El número de combinaciones y permutaciones posibles es grande, pero no debería ocultar el diseño principal de la estructura de ingresos que se requiere: la clasificación por estatus de ingresos a largo plazo de una generación determinada y de sus descendientes inmediatos.\\

\textbf{El corolario importante es que el estudio de los cambios a largo plazo en la distribución de la renta debe distinguir entre los cambios en las participaciones de los grupos residentes dentro de una o dos generaciones y los cambios en las participaciones de la renta de los grupos que, juzgados por sus niveles seculares, migran hacia arriba o hacia abajo en la escala de la renta.}

\section*{I. \centering Tendencias en la desigualdad del ingreso}
Los datos corresponden a Estados Unidos, Inglaterra y Alemania: una muestra escasa, pero al menos un punto de partida para algunas inferencias sobre los cambios a largo plazo en los países actualmente desarrollados. La conclusión general sugerida es que la distribución relativa del ingreso, medida por la incidencia del ingreso anual en clases bastante amplias, se ha estado moviendo hacia la igualdad, con estas tendencias particularmente notables desde la década de 1920, pero comenzando tal vez en el período anterior a la primera guerra mundial. Como dato, en Alemania en su conjunto, la desigualdad relativa de ingresos cae de manera bastante pronunciada desde $1913$ hasta la década de $1920$, aparentemente debido a la destrucción de grandes fortunas e ingresos de propiedad durante la guerra y la inflación; pero luego comienza a volver a los niveles anteriores a la guerra durante la depresión del $1930$. Esto justifica una impresión provisional de constancia en la distribución relativa del ingreso antes de impuestos, seguida de cierta reducción de la desigualdad relativa del ingreso después de la Primera Guerra Mundial o antes. Hay que subrayar tres aspectos de este hallazgo.
\begin{enumerate}[1.]
    \item Se deduce que la distribución del ingreso después de los impuestos directos e incluidas las contribuciones gratuitas del gobierno mostraría una reducción aún mayor de la desigualdad en los países desarrollados con distribuciones de tamaño de los ingresos antes de impuestos y ex-beneficios del gobierno similares a las de los Estados Unidos y el Reino Unido.
    \item Si las participaciones de los grupos clasificados por su posición de ingreso anual pueden verse como aproximaciones a las participaciones de los grupos clasificados por sus niveles de ingreso seculares, una participación porcentual constante de un grupo dado significa que su ingreso real per cápita está aumentando al mismo ritmo a la misma tasa que el promedio de todas las unidades del país; y una reducción en la desigualdad de las proporciones significa que el ingreso per cápita de los grupos de ingresos más bajos está aumentando a un ritmo más rápido que el ingreso per cápita de los grupos de ingresos más altos.
    \item ¿Las distribuciones por ingresos anuales reflejan adecuadamente las tendencias en la distribución por ingresos seculares?. A medida que la tecnología y el desempeño económico alcanzan niveles más altos, los ingresos están menos sujetos a perturbaciones transitorias, no necesariamente del orden cíclico que puede reconocerse y permitirse por referencia a la cronología del ciclo económico, sino de un tipo más irregular. Si en los primeros años la fortuna económica de las unidades estuvo sujeta a mayores vicisitudes, malas cosechas para algunos agricultores, pérdidas por calamidades naturales para algunas unidades comerciales no agrícolas. Si la proporción general de empresarios individuales cuyos ingresos estuvieron sujetos a tales calamidades, estas distribuciones anteriores del ingreso se verían más afectadas por perturbaciones transitorias. En estas distribuciones anteriores, los temporalmente desafortunados podrían abarrotar los quintiles inferiores y reducir sus acciones indebidamente, y los temporalmente afortunados podrían dominar el quintil superior y aumentar su participación indebidamente, más proporcionalmente que en las distribuciones de años posteriores. De ser así, las distribuciones por ingreso promedio a más largo plazo podrían mostrar una menor reducción de la desigualdad que las distribuciones por ingreso anual; incluso podrían mostrar una tendencia opuesta.
\end{enumerate}
El punto importante es que la calificación sea relevante; sugiere la necesidad de un mayor estudio si queremos aprender mucho de los datos disponibles sobre la estructura secular del ingreso; y es probable que tal estudio arroje resultados de interés en sí mismos por su relación con el problema de las tendencias en la inestabilidad temporal de los flujos de ingresos hacia unidades individuales o hacia grupos de unidades económicamente significativos en diferentes sectores de la economía nacional.

\section*{II. \centering Un intento de explicación} 
La presente entrega de especulaciones iniciales puede introducirse diciendo que la constancia a largo plazo, y mucho menos la reducción, de la desigualdad en la estructura secular de ingresos es un rompecabezas. 
Porque hay al menos dos grupos de fuerzas en la operación a largo plazo de los países desarrollados que contribuyen a aumentar la desigualdad en la distribución del ingreso antes de impuestos y excluyendo las contribuciones de los gobiernos. 
\begin{enumerate}[1.]
    \item El primer grupo se relaciona con la concentración del ahorro en los tramos de ingresos más altos. Según todos los estudios recientes sobre la distribución del ingreso entre consumo y ahorro, solo ahorran los grupos de ingresos más altos; los ahorros totales de los grupos por debajo del decil superior son bastante cercanos a cero. Por ejemplo, el 5 por ciento superior de las unidades en los Estados Unidos parece representar casi dos tercios de los ahorros de las personas. Lo que es particularmente importante es que la desigualdad en la distribución de los ahorros es mayor que la de la distribución de los ingresos de la propiedad y, por lo tanto, de los activos. 
    \item La segunda fuente del rompecabezas radica en la estructura industrial de la distribución del ingreso. Un acompañamiento invariable del crecimiento en los países desarrollados es el alejamiento de la agricultura, un proceso que suele denominarse industrialización y urbanización. Por lo tanto, la distribución del ingreso de la población total, en el modelo más simple, puede verse como una combinación de las distribuciones del ingreso de las poblaciones rural y urbana. Lo poco que sabemos de las estructuras de estos dos componentes de la distribución del ingreso revela que: (a) el ingreso promedio per cápita de la población rural suele ser más bajo que el de la urbana; (b) la desigualdad en las participaciones porcentuales dentro de la distribución de la población rural es algo más estrecha que en la de la población urbana. Operando con este modelo simple, ¿a qué conclusiones llegamos? En primer lugar, siendo iguales todas las demás condiciones, el peso creciente de la población urbana significa una participación creciente de la más desigual de las distribuciones de los dos componentes. En segundo lugar, la diferencia relativa en el ingreso per cápita entre las poblaciones rural y urbana no necesariamente desciende en el proceso de crecimiento económico: de hecho, hay algunas pruebas que sugieren que, en el mejor de los casos, es estable y tiende a aumentar debido a que la productividad per cápita en actividades urbanas aumenta más rápidamente que en la agricultura. Si esto es así, la desigualdad en la distribución del ingreso total debería aumentar.\\
\end{enumerate}
    Entonces surgen dos preguntas: 
    \begin{itemize}
	\item ¿Por qué la participación de los grupos de ingresos más altos no muestra un aumento en el tiempo si la concentración del ahorro tiene un efecto acumulativo?.
	\item ¿Por qué disminuye la desigualdad de ingresos y, en particular, por qué aumenta la participación de los grupos de menores ingresos si aumentan tanto el peso de la distribución del ingreso urbano más desigual como la diferencia relativa entre los ingresos per cápita urbano y rural?.
    \end{itemize}

\subsection*{Factores que contrarrestan la concentración del ahorro}
Un grupo de factores que contrarrestan el efecto acumulativo de la concentración de los ahorros en las participaciones de los ingresos altos es la intensidad legislativa y las decisiones $"$políticas$"$. Estos pueden estar destinados a limitar la acumulación de propiedad directamente a través de impuestos a la herencia y otros gravámenes de capital explícitos. 
Pueden producir efectos similares indirectamente, por ejemplo, por inflación permitida o inducida por el gobierno que reduce el valor económico de la riqueza acumulada almacenada en valores de precio fijo u otras propiedades que no responden completamente a los cambios de precios; o por la restricción legal del rendimiento de la propiedad acumulada, como sucedió recientemente en forma de controles de rentas o de tasas de interés a largo plazo artificialmente bajas mantenidas por el gobierno para proteger el mercado de sus propios bonos. 
El resultado de tales cambios sería una presión cada vez mayor de las decisiones legales y políticas sobre la participación de los ingresos más altos, aumentando a medida que un país avanza hacia niveles económicos más altos.\\

Pasamos a otros tres grupos menos obvios de factores que compensan los efectos acumulativos de la concentración del ahorro. El primero es demográfico. En los países actualmente desarrollados ha habido tasas diferenciales de aumento entre los ricos y los pobres, habiéndose extendido primero el control familiar a los primeros. Por lo tanto, incluso sin tener en cuenta la migración, se puede argumentar que el $5$ por ciento superior de $1870$ y sus descendientes representarían un porcentaje significativamente menor de la población en $1920$. 
El $5$ por ciento superior de la población en $1920$ está, por lo tanto, compuesto sólo en parte por los descendientes del $5$ por ciento superior de $1870$; quizás la mitad o una fracción mayor debe haberse originado en los tramos de ingresos más bajos de $1870$. Esto significa que el período durante el cual se puede suponer que los efectos de la concentración de los ahorros se acumularon para aumentar la participación en el ingreso de cualquier grupo ordinal fijo dado (ya sea sea el $1$, $5$ o $10$ por ciento superior de la población) es mucho más corto que los cincuenta años en el lapso; y, por lo tanto, estos efectos son mucho más débiles de lo que habrían sido si el $5$ por ciento más rico de $1870$ lo hubiera hecho, a través de las filas del $5$ por ciento más rico de la población de $1920$. Aunque el efecto acumulativo de los ahorros puede ser elevar el relativo ingreso de una proporción superior de la población total que disminuye progresivamente, su efecto sobre la proporción relativa de una proporción superior fija de la población es mucho menor.\\

El segundo grupo de fuerzas reside en la naturaleza misma de una economía dinámica con relativa libertad de oportunidades individuales. En una sociedad de este tipo, el cambio tecnológico es desenfrenado y los activos de propiedad que se originaron en industrias más antiguas casi inevitablemente tienen un peso proporcional decreciente en el total debido al crecimiento más rápido de las industrias más jóvenes. 
A menos que los descendientes de un grupo de altos ingresos logren trasladar sus activos acumulados a nuevos campos y participar con nuevos empresarios en la creciente participación de las industrias nuevas y más rentables, es probable que los rendimientos a largo plazo de sus propiedades sean significativamente más bajos que los de los participantes más recientes en la clase de tenedores de activos sustanciales. Hay, entre los grupos de ingresos altos de hoy, muchos descendientes de los grupos de ingresos altos de más de tres o más años. Pero el adagio es realista en el sentido de que una secuencia larga e ininterrumpida de conexión con industrias emergentes y, por lo tanto, con fuentes importantes de ingresos continuos de grandes propiedades es extremadamente rara; que los grandes y exitosos empresarios de hoy rara vez son hijos de los grandes y exitosos empresarios de ayer.\\

El tercer grupo de factores lo sugiere la importancia, incluso en los tramos de ingresos más altos, de los ingresos por servicios. En un momento dado, solo una parte limitada del diferencial de ingresos de un grupo superior se explica por la concentración de los rendimientos de la propiedad: gran parte proviene del alto nivel de ingresos por servicios (ingresos profesionales y empresariales y similares). 
Es probable que el aumento secular de los ingresos superiores debido a esta fuente sea menos marcado que el de los ingresos por servicios de los tramos inferiores, y por dos razones algo diferentes. En primer lugar, en la medida en que los altos niveles de ingresos por servicios de determinadas unidades superiores se deban a la excelencia individual (como es el caso de muchas actividades profesionales y empresariales), hay mucho menos incentivo y posibilidad de mantener dichos ingresos en niveles relativamente altos continuados. Por lo tanto, no es probable que los ingresos por servicios de los descendientes de una unidad de nivel inicialmente alto muestren una tendencia ascendente tan fuerte como los ingresos de la gran masa de población en los niveles de ingresos más bajos. En segundo lugar, una parte sustancial de la tendencia creciente del ingreso per cápita se debe al cambio entre industrias, es decir, un cambio de trabajadores de industrias de ingresos más bajos a industrias de ingresos más altos. \\

Estos tres grupos de factores, incluso sin tener en cuenta la intervención legislativa y política como se indicó anteriormente, son todos características de una economía dinámica en crecimiento.  Las diferencias en la tasa de aumento natural entre los grupos de ingresos altos y bajos son ciertas solo para una población en rápido crecimiento, con o sin inmigración, pero acompañadas por tasas de mortalidad y tasas de natalidad decrecientes, un patrón demográfico asociado. 
El impacto de las nuevas industrias sobre la obsolescencia de la riqueza ya establecida como fuente de ingresos de la propiedad está claramente en función del rápido crecimiento, y cuanto más rápido sea el crecimiento, mayor será el impacto.
El efecto de los cambios entre industrias en el aumento del ingreso per cápita, particularmente de los grupos de ingresos más bajos, también es una función del crecimiento, ya que solo en una economía en crecimiento hay un gran cambio en la importancia relativa de los diversos sectores industriales. \textbf{Entonces se puede decir, en general, que el factor básico que milita contra el aumento de la participación de los ingresos altos que se produciría por los efectos acumulativos de la concentración del ahorro, es el dinamismo de una sociedad económica libre y en crecimiento.}\\

Sin embargo, si bien la discusión responde a la pregunta original, no arroja una respuesta determinada sobre si la tendencia en la participación de los ingresos de los grupos superiores es ascendente, descendente o constante. Incluso para la pregunta específica discutida, una respuesta determinada depende del equilibrio relativo de factores: la concentración continua de ahorros que produce una participación creciente y las fuerzas compensatorias que tienden a cancelar este efecto. 
Para saber cuál será probablemente la tendencia de las acciones de ingresos altos, necesitamos saber mucho más sobre el peso de estas presiones en conflicto. Además, la discusión ha sacado a la superficie factores que, por sí mismos, pueden causar una tendencia ascendente o descendente en la participación de los grupos de ingresos altos y, por lo tanto, en la desigualdad de ingresos, en las distribuciones del ingreso anual o secular. 
Por ejemplo, los nuevos entrantes en los grupos superiores los $"$migrantes$"$ ascendentes, que ascienden ya sea debido a una habilidad excepcional o apego a nuevas industrias o por una variedad de otras razones, pueden estar ingresando al grupo superior fijo de, digamos, los $5$ mejores por ciento,  con un diferencial de ingresos, ya sea anual o de largo plazo, que puede ser relativamente mayor que el de los entrantes en la generación anterior. 
Nada en el argumento hasta ahora excluye esta posibilidad, lo que significaría un aumento en la participación de los grupos de ingresos altos, incluso si la participación de la antigua parte $"$residente$"$ permanece constante o incluso disminuye. Incluso sin tener en cuenta otros factores que se señalarán en la siguiente sección, no se puede derivar ninguna conclusión firme en cuanto a las tendencias de las acciones de ingresos altos del modelo básico discutido. La búsqueda de más datos podría arrojar evidencia que permitiría una conclusión razonablemente aproximada pero determinada; pero no tengo tal evidencia a la mano.

\subsection*{El cambio de los sectores agrícolas a los no agrícolas}
¿Qué ocurre con la tendencia a una mayor desigualdad debida al paso de los sectores agrícolas a los no agrícolas? Dada la importancia de la industrialización y la urbanización en el proceso de crecimiento económico, es preciso estudiar sus implicaciones en la evolución de la distribución de la renta, aunque no dispongamos de los datos necesarios ni de un modelo teórico razonablemente completo.\\

Las implicaciones se pueden explicar más claramente con la ayuda de una ilustración numérica (véase el cuadro I). En esta ilustración tratamos dos sectores: la agricultura (A) y todos los demás (B). Para cada sector suponemos distribuciones porcentuales de la renta total del sector entre los deciles del mismo: una distribución (E) es de desigualdad moderada, con porcentajes que comienzan en el $5,5\%$ para el decil más bajo y aumentan 1 punto porcentual de decil a decil hasta alcanzar el $14,5\%$ para el decil superior; la otra distribución (U) es mucho más desigual, con porcentajes que comienzan en el $1\%$ para el decil más bajo y aumentan 2 puntos porcentuales de decil a decil hasta alcanzar el $19\%$ para el decil superior. Asignamos la renta per cápita a cada sector: $50$ unidades a $A$ y $100$ unidades a $B$ en el caso I (líneas 1-10 de la ilustración); $50$ a $A$ y $200$ a $B$ en el caso II (líneas 11-20). Por último, dejamos que la proporción de los números del sector $A$ en el número total disminuya de $0,8$ a $0,2$.\\

Parece más plausible suponer que en períodos anteriores de industrialización, aun cuando la población no agrícola todavía era relativamente pequeña en el total, la distribución de su ingreso era más desigual que la de la población agrícola. Esto sería particularmente así durante los períodos en que la industrialización y la urbanización avanzaban a buen ritmo y la población urbana estaba siendo aumentada, y con bastante rapidez, por inmigrantes, ya sea de las zonas agrícolas del país o del extranjero. En estas condiciones, la población urbana abarcaría toda la gama, desde las posiciones de bajos ingresos de los recién llegados hasta las cimas económicas de los grupos establecidos de ingresos más altos. Podría suponerse que las desigualdades de ingresos urbanos son mucho más amplias que las de la población agrícola que estaba organizada en empresas individuales relativamente pequeñas.\\
Los supuestos básicos utilizados en todo momento son que la renta per cápita del sector B (no agrícola) es siempre mayor que la del sector A; que la proporción del sector A en el número total disminuye; y que la desigualdad de la distribución de la renta dentro del sector A puede ser tan amplia como la del sector B, pero no más.\\

En primer lugar, si el diferencial de renta per cápita aumenta, o si la distribución de la renta es más desigual para el sector $B$ que para el sector $A$, o si se dan ambas condiciones, el aumento a lo largo del tiempo del peso relativo del sector $B$ provoca un marcado incremento de la desigualdad en la distribución de la renta en todo el país.\\

En segundo lugar, si la distribución intrasectorial de la renta es la misma para ambos sectores, y la ampliación de la desigualdad en la distribución de la renta a nivel nacional se debe únicamente al aumento del diferencial de renta per cápita a favor del sector $B$, dicha ampliación es mayor cuando las distribuciones intrasectoriales de la renta se caracterizan por una desigualdad moderada en lugar de amplia. Así, si las distribuciones intrasectoriales son del tipo $E$, el rango de la distribución a nivel nacional se amplía de $23,7$ a $26,3$ a medida que la proporción de $A$ disminuye de $0,8$ a $0,2$ y que la relación entre la renta per cápita del sector $B$ y la del sector $A$ pasa de $2$ a $4$ (véase la línea 4, col. 1, y la línea 14, col. 7). Si se utilizan las distribuciones U, el rango, en idénticas condiciones, sólo se amplía de 36,8 a 37,9 (véase la línea 7, col. 1, y la línea 17, col. 7). Esta diferencia se pone de manifiesto más claramente por el cambio en la cuota del primer quintil, que es el que más sufre la ampliación de la desigualdad: para la distribución E, la cuota cae de 10,5 (línea 2, col. 1) a 5,9 (línea 12, col. 7); para la distribución U, de 3,8 (línea 5, col. 1) a 3,1 (línea 15, col. 7).\\

En tercer lugar, si el diferencial de renta per cápita entre sectores es constante, pero la distribución intrasectorial de $B$ es más desigual que la de $A$, la desigualdad creciente en la distribución a nivel nacional es tanto mayor cuanto menor sea el diferencial de renta per cápita supuesto. Así, para un diferencial de $2$ a $1$, el rango se amplía de $28,3$ cuando la proporción de $A$ es de $0,8$ (línea 10, col. 1) a $36,0$ en el punto máximo cuando la proporción de $A$ es de $0,5$ (línea 10, col. 4) y sigue siendo de $33,8$ cuando la proporción de $A$ baja a $0,2$ (línea 10, col. 7). Para un diferencial de renta per cápita de $4$ a $1$, la ampliación del rango en el máximo es sólo de $44,2$ (línea 20, col. 1) a $49,8$ (línea 20, col. 2) y luego el rango desciende a $37,2$ (línea 20, col. 7), muy por debajo del nivel inicial.\\

Cuarto, los supuestos utilizados en la ilustración numérica de un aumento en las proporciones del número total en la sección B, de una mayor desigualdad en la distribución dentro del sector B, y del creciente exceso del ingreso per cápita en B sobre el de A, producen una disminución en la participación del primer quintil que es mucho más conspicuo que el aumento en la participación del quinto quintil. Así, la participación del 1er quintil, con la proporción de A en 0,8, distribución en B más desigual que en A, y un diferencial de ingreso per cápita de 2 a 1, es 9,3 (línea 8, col. 1). A medida que cambiamos a una proporción de A de 0,2 y un diferencial de ingreso per cápita de 4 a 1, la participación del primer quintil cae a 3,8 (línea 18, columna 7). En las mismas condiciones, la participación del quintil 5 cambia de 37,7 (línea 9, col. 1) a 40,9 (línea 19, col. 7).\\

Quinto, incluso si el diferencial en el ingreso per cápita entre los dos sectores permanece constante y las distribuciones intrasectoriales son idénticas para los dos sectores, el cambio en las proporciones de los números produce cambios leves pero significativos en la distribución del país como un todo entero. En general, a medida que la proporción de A se desplaza de 0,8 hacia abajo, el rango tiende primero a ensancharse y luego a disminuir. Cuando el diferencial de ingreso per cápita es bajo (2 a 1), la ampliación del rango alcanza un pico cercano a la mitad de la serie, es decir, en una proporción de A igual a 0,6 (líneas 4 y 7); y los movimientos en el rango tienden a ser bastante limitados. Cuando el diferencial de ingreso per cápita es grande (4 a 1), el rango se contrae tan pronto como la proporción de A pasa el nivel de 0.7, y la disminución en el rango es bastante sustancial (líneas 14 y 17).\\

En sexto lugar, de particular importancia para las participaciones de los grupos de ingresos altos es el hallazgo de que la participación del quintil superior disminuye a medida que la proporción de A cae por debajo de cierta fracción bastante alta del número total. No hay un solo caso en la ilustración en el que la participación del quinto quintil no disminuya, ya sea a lo largo o ancho de un segmento sustancial de la secuencia en el movimiento descendente de la proporción de A de 0,8 a 0,2. En las líneas 6 y 9, la participación del quinto quintil desciende más allá del punto en que la proporción de A es 0,6; y en el resto de renglones relevantes la tendencia a la baja en la participación del quintil 5 se establece antes. La razón radica, por supuesto, en el hecho de que al aumentar la industrialización, el peso creciente del sector no agrícola, con su mayor ingreso per cápita, eleva el ingreso per cápita de toda la economía; y, sin embargo, el ingreso per cápita dentro de cada sector y las distribuciones intrasectoriales se mantienen constantes. En tales condiciones, las proporciones superiores no disminuirían sólo si hubiera un aumento mayor en el ingreso per cápita del sector B que en el del sector A; o el aumento de la desigualdad en la distribución intrasectorial del sector B.\\

Sabemos que la renta per cápita es mayor en el sector B que en el sector A; que, en el mejor de los casos, el diferencial de ingresos per cápita entre los sectores A y B ha sido bastante constante (por ejemplo, en los Estados Unidos) y quizás ha aumentado con mayor frecuencia; que la proporción del sector A en números totales ha disminuido. Entonces, si comenzamos con una distribución intrasectorial de B más desigual que la de A, esperaríamos los resultados sugeridos por las líneas 8-10 o 18-20. En el primer caso, el rango se amplía a medida que la proporción de A cae de 0,8 a 0,5, y luego se estrecha. En el último caso, el rango declina más allá del punto en el que la proporción de A es 0,7. Pero en ambos casos, la participación del 1er quintil disminuye, y de manera bastante apreciable y continua (véanse las líneas 8 y 18). La magnitud y continuidad de la disminución son en parte el resultado de los supuestos específicos realizados; pero estaría justificado argumentar que dentro de los amplios límites sugeridos por la ilustración, el supuesto de una mayor desigualdad en la distribución intrasectorial para el sector B que para el sector A, arroja una tendencia a la baja en la participación de los grupos de menores ingresos. Sin embargo, no encontramos tal tendencia en la evidencia empírica que tenemos. ¿Podemos suponer que en los períodos anteriores la distribución interna del sector B no era más desigual que la del sector A, a pesar de los indicios más recientes de que la distribución del ingreso urbano es más desigual que la rural?.\\

Obviamente, hay lugar para la conjetura. Parece más plausible suponer que en períodos anteriores de industrialización, aun cuando la población no agrícola todavía era relativamente pequeña en el total, la distribución de su ingreso era más desigual que la de la población agrícola. Esto sería particularmente así durante los períodos en que la industrialización y la urbanización avanzaban a buen ritmo y la población urbana estaba siendo aumentada, y con bastante rapidez, por inmigrantes, ya sea de las zonas agrícolas del país o del extranjero. En estas condiciones, la población urbana abarcaría toda la gama, desde las posiciones de bajos ingresos de los recién llegados hasta las cimas económicas de los grupos establecidos de ingresos más altos. Podría suponerse que las desigualdades de ingresos urbanos son mucho más amplias que las de la población agrícola que estaba organizada en empresas individuales relativamente pequeñas (las unidades a gran escala eran más raras entonces que ahora).\\

Si admitimos el supuesto de una mayor desigualdad distributiva en el sector B, la participación de los tramos de menores ingresos debería haber mostrado una tendencia a la baja. Sin embargo, el resumen anterior de la evidencia empírica indica que durante los últimos 50 a 75 años no ha habido una ampliación de la desigualdad de ingresos en los países desarrollados sino, por el contrario, una reducción en las últimas dos a cuatro décadas. De ello se deduce que la distribución intrasectorial, ya sea para el sector A o para el sector B, debe haber mostrado una reducción suficiente de la desigualdad para compensar el aumento requerido por los factores discutidos. Específicamente, las participaciones de los grupos de menores ingresos en los sectores A y/o B deben haber aumentado lo suficiente como para compensar la disminución que de otro modo se habría producido por una combinación de los elementos que se muestran en la ilustración numérica.\\

Este estrechamiento de la desigualdad, el aumento compensatorio en las proporciones de los tramos más bajos, muy probablemente ocurrió en la distribución del ingreso de los grupos urbanos, en el sector B. Si bien también puede haber estado presente en el sector A, habría tenido un efecto más limitado. efecto sobre la desigualdad en la distribución del ingreso a nivel nacional debido a la rápida disminución del peso del sector A en el total. Tampoco era probable tal reducción de la desigualdad de ingresos en la agricultura: con la industrialización, un nivel más alto de tecnología permitió unidades de mayor escala y, en los Estados Unidos por ejemplo, agudizó el contraste entre los grandes y exitosos agricultores comerciales y los aparceros de subsistencia de la agricultura. Sur. Además, dado que aceptamos el supuesto de una desigualdad inicialmente menor en la distribución interna del ingreso en el sector A que en el sector B, cualquier reducción significativa de la desigualdad en el primero es menos probable que en el segundo.

Por lo tanto, podemos concluir que la mayor compensación a la ampliación de la desigualdad de ingresos asociada con el cambio de la agricultura y el campo a la industria y la ciudad debe haber sido un aumento en la participación en el ingreso de los grupos más bajos dentro del sector no agrícola de la población. . Esto proporciona una pista para la exploración en lo que me parece una dirección muy prometedora: la consideración del ritmo y el carácter del crecimiento económico de la población urbana, con particular referencia a la posición relativa de los grupos de bajos ingresos. Hay mucho que decir sobre la noción de que una vez que pasaron las primeras fases turbulentas de industrialización y urbanización, una variedad de fuerzas convergieron para reforzar la posición económica de los grupos de bajos ingresos dentro de la población urbana. El mismo hecho de que después de un tiempo, una proporción cada vez mayor de la población urbana fuera "nativa", es decir, nacida en las ciudades en lugar de en las áreas rurales, y por lo tanto más capaz de aprovechar las posibilidades de la vida de la ciudad en preparación para la economía lucha, significó una mejor oportunidad para la organización y la adaptación, una mejor base para asegurar una mayor participación en el ingreso de lo que era posible para la nueva población "inmigrante" que venía del campo o del extranjero. También debe tenerse en cuenta la creciente eficiencia de la población urbana establecida de mayor edad. Además, en las sociedades democráticas, el creciente poder político de los grupos urbanos de bajos ingresos condujo a una variedad de leyes protectoras y de apoyo, muchas de las cuales apuntaban a contrarrestar los peores efectos de la rápida industrialización y urbanización y a respaldar los reclamos de las amplias masas por proporciones más adecuadas de los ingresos crecientes del país. El espacio no permite la discusión de las consideraciones demográficas, políticas y sociales que podrían aplicarse para explicar las compensaciones de cualquier declive en los grupos, declives que de otro modo serían deducibles de las tendencias sugeridas en la ilustración numérica.


\section*{\centering III. Otras tendencias relacionadas con las de la desigualdad de ingresos}




