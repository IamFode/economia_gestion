\chapter{Introducción a la escritura académica}

\section{¿Por qué hacemos investigación?}
Un proyecto de investigación requiere que:
\begin{itemize}
    \item Investigues un tema.
    \item obtengas una compresión de sus aspectos esenciales.
    \item Divulgue sus hallazgos.
\end{itemize}

\section{Aprendiendo las conversaciones de la escritura académica}

Tres convenciones de investigación clave:
\begin{itemize}
    \item Análisis: Clasifique los temas principales de su estudio y proporcione un análisis detallado de cada uno en defensa de su tesis.
    \item Evidencia: Proporcione proposiciones y declaraciones bien razonadas que estén respaldadas por hechos, detalles y evidencia con la documentación adecuada. 
    \item Discusión: Relacione las implicaciones de sus hallazgos y los méritos del estudio, ya sea las técnicas poéticas de un autor, un movimiento histórico o un problema social.
\end{itemize}

\section{Comprender y evitar el plagio}
La convención más importante de la escritura académica es el principio de dar el debido crédito al trabajo de otros. El uso no reconocido de las oraciones, frases o terminología de otra persona es plagio, así que proporcione una cita y use comillas. Del mismo modo, el uso no reconocido de las ideas, la investigación o el enfoque de otra persona también es plagio, así que escriba paráfrasis cuidadosas.

\paragraph{Evitar el plagio no intencional} 
Las siguientes pautas lo ayudarán a evitar el plagio no intencional. 
\begin{itemize}
    \item Citación. Informe a los lectores cuando toma prestado de una fuente introduciendo una cita o paráfrasis con el nombre de su autor. 
    \item Comillas. Encierre entre comillas todas las palabras, frases y oraciones citadas. 
    \item Parafrasear. Proporcione una cita para indicar la fuente de una paráfrasis tal como lo hace con las citas. 
    \item Citas y notas entre paréntesis. Utilice uno de los estilos de documentación académica (MLA, APA, CMS o CSE) para proporcionar citas específicas en el texto para cada fuente de acuerdo con las convenciones de la disciplina en la que está escribiendo. 
    \item Obras citadas o páginas de referencias. Proporcione una entrada de bibliografía completa al final de su artículo para cada fuente que utilice, conforme a los estándares del estilo de documentación que está utilizando.
\end{itemize}


\section{Comprender una tarea de investigación}
\subsection{Comprender la terminología}
En las ciencias, sus experimentos y pruebas generalmente requerirán una discusión de las implicaciones de sus hallazgos.

\subsubsection{Evaluación}
Para evaluar, primero debe establecer criterios claros de juicio y luego explicar cómo el sujeto cumple con estos criterios. Por ejemplo, las evaluaciones de los miembros de la facultad por parte de los estudiantes se basan en un conjunto de criterios expresados: un interés en el progreso de los estudiantes, un conocimiento profundo de la materia, etc. 
\begin{center}
    \textbf{Su primer paso debe ser crear sus criterios}
\end{center}

\subsubsection{Interpretación}
as preguntas a menudo apuntan hacia la interpretación: ¿Qué significa este pasaje? ¿Cuáles son las implicaciones de estos resultados? ¿Qué nos dice este dato? ¿Puedes explicar tu lectura del problema a otros?.\\

\subsubsection{Definición}
Ejemplo: La causa fundamental de las rupturas en las relaciones es el egoísmo. Este tema requerirá una definición de egoísmo y ejemplos de cómo debilita las relaciones.\\
Una buena definición suele incluir tres elementos: 
\begin{itemize}
    \item el sujeto (dieta baja en grasas); 
    \item la clase a la que pertenece el sujeto (dietas en general); 
    \item y las diferencias entre otros en esta clase (bajo en carbohidratos o Atkins). 
\end{itemize}
La definición casi siempre se convertirá en parte de su trabajo cuando parte de la terminología es subjetiva. Si argumenta, por ejemplo, que los experimentos médicos en animales son crueles e inhumanos, es posible que deba definir lo que quiere decir con crueldad y explicar por qué se deben aplicar estándares humanitarios a animales que no son humanos. Por lo tanto, la definición podría servir como su tesis principal. 

\subsubsection{Propuesta}
\textbf{Una propuesta le dice al lector: Deberíamos hacer algo.} A menudo tiene aplicaciones prácticas, como se muestra en este ejemplo: Para mantener la integridad académica, los administradores universitarios deben promulgar políticas estrictas y castigos por hacer trampa y plagiar.\\
Una propuesta exige acción: un cambio en la política, un cambio en la ley y, a veces, una alteración de los procedimientos aceptados.\\
Una propuesta exige consideraciones especiales:
\begin{itemize}
    \item Primero, los escritores deben convencer a los lectores de que existe un problema y es lo suficientemente grave como para merece una acción.
    \item Segundo, los escritores deben explicar las consecuencias para convencer al lector de que la propuesta tiene validez.
    \item Tercero, los escritores deben abordar las posiciones opuestas, las propuestas en competencia y las soluciones alternativas. 
\end{itemize}

El escritor deberá tomar nota de los puntos de vista opuestos y considerarlos en el artículo.

\subsubsection{Argumento causal}
A diferencia de las propuestas, que predicen las consecuencias, los argumentos causales muestran que existe una condición debido a circunstancias específicas, es decir, algo ha causado o creado esta situación y necesitamos saber por qué.

\subsubsection{Comparación, incluida la analogía}
Un argumento a menudo compara y compara un tema con otra cosa. Es posible que le pidan que compare un par de poemas o que compare los mercados bursátiles.\\
Una analogía es una comparación figurativa que le permite al escritor trazar varios paralelos de similitud Por ejemplo, el sistema circulatorio humano es como un sistema de transporte con un centro, un sistema de carreteras y una flota de camiones para transportar la carga.

\subsubsection{Procedencia}
Como ejemplo. Si el investigador puede probar que otra planta en otra parte del país arruinó el medio ambiente, entonces el investigador tiene un precedente de cuán dañina puede ser tal operación.

\subsubsection{Implicaciones}
En algún momento, sin embargo, se espera que explique sus hallazgos, llegue a conclusiones y discuta las implicaciones de su investigación científica.

\section{Establecer un cronograma de investigación}

\begin{itemize}
    \item  Encontrar y delimitar un tema. Su tema debe tener una pregunta o argumento incorporado para que pueda interpretar un problema y citar las opiniones encontradas en los materiales de su curso.
    \item  Redacción de una tesis y propuesta de investigación. Incluso si no está obligado a crear una propuesta de investigación formal, debe redactar algún tipo de plan para ayudar a dirigir y organizar su investigación antes de comenzar a leer en profundidad. Capítulo 3 2f 2g.
    \item Lectura y creación de una bibliografía de trabajo. La lectura preliminar establece la base para su investigación, ayudándole a descubrir la cantidad y calidad de las fuentes disponibles. Si no puede encontrar mucho, su tema es demasiado limitado. Si encuentra demasiadas fuentes, su tema es demasiado amplio y debe reducirse. Capítulo 4 y 5.
    \item  Creación de notas. Comience a ingresar notas en una revista de investigación digital o impresa. Algunas notas serán resúmenes, otras serán citas cuidadosamente seleccionadas de las fuentes y algunas serán paráfrasis escritas con su propia voz. Capítulo 9.
    \item  Organizar y esquematizar. Es posible que deba crear un esquema formal; los esquemas formales e ideas adicionales para organizar sus ideas se presentan en las secciones. Capítulo 9 9h y 9i.
    \item  Redacción del documento. Mientras escribe, deje que su instructor escanee el borrador para brindarle comentarios y orientación. Él o ella podrían ver más complicaciones para su exploración y también alejarlo de cualquier conclusión simplista. La redacción es también una etapa para la revisión por pares, en la que uno o dos compañeros de clase miran tu trabajo. La Sección 13a, páginas 237–239, brinda más detalles sobre la revisión por pares. Los capítulos 10 a 12 explican aspectos de la redacción del documento. 
    \item  Dar formato al papel. El diseño adecuado del documento coloca su documento dentro del formato requerido para su disciplina, como el sistema numérico para un proyecto científico o el estilo APA para un documento educativo. Los capítulos 14 a 17 proporcionan las pautas para las diversas disciplinas. 
    \item  Escribir una lista de sus referencias. Deberá enumerar en el formato adecuado las diversas fuentes utilizadas en su estudio. Los capítulos 14 a 17 proporcionan pautas de documentación.
    \item  Revisión y corrección. Al final del proyecto, debe ser consciente de examinar el manuscrito y hacer todas las correcciones necesarias. Con la ayuda de las computadoras, puede revisar la ortografía y algunos aspectos de estilo. El Capítulo 13 da consejos sobre revisión y edición. El Glossasry es una lista de términos que explica aspectos de forma y estilo. 
    \item  Envío del manuscrito. Al igual que todos los escritores, necesitará en algún momento "publicar" el documento y entregarlo a la audiencia, que podría ser su instructor, sus compañeros de clase o quizás un grupo más grande. Planifique con suficiente antelación para cumplir con este plazo final. Puede presentar el documento de varias maneras: en papel, por correo electrónico a su instructor, en una unidad flash USB, en un buzón o en su propio sitio web.
\end{itemize}


\chapter{Selección de temas}

Este capítulo traza una dirección para su proyecto de investigación: 
\begin{itemize}
    \item Relacionar ideas personales con un problema académico.
    \item Hablar con otros para refinar el tema. 
     \item Refinar su tema a través de fuentes en línea.
     \item Utilizar bases de datos y recursos electrónicos para perfeccionar su tema.
     \item Desarrollar una declaración de tesis, entimema o hipótesis.
     \item Redacción de una propuesta de investigación.
 \end{itemize}

Tenga en cuenta que un tema académico requiere investigación y resolución de problemas.

\paragraph{Limitar un tema general a un tema académico}

A diferencia de un tema general, un tema académico debe:

\begin{itemize}
    \item Examine un tema limitado, no un tema amplio. 
    \item Diríjase a lectores conocedores y llévelos a otra plataforma de conocimiento. 
    \item Tener un propósito serio, uno que exija el análisis de los problemas, argumente desde una posición y explique detalles complejos. 
    \item Cumplir con las expectativas del instructor y ajustarse a los requisitos del curso.
\end{itemize}

\subsection{Relacionar sus ideas personales con un problema académico}
Comience con dos actividades: 
\begin{enumerate}[1.]
    \item Relacione sus experiencias con problemas académicos y disciplinas académicas. 
    \item Especular sobre el tema enumerando problemas, haciendo preguntas, participando en la escritura libre y utilizando otras técnicas de generación de ideas.
\end{enumerate}

\subsubsection{Conexión de la experiencia personal con temas académicos}
Aprenda el lenguaje especial de la disciplina académica y utilícelo. Cada campo de estudio, ya sea sociología, geología o literatura, tiene palabras para describir su enfoque analítico de temas, como la demografía de un público objetivo (marketing), la función de bucles y matrices (informática), el simbolismo de Maya La poesía de Angelou (literatura) y la observación de sujetos humanos (psicología). Parte de su tarea es aprender la terminología y usarla apropiadamente.

\subsubsection{Especular sobre su tema para descubrir ideas y concentrarse en los problemas}
En algún momento, es posible que necesite sentarse, relajarse y usar su imaginación para contemplar las cuestiones y los problemas que vale la pena investigar. Las ideas se pueden generar de las siguientes maneras:

\paragraph{Escritura libre}
Para escribir libremente, simplemente concéntrate en un tema y escribe lo que te venga a la mente. No se preocupe por la gramática, el estilo o la caligrafía, pero siga escribiendo sin parar durante una página más o menos para desarrollar frases valiosas, comparaciones, anécdotas personales y pensamientos específicos que ayuden a enfocar los temas de interés.

\paragraph{Listado de palabras clave}

\paragraph{Organizar palabras clave en un esquema aproximado}

\paragraph{Agrupación}
Otro método para descubrir la jerarquía de sus temas principales y subtemas es agrupar ideas en torno a un tema central. El conjunto de temas relacionados puede generar multitud de ideas interconectadas.

\paragraph{Reducción por comparación }
La comparación limita una discusión a diferencias específicas. Dos obras cualesquiera, dos personas cualesquiera, dos grupos cualesquiera pueden servir como base para un estudio comparativo. Los historiadores comparan a los comandantes de la Guerra Civil Robert E. Lee y Ulysses S. Grant. Los politólogos comparan conservadores y liberales. Los estudiosos de la literatura comparan los méritos del verso libre y los del verso formal.

\paragraph{Haciendo preguntas}
La investigación es un proceso de búsqueda de respuestas a preguntas. Por lo tanto, los investigadores más efectivos son aquellos que aprenden a hacer preguntas y buscar respuestas. Plantear preguntas sobre el tema puede proporcionar límites claros para el documento. Estira tu imaginación con preguntas para desarrollar un tema claro.


\section{Hablar con otros para refinar el tema}
\subsection{Entrevistas personales y discusiones}
Al igual que algunos investigadores, es posible que deba consultar formalmente con un experto en el tema o explorar un tema de manera informal mientras toma un café o un refresco con un colega, pariente o compañero de trabajo.

\subsection{Grupos de discusión en línea}
¿Qué dicen otras personas sobre tu tema?

\paragraph{Explorar ideas con otros}
\begin{itemize}
    \item Consulte con su instructor. 
    \item Discuta su tema con tres o cuatro compañeros de clase. 
    \item Escuche las preocupaciones de los demás. 
    \item Lleve a cabo una discusión o una entrevista. 
    \item Únase a un grupo de discusión en la web. 
    \item Tome notas cuidadosas. 
    \item Ajuste su investigación en consecuencia.
\end{itemize}

\section{Uso de búsquedas en línea para refinar su tema}
\subsection{Uso de un directorio de materias Online}
\subsection{Uso de una búsqueda de palabras clave en Internet}
La mayoría de las bases de datos en línea y los sitios de búsqueda web incluyen el uso de términos de búsqueda booleanos, específicamente AND, OR y NOT.

\subsection{Uso de las bases de datos electrónicas de la biblioteca para encontrar y delimitar un tema}

\begin{itemize}
    \item Seleccione una base de datos. Algunas bases de datos, como InfoTrac y ProQuest, son generales; utilícelos para encontrar un tema. Otras bases de datos se enfocan en una disciplina; por ejemplo, los índices ERIC buscan solo fuentes educativas específicas. Estas bases de datos lo llevarán rápidamente a una lista de artículos sobre su tema. 
    \item Enumere palabras clave o una frase para describir su tema, entre comillas. Evite usar una sola palabra general. Por ejemplo, la palabra silvicultura en la base de datos de la Biblioteca Electrónica produjo más de 5.000 sitios posibles. La frase de dos palabras “conservación forestal” produjo un número más manejable de sitios. Esta es una de las entradas: “Un nuevo año para la política forestal”. Jami Westerhold. bosques americanos. 118.4 (invierno de 2013) p32. 
    \item Examine las distintas entradas en busca de posibles temas. Busque artículos relevantes, explore las descripciones, lea los resúmenes y, cuando encuentre algo valioso, imprima el texto completo, si está disponible.
\end{itemize}

\section{Uso del catálogo de libros electrónicos de la biblioteca para encontrar un tema}
Los instructores esperan que cite información de algunos libros, y el índice de libros de la biblioteca sugerirá temas y confirmará que su tema ha sido tratado con estudios profundos en forma de libro, no solo en Internet o en revistas.\\
Inspeccione la tabla de contenido de un libro para encontrar temas de interés.

\section{Desarrollo de una declaración de tesis, entimema o hipótesis}
Comience a pensar en términos de una idea central. Cada uno tiene una misión separada: 
\begin{itemize}
    \item Una declaración de tesis avanza una conclusión que el escritor defenderá: Al contrario de lo que han adelantado algunos filósofos, los seres humanos siempre han participado en guerras. 
    \item Un entimema usa una cláusula porque para hacer una afirmación que el escritor defenderá: nunca ha habido un “buen salvaje”, como tal, porque incluso los seres humanos prehistóricos libraron guerras frecuentes por numerosas razones. 
    \item Una hipótesis es una teoría que debe probarse en el laboratorio, en la literatura y/o mediante investigación de campo para demostrar su validez: los seres humanos están motivados por instintos biológicos hacia el derrocamiento físico de los enemigos percibidos.
\end{itemize}

\subsection{Tesis}
Una declaración de tesis expande su tema en una propuesta académica, una que intenta probar y defender en su artículo. Se debe encontrar un enfoque crítico.\\

Un tema puede generar varias cuestiones entre las que el escritor puede elegir: 
\begin{itemize}
    \item Enfoque biológico: los alimentos funcionales pueden ser una adición prometedora a la dieta de aquellos que desean evitar ciertas enfermedades. 
    \item Enfoque económico: los alimentos funcionales pueden convertirse en un arma económica en la batalla contra el aumento de los costos de atención médica. 
    \item Enfoque histórico: Otras civilizaciones, incluidas las tribus primitivas, conocen las propiedades curativas de los alimentos desde hace siglos. ¿Por qué dejamos que la química moderna nos cegara a sus beneficios?
\end{itemize}

\subsection{Entimema}
Entimema: los niños hiperactivos necesitan medicación porque el TDAH es un trastorno médico, no un problema de comportamiento. Entimema: Debido a que la gente está muriendo en todo el mundo por la escasez de agua, los países con abundancia de agua tienen la obligación ética de compartirla.

\subsection{Hipótesis}
Una hipótesis propone una teoría o sugiere una explicación para algo. Tipos de hipótesis

\subsubsection{La hipótesis teórica}
La discriminación contra las mujeres jóvenes en el salón de clases, conocida como defraudación, perjudica a las mujeres académica, social y psicológicamente. Aquí, el estudiante producirá un estudio teórico citando literatura sobre cambios cortos.

\subsubsection{La hipótesis condicional}
La diabetes se puede controlar con medicamentos, control, dieta y ejercicio. Se deben cumplir ciertas condiciones. El control dependerá de la capacidad del paciente para realizar las cuatro tareas adecuadamente para probar la hipótesis válida.

\subsubsection{La hipótesis relacional}
El tamaño de la clase afecta la cantidad de tareas escritas de los instructores de redacción. Este tipo de hipótesis afirma que a medida que cambia una variable, también lo hace otra, o afirma que algo es más o menos que otro. Podría probarse examinando y correlacionando el tamaño de la clase y las tareas, un tipo de investigación de campo.

\subsubsection{La hipótesis causal}
El juguete de un niño está determinado por los comerciales de televisión. Esta hipótesis causal asume la ocurrencia mutua de dos factores y afirma que un factor es responsable del otro. El estudiante que es padre puede realizar una investigación para probar o refutar la suposición. Una revisión de la literatura también podría servir al escritor.

\section{Redacción de una propuesta de investigación}
Una propuesta de investigación se presenta en una de dos formas:
\begin{itemize}
    \item Un breve párrafo para identificar el proyecto para usted y su instructor, o 
    \item un informe formal de varias páginas que brinda información general, su justificación para realizar el estudio, una revisión de la literatura, sus métodos y las conclusiones que espera probar.
\end{itemize}

\subsection{La propuesta breve}
Una propuesta breve identifica cinco ingredientes esenciales de su trabajo:
\begin{itemize}
    \item El tema específico.
    \item El propósito del artículo (para explicar, analizar o argumentar) 
    \item La audiencia prevista (general o especializada) 
    \item Su voz como escritor (informador o defensor) 
    \item La declaración de tesis preliminar o hipótesis de apertura
\end{itemize}
Ejemplo:\\
El mundo se está quedando sin agua dulce mientras bebemos nuestro Evian. Sin embargo, la moda del agua embotellada indica algo: no confiamos en el agua fresca del grifo. Tenemos una crisis emergente en nuestras manos, y algunas autoridades pronostican guerras mundiales por los derechos de agua. El tema del agua toca casi todas las facetas de nuestras vidas, desde los rituales religiosos y el suministro de alimentos hasta las enfermedades y la estabilidad política. Podríamos enmarcar esta hipótesis: el agua pronto reemplazará al petróleo como el recurso económico más preciado por las naciones del mundo. Sin embargo, esa afirmación resultaría difícil de defender y puede no ser cierta en absoluto. Más bien, debemos mirar a otra parte, al comportamiento humano y a la responsabilidad humana de preservar el medio ambiente para nuestros hijos. En consecuencia, este documento examinará (1) los problemas relacionados con la oferta y la demanda, (2) las luchas de poder político que pueden surgir y (3) las implicaciones éticas para quienes controlan el suministro disperso de agua dulce en el mundo.

\subsection{La propuesta larga}
\begin{enumerate}[1.]
    \item Una portada con el título del proyecto, su nombre y la persona o agencia a la que le envía la propuesta.
    \item Si es necesario, agregue un resumen que resuma su proyecto en 50 a 100 palabras (consulte la página 327 para obtener información adicional).
    \item Incluya una declaración de propósito con su justificación para el proyecto. En esencia, esta es su declaración de tesis o hipótesis, junto con su identificación de la audiencia a la que se dirigirá su trabajo y el papel que desempeñará como investigador y defensor.
    \item Una declaración de cualificación que explique su experiencia y, quizás, las cualidades especiales que aporta al proyecto. 
    \item Una revisión de la literatura, que analiza los artículos y libros que ha examinado en su trabajo preliminar (consulte las páginas 146–152 para obtener una explicación y otro ejemplo).
    \item Una descripción de sus métodos de investigación, que es el diseño de los materiales que necesitará, su cronograma y, en su caso, su presupuesto. Estos elementos son a menudo parte de un estudio científico, así que consulte los Capítulos 15 y 17 para el trabajo en las ciencias sociales, físicas y biológicas.
\end{enumerate}

\paragraph{Explicar su propósito en la propuesta de investigación}
Los trabajos de investigación cumplen varias tareas: 
\begin{itemize}
    \item Explican y definen el tema. 
    \item Analizan los temas específicos. 
    \item Convencen al lector con el peso de la evidencia.
\end{itemize}
\begin{enumerate}[1.]
    \item  Use la explicación para revisar y detallar datos fácticos. Sarah Bemis explica cómo se puede controlar la diabetes (véanse las páginas 364–374), y Clare Grady explica las presiones asociadas con la carrera espacial en la década de 1960 (véanse las páginas 349–354). 
    \item Usar el análisis para clasificar varias partes del tema e investigar cada una en profundidad. Ashley Irwin examina las emociones en la poesía generadas por eventos trágicos de la vida (páginas 246–254) y Caitlin Kelley analiza la importancia del recreo para los estudiantes de primaria (páginas 328–335). 
    \item Usar la persuasión para cuestionar las actitudes generales sobre un problema y luego afirmar nuevas teorías, proponer una solución, recomendar un curso de acción o, al menos, invitar al lector a un diálogo intelectual.
\end{enumerate}

\section{Tu proyecto de investigación}
\begin{enumerate}[1.]
    \item Intente conectar cualquiera de sus intereses personales con una idea académica sobre el tema que ha elegido investigar. Deberá concentrarse en los problemas que podría investigar. Usa la escritura libre (páginas 32 y 33) para ayudarte con esto. Sin preocuparte por cuestiones gramaticales o estilísticas, anota los pensamientos que se te ocurran sobre el tema que has elegido. Escribe durante una página o dos para aclarar tus pensamientos. Le ayudará a desarrollar el camino a lo largo del cual procederá su investigación. 
    \item Una vez que haya elegido su tema, enumere las palabras clave básicas que encuentre en la literatura sobre ese tema. Organícelos de una manera (dividiendo las palabras clave en temas principales y secundarios) que le ayudarán a concentrarse en la dirección de su investigación (página 33). 
    \item Utilice grupos de chat en línea, foros, etc. para interactuar con otras personas, conocer sus ideas y experiencias, e incluso obtener su opinión sobre su tema de investigación. Pueden señalar aspectos de su investigación que no había considerado. Además, utilice los recursos en línea: Yahoo! Directory, puede ayudarlo a investigar su tema en categorías amplias; Google Scholar puede dirigirlo a trabajos académicos relevantes realizados por otros. Utilice expresiones booleanas, para una búsqueda de palabras clave a fin de centrarse en su tema y restringir su búsqueda a fuentes relevantes. Busque en las bases de datos que puedan estar disponibles en su biblioteca. También puede buscar en el catálogo de libros electrónicos de la biblioteca. 
    \item Estructura tu idea central en la forma de una conclusión a la que has llegado (enunciado de tesis), o una afirmación que estás haciendo (entimema). Incluso podría estructurar la idea en una declaración que se probará (hipótesis) en su investigación. Escriba un borrador de su propuesta de investigación. Puede discutir la propuesta de investigación con otras personas, incluido su instructor, personas mayores o compañeros en el campo relevante, antes de finalizarla.
\end{enumerate}


\chapter{Organización de ideas y establecimiento de objetivos}
Este capítulo proporciona ideas para trazar la dirección de su investigación:
\begin{itemize}
    \item Usar un orden claro para el curso de su trabajo de investigación 
    \item Utilizar su propuesta de investigación para dirigir la toma de notas.
    \item Enumerar términos y frases clave para tomar notas.
    \item Hacer preguntas para identificar problemas. 
    \item Organizar ideas con modos de desarrollo.
    \item Trazar la dirección de su investigación con su tesis.
\end{itemize}

\section{Usar un orden básico para trazar el curso de su trabajo}
El documento debe proporcionar estos elementos:
\begin{itemize}
    \item Identificación del problema o tema.
    \item Una revisión de la literatura sobre el tema.
    \item Su tesis o hipótesis.
    \item Análisis de los problemas.
    \item Presentación de evidencia.
    \item Interpretación y discusión de los hallazgos.
\end{itemize}

En todos los casos, debe generar la dinámica del artículo
\begin{itemize}
    \item generando anticipación en la introducción, 
    \item investigando los problemas en el cuerpo y 
    \item brindando un juicio final. 
\end{itemize}
De esta manera, satisfará las demandas del lector académico, quien esperará que usted: 

\begin{itemize}
    \item Examine un problema 
    \item Cite literatura pertinente al respecto 
    \item Ofrezca sus ideas e interpretación de él
\end{itemize}
Los tres son necesarios en casi todos los casos. En consecuencia, su organización inicial determinará, en parte, el éxito de su trabajo de investigación.

\section{Uso de su propuesta de investigación para dirigir su toma de notas}
Su propuesta de investigación, si desarrolló una, presenta temas dignos de investigación.

\section{Listado de términos y frases clave para establecer instrucciones para tomar notas}
Siga dos pasos bastante simples: 
\begin{itemize}
    \item Anote ideas o palabras en una lista aproximada y 
    \item amplíe la lista para mostrar una jerarquía de ideas principales y secundarias.
\end{itemize}
Lo que busca en este punto son términos que acelerarán su búsqueda en Internet y en los índices de la biblioteca.

\section{Escribir un borrador}
Tan pronto como sea posible, organice su terminología clave en un breve esquema, organizando las palabras y frases en una secuencia ordenada.\\

Para evitar esta forma de plagio intencional y agregar credibilidad a sus propias ideas, el estudiante puede agregar el nombre de la fuente y mezclar la cita directa en su investigación: Según Arthur Ferrill, “La guerra organizada no era nueva; se había practicado durante un milenio en tiempos prehistóricos”.

\section{Uso de preguntas para identificar problemas}
Las preguntas pueden invitarlo a desarrollar respuestas en sus notas.

\section{Establecimiento de objetivos mediante el uso de patrones organizacionales}
Trate de anticipar los tipos de desarrollo o patrones organizacionales que necesitará para construir párrafos efectivos y explorar su tema completamente. Luego basa tus notas en los modos de desarrollo:
\begin{itemize}
    \item Definición, 
    \item comparación y contraste, 
    \item proceso, 
    \item ilustración, 
    \item causa y efecto, 
    \item clasificación, 
    \item análisis y descripción. 
\end{itemize}
Aquí hay una lista de un estudiante que estudió los temas de la donación de órganos y tejidos: Defina donación de tejidos. Contrastar mitos, puntos de vista religiosos y consideraciones éticas. Ilustre la donación de órganos y tejidos con varios ejemplos. Utiliza estadísticas y datos científicos. Busque las causas de la renuencia de una persona a firmar una tarjeta de donante. Determinar las consecuencias de la donación con un enfoque en salvar la vida de los niños. Lea y use un estudio de caso sobre la muerte de un niño y la donación de órganos por parte del público. Explore las etapas paso a paso del proceso de donación de órganos. Clasificar los tipos y analizar el problema. Dé ejemplos narrativos de varias personas cuyas vidas fueron salvadas. Con esta lista en la mano, un escritor puede buscar material para desarrollar como contraste, proceso, definición, etc.\\

Intente desarrollar cada elemento importante de su lista en un párrafo completo. Escribe un párrafo de definición. Escribe un párrafo para comparar y contrastar las actitudes expresadas por las personas sobre la donación de órganos. Luego escribe otro párrafo que dé cuatro o cinco ejemplos. Al hacerlo, estará bien encaminado para desarrollar el documento.\\

\section{Uso de enfoques en todo el plan de estudios para trazar sus ideas}
Cada campo académico brinda una visión especial de un tema determinado.

\section{Usar su tesis para trazar la dirección de su investigación}
A menudo, la declaración de tesis establece la dirección del desarrollo del artículo.

\subsection{Organización por temas}
La declaración de la tesis puede obligar al escritor a abordar varios temas y posiciones. \\
Tesis: Los malentendidos sobre la donación de órganos distorsionan la realidad y ponen serios límites a la disponibilidad de aquellas personas que necesitan un ojo, un hígado o un corazón sano. \\
Problema 1. Muchos mitos engañan a las personas haciéndoles creer que la donación no es ética. \\
Problema 2. Algunos temen que, como pacientes, puedan terminar antes de tiempo debido a sus partes del cuerpo. \\
Problema 3. Las opiniones religiosas a veces se interponen en el camino de la donación. \\

El bosquejo anterior, aunque breve, le da a este escritor tres categorías que requieren una investigación detallada en apoyo de la tesis. La toma de notas se puede centrar en estos tres temas.


\subsection{Organización por causa y efecto}
En otros casos, la declaración de tesis sugiere desarrollo por cuestiones de causa/efecto. \\

Nótese que la tesis del próximo escritor sobre los valores educativos de la televisión señala el camino hacia cuatro áreas muy diferentes que merecen ser investigadas. Formulando una tesis efectiva, 2f, páginas 42–45. Tesis: La televisión puede tener efectos positivos en el desarrollo del lenguaje de un niño. Consecuencia 1. La televisión introduce nuevas palabras. Consecuencia 2. La televisión refuerza el uso de palabras y la sintaxis correcta. Consecuencia 3. Los clásicos literarios cobran vida verbalmente en la televisión. Consecuencia 4. La televisión proporciona los ritmos sutiles y los efectos musicales de los oradores consumados.

\paragraph{Evaluación de su plan general }
\begin{enumerate}[1.]
    \item ¿Cuál es mi tesis? ¿Mis notas y registros defenderán e ilustrarán mi propuesta? ¿Es la evidencia convincente? 
    \item ¿He encontrado el mejor plan para desarrollar la tesis con elementos de argumentación, evaluación, causa/efecto o comparación? 
    \item 3. ¿Debo usar una combinación de elementos, es decir, necesito evaluar el tema, examinar las causas y consecuencias y luego exponer el argumento?
\end{enumerate}


\subsection{Organización por Interpretación y Evaluación}
La evaluación evolucionará a partir de declaraciones de tesis que juzguen un tema según un conjunto de criterios, como el análisis de un poema, una película o una exhibición en un museo. Observe cómo la declaración de tesis del próximo estudiante requiere una interpretación del personaje de Hamlet.
Lista de verificación Evaluación de su plan general 1. ¿Cuál es mi tesis? ¿Mis notas y registros defenderán e ilustrarán mi propuesta? ¿Es la evidencia convincente? 2. ¿He encontrado el mejor plan para desarrollar la tesis con elementos de argumentación, evaluación, causa/efecto o comparación? 3. ¿Debo usar una combinación de elementos, es decir, necesito evaluar el tema, examinar las causas y consecuencias y luego exponer el argumento?.\\

Tesis: Shakespeare manipula los escenarios de los soliloquios de Hamlet para descubrir su naturaleza inestable y pronosticar su fracaso. 1. Su alma está oscura por el incesto de su madre. 2. Parece impotente en comparación con el actor. 3. Es atraído por el magnetismo de la muerte. 4. Se da cuenta de que no puede realizar actos crueles y antinaturales. 5. Se avergüenza de su inactividad en comparación.

\subsection{Organización por comparación}
A veces, una declaración de tesis estipula una comparación sobre el valor de los dos lados de un problema, como se muestra en el bosquejo preliminar de un estudiante: Tesis: La disciplina a menudo implica castigo, pero el abuso infantil agrega otro elemento: la gratificación del adulto. Comparación 1. Una nalgada tiene en el corazón el interés del niño, pero una paliza o un azote no tienen ningún valor redentor. Comparación 2. Los tiempos de espera le recuerdan al niño que las relaciones son importantes y deben ser apreciadas, pero los bloqueos en un armario solo promueven la histeria y el miedo. Comparación 3. El ego y los intereses egoístas de los padres a menudo tienen prioridad sobre el bienestar del niño o los niños.

\section{Tu proyecto de investigación}
\begin{enumerate}[1.]
    \item  Usando las pautas organizativas explicadas en este capítulo, haga una lista, como la de la página 55, para explorar su tema en profundidad. Con base en los elementos de la lista, use los modos de desarrollo para construir párrafos que analicen el tema en detalle. Encontrará que la lista es útil para buscar material para su investigación. 
    \item Determine cómo se desarrollará su trabajo en función de su declaración de tesis. Realice una investigación detallada sobre temas y posiciones para respaldar su tesis. Tome notas sobre los temas relevantes. Asegúrese de que los puntos que elija para defender e ilustrar su propuesta sean convincentes. 
    \item Con base en los patrones de desarrollo discutidos en las páginas 58 y 59, ordene su evidencia y argumentos basados en (a) interpretación y evaluación, (b) comparación o (c) una combinación de ambos. 
    \item Evalúe su progreso general hasta ahora. ¿Está satisfecho con su tesis? ¿Ha cubierto su tema de investigación desde múltiples ángulos académicos? ¿Has decidido cómo vas a proceder?
\end{enumerate}

\chapter{Búsqueda de recursos basados en la web}

