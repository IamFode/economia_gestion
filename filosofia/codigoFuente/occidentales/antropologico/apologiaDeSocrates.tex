\begin{center}
    \huge Apología de Sócrates (Alejandro G. Vigo)\\
    \vspace*{0.5cm}
    \large Platón\\
    Editorial universitaria los clásicos.\\
    \vspace{1cm}
    \Large Apuntes por Fode (Christian L. Paredes Aguilera).
    \vspace{1.5cm}
\end{center}

\section*{\center Introdución}

\begin{multicols}{2}

La apología tiene, sin duda, una especial importancia como fuente para el conocimiento de aspectos fundamentales de la figura de Sócrates. El escrito provee un rico y vivido retrato del modo en que Sócrates enfrentó la instancia decisiva, en la que debió probar la firmeza de sus propias convicciones frente a la amenaza cierta de la muerte. \\\\
Junto a la conciencia de los límites del propio saber, la actitud socrática aparece así, al mismo tiempo, como fundada en ciertas convicciones sólidas acerca de lo que es mejor o más valioso como también acerca del modo de vida que resulta preferible para el hombre. Y tales convicciones son lo suficientemente fuertes como para que Sócrates prefiera morir perseverando en ellas, antes que seguir viviendo a costa de sacrificarlas y dañar así su propia alma.\\\\
\textbf{Sócrates no sabe cómo definir las nociones morales básicas como la piedad, la justicia o la valentía, que son objeto habitual de sus indagaciones, pero sabe que lo más importante son la virtud y los bienes del alma, al punto de estar dispuesto a morir, con tal de no actuar de modo contrario a sus convicciones}\\\\

\textbf{Pues si la ignorancia es un mal para el alma, tanto más lo será allí donde no es reconocida como tal y puede así incluso pasar falsamente por conocimiento.}

\end{multicols}

\section*{\center Proemio (17a-18a)}
pag 34
