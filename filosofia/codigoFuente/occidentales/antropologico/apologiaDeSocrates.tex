\begin{center}
    \huge Apología de Sócrates (Alejandro G. Vigo)\\
    \vspace*{0.5cm}
    \large Platón\\
    Editorial universitaria los clásicos.\\
    \vspace{1cm}
    \Large Apuntes por Fode (Christian L. Paredes Aguilera).
    \vspace{1.5cm}
\end{center}

\section*{\center Introdución}

\begin{multicols}{2}

La apología tiene, sin duda, una especial importancia como fuente para el conocimiento de aspectos fundamentales de la figura de Sócrates. El escrito provee un rico y vivido retrato del modo en que Sócrates enfrentó la instancia decisiva, en la que debió probar la firmeza de sus propias convicciones frente a la amenaza cierta de la muerte. \\\\
Junto a la conciencia de los límites del propio saber, la actitud socrática aparece así, al mismo tiempo, como fundada en ciertas convicciones sólidas acerca de lo que es mejor o más valioso como también acerca del modo de vida que resulta preferible para el hombre. Y tales convicciones son lo suficientemente fuertes como para que Sócrates prefiera morir perseverando en ellas, antes que seguir viviendo a costa de sacrificarlas y dañar así su propia alma.\\\\
\textbf{Sócrates no sabe cómo definir las nociones morales básicas como la piedad, la justicia o la valentía, que son objeto habitual de sus indagaciones, pero sabe que lo más importante son la virtud y los bienes del alma, al punto de estar dispuesto a morir, con tal de no actuar de modo contrario a sus convicciones}\\\\

\textbf{Pues si la ignorancia es un mal para el alma, tanto más lo será allí donde no es reconocida como tal y puede así incluso pasar falsamente por conocimiento.}

\end{multicols}

\section*{\center II Plan de defensa}
\begin{multicols}{2}

$"$\textbf{Sócrates explica que no teme tanto a los nuevos cargos que se le hacen, sino más bien a la posibilidad de que los jueces estén todavía sujetos a la influencia de las muchas calumnias sobre su persona que oyeron desde niños, de parte de personas mayores a cargo de su crianza y educación.}$"$ \\\\
\textbf{En efecto, también en el pasado, y hace ya muchos años, han surgido ante ustedes muchos acusadores míos, que tampoco dijeron nada verdadero. A ellos temo más que a Ánito y sus compañeros. Pues, aunque también éstos son temibles, aquéllos lo son todavía más, ya que, teniendo a cargo a la mayoría de ustedes desde la infancia®, los persuadían, haciendo acusaciones completamente falsas en mi contra, de que hay un tal Sócrates, sabio varón, que especula acerca de los fenómenos celestes e investiga todas las cosas subterráneas, y que convierte al argumento más débil en el más fuerte.}

\end{multicols}

\section*{\center III Defensa de Sócrates}
\begin{multicols}{2}

    \begin{enumerate}[\bfseries 1.]

	\item \textbf{Defensa contra las primeras acusaciones (19a-24b)}\\\\
	    Los cargos principales son:
	    \begin{enumerate}[\bfseries a.]
		\item Indagar las cosas subterráneas y las del cielo,
		\item Convertir el argumento más débil en el más fuerte y
		\item Enseñar a otros lo indicado en a) y b).
	    \end{enumerate}

	    Pero, desde luego, nada de eso es verdad. Y si oyen a alguien decir que yo intento instruir a los hombres y gano así dinero, tampoco esto es e cierto. \\ 
	    El $'$prejuicio$'$ en su contra se refiere, en cambio, a su falsa imagen pública de filósofo natural y sofista, sospechoso de ateísmo y de corromper a la juventud.\\\\
	    Sócrates hace reseña a la $"$profecía$"$ del Oráculo, que menciona que Sócrates era el más sabio entre los sabios, pero Sócrates incrédulo pone a prueba dicha profecía. Para ello se dirige a uno de los que parecía ser más sabios, pero al entablar conversación con este, se da cuenta que no es como creía ser, en ese punto es donde empieza a ganar enemigos en su contra.\\
	    Al ir a comprobar a cada sabio Sócrates se dio cuenta que los que dicen ser sabios no lo eran, y los que eran inferiores, eran hombres mejor dotados para ser sensatos.\\
	    En efecto, después de los políticos me dirigí a los poetas, tanto a los que escriben tragedias como a los que escriben ditirambos, y también al resto, en la idea de que allí podría descubrirme flagrantemente a mí mismo como más ignorante que ellos. Entonces, tomando aquéllas entre sus obras que me parecían estar mejor realizadas, les preguntaba qué querían decir, a fin de poder a la vez aprender algo de ellos. Pues bien, me avergüenza, señores, decirles la verdad, pero debo decirla. En efecto, por así decir, prácticamente todos los aquí presentes podrían dar mejores explicaciones que ellos acerca de las obras que ellos mismos han compuesto.\\
	    \textbf{Pero en realidad, señores, parece que es el dios el que es sabio, y que en aquel oráculo quería decir lo siguiente: que la sabiduría humana vale poco y nada.} A esto se agrega que los jóvenes que disfrutan de mayor tiempo libre, es decir, los hijos de los más ricos comienzan, por su propia iniciativa, a seguirme, pues los divierte oír cómo someto a examen a esos individuos, y a menudo ellos mismos me imitan y tratan entonces de someter a examen a otros.

	\item \textbf{Defensa contra la acusación de Meleto (24b - 28a)}\\\\
	    Afirma que cometo el delito de corromper a los jóvenes. Yo, por mi parte, afirmo, señores atenienses, que es Meleto quien delinque, puesto que se toma en broma algo serio al llevar a juicio con ligereza a las personas, fingiendo estar ocupándose y cuidando de asuntos de los que jamás se preocupó. Es decir, hacer bien a los jovenes. \\
	    \textbf{Ekklesiastaí = miembros de la Ekklesía, la Asamblea de Atenas a la cual podían acudir todos los ciudadanos para discutir y votar sobre asuntos relativos a las políticas del Estado.}\\\\ 
	    Ciertamente, resulta evidente, a mi juicio, que éste se contradice a sí mismo en su acusación, como si dijera: $"$Sócrates comete delito al no creer en dioses, pero creyendo en dioses$"$.\\

	\item \textbf{La misión divina de Sócrates (28a-34b)}\\\\
	    $"$Tras haberse defendido de las acusaciones formales e informales en su contra, Sócrates pasa a explicar la motivación última de su actividad de indagación de sus conciudadanos, a la cual presenta como una misión que le ha sido asignada por el dios Apolo a través de su oráculo. Sócrates intenta presentar su actitud como coherente y fundada en ciertas convicciones de tipo ético y religioso, y, con ello, da también razones para justificar su decisión de no abandonarla.$"$\\\\

	    \textbf{$"$A la hora de cumplir una misión o una orden uno no debe abandonar su puesto, no importa si éste fue escogido por uno mismo o asignado por alguien superior.$"$}\\\\

	    Pues lo que me condenará, si soy condenado, no son Meleto ni Ánito, sino más bien la calumnia y la envidia de la gente. Lo mismo que ha condenado ya a muchos otros hombres hob nestos <me> condenará, creo, también <a mí>. Pues no hay temor de que <esto> se detenga conmigo.\\
	     Porque tenerle miedo a la muerte no es otra cosa, señores, que creer ser sabio sin serlo, ya que es creer saber lo que no se sabe.\\
	     \textbf{O e las riquezas no surge la virtud, sino que es por causa de la virtud como las riquezas y las demás cosas llegan a ser, todas ellas, buenas para los seres humanos, tanto en lo privado como en lo público}

    \item \textbf{Epílogo (34b-35d)}\\\\
pag 83


    \end{enumerate}

\end{multicols}
