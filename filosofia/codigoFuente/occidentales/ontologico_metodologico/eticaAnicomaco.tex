\begin{center}
    \huge Ética a Nicómaco\\
    \vspace*{0.5cm}
    \large Aritóteles\\
    Traducido por Carola Tognetti\\
    Greenbooks editore\\
    Edición digital\\
    \vspace{1cm}
    \Large Apuntes por Fode (Christian L. Paredes Aguilera).
    \vspace{1.5cm}
\end{center}

\part*{Libro primero}
\textbf{De los morales de Aristóteles, escritos a Nicomaco, su hijo y por esta causa llamados nicomaquios}

\begin{multicols}{2}

\section*{capítulo I}
Cualquier arte y cualquier doctrina, y asimismo toda acción y elección, parece que a algún bien es enderezada.

\section*{capítulo II}
\textbf{Porque aunque lo bien para un particular es asimismo bien para una república, mayor, con todo, y más perfecto parece ser para procurarlo y conservarlo el bien de una república. Porque bien es de amar el bien de uno, pero más ilustre y más divina cosa es hacer bien a una nación y a muchos pueblos. Esta doctrina, pues, que es ciencia de república, propone tratar de todas estas que es cosas.}\\

\section*{capítulo III}
Porque las cosas honestas y justas de que trata la disciplina de república, tienen tanta diversidad y escuridad, que parece que son por sola ley y no por naturaleza, y el mismo mal tienen en sí las cosas buenas, pues acontece muchos por causa de ellas ser perjudicados. Pues se ha visto perderse muchos por el dinero y riquezas, y otros por su valentía.

\section*{capítulo IV}
En cuanto al nombre, cierto casi todos lo confiesan, porque así el vulgo, como los más principales, dicen ser la felicidad el sumo bien, y el vivir bien y el obrar bien juzgan ser lo mismo que el vivir prósperamente; pero en cuanto al entender qué cosa es la felicidad, hay diversos pareceres, y el vulgo y los sabios no lo determinan de una misma manera. Porque el vulgo juzga consistir la felicidad en alguna de estas cosas manifiestas y palpables, como en el regalo, o en las riquezas, o en la honra, y otros en otras cosas. Y aun muchas veces a un mismo hombre le parece que consiste en varias cosas, como al enfermo en la salud, al pobre en las riquezas; y los que su propria ignorancia conocen, a los que alguna cosa grande dicen y que excede la capacidad dellos, tienen en gran precio.

\section*{capítulo V}
\textbf{Pero nosotros volvamos al propósito. Porque el bien y la felicidad paréceme que con razón la juzgan, según el modo de vivir de cada uno. Porque el vulgo y gente común por la suma felicidad tienen el regalo, y por esto aman la vida de regalo y pasatiempo.}\\
\textbf{ El vulgo, pues, a manera de gente servil, parece que del todo eligen vida más de bestias que de hombres, y parece que tienen alguna excusa, pues muchos de los que están puestos en dignidad, viven vida cual la de Sardanápalo. Pero los ilustres y para el tratar las cosas aptos, la honra tienen por su felicidad; porque éste casi es el fin de la vida del gobierno de república.}\\

\section*{capítulo VI}
porque siendo ambas cosas amadas, como a más divina cosa es bien hacer más honra a la verdad.\\
Pues ¿de qué manera se dicen bienes? Porque no parece que se digan como las cosas que acaso tuvieron un mismo nombre, sino que se llamen así, por ventura, por causa que, o proceden de una misma cosa, o van a parar a una misma cosa, o por mejor decir, que se digan así por analogía o proporción.\\ 
Si hay una común Idea de todos los bienes, la cual, como el mismo Aristóteles lo dice, es ajena de la moral filosofía, y por esto se ha de tener con ella poca cuenta, vuelve agora a su propósito y prueba cómo la felicidad no puede consistir en cosa alguna de las que por causa de otras se desean, porque las tales no son del todo perletas, y \textbf{la felicidad parece, conforme a razón, que ha de ser tal, que no le falte nada.}

\section*{capítulo VII}
si de esta misma manera presuponemos que el proprio oficio del hombre es vivir alguna manera de vida, y que ésta es el ejercicio y obras del alma hechas conforme a razón, el oficio del buen varón será, por cierto, hacer estas cosas bien y honestamente. \textbf{Vemos, pues, que cada cosa conforme a su propria virtud alcanza su remate y perfección, lo cual si así es, el bien del hombre consiste, por cierto, en ejercitar el alma en hechos de virtud, y si hay muchos géneros de virtud, en el mejor y más perfecto, y esto hasta el fin de la vida.}\\

\section*{capítulo VIII}
pag 23


\end{multicols}

