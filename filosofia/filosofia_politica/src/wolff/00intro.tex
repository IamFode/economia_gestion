\chapter*{Introducción}
\addcontentsline{toc}{chapter}{Introducción}

Para aproximarse a la filosofía política, podemos tomar en cuenta dos preguntas:

\begin{enumerate}[\bfseries 1.]

    \item ¿Quién consigue qué?.

	Que tiene que ver con la distribución de bienes:
	\begin{itemize}
	    \item Materiales,
	    \item derechos,
	    \item libertades.
	\end{itemize}

	{\color{blue}
	Pregunta clave:
	\begin{itemize}
	    \item ¿En virtud de qué debería la gente poseer propiedades?
	    \item ¿Qué derechos y libertades debería tener?
	\end{itemize}
	}


    \item ¿Quién lo dice?.

	Que tiene que ver con la distribución de otro bien:
	\begin{itemize}
	    \item El poder político.
	\end{itemize}

	Locke definió el poder político como:
	\begin{center}
	    \textit{«el derecho de aprobar leyes y sancionarlas mediante la pena capital y, por consiguiente, mediante todas las demás penas inferiores».}
	\end{center}

	{\color{blue}
	Pregunta clave:
	\begin{itemize}
	    \item ¿Quién debería tener un poder así?
	\end{itemize}
	}

\end{enumerate}

En cuanto reflexionamos sobre estas cuestiones surgen los rompecabezas.

{\color{blue}
\begin{itemize}
    \item ¿Existe alguna buena razón por la que una persona deba poseer más propiedad que otra?
    \item ¿Existe algún límite justificado a mi libertad?
    \item ¿Qué relación debería haber entre el poder político y el éxito económico?
    \item ¿Debería existir una conexión entre la posesión de riqueza y la posesión de poder político?
    \item ¿Cómo justifica que alguien tenga poder político legítimo sobre mi y tenga derecho a forzarme a hacer ciertas cosas?
\end{itemize}
}

El filósofo político tiene cómo interés principal las siguentes cuestiones:

{\color{blue}
\begin{itemize}
    \item ¿Qué regla o principio debería gobernar la distribución de bienes?.
    \item ¿Qué distribución de la propiedad seria justa o equitativa?.
    \item ¿Qué derechos y libertades debería tener?.
    \item ¿Según qué criterios ideales, o normas, debería regirse la distribución de bienes en una sociedad?.
\end{itemize}
}

De esta forma, podemos sentirnos cómodos tanto con el alegato que hace el anarquista a favor de la autonomía del individuo, como con la defensa del autoritarista a favor del estado. 

Por consiguiente, una de \textbf{las tareas del filósofo político consiste en determinar el equilibrio correcto entre la autonomía y la autoridad, en otras palabras, determinar la distribución adecuada del poder político.}

La filosofía política es una disciplina \textit{normativa}. Es decir, pretenderá descubrir cómo deberían ser las cosas. Contrario, a un sociólogo o un historiador que su perspectiva es descriptiva. Ahora bien, investigar cómo son las cosas (descriptivo), ayuda a explicar cómo pueden ser.


Todos nosotros tenemos voz y voto para decidir cómo debería ser las cosas. Los que decidan quedarse al margen, se encontrarán con que otros han tomado las decisiones políticas por ellos, les agrade o no.

El punto de partida natural es el \textbf{poder político}, el derecho de ordenar.

{\color{blue}
\begin{itemize}
    \item ¿Por qué razón deben algunos tener el derecho de aprobar leyes que regulen el comportamiento de las demás personas? Supongamos que nadie tuviera ese derecho. ¿Cómo sería la vida entonces?.
    \item ¿qué sucedería en un «estado de naturaleza» sin gobierno? ¿Podría uno soportar vivir así? ¿Constituiría ello una mejora con respecto a la situación presente?.
\end{itemize}
}

Si aceptamos una vida bajo un gobierno. Entonces,

{\color{blue}
\begin{itemize}
    \item ¿Se sigue de ello que tenemos el deber moral de hacer lo que el estado nos ordene? ¿Existe algún otro argumento que conduzca a esa misma conclusión?.
    \item Si tenemos un estado, ¿cómo deberíamos organizarlo? ¿Debería ser democrático? ¿Qué significa decir que un estado es democrático?.
    \item ¿Cuánto poder debería tener el estado? O visto desde el otro lado: ¿de cuánta libertad debería gozar el ciudadano?
\end{itemize}
}

Ahora, bien para evitar una tiranía tenemos el derecho a la libertad. Pero al proporcionar esta libertad:

{\color{blue}
\begin{itemize}
    \item ¿Les otorgamos también la libertad de adquirir y disponer de propiedades del modo que crean conveniente? ¿O tal vez existen restricciones justificadas a la actividad económica en nombre de la libertad o la justicia?
\end{itemize}
}

Los primeros 5 temas versan sobre cuestiones de un interés permanente:

\begin{enumerate}[\bfseries 1.]
    \item El estado de naturaleza.
    \item El estado.
    \item La democracia.
    \item La libertad.
    \item La propiedad.
\end{enumerate}

