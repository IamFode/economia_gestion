\chapter{El estado de naturaleza}

Nos preguntamos:

\begin{center}
    ¿Cómo sería la vida en un estado natural, un mundo si gobierno?
\end{center}

Dado que no se podría obolír el estado, lo mejor que podemos hacer es realizar un experimento mental. Nos imaginamos que nadie posee el poder político y empezamos a averiguar como sería la vida en tales condiciones.

\begin{itemize}
    \item ¿Hubo alguna vez un estado de natural
    \item ¿es posible vivir en un estado de naturaleza? 
    \item ¿qué aspecto tendría el mundo?.
\end{itemize}

Jean-Jacques Rousseau (1712-1778) al igual que John Locke, pensaba que sí había ejemplos de personas viviendo en un estado natural, como por ejemplo la América del siglo XVII.

Comencemos suponiendo que los seres humanos podrían vivir en el mundo sin estado. ¿Qué aspecto tendría el mundo?.

\textbf{Hobbes, Leviatán, p 186:}

\begin{center}
    En una condición así [en el estado de naturaleza], no hay lugar para el trabajo, ya que el fruto del mismo se presenta como incierto; y, consecuentemente, no hay cultivo de la tierra; no hay navegación, y no hay uso de productos que podrían importarse por mar; no hay construcción de viviendas, ni de instrumentos para mover y transportar objetos que requieren la ayuda de una fuerza grande; no hay conocimiento en toda la faz de la tierra, no hay cómputo del tiempo; no hay artes; no hay letras; no hay sociedad. Y lo peor de todo, hay un constante miedo y un constante peligro de perecer con muerte violenta. Y la vida del hombre es solitaria, pobre, desagradable, brutal y corta.
\end{center}

\textbf{Leviatán, habla de los males de la guerra civil y la anarquía.} Donde, menciona que no hay nada peor que una vida sin protección del estado y, por consiguiente, es crucial que exista un gobierno fuerte que impida que caigamos en una guerra de todos contra todos. Ya que, nos conducirá a una situación de extremo conflicto.

¿por qué creyó Hobbes que el estado de naturaleza sería tan desesperado, una situación de guerra, una situación de constante temor y peligro de perecer en una muerte violenta?

Para Hobbes, la filosofía política se ocupa en primer lugar del estudio de la naturaleza humana. Para ello se tiene dos claves para comprenderla:
\begin{enumerate}
    \item Autoconocimiento de los seres humanos.
    \item Conocimiento de los principios generales de la física.
\end{enumerate}
Hobbes, como buen materialista, pensaba que para comprender la naturaleza humana uno tiene que comprender primero el "cuerpo" o la materia de la cual estamos totalmente compuestos, a través de la teoría de la conservación del movimiento. Y ver que cuando una cosa está en movimiento estará continuamente en movimiento a menos que alguna otra cosa la detenga. Por ejemplo, la sensación es una presión sobre un órgano.

Por otro lado, Hobbes decía que la llamada felicidad es la que nos conduce a la guerra en el estado de naturaleza y el temor a la muerte es lo que conducirá a crear un estado. ¿por qué creía Hobbes esto?

Porque el poder de un  hombre lo constituye los medios que tiene a la mano para obtener un bien futuro que se le aparenta como bueno, y la búsqueda de poder es por naturaleza competitiva ya que puede lograr tener felicidad.

La siguiente suposición que hace Hobbes es que los seres humanos somos por naturaleza iguales, por el hecho de que deberíamos respectar a los demás y tratarnos mutuamente con cuidado y preocupación, luego supone que existe un importante escasez de bienes y por último supone que existe incertidumbre. 


