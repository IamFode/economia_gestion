\chapter{El estado de naturaleza}

Nos preguntamos:

\begin{center}
    ¿Cómo sería la vida en un estado natural, un mundo si gobierno?
\end{center}

Dado que no se podría obolír el estado, lo mejor que podemos hacer es realizar un experimento mental. Nos imaginamos que nadie posee el poder político y empezamos a averiguar como sería la vida en tales condiciones.

\begin{itemize}
    \item ¿Hubo alguna vez un estado de natural
    \item ¿es posible vivir en un estado de naturaleza? 
    \item ¿qué aspecto tendría el mundo?.
\end{itemize}

Jean-Jacques Rousseau (1712-1778) al igual que John Locke, pensaba que sí había ejemplos de personas viviendo en un estado natural, como por ejemplo la América del siglo XVII.

Según Thomas Hobbes (Leviatán, p 186), no hay nada peor que una vida sin protección del estado y, por consiguiente, es crucial que exista un gobierno fuerte que impida que caigamos en una guerra de todos contra todos. Ya que, nos conducirá a una situación de extremo conflicto.

Para entender mejor, Hobbes pensaba que la filosofía política debía empezar por el principio, por el estudio de naturaleza humana. Para esto se tiene dos claves:

\begin{enumerate}
    \item Autoconoimiento.
    \item Conocimiento de los principios generales de la física.
\end{enumerate}

Hobbes, como buen materialista, pensaba que para comprender la naturaleza humana uno tiene que comprender primero el "cuerpo" o la materia de la cual estamos totalmente compuestos.

Dado que el movimiento es continuo a menos que alguna cosa la detenga, Hobbes pensaba que el humano se cansa y desea el reposo tan sólo cuando otro movimiento actúa sobre nosotros. Así, utilizó el principio de la conservación del movimiento para desarrollar una concepción materialista y mecanicista del ser humano.
