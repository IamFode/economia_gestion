\documentclass[11pt]{book}
\usepackage[T1]{fontenc}
\usepackage[utf8]{inputenc}
\usepackage[spanish]{babel}
\parindent = 0cm
\usepackage{amsmath}
\usepackage{amsfonts}
\usepackage{amssymb}
\usepackage{graphicx}
\usepackage{rotating}
\usepackage{enumerate}
\usepackage{geometry}
\usepackage{booktabs}
\geometry{hmargin={2.5cm,3cm},vmargin={3cm,3cm}}
\pagestyle{empty}
\usepackage[usenames,dvipsnames,svgnames,table]{xcolor}

\begin{document}
\author{Christian Paredes Aguilera}
\title{Modelo CANVAS}
\date{21/12/2018}
\maketitle
\let\cleardoublepage\clearpage
\part{Lienzo}
\let\cleardoublepage\clearpage
\chapter{Segmentos de Mercado}
\let\cleardoublepage\clearpage
\subsubsection{Preguntas}
\begin{itemize}
\item\textcolor{blue}{¿ Para quien creamos valor ?}
\item\textcolor{blue}{¿ Cuales son nuestros clientes mas importantes ?}
\end{itemize}
\vspace{0.5cm}
Podría ir orientado a:\\
\begin{itemize}
\item MERCADO DE MASAS\\
\item MERCADO SEGMENTADO\\ 

\item NICHO DE MERCADO\\
\item MERCADO DIVERSIFICADO\\
\item PLATAFORMAS MULTILATERALES (O MERCADOS MULTILATERALES
\end{itemize}
  
\part{Propuesta de valor}
\paragraph{definición de propuesta de valor}
Descripción de los beneficios que pueden esperar los clientes de tus productos y servicios.\\
\textbf{palabra clave: Crear valor}\\ 
Se debe solucionar o satisfacer una necesidad del cliente, como también tener ventajas antes los competidores
\section{Preguntas}
\begin{itemize}
\item\textcolor{blue}{¿ Que valor proporcionamos a nuestros clientes ?}
\item\textcolor{blue}{¿ Que problema de nuestros clientes ayudamos a solucionar ?}
\item\textcolor{blue}{¿ Que necesidades de nuestros clientes satisfacemos ?}
\item\textcolor{blue}{¿ Que paquetes de productos o servicios ofrecemos a cada segmento de mercado ?}
\end{itemize}
\begin{itemize}
\item NOVEDAD\\

\item MEJORA DEL RENDIMIENTO\\

\item PERSONALIZADO\\

\item EL TRABAJO, HECHO\\

\item DISEÑO\\

\item MARCA/STATUS\\

\item PRECIO\\

\item REDUCCIÓN DE COSTES\\

\item REDUCCIÓN DE RIESGO\\

\item ACCESIBILIDAD

\item COMODIDAD/UTILIDAD\\

\end{itemize}



\chapter{Lienzo}
Tiene dos lados 
\begin{enumerate}
\item Perfil del cliente (pag 10)
\item Mapa de valor(pag. 26 )
\end{enumerate}
Para luego conseguir el ENCAJE 
\subsubsection{Encaje}
\paragraph{Crear Valor}
El conjunto de beneficios de la propuesta de valor que diseñas para atraer a los clientes.
\paragraph{Observar a los clientes}
El conjunto de características del cliente que asumes, observas y verificas en el mercado.
\section{perfil del cliente}
\subsection{Trabajos del cliente}
Los trabajos describen las actividades que tus clientes intentan resolver en su vida laboral o personal. como por ejemplo:
\begin{itemize}
\item Tareas que intentan terminar.
\item Problemas que intentan solucionar.
\item Necesidades que intentan satisfacer.
\end{itemize}
Asegurase de adoptar la perspectiva del cliente cuando investigues los trabajos.
Distingue entre los tres tipos de rtabajos que el cliente quiere resolver y los trabajos de apoyo
\begin{enumerate}
\item \textbf{Trabajos funcionales}\\
Aquellos en los que tus clientes intentan realizar o terminar una tarea especifica, o solucionar un problema. Por ejemplo: \\
\begin{itemize}
\item Cortar el césped.
\item Intentar comer sano.
\item Redactar un informe.
\end{itemize}
\item \textbf{Trabajos sociales}\\
Aquellos en los que tus clientes quieren quedar bien, ganar poder o estatus. Estos trabajos describen cómo quieren que los perciban los demás. Por ejemplo:
\begin{itemize}
\item Ir a la moda. 
\item Ser considerado competente.
\end{itemize} 

\item \textbf{Trabajos personales/emocionales}\\
Aquellos en los que tus clientes buscan alcanzar un estado emocional específico. Por ejemplo:
\begin{itemize}
\item Sentirse bien.
\item Encontrar la paz mental.
\item Lograr sensaciones de seguridad en el puesto de trabajo.
\end{itemize}

\end{enumerate}
\begin{itemize}
\item \textbf{Trabajos de apoyo}\\
Los clientes también realizan trabajos de apoyo en el contexto relacionado con la adquisición y el consumo de valor. Surgen en tres papeles diferentes:
\begin{itemize}
\item COMPRADOR DE VALOR.\\
Trabajos relacionados con la compra de valor. Por ejemplo:
\begin{itemize}
\item Comparar ofertas.
\item Decidir que productos adquirir.
\item Esperar en una cola para pagar.
\item Completar un compra u optar por el envío de un producto o servicio.
\end{itemize}
\item COCREADOR DE VALOR.\\
Trabajos relacionados con la cocreación de valor con tu empresa. Por ejemplo:
\begin{itemize}
\item Publicar opiniones y comentarios de productos
\item Participar en el diseño de un producto o servicio.
\end{itemize}

\item TRANSFERIDOR DE VALOR.\\
Trabajos relacionados con el fin del ciclo de vida de una propuesta de valor. Por ejemplo:
\begin{itemize}
\item Cancelar una suscripción
\item Deshacerse de un producto transferido a terceros o revenderlo.
\end{itemize}
\end{itemize}
\end{itemize}
\paragraph{Importancia del trabajo}
Es importante reconocer que no todos los trabajos tienen la misma importancia para tu cliente

\subsection{Frustraciones del cliente}
Las  frustraciones del cliente es lo que molesta a tus clientes antes, durante y después de intentar resolver un trabajo o simplemente, lo que les impide resolverlo.\\
También describe los riesgos.\\
Trata de identificarlos tres tipos de frustraciones del cliente y lo severas que pueden ser.
\begin{enumerate}
\item \textbf{ Características, problemas y resultados no deseados.}\\
Las frustraciones son:
\begin{itemize}
\item \textit{Funcionales.}\\
Pueden ser que funcionen o no funcionen bien.
\item \textit{Sociales.}\\
Quedo mal haciendo esto.
\item \textit{Secundarias.}\\
Es un fastidio tener que ir a comprar a la tienda.
\item \textit{Sensaciones no deseadas}.\\
Es aburrido correr en el gimnasio o este diseño es feo.
\end{itemize}
\item \textbf{ Obstáculos }\\
Los elementos que impiden que los clientes empiecen un trabajo o que los hacen ir mas lentos (Me falta tiempo para hacer esto como es debido)
\item \textbf{ Riesgos (resultados potenciales no deseados) }\\
Lo que podría salir mal y tener importantes consecuencias negativas.
\paragraph{Intensidad de las frustraciones}
Para un cliente una frustración puede ser extrema o moderada, de modo similar a que los trabajos pueden ser importantes o insignificantes.
\subparagraph{Consejo: Concreta las frustraciones}
para distinguir claramente trabajos, frustraciones y alegrías, describelos de la manera mas concreta posible.
\subsubsection{Las siguiente lista de preguntas desencadenantes puede ayudarte a pensar en varias frustraciones potenciales del cliente}
\begin{itemize}
\item \textcolor {blue}{¿ Como definen tus clientes que algo es demasiado costoso ?, ¿ Lleva mucho tiempo hacerlo, cuesta demasiado dinero o exige esfuerzos considerables ?}
\item \textcolor {blue}{¿ Que provoca que tus clientes se sientan mal ?. ¿ Que molestias, fastidios o quebraderos de cabeza tienen ?}
\item \textcolor {blue}{¿ En qui fallan para tus clientes las actuales propuesta de valor ?, ¿ Que elementos les faltan ?, ¿ Hay cuestiones de rendimiento que les molesten o mencionan fallos ?}
\item \textcolor {blue}{¿ Cuales son los principales retos y dificultades con los que se encuentran los clientes ?, ¿ Entienden como funcionan las cosas, tienen dificultades haciendo algunos trabajos o se resisten a hacer algunos determinados por motivos específicos ?}
\item \textcolor {blue}{¿ Con que consecuencias sociales negativas se topan o cuales temen ?. ¿ Les asusta una pérdida de prestigio, poder, confianza o estatus ?}
\item \textcolor {blue}{¿ Que riesgo temen tus clientes ?, ¿ Los técnicos, sociales o financieros ?, ¿ Se preguntan que podría salir mal ?}
\item \textcolor {blue}{¿ Que les hace perder el sueño ?. ¿ Cuales son sus grandes preocupaciones ?}
\item \textcolor {blue}{¿ Que errores comunes comenten tus clientes ?, ¿ Utilizan una solución de manera equivocada ?}
\item \textcolor {blue}{¿ Cuales son las barreras que impiden que tus clientes adopten una propuesta de valor ?, ¿ Hay costes de inversión iniciales, una curva de aprendizaje pronunciada u otros obstáculos que impidan su adopción ?}
\end{itemize}
\end{enumerate}
\subsection{Alegrías del cliente}
Las alegrías describen los resultados y beneficios que quieren tus clientes. Algunas son:
\begin{itemize}
\item Necesarias.
\item Esperadas.
\item Deseadas.
\item Sorpresas.
\end{itemize}
Entre ellas se incluyen las alegrías:
\begin{itemize}
\item Sociales.
\item utilidad funcional.
\item las emociones positivas.
\item Ahorros en costes.
\end{itemize}
\emph{
Trata de identificar cuatro tipos de alegrías desde el punto de vista de los resultados y beneficios.}
\begin{enumerate}
\item\textbf{ Alegrías necesarias.}\\
Se trata de alegrías sin las cuales una solución no funcionaria.
Por ejemplo la expectativa de un celular es que podamos llamar con el.
\item \textbf{Alegrías esperadas.}\\
Son alegrías relativamente básicas que esperamos de una solución incluso cuando podría funcionar sin ellas. Desde que Apple lanzó el iphone esperamos que estén bien diseñadas y bonitas.
\item \textbf{Alegrías deseadas.}\\
Los que van mas allá de lo que esperamos de una solución, pero que nos encantaría tener si pudiéramos.
\item \textbf{Alegrías inesperadas.}\\
Aquellas que van mas allá de las experiencias y deseos de los clientes. Ni siquiera las mencionarían si les preguntaras. Innovación
\end{enumerate}
\paragraph{Relevancia de la alegría}
Una alegría puede resultar esencial o agradable para el cliente.
\subparagraph{Consejo: Concreta las alegrías}
Igual que con las frustraciones, es preferible describir las alegrías de la manera mas concreta posible para distinguir claramente entre trabajos, frustraciones y alegrías.
\subsubsection{Las siguiente lista de preguntas desencadenantes puede ayudarte a pensar en varias alegrías potenciales del cliente}
\begin{itemize}
\item \textcolor{blue}{ ¿ Qué ahorros harían felices a tus clientes ? ¿ Qué ahorros valorarían
desde el punto de vista del tiempo, dinero y esfuerzo ? }
\item \textcolor{blue}{ ¿ Qué niveles de calidad esperan? Y ¿ de cuáles quisieran más o menos ?}
\item \textcolor{blue}{ ¿ Cómo satisfacen las actuales propuestas de valor a los clientes? ¿ Con qué
características específicas disfrutan? ¿ Qué rendimiento y calidad esperan? }
\item \textcolor{blue}{ ¿ Qué les haría la vida más fácil a tus clientes? ¿ Podría haber una curva de
aprendizaje más plana, más servicios, o costes de propiedad más bajos?
 }
\item \textcolor{blue}{ ¿ Qué consecuencias sociales positivas desean tus clientes? ¿ Qué les hace
quedar bien? ¿ Qué aumenta su poder o estatus? }
\item \textcolor{blue}{ ¿ Qué buscan más los clientes? ¿ Buen diseño, garantías, más
características o que éstas sean más específicas?
 }
\item \textcolor{blue}{ ¿ Con qué sueñan los clientes? ¿ Qué aspiran conseguir? O ¿ qué
representaría un alivio para ellos? }
\item \textcolor{blue}{ ¿ Cómo miden tus clientes el éxito o el fracaso? ¿ Cómo calculan el
rendimiento o el coste? }
\item \textcolor{blue}{ ¿ Qué aumentaría la probabilidad de que tus clientes adopten
una propuesta de valor? ¿ Desean un coste más bajo, menos
inversión, menor riesgo o mejor calidad? }
\end{itemize}
\subsection{Perfil de un lector de libros de empresa}
se puede explorar realizando un esquema de un perfil basado en el aspecto que tu crees que puedan tener tus clientes potenciales. es un excelente punto de partida para preparar entrevistas y pruebas relacioandas con tus asunciones sobre los trabajos.\\
Las \textbf{Alegrías} son:
\begin{itemize}
\item Beneficios.
\item Resultados.
\item Características.
\end{itemize}
Que los clientes exigen o desean.
\textcolor{red}{Leer ejemplo pagina 19 libro 2}
\subsection{Clasificar trabajos, frustraciones y alegrías}
\begin{itemize}
\item Investiga que trabajos considera importantes o insignificantes la mayoría.
\item Averigua que frustraciones les parece extremas frente a las que simplemente consideran moderadas.
\item Descubre que alegrías consideras esenciales y cuáles agradables.
\end{itemize}
\begin{enumerate}
\item \textbf{importancia de la tarea}\\
Clasifica los trabajos según la importancia que tienen para los clientes.
\item \textbf{Intensidad de las frustraciones}\\
Clasifica las frustraciones en función de lo extremas que son a ojos del cliente.
\item \textbf{Relevancia de la alegría}\\
Clasifica las alegrías según lo esenciales que sean a ojos del cliente
\end{enumerate}
\textcolor{red}{Ver pagina 21}
\subsection{Ponte en lugar del cliente}
\paragraph{OBJETIVO}
Visualiza que les importa a tus clientes en un formato que se pueda compartir
\paragraph{RESULTADO}
Perfil conciso del cliente de una página.\\\\
\textcolor{green}{¿ Comprendes realmente sus trabajos, frustraciones y alegrías ?. Elabora un perfil del cliente}
\begin{enumerate}
\item \textbf{Selecciona el segmento de clientes}\\
Selecciona un segmento de clientes sobre el que quieras hacer el perfil.
\item \textbf{ Identifica los trabajos del clientes}\\
Pregunta a tus clientes que trabajos intentan acabar. Indicalos todos, anotalos cada uno de ellos en una nota autoadhesiva.
\item \textbf{Identifica las frustraciones del cliente}\\
\textcolor{blue}{¿ Que frustraciones tiene tus clientes ?}. Anota las que se te ocurra, incluyendo obstáculos y riesgos.
\item \textbf{Identifica las alegrías del cliente}\\
\textcolor{blue}{¿ Que resultados y beneficios quieren conseguir ?}. Escribe todas las alegrías que se te ocurran
\item \textbf{Prioriza trabajos, frustraciones y alegrías}\\
Ordena los trabajos, frustraciones y alegrías en columnas. Coloca arriba en cada uno los trabajos mas importantes, las frustraciones mas extremas y las alegrías esenciales, y abajo las frustraciones moderadas y las alegrías que estaría bien tener.
\end{enumerate} 
\subsection{Buenas prácticas para identificar trabajos, frustraciones y alegrías}
\begin{itemize}
\item ERRORES COMUNES.
\begin{enumerate}
\item Mezclar varios segmentos de clientes en un mismo perfil.
\item Mezclar trabajos y resultados.
\item Centrarse únicamente en los trabajos funcionales y olvidar los sociales y emocionales.
\item hacer listas de trabajos, frustraciones y alegrías con su propuesta de valor en mente.
\item Identificar muy pocos trabajos, frustraciones y alegrías.
\end{enumerate}
\item BUENAS PRÁCTICAS.
\begin{enumerate}
\item Haz un lienzo de la propuesta de valor para cada segmento de clientes.
\item Los trabajos son las tareas que los clientes intentan resolver, los problemas aquello que quieren solucionar o las necesidades que desean satisfacer, mientras que las alegrías son los resultados concretos que quieren conseguir o, en el caso de las frustraciones, eliminar.
\item A veces los trabajos sociales o emocionales son mas importantes que los trabajos funcionales. Puede ser quedar bien con los demás que solucionar un problema técnico.
\item Al elaborar un mapa del cliente deberás proceder como un antropólogo y olvidar lo que ofreces.\\
Ve mas allá de los trabajos, frustraciones y alegrías que pretendes o esperas abordar con tu propuesta de valor.
\item Un buen perfil del cliente esta lleno de autoadhesivas, ya que la mayoría tiene montón de frustraciones y espera o desea tener muchas alegrías .
\item Procura que las frustraciones y alegrías sean tangibles y concretas. Es decir anota cuantitativamente exacto lo que quiere tu cliente.
\end{enumerate}
\end{itemize}
\paragraph{Frustraciones frente a alegrías}
Es muy común adherir al perfil del cliente elementos opuesto de alegrías y frustraciones. por ejemplo Si uno de los trabajos que tiene que resolver es ganar mas dinero, puede que empieces añadiendo aumentar de salario en alegrías y reducción de salario en frustraciones.\\
 Hay un modo de hacerlo mejor:
\begin{itemize}
\item Averigua exactamente cuanto dinero mas espera ganar el cliente para considerar una alegría e investiga que, redacción supondría una frustraciones.
\item En frustraciones añade, añade las barreras que impiden o hacen difícil resolver un trabajo. En nuestro ejemplo la frustración podría ser Mi empresa no concede aumentos.
\item En frustraciones, añade los riesgos relacionados con no resolver el trabajo. En nuestro ejemplo la frustración podría ser: Quizá no pueda pagarles la matricula de la universidad a mis hijos.
\end{itemize}
\paragraph{Pregúntales varias veces Por que hasta que comprendas de verdad los trabajos que tiene que resolver tu cliente}
Debes tomar en cuenta que solo comprendes sus trabajos de manera superficial, por eso no te conformes hasta que llegues a comprender de verdad lo que hay detrás de los trabajos que realmente motivan a los clientes.
\section{Mapa de valor}
\subsection{Productos y servicios}
Se trata sencillamente de una lista de lo que ofreces. Es importante recordar que los productos y servicios no crean valor por si solos solo en relación con un segmento de clientes específicos y sus trabajos, frustraciones y alegrías.
\paragraph{Relevancia}
Es esencial admitir que para tus clientes no todos los productos y servicios tiene la misma relevancia. Algunos son esenciales para tu propuesta de valor, otros son simplemente agradables.
\subsection{Aliviadores de frustraciones}
Los aliviadores de frustraciones describen como de manera exacta tus productos y servicios alivian las frustraciones especificas de tus clientes.\\
Resumir de manera explicita como pretender eliminar o reducir los problemas que aquejan.\\
las buenas propuestas se centran en las frustraciones que extremas importan a los clientes. No hace falta aliviar todas las frustraciones si no las esenciales.\\
La siguiente lista de preguntas puede ayudar a pensar maneras distintas de como tus productos y servicios pueden aliviar las frustraciones de los clientes.\\\\
\textbf{Pregúntate: \textcolor{blue}{¿ Podrían tus productos y servicios.... ?}}
\begin{itemize}
\item \textcolor{blue}{¿ Generar ahorros ? Desde el punto de vista del tiempo, dinero o esfuerzo}
\item \textcolor{blue}{¿ Hacer que tus clientes se sientan mejor ? Eliminando frustraciones, molestias y otros elementos que les provoca dolores de cabeza}
\item \textcolor{blue}{¿ Arreglar soluciones de bajo rendimiento ? Introduciendo novedades, mejor rendimiento y calidad}
\item \textcolor{blue}{¿ Poner fin a las dificultades y retos con los que se encuentran tus clientes ? Haciendo las cosas mas fáciles o eliminando obstáculos}
\item \textcolor{blue}{¿ Borrar consecuencias sociales negativas a las que se enfrentan o temen ? Desde el punto de vista de la pérdida de prestigio, poder, confianza o estatus}
\item \textcolor{blue}{¿ Eliminar riesgos que les asustan ? De tipo financiero, social, técnico o cosas que podrían salir mal}
\item \textcolor{blue}{¿ Ayudar a tus clientes a dormir mejor ? Abordando cuestiones significativas, disminuyendo o eliminando preocupaciones}
\item \textcolor{blue}{¿ Limitar o erradicar los errores habituales que comenten ? Ayudándoles a usar una solución de manera adecuada}
\item \textcolor{blue}{¿ Eliminar barreras que hacen que tus clientes no adopten propuesta de valor ? Introduciendo costes de inversión iniciales mas bajos o eliminándoles con una curva de aprendizaje mas plana o suprimiendo otros obstáculos que impiden adopción}
\end{itemize}
\subsection{Creadores de alegrías}
Describen como tus productos y servicios crean alegrías para el cliente. Resumen de manera explícita como pretendes producir resultados y beneficios que tus clientes esperan, esperan o con lo que se sorprenderían entre las que se incluyen:
\begin{itemize}
\item Utilidad funcional.
\item Alegrías sociales.
\item Emociones positivas.
\item Ahorro de costes.
\end{itemize}

\paragraph{\textcolor{blue}{¿ Podrían tus productos y servicios.... ?}}
\begin{itemize}
\item \textcolor{blue}{¿ Generar ahorros que les interesen a tus clientes? Desde el
punto de vista del tiempo, dinero o esfuerzo}
\item \textcolor{blue}{¿ Producir resultados que esperan o que exceden sus
expectativas? Ofreciendo niveles de calidad o variando la
cantidad de algún elemento}
\item \textcolor{blue}{¿ Ofrecer un mejor rendimiento que las actuales propuestas de valor y dejar encantados
a tus clientes? En cuanto a características específicas, rendimiento o calidad.}
\item \textcolor{blue}{¿ Hacerles la vida o el trabajo más fácil? A través de una mejor usabilidad,
accesibilidad, más servicios o un coste de propiedad más bajo.
}
\item \textcolor{blue}{¿ Crear consecuencias sociales positivas? Haciéndoles quedar bien
o ayudándoles a obtener un aumento de poder o estatus.}
\item \textcolor{blue}{¿ Hacer algo específico que los clientes buscan? Desde el punto de vista del
buen diseño, las garantías o tener más características más específicas.}
\item \textcolor{blue}{¿ Cumplir un deseo con el que sueñan? Ayudándoles a conseguir
sus aspiraciones o librándoles de un apuro.}
\item \textcolor{blue}{¿ Producir resultados positivos que se correspondan con los criterios de éxito
y de fracaso que tienen tus clientes? Desde el punto de vista de un
rendimiento mejor o de costes más bajos.}
\item \textcolor{blue}{¿ Ayudar a que la adopción sea más fácil? Mediante un coste más bajo,
menos inversiones, menor riesgo, o mejor calidad, rendimiento o diseño}
\end{itemize}
Asegúrate de diferenciar entre esenciales y agradable.

\subsection{Planificando la propuesta de valor de diseñando la propuesta de valor}
Céntrate en aquellas frustraciones y alegrías que para el cliente marcarán la diferencia.

\textcolor{red}{Revisar paginas 34 y 34 libro 2}

\subsection{Realiza un mapa de como crean valor tus productos y servicios}
\paragraph{Objetivo}
Describe de manera explícita como crean valor tus productos y servicios.
\paragraph{Resultado}
Un mapa de una página de creación de valor.
\subsubsection{Instrucciones}
Esboza el mapa de valor de una de tus propuestas existentes. Utiliza una que tenga como objetivo el segmento de clientes del que hiciste el perfil de clientes.
\begin{enumerate}
\item \textbf{Haz una lista de productos y servicios.} \\
Haz una lista de todos los productos y servicios de la propuesta de valor que ya tienes.
\item \textbf{Resume los aliviadores de frustraciones.} \\
Resume cómo tus productos y servicios ayudan a los clientes en la actualidad a aliviar frustraciones eliminando resultados, obstáculos o riesgos no deseados.
\item \textbf{Resume los creadores de alegrías.}\\
Explica cómo tus productos y servicios crean en la actualidad resultados y beneficios esperados o deseados para los clientes. Utiliza una nota autoadhesiva para cada creador de alegrías
\item \textbf{Clasifica por orden de importancia.}\\
Clasifica productos y servicios, aliviadores de frustraciones y creadores de alegrías según lo esencial que sean para los clientes.
\end{enumerate}
\subsection{Buenas practicas para la creación del mapa de valor}
\subsubsection{Errores comunes}
\begin{itemize}
\item Hacer una lista de tus productos y servicios en lugar de hacerla solo de aquellos que tienen como objetivo un segmento especifico.
\item Añadir productos y servicios a los campos de aliviadores de frustraciones y creadores de alegrías.
\item Ofrecer aliviadores de frustraciones y creadores de alegrías que no tienen nada que con las frustraciones y alegrías del perfil del cliente
\item Intentar abordar de manera poco realista todas las frustraciones y alegrías del cliente.
\end{itemize}
\subsubsection{Buenas prácticas}
\begin{itemize}
\item Los productos y servicios crean valor solo en relación con un segmento de cliente específico. Hay una lista solo de los paquetes de productos y servicios que forman de manera conjunta una propuesta de valor para un segmento específico.
\item Los aliviadores de frustraciones y creadores de alegrías son explicaciones o características que explicitan la creación de valor de tus productos y servicios. Entre los ejemplos se incluyen: Ayuda a ahorrar tiempo y Bien diseñado.
\item Recuerda que los productos y servicios no crean valor en términos absolutos. Siempre es en comparación con los trabajos, frustraciones y alegrías del cliente.
\item Se consciente de que las buenas propuestas de valor son las que toman decisiones sobre qué trabajos, frustraciones y alegrías ocuparse y sobre cuáles renunciar. No hay ninguna que se encargue de todo. Si tu mapa de valor así lo indica, seguramente será porque no eres honesto con todos los trabajos, frustraciones y alegrías qe deberían estar en tu perfil de cliente.
\end{itemize}
\section{Encaje}
Al encaje se llega cuando los clientes e ilusionan con tu propuesta de valor. Esforzarse para conseguir el objetivo es la esencia del diseño de la propuesta de valor.
\paragraph{¿ Encaje ?}
\textcolor{red}{Revisar paginas 44 y 45 libro 2}
\subsection{Verifica tu encaje}
\paragraph{Objetivo}
Comprueba que te estas ocupando de lo que les importa a los clientes
\paragraph{Resultado}
Conexión entre productos y servicios y los trabajos, frustraciones y alegrías del cliente.
\begin{enumerate}
\item \textbf{Instrucciones}\\
Ve a buscar el mapa de la propuesta de valor y el perfil del segmento de cliente que completaste. Revisa uno por uno los aliviadores de frustraciones y los creadores de alegrías y comprueba si encaja con algún trabajo, frustración o alegría del cliente.
\item \textbf{Resultados}\\
Si un aliviador de frustraciones o un creador de alegrías no encaja en nada, puede que no este creando valor para el cliente.
\end{enumerate}
\subsection{Tres tipos de encaja}
Buscar el encaje es el proceso que consiste en diseñar propuestas de valor que cubren los trabajos, frustraciones y alegrías que realmente importan al cliente. El encaje entre lo que ofrece la empresa y lo que quiere el cliente es el primer requisito para lograr una propuesta de valor.
\subsubsection{Sobre el papel}
\paragraph{1. Encaje problema-solución} 
Tiene lugar cuando:
\begin{itemize}
\item Tienes pruebas de que a los clientes les importan determinados trabajos, frustraciones y alegrías.
\item Has diseñado una propuesta de valor que aborda esos trabajos, frustraciones y alegrías.
\end{itemize}
Aun no se tiene pruebas que al cliente le importa tu propuesta de valor.\\
Realiza diferentes prototipos de propuesta de valor, para elegir los que produzcan mayo encaje.
\subsubsection{En el Mercado}
\paragraph{2. Encaje producto-mercado}
Tiene lugar cuando:
\begin{itemize}
\item Tienes pruebas de que tus productos y servicios, los aliviadores de frustraciones y los creadores de alegrías realmente crean valor para el cliente y encuentran tracción en el mercado
\end{itemize}
Te darás cuenta que muchas de tus ideas no crean valor para el cliente, por el cual tendrás que diseñar propuesta nuevas.

\subsubsection{En el banco}
\paragraph{3. Encaje modelo de negocio}
Tiene lugar cuando:
\begin{itemize}
 \item Tienes pruebas de que tu propuesta de valor puede insertarse en un modelo de negocio rentable y escalable.
\end{itemize}
Una gran propuesta de valor sin un gran modelo de negocio, puede suponer un éxito por debajo de lo esperado, por esta razón tienen que ir de la mano para generar ingresos mayores a lo gastado.
\subsection{Encajes Múltiples}
Algunos modelos sólo funcionan con una combinación de varias propuestas de valor y segmentos de mercado. Para ello es necesario que cada propuesta de valor encaje con su segmento de cliente.
\subsubsection{Intermediarios}
Cuando un negocio vende a través de un intermediario, necesita complacer a dos clientes. Sin una propuesta de valor clara al intermediario puede que no llegue al cliente final con el mismo impacto.\\
Por esta razón se necesita dos propuestas de valor que encajen a l segmento de cliente.
\subsubsection{Plataformas}
Solo funcionan cuando interactúan dos o mas actores y extraen valor del mismo modelo de negocio. Por ejemplo no es el mismo cliente un cliente internacional que un local.
\subsection{Ir al cine (ejemplo)}
\textcolor{red}{revisar páginas 54 y 55 2º libro}\\
\textbf{Al preparar un perfil del cliente, tu objetivo es desvelar lo que realmente le motiva, más que describir sus características socioeconómicas.\\ Investiga lo que quieren conseguir, los motivos que hay detrás, sus objetivos y lo que les frena.}
\subsection{El mismo cliente diferentes contextos}
Las prioridades cambian dependiendo del contexto del cliente. Es fundamental que lo tengas en cuenta antes de pensar en una propuesta de valor.\\
Preguntas que realizar:
\begin{itemize}
\item \textcolor{blue}{¿ Cuando ?}
\item \textcolor{blue}{¿ Donde ?}
\item \textcolor{blue}{¿ Con quien ?}
\item \textcolor{blue}{¿ Limitaciones ?}
\end{itemize}
\textcolor{red}{Revisar pagina 57 2º libro}
\subsection{El mismo cliente soluciones distintas}
Esfuérzate en comprender lo que realmente les importa a tus clientes, investiga sus trabajos, frustraciones y alegrías mas allá de los que abordan directamente tu propuesta de valor, ya que la hipercompetitividad abunda en el mundo.\\
\textcolor{red}{Revisar ejemplo pagina 59}\\
\textcolor{red}{Revisar resumen paginas 60 y 61 2º libro}
\chapter{Diseñar}
\textbf{DANDO FORMA A LAS IDEAS}
Se trata de un ciclo continuo que consiste en:
\begin{itemize}
\item Diseñar.
\item Investigar sobre los clientes.
\item Volver a estructurar tus ideas.
\end{itemize}
\textbf{DIEZ CARACTERÍSTICAS DE LAS GRANDES PROPUESTAS DE VALOR}
\begin{enumerate}
\item Se afianzan en grandes modelos de negocio.
\item Se centran en los trabajos, frustraciones y alegrías que mas les importan a los clientes.
\item Se centran en trabajos no solucionados.
\item Tienen como objetivo pocos trabajos, frustraciones y alegrías, pero se centran en ellos extremadamente bien.
\item Van mas allá de los trabajos funcionales y abordan los emocionales y sociales.
\item Están en consonancia con el modo en que los clientes miden el éxito.
\item Se concentran en los trabajos, frustraciones y alegrías que tiene mucha gente o por los que pagarán mucho dinero.
\item Se diferencian de las competencias en los trabajos, frustraciones y alegrías que les importan a los clientes.
\item Superan a la competencia de manera significativa por lo menos en un ámbito.
\item Son difíciles de copiar.
\end{enumerate}
\section{Posibilidades de prototipos}
\subsection{¿ Que es el prototipo ?}
\textcolor{red}{Revisar paginas 76 y 77 de 2º libro}
\subsection{10 principios del prototipo}
\begin{enumerate}
\item \textbf{ Que sean visuales y tangibles }
\item \textbf{ Adopta una mente de principiante }\\ Hay prototipos de lo que no se puede hacer, explora con mentalidad abierta.
\item \textbf{ No te enamores de las primeras ideas, crea alternativas}
\item \textbf{ Siéntese cómodo en un estado liquido }\\ En la primera etapa del proceso no esta clara la dirección adecuada, no solidifiques las cosas demasiado rápido.
\item \textbf{ Empieza con baja fidelidad, itera y perfección }\\ Es difícil descartar prototipos pulidos, perfeccionalos cuando tengas mas conocimiento sobre que funcionara y que no.
\item \textbf{ Muestra, pronto tu trabajo. Busca la crítica }\\ Solifica feedback pronto.
\item \textbf{ Aprende mas rápido fracasando antes, repetidas veces y con poco dinero } \\ El miedo al fracaso frena a muchas personas explorar. Véncelo con una cultura de creación rápida.
\item \textbf{ Usa técnicas creativas}
\item \textbf{ Crea modelos SHREK }\\ Son prototipos extremos que seguramente no construirás. Aprovecha para fomentar el debate y el aprendizaje.
\item \textbf{ Haz un seguimiento de lo que aprendes, de las nuevas percepciones y de lo que progresas }
\end{enumerate}
\subsection{Haz visibles las ideas con dibujos en servilletas}
\paragraph{Objetivo}
Visualiza rápidamente ideas para propuesta de valor.
\paragraph{Resultado}
Prototipos en forma de dibujos en servilletas.
\subsubsection{Los mejores dibujos en servilletas}
\begin{itemize}
\item Contienen solo una idea central (después se puede fusionar ideas).
\item Explica de que va la idea, como funciona.
\item Simplifica las cosas para entenderlo de un vistazo.
\item se puede vender en un plazo de 10 a 30 segundos.
\end{itemize}
PASOS A SEGUIR
\begin{enumerate}
\item \textbf{ Brainstorming - 15-20 minutos }\\ Utiliza preguntas desencadenantes (pag. 15,17,31,33)o preguntas del tipo y si. aca la cantidad de preguntas es mejor que la calidad.
\item \textbf{Dibujar 12-15 minutos }\\ 
\item \textbf{ Vender 30 segundos }\\ Asegúrate que haya mucha diversidad.
\item \textbf{ Explotación }\\ Todas las ideas en servilletas se expondrán en la pared.
\item \textbf{ Puntocracia 10-15 minutos }\\ Se tiene que votar o elegir las ideas favoritas
\item \textbf{ Prototipo }\\ Se sigue esbozando el lienzo de propuesta de valor.
\end{enumerate}
\subsection{Crea rápidamente posibilidades con ad-libs}
\paragraph{Objetivo}
Da forma rápidamente a direcciones posibles de propuesta de valor.
\paragraph{Resultado}
Prototipos alternativos en forma de frases listas para vender. 
\\\\
\textbf{NUESTROS} productos y servicios \textbf{AYUDAN} al segmento del cliente \textbf{QUE QUIEREN } trabajos \textbf{PARA} el verbo que tu elijas reducir evitar \textbf{Y} el verbo que tu elijas aumentar reducir \textbf{A DIFERENCIA DE } la propuesta de valor de la competencia.\\
\textcolor{red}{Revisar paginas 82 y 83 libro 2º}
\subsection{Precisa las ideas con los lienzos de la propuesta de valor}
\paragraph{Objetivo}
Esboza explícitamente como crean valor para el cliente las diferentes ideas.
\paragraph{Resultado}
Prototipos alternativos en forma de lienzos de la propuesta de valor.
\\ NO tengas miedo de crear prototipos de direcciones radicales aunque sepas que es poco probable que las apliques.
\section{Puntos de partida}

\subsection{Genera ideas con limitaciones de diseño}
\paragraph{Objetivo}
Forzarte a pensar con una perspectiva diferente
\paragraph{Resultado}
Ideas que difieren de tus propuestas de valor y modelos de negocio habituales
\begin{itemize}
\item \textbf{ Serviciación }\\
Limitación: Pasar de vender una propuesta de valor basada en un producto a otra basada en un servicio que genera ingresos a partir de un modelo de suscripción.\\
hilti paso de vender herramientas a alquilarlas
\item \textbf{ Hojas y maquinilla de afeitar }\\
Limitación: Crear una propuesta de valor compuesta por un producto base y un consumible que genere ingresos recurrentes.\\
Nespresso Pasar de un negocio transaccional a un negocio de ingresos concurrentes basados en productos consumibles para su máquina de café.

\item \textbf{ Marcador de tendencias }\\
Limitación: Convertir (una innovación de) la tecnología a una tendecnia de moda\\
Swatch conquistó el mundo convirtiendo un reloj de plástico en tendencia mundial debido a que se podía fabricar a bajo costo
\item \textbf{ De bajo coste }\\
Limitación: Reducir el valor central de la propuesta de valor a sus características básicas, apuntar con precios bajos a un segmento de clientes desatendido, y vender el resto como una propuesta de valor adicional.
\item \textbf{ Plataforma }\\
Limitación: construir un modelo de plataforma que conecte varios actores con una propuesta de valor específica para cada uno.\\
Airbnb Permitió que hogares de todo el mundo fueran accesibles a viajeros al conectarlos con personas que quieren alquilar apartamentos a corto plazo.  
\end{itemize}
\textcolor{red}{revisar ejemplos paginas 90 y 91 2º libro}
\subsection{Aportar grandes ideas a la mesa con libros y revistas}
Utiliza libros y revistas de éxito para generar ideas frescas para propuestas de valor y negocios de modelo nuevos e innovadores.
\paragraph{Objetivo}
Ampliar horizontes y generar ideas frescas.
\paragraph{Resultado}
Ideas que incorporen temas relevantes y últimas tendencias.
\begin{enumerate}
\item \textbf{ Selecciona libros }\\
Coloca en la mesa variedad de libros y revistas que representen una tendencia un tema importante o una gran idea. Pide a los participantes del taller que elijan uno cada uno.
\item \textbf{ Hojea el material y extrae ideas }\\
Los participantes hojean y comentan las mejores ideas en notas antoadhesivas.
\item \textbf{ Comparte y discute }\\
Comparten y plasman sus ideas en una pizarra.
\item \textbf{ Posibilidad de brainstorming }\\
Cada grupo genera tres ideas nuevas de propuesta de valor basada en discusiones.
\item \textbf{ Vende }\\
Cada grupo comparte sus propuesta de valor alternativas con los otros grupos.

\end{enumerate}
 \textcolor{red}{Revisar consejos pagina 93 2º libro}
\subsection{Push frente a Pull}
\textbf{Push} indica que el diseño de tu propuesta de valor comienza en una tecnología o innovación que posees.\\
\textbf{pull} Empiezas con un trabajo, frustración o alegría manifiestos del cliente.
\subsubsection{Impulso a la tecnología (Push)}
Se trata de una solución en busca de un problema.\\
Explora propuestas de valor que se basen en tus inventos, innovaciones.
\subsubsection{Demanda del mercado (Pull)}
Un problema en busca de solución.
Aprende que tecnologías y otros recursos se necesitan para cada prototipo de propuesta de valor.
\subsection{Push: Tecnología a la búsqueda de trabajos, frustraciones y alegrías}
\paragraph{Objetivo}
Practica el enfoque basado en la tecnología sin riesgo.
\paragraph{Resultado}
Mejora de las habilidades.
\subsubsection{Empieza con la solución}
es decir empieza por el mapa de valor.
\begin{enumerate}
\item \textbf{ Diseña }\\
Diseña una propuesta de valor basada en la tecnología. 
\item \textbf{ Idea }\\
Propón una idea para una propuesta de valor.
\item \textbf{ Segmenta }\\
Selecciona un segmento de clientes al que podría interesarle esta propuesta de valor y que estaría dispuesta a pagar por ella.
\item \textbf{ Prepara el perfil }\\
Haz un esbozo del perfil del cliente. Estima los trabajos frustraciones y alegrías que habrá que abordar
\item \textbf{ Esboza }\\
Perfecciona la propuesta de valor esbozando como terminará con las frustraciones y creará alegrías.
\item \textbf{ Evalúa }\\
Evalúa el encaje entre el perfil del cliente y la propuesta de valor diseñada.
\end{enumerate}
\textcolor{red}{Revisar ejemplo y consejos paginas 96 y 97 2º libro}
\subsection{Pull: Identifica trabajos de gran valor}
Los creadores de grandes propuestas de valor dominan el arte de centrarse en los trabajos, frustraciones y alegrías que importan.\\
\textcolor{blue}{¿ Como sabrás en cuáles de ellos centrarte ?}\\
Identifica trabajos de gran valor preguntándote si son:
\begin{itemize}
\item Importantes.
\item Tangible.
\item Lucrativos.
\item y si aun no se han resuelto.
\end{itemize}
\subsubsection{Los trabajos de gran valor se caracterizan por frustraciones y alegrías que son:}
\begin{itemize}
\item \textbf{ Importantes }\\
Cuando el éxito o el fracaso del cliente al resolver el trabajo supone alegrías esenciales o frustraciones extremas respectivamente.
\begin{enumerate}
\item \textcolor{blue}{¿ Fracasar en el trabajo supone frustraciones extremas ?}
\item \textcolor{blue}{¿ Fracasar en el trabajo supone perderse alegrías esenciales ?}
\end{enumerate}
\item \textbf{ Tangibles }\\
Cuando las frustraciones o alegrías relacionados con el trabajo se pueden experimentar de inmediato o a menudo.
\begin{enumerate}
\item \textcolor{blue}{¿ Sientes la frustraciones ?}
\item \textcolor{blue}{¿ Sientes la alegría ?}
\end{enumerate}
\item \textbf{ No solucionadas }\\
Cuando las propuestas de valor no ayudan a aliviar frustraciones ni a crear alegrías deseadas satisfactoriamente.
\begin{enumerate}
\item \textcolor{blue}{¿ Hay frustraciones no resueltas ?}
\item \textcolor{blue}{¿ Hay alegrías que no se han conseguido ?}
\end{enumerate}
\item \textbf{ Lucrativas }\\
Cuando hay mucha gente con el trabajo que tiene frustraciones y alegrías relacionadas o un número pequeño de clientes está dispuesto a pagar por el servicio premium.
\begin{enumerate}
\item \textcolor{blue}{¿ Hay muchas personas con este trabajo, frustraciones, o alegrías ?}
\item \textcolor{blue}{¿ Hay pocos dispuestos a pagar mucho ?}
\end{enumerate}
\item \textbf{ Trabajos de gran valor }\\
Concéntrate en trabajos de gran valor, en las frustraciones y alegrías relacionadas
\end{itemize}
\textcolor{red}{Revisar paginas 89 y 99 2º libro}
\subsection{Pull: Selección del trabajo}
\paragraph{Objetivo}
Identifica trabajos de gran valor para el cliente en los que podrías centrar.
\paragraph{Resultado}
Clasificación de trabajos del cliente desde tu perspectiva.
\paragraph{Consejos}
\begin{itemize}
\item Completa este ejercicio obteniendo conocimiento del cliente (pagina 106). 
\item Experimentos que generan datos (pagina 216).
\end{itemize}
\textcolor{red}{Revisar cuadros paginas 100 y 101 2º libro}
\subsection{Seis maneras de innovar a partir del perfil del cliente}
\begin{enumerate}
\item \textbf{ ¿ Puedes abordar mas trabajos ? }\\
Aborda un grupo de trabajos mas completo que incluya los relacionados y los auxiliares.
\item \textbf{ ¿ Puedes pasar a un trabajo mas importante ? }\\
Ayuda a los clientes a resolver un trabajo distinto de aquellos en los que se centran la mayoria de las propuestas de valor.
\item \textbf{ ¿ Puedes ir mas allá de los trabajos funcionales ? }\\
Mirar mas allá de los trabajos funcionales y crear valor nuevo cumpliendo trabajos sociales y emocionales mas importantes.
\item \textbf{ ¿ Puedes ayudar a mas clientes a resolver un trabajo ? }\\
Ayuda a mas personas a resolver un trabajo que de otra manera era demasiado complejo o demasiado caro.
\item \textbf{ ¿ Resolver un trabajo cada vez mejor ? }\\
Ayuda a los clientes a resolver mejor un trabajo añadiendo una serie de pequeñas mejoras en una propuesta de valor existente.
\item \textbf{ ¿ Ayudar a un cliente a resolver mejor un trabajo de manera radical ? }\\
Es lo que ocurre en los mercados nuevos, cuando una propuesta de valor nueva supera el modo anterior de ayudar a un cliente a resolver un trabajo.
\end{enumerate}
\section{Comprender a los clientes}
\subsection{Seis técnicas para conocer mejor al cliente}
\subsubsection{El detective de datos}
Realizar el trabajo mediante documentación. Toma datos externos a tu sector industrial y estudia las analogías, elementos opuestos y los pintos en común.\\ ver pagina 108

\subsubsection{El periodista}
Un meto fácil es hablar con los clientes. ojo Los clientes no siempre saben lo que quieren y su comportamiento real es distinto. ver pagina 110
\subsubsection{El antropólogo}
Observa a los clientes potenciales en el mundo real para entender como se comportan. Estudia en que trabajos se centran y como los resuelven, anota que frustraciones les molesta y que alegrías quieren lograr. Ojo es difícil conocer lo que piensan los clientes sobre ideas nuevas. ver pagina 114
\subsubsection{El imitador}
Se tu cliente y utiliza tus productos y servicios, Extrae experiencias como un cliente insatisfecho.
\subsubsection{El cocreador}
Integra a los clientes en el proceso de creación de valor y aprende de ellos. Trabaja a su lado para explorar y desarrollar ideas nuevas.

\subsubsection{El científico}
Invita a los clientes entrar en un experimento y aprende de sus resultados.
\subsection{El detective de datos: Empieza con información previa}
\subsubsection{Google Trends}
Compara tres términos de búsqueda que representen tres tendencias diferentes relacionadas con tu idea.
\subsubsection{Planificador de palabras clave}
Descubre que es popular entre tus clientes encontrando los cinco términos de búsqueda mas buscados relacionados con tu idea ¿ Con que frecuencia se buscan ?
\subsubsection{Datos del censo del gobierno, del banco mundial, del FMI, entre otros}
Identifica los datos del gobierno que sean relevantes para tu idea y que tienes a tu alcance gracias a la web.
\subsubsection{Informes de investigaciones externos}
Identifica tres informes de investigación que estén fácilmente disponibles y que puedan servirte como punto de partida para preparar tu propia investigación del clinete y dela propuesta de valor.
\subsubsection{Analítica de las redes sociales}
Las empresas y marcas existentes deberían:
\begin{itemize}
\item Identificar en las redes sociales a las personas, mas influyentes relacionadas con su marca.
\item Localizar las 10 cosas positivas y negativas que se dicen con mas frecuencia sobre ellos en las redes sociales.
\end{itemize}
\subsubsection{gestión de la relación con el cliente}
Haz una lista de tres preguntas, quejas y solicitudes que recibes mas en tu interacción diaria con los clientes.
\subsubsection{Realiza un seguimiento de los clientes en tu página web}
\begin{itemize}
\item Haz una lista de las tres maneras que tienen los clientes de llegar a tu sitio web (busqueda, referencias)
\item Encuentra los destinos mas y menos populares en tu página web.
\end{itemize}
\subsubsection{Minería de datos}
La empresa existente debería extraer sus datos para:
\begin{itemize}
\item Identificar patrones que podrían ser útiles para su idea nueva.
\end{itemize}
\subsection{El periodista: Entrevista a tus clientes}
\paragraph{Objetivo}
Conocer mejor al cliente.
\paragraph{Resultado}
Primeros perfiles del cliente ligeramente validados.
\subsubsection{Entrevista}
\begin{enumerate}
\item \textbf{Crea un perfil del cliente}\\
Haz un bosquejo de los trabajos, frustraciones y alegrías que crees que caracterizan al cliente que tienes como objetivo y clasificalos por orden de importancia.
\item \textbf{Elabora un guión para la entrevista}\\
Pregúntate que quieres saber, obtén alas preguntas a partir del perfil del cliente. pregunta por los trabajos, frustraciones y alegrías mas importantes.
\item \textbf{ Realiza la entrevista }\\
Llévala a cabo siguiendo las normas básicas para realizar entrevistas pagina 112 y 113 2º libro
\item \textbf{ Plasma la información }\\
En un perfil del cleinte vacio haz un esquema de los trabajos, frustraciones y alegrías que has aprendido con la entrevista. Asegúrate de pasmar lo que hayas averiguado sobre los modelos de negocio.
\item \textbf{ Revisa la entrevista }\\
Evalúa si debes revisar las preguntas basándose en lo que has aprendido.
\item \textbf{ Busca patrones }\\
\begin{itemize}
\item \textcolor{blue}{¿ Encuentras trabajos, frustraciones y alegrías similares ?}
\item \textcolor{blue}{¿ Que destaca ?}
\item \textcolor{blue}{¿ Que es similar y que difiere entre los entrevistados ?}
\item \textcolor{blue}{¿ Por que son parecidos o distintos ?}
\item \textcolor{blue}{¿ Puedes detectar contextos específicos (recurrentes) que influyan en los trabajos, frustraciones y alegrías ?}
\end{itemize}
\item \textbf{ Sintetiza }\\
Elabora un perfil sintetizado para cada segmento de clientes que surja de todas tus entrevistas. Escribe tus ideas mas importantes en notas autoadhesivas.
\end{enumerate}
\subsection{Reglas básicas para entrevistar}
\begin{enumerate}
\item \textbf{ Adopta una mente de principiante }\\
Escucha con atención y evita las interpretaciones. 
Explora sobre todo los trabajos, frustraciones y alegrías inesperados.
\item \textbf{ Escucha mas de lo que hablas }\\
Tu objetivo es escuchar y aprender, no informar, impresionar o convencer al cliente de nada.
\item \textbf{ Busca hechos, no opiniones }\\
NO pregunte: ¿ Le gustaría...?\\
Pregunta: ¿ Cuando fue la última vez que... ?
\item \textbf{ Pregunta por que para llegar a las motivaciones reales }\\
Pregunta ¿ Por que necesita hacer... ?\\
pregunta ¿ Por que es importante para usted ?\\
Pregunta ¿ Por que es una frustración ?

\item \textbf{ El objetivo de las entrevistas para concer a los clientes no es vender, sino aprender }\\
NO preguntes: ¿ Compraría nuestra solución?\\
Sino: ¿ Cual es su criterio de decisión cuando realiza una compra ?
\item \textbf{ No menciones soluciones, es demasiado pronto }\\
NO expliques nuestra solución hace...\\
Pregunta: ¿ Cuales son los elementos mas importantes con los que tiene problemas ?

\item \textbf{ Haz un seguimiento }\\
Pide permiso para guardar la información de contacto de tu entrevistado para volver a hacerle mas preguntas o para probar los prototipos.
\item \textbf{ Al final deja siempre la puerta abierta }\\
Pregunta: ¿ Con quien mas debería hablar ?
\end{enumerate}
\subsection{El antropólogo: Sumérgete en el mundo del cliente}
\subsubsection{B2C de empresa a consumidor: vive con la familia}
Quédate unos días en casa de uno de los clientes potenciales y convive con su familia. Participa en las rutinas diarias. Descubre lo que motiva a esa persona.
\subsubsection{B2B (de empresa a empresa): Trabaja/consulta en compañía}
Pasa un tiempo trabajando con un cliente potencial en un compromiso de consultoría, por ejemplo, observa ¿ Que le quita el sueño a esa persona ?
\subsubsection{¿B2B/B2C?}
¿ Como podrías adentrarte en la vida de tu cliente potencial ? Se creativo. Ve mas allá de los límites habituales.
\subsubsection{B2C: Observar el comportamiento de compra}
Ve a una tienda donde compren tus clientes potenciales y observa a la gente durante diez horas mira ¿ Puedes detectar algún patrón ?
\subsubsection{B2C: Haz de sombra de tu cliente por un día}
Conviértete en la sombra de tu cliente potencial y síguele durante un día. Anota todos los trabajos, frustraciones y alegrías que observes. Haz un seguimiento de su tiempo. Sintetiza. Aprende. 
\subsection{Identifica patrones en la investigación del cliente}
\paragraph{Objetivo}
Precisar a tu cliente.
\paragraph{Resultado}
Perfil sintetizado del cliente.
\subsubsection{Patrones}
Analiza tus datos e intenta detectar patrones una vez hayas logrado reunir bastante información sobre tu cliente.
Busca clientes con trabajos, frustraciones y alegrías, similares y elabora perfiles separados.
\begin{enumerate}
\item \textbf{ Muestra }\\
Muestra en una pared todos los perfiles del cliente resultado de tu investigación.
\item \textbf{ Agrupa y segmenta }\\
Agrupa perfiles del cliente similares en uno o mas segmentos independientes si logras identificar patrones en sus trabajos, frustraciones y alegrías.
\item \textbf{ Sintetiza }\\
Sintetiza los perfiles de cada segmento en un único perfil principal identifica trabajos frustraciones y alegrías mas comunes y utiliza etiquetas separadas para describir en los perfiles principales.
\item \textbf{ Diseña }\\
Empieza a crear tus primeros prototipos de propuesta de valor. 
\end{enumerate}
\subsection{Encuentra a tu primer evangelista}
Son los primeros clientes que se arriesgan a utilizar el producto o servicio.
\begin{enumerate}
\item \textbf{ Tiene un problema o una necesidad }\\
En otras palabras, hay que resolver un trabajo.
\item \textbf{ Es consciente de que tiene un problema }\\
El cliente comprende que hay un problema o trabajo.
\item \textbf{ Busca de manrea activa una solución }\\
El cliente está buscando una solución y tiene un calendario para encontrarlo.
\item \textbf{ Ha encontrado una solución con varias piezas sueltas }\\
El trabajo es tan importante, que el clinete ha tenido que buscar de manera apresurada una solución provisional.
\item \textbf{ tiene o puede contar con un presupuesto }\\
El cliente se ha comprometido a un presupuesto o puede contar con una rápidamente para adquirir una solución.
\end{enumerate}
\section{Tomar decisiones}
\subsection{10 preguntas para evaluar tu propuesta de valor}
\paragraph{Objetivo}
Descubrir potencial para mejorar tu propuesta de valor. 
\paragraph{Resultado}
Evaluación de la propuesta de valor.
\subsubsection{Preguntas}
\begin{enumerate}
\item \textcolor{blue}{¿ Está insertada en un gran modelo de negocio ?}
\item \textcolor{blue}{¿ Se centra en los trabajos mas importantes, frustraciones mas extremas y alegrías mas esenciales ?}
\item \textcolor{blue}{¿ Se centra en trabajos no solucionados, frustraciones no resueltas y alegrías no obtenidas ?}
\item \textcolor{blue}{¿ Se concentra sólo en aliviadores de frustraciones y creadores de alegrías pero lo hace extremadamente bien ?}
\item \textcolor{blue}{¿ Aborda a la vez trabajos funcionales emocionales y sociales ?}
\item \textcolor{blue}{¿ Se alinea con el modelo en el que os clientes miden el éxito ?}
\item \textcolor{blue}{¿ Se concentra en los trabajos, frustraciones y alegrías que tienen un gran número de clientes o por los que unos pocos estarían dispuestos a pagar mucho dinero ?}
\item \textcolor{blue}{¿ Se diferencia de la competencia de manera significativa ?}
\item \textcolor{blue}{¿ Supera de manera sustancial a la competencia por lo menos en un ámbito ?}
\item \textcolor{blue}{¿ Es difícil de copiar ?}
\end{enumerate}
\subsection{Simula la voz del cliente}
\paragraph{Objetivo}
Hacer una prueba de estrés en la sala de reuniones.
\paragraph{Resultado}
Propuesta de valor mas sólida antes de validarla en el mercado.\\
\textcolor{red}{Revisar paginas 124 y 125 2º libro}
\subsection{Comprende el contexto}
Las propuestas de valor y los modelos siempre se diseñan en un contexto. Reduce el zoom de los modelos para identificar el entorno en el que estás diseñando y tomando decisiones sobre los prototipos que quieres realizar.
\subsubsection{Entorno}
\begin{itemize}
\item \textbf{ Fuerzas de la industria }\\
Actores clave en tu espacio, como competidores, actores de la cadena de valor, proveedores de tecnología y mas
\item \textbf{ Fuerzas macroeconómicas }\\
Macrotendencias, como las condiciones globales del mercado, acceso a recursos, precios de los productos y demás.
\item \textbf{ Tendencias clave }\\
Tendencias clave que dan forma a tu espacio, como las innovaciones en tecnología, las limitaciones en las normativas, tendencias sociales.
\item \textbf{ Fuerzas del mercado }\\
Cuestiones clave del cliente en tu espacio, como puede ser los segmentos en crecimiento, costes de sustitución, trabajos frustraciones y alegrías cambiantes.
\end{itemize}
\textcolor{red}{Revisar pagina 127 2º libro}
\subsection{Diseñando la propuesta de valor frente a los competidores}
Evalúa con las de la competencia analizándola en un lienzo estratégico. \\
\textcolor{red}{Revisar paginas 128 y 129 2º libro}
\subsection{Compara tu propuesta de valor con la de la competencia}
\paragraph{Objetivo}
Comprender como valorar tu rendimiento en comparación con el de los demás
\paragraph{Resultado}
Comparación visual con la competencia.
\subsubsection{instrucciones}
\begin{enumerate}
\item \textbf{ Selecciona una propuesta de valor }\\
Selecciona (el prototipo de) la propuesta de valor que quieres comparar.
\item \textbf{ Selecciona los factores de competencia }\\
En el eje de las abscisas excoje los aliviadores de frustraciones y los creadores de alegrías que quieras comparar con la competencia.
\item \textbf{ Puntúa tu propuesta de valor }\\
En el eje de las ordenadas que represente el rendimiento de tu propuesta de valor, añade una escala que vaya de bajo a alto o de 0 a 10. 
\item \textbf{ Añade propuestas de valor de la competencia }\\
Añade en el eje x aliviadores de frustraciones y creadores de alegrías de la propuesta de valor de la competencia.
\item \textbf{ Puntúa las propuestas de valor de la competencia  }\\
Marca como se desenvuelven las propuestas de valor de la competencia, igual que has hecho con la tuya.
\item \textbf{ Analiza tu zona óptima }\\
Analiza las curvas y descubre oportunidades. Pregúntate si te diferencias de los competidores con tu propuesta de valor.
\end{enumerate}
\subsection{Evita el asesinato cognitivo para obtener mejor feedback}
\subsubsection{Buenas prácticas en presentaciones}
% Table generated by Excel2LaTeX from sheet 'Hoja1'
\begin{table}[htbp]
  \centering
  
    \begin{tabular}{ll}
    \multicolumn{1}{c}{SI} & \multicolumn{1}{c}{NO} \\
    Sencilla & Compleja  \\
    Tangible & Abstracta \\
    Presentar sólo lo importante & Presentar todo lo que sabes \\
    Centrada en el cliente & Centrada en las características \\
    Un elemento de información después de otro & Toda la información de golpe \\
    En el soporte adecuado & Sin soporte visual \\
    Línea argumental & Flujo de información aleatorio \\
    \end{tabular}%
  \label{tab:addlabel}%
\end{table}%

\subsubsection{Pasos a seguir}
\begin{enumerate}
\item Empieza con un lienzo vacío. con una breve introducción al mismo.
\item Empieza tu presentación en el punto que tenga mas sentido. Puedes empezar con los productos o los trabajos.
\item Al explicar ve añadiendo las notas autoadhesivas una detrás de otra.
\end{enumerate}
\subsection{Perfecciona el arte de la critica}
\subsubsection{Distingue entre tres tipos de feedback}
\begin{itemize}
\item Opinión.
\item Experiencia.
\item Datos (del mercado).

\end{itemize}

\textcolor{red}{Revisar paginas 134 y 135 recomendado 2º libro}
\subsection{Recopila feedback eficaz con los sombreros para pensar de De Bono}
\paragraph{Objetivo}
Recopilar feedback de manera efectiva y evitar discusiones largas
\paragraph{Resultado}
Comprender lo positivo o negativo de las ideas y como se pueden mejorar.
\subsection{Sombreros de De Bono}
\begin{enumerate}
\item \textbf{Vender}
El equipo presenta su idea y propuesta de valor o su lienzo de modelo de negocio
\item \textbf{Sombrero blanco}
Los miembros del público hacen preguntas para esclarecer y comprender la idea a fondo.
\item \textbf{Sombrero negro}
Los participantes escriben en una nota autoadhesiva el motivo por el cual es mala idea.\\
El facilitador va poniendo en la pizarra el feedback que ha recogido mientras los participantes leen en voz alta.
\item \textbf{Sombrero amarillo}
Los participantes escriben en una nota autoadhesiva el motivo por el cual es buena la idea.\\
El facilitador va poniendo en la pizarra el feedback que ha recopilado y los participantes lo leen en voz alta.
\item \textbf{Sombrero verde}
Se inicia el debate. Los participantes aportan sugerencias con respecto a como desarrollar las ideas que se han presentado.
\item \textbf{Ajuste}
El equipo que presenta ahusta la idea teniendo en cuenta el feedback de los sobreros blanco, negro, amarillo y verde.

\end{enumerate}
\subsection{Vota visualmente con la puntocracia}
\paragraph{Objetivo}
Visualizar las preferencias de un grupo y evitar discusiones largas
\paragraph{Resultado}
Selección rápida de ideas
\subsection{Puntocracia}
\begin{enumerate}
\item \textbf{ Galería de ideas }\\
Las ideas o lienzos se exponen en un muro como una galería de opciones.
\item \textbf{ Pegatinas }\\
Se reparte el mismo número de pegatinas (10, por ejemplo) a cada participante del taller, y cada una de ellas cuenta como un voto.
\item \textbf{ Criterios }\\
Se defienden los criterios de vtación. Por ejemplo se pide a los participantes que coloquen una pegatina en sus ideas favoritas
\item \textbf{ Votar }\\
Los participantes pueden poner todas sus pegatinas en una idea o distribuirlas entre varias distintas.
\item \textbf{ Contar }\\
Se hace un recuento de pegatinas y se destacan las ideas preferidas.
\end{enumerate}
\subsection{Define criterios y selecciona prototipos}
\paragraph{Objetivo}
Seleccionar entre una serie de alternativas.
\paragraph{Resultado}
Clasificación de prototipos.
\subsection{Criterios}
\begin{enumerate}
\item \textbf{ Brainstorming de criterios }\\
Busca todos los criterios posibles para evaluar el atractivo de tus prototipos.\\
\textcolor{red}{Revisar página 140 2º libro}

\item \textbf{ Selecciona los criterios }\\
Selecciona los criterios mas importantes para tu equipo y organización.
\item \textbf{ Puntúa los prototipos (de a, bajo a 10 ,alto) }\\
Puntúa cada idea con los criterios que hayan elegido.
\item \textbf{ Desarrolla el prototipo y explora el mercado }\\
Desarrolla tu prototipo y pruébalo en el mercado para saber si realmente tiene potencial.
\end{enumerate}
\section{ Encontrar el modelo de negocio adecuado}
\subsection{Crea valor para tu cliente y para el negocio}
Esta sección muestra que acertar con el modelo de negocio y la propuesta de valor es un proveso que hay que repetir una y otra vez hasta dar en el clavo.\\
\begin{itemize}
\item \textit{PARA CREAR VALOR PARA TU NEGOCIO DEBES CREAR VALOR PARA TU CLIENTE}
\item \textit{PARA CREAR VALOR PARA TU CLIENTE DE MANERA SOSTENIBLE, DEBES CREAR VALOR PARA TU NEGOCIO}
\end{itemize}
\textcolor{red}{Revisar modelo pagina 145 2º libro}
\subsection{Azuri (Eight 19): Convertir la tecnología solar en negocio viable}
\begin{enumerate}
\item \textbf{ Idea inicial }\\
Una oportunidad.
\item \textbf{ Observar }\\
La barrera de costes.
\item \textbf{ Diseño }\\
¿ y si ?
\item \textbf{ Iteración 2 }\\
Idea para un modelo de negocio
\item \textbf{ Observa }\\
La barrera de no tener banco.
\item \textbf{ Diseño }\\
Solución de baja tecnología.
\item \textbf{ Iteración 3 }\\
Idea para el modelo de negocio Azuri. 
\end{enumerate}
\textcolor{red}{Revisar ejemplo detenidamente paginas 146 al 151 2º libro}
\subsection{De la propuesta de valor al modelo de negocio}
\paragraph{Objetivo}
Practicar sin riesgo la conexión entre propuesta de valor y modelo de negocio.
\paragraph{Resultado}
Mejora de las habilidades.
\subsubsection{Parte A: Diseña el modelo de negocio completo}
\subsubsection{A1: Primer plano}
Crea un prototipo de un modelo de ingresos. Selecciona canales de distribución y define las relaciones que se podrían adoptar con el cliente.
\subsubsection{A2: Segundo plano}
Añade los recursos clave, actividades clave y socios necesarios para que el modelo funcione y utilízalo para calcular la estructura de costes.
\subsubsection{A3: Evaluación}
Evalúa tu prototipo y detecta posibles debilidades del modelo de negocio.
\subsubsection{Parte B: Revisar la propuesta de valor}
Evalúa las debilidades de tu primer prototipo de modelo de negocio. Pregúntate como podrías mejorar o cambiar tu propuesta de valor inicial.
\subsubsection{B1: ¿ Propuesta de valor nueva?}
\textcolor{blue}{ ¿ Podría haber otra propuesta de valor completamente diferente para la misma tecnología ? }
\textcolor{red}{Ver paginas 106 y 216 2º libro}
\subsubsection{B2: ¿ Segmento nuevo ?}
\textcolor{blue}{¿ mantendrás el mismo segmento de clientes o pasarás a uno totalmente distinto, quizça mas grande ?}
\subsubsection{B3: ¿ Pulir o eliminar tu perfil ?}
\textcolor{blue}{¿ Podrías pulir tu perfil de cliente o necesitas describir uno totalmente nuevo porque has cambiando tu segmento de clientes ?}
\subsubsection{B4: ¿ Cambiar o eliminar tus beneficios ?}
\textcolor{blue}{¿ Necesitas cambiar o eliminar tus beneficios que creaba tu propuesta de valor porque el perfil del cliente ha cambiado ?}
\subsubsection{B5: ¿ Has conseguido el encaje ?}
\textcolor{blue}{¿ Has conseguido el encaje entre tu nuevo perfil de cliente y tu propuesta de valor que abas de diseñar ?}
\textcolor{red}{Revisar  $"$encaje$"$ pagina 40 2º libro}
\subsection{Pruebas de estrés con cifras: Una ilustración medicotécnica}
\emph{Una propuesta de valor sin un modelo de negocio financieramente sólido no te permitirá llegar muy lejos.}\\\\
Juega con distintos modelos de negocio y suposiciones financieras para encontrar el mejor. \\
\textcolor{red}{Revisar exhaustivamente paginas 154 y 155 2º libro}
\subsection{Siete preguntas para evaluar tu diseño del modelo de negocio}
\paragraph{Objetivo}
Descubrir potencial para mejorar tu modelo de negocio.
\paragraph{Resultado}
Evaluación del modelo de negocio.
\subsubsection{Preguntas}
\begin{enumerate}
\item \textbf{ Cambiar costes }\\
\textcolor{blue}{ ¿ Como de fácil o de difícil lo tienen los clientes para cambiarse a otra empresa ? }
% Table generated by Excel2LaTeX from sheet 'Hoja1'
\begin{table}[htbp]
  \centering

    \begin{tabular}{p{11.855em}}
    Tengo a mis clientes atados durante varios años \\
    \multicolumn{1}{c}{10} \\
    \multicolumn{1}{c}{9} \\
    \multicolumn{1}{c}{8} \\
    \multicolumn{1}{c}{7} \\
    \multicolumn{1}{c}{6} \\
    \multicolumn{1}{c}{5} \\
    \multicolumn{1}{c}{4} \\
    \multicolumn{1}{c}{3} \\
    \multicolumn{1}{c}{2} \\
    \multicolumn{1}{c}{1} \\
    Nada impide que mis clientes me abandonen \\
    \end{tabular}%
  \label{tab:addlabel}%
\end{table}%

\item \textbf{ Ingresos recurrentes }\\
\textcolor{blue}{¿ Cada venta es un esfuerzo nuevo o dará como resultado casi garantizado nuevos ingresos y compras ?}
% Table generated by Excel2LaTeX from sheet 'Hoja1'
\begin{table}[htbp]
  \centering

    \begin{tabular}{p{10em}}
    El 100 \% de mis ventas conduce automáticamente a ingresos recurrentes \\
    \multicolumn{1}{c}{10} \\
    \multicolumn{1}{c}{9} \\
    \multicolumn{1}{c}{8} \\
    \multicolumn{1}{c}{7} \\
    \multicolumn{1}{c}{6} \\
    \multicolumn{1}{c}{5} \\
    \multicolumn{1}{c}{4} \\
    \multicolumn{1}{c}{3} \\
    \multicolumn{1}{c}{2} \\
    \multicolumn{1}{c}{1} \\
    El 100 \% de mis ventas son transaccionales \\
    \end{tabular}%
  \label{tab:addlabel}%
\end{table}%

\item \textbf{ Ganancias frente a gastos }\\
\textcolor{blue}{ ¿ Percibes ingresos antes de incurrir en gastos ? }
% Table generated by Excel2LaTeX from sheet 'Hoja1'
\begin{table}[htbp]
  \centering

    \begin{tabular}{p{12.285em}}
    Percibo el 100 \% de mis ingresos antes de incurrir en los gastos de los productos vendidos \\
    \multicolumn{1}{c}{10} \\
    \multicolumn{1}{c}{9} \\
    \multicolumn{1}{c}{8} \\
    \multicolumn{1}{c}{7} \\
    \multicolumn{1}{c}{6} \\
    \multicolumn{1}{c}{5} \\
    \multicolumn{1}{c}{4} \\
    \multicolumn{1}{c}{3} \\
    \multicolumn{1}{c}{2} \\
    \multicolumn{1}{c}{1} \\
    Incurro en el 100 \% de los gastos de los productos vendidos antes de percibir ingresos \\
    \end{tabular}%
  \label{tab:addlabel}%
\end{table}%

\item \textbf{ Estructura de costes para cambiar las reglas del juego }\\
\textcolor{blue}{¿ Tu estructura de costes es notablemente diferente y mejor que la de tus competidores ?}

% Table generated by Excel2LaTeX from sheet 'Hoja1'
\begin{table}[htbp]
  \centering

    \begin{tabular}{p{11.5em}}
    Mi estructura de costes es como mínimo un 30 \% mas baja que la de mis competidores \\
    \multicolumn{1}{c}{10} \\
    \multicolumn{1}{c}{9} \\
    \multicolumn{1}{c}{8} \\
    \multicolumn{1}{c}{7} \\
    \multicolumn{1}{c}{6} \\
    \multicolumn{1}{c}{5} \\
    \multicolumn{1}{c}{4} \\
    \multicolumn{1}{c}{3} \\
    \multicolumn{1}{c}{2} \\
    \multicolumn{1}{c}{1} \\
    Mi estructura de costes es como mínimo un 30 \% mas alta que la de mis competidores \\
    \end{tabular}%
  \label{tab:addlabel}%
\end{table}%

\item \textbf{ Los otros hacen el trabajo }\\
\textcolor{blue}{¿ Tu modelo de negocio permite que los clientes o terceros creen valor gratis por ti ? }

% Table generated by Excel2LaTeX from sheet 'Hoja1'
\begin{table}[htbp]
  \centering

    \begin{tabular}{p{11em}}
    Todo el valor que se crea en mi modelo de negocio se crea gratis por terceros \\
    \multicolumn{1}{c}{10} \\
    \multicolumn{1}{c}{9} \\
    \multicolumn{1}{c}{8} \\
    \multicolumn{1}{c}{7} \\
    \multicolumn{1}{c}{6} \\
    \multicolumn{1}{c}{5} \\
    \multicolumn{1}{c}{4} \\
    \multicolumn{1}{c}{3} \\
    \multicolumn{1}{c}{2} \\
    \multicolumn{1}{c}{1} \\
    Incurro en gastos en todo el valor que se crea en mi modelo de negocio \\
    \end{tabular}%
  \label{tab:addlabel}%
\end{table}%


\item \textbf{ Escalabilidad }\\
\textcolor{blue}{¿ Que facilidad tienes de crecer sin toparte con obstáculos (infraestructura, atención al cliente, contratar personal) ?}
% Table generated by Excel2LaTeX from sheet 'Hoja1'
\begin{table}[htbp]
  \centering

    \begin{tabular}{p{12.855em}}
    Mi modelo de negocio no tiene límites para crecer \\
    \multicolumn{1}{c}{10} \\
    \multicolumn{1}{c}{9} \\
    \multicolumn{1}{c}{8} \\
    \multicolumn{1}{c}{7} \\
    \multicolumn{1}{c}{6} \\
    \multicolumn{1}{c}{5} \\
    \multicolumn{1}{c}{4} \\
    \multicolumn{1}{c}{3} \\
    \multicolumn{1}{c}{2} \\
    \multicolumn{1}{c}{1} \\
    Para cercer con mi modelo de negocio se requieren recursos y esfuerzos considerables \\
    \end{tabular}%
  \label{tab:addlabel}%
\end{table}%

\item \textbf{ Protección frente a la competencia }\\
\textcolor{blue}{¿ Como te protege de la competencia tu modelo de negocio ?}
% Table generated by Excel2LaTeX from sheet 'Hoja1'
\begin{table}[htbp]
  \centering

    \begin{tabular}{p{12.43em}}
    Mi modelo de negocio ofrece barreras significativas difíciles de superar \\
    \multicolumn{1}{c}{10} \\
    \multicolumn{1}{c}{9} \\
    \multicolumn{1}{c}{8} \\
    \multicolumn{1}{c}{7} \\
    \multicolumn{1}{c}{6} \\
    \multicolumn{1}{c}{5} \\
    \multicolumn{1}{c}{4} \\
    \multicolumn{1}{c}{3} \\
    \multicolumn{1}{c}{2} \\
    \multicolumn{1}{c}{1} \\
    Mi modelo de negocio no tiene barreras, soy vulnerable a la competencia \\
    \end{tabular}%
  \label{tab:addlabel}%
\end{table}%

\end{enumerate}
\textcolor{red}{Revisar ejemplos paginas 156 y 157}
\section{Diseñar en empresas establecidas}
\subsubsection{Adoptar la actitud adecuada para inventar o mejorar}
\subsubsection{Inventa}
\paragraph{Objetivo}
Diseña propuesta de valor nuevas independientemente de las limitaciones potenciales generadas por las propuestas de valor y los modelos de negocio actuales.
\paragraph{Ayuda a}
\begin{itemize}
\item hacer una propuesta de valor proactiva por el futuro.
\item Enfrentarse a una crisis.
\item Surgimiento de tecnología, normativas y otros elementos que cambien las reglas del juego.
\item Respuesta a una propuesta de valor disruptiva de un competidor.
\end{itemize}
\paragraph{Objetivos financieros}
Crecimiento de los ingresos anuales del 50 % como mínimo.
\paragraph{Riesgo e incertidumbre}
Altos
\paragraph{Conocimiento del cliente}
Bajo, potencialmente inexistente.
\paragraph{Modelo de negocio}
Requiere adaptarse a cambios radicales.
\paragraph{Actitud hacia el fracaso}
Parte de aprendizaje y proceso de iteración.
\paragraph{Mentalidad}
Abierta a explorar nuevas posibilidades.
\paragraph{Enfoque del diseño}
Cambio radical de la propuesta de valor y el modelo de negocio.
\paragraph{Actividades principales}
\begin{itemize}
\item Investigar.
\item Probar.
\item Evaluar.
\end{itemize}

\subsection{Mejora}
\paragraph{Objetivo}
Mejorar tus propuesta de valor actuales sin cambiarles por completo y sin que afecten al modelo o modelo de negocio sobre el que se apoyan.
\paragraph{Ayuda a}
\begin{itemize}
\item Renovar productos y servicios anticuados.
\item Asegurar o mantener el encaje.
\item Mejorar el potencial de beneficios o la estructura de costes.
\item Mantener el crecimiento.
\item Abordar quejas de los clientes.
\end{itemize}
\paragraph{Objetivos financieros}
Crecimiento de los ingresos anuales del 0 al 15 % o superior.
\paragraph{Riesgo e incertidumbre}
Bajos.
\paragraph{Conocimiento del cliente}
Alto.
\paragraph{Modelo de negocio}
Poco cambio.
\paragraph{Actitud hacia el fracaso}
No entra en las opciones.
\paragraph{Mentalidad}
Centrada en mejorar aspectos.
\paragraph{Enfoque del diseño}
Cambios y ajustes progresivos a la propuesta de valor existente.
\paragraph{Actividades principales}
Pulir, planear y ejecutar.

\subsection{Termino medio: Ampliar}
El objetivo es buscar propuestas de valor nuevas que amplíen el modelo de negocio en el que se basan sin tener que modificar demasiados aspectos.

\subsection{El libro de empresa del futuro} \textcolor{red}{Ver detalladamente ejemplo de las paginas 162 y 163 2º libro}
\subsection{Reinvéntate y pasa de ofrecer a productos a ofrecer servicios}
\textcolor{red}{Ver detalladamente ejemplo de las paginas 164 y 165 2º libro}
\subsection{El ambiente perfecto para un taller}
\subsubsection{¿ Quien debería participar ?}
Invita a personas de orígenes distintos, sobre todo cuando sabes que supondrá, un impacto considerable. Su aprobación es fundamental.
\subsubsection{¿ Cual debería ser el formato ?}
Como norma general, en las primeras fases del diseño de la propuesta de valor, cuantos mas puntos de vista, mejor. Con diez o mas participantes, puedes explorar varias alternativas en paralelo. En las fases posteriores tener menos participantes suele ser mejor.
\subsubsection{¿ como puede usarse el espacio como instrumento ?}
Los espacios suelen ser un instrumento infravalorado para crear talleres que den resultados excepcionales, elige un lugar amplio, inspirador y poco habitual.
\subsubsection{¿ Que herramientas y materiales se necesitan ?}
Prepara una zona de autoservicio con pósters de lienzo, notas autoadhesivas, papel, cinta adhesiva, rotuladores y otras herramientas para que los participantes tengan lo que necesitan.
\textcolor{blue}{ Revisar pagina 167 2º libro }
\subsection{Organiza tu taller}
Un buen taller produce resultados tangibles y prácticos.
\subsubsection{ Principios de diseño para un buen taller}
\begin{itemize}
\item Crea un programa con una línea clara que muestre a los participantes como emergerán las propuesta de valor y modelos de negocio en una nueva versión mejorada.
\item Conduce a los participantes por un recorrido de varios pasos centrándote en una tarea o módulo por vez.
\item Evita la cháchara y favorece las interacciones estructuradas con herramientas como los lienzos o procesos como los sombreros para pensar.
\item Alterna entre el trabajo en grupos pequeños (4 a 6 personas)y las cesiones plenarias para las presentaciones y la puesta en común.
\item Gestiona el tiempo de manera estricta para cada módulo, sobre todo para la creación de prototipos. utiliza un temporizador visible para toso los participantes.
\item Diseña el programa como una serie de iteraciones para la misma propuesta de valor o modelo de negocio. Diseña, critica, repite y vuelve a repetir.
\item Evita las actividades lentas después de comer.
\end{itemize}
\subsubsection{Pasos a seguir}
\subsubsection{Antes del taller}
Haz los deberes y recopila información del cliente (Página 106)
\subsubsection{Posibilidades de prototipos}
\begin{itemize}
\item Preguntas desencadenantes. páginas 15, 17, 31, 33.
\item Elaborar perfil del cliente. páginas 22.
\item Hacer un mapa de la propuesta de valor. paginas 36.
\item Dibujos en servilletas. Páginas 80.
\item Ad libs. Páginas 82.
\item Precisa las ideas con los lienzos de la propuesta de valor. Páginas 84.
\item Limitaciones. Páginas 90.
\item Ideas nuevas con libros. Páginas 92.
\item ejercicios Push/Pull. Páginas 94.
\item Seis maneras de innovar. Páginas 102. 
\end{itemize}
\subsubsection{Tomar decisiones}
\begin{itemize}
\item Clasifica trabajos, frustraciones y alegrías. Páginas 20.
\item Comprueba tu encaje. Páginas 94.
\item Selección de trabajo. Páginas 100.
\item 10 preguntas. Páginas 122.
\item Voz del cliente. Páginas 124.
\item Evalúa el entorno. Páginas 126. 
\end{itemize}
\subsubsection{Continuo proceso de repetición del modelo de negocio}
\begin{itemize}
\item Ir y venir constante. Páginas 152.
\item Estimación de cifras. Páginas 154.
\item Siente preguntas del modelo de negocio. Páginas 156. 
\end{itemize}
\subsubsection{Preparando las pruebas}
\begin{itemize}
\item Extraer hipótesis. Páginas 200.
\item Priorizar hipótesis. Páginas 202.
\item Probar el diseño. Páginas 204.
\item Elegir una mezcla de experimentos. Páginas 216.
\item Prueba el mapa de carretera. Páginas 242-245.
\end{itemize}
\subsubsection{Después del taller.}
Empieza a probar tus propuestas de valor y modelos de negocio en el mundo real. Páginas 172.

\chapter{Probar}
\subsection{Empieza a experimentar para reducir el riesgo}
Cuando empiezas a explorar ideas nuevas, sueles encontrarte en un estado de máxima incertidumbre. es mejor que que hagas pruebas con  experimentos baratos, prototipos y pilotos. 
\subsection{Los planes de negocio frente a los procesos de experimentación}
Antes el paso en cualquier aventura empresarial era redactar un plan de negocio. Ahora sabemos mas, Los planes de negocio son documentos estupendos que funcionan en un entorno conocido con la certeza suficiente. Desgraciadamente, en las nuevas aventuras empresariales suelen haber un alto nivel de incertidumbre. Por ello, probar las ideas de manera sistemática para descubrir que funciona y que no es un enfoque mucho mejor que escribir un plan. Hasta se podría afirmar que los planes maximizan el riego. Su naturaleza sin imperfecciones crea la ilusión de que con una buena ejecución poco puede salir mal. \\
Se debe:
\begin{itemize}
\item Experimentar.
\item Aprender.
\item Adaptarse.
\end{itemize}
Para gestionar este cambio e ir reduciendo el riesgo y la incertidumbre paulatinamente. Este proceso se llama desarrollo de clientes y lean startup.

% Table generated by Excel2LaTeX from sheet 'Hoja1'
\begin{table}[htbp]
  \centering

    \begin{tabular}{rcl}
    \multicolumn{1}{c}{\textbf{PLANES DE NEGOCIOS}} & \multicolumn{1}{p{7.715em}}{Aplicando a nuevas aventuras empresariales} & \multicolumn{1}{c}{\textbf{EXPERIMENTACIÓN}} \\
    Nostros sabemos & \textbf{Actitud} & Nuestros clientes y socios saben \\
    Plan de negocio & \textbf{Herramientas} & Lienzos de la PV y del modelo de negocio \\
    Planificación & \textbf{Proceso} & Desarrollo de clientes y lean startup \\
    En la oficina & \textbf{Donde} & Fuera de la oficina \\
    Ejecución de un plan & \textbf{Se centra en } & Experimentación \\
    Hechos históricos de éxitos pasados & \textbf{Base de la decisión} & Hechos y resultados de los experimentos \\
    No se trata adecuadamente & \textbf{Riesgo} & Minimizado mediante aprendizaje \\
    Se evita & \textbf{Fracaso} & Aceptado como modo de aprender y mejorar \\
    Enmascarada bajo un plan detallado & \textbf{Incertidumbre} & Reconocida y reducida mediante experimentos \\
    Hojas de cálculo y documentos & \textbf{Detalle} & Depende de los n de datos de los experimentos \\
    Asunciones & \textbf{Números} & Basados en los datos \\
    \end{tabular}%
  \label{tab:addlabel}%
\end{table}%
\vspace{5cm}
\subsection{Diez principios de las pruebas}
Aplica estos diez principios cuando empieces a probar tus ideas para propuestas de valor con una serie de experimentos.
\begin{enumerate}
\item \textbf{ Asume que los datos superan las opiniones }\\
Los datos del mercado superan sea lo que sea que piense tu jefe, tus inversores, tú o cualquier otro.
\item \textbf{ Aprende mas rápido y reduce el riesgo aceptando el fracaso }\\
Probar ideas incluye el fracaso, no obstante, fracasar por poco dinero y en poco tiempo contribuye a aprender algo que reduce el riesgo.
\item \textbf{ Haz pruebas pronto y perfecciona después }\\
Recopila información con experimentos baratos y en su fase inicial, antes de haber pensado cada detalle o descrito tus ideas profusamente
\item \textbf{ Experimentos diferente a realidad }\\
Recuerda que los experimentos son una lente a través de la cual intentas comprender la realidad. Son un buen indicador, pero son distintos a la realidad.
\item \textbf{ Equilibra aprendizaje y visión }\\
Integra los resultados de las pruebas sin darle la espalda a tu visión.
\item \textbf{ Identifica aquello que se carga las ideas }\\
Empieza probando las asunciones mas importantes: las que podrían hacer explotar tu idea. 
\item \textbf{ Primero comprende a los clientes }\\
Haz pruebas de los trabajos, frustraciones y alegrías del cliente antes de probar lo que podrías ofrecerles.
\item \textbf{ Ahz que se pueda medir }\\
Las buenas pruebas contribuyen al aprendizaje que se puede medir y te proporciona información que se puede aplicar.
\item \textbf{ Acepta que no toso los hechos son iguales }\\
Puede que los entrevistados te digan una cosa y luego otra. Considera el grado de fiabilidad de tus datos.
\item \textbf{ Prueba el doble de veces las decisiones irreversibles }\\
Asegúrate de que las decisiones que tienen un impacto irreversible estén especialmente bien fundamentados. 
\end{enumerate}
\subsection{Presentando el proceso de desarrollo de clientes}
El desarrollo de clientes es un proceso de cuatro pasos inventado por Steve Blank.
La premisa básica es que en la oficina no hay hechos, así que debes probar con tus clientes y las partes interesadas.\\
Utilizaremos el proceso de desarrollo de clientes para probar  asunciones que se esconden detrás de los lienzos de la propuesta de valor y del modelo de negocio.
\begin{itemize}
\item \textbf{ Descubrimiento del cliente }\\
Sal a la calle a descubrir los trabajos, frustraciones y alegrías de tu cliente, Investiga lo que podrías ofrecerles para acabar con sus frustraciones y crear alegrías.
\item \textbf{ Validación del cliente }\\
Haz experimentos para probar si los clientes valoran cómo tus productos y servicios pretenden mitigar frustraciones y crear alegrías
\item \textbf{ Creación de clientes }\\
Empieza a generar demanda en el usuario final. Lleva a tus clientes a tus canales de ventas y empieza a escalar el negocio.
\item \textbf{ Creación de empresa }\\
Haz la transición de una organización temporal diseñada para buscar y experimentar hacia una estructura centrada en ejecutar un modelo validado.
\end{itemize}
\subsubsection{Buscar frente a ejecutar}
El objetivo de fase de búsqueda es experimentar y aprender que propuestas de valor se pueden vender y que modelos de negocio podrían funcionar. Tus lienzos cambiaran de manera radical y se irán ajustando constantemente. Hasta no haber validado tus ideas no pasarás al modo y escala de ejecución.
\textcolor{red}{El manual del emprendedor, la guía paso a paso para crear una gran empresa, Blanks, Dorf B}
\subsection{Integra los principios del Lean Startup}
La idea consiste en eliminar la incertidumbre y el estancamiento del desarrollo del producto creado, haciendo pruebas y aprendiendo continuamente en un proceso iterativo.
\begin{enumerate}
\item \textbf{ Generar una hipótesis }\\
Empieza por los lienzos de la propuesta de valor y del modelo de negocio a definir las hipótesis criticas que hay detrás de tus ideas para poder diseñar los experimentos adecuados.
\item \textbf{ Diseñar/construir }\\
Diseña a construye un objeto específicamente concebido para probar tus hipótesis, obtener información y aprender. Podría ser un prototipo conceptual, un experimento, o simplemente un prototipo básico (PMV) de los productos o servicios que quieres ofrecer. 
\item \textbf{ Aprender }\\
Analiza el rendimiento del objeto, compáralo con tus hipótesis iniciales y extrae conclusiones. Pregunta que pensabas que pasaría. Describe lo que a pasado realmente. Después resume lo que cambiarías y como lo harás. 
\item \textbf{ Medir }\\
Mide el rendimiento del objeto que has diseñado o construido.
\end{enumerate}
\subsection{Aplica el ciclo construir-medir-aprender}
\subsubsection{Prototipos conceptuales}
Diseña prototipos conceptuales rápidos para dar forma a tus ideas. 
\begin{itemize}
\item \textbf{ Diseña/construye }\\
Lienzos del modelo de negocio o de la propuesta de valor para dar forma a tus ideas a lo largo del proceso.
\item \textbf{ Mide }\\
Rendimiento del prototipo conceptual, el encaje entre el perfil del cliente y el mapa de valor, cifras aproximadas, evaluación del diseño con las siete preguntas del modelo de negocio.
\item \textbf{ Aprende }\\
Si debes adaptar tus prototipos conceptuales y por que el rendimiento económico asumido de tu modelo de negocio, El encaje asumido, Que hipótesis debes probar.
\end{itemize}

\subsubsection{Hipótesis}
Diseña y construye experimentos para probar las hipótesis que deben ser ciertas para que tu idea tenga éxito. Empieza con las hipótesis mas críticas que podrían acabar con tu idea.
\begin{itemize}
\item \textbf{ Diseña/construye }\\
Entrevistas, observación y experimentos para probar las asunciones iniciales de tu PV y modelo de negocio derivadas de los prototipos conceptuales.
\item \textbf{ Mide }\\
Lo que pasa realmente en tus experimentos comparado con lo que creías que pasaría.
\item \textbf{ Aprende }\\
Si es necesario cambiar alguno de los componentes básicos del modelo de negocio o del lienzo de la propuesta de valor y por que.
\end{itemize}

\subsubsection{Productos y servicios}
Construye productos mínimos viables, para probar tus propuestas de valor. Son prototipos con un conjunto mínimo de características diseñados específicamente para aprender, mas que para vender.
\begin{itemize}
\item \textbf{ Diseña/construye }\\
Productos mínimos viables con los beneficios y características que quieras probar
\item \textbf{ Mide }\\
Si tus productos y servicios realmente alivian frustraciones y crean alegrías para los clientes.
\item \textbf{ Aprende }\\
Si debes cambiar los productos de tu propuesta de valor y por que. Que aliviadores de frustraciones y creadores de alegrías funcionan y cuáles no.
\end{itemize}

\section{Que probar}
\subsection{Probando el círculo}
Prueba que trabajos, frustraciones y alegrías le importan mas a un cliente realizando experimentos que generen datos que vayan mas allá de tu estudio inicial del cliente. Solo después de esto deberás comenzar con la propuesta de valor. Así no pierdes tiempo con productos y servicios que no les interesa.\\ 
APORTA DATOS QUE MUESTREN LO QUE LES IMPORTA A LOS CLIENTES (EL CIRCULO) ANTES DE CENTRARSE EN COMO AYUDARLES (EL CUADRADO)
\subsubsection{Empieza con los trabajos, frustraciones y alegrías}
Debemos confirmar con datos que nuestros esbozos de perfiles, nuestra investigación inicial, nuestras observaciones y conclusiones de las entrevistas eran correctas.\\
\textcolor{blue}{¿ Tienes datos que muestren...}
\begin{itemize}
\item \textcolor{blue}{ que alegrías importan mas a los clientes? }
\item \textcolor{blue}{ cuales son las mas esenciales? }
\item \textcolor{blue}{ que frustraciones importan mas a los clientes? }
\item \textcolor{blue}{ cuales son las mas extremas? }
\item \textcolor{blue}{ que trabajos importan mas a los clientes? }
\item \textcolor{blue}{ cuales son las mas importantes? }
\end{itemize}
\textcolor{red}{ Revisar página 191 2º libro }
\subsection{Probando el cuadrado}
Prueba si los clientes pretenden ayudarles y cuánto les importa. Diseña experimentos que muestren que tus productos y servicios acaban con frustraciones y crean alegrías que importan a los clientes.\\
APORTA DATOS QUE DEMUESTREN QUE TUS CLIENTES LES IMPORTA COMO TUS PRODUCTOS Y SERVICIOS ACABAN CON LAS FRUSTRACIONES Y CREAN ALEGRÍAS.
\subsubsection{El arte de probar propuesta de valor}
Probar cuánto les importa tu propuesta de valor a tus clientes es un arte porque el objetivo es hacerlo de la manera mas barata y rápida posible sin implementar la propuesta de valor en su totalidad.\\
Debes probar el gusto de tu cliente por tus productos y servicios con los aliviadores de frustraciones y servicios con los aliviadores de frustraciones y creadores de alegrías uno por uno, diseñando experimentos que sean medibles, aporten conclusiones y que te permitan aprender y mejorar. (pagina 214). Procura en centrarte siempre en encontrar la manera mas sencilla, rápida y barata de probar un aliviador de frustraciones o un creador de alegrías, para luego crear prototipos.\\\\
\textcolor{blue}{¿ Tienes datos que que demuestren.... }
\begin{itemize}
\item \textcolor{blue}{ cual de tus productos y servicios esperan realmente los clientes? }
\item \textcolor{blue}{ cuales son los que mas esperan? }
\item \textcolor{blue}{ cual de tus creadores de alegrías realmente necesitan o desean los clientes? }
\item \textcolor{blue}{ cuales son los que mas ansían? }
\item \textcolor{blue}{ cual de tus aliviadores de frustraciones ayuda a tus clientes con su problemas? }
\item \textcolor{blue}{ cuales son los que mas anhelan? }
\end{itemize}
\textcolor{red}{ Revisar página 192 2º libro }
\subsection{Probando el rectángulo}
Prueba las asunciones mas criticas relativas al modelo de negocio en el que está insertada tu propuesta de valor. Recuerda: Hasta las mejores propuestas de valor pueden fracasar si no hay un buen modelo de negocio. Que crea valor no sólo para tus clientes si no para tu negocio.
APORTA DATOS QUE DEMUESTREN QUE ES PROBABLE QUE FUNCIONE EN MODO CON EL QUE PRETENDES CREAR, OFRECER Y CAPTURAR VALOR.
\subsubsection{No descuides probar tu modelo de negocio}
Una propuesta de valor no tiene demasiado valor si no tiene los canales adecuados para legar a los clientes.\\
\textcolor{blue}{¿ Tienes pruebas que demuestren...}
\begin{itemize}
\item \textcolor{blue}{ que tendrás acceso a los socios necesarios para que funcione el modelo?}
\item \textcolor{blue}{ que tendrás acceso a los recursos necesarios para crear valor? }
\item \textcolor{blue}{ que serás capaz de realizar las actividades necesarias para crear valor? }
\item \textcolor{blue}{ como lograrás adquirir y retener clientes? }
\item \textcolor{blue}{ que puedes generar mas ingresos que los gastos en los que incurres? }
\item \textcolor{blue}{ como generarás ingresos a partir de los clientes? }
\item \textcolor{blue}{ a través de que canales podrás llegar a los clientes? }
\end{itemize}
\textcolor{red}{Revisar página 195}
\section{Haciendo pruebas paso a paso}
\subsection{Visión general del proceso de pruebas}
\begin{enumerate}
\item Extraer hipótesis. (página 200)
\item Priorizar hipótesis. (página 202)
\item Diseño de pruebas. (página 204)
\item Priorizar pruebas. (página 205)
\item Ejecutar pruebas. (página 205)
\item Plasmar aprendizajes. (página 206)
\item Progresar. (páginas 242-245)
\end{enumerate}
\subsection{Extraer tu hipótesis: ¿ Que debe ser cierto para que funcione tu idea ?}
Utiliza los lienzos de la propuesta de valor y del modelo de negocio para identificar que probar antes de salir a la calle. Define las cosas mas importantes que deben ser ciertas para que funcione tu idea.
\subsubsection{Definición de Hipótesis de un negocio}
Aquello que debe ser cierto para que tu idea funcione de forma parcial o total, pero que aún no se ha validado.\\\\
PARA TRIUNFAR, PREGÚNTATE, QUE DEBE SER CIERTO SOBRE... 
\begin{itemize}
\item TU MODELO DE NEGOCIO.
\item TU PROPUESTA DE VALOR.
\item TU CLIENTE.
\end{itemize}
\textcolor{red}{Revisar ejemplo páginas 200 y 201}
\subsection{Prioriza tus hipótesis: Qué podría acabar con tu negocio}
No todas las hipótesis son iguales de decisivas. Algunas pueden acabar con tu negocio, mientras que otras sólo importan una vez aciertan con las mas importantes.
Empieza a  priorizar qué es decisivo para la supervivencia.\\
\textbf{Clasifica todas las hipótesis según lo decisivas que sean para que tu idea sobreviva y prospere:\\
+ Decisivas para la supervivencia.\\ 
.\\
.\\
.\\
.\\
- Menos decisivas para la supervivencia. }\\\\
\begin{itemize}
\item IDENTIFICA A LOS ASESINOS DE NEGOCIOS. SE TRATA DE LAS HIPÓTESIS IMPRESCINDIBLES PARA LA SUPERVIVENCIA DE TU IDEA. ¡ PRUÉBALAS PRIMERO!
\item ¿ QUÉ PRIORIDADES IMPORTAN MAS? 
\end{itemize}
\textcolor{red}{Revisar ejemplo página 203 2º libro}
\subsection{Diseña tus experimentos con la tarjeta de pruebas}
Estructura todos tus experimentos con esta sencilla tarjeta de pruebas. Empieza probando las hipótesis mas decisivas.
\begin{enumerate}
\item \textbf{ Diseña un experimento }\\

\item \textbf{ Diseña una serie de experimentos para las hipótesis mas decisivas }\\

\item \textbf{ Clasifica las tarjetas de pruebas }\\

\item \textbf{ Haz experimentos }\\

\end{enumerate}
\textcolor{red}{Revisar ejemplos exhaustivamente paginas 204 y 205}\\\\
\begin{itemize}
\item ¿ COMO APRENDERÉ ?.
\item REPITE.
\item ¿ DÓNDE PUEDO APRENDER MAS LO ANTES POSIBLE ?.
\end{itemize}
\subsection{Plasmas tus conclusiones con la tarjeta de aprendizaje}
Estructura tus conclusiones con esta sencilla tarjeta de aprendizaje.\\
\textcolor{red}{Revisar ejemplo exhaustivamente página 206}\\\\
\begin{itemize}
\item HAS EXPERIMENTADO Y APRENDIDO. ¿ y AHORA QUE ?
\end{itemize}
\begin{itemize}
\item \textbf{ Invalidada }\\
\begin{itemize}
\item Vuelve a la pizarra: PIVOTA.\\
Busca nuevos segmentos alternativos, propuestas de valor o modelos de negocio para que tus ideas funcionen cuando tus pruebas invalidan tus primeros intentos.
\end{itemize}
\item \textbf{ Aprende más }\\
\begin{itemize}
\item Busca la confirmación.\\
Diseña y realiza mas pruebas cuando los experimentos rápidos basados en una cantidad pequeña de datos indican la necesidad de tomar acciones drásticas.\\
\item Profundiza tu conocimiento.\\
Diseña y haz mas pruebas para comprender por qué hay una tendencia una vez la hayas descubierto.
\end{itemize}
\item \textbf{ Validada }\\
\begin{itemize}
\item Avanza al siguiente elemento clave.\\
Pasa a probar tu siguiente hipótesis cuando estés satisfecho con tus conclusiones y la fiabilidad de los datos.
\item Ejecuta.\\
Cuando estés satisfecho con la calidad de tus conclusiones y la fiabilidad de tus datos, puedes empezar a ejecutarlo todo basándose en tus descubrimientos.
\end{itemize}
\end{itemize}
\textcolor{red}{Revisar página 207}
\subsection{Como aprendes de rápido}
Lo único que se interpone entre tú y descubrir lo que los clientes y socios realmente esperan es la coherencia y la velocidad con la que tu equipo y tú avancen durante el ciclo de diseñar/construir, medir y aprender, a esto se lo conoce como tiempo de ciclo.\\
Por el hecho de que no sabes como reaccionarán los clientes es necesario que los primeros experimentos sean extremadamente rápidos y así saques conclusiones y te puedas adaptar rápido.
% Table generated by Excel2LaTeX from sheet 'Hoja1'
\begin{table}[htbp]
  \centering

    \begin{tabular}{cc}
    \multicolumn{2}{c}{\textbf{+}} \\
    \multicolumn{2}{c}{\textbf{Aprendizaje rápido}} \\
    \midrule
    \multicolumn{1}{l|}{\textit{\textbf{ULTRARRÁPIDO}}} & \multicolumn{1}{l}{\textit{Dibujos en servilletas}} \\
    \multicolumn{1}{r|}{} & \multicolumn{1}{l}{\textit{Leinzo MN y PV}} \\
    \multicolumn{1}{l|}{\textit{\textbf{RÁPIDO}}} & \textit{Entrevistas a clientes, socios y partes interesadas} \\
    \multicolumn{1}{r|}{} & \multicolumn{1}{l}{\textit{Biblioteca de experimentos}} \\
    \multicolumn{1}{l|}{\textit{\textbf{LENTO}}} & \multicolumn{1}{l}{\textit{Plan de negocio}} \\
    \multicolumn{1}{l|}{\textit{\textbf{MUY LENTO}}} & \multicolumn{1}{l}{\textit{Estudios de mercado encargado a terceros}} \\
    \multicolumn{1}{l|}{\textit{\textbf{MUY LENTO}}} & \multicolumn{1}{l}{\textit{Estudio piloto}} \\
    \midrule
    \multicolumn{2}{c}{\textbf{Aprendizaje lento}} \\
    \multicolumn{2}{c}{\textbf{-}} \\
    \end{tabular}%
  \label{tab:addlabel}%
\end{table}%
\subsection{Instrumentos para el aprendizaje}
Da forma a tus ideas en poco tiempo para compartirlas, verificarlas, iterarlas o generar hipótesis para probarlas. 
\begin{itemize}
\item Obtén primeras conclusiones del mercado de forma rápida.
\item Utiliza toda la gama de experimentos (página 214), empieza con los rápidos cuando el grado de incertidumbre sea alto.
\item Los planes de negocios son documentos más perfeccionados y normalmente más estéticos. Redáctalo sólo cuando tengas datos claros y te estés aproximando a la fase de ejecución.
\item Los estudios de mercado suelen ser costosos y lentos. No son una herramienta de búsqueda óptima porque permiten adaptarse a las circunstancias con rapidez.
\item Un estudio piloto suelen ser la manera por defecto de probar una idea dentro de una empresa. Sin embargo, debería ir precedido de herramientas de aprendizaje mas rápidas y baratas.
\end{itemize}
CUANTO MAS RÁPIDO ITERES, APRENDES MAS Y TRIUNFAS MAS DEPRISA.
\subsection{Cinco trampas de datos a evitar}
Evita el fracaso pensando de manera crítica sobre tus datos. 
\subsubsection{La trampa del salso positivo}
\paragraph{Riesgo}
Ver cosas que no están.
\paragraph{Sucede}
Cuando los datos de tus pruebas te hacen creer, por ejemplo, que tu cliente tiene una frustración, cuando en realidad no es así. \\
Diseñar distintos experimentos para la misma hipótesis.
\subsubsection{La trampa del falso negativo}
\paragraph{Riesgo} No ver cosas que si están
\paragraph{Sucede} Cuando tu experimento no detecta, por ejemplo, un trabajo del cliente que supuestamente debía develar.
\subsubsection{La trampa del máximo local}
\paragraph{Riesgo} Perderse el potencial real.
\paragraph{Sucede}
Cuando llevas a cabo experimentos que optimizan un máximo local e ignoran una oportunidad mayor. Por ejemplo, un resultado positivo de las pruebas puede hacer que termines con un modelo menos rentable cuando existe uno mas rentable.\\
Concentrare mas en aprender que en optimizar.
\subsubsection{La trampa del máximo agotado}
\paragraph{Riesgo}
Pasar por alto las limitaciones de un mercado.
\paragraph{Sucede}
Cuando crees que una oportunidad es mayor de lo que realmente es. Por ejemplo, cuando crees que estás probando una muestra de una gran población, pero en realidad es de la población entera.\\ ve mas allá
\subsubsection{La trampa de los datos equivocados}
\paragraph{Riesgo}Buscar en el sitio equivocado.

\paragraph{Sucede} Cuando abandonas una oportunidad porque estas trabajando con los datos equivocados. Por ejemplo, puede que abandones una idea porque los clientes con los que estas haciendo pruebas no estén interesados y no te das cuenta de que hay gente su lo está.
\\ Vuelve a diseñar otras alternativas antes de desistir.
\section{Biblioteca de experimentos}
\subsection{Elige una combinación de experimentos}
Cualquier experimentos tienes sus fortalezas y sus debilidades.
Para diseñar debemos tener en cuenta:
\begin{itemize}
\item Coste.
\item Fiabilidad de los datos.
\item Tiempo requerido.
\end{itemize}
Como norma generar empieza con lo más barato cuando haya mucha incertidumbre y aumenta el gasto cuando la certeza vaya creciendo.
\paragraph{Definición de experimento}
Procedimiento que valida una propuesta de negocio y que genera datos.
Ten dos cosas en mente cuando prepares tu combinación.
\subsubsection{Lo que los clientes dicen y hacen son dos cosas distintas}
Utiliza experimentos que aporten datos verbales como punto de partida. Consigue que los clientes realicen acciones y se involucren en ellas. (Por ejemplo interactuar con un prototipo) Para generar fatos mas sólidos.
\subsubsection{Los clientes se compartan de manera distinta si tú estás o no estás}
Durante el contacto personal directo con los clientes, puedes descubrir por que hacen o dicen algo y obtener información sobre cómo mejorar tu propuesta de valor. 
Utiliza estas técnicas en las fases iniciales del proceso de diseño. 
\subsection{Genera datos con una llamada a la acción}
Utiliza Experimentos para probar si los clientes están interesados.
\paragraph{Definición de llamada a la acción}
Provoca que un individuo realice una acción. Se usa en un experimento para probar una o más hipótesis.
\subsubsection{Utiliza experimentos para probar...}
\paragraph{Interés y relevancia}
Demuestra que los clientes potenciales están interesados de verdad y no lo dicen por decir. Muestra que tus ideas son lo suficientemente relevantes para que lleven a cabo llamadas de acción.
\paragraph{Prioridades y preferencias}
Muestra qué los trabajos, frustraciones y alegrías valoran mas tus clientes y socios potenciales y cuales valoran menos. Aporta datos que indiquen que características de tu propuesta de valor prefieren.
\paragraph{Si están dispuestos a pagar}
Ofrece datos que muestren que harán con el dinero lo que prometen.
\subsection{Seguimiento de anuncios}
utiliza esta el seguimiento de anuncios para explorar los trabajos, frustraciones, alegrías y el interés potencial de tus clientes por tu propuesta de valor o su ausencia. 
\subsubsection{Prueba el interés del ciente con Google AdWords}
\begin{enumerate}
\item \textbf{ Selecciona los términos de búsqueda }\\
Ya sea el trabajo, frustración o alegría de un cliente o el interés por una propuesta de valor. 
\item \textbf{ Diseña tu anuncio/prueba }\\
Diseña tu anuncio prueba con un titulo, un enlace a una página de aterrizaje y una nota.
\item \textbf{ Lanza tu campaña }\\
Define un presupuesto para tu campaña de pruebas de anuncio y lánzala. Paga sólo por los clics en el anuncio, que son los que representan interés.
\item \textbf{ Mide los clics }\\
Descubre cuánta gente hace clic en tu anuncio. Que no haya clics puede indicar falta de interés.
\end{enumerate}
\paragraph{Donde aplicarlo}
Prueba el interés del proceso para conocer la existencia de los trabajos, frustraciones y alegrías del cliente y el interés por una propuesta de valor concreta.
\subsubsection{URL del seguimiento}
Establece un URL de seguimiento para verificar el interés de os clientes.
\begin{enumerate}
\item \textbf{ Fabrica una URL única }\\
Haz un enlace único y del que puedas hacer un seguimiento para obtener información mas detallada sobre tus ideas con un servicio como goo.gl
\item \textbf{ Vende y haz un seguimiento }\\
Explica tu idea a un cliente o socio potenciales durante la reunión o después, proporciónale la URL y explicale que conduce a información mas detallada.
\item \textbf{ Descubre el interés real }\\
Averigua si el cliente ha utilizado el enlace o no. Si no ha clicado en el, puede indicar falta de interés o que hay trabajos, frustraciones y alegrías mas importantes que los que aborda tu idea.
\end{enumerate}
\subsection{Catálogo de PMV}
El producto mínimo variable PMV es un concepto popularizado por el movimiento Lean Startup para probar de manera eficiente el interés por un producto antes de construirlo por completo. \\\\
HAZLO REAL CON UNA REPRESENTACIÓN DE UNA PROPUESTA DE VALOR.
\begin{enumerate}
\item \textbf{ Hoja de datos  }\\
Especificaciones de tu propuesta de valor imaginada.
\item \textbf{ Folleto }\\
Un modelo de folleto de tu propuesta de valor imaginada.
\item \textbf{ Storyboard }\\
ilustración del escenario de un cliente que muestre tu propuesta de valor imaginada. Dibujada 
\item \textbf{ Página de aterrizaje }\\
Pagina web que resume tu propuesta de valor imaginada. 
\item \textbf{ La caja del producto }\\
Prototipo del packaging de tu propuesta de valor imaginada. 
\item \textbf{  Video }\\
Video mostrando tu propuesta de valor imaginada o explicando como funcionas. 
\item \textbf{ Prototipo de aprendizaje }\\
Prototipo que funcione de tu propuesta de valor con el conjunto de características mas básico necesario para poder aprender.
\item \textbf{ Mago de Oz }\\
Prepara un montaje que parezca una propuesta de valor real en funcionamiento y desempeña manualmente las tareas del producto o servicio que normalmente estarían automatizadas
\end{enumerate}
\subsection{Ilustraciones storyboards y escenarios}
Comparte ilustraciones, storyboards y escenarios relacionados  con las ideas de tu propuesta de valor con tus clientes para saber lo que de verdad les importa.
\begin{enumerate}
\item \textbf{ Crea prototipos de propuesta de valor alternativas }\\
PResetna carios prototipos alternativos para el mismo segmento de clientes, Apuesta por la diversidad. 
\item \textbf{ Define los escenarios }\\
Haz escenarios y storyborads que describan como experimentará un cliente cada propuesta e valor en un entorno del mundo real.
\item \textbf{ Crea imágenes visuales convincentes }\\
Utiliza un ilustrador para consolidar tus esbozos y que se conviertan en imágenes convincentes que hagan la experiencia del cliente clara y tangible.
\item \textbf{ Haz pruebas con los clientes }\\
reúnete con los clientes y presenta los distintos escenarios, storyboards e ilustraciones para iniciar una conversación, provocar reacciones, y aprender que les importa. Haz que los clientes clasifiquen las propuestas de valor y que vayan de las mas valiosas a las que les ayudan menos.
\item \textbf{ Saca conclusiones y adáptate}\\
utiliza la información de tus reuniones con los clientes. Decide que propuestas de valor seguirás explorando, cuáles abandonaras y cuáles adoptarás.
\end{enumerate} 
\subsubsection{Preguntas para hacerles a los clientes}
\begin{enumerate}
\item \textcolor{blue}{¿ Que propuestas de valor crean realmente valor para usted ?}
\item \textcolor{blue}{¿ Cuáles deberíamos mantener y seguir desarrollando y cuáles deberíamos abandonar ?}
\item \textcolor{blue}{¿ Que le falta ?}
\item \textcolor{blue}{¿ Que deberíamos quitar ?}
\item \textcolor{blue}{¿ Que deberíamos añadir ?}
\item \textcolor{blue}{¿ Que deberíamos reducir ?}
\item \textcolor{blue}{¿ Por que ? para obtener comentarios cualitativos}
\end{enumerate}
\subsection{Experimentos a tamaño real}
Haz que tus clientes interactúen con prototipos a tamaño real y con réplicas de experiencias de servicios reales.\\ 
\textcolor{red}{Revisar ejemplo paginas 226 y 227}
\subsection{Landing Page}
Un producto mínimo viable puede ser una página de aterrizaje, Una única página web que describe una propuesta de valor. o algunos aspectos de ella. Se invita al visitante a una llamada de acción por ejemplo registrándose con correo electrónico o simplemente a la tasa de conversión del número de visitantes.\\
EL OBJETIVO DE UNA LANDING PAGE ES VALIDAR UNA O MAS HIPÓTESIS, NO REUNIR E-MAILS O VENDER, POR MAS QUE SEA UNA CONSECUENCIA POSITIVA DEL EXPERIMENTO.
\begin{itemize}
\item \textbf{ ¿ Cuándo? }\\
Haz pruebas al principio para ver si los trabajos, frustraciones y alegrías son lo suficientemente importantes para el cliente y así lleven a cabo una acción.
\item \textbf{ Variaciones }\\
Combínalos con los split-tests para investigar preferencias o alternativas que funcionen mejor que otras. Mide la actividad por clics con los llamados mapas de calor para saber cuántos visitantes han entrado en tu página.
\end{itemize}
\subsubsection{Pasos a seguir}
\begin{itemize}
\item \textbf{ Tráfico }\\
Genera tráfico hacia tu página de aterrizaje ( PMV ) mediante anucnios, redes sociales o tus canales existentes. Asegúrate de dirigirte a los clientes objetivos de los que quieres aprender.
\item \textbf{ Título }\\
Redacta un título que hable a tus clientes potenciales y presente la propuesta de valor.
\item \textbf{ Propuesta de valor }\\
Redacta las técnicas que hemos descrito previamente para hacer que tu propuesta de valor sea clara y tangible para los clientes.
\item \textbf{ Llamada a la acción }\\
haz que los visitantes de la página web realicen una acción de la que puedas aprender, como registrar con el correo electrónico, compra falsa, pre-compra. Limita las llamadas a la acción para optimizar el aprendizaje.
\item \textbf{ Comunícate }\\
Dirígete a la gente que realizó tu llamada a la acción e investiga qué les motivó lo suficiente para hacerla. Descubre sus trabajos, frustraciones y alegrías. Evidentemente, esto requiere recoger información de contacto durante la llamada de acción.\\
\textcolor{red}{Ver diagrama de la página 229 }
\end{itemize}
\subsection{Split testing}
El split-testing, también conocido como test A/B, es una técnica para comparar como funcionan dos o mas opciones.
Acá lo usaremos para comparar propuestas de valor alternativas con clientes.
\begin{itemize}
\item \textbf{ Realizar split-test }\\
Se sueñe utilizar de esta técnica dos o mas variaciones de una página web. Pueden ser distintas en su propuesta de valor o diseño.
\item \textbf{¿ Que probar? }\\
\begin{itemize}
\item Características alternativas.
\item Precios.
\item Descuentos.
\item Textos.
\item Packaging.
\item Variaciones de un sitio web.
\end{itemize}
\item \textbf{ Llamada a la acción }\\
\textcolor{blue}{¿ Cuantos de los individuos sometidos a la prueba han realizado la llamada a la acción ?}
\begin{itemize}
\item Compra.
\item Registro con correo electrónico.
\item Clic en botón.
\item Encuesta.
\item Finalización de cualquier otra tarea.
\end{itemize}
\end{itemize}
\subsubsection{Split-tests del título de este libro}
\textcolor{red}{Revisar páginas 230 y 231}
\subsection{Innovation Games}
Es una metodología que te ayuda a diseñar mejor propuestas de valor mediante el juego colaborativo con clientes potenciales. Se puede jugar online o de manera presencial.
\begin{enumerate}
\item \textbf{ Compra una característica }\\
Tarea: Priorizar que características esperan mas los clientes.
\item \textbf{ La caja del producto }\\
Tarea: Comprender los trabajos, frustraciones y alegrías de tus clientes y las propuestas de valor que les gustarían.
\item \textbf{ La lancha motora }\\
Tarea: Identificar las frustraciones mas extremas que impiden que los clientes puedan resolver sus trabajos.
\end{enumerate}
\subsubsection{La lancha Motora}
Es un juego sencillo que trata de comprender las frustraciones de tu clientes. Haz que estos indiquen de manera explicita los problemas, obstáculos, y riesgos que les impiden resolver sus trabajos con éxitos usando la analogía de un barco frenado por anclas.
\begin{enumerate}
\item \textbf{ Preparación }\\
Prepara un póster grande con una lancha motora flotando en el mar.
\item \textbf{ Identifica las frustraciones }\\
Invita a los clientes a identificar los problemas, obstáculos y riesgos que les impiden resolver sus trabajos con éxito. Se debe anotar cada cuestión en una nota autoadhesiva grande. Pideles que coloquen las notas como si fueran anclas en la lancha; cuanto mas abajo este el ancla, mas extrema es la frustración.
\item \textbf{ Análisis }\\
Compara los resultados de este ejercicio con tu conocimiento previo de lo que impedía a los clientes resolver sus trabajos.
\end{enumerate}
Utiliza un barco de velas si quieres trabajar simultáneamente con frustraciones y alegrías. Las velas te permiten preguntar \textcolor{blue}{¿ Que hace que el barco vaya mas rápido?}
\subsubsection{La caja del producto}
En este juego pides a los clientes que diseñen la caja del producto que representa la propuesta de valor que les gustaría comprarte. Aprenderas que les importa a tus clientes y con que características se entusiasman.
\begin{enumerate}
\item \textbf{ Diseñar }\\
Dales a tus clientes una caja de cartón y pídeles literalmente una caja de producto que comprarían. La caja debería contener los mensajes clave de marketing, las principales características y los beneficios esenciales que ellos esperarían de tu propuesta de valor.
\item \textbf{ Vender }\\
Pide a tus clientes que se imaginen que están vendiendo tu producto en una feria. Finge que eres un potencial comprador escéptico y hay que tu cliente te venda la caja.
\item \textbf{ Plasmar }\\
Observa y anota que mensajes, características y beneficios mencionan los clientes en la caja y que aspectos concretos destacan durante tu intervención, Identifica sus trabajos, frustraciones y alegrías.
\end{enumerate}
\subsubsection{Compra una característica}
Es un juego donde los clientes priorizan entre una lista de características predefinidas que aun no existen de la propuesta de valor. Los clientes tienen un presupuesto limitado para comprar sus características preferidas, a la que tu pones precio basándote en factores del mundo real.
\begin{enumerate}
\item \textbf{ selecciona y pon precio a las características }\\
Selecciona las características con las que quieres probar las preferencias del cliente. Ponles precio basándote en el coste de desarrollo, el precio de mercado y otros factores. 
\item \textbf{ Define el presupuesto }\\
Los participantes compran características como grupos, pero cada participante obtiene un presupuesto personal que puede asignar de manera individual, asegúrate que el participante auna recursos y que el precio total les obligue a tomar decisiones difíciles entre las características que quieren.
\item \textbf{ Haz que los participantes compren }\\
Invita a los participantes a repartir su presupuesto entre las características que quieren.
\item \textbf{ Analiza resultados }\\
Analiza que características tienen mas tracción y la gente las compra y cuales no.
\end{enumerate}
\subsection{Ventas simuladas}
Una manera de probar el interés sincero del cliente es preparar una venta simulada antes de que la propuesta de valor exista. El objetivo es hacer creer a los clientes que están haciendo una compra real. Se puede realizar fácilmente en un contexto online o en uno físico.
\paragraph{Online}
\begin{itemize}
\item Compruebe cuantes personas hacen clic en comprar ahora.\
\item Descubre como influye el precio en el interés del cliente. Combínalo con el test A/B (pag. 230), para saber de la elasticidad de demanda y el precio óptimo.
\item Consigue datos fidedignos simulando una transacción con la información de la tarjeta de crédito del cliente. Es la prueba mas solida de la demanda. (pag. 237)
\paragraph{Mundo offline}
\begin{itemize}
\item Introducir productos que aún no existen en un número limitado de catálogos. (De ventas por correo).
\item Vender un producto en un solo punto de venta durante un tiempo limitado.
\end{itemize}
\subsubsection{Preventas}
EL objetivo es explorar al cliente. 
\end{itemize}
\paragraph{Online}
Kickstarter te ofrece anunciar un proyecto y si a los clientes les gusta pueden hacer un donativo. 
\paragraph{Mundo offline}
Las promesas, las cartas de intenciones y las firmas, son una técnica convincente para probar la disposición de compra de cliente.
\section{Reunirlo todo}
\subsection{El proceso de pruebas}
Utiliza todas las herramientas que has aprendido para describir lo que debes probar y cómo lo harás para transformar tu idea en realidad.
\begin{itemize}
\item \textbf{ Que probar }\\
Con los lienzos de propuesta de valor y del modelo de negocio., trazas como crees que tu idea podría convertirse en un éxito. Comienza una serie de experimentos.
\item \textbf{ Cómo probar }\\
Con la tarjeta de aprendizaje, describes exactamente cómo verificarás tus hipótesis mas importantes y que medirás. 
\item \textbf{ Y ahora ¿ qué? }\\
No pierdas de vista el objetivo y asegúrate de que están progresando. Haz un seguimiento de si estás avanzando correctamente de tu idea inicial hacia un negocio rentable y escalable mediante un encaje problema solución, producto mercado y de modelo de negocio.
\end{itemize}
\subsubsection{Pasos a seguir}
\begin{enumerate}
\item \textbf{ (Vuelve a) dar forma a tus ideas }\\

\item \textbf{ Extrae tus hipótesis }\\

\item \textbf{ Diseña tus pruebas }\\

\item \textbf{ Entra en el bucle del aprendizaje }\\
\begin{itemize}
\item Medir
\item Aprender
\item Construir
\end{itemize}
\item \textbf{ Plasma aprendizajes y acciones siguientes }\\
\begin{itemize}
\item INVALIDADA.- Pivotar o iterar
\item INCIERTA.- Probar mas.
\item VALIDADA.- Progresar hacia el siguiente elemento.
\end{itemize}
\item \textbf{ Mide el progreso }\\
\item Descubrimiento del cliente.
\item Validación del cliente.
\item Creación del cliente.
\item Creación de la empresa.
\end{enumerate}
\textcolor{red}{Revisar exhaustivamente páginas 240 y 241}
\subsection{Mide tu progreso}
El proceso de prueba te permite reducir la incertidumbre y te acerca a convertir tu idea en un negocio real.
Haz un seguimiento de las actividades que haz realizado y os resultados que haz conseguido.

\textcolor{red}{Revisar exhaustivamente páginas 242 y 243}
\subsection{La tabla de progresos}
Utiliza para gestionar y hacer un seguimiento de tus pruebas y evaluar cuánto progreso hacia el éxito vas consiguiendo.
\begin{itemize}
\item \textbf{ Que he probado ya }\\
Utiliza los lienzos de l propuesta de valor y del modelo de negocio para hacer un seguimiento de que elementos has probado, validado o invalidado.
\item \textbf{ Que estoy probando y que he aprendido }\\
Haz un seguimiento de las pruebas que estas planeando, construyendo, midiendo y digiriendo para aprender y hacer explicitas tus conclusiones
\item \textbf{ Cuánto he progresado }\\
Guarda registro de cuánto estás progresando.
\end{itemize}
\textcolor{red}{Rivisar exhaustivamente páginas 244 y 245}
\subsection{Owlet: progreso constante con diseño y pruebas sistemáticas}
Dispositivo inalámbrico sobre la sangre, el oxígeno, el ritmo cardíaco y el sueño del bebe.
\textcolor{red}{Revisar ejemplo de las páginas 246 a la 251 }
\chapter{Ajustar}
\subsection{Crear alineación}
El lienzo de a propuesta de valor es una excelente herramienta para alinear. Te ayuda a comunicar a las distintas partes interesadas en que trabajos, frustraciones y alegrías del cliente te estás concentrando, y explica como tus productos y servicios alivian frustraciones y crean alegrías.
\subsubsection{Crear mensajes alineados}
\begin{itemize}
\item Publicidad.
\item Pachaging.
\item Presentaciones.
\item Vídeos explicativos.
\item Guiones de ventas.
\end{itemize}
\subsubsection{Alinear partes interesadas internas y externas}
\begin{itemize}
\item \textbf{ Ventas }\\
Ayuda a entender al departamento de ventas a que segmentos deben apuntar y explica que son los trabajos, frustraciones y alegrías del cliente.
\item \textbf{ Socios (de canal) }\\
Invita a los socios (de canal) y explícales tu propuesta de valor, ayúdales a comprender porque les encantaran tus productos y servicios a tus clientes destacando los aliviadores de frustraciones y creadores de alegrías.
\item \textbf{ Marketing }\\
Crea mensajes de marketing basados en los trabajos, frustraciones y alegrías con los que ayudan tus productos y servicios. Alinea los mensajes del cliente en todo momento, desde la publicidad hasta el diseño del packaging.
\item \textbf{ Empleados }\\
Haz entender a todos los empleados que trabajos, frustraciones y alegrías del cliente tienes como objetivo y resume exactamente como tus productos y servicios crearán valor para tus clientes.
\item \textbf{ Accionistas }\\
Explica exactamente a tus accionistas como pretendes crear valor para tus clientes. Aclara como la propuesta de valor reforzará tu modelo de negocio y creará una ventaja competitiva.
\end{itemize}
\subsection{Medir y controlar}
Utiliza los lienzos de la propuesta de valor y del modelo de negocio para crear indicadores del rendimiento y controlarlos una vez tu propuesta de valor esté operativa en el mercado.Haz seguimiento de:
\begin{itemize}
\item Modelo de negocio.
\item Propuesta de valor.
\item Satisfacción del cliente. Circulo.
\end{itemize}
\textcolor{red}{Revisar exhaustivamente páginas 262 y  263 2º libro}
\subsection{Mejorar sin cesar}
Utiliza los mismo procesos y herramientas de la fase de pruebas y control para mejorar tu propuesta de valor una vez esté en el mercado.
\textcolor{red}{Revisar exhaustivamente páginas 264 y  265 2º libro}
\subsection{Reinventate constantemente}
Las empresas con éxito crean propuesta de valor que se venden en modelos de negocio que funcionan. Lo hacen continuamente, mientras tiene éxito siguen creando propuestas de valor y modelos de negocio nuevos.
\subsubsection{Cinco cosas que es conveniente recordar al crear ventajas transitorias}
\begin{enumerate}
\item Tómate la exploración de propuestas de valor y modelos de negocios nuevos tan seriamente como la ejecución de los ya existentes
\item Invierte en experimentar continuamente propuestas de valor y modelos de negocio nuevos en lugar de hacer grandes apuestas ambicionas inciertas.
\item Reinvéntate mientras tienes éxito. NO esperes a que te obliguen a hacerlo una crisis.
\item Tómate las nuevas ideas y oportunidades como un modo de activar y movilizar a empleados y clientes en lugar de verlo como un esfuerzo arriesgado.
\item Utiliza los experimentos con los clientes como criterio para juzgar nuevas ideas y oportunidades en lugar de tener en cuenta las opiniones de jefes, estrategas o expertos.
\end{enumerate}
\subsubsection{Preguntate continuamente...}
\begin{itemize}
\item \textbf{ Entorno}\\
\begin{itemize}
\item \textcolor{blue}{¿ Que elementos de tu entorno están cambiando ?}
\item \textcolor{blue}{¿ Que suponen los cambios de mercado, tecnológicos, normativos, macroeconómicos o competitivos para tus propuestas de valor y modelos de negocio ?}
\item \textcolor{blue}{¿ Ofrecen esos cambios una oportunidad para explorar posibilidades nuevas o podrían ser una amenaza que podría alterarte ?}
\end{itemize}
\item \textbf{ Modelos de negocio }\\
\begin{itemize}
\item \textcolor{blue}{¿ Está a punto de caducar tu modelo de negocio ?}
\item \textcolor{blue}{¿ Necesitas añadir actividades o recursos nuevos ?}
\item \textcolor{blue}{¿ Los existentes ofrecen una oportunidad para expandir tu modelo de negocio ?}
\item \textcolor{blue}{¿ Podrías reforzar tu modelo de negocio existente o construir otros completamente nuevos ?}
\item \textcolor{blue}{¿ Encaja en el futuro tu cartera de modelos de negocio ?}
\end{itemize}
\item \textbf{ Propuesta de valor }\\
\begin{itemize}
\item \textcolor{blue}{Se debe desarrollar y hacer frente a nuevas oportunidades, en lugar de buscar ventajas competitivas cada vez mas insostenibles en el tiempo}
\end{itemize}
\end{itemize}
\subsection{Taobao: Reinvertar el comercio electrónico}
\textcolor{red}{Revisar ejemplo de las páginas 268 al 271}












\chapter{Canales}

\subsubsection{Preguntas}
\begin{itemize}
\item\textcolor{blue}{¿ Que canales prefieren nuestros segmentos de mercado ? }
\item\textcolor{blue}{¿ Como establecemos el actualmente el contacto con nuestros clientes  ? }
\item\textcolor{blue}{¿ Como se conjugan nuestros canales ? }
\item\textcolor{blue}{¿ Cuales canales tienen mejores resultados ? }
\item\textcolor{blue}{¿ Cuales son mas rentables ? }
\item\textcolor{blue}{¿ Como se integran en las actividades diarias de los clientes ? }
\end{itemize}
\section{Tipos de canal}
\subsection{Equipo Comercial}
\subsection{Ventas en Internet}
\subsection{Tiendas propias}
\subsection{Tiendas de socios}
\subsection*{Mayoristas}
\section{Fases de Canal}
\subsection{Información}
\textcolor{red}{¿ Como damos a conocer los productos y servicios de nuestra empresa ?}
\subsection{Evaluación}
\textcolor{red}{¿ Como ayudamos a nuestros clientes a evaluar nuestra propuesta de valor ?}
\subsection{Compra}
\textcolor{red}{¿ Como pueden comprar los clientes nuestros productos y servicios ?}
\subsection{Entrega}
\textcolor{red}{¿ Como entregamos a los clientes nuestra propuesta de valor ?}
\subsection{Posventa}
\textcolor{red}{¿ Que servicio de atención postventa ofrecemos ?}


\chapter{Relaciones con clientes}
pueden estar basadas en los fundamentos siguientes:\\

\begin{itemize}
\item Captación de cliente\\
\item Fidelización de Clientes\\
\item Estimulación de las ventas (Ventas sugestivas)
\end{itemize}
\section{Preguntas}
\begin{itemize}
\item\textcolor{blue}{¿ Que tipo de relación esperan los diferentes segmentos de mercado ?}
\item\textcolor{blue}{¿ Que tipo de relaciones hemos establecido ?}
\item\textcolor{blue}{¿ Cual es su coste ?}
\item\textcolor{blue}{¿ Como se integran en nuestro modelo de negocios ?}
\end{itemize}
\section{Categoría de relaciones con clientes}
\begin{itemize}
\item ASISTENCIA PERSONAL \\

\item ASISTENCIA PERSONAL EXCLUSIVA\\

\item AUTOSERVICIO\\

\item SERVICIOS AUTOMÁTICOS\\

\item COMUNIDADES\\

\item CREACIÓN COLECTIVA\\ 

\end{itemize}
\chapter{Fuentes de Ingreso}
Un modelo de negocio puede implicar dos tipos diferentes de fuentes de ingreso 
\begin{enumerate}
\item Ingresos por transacciones derivados de pagos puntuales de clientes.
\item   Ingresos recurrentes derivados de pagos periódicos realizados a 
cambio del suministro de una propuesta de valor o del servicio 
posventa de atención al cliente.
\end{enumerate}	
\section{Preguntas}
\begin{itemize}
\item\textcolor{blue}{¿ Por qué valor están dispuestos a pagar nuestros clientes ?}
\item\textcolor{blue}{¿ Por qué pagan actualmente? }
\item\textcolor{blue}{¿ Cómo pagan actualmente?}
\item\textcolor{blue}{¿ Cómo les gustaría pagar? }
\item\textcolor{blue}{¿ Cuánto reportan las diferentes 
fuentes de ingresos al total de ingresos?}
\section{ Formas de generar fuentes de ingresos}
\begin{itemize}
\item VENTA DE ACTIVOS\\
\item CUOTA POR USO\\
\item CUOTA DE SUSCRIPCIÓN\\
\item PRÉSTAMO/ALQUILER/LEASING
\item CONCESIÓN DE LICENCIAS\\
\item GASTOS DE CORRETAJE\\
\item PUBLICIDAD\\ 
\end{itemize}
\subsection{Mecanismos de fijación de precios}
Existen dos mecanismos de fijación de precios principales:\\
\subsubsection{Fijo}
\paragraph{Lista de precios fija}
Precios fijos para productos servicios y otras propuestas de valor individuales.
\paragraph{Según características del producto}
El precio depende de la cantidad o la calidad de la propuesta de valor
\paragraph{Según segmento de mercado}
el precio depende del tipo y las características de un segmento de mercado.
\paragraph{Según volumen}
El precio depende de la cantidad adquirida
\subsection{Dinámico}
Los precios cambian en función del mercado
\paragraph{Negociación}
El precio se negocia entre dos o más socios y depende de las habilidades o el poder de negociación
\paragraph{de la rentabilidad}
El precio depende del inventario y del momento de la compra (suele utilizarse en recursos perecederos, como habitaciones de hotel o plazas de avión)
\paragraph{Subastas}
El precio se determina en una licitación
\end{itemize}
\chapter{Recursos clave}
En este módulo se describen los activos más importantes para que un modelo de negocio funcione.\\
Los recursos clave pueden ser físicos, económicos, intelectuales o humanos.
\section{Preguntas}
\begin{itemize}
\item\textcolor{blue}{¿ Qué recursos clave requieren nuestras propuestas de valor, canales de distribución, relaciones con clientes y fuentes de ingreso ?}
\end{itemize}
\section{Categorías}
\begin{itemize}
\item Físicos
\item Intelectuales
\item Humanos 
\item Económicos 
\end{itemize}
\chapter{Actividades clave}
Se describe las acciones más importantes que debe emprender una empresa para que su modelo de negocios funcione
\section{Preguntas}
\begin{itemize}
\item\textcolor{blue}{¿ Qué actividades clave requieren nuestras propuestas de valor, canales de distribución, relaciones con clientes y fuentes de ingreso ?}
\end{itemize}
\section{Categorias}
\begin{itemize}
\item Producción 
\item Resolución de los problemas
\item Plataforma/red
\end{itemize}
\chapter{Asociaciones clave}
Se describe la red de proveedores y socios que contribuyen al funcionamiento de un modelo de negocios.\\
Podemos hablar de cuatro tipos de asociaciones:\\
\begin{itemize}
\item Alianzas estratégicas entre empresas no competidoras
\item Coopetición: Asociaciones estratégicas entre empresas competidoras
\item Joint ventures: (Empresas conjuntas) para crear nuevos negocios
\item Relaciones cliente-proveedor para garantizar la fiabilidad de los suministros
\end{itemize}
\section{Preguntas}
\begin{itemize}
\item\textcolor{blue}{¿ Quiénes son nuestro socios clave ?}
\item\textcolor{blue}{¿ Quiénes son nuestros proveedores clave ?}
\item\textcolor{blue}{¿ Qué recursos clave adquirimos a nuestros socios ?}
\item\textcolor{blue}{¿ Qué actividades clave realizan los socios ?}
\end{itemize}
\section{Tres motivaciones para establecer asociaciones}
\begin{itemize}
\item Optimización y economía de escala
\item Reducción de riesgos e incertidumbre
\item 
\end{itemize}
\chapter{Estructura de costes}
Se describen todos los costes que implica la puesta en marcha de un modelo de negocios.\\
\section{Preguntas}
\begin{itemize}
\item\textcolor{blue}{¿ Cuáles son los costes más importantes inherentes a nuestro modelo de negocios ?} 
\item\textcolor{blue}{¿ Cuáles son los recursos clave más caros ?}
\item\textcolor{blue}{¿ Cuáles son las actividades clave más caras ?}
\end{itemize}
\subsection{Según costes}
El objetivo de los modelos de negocios basados en los costes es recortar gastos en donde sea posible
\subsection{Según valor}
Algunas empresas se centran en la propuesta de valor antes de considerar los costes
\section{Característica de las estructuras de costes}
\begin{itemize}
\item Costes fijos 
\item Costes variables
\item Economía a escala
\item Economía de campo
\end{itemize}


\part{Patrones}
\let\cleardoublepage\clearpage
\let\cleardoublepage\clearpage
\chapter{Desegregación de modelo de negocio}
\section{Definición}
El concepto de empresa desagregada sostiene que existen fundamentalmente tres tipos de actividades empresariales diferentes:\\
Se asegura que cada una de estas actividades y solo concentrarse en una por el conflicto que pueden acarrear cada una de estas. \\
\begin{itemize}
\item Relaciones con clientes.
\item Innovación de productos.
\item Infraestructuras.  
\end{itemize}
\subsection{Uno}
Se sugiere que la empresa debería centrarse en una de las 3 disciplinas que proponen
\begin{itemize}
\item Excelencia operativa.
\item Liderazgo del producto.
\item Intimidad con el Cliente.
\end{itemize}
\subsection{Dos}
\subsubsection{Relaciones con clientes}
Consiste en buscar y conseguir clientes y en establecer relaciones con ellos.\\ La empresa se puede centrar en captar clientes.
\subsubsection{Innovación de productos}
Desarrollar nuevos productos y servicios que resulten atractivos.\\
Se lo puede dejar a empresas creativas mas pequeñas.
\subsubsection{Infraestructuras}
Consiste en construir y gestionar plataformas para tareas repetitivas y volúmenes elevados.\\
Nos podemos a terceros que puedan crear o darnos el servicio a escala.\\\\

\begin{flushright}
\textcolor{red}{Revisar pagina 64 y 65}
\end{flushright}
\chapter{La larga cola}
\section{Definición}
El principio de los modelos de negocio de larga cola es vender menos de más, ofrecer una amplia gama de productos especializados que, por separado, tienen un volumen de ventas relativamente bajo.
\section{Tres Factores Económicos}
\subsection{Democratización de las herramientas de producción}
La bajada de los precios de la tecnología Permitió que los usuarios individuales accediesen a herramientas que hace tan solo unos años tenían precios elevados.Ahora millones pueden acceder a hacer producciones profesionales.
\subsection{Democratización de la distribución}
Internet ha convertido la distribución de contenido digital en un producto básico y ha reducido drásticamente los costes de inventario, las comunicaciones y las transacciones, abriendo así nuevos mercados para los productos especializados.
\subsection{Bajada de los costes de búsqueda para coordinar la oferta y la demanda}
El verdadero desafío que plantea la venta de contenido especializado es encontrar compradores que puedan estar interesados. Los potentes motores de búsqueda y recomendación, las calificaciones de los usurarios y las comunidades de interés han facilitado esta tarea enormemente.
\begin{flushright}
\textcolor{red}{Revisar pagina 75}
\end{flushright}
\chapter{Plataformas multilaterales}
\section{Definición}
Reúnen a dos o más grupos de clientes distintos pero interdependientes. Este tipo de plataformas solamente son valiosas para u grupo de clientes si los demás grupos de clientes también están presentes, crea valor al permitir la interacción entre los diferentes grupos.
\subparagraph{Efecto Red}
Valor que aumenta a medida que aumenta el número de usuarios.
\begin{flushright}
\textcolor{red}{Revisar pagina 87}
\end{flushright}
\chapter{Gratis como modelos de negocio}
\section{Definición}
Al menos un segmento de mercado se beneficia constantemente de una oferta gratuita.\\ Una parte del modelo financia los productos o servicios que ofrecen gratuitamente a otra parte o segmento.
\section{Patrones}
\subsection{Oferta gratuita basada en una plataforma multilateral (publicidad)}
Consiste en subvencionar con publicidad.\\
Por un lado se atrae a los usuarios con contenido, productos o servicios gratuitos, mientras que en el otro se generan ingresos mediante la venta de espacio a los anunciantes.
\begin{flushright}
\textcolor{red}{Revisar pagina 95}
\end{flushright} 
\subsection{Servicios básicos gratuitos con servicios premium opcionales (modelo freemium)}
Se refiere a un modelo de negocio, basado principalmente en internet, que combina servicios básicos gratuitos con servicios premium de pago. El modelo freemium se caracteriza por contar con una amplia base de usuarios que disfrutan de una oferta gratuita sin condiciones.
\begin{flushright}
\textcolor{red}{Revisar pagina 102 y 103}
\end{flushright} 
\subsection{Modelo del cebo y el anzuelo (bait and hook)}
Una oferta inicial atractiva, económica o gratuita fomenta la compra repetida de productos o servicios relacionados en el futuro, este patrón también se conoce como modelo de "reclamo publicitario" (loss leader) o "de cuchilla y hoja de afeitar" (razor and blades).\\
El \textbf{modelo de reclamo publicitario} se refiere a una oferta inicial subvencionada, en la que incluso se pierde dinero, cuyo objetivo es generar beneficios con las compras relacionadas posteriormente.\\
La \textbf{cuchilla y la hoja de afeitar} es un modelo de negocio con una oferta inicial para ganar dinero con las ventas posteriores. ejemplo: Gillete y sus hojas de afeitar.
\subparagraph{Lock-in}
Estrecho vinculo entre el producto inicial y los productos o servicios complementarios.
\begin{flushright}
\textcolor{red}{Revisar pagina 107}
\end{flushright} 
\chapter{Modelos de negocio abiertos}
\section{Definición}
Se pueden utilizar para crear y captar valor mediante la colaboración sistemática con socios externos.\\
Se puede realizar de \textit{afuera hacia adentro} aprovechando las ideas externas de la empresa o \textit{de dentro afuera} proporcionando a terceros ideas o activos que no se estén utilizando en la empresa.\\
Las empresas pueden crear mas valor y explotar mejor sus procesos de investigación si integran conocimiento, objetos de propiedad intelectual y productos externos en su trabajo de innovación. \\
Los productos, tecnologías, conocimientos y objeto de propiedad intelectual que se utilizan en la empresa se pueden poner a disposición de terceros, mediante licencias, joins ventures o spin-offs (Empresas segregadas)
\section{Dodelos de innovación}
\subsection{Fuera adentro}
La empresa integra ideas, tecnología u objetos de propiedad intelectual externos en sus procesos de desarrollo y comercialización. 
\subsection{Dentro afuera}
La empresa concede licencias o vende sus tecnologías u objetos de propiedad intelectual, los activos que no utiliza.
% Table generated by Excel2LaTeX from sheet 'Hoja1'
\begin{table}[htbp]
  \centering
  \caption{Principios de innovación}
  \vspace{1cm}
    \begin{tabular}{p{17.5em}|p{17.5em}}
    \multicolumn{1}{c}{\textcolor[rgb]{ .267,  .329,  .416}{\textbf{Cerrada}}} & \multicolumn{1}{c}{\textcolor[rgb]{ .267,  .329,  .416}{\textbf{Abierta}}} \\
\midrule
    Los talentos de nuestro sector trabajan para nosotros & Debemos trabajar tanto con talentos de la empresa como con talentos externos  \\
    \midrule
    Para beneficiarnos del trabajo de (I+D), debemos encargarnos del descubrimiento, el desarrollo y la provisión de valor & El trabajo del I+D externo puede crear un valor notable, los procesos internos de I+D son necesarios para acreditar parte de este valor. \\
    \midrule
    Si realizamos la mejor instigación del sector, ganaremos & No tenemos que investigar para beneficiarnos de la investigación. \\
    \midrule
    Si generamos la mayoría de las ideas del sector o las mejores, ganaremos & Si utilizamos las mejores ideas internas y externas, ganamos. \\
    \midrule
    Debemos controlar nuestro proceso de innovación para que la competencia no se beneficie de nuestras ideas & Debemos rentabilizar el uso de nuestras innovaciones por parte de terceros, así como adquirir objetos de propiedad intelectual (PI) de terceros, siempre que vayan a favor de nuestros intereses. \\
    \end{tabular}%
  \label{tab:addlabel}%
\end{table}%
\begin{flushright}
\textcolor{red}{Revisar pagina 116 y 117}
\end{flushright} 

\part{Diseño}
\let\cleardoublepage\clearpage
\let\cleardoublepage\clearpage
\chapter{Aportaciones de clientes}
\section{Creación de modelos de negocio a partir de aportaciones de clientes}
El reto consiste en conocer perfectamente el tipo de cliente en los que se debe basar el modelo de negocio.\\
Ver con los ojos del cliente. El concepto no implica tomar la visión del cliente como único punto de partida para una iniciativa de innovación, sino tener en cuenta su perspectiva a la hora de evaluar el modelo de negocio. El éxito se basa en una comprensión profunda del cliente, su entorno, sus rutinas diarias, sus preocupaciones y sus aspiraciones.\\
Se debe saber que clientes tomar en cuenta y que clientes ignorar.
\subsection{Modelo de negocio centrado en los clientes}
\begin{itemize}
\item\textcolor{blue}{¿ Qué servicios necesitan nuestros clientes y cómo podemos ayudarles ?}
\item\textcolor{blue}{¿ Qué aspiraciones tienen nuestros clientes y cómo podemos ayudarles a alcanzarlas ?}
\item\textcolor{blue}{¿ Qué trato prefieren los clientes ?} 
\item\textcolor{blue}{¿ Cómo podemos adaptarnos mejor a sus actividades cotidianas ?}
\item\textcolor{blue}{¿ Qué relación esperan los clientes que establezcamos con ellos ?}
\item\textcolor{blue}{¿ Por qué valores están dispuestos a pagar nuestros clientes ?}
\end{itemize}
\subsection{Mapa de empatía}
\textcolor{red}{¿ POR QUË ESTA DISPUESTO A PAGAR UN CLIENTE ?}
\subsubsection{¿ Qué ve ?}
Describe qué ve el cliente en su entorno
\begin{itemize}
\item\textcolor{blue}{¿ Qué aspecto tiene ?}
\item\textcolor{blue}{¿ Qué lo rodea ?}
\item\textcolor{blue}{¿ Quiénes son sus amigos ?}
\item\textcolor{blue}{¿ A qué tipos de ofertas está expuesto diariamente (En contraposición a todas las ofertas del mercado) ?}
\item\textcolor{blue}{¿ A qué problema se enfrenta ?}
\end{itemize}

\subsubsection{Qué oye}
Describe cómo afecta el entorno al cliente
\begin{itemize}
\item\textcolor{blue}{¿ Qué dicen sus amigos  ?, ¿ Su cónyuge ?}
\item\textcolor{blue}{¿ Quién es la persona que más le influye ?, ¿ Como le influye ?}
\item\textcolor{blue}{¿ Qué canales multimedia le influyen ?}
\end{itemize}
\subsubsection{¿ Qué piensa y siente en realidad ?}
Intenta averiguar qué pasa en la mente del cliente
\begin{itemize}
\item\textcolor{blue}{¿ Qué es lo más importante para el cliente (Aunque no lo diga explicitamente) ?}
\item\textcolor{blue}{Imagina sus emociones ¿ Que lo conmueve ?}
\item\textcolor{blue}{¿ Qué le quita el sueño ?}
\item\textcolor{blue}{ Intenta describir sus sueños y aspiraciones }
\end{itemize}
\subsubsection{¿ Qué dice y hace ?}
Imagina qué diría o cómo se comportaría el cliente en público
\begin{itemize}
\item\textcolor{blue}{¿ Cuál es su actitud ?}
\item\textcolor{blue}{¿ Qué podría estar contando a los demás ?}
\item\textcolor{blue}{Presta atención a las posibles incongruencias entre lo que dice un cliente y lo que piensa o siente en realidad }
\end{itemize}
\subsubsection{¿ Qué esfuerzos hace el cliente ?}
\begin{itemize}
\item\textcolor{blue}{¿ Cuáles son sus mayores frustraciones ?}
\item\textcolor{blue}{¿ Qué obstáculos se interponen entre el cliente y sus deseos o necesidades ?}
\item\textcolor{blue}{¿ Qué riesgos teme asumir ?}
\end{itemize}
\subsubsection{¿ Qué resultados obtiene el cliente ?}
\begin{itemize}
\item\textcolor{blue}{¿ Qué desea o necesita conseguir en realidad ?}
\item\textcolor{blue}{¿ Qué baremos utiliza para medir el éxito ?}
\item\textcolor{blue}{ Piensa en algunas estrategias que podría utilizar para alcanzar sus objetivos }
\end{itemize}
\subsection{Cómo utilizar el mapa de empatia (con el cliente)}
\begin{enumerate}[1.]
\item Realizar una cesión de brainstorming para identificar todos los segmentos de mercado a los que podría dirigir tu modelo de negocio
\item Elige tres candidatos prometedores
\item Selecciona uno para el primer ejercicio de creación de perfil
\item Asigna al cliente un nombre y una serie de características demográficas 
\end{enumerate}
\subsection{Creación del perfil de un cliente B2B con el mapa de empatía}
EL objetivo es definir el punto de vista de un cliente para cuestionarse constantemente las premisas del modelo de negocio. La creación de un perfil de cleinte te permite responder más acertadamente a preguntas como las siguientes:
\begin{itemize}
\item\textcolor{blue}{ ¿ Esta propuesta de valor soluciona algún problema real del cliente ? }
\item\textcolor{blue}{ ¿ El cliente está realmente dispuesto a pagar por esto ? }
\item\textcolor{blue}{ ¿ Cómo prefiere que se establezca la comunicación ? }
\end{itemize}
\begin{flushright}
\textcolor{red}{Revisar pagina 132 y 133}
\end{flushright}
\chapter{Ideación}
\section{Generación de nuevas ideas de modelo de negocio}
Exige el dominio del arte de la ideación. No mira el pasado si no que crea mecanismos nuevos para permitir crear valor y percibir ingresos. 
Consiste en desafiar las normas para diseñar modelos originales que satisfagan las necesidades desatendidas, nuevas u ocultas de los clientes.\\
Para encontrar opciones nuevas o mejores es necesario engendrar un puñado de ideas para después elegir las mas apropiadas.
\subsection{Epicentro de la innovación en modelo de negocio}

\paragraph{Recursos}
Las innovaciones basadas en recursos nacen de la infraestructura o las asociaciones existentes de una empresa y tienen como objetivo ampliar o transformar el modelo de negocio. 
\paragraph{Oferta}
Las innovaciones basadas en oferta crean nuevas propuestas de valor que afectan a otros módulos del modelo de negocio.
\paragraph{Clientes}
Las innovaciones basadas en los clientes tienen su origen en las necesidades de los clientes, un acceso mas sencillo o una mayor comodidad. Al igual que todas las innovaciones derivadas de un solo epicentro, estas innovaciones afectan a otros módulos del modelo de negocio.
\paragraph{Finanzas}
Se trata de una innovación basadas en nuevas fuentes de ingresos, mecanismos de fijación de precios o estructuras de costes reducidas que afectan a otros módulos del modelo de negocio.
\paragraph{Varios epicentros}
Las innovaciones que tienen su origen en varios epicentros pueden tener un impacto significativo en varios módulos.
\subsection{El poder de las preguntas $"$y si$"$}
 a menudo tenemos dificultades para concebir modelos de negocio innovadores porque nuestro pensamiento se ve reprimido por el statu quo. La preguntas del tipo $"$ y si $"$  desafían las premisas convencionales.
\begin{flushright}
\textcolor{red}{Revisar pagina 140 y 141}
\end{flushright}
\section{El proceso de ideación}
\subsection{Formación de equipo}
\textcolor{blue}{ ¿ Nuestro equipo es lo suficientemente heterogéneo como para generar ideas de modelo de negocio novedosas ? }\\
Los miembros deben ser diversos en cuanto a antigüedad, edad, grado de experiencia, unidad empresarial, conocimiento de los clientes y especialización profesional.
\subsection{Inmersión}
\textcolor{blue}{ ¿ Que elementos debemos estudiar antes de generar ideaas del modelo de negocio ? }\\
Lo ideal es que el equipo se someta a una fase de inmersión que incluya, por ejemplo, la investigación general, el estudio del cliente actuales o potenciales, el escrutinio de nuevas tecnologías o la evaluación de modelos de negocios.
Se podría realizar un mapa de empatía.
\subsection{Expansión}
\textcolor{blue}{ ¿ Qué innovación se nos ocurre para los diferentes módulos del modelo de negocio ? }\\
El equipo amplia el abanico de soluciones potenciales con el objetivo de generar tantas ideas como sea posible. En esta fase lo importante es la cantidad y no la calidad. Es importante que se formulen ideas y no criticas.
\subsection{Selección de criterios}
\textcolor{blue}{ ¿ Cuales son los criterios más importantes para establecer un orden de prioridades para nuestras ideas de modelo de negocio ? }\\
Tras ampliar el abanico de soluciones posibles, el equipo debería definir criterios para reducir el número de ideas a un número manejable. Los criterios deben ser específicos del contexto empresarial, aunque pueden incluir factores como el tiempo del aplicación estimado, el potencial de generación de ingresos, la posible reticencia de los clientes y el impacto sobre la ventaja competitiva.
\subsection{Creación de prototipos}
\textcolor{blue}{ ¿ Que aspecto tiene el modelo de negocio completo correspondiente a cada una de las ideas seleccionadas ? }\\
Con los criterios definidos, el equipo debería ser capaz de reducir el número de ideas a una lista prioritaria de entre tres y cinco innovaciones Utilizando el lienzo de modelo de negocio para esquematizar y comentar las diferentes ideas como prototipos de modelo de negocio.
\section{Constitución de un equipo heterogéneo}
Un equipo de innovación en modelos de negocios diversos tienen miembros:
\begin{itemize}
\item De diferentes unidades empresariales
\item De edades diferentes
\item Con diferentes especialidades
\item Con una antigüedad diferente
\item Con diferentes grados de experiencia
\item Con diferente bagaje cultural
\end{itemize}
\section{Normas para la sesión de brainstorming}
Para que la generación de ideas tenga éxito es necesario seguir una serie de normas. La aplicación de estas normas ayudará a aumentar el número de ideas útiles generadas.
\subsection{Concentración}
Empieza con una exposición detalla del problema. Lo ideal es que el problema esté relacionado con la necesidad de un cliente. No permitas que la discusión se aleje demasiado, del tema central y vuelve siempre al problema expuesto.
\subsection{Aplicación de normas}
En primer lugar, aclara las normas por las que se regirá la sesión de brainstorming y aplicalas. Las normas mas importantes son:\\
\begin{itemize}
\item Deja las críticas para después
\item Una conversación a la vez
\item Promueve las ideas alocadas
\end{itemize}
\subsection{Pensamiento visual}
Anota las ideas o realiza un esquema en una superficie que todos puedan ver. Una buena forma de recopilar las ideas es anotarlas en notas autoadhesivas y pegarlas en la pared. De esta manera, podrás cambiar las ideas de lugar y reagruparlas.
\subsection{Preparación}
Prepara la sesión de brainstorming con un ejercicio de inmersión que éste relacionado con el problema como, por ejemplo, salida exterior, un debate con los clientes o cualquier otro método que sumerja al equipo en temas relacionados con el problema expuesto.\\
\textcolor{red}{Revisar pagina 145 (ejercicio de la vaca)}
\chapter{Visual}
\section{El valor del pensamiento visual}
Entendemos por pensamiento visual el uso de herramientas visuales como la fotografías, esquemas, diagramas y notas autoadhesivas, para crear significado y establecer un debate al respecto.
Es difícil comprender un modelo de negocio sin antes dibujarlo. 
\subsection{Visualización con notas autoadhesivas}
Son notas post it, contenedores de ideas que se pueden pegar, quitar y cambiar de un módulo a otro con facilidad.
\begin{itemize}
\item \textbf{Pautas sencillas para visualización de las notas autoadhesivas.}
\begin{enumerate}
\item Utiliza rotuladores gruesos.
\item Escribe un solo elemento en cada nota.
\item Escribe pocas palabras en cada nota para captar la esencia.
\end{enumerate}
\end{itemize}
\subsection{Visualización con dibujos}
Los dibujos pueden tener una fuerza incluso mayor que las notas autoadhesivas, ya que las personas reaccionan más enérgicamente ante una imagen que ante una palabra. Las imágenes envían un mensaje instantáneo.
Realizar un bosquejo de las necesidades de un segmento de mercado resulta un método eficaz para explotar diferentes técnicas visuales.
\section{Cuatro procesos que el pensamiento visual ayuda mejorar}
\subsection{Captación de la esencia}
\begin{enumerate}
\item \textbf{Gramática visual.}\\
El modelo de negocio es una guía visual y textual con toda la información necesaria para esbozar 
\item \textbf{Visión global.}\\
El lienzo de modelo de negocios simplifica visualmente la realidad de una empresa con todos sus procesos estructuras y sistemas.
\item \textbf{Conocimientos de las relaciones.}\\
Para comprender un modelo de negocio no basta con conocer los elementos que lo componen, sino que es necesario captar las interdependencias entre los elementos, las cuales se expresan mejor con imágenes. 
\end{enumerate}
\subsection{Mejora del dialogo}
\begin{enumerate}
\item \textbf{Punto de referencia común.}\\
Todos tenemos ideas preconcebidas, por lo que mostrar una imagen donde esas premisas tácitas se convierten en información explícita contribuye a un diálogo más eficaz.
\item \textbf{Idioma común.}\\
El lienzo de modelo de negocios proporciona un vocabulario y una gramática para que las personas se entiendan mejor, así convirtiéndose en una potente herramienta para mantener un debate focalizdo.
\item \textbf{Consenso.}\\
La visualización de los modelos de negocio como un grupo es la forma más eficiente para alcanzar un consenso. Cuando varios especialistas colaboran en el dibujo de un modelo de negocios, todos los participantes adquieren conocimientos sobre los componentes individuales y así llegar a un consenso.
\end{enumerate}
\subsection{Exploración de ideas}
\begin{enumerate}
\item \textbf{Desencadenante de ideas.}\\
El lienzo es como el lienzo de un artista. Cuando un artista empieza a pintar, suele tener en mente una idea vaga, no empieza pintando una esquina del lienzo y continúa el trabajo de forma secuencial, si no empieza donde le dicte su inspiración y construye el cuadro de forma orgánica.
\item \textbf{Juego.}\\
Con los elementos de notas autoadhesivas puedes comprobar qué sucede al quitar determinados elementos o insertar otros nuevos. por ejemplo que sucede si quitásemos el segmento de mercado menos rentable. y si te desases de los clientes menos rentables podrías reducir costos?
\end{enumerate}
\subsection{Mejora de la comunicación}
\begin{enumerate}
\item \textbf{Divulgación de información en toda la empresa.}\\
Todas las personas que forman parte del negocio deben entender su modelo de negocios, ya que todos tienen el potencial de contribuir a su mejora.
\item \textbf{Venta interna.}\\
En las empresas, es frecuente que las ideas y los planes tengan que venderse a diferentes unidades internas para conseguir su apoyo o fondos. Una buena historia visual puede aumentar tus posibilidades de éxito.
\item \textbf{Venta externa.}\\
Del mismo modo los empleados deben vender las ideas internamente, los empresarios deben venderlos a otras empresas o personas, como inversores o posibles colaboradores. 
\end{enumerate}
\section{Un tipo de visualización para cada necesidad}
Las representaciones visuales de modelo de negocio requieren diferentes niveles de detalle, según el objetivo de cada uno.\\
\textcolor{red}{Revisar pagina 156 (Ejemplos visuales, ejemplo: Sellaband artistas musicales)}
\section{Narración de una historia visual}
Una forma eficaz de explicar un modelo de negocio seria contar una historia imagen por imagen. Se podría dibujar las diferentes partes de una en una.
\section{Ejercicio de narración visual}
\begin{enumerate}
\item \textbf{Esquematiza el modelo de negocio}
\begin{itemize}
\item Para empezar, escribe un esquema sencillo del modelo de negocios.
\item Escribe los diferentes elementos del modelo de negocio en una nota autoadhesiva.
\item El esquema se puede hacer de forma individual o en grupo.
\end{itemize}
\item \textbf{Dibuja los elementos del modelo de negocio}
\begin{itemize}
\item Coge las notas autoadhesivas de una en una, y sustitúyelas por el dibujo que refleje su contenido.
\item Dibuja imágenes sencillas y omite los detalles.
\item La calidad del dibujo no tiene importancia siempre que el mensaje esté claro.
\end{itemize}
\item \textbf{Define el guión}
\begin{itemize}
\item Ordena las notas autoadhesivas en la secuencia que desees contar la historia.
\item Prueba diferentes posibilidades, por ejemplo, puedes empezar por los segmentos de mercado o por la propuesta de valor.
\item Básicamente, cualquier punto de partida es válida si permite contar la historia con eficacia.
\end{itemize}
\item \textbf{Cuenta la historia}
\begin{itemize}
\item Muestra de una en una las imágenes dibujadas en las notas autoadhesivas para contar la historia del modelo de negocio.\\
\textbf{Nota: } Según el contexto y tus preferencias personales, quizá sea preferible que utilices un powerpoint o keynote. No obstante, la presentación de diapositivas probablemente no tendrá el mismo impacto positivo que el método de las notas autoadhesivas.
\end{itemize}
\end{enumerate}
\chapter{Creación de prototipos}
\section{El valor de los prototipos}
Los prototipos tiene como objeto el debate, el análisis y la corrección de un concepto.\\
Un prototipo puede cobrar la forma de un simple bosquejo, un concepto muy estudiado descrito en un lienzo de modelo de negocio o una hoja de cálculo que simule la mecánica financiera de una nueva empresa.
Se debe mencionar que un prototipo no es un borrador de modelo de negocio real , si no una herramienta para reflexionar sobre las direcciones que podría tomar el modelo de negocio.
\begin{itemize}
\item ¿ Que supondría para el modelo la adición de otro segmento de mercado ?
\item ¿ Que consecuencia tendría la eliminación de un recurso caro ?
\item ¿ Y si regalamos algo y cambiamos la fuente de ingresos por otra más innovadora ?
\end{itemize}
Para entender bien los pro y contra de las diferentes posibilidades y avanzar en nuestro análisis, necesitamos varios prototipos de nuestro modelo de negocio con diferentes niveles de detalle.\\
La interacción con prototipos propicia la generación de ideas mucho más que el debate. Los prototipos de negocio pueden ser provocadores, o incluso un poco alocados, de manera que nos obliga a exprimir la imaginación. Es entonces cuando se convierten en indicadores que apuntan en direcciones insospechadas. \\
El termino $"$ análisis $"$ debería hacer referencia a la búsqueda incansable de la mejor solución, y la única forma de seleccionar un prototipo para su perfeccionamiento y ejecución es decir realizar un análisis exhaustivo.\\ \vspace{0.2cm}

\textit{Estamos convencidos de que los modelos de negocio nuevos y revolucionarios nacen de un análisis profundo e incesante.}\\\\
\textbf{Pensamiento nuevo}
\begin{itemize}
\item Carios modelos de negocio en un mismo sector y en otros sectores.
\item De dentro afuera: Los modelos de negocio transforman los sectores.
\item Búsqueda preliminar de modelos de negocio.
\item Centrado en el diseño.
\item Centrado en el valor y la eficiencia. 
\end{itemize}
\section{Actitud de diseño}
Vemos un prototipo como algo que simplemente hay que pulir. En el mundo del diseño, los prototipos tienen una función en los procesos de visualización y comprobación previos a la aplicación, pero tienen también otra función muy importante, constituyen una herramienta de análisis. En este sentido, son una ayuda para la explotación de otras posibilidades. Nos ayudan a entender mejor las opciones potenciales.
\section{Prototipos a diferentes escalas}
No se trata únicamente de desarrollar ideas que pretendes aplicar, sino de explorar ideas nuevas, aunque sean absurdas o imposibles.
\subsection{Dibujo de una servilleta}
\textbf{Esboza y da forma a una idea indefinida}
Describe la idea sólo con elementos clave.
\begin{itemize}
\item Esboza la idea.
\item Incluye la propuesta de valor.
\item Incluye los principales fuentes de ingresos.
\end{itemize}
\subsection{Lienzo elaborado}
\textbf{Investiga que hace falta para que la idea funcione}
\begin{itemize}
\item Desarrolla un lienzo completo.
\item Reflexiona sobre la lógica empresarial.
\item Valora el potencial del mercado.
\item Comprende las relaciones entre los módulos.
\item Haz una comprobación rápida de los hechos.
\end{itemize}
\subsection{Plan de negocio}
\textbf{Estudia la viabilidad de la idea}
Convierte el lienzo detallado en una hoja de cálculo para calcular el potencial de beneficio.
\begin{itemize}
\item Crea un lienzo completo.
\item Incluye datos clave.
\item Calcule los costes e ingresos.
\item Calcula los beneficios potenciales.
\item Estudia varios escenarios relacionados con las finanzas y basados en diferentes ideas preconcebidas. 
\end{itemize}
\subsection{Prueba de campo}
\textbf{Investiga la aceptación de los clientes y la factibilidad}
\begin{itemize}
\item Prepara un plan de negocio justificado para el nuevo modelo.
\item Incluye clientes actuales a futuros en la prueba de campo.
\item Comprueba la propuesta de valor, los canales, los mecanismos de fijación de precios y otros elementos del mercado.
\end{itemize}
\textcolor{red}{Revisar ejemplo paginas 166, 167 y esquema pagina 168 y 169}
\begin{enumerate}
\item Decidir. 
\item Analizar.
\begin{itemize}
\item Diseñar.
\item Provocar.
\item Crear prototipos.
\end{itemize}
\item Ejecutar.
\end{enumerate}
\chapter{Narración de historia}
\section{El valor de la narración de historias}
La narración de historias te ayudará a explicar eficazmente en qué consiste el modelo, ya que acaba con la incredulidad ante lo desconocido.
\section{¿ Por qué contar una historia ?}
\subsection{Presentación de lo nuevo}
Algunas ideas pueden ser buenas, otras mediocres y otras completamente inútiles. NO obstante las ideas mas brillantes puede tener obstáculos para superar todos los niveles de dirección y convertirse en estrategias empresariales.
\paragraph{Materializa lo nuevo}
Explicar un modelo de negocios nuevo, que no se haya puesto en práctica, es como explicar un cuadro solo con palabras. Contar una historia sobre cómo crea valor un modelo es como aplicar colores brillantes a un lienzo, lo hace tangible.
\subsection{Venta a los inversores}
La pregunta de los inversores es:
\begin{itemize}
\item ¿ Como crearás valor para los clientes y como ganarás dinero haciéndolo ?
\end{itemize}
\paragraph{Aclaración}
Contar una historia que muestre cómo el modelo de negocio soluciona el problema de un cliente es una forma clara de presentar la idea al público. Las historias captan el interés necesario para después explicar detalladamente el modelo.
\subsection{Implicación de los empleados}
Las personas deben comprender a la perfección el nuevo modelo y lo que supone. Para ello presentamos un modelo con una historia interesante.
\paragraph{Implicación de la persona}
Las personas se rigen más por las historias que por la lógica. Presenta la lógica del modelo con una narración atractiva que familiarice al público con lo nuevo desconocido.
\section{Cómo convertir un modelo de negocio en tangible}
El objetivo de contar una historia es presentar un modelo de negocio nuevo de forma tangible y atractiva.\\
La historia debe ser sencilla y tener un único protagonista.
\subsection{Empresa perspectiva}
\paragraph{Empleado} Se cuenta la historia contada desde la perspectiva del empleado. Convierte al empleado en protagonista y haz que demuestre porque es lógico el nuevo modelo. el motivo puede ser que el empleado observa los problemas frecuentes de los clientes. Y en comparación del modelo anterior, el nuevo modelo hace uso mejorado de los recursos, costos, etc.
\subsection{Cliente perspectiva}
\paragraph{Tareas del cliente}
La perspectiva del cliente es un punto potente para la historia.
\begin{enumerate}
\item Utilizar un cliente como protagonista.
\item Contar la historia desde su punto de vista.
\item Muestra los retos al que se enfrenta y los trabajos que debe hacer.
\item Explicar que valor crea la empresa.
\item Incluir los servicios que recibe el cliente.
\item El lugar que ocupa por las que esta dispuesto a pagar.
\item Explicar por que la empresa le hace la vida mas fácil y como lo hace.
\item Con que recursos y mediante que actividades.
\item El desafío es que la historia suene auténtica.
\end{enumerate}
\section{Como hacer el futuro tangible}
las historias suelen ser una técnica excelente para desdibujar la linea que separa la realidad de la ficción.
\subsection{¿ Cuál es el modelo de negocio para el futuro ?}
\paragraph{Inspiración de ideas}
A veces, el único objetivo de una historia es desafiar el statu quo de una empresa. Esta historia debe dar vida a un entorno competitivo del futuro, donde el modelo de negocio actual sea cuestionado o incluso obsoleto.
\paragraph{Justificación de cambios}
A veces la empresa tiene una idea muy clara de cómo evolucionará el panorama competitivo. En este caso un modelo de negocios nuevo es perfecto para una empresa pueda competir en el nuevo panorama.
\section{Técnicas}
En función de la situación y del publico, elige la técnica adecuada cuando conozca a tu público y el contexto en que presentarás la historia.
% Table generated by Excel2LaTeX from sheet 'Hoja1'
\begin{table}[htbp]
  \centering
  \begin{turn}{90}
    \begin{tabular}{p{7em}|p{8.5em}p{8.5em}p{8.5em}p{8.5em}l}
    \multicolumn{1}{r}{} & Discurso e imagen & videoclip & juego de rol & texto e imagen & \multicolumn{1}{p{8.5em}}{tira cómica} \\
    \midrule
    Descripción & cuenta la historia de un protagonista y su entorno con una o varias imágenes & cuenta la historia de un protagonista y su entorno con video para desdibujar la línea que separa la realidad de la ficción & Asigna un papel de protagonista a cada uno de los participantes para que la situación parezca real y tangible & cuenta la historia de un protagonista y su entorno ediante texto e imágenes & \multicolumn{1}{p{8.5em}}{utiliza una tira cómica para contar la historia de un protagonista de forma tangible} \\
    Cuándo & presntación en grupo o confrencia  & presentación ante un público numeros o uso interno, para la toma de decisiones con importantes implicaciones económicas & talleres donde los participantes presentan ideas nuevas para modelos de negocio & informes o reporducciones ante un público númeroso & \multicolumn{1}{p{8.5em}}{informes o presntaciones ante un público numeroso} \\
    Tiempo y coste & bajo  & medio - alto & bajo  & bajo  & medio - bajo \\
    \end{tabular}%
  \label{tab:addlabel}%
    \end{turn}
\end{table}%
\\
\textcolor{red}{Revisar ejemplo pagina 179}

\chapter{Escenarios}
\section{Diseño de modelo de negocios basado en escenarios}
Su función principal es aportar al proceso de desarrollo del modelo de negocio un contexto de diseño especifico y detallado.
\subsection{Exploración de ideas}
Los escenarios de clientes nos guían en el diseño de modelos de negocio, nos ayudan con cuestiones como:
\begin{itemize}
\item Identificación de canales adecuados.
\item Las relaciones que debemos establecer.
\item Las soluciones por las que los clientes están dispuestos a pagar .
\end{itemize}
por otro lado se describen aspectos relacionados con el cliente:
\begin{itemize}
\item Como se utiliza los productos o servicios.
\item Qué tipos de clientes los utilizan. 
\item Cuales son las preocupaciones.
\item Cuales los objetivos.
\item Cuales los deseos.
\end{itemize}
\textcolor{red}{Revisar ejemplo paginas 184 y 185}
\subsection{Escenarios futuros}
Se describe el entorno en que un modelo de negocio competirá en el futuro.\\
El objetivo no es predecir el futuro si no imaginar varios futuros con detalles concretos.
\paragraph{Paso a Seguir}
\begin{itemize}
\item Inventar una serie de escenarios que presenten el futuro.
\end{itemize}
\textcolor{red}{Revisar ejemplo paginas 186 y 187}
\section{Escenarios futuros y nuevos modelos de negocio}
\begin{enumerate}
\item Desarrolla una serie de escenarios futuros basados en dos o mas criterios.
\item Describe cada caso con una historia que incluya los principales elementos del caso.
\item Desarrolla uno o varios modelos de negocio adecuados para cada escenario.
\end{enumerate}

\part{Estrategia}
\section{Introducción}
En este apartado veremos como reinterpretar la estrategia a través de la lente del lienzo de modelo de negocio. La reinterpretación te hará cuestionar constructivamente los modelos de negocio establecidos y analizar de forma estratégica el entorno de actuación de tu modelo.
\chapter{Entorno del modelo de negocio:\\ Contexto, factores de diseño y restricciones}
Los modelos de negocio se diseñan y aplican en entornos específicos.
\\ Un conocimiento profundo del entorno de la empresa te ayudará a concebir modelos de negocio mas fuertes y competitivos.\\
Es recomendable esbozar las cuatro áreas mas importantes del entorno:
\begin{itemize}
\item Fuerzas de mercado.
\item Fuerzas de la industria.
\item Tendencias clave.
\item Fuerzas macroeconómicas.
\end{itemize}
\section{Fuerzas del mercado}
\subsection{Cuestiones de mercado}
Identificar los aspectos que impulsan y transforman el mercado desde el punto de vista del cliente y la oferta.

\begin{itemize}
\item\textcolor{blue}{¿ Cuales son las cuestiones con un mayor impacto en el panorama del cliente ?}
\item\textcolor{blue}{¿ Que cambios se están produciendo ?}
\item\textcolor{blue}{¿ Hacia dónde va el mercado ?}
\end{itemize}
\subsection{Segmentos de mercado}
Identifica los principales segmentos de mercado, describe su capacidad generadora e intenta descubrir nuevos segmentos.

\begin{itemize}
\item\textcolor{blue}{ ¿ Cuales son los segmentos de mercado mas importantes ? }
\item\textcolor{blue}{ ¿ Que segmentos tiene mayor potencial de crecimiento ? }
\item\textcolor{blue}{ ¿ Que segmentos están decayendo ? }
\item\textcolor{blue}{ ¿ Que segmentos periféricos requieren atención ? }
\end{itemize}
\subsection{Necesidades y demandas}
Refleja las necesidades de mercado y estudia el grado en que están atendidas.
\begin{itemize}
\item\textcolor{blue}{ ¿ Que necesitan los clientes ? }
\item\textcolor{blue}{ ¿ Cuales son las necesidades menos atendidas ? }
\item\textcolor{blue}{ ¿ Que servicios quieren los clientes en realidad ? }
\item\textcolor{blue}{ ¿ Donde está aumentando la demanda y donde está decayendo ? }
\end{itemize}
\subsection{Costes de cambio}
Describe los elementos relacionados con el cambio de los clientes a la competencia.
\begin{itemize}
\item\textcolor{blue}{ ¿ Que vincula a los clientes a una empresa y su oferta ? }
\item\textcolor{blue}{ ¿ Que costes de cambio impiden que los clientes se vayan a la competencia ? }
\item\textcolor{blue}{ ¿ Los clientes tienen a su alcance otras ofertas similares ? }
\item\textcolor{blue}{ ¿ Que importancia tiene la marca ? }
\end{itemize}
\subsection{Capacidad generadora de ingresos}
Identifica a los elementos relacionados con la capacidad generadora de ingresos y de fijación de precios
\begin{itemize}
\item\textcolor{blue}{ ¿ Por que están dispuestos a pagar los clientes ? }
\item\textcolor{blue}{ ¿ Donde se puede conseguir un margen de beneficios mayor ? }
\item\textcolor{blue}{ ¿ Los clientes tienen a su alcance productos y servicios mas baratos ? }
\end{itemize}
\textcolor{red}{ejemplos pagina 203}
\section{Fuerzas de la industria}
\subsection{Competidores (incumbentes)}
Identifica a los competidores incumbentes y sus puntos fuertes relativos
\begin{itemize}
\item\textcolor{blue}{¿ Quienes son nuestros competidores ?}
\item\textcolor{blue}{¿ quienes son los principales jugadores de nuestro sector ?}
\item\textcolor{blue}{¿ Cuales son sus ventajas o desventajas competitivas ?}
\item\textcolor{blue}{Describe su oferta principal}
\item\textcolor{blue}{¿ En que segmentos de mercado se centran ?}
\item\textcolor{blue}{¿ Que estructura de costes tienen ?}
\item\textcolor{blue}{¿ Que influencias ejercen sobre nuestros segmentos de mercado, fuentes de ingreso y márgenes ?}
\end{itemize}
\subsection{Nuevos jugadores (tiburones)}
Identifica a los nuestros jugadores especuladores y determina si compiten con un modelo de negocio diferente al tuyo.
\begin{itemize}
\item\textcolor{blue}{¿ Quienes son los nuestros jugadores del mercado ?}
\item\textcolor{blue}{¿ En que se distinguen ?}
\item\textcolor{blue}{¿ Cuales son sus ventajas o desventajas competitivas ?}
\item\textcolor{blue}{¿ Que barreras deben superar ?}
\item\textcolor{blue}{¿ Cuales son sus propuestas de valor ?}
\item\textcolor{blue}{¿ En que segmentos de mercado se centran ?}
\item\textcolor{blue}{¿ Que estructura de costes tienen ?}
\item\textcolor{blue}{¿ Que influencia ejercen sobre tus segmentos de mercado, fuentes de ingresos y márgenes ?}
\end{itemize}
\subsection{Productos y servicios sustitutos}
Describe los posibles sustitutos de tus ofertas, incluidos los que proceden de otros mercados e industrias.
\begin{itemize}
\item\textcolor{blue}{¿ Que productos o servicios podrían sustituir a los nuestros ?}
\item\textcolor{blue}{¿ Cuanto cuestan en comparación con los nuestros ?}
\item\textcolor{blue}{¿ De que tradición de modelo de negocio proceden estos productos sustitutos (por ejemplo trenes de alta velocidad o aviones, teléfonos móviles, o cámaras. skype o empresas de telefonía a larga distancia) ?}
\end{itemize}
\subsection{Proveedores y otros actores de la cadena de valor}
Describe a los principales incumbentes de la cadena de valor de tu mercado e identifica nuestros jugadores emergentes.
\begin{itemize}
\item\textcolor{blue}{¿ Cuales son los principales jugadores de la cadena de valor ?}
\item\textcolor{blue}{¿ En que grado depende tu modelo de negocio de otros jugadores ?}
\item\textcolor{blue}{¿ Están emergiendo jugadores periféricos ?}
\item\textcolor{blue}{¿ Cuales son los mas rentables ?}
\end{itemize}
\subsection{Inversores}
Especifica a los actores que pueden influir en la empresa y el modelo de negocios
\begin{itemize}
\item\textcolor{blue}{¿ Que inversores podrían influir en tu modelo de  ?}
\item\textcolor{blue}{¿ Que grado de influencia tienen los inversores, los trabajadores, el gobierno y los grupos de presión ?}
\end{itemize}
\textcolor{red}{revisar pagina 205}
\section{Tendencias clave}
\subsection{Tendencias tecnológicas}
Identifica las tendencias tecnológicas que podrían poner en peligro tu modelo de negocio su evolución o mejora.
\begin{itemize}
\item\textcolor{blue}{¿ Cuales son las principales tendencias tecnológicas dentro y fuera del mercado ?}
\item\textcolor{blue}{ ¿ Que tecnologías presentan oportunidades importantes o amenazas disruptivas ? }
\item\textcolor{blue}{ ¿ Que nuevas tecnologías empiezan a utilizar los clientes periféricos ? }
\end{itemize}
\subsection{Tendencias normalizadoras}
Describe las normativas y tendencias normalizadoras que afectan al modelo de negocio
\begin{itemize}
\item\textcolor{blue}{ ¿ Que tendencias normalizadoras afectan a tu mercado ? }
\item\textcolor{blue}{ ¿ Que normas afectan a tu modelo de negocio ? }
\item\textcolor{blue}{ ¿ Que normativas e impuestos afectan a la demanda de los clientes ? }
\end{itemize}
\subsection{Tendencias sociales y culturales}
Identifica a las principales tendencias sociales que podrían afectar al modelo de negocio
\begin{itemize}
\item\textcolor{blue}{ Describe las principales tendencias sociales  }
\item\textcolor{blue}{ ¿ Que cambios en los calores culturales o sociales afectan a tu modelo de negocio ? }
\item\textcolor{blue}{ ¿ Que tendencias pueden influir en el comportamiento de los compradores ? }
\end{itemize}
\subsection{Tendencias socioeconómicas}
Describe las principales tendencias solcioeconómicas para tu modelo de negocio
\begin{itemize}
\item\textcolor{blue}{ ¿ Cuales son las principales tendencias demográficas ? }
\item\textcolor{blue}{ ¿ Como describirías la distribución de la riqueza y los ingresos en tu mercado (por ejemplo vivienda, asistencia sanitaria, ocio, etc.) ? }
\item\textcolor{blue}{ ¿ Que parte de la población vive en zonas urbanas, en oposición a emplazamientos rurales ? }
\end{itemize}
\textcolor{red}{ Revisar pagina 207 }
\section{Fuerzas macroeconómicas}
\subsection{Condición del mercado global}
Esboza condiciones generales actuales desde una perspectiva macroeconómica
\begin{itemize}
\item\textcolor{blue}{ ¿ La economía se encuentra en una fase de auge o decadencia ? }
\item\textcolor{blue}{ Describe el sentimiento general del mercado }
\item\textcolor{blue}{ ¿ Cual es la tasa de crecimiento del PIB ? }
\end{itemize}
\subsection{Mercado de capital}
Describe las condiciones actuales del mercado de capitales con relación a tus necesidades de capital.
\begin{itemize}
\item\textcolor{blue}{ ¿ En que estado se encuentran los mercados de capitales ? }
\item\textcolor{blue}{ ¿ Es fácil obtener fondos para tu mercado ? }
\item\textcolor{blue}{ ¿ El capital inicial, el capital de riesgo, los fondos públicos, el capital del mercado y los créditos son de fácil acceso ? }
\end{itemize}
\subsection{Productos básicos y otros recursos}
Resalta los precios actuales y las tendencias de precios de los recursos necesarios para tu modelo de negocio
\begin{itemize}
\item\textcolor{blue}{ Describe el estado actual de los mercados de productos básicos y otros recursos vitales para tu negocio (por ejemplo, precio del petróleo y costes del trabajo) }
\item\textcolor{blue}{ ¿ Es fácil obtener los recursos necesarios para ejecutar el modelo de negocio (por ejemplo, atraer a los grandes talentos) ? }
\item\textcolor{blue}{ ¿ que coste tienen ? }
\item\textcolor{blue}{ ¿ En que dirección van los precios ? }
\end{itemize}
\subsection{Infraestructura económica}
Describe la infraestructura económica de tu mercado
\begin{itemize}
\item\textcolor{blue}{ ¿ Es buena la infraestructura (pública) del mercado? }
\item\textcolor{blue}{ ¿ Como describirías el transporte, el comercio, la calidad de la educación y el acceso a los proveedores y los clientes ? }
\item\textcolor{blue}{ ¿ Son muy elevados los impuestos individuales y corporativos ? }
\item\textcolor{blue}{ ¿ Son buenos los recursos públicos disponibles para las empresas ? }
\item\textcolor{blue}{ ¿ Como calificarías la calidad de vida ? }
\end{itemize}
\textcolor{blue}{ Revisar pagina 209 }
\section{¿ Como debería evolucionar el modelo de negocio en vista del cambiante entorno ?}
Un modelo de negocio competitivo que funciona en el entorno actual podría quedarse obsoleto o anticuado mañana. Para ello es posible formular una serie de hipótesis sobre el futuro que nos guíen en el diseño de los modelo de negocio futuros.
Las premisas sobre la evolución de las fuerzas del merado, las fuerzas de la industria, las tendencias clave y las fuerzas macroeconómicas nos proporcionan un espacio de diseño en el que desarrollar posibles opciones de modelo de negocio y prototipos para el futuro.
\chapter{Evaluación de modelos de negocio}
Permite evaluar su posición en el merado y adaptarse en función de los resultados, si no se realizan revisiones frecuentes es posible que no se detecten a tiempo los problemas del modelo de negocio, lo que podría tener como resultado la desaparición de una empresa.
En este capitulo adaptaremos el punto de vista de un modelo de negocio existente y analizaremos las fuerzas externas de dentro afuera. \\
\textcolor{red}{Revisar ejemplo paginas 214 y 215}
\section{Análisis DAFO detallado de los diferentes módulos}
La evaluación de la integridad general del modelo de negocio es fundamental. Combinado con el lienzo de modelo de negocios permite realizar una evaluación rigurosa del modelo de negocio de la empresa y sus módulos.
\\ El análisis DAFO plantea cuatro grandes preguntas
\begin{itemize}
\item\textcolor{blue}{ ¿ Cuales son los puntos débiles y los puntos fuertes de una empresa ? }
Evalúa los aspectos internos de la empresa
\item\textcolor{blue}{ ¿ Que oportunidad tiene la empresa y a que amenazas potenciales se enfrentan ? }
Estudian la posición de la empresa en su entorno. 
\end{itemize}
Es interesante plantear estas cuatro preguntas con relaciñon al modelo de negocio y a cada uno de sus nueve módulos
% Table generated by Excel2LaTeX from sheet 'Hoja1'
\begin{table}[htbp]
  \centering
  \caption{DAFO}
    \begin{tabular}{c|cc}
    \multicolumn{1}{c}{} & Utiles & Perjudiciales \\
\cmidrule{2-3}    Internas & \textbf{Fortalezas} & \textbf{Debilidades} \\
    Externas & \textbf{Oportunidades} & \textbf{Amenazas} \\
    \end{tabular}%
  \label{tab:addlabel}%
\end{table}%


% Table generated by Excel2LaTeX from sheet 'Hoja1'
\begin{table}[htbp]
  \centering
  \caption{Evaluación valor, costes e ingresos}
    \begin{tabular}{p{14.285em}ccccc|cccccp{12em}}
    \textcolor[rgb]{ 1,  0,  0}{\textbf{Evaluación de la propuesta de valor}} & \multicolumn{5}{p{3.225em}}{\textcolor[rgb]{ .557,  .663,  .859}{+}} & \multicolumn{5}{p{3.225em}}{-}        & \multicolumn{1}{r}{} \\
    Nuestras propuestas de valor están en consonancia con las necesitades de los clientes & 5     & 4     & 3     & 2     & 1     & 1     & 2     & 3     & 4     & 5     & Nuestras propuestas de valor y las necesitades de los clientes no estan en consonancia \\
    Nuestras propuestas de valor tienen un potente efecto de red & 5     & 4     & 3     & 2     & 1     & 1     & 2     & 3     & 4     & 5     & Nuestras propuestas de valor no tienen un potente efecto de red \\
    Hay fuertes sinergias entre nuestros productos y servicios & 5     & 4     & 3     & 2     & 1     & 1     & 2     & 3     & 4     & 5     & No hay fuertes sinergias entre nuestros productos y servicios \\
    Nuestros clientes están muy satisfechos & 5     & 4     & 3     & 2     & 1     & 1     & 2     & 3     & 4     & 5     & Recibimos quejas frecuentes \\
    \multicolumn{1}{r}{} &       &       &       &       & \multicolumn{1}{c}{} &       &       &       &       &       & \multicolumn{1}{r}{} \\
    \textcolor[rgb]{ 1,  0,  0}{\textbf{Evaluacion costes e ingresos}} &       &       &       &       & \multicolumn{1}{c}{} &       &       &       &       &       & \multicolumn{1}{r}{} \\
    \multicolumn{1}{l}{\textbf{Evaluación de ingresos}} &       &       &       &       & \multicolumn{1}{c}{} &       &       &       &       &       & \multicolumn{1}{r}{} \\
    Tenemos márgenes elevados & 5     & 4     & 3     & 2     & 1     & 1     & 2     & 3     & 4     & 5     & Nuestros margenes son reducidos \\
    Nuestros ingresos son predecibles & 5     & 4     & 3     & 2     & 1     & 1     & 2     & 3     & 4     & 5     & Nuestros ingresos son impredecibles \\
    Tenemos fuentes de ingresos recurrentes y compras repetidas frecuentes & 5     & 4     & 3     & 2     & 1     & 1     & 2     & 3     & 4     & 5     & Nuestros ingresos son transaccionales y tenemos pocas compras repetidas \\
    Tenemos fuentes de ingreso diversificadas & 5     & 4     & 3     & 2     & 1     & 1     & 2     & 3     & 4     & 5     & Dependemos de una sola fuente de ingreso \\
    Nuestras fuentes de ingreso son sostenibles & 5     & 4     & 3     & 2     & 1     & 1     & 2     & 3     & 4     & 5     & La sostenibilidad de nuestros ingresos son cuestionables \\
    Percibimos ingresos antes de incurrir en gastos & 5     & 4     & 3     & 2     & 1     & 1     & 2     & 3     & 4     & 5     & Tenemos que incurrir en muchos gastos antes de percibir ingresos \\
    Cobramos a nuestros clientes por lo que están dispuestos a pagar & 5     & 4     & 3     & 2     & 1     & 1     & 2     & 3     & 4     & 5     & No cobramos a los clientes cosas por las que están dispuestos a pagar \\
    Nuestros mecanismos de fijación de precios incluyen todas las oportunidades de ingresos & 5     & 4     & 3     & 2     & 1     & 1     & 2     & 3     & 4     & 5     & Nuestros mecanismos de fijación dejan dinero sobre la mesa \\
    \multicolumn{1}{r}{} &       &       &       &       & \multicolumn{1}{c}{} &       &       &       &       &       & \multicolumn{1}{r}{} \\
    \multicolumn{1}{l}{\textbf{Evaluación de costes}} &       &       &       &       & \multicolumn{1}{c}{} &       &       &       &       &       & \multicolumn{1}{r}{} \\
    Nuestros costes son predecibles & 5     & 4     & 3     & 2     & 1     & 1     & 2     & 3     & 4     & 5     & Nuestros costes son impredecibles \\
    Nuestra estructura de costes se adecua perfectamente a nuestro modelo de negocio & 5     & 4     & 3     & 2     & 1     & 1     & 2     & 3     & 4     & 5     & Nuestra estructura de costes y nuestro modelo de negocio no están en consonancia \\
    Nuestras operaciones son rentables & 5     & 4     & 3     & 2     & 1     & 1     & 2     & 3     & 4     & 5     & Nuestras operaciones no son rentables \\
    Aprovechamos las economías de escala & 5     & 4     & 3     & 2     & 1     & 1     & 2     & 3     & 4     & 5     & No aprovechamos las economías de escala \\
    \end{tabular}%
  \label{tab:addlabel}%
\end{table}%


% Table generated by Excel2LaTeX from sheet 'Hoja1'
\begin{table}[htbp]
  \centering
  \caption{Add caption}
    \begin{tabular}{p{14.285em}ccccc|cccccp{12em}}
    \multicolumn{1}{l}{\textcolor[rgb]{ 1,  0,  0}{\textbf{Evaluación de infraestructura}}} &       &       &       &       & \multicolumn{1}{c}{} &       &       &       &       &       & \multicolumn{1}{r}{} \\
    \multicolumn{1}{l}{\textbf{Evaluación de recursos clave}} &       &       &       &       & \multicolumn{1}{c}{} &       &       &       &       &       & \multicolumn{1}{r}{} \\
    La competencia no puede imitar fácilmente nuestros recursos clave & 5     & 4     & 3     & 2     & 1     & 1     & 2     & 3     & 4     & 5     & Nuestros recursos clave se pueden imitar fácilmente \\
    Las necesidades de recursos son predecibles & 5     & 4     & 3     & 2     & 1     & 1     & 2     & 3     & 4     & 5     & Las necesidades de recursos no son predecibles \\
    Aplicamos los recursos clave en la cantidad adecuada y en el momento adecuado & 5     & 4     & 3     & 2     & 1     & 1     & 2     & 3     & 4     & 5     & Tenemos problemas para aplicar los recursos adecuados en el momento adecuado \\
    \multicolumn{1}{r}{} &       &       &       &       & \multicolumn{1}{c}{} &       &       &       &       &       & \multicolumn{1}{r}{} \\
    \multicolumn{1}{l}{\textbf{Evaluación de actividades clave}} &       &       &       &       & \multicolumn{1}{c}{} &       &       &       &       &       & \multicolumn{1}{r}{} \\
    Realizamos nuestras actividades clave de forma efi ciente & 5     & 4     & 3     & 2     & 1     & 1     & 2     & 3     & 4     & 5     & Realizamos nuestras actividades clave de forma inefi ciente \\
    Nuestras actividades clave son difíciles de copiar & 5     & 4     & 3     & 2     & 1     & 1     & 2     & 3     & 4     & 5     & Nuestras actividades clave son fáciles de copiar \\
    La ejecución es de alta calidad & 5     & 4     & 3     & 2     & 1     & 1     & 2     & 3     & 4     & 5     & La ejecución es de baja calidad \\
    El equilibrio entre trabajo interno y colaboración externa es ideal & 5     & 4     & 3     & 2     & 1     & 1     & 2     & 3     & 4     & 5     & Realizamos muchas o muy pocas actividades internamente \\
    \multicolumn{1}{r}{} &       &       &       &       & \multicolumn{1}{c}{} &       &       &       &       &       & \multicolumn{1}{r}{} \\
    \textbf{Evaluación de asociaciones clave} &       &       &       &       & \multicolumn{1}{c}{} &       &       &       &       &       & \multicolumn{1}{r}{} \\
    Estamos especializados y trabajamos con socios cuando es necesario & 5     & 4     & 3     & 2     & 1     & 1     & 2     & 3     & 4     & 5     & No estamos especializados ni colaboramos con socios lo sufi ciente \\
    Tenemos buenas relaciones profesionales con los socios clave & 5     & 4     & 3     & 2     & 1     & 1     & 2     & 3     & 4     & 5     & Las relaciones profesionales con los socios clave son confl ictivas \\
    \end{tabular}%
  \label{tab:addlabel}%
\end{table}%


% Table generated by Excel2LaTeX from sheet 'Hoja1'
\begin{table}[htbp]
  \centering
  \caption{Add caption}
    \begin{tabular}{lccccc|cccccl}
    \multicolumn{1}{p{14.285em}}{\textcolor[rgb]{ 1,  0,  0}{\textbf{Evaluacion de la interacción con los clientes}}} &       &       &       &       & \multicolumn{1}{c}{} &       &       &       &       &       &  \\
    \multicolumn{1}{p{14.285em}}{\textbf{Evaluación de segmento de mercado}} &       &       &       &       & \multicolumn{1}{c}{} &       &       &       &       &       &  \\
    \multicolumn{1}{p{14.285em}}{El índice de migración de clientes es bajo} & 5     & 4     & 3     & 2     & 1     & 1     & 2     & 3     & 4     & 5     & \multicolumn{1}{p{12em}}{El índice de migración de clientes es elevado} \\
    \multicolumn{1}{p{14.285em}}{La cartera de clientes está bien segmentada} & 5     & 4     & 3     & 2     & 1     & 1     & 2     & 3     & 4     & 5     & \multicolumn{1}{p{12em}}{La cartera de clientes no está segmentada} \\
    \multicolumn{1}{p{14.285em}}{Captamos nuevos clientes constantemente} & 5     & 4     & 3     & 2     & 1     & 1     & 2     & 3     & 4     & 5     & \multicolumn{1}{p{12em}}{No captamos nuevos clientes} \\
          &       &       &       &       & \multicolumn{1}{c}{} &       &       &       &       &       &  \\
    \multicolumn{1}{p{14.285em}}{\textbf{Evaluación de canales de distribución}} &       &       &       &       & \multicolumn{1}{c}{} &       &       &       &       &       &  \\
    \multicolumn{1}{p{14.285em}}{Nuestros canales son muy efi cientes} & 5     & 4     & 3     & 2     & 1     & 1     & 2     & 3     & 4     & 5     & \multicolumn{1}{p{12em}}{Nuestros canales son inefi cientes} \\
    \multicolumn{1}{p{14.285em}}{Nuestros canales son muy efi caces} & 5     & 4     & 3     & 2     & 1     & 1     & 2     & 3     & 4     & 5     & \multicolumn{1}{p{12em}}{Nuestros canales son inefi caces} \\
    \multicolumn{1}{p{14.285em}}{Los canales establecen un contacto estrecho con los clientes} & 5     & 4     & 3     & 2     & 1     & 1     & 2     & 3     & 4     & 5     & \multicolumn{1}{p{12em}}{Los canales no establecen un contacto adecuado con los clientes potenciales} \\
    \multicolumn{1}{p{14.285em}}{Los clientes pueden acceder fácilmente a nuestros canales} & 5     & 4     & 3     & 2     & 1     & 1     & 2     & 3     & 4     & 5     & \multicolumn{1}{p{12em}}{Nuestros canales no llegan a los clientes potenciales} \\
    \multicolumn{1}{p{14.285em}}{Los canales están perfectamente integrados} & 5     & 4     & 3     & 2     & 1     & 1     & 2     & 3     & 4     & 5     & \multicolumn{1}{p{12em}}{Los canales no están bien integrados} \\
    \multicolumn{1}{p{14.285em}}{Los canales proporcionan economías de campo (expansión)} & 5     & 4     & 3     & 2     & 1     & 1     & 2     & 3     & 4     & 5     & \multicolumn{1}{p{12em}}{Los canales no proporcionan economías de campo} \\
    \multicolumn{1}{p{14.285em}}{Los canales se adecuan a los segmentos de mercado} & 5     & 4     & 3     & 2     & 1     & 1     & 2     & 3     & 4     & 5     & \multicolumn{1}{p{12em}}{Los canales no se adecuan a los segmentos de mercado} \\
          &       &       &       &       & \multicolumn{1}{c}{} &       &       &       &       &       &  \\
    \multicolumn{1}{p{14.285em}}{\textbf{Evaluación de relación con nuestros clientes}} &       &       &       &       & \multicolumn{1}{c}{} &       &       &       &       &       &  \\
    Estrecha relación con los clientes & 5     & 4     & 3     & 2     & 1     & 1     & 2     & 3     & 4     & 5     & Poca relación con los clientes \\
    \multicolumn{1}{p{14.285em}}{La calidad de la relación está en consonancia con los segmentos de mercado} & 5     & 4     & 3     & 2     & 1     & 1     & 2     & 3     & 4     & 5     & \multicolumn{1}{p{12em}}{La calidad de la relación no está en consonancia con los segmentos de mercado} \\
    \multicolumn{1}{p{14.285em}}{Las relaciones vinculan a los clientes mediante un elevado coste de cambio} & 5     & 4     & 3     & 2     & 1     & 1     & 2     & 3     & 4     & 5     & El coste de cambio es bajo \\
    Nuestra marca es fuerte & 5     & 4     & 3     & 2     & 1     & 1     & 2     & 3     & 4     & 5     & Nuestra marca es débil \\
    \end{tabular}%
  \label{tab:addlabel}%
\end{table}%

\section{Evaluación de amenazas}
% Table generated by Excel2LaTeX from sheet 'Evaluación de amenazas'
\begin{table}[htbp]
  \centering
  \caption{Add caption}
    \begin{tabular}{p{26.145em}ccccc}
    \textcolor[rgb]{ 1,  0,  0}{\textbf{Amenazas para la propuesta de valor}} &       &       &       &       &  \\
    ¿Hay productos y servicios sustitutos disponibles? & 1     & 2     & 3     & 4     & 5 \\
    ¿La competencia amenaza con ofrecer un precio mejor o más valor? & 1     & 2     & 3     & 4     & 5 \\
    \multicolumn{1}{l}{} &       &       &       &       &  \\
    \textcolor[rgb]{ 1,  0,  0}{\textbf{Amenazas para los costes/ingresos}} &       &       &       &       &  \\
    \textbf{Amenazas de ingresos} &       &       &       &       &  \\
    ¿La competencia pone en peligro nuestros márgenes de benefi cios? ¿Y la tecnología? & 1     & 2     & 3     & 4     & 5 \\
    ¿Dependemos excesivamente de una o varias fuentes de ingresos? & 1     & 2     & 3     & 4     & 5 \\
    ¿Qué fuentes de ingresos podrían desaparecer en el futuro? & 1     & 2     & 3     & 4     & 5 \\
    \multicolumn{1}{l}{} &       &       &       &       &  \\
    \textbf{Amenzas de costes} &       &       &       &       &  \\
    ¿Qué costes amenazan con volverse impredecibles? & 1     & 2     & 3     & 4     & 5 \\
    ¿Qué costes amenazan con aumentar más rápido que los ingresos que generan? & 1     & 2     & 3     & 4     & 5 \\
    \multicolumn{1}{l}{} &       &       &       &       &  \\
    \textcolor[rgb]{ 1,  0,  0}{Amenazas para la infraestructura} &       &       &       &       &  \\
    \textbf{Amenazas para recursos clave} &       &       &       &       &  \\
    ¿Podríamos hacer frente a una disrupción en el suministro de determinados recursos? & 1     & 2     & 3     & 4     & 5 \\
    ¿La calidad de nuestros recursos se ve amenazada de alguna manera? & 1     & 2     & 3     & 4     & 5 \\
    \multicolumn{1}{l}{} &       &       &       &       &  \\
    \textbf{Amenzas para Actividades clave} &       &       &       &       &  \\
    ¿Qué actividades clave podrían interrumpirse? & 1     & 2     & 3     & 4     & 5 \\
    ¿La calidad de nuestras actividades se ve amenazada de alguna manera? & 1     & 2     & 3     & 4     & 5 \\
    \multicolumn{1}{l}{} &       &       &       &       &  \\
    \textbf{Amenazas para relaciones clave} &       &       &       &       &  \\
    ¿Corremos el peligro de perder clientes? & 1     & 2     & 3     & 4     & 5 \\
    ¿Nuestros socios podrían colaborar con la competencia? & 1     & 2     & 3     & 4     & 5 \\
    ¿Dependemos demasiado de determinados socios? & 1     & 2     & 3     & 4     & 5 \\
    \multicolumn{1}{l}{} &       &       &       &       &  \\
    \textcolor[rgb]{ 1,  0,  0}{Amenazas para la interacción con los clientes} &       &       &       &       &  \\
    \textbf{Amenazas para el segmento de merado} &       &       &       &       &  \\
    ¿Nuestro mercado podría saturarse en breve? & 1     & 2     & 3     & 4     & 5 \\
    ¿La competencia pone en peligro nuestra cuota de mercado? & 1     & 2     & 3     & 4     & 5 \\
    ¿Qué probabilidades hay de que nuestros clientes se vayan? & 1     & 2     & 3     & 4     & 5 \\
    ¿A qué velocidad aumentará la competencia en nuestro mercado? & 1     & 2     & 3     & 4     & 5 \\
    \multicolumn{1}{l}{} &       &       &       &       &  \\
    \textbf{Amenazas para canales de distribución} &       &       &       &       &  \\
    ¿La competencia pone en peligro nuestros canales? & 1     & 2     & 3     & 4     & 5 \\
    ¿Es posible que los clientes dejen de utilizar nuestros canales? & 1     & 2     & 3     & 4     & 5 \\
    \multicolumn{1}{l}{} &       &       &       &       &  \\
    \textbf{Amenazas para las relaciones con clientes} &       &       &       &       &  \\
    ¿Alguna de las relaciones con clientes corre el peligro de deteriorarse? & 1     & 2     & 3     & 4     & 5 \\
    \end{tabular}%
  \label{tab:addlabel}%
\end{table}%

\section{Evaluación de oportunidades}
% Table generated by Excel2LaTeX from sheet 'Evaluación de oportunidades'
\begin{table}[htbp]
  \centering
  \caption{Add caption}
    \begin{tabular}{p{27.93em}rrrrr}
    \textcolor[rgb]{ 1,  0,  0}{Oportunidades de la propuesta de valor} &       &       &       &       &  \\
    ¿Podríamos generar ingresos recurrentes si convertimos nuestros productos en servicios? & 1     & 2     & 3     & 4     & 5 \\
    ¿Podríamos mejorar la integración de nuestros productos o servicios? & 1     & 2     & 3     & 4     & 5 \\
    ¿Qué otras necesidades de los clientes podríamos satisfacer? & 1     & 2     & 3     & 4     & 5 \\
    ¿Qué complementos o ampliaciones admite nuestra propuesta de valor? & 1     & 2     & 3     & 4     & 5 \\
    ¿Qué tareas adicionales podríamos realizar para nuestros clientes? & 1     & 2     & 3     & 4     & 5 \\
    \multicolumn{1}{l}{} &       &       &       &       &  \\
    \textcolor[rgb]{ 1,  0,  0}{Oportunidades de costes7 ingresos} &       &       &       &       &  \\
    \textbf{Oportunidades de ingresos} &       &       &       &       &  \\
    ¿Podemos sustituir los ingresos por transacción por ingresos recurrentes? & 1     & 2     & 3     & 4     & 5 \\
    ¿Por qué otros elementos estarían dispuestos a pagar los clientes? & 1     & 2     & 3     & 4     & 5 \\
    ¿Tenemos oportunidades de venta cruzada con los socios o dentro de la empresa? & 1     & 2     & 3     & 4     & 5 \\
    ¿Qué fuentes de ingresos podríamos añadir o crear? & 1     & 2     & 3     & 4     & 5 \\
    ¿Podemos elevar los precios? & 1     & 2     & 3     & 4     & 5 \\
    \multicolumn{1}{l}{} &       &       &       &       &  \\
    \textbf{Oportunidades de costes} &       &       &       &       &  \\
    ¿Qué costes podemos reducir? & 1     & 2     & 3     & 4     & 5 \\
    \multicolumn{1}{l}{} &       &       &       &       &  \\
    \textcolor[rgb]{ 1,  0,  0}{Oportunidades de infraestructura} &       &       &       &       &  \\
    \textbf{Oportunidades de recursos clave} &       &       &       &       &  \\
    ¿Podríamos utilizar recursos más baratos para obtener los mismos resultados? & 1     & 2     & 3     & 4     & 5 \\
    ¿Qué recursos clave podríamos adquirir a los socios? & 1     & 2     & 3     & 4     & 5 \\
    ¿Qué recursos clave están poco explotados? & 1     & 2     & 3     & 4     & 5 \\
    ¿Tenemos objetos de propiedad intelectual sin utilizar que podrían ser valiosos para terceros? & 1     & 2     & 3     & 4     & 5 \\
    \multicolumn{1}{l}{} &       &       &       &       &  \\
    \textbf{Oportunidades de Actividades Clave} &       &       &       &       &  \\
    ¿Podríamos estandarizar algunas actividades clave? & 1     & 2     & 3     & 4     & 5 \\
    ¿Cómo podríamos mejorar la efi ciencia en general? & 1     & 2     & 3     & 4     & 5 \\
    ¿El soporte de TI podría aumentar la efi ciencia? & 1     & 2     & 3     & 4     & 5 \\
    \multicolumn{1}{l}{} &       &       &       &       &  \\
    \textbf{Oportunidades de Asociaciones clave} &       &       &       &       &  \\
    ¿Hay oportunidades de externalización? & 1     & 2     & 3     & 4     & 5 \\
    ¿Una mayor colaboración con los socios nos permitiría concentrarnos en nuestra actividad empresarial principal? & 1     & 2     & 3     & 4     & 5 \\
    ¿Hay oportunidades de venta cruzada con los socios? & 1     & 2     & 3     & 4     & 5 \\
    ¿Los canales de socios podrían ayudarnos a mejorar el contacto con los clientes? & 1     & 2     & 3     & 4     & 5 \\
    ¿Los socios podrían complementar nuestra propuesta de valor? & 1     & 2     & 3     & 4     & 5 \\
    \end{tabular}%
  \label{tab:addlabel}%
\end{table}%
% Table generated by Excel2LaTeX from sheet 'Evaluación de oportunidades'
\begin{table}[htbp]
  \centering
  \caption{Add caption}
    \begin{tabular}{p{27.93em}rrrrr}
    \textcolor[rgb]{ 1,  0,  0}{\textbf{Oportunidades de interacción con clientes}} &       &       &       &       &  \\
    \textbf{Oportunidades de segmento de mercado} &       &       &       &       &  \\
    ¿Cómo podríamos benefi ciarnos de un mercado creciente? & 1     & 2     & 3     & 4     & 5 \\
    ¿Podríamos atender nuevos segmentos de mercado? & 1     & 2     & 3     & 4     & 5 \\
    ¿Podríamos atender mejor a nuestros clientes con una segmentación más depurada? & 1     & 2     & 3     & 4     & 5 \\
    \multicolumn{1}{l}{} &       &       &       &       &  \\
    \textbf{Oportunidades de canales de distribución} &       &       &       &       &  \\
    ¿Cómo podríamos mejorar la efi ciencia o efectividad del canal? & 1     & 2     & 3     & 4     & 5 \\
    ¿Podríamos mejorar la integración de nuestros canales? & 1     & 2     & 3     & 4     & 5 \\
    ¿Podríamos buscar nuevos canales de socios complementarios? & 1     & 2     & 3     & 4     & 5 \\
    ¿Podríamos aumentar el margen si servimos a los clientes directamente? & 1     & 2     & 3     & 4     & 5 \\
    ¿Podríamos acompasar mejor los canales con los segmentos de mercado? & 1     & 2     & 3     & 4     & 5 \\
    \multicolumn{1}{l}{} &       &       &       &       &  \\
    \textbf{Oportuniades de relaciones con clientes} &       &       &       &       &  \\
    ¿Se puede mejorar el seguimiento de los clientes? & 1     & 2     & 3     & 4     & 5 \\
    ¿Podríamos estrechar las relaciones con los clientes? & 1     & 2     & 3     & 4     & 5 \\
    ¿Podríamos aumentar la personalización? & 1     & 2     & 3     & 4     & 5 \\
    ¿Cómo podríamos aumentar los costes de cambio? & 1     & 2     & 3     & 4     & 5 \\
    ¿Hemos identifi cado y eliminado los clientes que no son rentables? Si no es así, ¿por qué no? & 1     & 2     & 3     & 4     & 5 \\
    ¿Tenemos que automatizar algunas relaciones? & 1     & 2     & 3     & 4     & 5 \\
    \end{tabular}%
  \label{tab:addlabel}%
\end{table}%

\section{Uso de los resultados del análisis DAFO para diseñar nuevas opciones de modelo de negocio}
Un análisis DAFO estructurado del modelo de negocio genera dos resultados: 
\begin{itemize}
\item Ofrece una instantánea del estado actual (puntos débiles y fuertes)
\item Sugiere algunas trayectorias para el futuro (oportunidades y amenazas) 
\end{itemize}
Esta valiosa información puede ayudarte a diseñar nuevas opciones de modelo de negocio para la empresa.
El análisis DAFO es una parte importante del proceso del diseño de prototipos de modelo de negocio (pag 160) y con suerte un nuevo modelo de negocio que podrás aplicar en el futuro.
\textcolor{red}{Gráfico pag 225}
\chapter{Perspectivas de los modelos de negocio sobre la estrategia del océano azul}
La estrategia del océano azul es un método potente para evaluar las propuestas de valor y los modelos de negocio, así como para explorar nuevos segmentos de mercado.
\\ En pocas palabras, la estrategia del océano azul consiste en crear industrias completamente nuevas a través de la diferenciación fundamental, en vez de competir en sectores existentes modificando los modelos establecidos. 
En lugar de superar la competencia en cuanto a rendimiento, abogan por la creación de espacios de mercados nuevos y desatendidos mediante lo que llaman INNOVACIÓN EN VALOR
\section{Esquema de las 4 acciones}
Se platea cuatro preguntas clave que desafían la lógica estratégica de un sector y el modelo de negocio establecido
\begin{enumerate}
\item\textcolor{blue}{De las variables que el sector da por sentadas ¿   Cuales se deben eliminar ?}
\item\textcolor{blue}{¿ Que variables se deben reducir muy por debajo de la norma del sector ?}
\item\textcolor{blue}{¿ Que variables se deben aumentar muy por encima de la norma del sector ?}
\item\textcolor{blue}{¿ Que variables, que el sector no haya ofrecido nunca, se deben crear ?}
\end{enumerate}
La idea principal es aumentar el valor y reducir los costes.\\
\textcolor{red}{mas detalle pagina 227}
% Table generated by Excel2LaTeX from sheet 'Hoja4'
\begin{table}[htbp]
  \centering
  \caption{Add caption}
    \begin{tabular}{p{17.285em}|p{16.715em}}
    \textbf{Eliminar} & \textbf{Aumnetar} \\
    \midrule
    De las variables con una gran competencia en el sector, Cuales puedes eliminar & ¿Qué variables se deben aumentar muy por encima de la norma del sector? \\
    \textbf{Reducir} & \textbf{Crear} \\
    \midrule
    ¿qué variables se deben reducir muy por debajo de la norma del sector? & ¿qué variables, que el sector no haya ofrecido nunca, se deben crear? \\
    \end{tabular}%
  \label{tab:addlabel}%
\end{table}%
\section{Combinación del esquema de la estrategia del océano azul con el lienzo de modelo de negocio}
\subsection{Lienzo de modelo de negocio}
El lienzo de modelo de negocio se divide en dos partes:
\begin{itemize}
\item A la derecha del lienzo centrada en el valor y los clientes
\item A la izquierda del lienzo, basada en los costes y la infraestructura pag. (49)
\end{itemize}
La modificación de los elementos situados a la derecha tienen ciertas implicaciones para los elementos del lado izquierdo. Cualquier reducción de elementos de canales o relaciones con clientes tiene una repercusión sobre los módulos recursos, actividades, costes, etc. 
\subsubsection{Innovación de valor}
La estrategia del océano azul consiste en aumentar el valor y reducir los costes de forma simultánea. 
Para ellos es necesario Identificar los elementos de la propuesta de valor que se pueden:
\begin{itemize}
\item Eliminar.
\item Reducir.
\item Aumentar.
\item Crear desde cero.
\end{itemize} 
Los objetivos son:
\begin{itemize}
\item Reducir los costes mediante la disminución o la eliminación de los componentes o servicios menos valiosos.
\item La mejora o creación de componentes y servicios de alto valor que no aumenten demasiado la base de costes.
\end{itemize}
\subsection{Combinación de ambos métodos}
La combinación de la estrategia del océano azul y el lienzo de modelo de negocio te permite analizar de forma sistemática el grado de innovación de un modelo de negocio.\\
\textcolor{red}{Ejemplo pag 225}
\section{Análisis de l lienzo con el esquema de las cuatro acciones}
\subsection{Estudio del impacto sobre los costes}
\begin{enumerate}
\item Identifica los elementos mas caros de la infraestructura y comprueba que sucede si los eliminas o reduces
\begin{itemize}
\item\textcolor{blue}{¿ Que elementos de valor desaparecen ?}
\item\textcolor{blue}{¿ Que tendrías que hacer para compensar su ausencia ?}
\end{itemize}
\item Identifica las inversiones en infraestructura que podrías hacer y analiza el valor que añadirían.
\begin{itemize}
\item\textcolor{blue}{¿ Que actividades, recursos y asociaciones tienen un coste mas elevado ?}
\item\textcolor{blue}{¿ Que sucede si reduces o eliminas alguno de estos factores de coste ?}
\item\textcolor{blue}{¿ Como podrías reemplazar, con elementos mas económicos, el valor perdido al reducir o eliminar los recursos actividades o asociaciones clave ?}
\item\textcolor{blue}{¿ Que valor crearían las nuevas inversiones planificadas ?}
\end{itemize} 
\end{enumerate}
\subsection{Estudio del impacto sobre la propuesta de valor}
Utiliza las cuatro preguntas del esquema de las cuatro
acciones parar iniciar la transformación de la propuesta de
valor.
Al mismo tiempo, valora su impacto en los costes y
averigua qué elementos deberías o podrías cambiar con
relación al valor (canales, relaciones, fuentes de ingresos y
segmentos de mercado)
\begin{itemize}
\item\textcolor{blue}{¿ Que componentes o servicios menos valiosos podrías eliminar o reducir ?}
\item\textcolor{blue}{¿ Que componentes o servicios se podrían mejorar o crear desde cero para ofrecer una nueva experiencia de cliente ?}
\item\textcolor{blue}{¿ Como afectaría los cambios de la propuesta de valor a los costes ?}
\item\textcolor{blue}{¿ Como afectaría los cambios de la propuesta de valor al cliente ?}
\end{itemize}
\subsection{Estudio del impacto sobre los clientes}
Aplica las cuatro preguntas del esquema a cada
uno de los módulos del modelo de negocio relativos al
cliente: canales, relaciones y fuentes de ingresos. Analiza
qué sucede en el ámbito de los costes si eliminas,
reduces, aumentas o creas elementos de valor.
\begin{itemize}
\item\textcolor{blue}{¿ En que nuevos segmentos de mercado podrías centrarte ?}
\item\textcolor{blue}{¿ Que segmentos podrías reducir o eliminar ?}
\item\textcolor{blue}{¿ Que servicios necesitan realmente los nuevos segmentos de mercado ?}
\item\textcolor{blue}{¿ Que canales de contacto prefieren estos clientes y que tipo de relación esperan ?}
\item\textcolor{blue}{¿ Como afectaría a los costes la atención de nuevos segmentos de mercado ?}
\end{itemize}
\chapter{Gestión de varios modelos de negocio}
Hay visionarios, revolucionarios y provocadores.
Se propone un esquema dos variables para determinar cual es la mejor forma de gestionar los modelos de negocio nuevos y tradicionales de forma simultánea. 
\begin{itemize}
\item La primera variable refleja la gravedad del conflicto entre los modelos
\item La segunda variable presenta la similitud estratégica.
\end{itemize}
Por otro lado el riesgo es otra variable a tomar en cuenta \\
\textcolor{blue}{¿ Es muy elevado el riesgo de que el nuevo modelo afecte negativamente al modelo existente en cuanto a imagen de marca, beneficio o responsabilidad legal ?}
Se debe considerar la posibilidad de una integración por fases o una segregación gradual de los modelos de negocio.
\textcolor{red}{Revisar paginas 233 al 239}
\part{Proceso}
\chapter{Proceso de diseño de modelos de negocio}
Se ligan los conceptos y las herramientas del libro con el fin de simplificar la tarea de configuración y puesta en marcha de una iniciativa de diseño de modelo de negocio.\\
El esfuerzo de innovación en modelo de negocio nace de una de las cuatro iniciativas:
\begin{itemize}
\item La crisis del modelo de negocio existente (en algunos casos, una experiencia próxima a la muerte)
\item El ajuste, la mejora o la defensa del modelo existente con el fin de adaptarlo a un entorno cambiante.
\item La comercialización de nuevas tecnologías, productos y servicios.
\item La preparación para el futuro mediante la búsqueda y la comprobación de modelos de negocio completamente nuevos que podrían reemplazar a los existentes.
\end{itemize}
\section{Innovación y diseño de modelos de negocio}
\paragraph{Satisfacción del mercado}
Satisfacer una necesidad desatendida del mercado.
\paragraph{Comercialización}
Comercializar una tecnología, producto o servicio nuevo, o explotar una propiedad intelectual existente.
\paragraph{Mejora del mercado}
Mejorar o desbaratar un mercado existente.
\paragraph{Creación de un mercado}
Crear un tipo de negocio totalmente nuevo.
\subsection{Retos}
\begin{itemize}
\item Encontrar el modelo adecuado.
\item Comprobar el modelo antes de su aplicación en el mundo real.
\item Persuadir el mercado para que adopte el nuevo modelo.
\item Adaptar el modelo constantemente en función de la respuesta del mercado.
\item Gestionar los puntos de incertidumbre.
\end{itemize}
\section{Factores específicos de las organizaciones consolidadas}
\paragraph{Reactivo}
Nace a raíz de una crisis con el modelo de negocio existente.
\paragraph{Adaptativo}
Ajuste, mejora o defensa del modelo de negocio existente.
\paragraph{Expansionista}
Lanzamiento de una tecnología, producto o servicio nuevo.
\paragraph{Proactivo/exploratorio}
Preparación para el futuro.
\subsection{Retos}
\begin{itemize}
\item Generar mercado para nuevos modelos.
\item Coordinar los modelos antiguos y nuevos.
\item Gestionar los interese creados.
\item Centrarse en los resultados a largo plazo.
\end{itemize}
\chapter{Actitud de diseño}
La coincidencia no suele desempeñar ningún papel en la innovación en modelos de negocio, aunque la innovación tampoco es exclusiva del genio creativo.
El reto que plantea la innovación en modelos de negocio es su falta de orden e imprevisibilidad.
\\ Para innovar es necesario:
\begin{itemize}
\item  Tener capacidad para gestionar la ambigüedad 
\item Tener capacidad de gestionar los puntos de incertidumbre.
\end{itemize}
Es mas probable obtener un modelo de negocio sólido si se lleva a cabo un proceso de:
\section{Exploración}
Consiste en una una combinación oportunista y desordenada de estudios de mercado de negocios y generación de ideas. 
\section{exploración y creación de prototipos}
La exploración y creación de prototipos da muchas ideas de posibilidades.
\\ tres pasos a seguir:
\begin{itemize}
\item Investigar y comprender.
\item Diseñar prototipos de modelos de negocio.
\item Aplicar el diseño de modelo de negocios.
\end{itemize}
\textcolor{red}{revisar gráfico pagina 247}
\chapter{Cinco fases}
Debemos aclarar que en el entorno actual, es preferible asumir que la mayoría de los modelos de negocio, incluso los que triunfan, tienen una vida útil corta. La gestión de evolución del modelo revelará que componentes siguen siendo relevantes y cuáles se han quedado obsoletos.

\textcolor{red}{Revisar pagina 249}
\section{Movilización}
Preparación de un proyecto de diseño de modelo de negocio de éxito
\subsection{Actividades}
\begin{enumerate}
\item Definición de los objetivos del proyecto.
\begin{itemize}
\item Incluir tareas como el establecimiento de las bases, el ámbito del proyecto y los objetivos principales.
\end{itemize}
\item Comprobación de las ideas preliminares para el negocio.
\item planificación.
\begin{itemize}
\item Abarca las primeras fases de un proyecto de diseño de negocio:
\begin{itemize}
\item Movilización.
\item Comprensión.
\item Diseño
\end{itemize}
\end{itemize}
\item Formación de un equipo.\\
La formación de un equipo de proyecto son tareas vitales. Es conveniente reunir a personas con amplia experiencia en:
\begin{itemize}
\item Gestión.
\item En el sector.
\item Ideas nuevas.
\item redes personales.
\item Profundo compromiso con la innovación en modelo de negocio.
\end{itemize}
\end{enumerate}
\subsection{Factores clave para el éxito}
\begin{enumerate}
\item Personas, experiencia y conocimiento adecuados.
\end{enumerate}
\subsection{Principales peligros}
\begin{enumerate}
\item Sobrevaloración de las ideas iniciales.
\end{enumerate}
Para comenzar durante la fase de movilización es recomendable realizar algunas comprobaciones preliminares de la idea básica. \\
se podría utilizar el método KILL / THRILL, que consiste en hacer un brainstorming de porque la idea funciona y otro el porque no funcionaria.
\subsection{Trabajo desde la perspectiva de la empresa establecida}
\subsubsection{Legitimidad del proyecto}
Cuando la empresa esta establecida, el modelo de negocio afecta a diferentes unidades empresariales, por lo que es recomendable que la alta dirección consiga la cooperación de cada unidad. 
\subsubsection{Gestión de los intereses creados}
Un nuevo negocio puede representar la amenaza para algunos.
\subsubsection{Equipo interdisciplinar}
El equipo ideal esta formado por diferentes unidades de la empresa con diferentes funciones.
\subsubsection{Orientación de los responsables de las decisiones}
Se debe capacitar a los responsables de toma de decisiones.

\section{Compresión}
Investigación y análisis de los elementos necesarios para el diseño del modelo de negocio.
\subsection{Actividades}
\begin{enumerate}
\item Análisis del entorno.
\item Estudio de los clientes potenciales.
\item Entrevista con expertos.
\item Estudio de los intentos anteriores (Ejemplos de fracasos y sus motivos)
\item Recopilación de ideas y opiniones.
\end{enumerate}
\subsection{Factores clave para el éxito}
\begin{enumerate}
\item Conocimiento exhaustivo de los posible mercados.
\item Superación de las barreras tradicionales que definen los mercados objetivos.
\end{enumerate}
\subsection{Principales peligros}
\begin{enumerate}
\item Alejamiento de los objetivos a causa de una investigación excesiva.
\item Investigación sesgada debido a un vínculo previo con una idea de negocio.
\end{enumerate}

La parálisis del análisis también se puede evitar con la creación de prototipos de modelo de negocio al principio del proceso. Este modelo te da la ventaja de obtener feedback inmediatamente. \textcolor {red}{Revisar página 160} \\
 La Investigación, la compresión y el diseño van de la mano.\\
Un área que requiere especial atención durante el proceso de investigación es la profundización en el conocimiento del cliente. El mapa de empatía con el cliente puede ser una buena opción \textcolor {red}{revisar pagina 131}
\\\\
La clave de éxito en esta fase es cuestionarse las premisas del sector y los patrones de modelo de negocio establecidos.
\subsection{Trabajo desde la perspectiva de la empresa establecida}
\subsubsection{Trazado/Evaluación de los modelos de negocio existentes}
Es adecuado la participación de empleados de la empresa para evaluar el modelo de negocio actual recopilando ideas y opiniones de los puntos débiles y fuertes. Así daremos paso a nuevos modelos de negocio.
\subsubsection{Desafío del statu quo}
Lo mas complicado es atreverse a mirar mas allá del modelo de negocio. ya que el statu quo suele ser el éxito en el pasado y, como tal, está muy arraigado en la cultura empresarial.
\subsubsection{Búsqueda fuera de la cartera de clientes actual}
Si tu objetivo es encontrar nuevos modelos de negocio lucrativos, es importante que no te limites a la cartera de clientes actual, la fuente de beneficios del futuro puede estar en cualquier sitio.
\subsubsection{Demostración del progreso}
Comenta o muestra un esbozo de modelo de negocio basado en los resultados de la investigación para demostrar tu progreso y así no demostrará una falta de productividad a la alta dirección.
\section{Diseño}
Adaptación y modificación del modelo de negocio según la respuesta del mercado.
\subsection{Actividades}
\begin{enumerate}
\item Sesión de brainstorming.
\item Creación de prototipos.
\item Pruebas.
\item Selección.
\end{enumerate}
\subsection{Factores clave para el éxito}
\begin{enumerate}
\item Colaboración con personas de toda la empresa.
\item Capacidad para ver mas allá del statu quo.
\item Exploración de varias ideas de modelo de negocio.
\end{enumerate}
\subsubsection{Principales peligros}
\begin{enumerate}
\item Atenuación o rechazo de las ideas atrevidas.
\item Enamorarse de las ideas demasiado rápido.
\end{enumerate}
\textbf{Los principales desafíos de la fase de diseño son:}
\begin{itemize}
\item la generación y adopción de modelos nuevos y atrevidos.
\item Adoptar una actitud de diseño orientada al análisis.
\item Intenta no enamorarte de las ideas demasiado rápido.
\item Reflexiona sobre las diferentes opciones de modelos de negocio antes de elegir el modelo que quieres aplicar.
\item Juega con varios modelos de asociación.
\item Busca fuentes de ingreso alternativas y estudia el valor de diversos canales de distribución.
\item experimenta con varios patrones de modelo de negocio para descubrir y analizar nuevas posibilidades. \textcolor{red}{Revisar página 52}
\end{itemize} 
\subsection{Trabajo desde la perspectiva de la empresa establecida}
\subsubsection{Evitar suavizar las ideas atrevidas.}
Las empresas establecidas tienden a aplacar las ideas de modelo de negocio atrevidas. En este caso, el reto es defender su osadía y al mismo tiempo asegurarse de que no se encuentren con grandes obstáculos.
\subsubsection{Diseño colaborativo}
También puedes aumentar las posibilidades de que se adopten ideas atrevidas, y posteriormente se apliquen.
\subsubsection{Convivencia de lo antiguo con lo nuevo}
Una de las cuestiones que plantea el diseño es si los modelos de negocio antiguos y nuevos se deberían separar o integrar en un mismo modelo. \textcolor{red}{Revisar pagina 232}
\subsubsection{No te centres en los resultados a corto plazo}
Evita centrarte en ideas que tengan un elevado potencial para generar ingresos el primer año.
La exploración de nuevos modelos de negocio requiere una perspectiva a largo plazo.
\section{Aplicación}
Aplicación efectiva del prototipo de modelo de negocio.
\subsection{Actividades}
\begin{enumerate}
\item Comunicación e implicación.
\item Ejecución.
\end{enumerate}
\subsection{Factores clave para el éxito}
\begin{enumerate}
\item Aplicación de las buenas prácticas.
\item Capacidad y voluntad para adaptar el modelo de negocio con rapidez.
\end{enumerate}
\subsection{Principales peligros}
\begin{enumerate}
\item Aceleración débil o ausente.
\end{enumerate}
generación de modelo de negocio se centra en:
\begin{itemize}
\item Compresión.
\item Desarrollo de modelo de negocios innovadores.
\end{itemize}
Una vez que se tenga el diseño final del modelo de negocio, es el momento de ponerlo en marcha. Para ello,  
\begin{itemize}
\item tendrás que definir todos los proyectos relacionados.
\item Especificar los objetivos.
\item Organizar la estructura legal.
\item Preparar el presupuesto.
\item Planificación detallada.
\item Redactar un documento de gestión de proyectos.
\begin{itemize}
\item Prestar atención a los puntos de incertidumbre.
\item Comparar las previsiones de riesgos y recompensas con los resultados reales.
\item Adoptar mecanismos que te permitan adaptar el modelo de negocio rápidamente en función de la respuesta del mercado.
\end{itemize}
\end{itemize}
\subsection{Trabajo desde la perspectiva de la empresa establecida}
\subsubsection{Gestión proactiva de las barreras}
La participación interdisciplinaria y activa te permite abordar directamente las cuestiones con el nuevo modelo de negocio antes de planificar su acción.
\subsubsection{Patrocinio del proyecto}
un elemento fundamental para el éxito es el apoyo de su patrocinador.
\subsubsection{Convivencia de los modelos antiguos y nuevos}
Es importante crear la estructura organizativa adecuada para el nueov modelo de negocio \textcolor{red}{Revisar Gestión de varios modelos de negocio pagina 232}
\subsubsection{Campaña de comunicación}
Para anunciar el nuevo modelo de negocio, debes realizar campaña de comunicación interna que tenga mucha visibilidad y utilice varios canales.
\section{Gestión}
Adaptación y modificación del modelo de negocio según la reacción del mercado.
\subsection{Actividades}
\begin{enumerate}
\item Análisis del entorno.
\item Evaluación constante del modelo de negocios.
\item Rejuvenecimiento o replanteamiento del modelo.
\item Gestión de las sinergias o conflictos entre modelos.
\end{enumerate}
\subsection{Factores clave para el éxito}
\begin{enumerate}
\item Perspectivas a largo plazo.
\item Proactividad.
\item Control de modelos de negocio.
\end{enumerate}
\subsection{Principales peligros}
Convertirse en una víctima del éxito no adaptarse.
\subsection{Trabajo desde la perspectiva de la empresa establecida}
\subsubsection{Control de modelos de negocio}
Contempla la posibilidad de crear un grupo de responsables del control del modelo de negocio que te ayude a mejorar su gestión en la empresa. 
\subsubsection{Gestión de sinergias y conflictos}
Coordinación de los modelos de negocio de forma que se pudiesen aprovechar las sinergias y evitar el gestionar conflictos.
\subsubsection{Cartera de modelos de negocio}
Una empresa establecida y exitosa deberá gestionar proactivamente una cartera de modelos de negocio.
\subsubsection{Mentalidad de un principiante}
Mantener la mentalidad de principiante contribuye a que no nos convirtamos en victimas de nuestros éxitos
\chapter{Visión general}
\section{Modelos de negocios sin ánimo de lucro}
Cualquier organización que cree y proporcione valor debe generar los ingresos suficientes para cubrir gastos, y por consiguiente tiene un modelo de negocio.\\
Se distingue dos categorías de modelos de negocio:
\begin{enumerate}
\item Modelos de negocio financiado por terceros.
\item Modelos de triple (costes ambientales, sociales y económicos) balance con una importante misión ecológica o social.
\end{enumerate}
\subsubsection{Modelos negociados por terceros}
Puede ser mas difícil crear valor para donantes que para destinatarios. \textcolor{red}{Revisar pagina 265}
\subsubsection{Modelos de negocios de triple balance}

El modelo pretende minimizar el impacto social y ambiental negativo para así maximizar el positivo.
\section{Diseño de modelos de negocio asistido por ordenador}






\end{document}
