\documentclass[10pt]{book} 
\usepackage[text=17cm,left=2.5cm,right=2.5cm, headsep=20pt, top=2.5cm, bottom = 2cm,letterpaper,showframe = false]{geometry} %configuración página
\usepackage{latexsym,amsmath,amssymb,amsfonts} %(símbolos de la AMS).7
\parindent = 0cm  %sangria
\usepackage{lmodern} % tipos de letras
\usepackage[T1]{fontenc} %acentos en español
\usepackage[spanish]{babel} %español capitulos y secciones
\usepackage{graphicx} %gráficos y figuras.
\pagestyle{empty}%elimina numeración de página

%-----------------------------------------%

\usepackage{titlesec} %formato de títulos
\usepackage[backref=page]{hyperref} %hipervinculos
\usepackage{multicol} %columnas
\usepackage{wrapfig} %Figuras al lado de texto
\usepackage{tikz}\usetikzlibrary{shapes.misc}
\usepackage{tikz,tkz-tab} % diseño de cajas
\usetikzlibrary{matrix,arrows, positioning,shadows,shadings,backgrounds,
calc, shapes, tikzmark}
\usepackage{tcolorbox, empheq} %cajas
\tcbuselibrary{skins,breakable,listings,theorems}
\usepackage{xparse} % cajas y entornos para teoremas etc
\usepackage{pstricks} %cambiar color de letra
\usepackage[Bjornstrup]{fncychap}%diseño de portada de capitulos
\usepackage{rotating}
\usepackage{enumerate}
\usepackage{booktabs}
\usepackage{synttree} 
\usepackage{chngcntr}
\usepackage{venndiagram}
\usepackage[all]{xy}%flechas
\counterwithout{footnote}{chapter}
\usepackage{xcolor}
\usetikzlibrary{datavisualization.formats.functions}
\usepackage{marginnote}%notas en el margen

%------------------------------------------

\newtheorem{axioma}{\large\textbf{Axioma}}
\newtheorem{teo}{\large\textbf{Teorema}}[chapter]%entorno para teoremas
\newtheorem{ejem}{{\it\textbf{ Ejemplo}}}[chapter]%entorno para ejemplos
\newtheorem{def.}{\textbf{Definición}}[chapter]%entorno para definiciones
\newtheorem{post}{\textbf{Postulado}}[chapter]%entorno de postulados
\newtheorem{col.}{\textbf{Corolario}}[chapter]
\newtheorem{ej}{\textbf{Ejercicio}}[chapter]
\newtheorem{prop}{\textbf{Propiedades}}[chapter]
\newtheorem{lema}{\textbf{Lema}}[chapter]



%----------Formato título de capítulos-------------

\usepackage{titlesec}
\renewcommand{\thechapter}{\arabic{chapter}}
\titleformat{\chapter}[display]
{\titlerule[2pt]
\vspace{4ex}\bfseries\sffamily\huge}
{\filleft\Huge\thechapter}
{2ex}
{\filleft}



%--------------------Documento-----------------------------

\begin{document}
\normalfont
\input xy
\xyoption{all}
\author{\Large Apuntes por FODE}
\title{Negocio de la Música \vspace{0.5cm} \\ \small Berklee College of Music}
\date{}
\pagestyle{empty}
\maketitle
\thispagestyle{empty}
\let\cleardoublepage\clearpage
\tableofcontents 								%indice

%------------------------------------------
 
\let\cleardoublepage\clearpage
\part{Fundamentos empresariales musicales}
\let\cleardoublepage\clearpage
\chapter{La industria de la música}
\section{Descripción general}
Los objetivos e aprendizaje son:
\begin{itemize}
\item Reconocer los principios generales para el éxito en la industria de la música.
\item Identifique los avances empresariales y tecnológicos que han impulsado la industria de la música a lo largo del tiempo.
\item Evaluar los cambios de paradigma que han definido cómo se consume la música.
\item Examine el panorama empresarial de la música actual y el impacto de las tecnologías digitales.
\end{itemize}
\subsection{Principios rectores de la industria musical}
Fui a ver una película en el año 2000. La película se llamaba $"$Next Friday$"$. Era una producción de Ice Cube, una película de muy bajo presupuesto, y de hecho, encabezó las ventas en taquilla esa semana. Había pasado ya un cuarto de la película cuando escuché una pista instrumental como fondo de una escena. Una gran canción, llena de ritmo, comencé a chasquear los dedos, y me dije, $"$Conozco esa canción,  y conozco al grupo también. ¿Cómo se llamaba esa canción? ¿Y cómo se llamaba el grupo?$"$ Y entonces la letra comenzó y escuché mi voz. Era una canción que grabé a mitad de los años 70s con el grupo Cameo para su primer disco, titulado $"$Cardiac Arrest$"$. La canción se llama $"$Rigor Mortis$"$, y por supuesto mi primera reacción fue,  $"$Estoy en una película, ¡fantástico!$"$ Me sentía tan bien, quería pararme  y bailar al ritmo de la música. Estoy seguro que cualquiera se sentiría igual  si escucharan su propia voz, ó se vieran en una película por primera vez. Pero entonces, mientras me sentía así, empecé a recordar cuando estaba con el grupo. Y me dije, $"$espera un minuto.\\ 
Cuando estaba con el grupo sólo ganaba \$ 100 a la semana, para tocar en vivo, para ensayar, para grabar.$"$ E incluso en los 70s, \$ 100 a la semana  no te llevaban muy lejos en la ciudad de Nueva York. Por aquel entonces, yo era el abogado de The OJays, miembros del salón de la fama,  y del fallecido, y grande del R\& B, Gerald Levert. Así que sabía cuánto dinero se paga a las disqueras y casas editoriales cuando usan su música  en ese tipo de formatos Así que ideé una frase que utilizo a menudo, y se la digo a todos mis alumnos.\\ 
Cada vez que la música es interpretada, alguien cobra. Cada vez que la música es interpretada, alguien cobra. Si eres un artista, un compositor, ó un productor, deberías de estar cobrando,  no siendo interpretado. Hay tres principios fundamentales para alcanzar el éxito actualmente en la industria musical. Las llamo Las Tres Grandes Ps. Las tres Ps son: 
\begin{itemize}
\item Producto fuerte.
\item Perspectiva adecuada.
\item Actitud Profesional.
\end{itemize}
Producto fuerte, Perspectiva adecuada y actitud Profesional son Las Tres Grandes Ps para alcanzar el éxito en la industria musical. ¿Qué es un producto fuerte? Un producto fuerte puede ser una gran canción, emparejada con una gran interpretación de un gran artista, producida por un gran productor en un gran estudio. De hecho, el estudio de grabación no tiene que ser un lugar enorme de los que graban digitalmente en 48 canales. Puede ser un estudio en el cuarto de una casa. Puedes crear un gran producto ahí. Me gustaría ampliar el sentido de la palabra $"$producto$"$ para que signifique más que sólo un producto grabado,  una grabación de audio. Puede ser un vídeo.\\
YouTube se ha convertido en una gran plataforma que produce productos fuertes, ayudando a que despeguen  las carreras de muchos artistas. \textbf{Un producto fuerte puede ser una aplicación. Puede ser una obra de software, puede ser un sitio web,  puede ser una marca.} Jay-Z ha hecho cosas maravillosas con su marca, al asociarse con Samsung. \textbf{Un producto fuerte es un activo cuyo valor  aumenta con los años}. Y en eso se basa realmente la industria musical,  activos que crecen en valor. La segunda gran P es Perspectiva adecuada. Perspectiva adecuada. Muchos artistas, cuando tienen una canción exitosa, su primera canción exitosa creen que van a ser millonarios en los siguientes seis meses,  ó un año. Lo dan por sentado, creen que va a suceder. Confía en mí, muchas veces la fama  no equivale a la fortuna. Se necesita una perspectiva adecuada  para entender eso, para entender el negocio musical y cómo funciona. Y, al mismo tiempo, \textbf{hay que reconocer que se tiene que amar este negocio. Tienes que sentir pasión por él para poder quedarte}. Si quieres estar en el negocio, tienes que amarlo. \\
Uno de mis colegas me contó una historia  sobre el gran Niles Rodgers. El gran Niles Rodgers es un guitarrista y productor, y comenzó con el grupo $"$Chic$"$ en los años 70. Tuvieron grandes éxitos como $"$Le Freak$"$ y $"$Everybody Dance$"$. Pasó a producir a grandes artistas, Madonna, David Bowie, Diana Ross. Incluso tuvo un éxito este año, actualmente está  en las listas de Billboard, con Daft Punk. Y dice que el primer contrato que firmó era, al mismo tiempo, el peor y el mejor contrato que ha firmado. Se refería a que claramente no obtuvo todo lo que él quería, en cuanto a dinero. Pero ese contrato supuso el medio por el cual  él pudo demostrarle su talento al mundo. Fue la oportunidad para mostrar  las cosas que él podía hacer. Y resulta que sigue trabajando, y ha tenido una larga carrera de más de 40 años. Y eso me lleva a la tercera gran P,  que es actitud Profesional. ¿Qué es la actitud Profesional? \textbf{Es importante tener una educación adecuada sobre el negocio detrás de la música}, lo cual estás haciendo  ahora mismo al tomar este curso. \textbf{Necesitas entender qué son los derechos de autor,  qué son los contratos, y las diversas disposiciones en los contratos,  de qué eres responsable.} Es muy importante contar con esa formación. Y \textbf{aquí está la última parte de la actitud profesional,  que es posiblemente más importante que la primera parte. Tener respeto por la gente con la que trabajes.} Respetar a todas las personas con las que entres en contacto  dentro de la industria musical. Uno nunca sabe cuándo  el pasante o la recepcionista puedan convertirse en el próximo presidente  de la disquera en la que trabajas. Así que ten en cuenta las Tres Grandes Ps para tener éxito en la industria musical. Producto fuerte, un producto fuerte es un activo cuyo valor  crece con el tiempo. Perspectiva adecuada, entender que  necesitas amar este negocio, tienes que tener pasión, y  que tienes que hacer como mi padre me aconsejó hace años, a pensar con una visión a largo plazo. Se necesita tiempo para desarrollar  una carrera en este negocio, pero lo que te puedo decir de representar  a clientes que han tenido carreras de 40 ó 20 años, es que se puede hacer. Entender la perspectiva adecuada,  y siempre tener una actitud profesional. Entiende bien los aspectos básicos de la industria. Y sobre todas las cosas, ten respeto por todas las personas que conozcas en esta industria,  porque nunca se sabe. Conocerás a las mismas personas al alcanzar el éxito, que al perderlo. Y recuerda, cada vez que la música es interpretada, alguien cobra. Cada vez que la música es interpretada, alguien cobra. Y si eres un artista, compositor o productor,  siento que deberías de estar cobrando. sin ser defraudado.
\section{Historia de la industria musical}
\subsection{1900-1950}
\textbf{El negocio de la música es la unión entre el arte y el comercio}, y se construye en torno a la innovación tecnológica  y el emprendimiento. Actualmente, experimentamos un cambio de paradigma pasando del formato físico  al formato digital. Y en el medio de este cambio, mucha gente ha dicho que la industria de la música está muriendo. La industria de la música está acabada. Bueno, quiero decirles que todos esos rumores son muy exagerados. Mientras que en la industria discográfica, el negocio de la música se ha reducido, otras áreas como las actuaciones en directo, se han incrementado notablemente. El aumento de ventas digitales tanto de álbumes como de singles, del streaming, y hasta los servicios de subscripción se han intensificado. Sin embargo, no es la primera vez que la gente cree  que la industria de la música está acabada. Esto se ha repetido en numerosas ocasiones a lo largo de los últimos 100 años de historia de la industria musical tal como la conocemos nosotros. El notable astrónomo y autor Dr. Carl Sagan dijo: \" Para entender tu futuro, debes conocer tu pasado\". Bien, vayamos al comienzo de la industria de la música tal como la conocemos. Volvamos a los años de 1890. En 1890, había espectáculos ambulantes, espectáculos de juglares, por desgracia, gente blanca con la caras pintada de negro y muchos negros con la cara negra, cantando canciones que el público disfrutaba. ¿Había un negocio más allá de la interpretación en vivo? Sí lo había. ¿Cuál era ese negocio? El negocio de las partituras. Las partituras se vendían como los discos en el siglo 20. Además de las actuaciones en vivo, las partituras eran el producto que se vendía Un negocio en expansión. Había emprendedores en esa época, compañías que empezaban: Shapiro, Bernstein y compañía. Estaba Mills Music. Compañías que todavía existen y que empezaron como compañías de publicidad. Imprimiendo  las partituras que vendían. Por aquel entonces, había familias de clase media que tenían pianos en sus salones, así que si iban a un espectáculo juglar o a un espectáculo de vodevil, a los pocos días iban a la tienda de música. La única manera en la que podían oír esas canciones que que tanto habían disfrutado noches atrás, era  comprando la partitura y llevándola a casa para empezar a tocar la música y  cantar las canciones. Fue un gran negocio durante unos 10 o 15 años, diría yo. Pero entonces apareció un nuevo producto, creado poco después del inicio del nuevo siglo. Un nuevo producto, una nueva forma de música y de canciones. La pianola. La primera reproducción mecánica de una canción, la pianola fue la primera reproducción mecánica en popularizarse Las perforaciones en el papel permiten que las teclas suenen en el momento justo y oyes las mismas melodías de la canción que has escuchado. ¿Cuál fue la reacción de las compañías editoras de aquel momento? Bueno, pueden apostar a que sus reacciones fueron,  un momento... ¡Nuestro negocio está acabado! ¿Quién va a querer comprar la partitura y tener que tocarla en el piano y cantar las canciones cuando pueden escuchar el piano  en la pianola? ¡La industria de la música está acabada! ¿Lo estaba? No. El Congreso se involucró corrigiendo la Ley de Copyright para asegurarse de que a los creadores de esas composiciones se les pagaran los derechos de autor. cada vez que uno de esos rollos de papel se vendiera. ¿Estaba la industria acabada? ¿Afectó al negocio de la edición? ni la ciencia. En lo absoluto. Yo logré tener éxito porque El negocio de la edición floreció incluso más, y las pianolas también. Un emprendedor en la industria de la grabación fue un hombre llamado Harry Pace. Harry Pace era el socio del gran fundador del blues, el fundador, el padre del blues, el gran WC Handy, quien a principios de los años veinte,  vio un mercado para las cantantes negras de blues. y creó una empresa en Harlem llamada Black Swan Records. Muchos creen que Motown fue la primera discográfica propiedad de negros, Pero no, hay que remontarse al inicio de la industria discográfica. Así, durante la década de 1920 las grabaciones se hicieron populares. y vender discos también. Los primeros discos populares fueron grabados por cantantes negras de blues. Mamie Smith, Ethel Waters, a comienzos de los años 20 se convirtieron en super estrellas de la canción como resultado de  su venta de discos en los años 20. ¿Cómo se sintieron los editores acerca de esto? Había algo de preocupación, por supuesto. Iban a pagarles los derechos de autor, pero seguramente pensaban, \" Dios mío, el negocio de la edición está acabado\" . La pianola no pudo con nosotros, pero la gente puede comprar discos la gente escucha las canciones cuando quiere ¿Por qué iban a querer comprar música? Al final, no hubo ningún problema. Las grabaciones ayudaron  a la industria editorial y a la venta de partituras que se expandió aún más. Con la década de 1930 llegó la Gran Depresión. El negocio musical se vio diezmado, y una vez más, la gente pensó que  la industria musical estaba acabada. La gente no podía permitirse comprar discos, ni los espectáculos de vodevil ir de gira, la gente no podía costeasrse sus entradas. Pero entonces, apareció otra innovación musical: la radio, que en realidad había surgido algunos años atrás, retransmitiendo, hasta el momento, principalmente noticias. Pero en la década de 1930, las emisoras de radio  decidieron traer a bandas de música. Los grupos no podían trabajar por su cuenta, así que los llevaron a los estudios y les dejaban tocar allí su música. Tanto la industria discográfica como la industria editorial, pensaron que esto les iba a traer problemas. No era el \" peer to peer\" lo que les molestaba, No era el \" peer to peer\" lo que preocupaba a la gente hace 100 años, Era a a e. A a e. Aire, las ondas de radio en el aire,  llegando a los oídos (ears) de los consumidores, era lo que inquietaba a las industrias de la grabación y de la edición. ¿Por qué alquien iba a querer imprimir partituras? ¿Por qué iban siquiera a comprar discos si pueden escuchar música gratis? Nuestra industria está acabada. ¿Lo estaba? ni la ciencia. En lo absoluto.\\
Yo logré tener éxito porque Hubo estrellas de la radio que se convirtieron en estrellas de la grabación, porque hubo un aumento de las ventas de discos como resultado de la promoción radiofónica. Las partituras siguieron vendiéndose. William Paley fue también un emprendedor,  y en 1927 compró 16 emisoras de radio, principalmente con el objetivo de publicitar el negocio tabaquero de su padre. Pero ese grupo de emisoras, la Columbia Broadcasting System, CBS, jugó un papel importante, en la industria musical y en los medios de comunicación en general. Pero entonces llegó la Segunda Guerra Mundial. Y el uso de esmalte fue restringido. El esmalte se utilizaba para crear discos, los llamaban 78's porque se movían alrededor del tocadiscos a 78 revoluciones por minuto. Éstos eran los llamados \"single\", Los llamaron \" singles\" porque sólo tenían  una canción en una de las caras, y otra canción en la otra. Pero no pudieron fabricarlos durante la guerra porque todos los materiales se destinaban a asuntos bélicos. La mayoría de los hombres estaban luchando en la guerra. Una vez más, la gente pensó que la industria estaba acabada. No va a suceder. Gracias a aquella innovación tecnológica: la radio. Esta vez fue la creación de La Cadena de Radio de las Fuerzas Armadas, 90 emisoras llevando sus ondas a oyentes de todo el mundo. No sólo para los americanos, era la primera vez que muchos extranjeros tenían la oportunidad de escuchar música americana. \\
¿Y qué clase de impacto tuvo esto en el resto del mundo? Están oyendo esta música y  nunca habían escuchado nada igual. Ritmos africanos. Contenido europeo. Es música popular, música para bailar tenía algo que era totalmente fascinante  para la gente de otros lugares. Así pues,  lo que pareció ser uno de los momentos  más duros tanto en el ámbito político, como en el ámbito social con la Guerra Mundial,  acabó siendo el impulso que la industria discográfica necesitaba para expandirse después de la guerra. Y hubo un gran crecimiento de la industria musical Cuando los soldados volvieron a casa, la venta de instrumentos se disparó. Seguramente hubieron soldados europeos en  el frente y en las trincheras que durante su tiempo libre escuchaban la Cadena de Radio de las Fuerzas Armadas, y descubrieron la gran música de la banda de Duke Ellington, Count Basie. Alguno de ellos seguramente dijo,  Si salgo de esta con vida, cuando vuelva a casa, tomaré un saxofón y lo haré sonar como $"$Lester Young.$"$ $"$Sí, quiero tocar como Pres$"$ Y créanme,  cuando volvieron a casa fue exactamente lo que hicieron. La venta de instrumentos se disparó y entonces apareció otro emprendedor que en 1945 aprovechó esta gran oportunidad. Este emprendedor vio una demanda  que había que cubrir ¿Qué demanda? Bueno, si los soldados que volvían a casa
se beneficiaban de la ley GI Bill, un programa gubernamental cuyo objetivo era que el Estado se encargara de la financiación de los estudios técnicos o unversitarios de los veteranos. Este emprendedor dijo: $"$Conozco a un montón de esos veteranos, que vuelven a casa con muchas ganas  de tocar como Lester Young.$"$  $"$Quieren formar parte de una banda$"$ Pero los únicos programas de pregrado o programas de nivel universitario, los programas avanzados de música se daban en  los conservatorios, que sólo enseñaban música clásica. Este emprendedor dijo:  $"$ No, eso no es lo que ellos quieren $"$ $"$ Creo que lo que quieren es entender y ser capaces de tocar en bandas, grandes bandas, tocar música jazz$"$. Así fue como en 1945, fundó la Schillinger House, les hablo de Lawrence Berk,  fundador de Berklee College of Music. Cuando hablo de la Schillinger House, me estoy refiriendo a la precursora de Beerklee College of Music. En los años 50 Lawrence Berk decidió cambiar el nombre en honor a su hijo, Lee Berk. Berk-Lee, un gran proyecto emprendedor  de mediados del siglo pasado, La innovación técnica y el emprendimiento, fueron el sello distintivo de la industria musical durante la primera mitad del siglo 20, y serían aún más importantes en la segunda mitad.
\subsection{1950-2000}
La innovación tecnológica y el emprendimiento fueron cruciales al desarrollo de la industria de la música en la primera parte del siglo XX. Pero a mediados de siglo había una serie de nuevas innovaciones y empresarios, que querían participar en el desarrollo posterior, particularmente en la industria discográfica. \\
Uno era Ahmet Ertegun, hijo de un inmigrante turco. Amaba la música negra. Amaba el blues. Le encantaba lo que sucedía en la escena del jazz. Una forma de baile del blues, llamada jump blues. Comenzó una compañía discográfica en Nueva York, con un socio llamado Atlantic Records. Y él era parte de una nueva ola de música llamada R y B, o Rhythm and Blues. No era el único, lo mismo estaba sucediendo en la costa oeste, con Art Rupe, quien era el dueño de Specialty Records. Y esa fue una compañía que grabó grandes blues y R y Cantantes B del lugar de nacimiento de la música popular, Nueva Orleans, Louisiana.\\
Se tuvo un avance tecnológico en la televisión. El desarrollo de la televisión. En la década de 1950, y su impacto en la radio. La radio a principios de siglo, desde el comienzo de la radio, realmente tenía lo que se llamaba serie dramática en serie, donde tenías actores de voz que entraban y representaban las partes en la radio. Bueno, los dueños de las estaciones de radio y todas las personas que participaron en ese proceso estaban preocupados cuando entró la televisión, porque pensaron que ese tipo de programación se transferiría por televisión. Y de hecho, lo hizo. Y realmente dejó un vacío en el mercado de la radio que estaba lleno por estaciones de radio que reproducen discos. En mayor grado que nunca. Y, por supuesto, eso hace que algunas personas también estén descontentas, pensaron que la reproducción de discos y la sustitución de la serie dramática en serie no tenía el mismo tipo de calidad, como lo hicieron los programas de radio en serie. \\
Jukeboxes. Una invención tecnológica completamente nueva a principios de los años 50, última parte de los años 40 y 50. También cambió la industria de la música. Permitía a las personas ir a un restaurante, a una farmacia, poner una moneda en la máquina y reproducir cualquier disco que quisieran. Bueno, eso fue un poco inquietante para varias personas. Los músicos en vivo sintieron que fueron desplazados, porque en lugar de traer bandas a los clubes, un club podría incluso tener una máquina de discos. Y podría desplazarlos. También tienes compañías discográficas que sintieron, si las personas pueden salir y tener los discos que quieren haber reproducido en cualquier momento,tal vez eso reduciría sus ventas récord. Bueno, eso no resultó ser el caso después de todo, porque burbujeaba debajo de todo esto, era una nueva forma de música.\\
Y ciertos cambios culturales, como tuviste baby boomers, los adolescentes nacen de soldados que regresaron de la guerra, que estaban llegando a la mayoría de edad a mediados de los años cincuenta y tenían muchos ingresos discrecionales y realmente compró esta nueva forma de música, que estaba basada en el blues, gospel, música pop y música R y B. Rock and roll, que explotó. La industria de la música explotó particularmente la industria discográfica, que pasó de un total de \$ 200 millones en ingresos en 1954, a más de 604 millones en 1958. Durante ese período de cuatro años, tuviste ese gran crecimiento.\\
El espíritu emprendedor continuó creciendo, incluso con algunos de los artistas durante este período. Sam Cook, cantante pop y cantante de R y B, comenzó su propia compañía discográfica. Comenzó su propia editorial y el gran Ray Charles renegoció su contrato con Atlantic Records decidió dejar esa compañía e irse con ABC registros, porque ABC Records le permitió tener sus propias grabaciones maestras. Así que este comienzo de emprendimiento continuó creciendo durante ese período. Otros talentos creativos también se interesaron en convertirse empresarios en la industria discográfica.\\
Un joven escritor de canciones de Detroit, Michigan, Barry Gordy había escrito un par de canciones, un par de golpes de hecho, para una gran R y B artistas en ese momento, Jackie Wilson. Finalmente decidió comenzar su propia compañía discográfica, en Detroit se llama Motown Records. Que pasó a ser mundialmente reconocido y inspirar a otros talentos creativos para comenzar sus propias etiquetas. Uno de esos otros talentos fue Herb Alpert, quien como líder de banda y un trompetista tuvo tremendos éxitos con su banda Tijuana Brass, y decidió asociarse con un gran disco nombró a Jerry Moss para iniciar A\& M Records a mediados de los años 60. La explosión de R y B, rock \& roll y La música soul llevó a la invasión británica a mediados de los años 60. Y los Beatles, que crecieron en Inglaterra, escuchando la gran música blues y También los grandes éxitos de Motown, le dan su propio giro y llegó a América y creó un fenómeno como ningún otro. La venta de registros explotó. En ese momento, 45 y singles en 45, dos caras, una canción en cada lado, fueron la configuración predominante, la configuración de ventas. Álbumes que realmente comenzaron en la década de 1950, también estaban comenzando a ganar algo de tracción. Y como el desarrollo de la música rock y El crecimiento de la música R y B continuó en los años 70, Los álbumes comenzaron a despegar y se volvieron muy, muy populares. También durante este período, en la década de 1960 tuvo el desarrollo de carrete para grabar cintas de audio y cinta de cassette, lo que cambió el juego. Permitió a las personas tomar la música de los discos de vinilo y transfiéralo a este tipo de configuración. A la configuración de la cinta de audio. Creó grandes problemas. La gente sentía que podían crear la suya, y no sentían que realmente crearan sus propias listas de reproducción. Y también pudieron hacer copias como nunca antes, para compartir con sus amigos Y eso se convirtió en una gran preocupación para la industria discográfica.\\
A medida que nos mudamos a los años 70 y 80, Empezaron a suceder otras cosas, otros avances tecnológicos. La música disco a mediados de los 70 condujo a una nueva ola de emprendedores. ingresando al negocio comenzando sus propias compañías de producción y etiquetas, vender singles de 12 pulgadas a las masas y a los clubes, Las discotecas tocaron estos singles sin parar toda la noche. Pero justo cuando la música disco llegó a su punto máximo a mediados de los 70 y en 78 o 79, casi tan rápido como su ascenso meteórico, había terminado.  Disco estaba muerto. La estación de radio de Chicago tuvo una fiesta en un juego de los Chicago Cubs, donde quemaron discos de discoteca e igual de rápido y el rápido ascenso que música disco hecha a mediados de los 70, tuvo una muerte igual de rápida. Y una vez más, los detractores dijeron que la industria de la música estaba muerta. Pero el hecho es que la década de 1980 trajo más innovación, eso creó un nuevo mercado para la industria de la música.\\
Sony lanzó el primer reproductor de cassette portátil, The Walkman, a principios de los años 80. Y transformó la industria. Ese fue el primero, bueno, realmente no el primero porque tenías un transistor radios en la década de 1950, que también transformaron la industria. Pero en la década de 1980, el Walkman, el casete personal El jugador permitió a las personas llevar su lista de reproducción en sus bolsillos, y escuchar la música que quisieran escuchar. Transformó la industria y le dio el tipo de impulso que necesitaba para intentar construirse de nuevo. Y luego, en 1981 y 1982, MTV, un nuevo Music Television Network surgió y creó el mercado de videos musicales que generó grandes éxitos durante la década de 1980. Y también creó grandes artistas y cambió la industria de una manera excelente. Por supuesto, no puedes ignorar el impacto del gran Michael Jackson, y su álbum Thriller que también proporcionó un impulso.\\
Se tuvo otro desarrollo tecnológico, el CD, que llegó a principios de los años 80. Y mucha gente, incluyéndome a mí, realmente tenía preguntas sobre. qué tan efectivo sería el CD y cómo realmente se pondría al día. Después de todo, el CD cuesta tres veces el costo de un disco de vinilo y un cassette. Y cuando lo escuché por primera vez, dije que nunca compraría ningún producto, que tendría que pagar el triple del precio. Podría obtener tres cassettes por el precio de un CD. No puede sonar tan bien. Todas las personas de marketing vieron de manera diferente, y realmente promocionaron el CD como el que tiene un sonido digital superior. Y también ser indestructible, recuerdo la primera vez que realmente escuché un CD en la casa de mi primo, en su sótano Y fue un álbum de Michael Jackson, cuando cerré los ojos y escuché la música, Era como si estuviera en el estudio con Michael.\\
Y eso es lo que realmente me cambió. Realmente podría ver el futuro, realmente vería que los CD evolucionarían, y ser la próxima evolución del negocio de la música. Ahora más o menos por la misma época en los años 70 y 80, cuando los álbumes se hicieron muy populares y vendido en grandes números tanto en casete como en vinilo, A medida que esta nueva configuración de CD comenzó a despegar en los años 80, Las grandes corporaciones realmente se interesaron por las etiquetas pequeñas e independientes. Realmente vieron el valor de la música y la venta de grabaciones. La inversión fue mucho menor que hacer una película, y el retorno podría ser mucho mayor, porque puedes invertir en muchos más artistas de lo que podrías en una película. Entonces se redujo el riesgo. Además, durante ese período de tiempo, Las prácticas contables de muchas de las etiquetas independientes experimentan un cambio. Se volvieron más confiables y más verificables. Y entonces las corporaciones realmente sintieron que esta era una industria preparada para inversión corporativa importante y eso es lo que hizo la industria. En la década de 1990, las ventas de CD despegaron ya que muchas personas intentaron reemplazar sus vinilos y colecciones de casetes, con los mismos registros esta vez en CD. El crecimiento de las ventas de CD en la década de 1990 fue notable, como resultado, las compañías discográficas y las editoriales se volvieron muy valiosas materias primas, y las grandes corporaciones comenzaron a invertir a lo grande, a medida que avanzaban los años 90, muchas de estas corporaciones comenzaron a fusionarse también. Pero sucedieron otras cosas en la industria de la música. En ese momento, cuando las compañías discográficas se llenaron de efectivo, de vender estos CD. Las compañías discográficas se hincharon muchas veces con empleados. Comenzaron a firmar demasiados actos. La calidad de los actos podría no haber estado en el nivel que deberían tener estado, para atraer al público. Durante ese tiempo algunas personas dirían que las compañías discográficas se pusieron tan codiciosos a finales del siglo XX que dejaron de lanzar singles. Y si realmente querías esa canción, ese single que realmente amabas, tenías que comprar el CD completo. Verdaderamente las compañías discográficas sintieron que, los buenos tiempos nunca terminarían, pero pronto descubrirían lo contrario.
\subsection{Napster y más allá}
La venta de discos llegó a un auge en 1990, de \$ 14 billones de dólares. Probablemente recordarás que dije que la venta musical en 1958 era de sólo \$ 604 millones de dólares. Por lo que el siglo terminó muy bien, y realmente fue un siglo de avance tecnológico e impulso empresarial. Pero al mismo tiempo, Napster apareció  en la escena musical, y reventó la burbuja de ventas de discos. Las ventas se desplomaron como resultado  del intercambio ilegal de archivos entre usuarios. Nadie sabía cómo detener los estragos que el intercambio ilegal estaba causando en la industria de la música hasta el año 2003, y fue Steve Jobs de Apple quien diseñó una nueva forma de acceder y comprar música en línea,   una distribución digital, un método, iTunes, que se volvió muy exitoso. También lo acompañó con hardware como el iPod y el Shuffle, que le permitió a la gente llevar la música a cualquier parte. Su música, su lista de reproducción.  Similar a lo que había pasado en los años 80 con el Walkman de Sony, un gran avance tecnológico. Curiosamente, en los años siguientes, iTunes hizo que los sencillos musicales regresaran a ser la configuración predominante en la industria musical. Me gusta decir que la gente tuvo que empezar  a vivir una vida sencilla, y me refiero a los años 60s y 70s, cuando los sencillos eran la configuración predominante de ventas. Y eso fue lo que pasó  en la primera década del siglo XXI.\\ Alrededor de esos años, también habían artistas como David Bowie, quien predijo y soñaba con que algún día, la música fuera  accesible de manera ilimitada. Veía la música fluir como agua, y creo que también soñaba con que se pagara por música como se paga por agua, de manera mensual. Hubo otro avance tecnológico. Radio por satélite. XM y Sirius Radio permitían recibir señales de radio vía satélite en los automóviles. Una parte interesante de esto es que ese fue el primer avance tecnológico por el que las personas estaban dispuestas a pagar de manera mensual para tener música. No era un concepto nuevo. Ciertamente. Las personas con televisión por cable han estado pagando desde los años 70 una factura mensual. De hecho, creo que unos 75 a 100 millones de personas pagan una factura mensual de cable, y esa era la idea. Que este tipo de pago y acceso a la música, no sólo poseer el producto físico, sino tener acceso ilimitado a la música, sería algo del futuro. Otro adelanto tecnológico emergente, fue el servicio de radio por internet, Pandora, que de hecho estaba construido bajo un modelo de servicio de dos niveles en donde la gente se podía registrar  para obtener el servicio y escucharlo gratis. Pero también tenían un servicio premium,  en donde se podía pagar una suscripción mensual y eliminar los anuncios entre canciones del servicio gratuito. Eventualmente, la industria discográfica se unió y trabajaron para sacar a Napster del negocio, pero se tenía el desarrollo de otros servicios para proporcionar música de una manera similar a lo que había sugerido David Bowie. De una manera donde la música fluyera como agua. No necesariamente se tendrían las copias físicas, pero la gente tendría acceso a ellas, en lugar de la propiedad. El internet y servicios móviles de streaming como Spotify y Rhapsody crean un nuevo mercado de gran crecimiento para servicios de streaming. En un formato multinivel, el servicio gratuito financiado por anuncios, y la suscripción premium que elimina la publicidad, ha estado creando nuevos ingresos para la industria musical que realmente tiene una oportunidad para seguir en le futuro. Spotify fue uno de los primeros servicios de streaming que fue capaz de negociar licencias con todas las disqueras principales y muchas independientes para permitir acceso ilimitado a un amplio  catálogo de música. Y los artistas han estado considerando las ramificaciones de estas ofertas especiales. Y básicamente los principales sellos discográficos y empresas editoriales, e incluso artistas de estudio, han recibido ofertas especiales,  al igual que el gobierno, a través de licencias obligatorias que estas nuevas empresas han estado dando, como Pandora y Spotify, ofertas especiales referentes a la tasa de regalías, para que puedan iniciar su negocio. Y con la esperanza de que este crezca y tengan un buen número de suscriptores. Y llegará a un punto en el cual todos podrán hacer más dinero. Y esa va a ser la gran carga para los artistas en los próximos años, asegurarse de estar bien representados y abogar por una mayor parte de los ingresos que se generen de estos servicios de suscripción y streaming digital, a medida que estas empresas comienzan a crecer. El avance de la tecnología digital crea todo tipo de oportunidades para un número de empresas a través de este medio. Lo vimos durante la primera década de este siglo. Derek Sivers, ex alumno de Berklee, comenzó CD Baby, un servicio digital que te permitía pedir CDs para que fueran enviados a tu casa, y ahora pueden incluso ser entregados digitalmente. Existía Sonicbids, una empresa creada por otro ex alumno de Berklee, Panos Panay, quien creó una agencia de talentos y reservas virtual, además de proporcionar otros servicios. Redes sociales, Facebook, Twitter. Existen sitios de crowdfunding y aplicaciones, Kickstarter, PledgeMusic, todos creando grandes oportunidades para la fusión de tecnología innovadora con el sector creativo de los artistas para llegar a un nuevo mercado durante este siglo. Muchas personas señalan el hecho de que el mercado de las actuaciones en vivo ha crecido, en cuanto a la generación de ingresos. Y mucha gente cree que los artistas, aunque tal vez no ganen mucho dinero de la venta de discos, o incluso de tener su música en streaming o siendo escuchados a través de varios servicios de suscripción, ¿cómo van a hacer dinero? Pues muchas personas piensan que pueden hacer dinero de las actuaciones en vivo, que los artistas pueden ganar dinero estando de gira. Deberían de poder tocar en vivo. Bueno, eso puede ser cierto. Hay un tremendo crecimiento en festivales. Hay un crecimiento en establecimientos nocturnos en todo el país. La pregunta será: ¿En qué medida pueden los artistas, los artistas emergentes, generar ingresos de esas fuentes? Sin duda, los artistas consolidados y los que se están presentando en los grandes  festivales están ganando mucho dinero. Pero, ¿qué hay del artista que está empezando en el club? Y aquí hay otro punto en el cual pensar. No todos los creadores de música son intérpretes. ¿Qué hay de los compositores? Nashville tiene un gran sistema de compositores, un gran sistema de letristas. Y un gran conjunto de compositores, y todo lo que hacen es escribir canciones. ¿Qué parte tiene en esta mezcla? Teniendo en cuenta el hecho de que las ventas están bajas, y los derechos de streaming y de suscripción no están en un punto en el cual puedan proporcionar la gran vida a la cual estos autores se habían acostumbrado. Así que estamos en medio del cambio. Sin duda van a crecer a lo largo del tiempo. Además, los artistas están haciendo enormes cantidades de dinero, los artistas más famosos, de otros tipos de oportunidades como apoyo a marcas y patrocinios, e incluso de desarrollo de productos. Fifty Cent no ha lanzado un disco  en más de ocho años, pero ¿ha tenido una carrera exitosa como empresario? Con la marca Vitamin Water, sin duda que sí. Hizo millones, cientos de millones de dólares por tener una participación accionaria en esa empresa. Están sucediendo cosas interesantes  en la industria de la música. Las posibilidades son infinitas, pero realmente tienes que averiguar  cómo formar parte de ella. ¿Vas a desarrollar una aplicación? ¿Vas a escribir la próxima gran canción? ¿Vas a ser un artista? ¿Vas a crear otro tipo de servicio que beneficiará a los artistas? ¿Dónde te encuentras? ¿Cuál es tu objetivo en esta nueva era digital emergente?.
\subsection{Estructura empresarial de la música actual}
El cambio de paradigma de lo físico a lo digital realmente tomado un peaje drástico en las principales discográficas.  Donde había seis sellos discográficos importantes en los años 80 y 9os, actualmente solo hay tres. Ahora, ¿cuáles son estos grandes sellos discográficos? ¿Qué hacen ellos? Bueno, no son como las etiquetas antiguas, donde tenías un emprendedor que podría haber sido artista o compositor, como Berry Gordy o Ahmet Ertegun, que realmente se involucró en el proceso creativo en gran medida. Estas etiquetas principales son realmente grandes conglomerados. Y su trabajo principal es la fabricación, marketing y distribución de música a través de diferentes plataformas. Es posible que no se involucren tanto, hasta cierto punto, en el nivel creativo. Lo que tienen son subsidiarias y sub-etiquetas que brindan ese tipo de servicios. Entonces tienes tres sellos discográficos principales actualmente. Sony Music Entertainment, una preocupación japonesa, Universal Music Group, propiedad del conglomerado francés Vivendi, y Warner Music Group, recientemente comprado por Leonard Blavatnik de Rusia. Estas son las principales etiquetas. Y debo decir que los principales fabricantes, comercializadores y distribuidores, de hasta el 85 al 90\% de los registros que se venden hoy. Llevan ese tipo de influencia. Todo bien. Hablemos de las principales discográficas y sus subsidiarias y subetiquetas. \\
Universal Music Group tiene una serie de sub-etiquetas influyentes y significativas y filiales como Island / Def Jam, el gran sello Interscope, Geffen Records, los discos Motown de Berry Gordy ahora se distribuyen a través de Universal Music Group. Y más recientemente, Universal Music Group compró el catálogo y contratos de EMI, Prioridad y Capitol Records, reduciendo el número de sellos principales de cuatro a tres. Sony Music Entertainment distribuye los grandes catálogos y los artistas asignados a Columbia, Epic Records, Arista, J Records y Jive Records, un gran cuadro de historia sellos que han tenido un tremendo efecto en la industria de la música. Y, por último, el Warner Music Group, distribuye etiquetas como Elektra, Atlantic y, por supuesto, la etiqueta Warner. Sabes, es sorprendente que, incluso en este día y época de bricolaje, encontrará muchos artistas que quieren firmar con las principales discográficas. Pero al mismo tiempo, tienes otros artistas que dirían: ¿por qué querrías firmar con un enorme empresa que puede no prestarte tanta atención? Bueno, las personas que sienten que necesitan estar con las grandes discográficas, como el prestigio de estar con una etiqueta importante. Sienten que eso les da influencia no solo como artista, sino pueden tener aspiraciones como productores, y Quiero trabajar con otros artistas que estén firmados con las etiquetas. Y al decir que están firmados con Columbia, Puede abrir nuevas oportunidades para ellos.\\
La mayoría de estos artistas también mencionan el hecho de que estos grandes sellos tienen Un sistema de distribución que es amplio y vasto, particularmente en el espacio físico. Y puedo decirte que eso es realmente cierto. Y aquí está la cosa, las grandes discográficas son tremendas para expandir su mercado. Me parece notable que una gran discográfica pueda decidir una fecha en particular. tener sus grabaciones en tiendas de discos a nivel internacional. Pueden adaptar todo su esfuerzo de marketing para el lanzamiento de ese registro, en esa fecha en particular, y Tengo que decir que estoy muy impresionado con los principales sellos discográficos. Como abogado que representa a un artista, luché contra ellos en varios temas. Pero hace años Pude obtener uno de los planes de marketing de mi cliente de su sello discográfico. Y mientras hojeo el plan de marketing Me sorprendió ver todas las diferentes áreas. La empresa realmente había estructurado actividades específicas de marketing y promoción. Y el costo de esas actividades, El costo de tener un anuncio en una revista nacional, miles de dólares. El costo para iniciar un club de fans de Internet o para promocionar a través de Internet, miles de dólares. El costo para pagar los gastos del artista a aparecer en una transmisión nacional o internacional o talk show televisado, miles de dólares. Entonces, es sorprendente ver que las compañías discográficas no solo pueden cubrir esos gastos, pero también tener una red de distribución que les permita tener ese récord en las tiendas, a nivel nacional e incluso internacional, en una fecha particular. Porque, como ese marketing está enfocado para esa fecha de lanzamiento, tienen el mecanismo. Y ciertamente desea su audiencia, a quién se dirigió ese marketing y promoción a, estar en condiciones de entrar a la tienda y recoger ese registro en esa fecha. Una tendencia creciente en el mundo de la música hoy es para etiquetas independientes para comenzar de forma independiente distribuida. Y a medida que construyen su mercado, afiliarse a las principales discográficas únicamente para su distribución. Este mercado realmente ha despegado en los últimos años, como has tenido grandes e independientes sellos independientes como Big Machine Records, que publica las grabaciones de Taylor Swift. Grandes bandas como Mumford \& Sons trabajando con la etiqueta independiente Glass Note que ahora es distribuida por un mayor. Fin de semana de vampiros con el grupo de mendigos. Adele también está afiliada al Beggars Group, y Es una fuerza tremenda en la industria.\\
Macklemore y Lewis tenían su propia etiqueta que se volvió tan populares por sí mismos que terminaron afiliando solo por propósitos de distribución a través de una de las principales etiquetas. Esa es un área en crecimiento de la industria. Pero ahora, veamos esto desde la perspectiva histórica. De vuelta en los años 50 y 60, e incluso en los años 70 hasta cierto punto, la mayoría de los artistas de grabación tuvieron que tener una etiqueta para que su producto llegue al mercado. Estoy hablando del pre-sintetizador, la pre-caja de ritmos, la edad previa a la muestra, cuando entraste en un estudio como artista, generalmente la mayoría de los artistas no tenían dinero para pagar el costo del tiempo de estudio, la contratación de un ingeniero. Pagar todos los reproductores de sesión que necesitabas para armar tu grabación. En aquellos días, en realidad tenía que tener un pianista o un tecladista. Tenías que tener un baterista, quiero decir, un verdadero baterista. Tal vez un bajo en la sección de ritmo o una guitarra real en la sección de ritmo. Y luego, después de terminar las pistas de ritmo básicas, tenías algo eso fue llamado, y todavía se llama hoy, endulzar. Cuernos y cuerdas. Jugadores individuales que tocaban instrumentos reales que tenían que venir al estudio. para colocar las pistas para completar la grabación. Además de eso, La mayoría de los artistas no tenían el dinero para fabricar sus discos. Confiaron en las principales discográficas, no en las principales discográficas, sino en un sello discográfico, sello discográfico independiente en ese momento también, para pagar El costo de fabricación. Y ciertamente no tenían el sistema de distribución cuando era estrictamente un mercado físico. Esa fue una de las formas en que las etiquetas tenían un dominio absoluto sobre el artista. Para llegar al mercado, más o menos, tenías que pasar por la relación de distribución de una etiqueta. Pero varias cosas han cambiado en los últimos 40, 50 años. Bueno, ¿cómo han cambiado? Bueno, ¿ahora realmente tienes que pagar los costos de grabación de un estudio?. Muchos artistas tienen estudios en sus habitaciones, en sus sótanos. ¿Tienes que tener un ingeniero? Gran parte de la ingeniería se puede hacer en su computadora portátil, ¿Tienes que fabricar un CD para ser lanzado?. No, puedes subir el archivo a SoundCloud o muchos de los otros tipos de salidas de internet que se pueden usar hoy en día. ¿Qué pasa con el marketing? Los sellos discográficos también proporcionaron marketing que el artista no podía pagar hace años. Bueno, los costos de marketing se han reducido como resultado de las redes sociales. puntos de venta en la creación de videos virales, incluso, YouTube, crear un tremendo y tremendo marketing y promoción para el artista. En otras palabras, tal vez realmente no sea necesario tener un sello discográfico hoy.\\
Una vez más, emprendimiento e innovación, entonces, si tiene el talento para poder grabarse y armar todo las piezas de la grabación, podría estar en una excelente posición para estar en el negocio. Recuerdo que Stevie Wonder años atrás fue uno de los primeros artistas tocar todos los instrumentos como sintetizadores. Y fue realmente notable durante ese tiempo porque los artistas estaban mirando el hecho de que tuvieron que traer varios músicos de sesión para tocar. Pero cuando entró el synclavier, entraron las cajas de ritmos, Stevie fue uno de los primeros artistas en grabar toda la música por su cuenta, y ese tipo de set la industria de la música en el camino de las personas capaces de crear su música a muy bajo costo utilizando todos estos nuevos avances tecnológicos. Entonces, ¿dónde estamos ahora? estamos en la era del bricolaje, hazlo tu mismo. ¿Puedes crear ese producto por tu cuenta? Tal vez tu puedas. ¿Puedes explotar todos los diversos medios de comunicación social para crear realmente un bombo, crea un gran video de YouTube que realmente va a maximizar tu exposición a la audiencia que quieres tener? ¿Vas a estar en una posición no solo para crear ese bombo inicial, y desarrollar esa audiencia inicial, pero ¿podrán desarrollar esa audiencia? y expandirlo al siguiente nivel? Eso es principalmente lo que las principales discográficas todavía dicen hasta hoy. Eso espera un minuto. Cuando realmente quieras llegar al mercado más amplio, ven a nosotros. Eso es lo que hacemos, podemos llevarlo de este nivel al siguiente. Y muchos artistas de bricolaje sienten esa presión una vez que tengan esa audiencia inicial de recepción y éxito. Si deben o no pasar al siguiente nivel con las etiquetas principales. O si deberían continuar intentando hacerlo usted mismo, bricolaje.\\
Bueno, aquí está la dificultad con el bricolaje, para que puedas pasar de ese nivel al siguiente, yo digo que es muy difícil hacerlo tú mismo, es muy difícil tener un bricolaje amplio, exposición masiva, exposición mundial. ¿Por qué? Porque si estás involucrado en el proceso creativo, y estás escribiendo tus canciones y estás grabando tus canciones y te gusta ese ritmo creativo, es un poco difícil tener que apagar eso y entrar a Twitter y publicar y actualizar, cortar un video de YouTube, publícalo y monitorízalo para ver dónde está realmente tendencia, y si eres tendencia o no. Se puede quitar del proceso creativo. Muchos músicos de bricolaje ahora están encontrando, en orden, y anticipo que los artistas de bricolaje llegarán al punto en que van para descubrir que necesitan a otras personas que les ayuden a llegar al siguiente nivel. En otras palabras, no es bricolaje, será DIO. Hazlo nosotros mismos. Y con eso quiero decir que pueden señalar a alguien que es un importante negocio de música y tiene la experiencia comercial para ayudar en la gestión de lo que están haciendo. Pueden encontrar a alguien experto en redes sociales y traerlos a su equipo para ayudarlos a expandir su carrera, pueden encontrar a alguien que tenga contactos con patrocinios y endosos para ayudarlos a atraer. Necesitaban financiación para que pudieran emprender otras empresas. En nombre de no solo esas empresas sino también sus propias empresas, sus propios productos, así que creo que nos estamos moviendo a una etapa no solo de bricolaje. Creo que nos estamos moviendo a una nueva etapa de DIO. Hazlo nosotros mismos. Creo que es importante para los nuevos artistas que tienen la capacidad de hacer esto. Rodearse del tipo de equipo de profesionales y expertos, en una gran cantidad de áreas. Para llevar su música y, lo más importante, su negocio, al siguiente nivel. Puedes ser ingeniero y créanme, el poderoso producto sigue siendo una de las grandes P's. Necesitarán personas que puedan poner y mezclar las grabaciones. que se mantiene con el tiempo, puedes ser un experto en redes sociales.  Los artistas necesitarán grandes personas para ayudarlos a navegar por Internet y la comercialización y promoción a través de esos medios. Si eres compositor, si eres un músico que también puede brindar servicios en el estudio. Un gerente, alguien que puede coordinar todas estas actividades en nombre del artista. Alguien que pudiera rastrear y organizar patrocinios y patrocinios. Tal vez puedas crear una nueva aplicación tecnológica eso proporcionará algún tipo de beneficio para expandir la carrera del artista. Todas estas vastas áreas y oportunidades solo crecerán a medida que más y Cada vez más personas aprovechan la nueva tecnología y la nueva era digital.
\subsection{Nuevos innovadores del milenio: ITunes y Spotify}
\subsubsection{iTunes}
Después de que el sitio de intercambio de archivos Napster volcó el negocio minorista de música tradicional y cesó sus operaciones debido a las acciones de infracción de derechos de autor, las principales empresas idearon alternativas digitales en 2002-2003. Press Play (Universal y Sony Music) y Music Net (Warner, BMG y EMI) finalmente fallaron debido a los engorrosos esquemas de administración de derechos digitales, los catálogos limitados, las interfaces de usuario torpes y los altos precios. Steve Jobs de Apple explotó el mercado en un punto de inflexión de la era digital de la industria de la música. Primero vino el iPod y luego iTunes. iTunes se convirtió rápidamente en la tienda de discos más grande del mundo (con Walmart en segundo lugar).
\subsubsection{Spotify}
Spotify (y servicios similares) están encabezando otra revolución en las ventas de música. Estos servicios interactivos encabezan la transformación del modelo de negocio de un modelo basado en productos a un modelo basado en acceso. Muchos profesionales de la industria piensan que esto podría permitir que el negocio sea mucho más grande que nunca. Algunos predicen que podría convertirse en un negocio de \$ 100 mil millones en los próximos 10 años.

\chapter{Registro de acuerdos}
\section{Registro de acuerdos desmitificados}
Esta lección va a ser muy interesante por que tratará de algo que es muy importante, no sólo para las compañías disqueras sino también para los artistas y esto es el contrato de grabación exclusiva del artista El contrato de grabación exclusiva del artista Durante la segunda mitad del último siglo, e incluso en este siglo, había pocas empresas. De hecho, actualmente hay tres grandes conglomerados que controlan cerca del 85 al 90\% de la distribución de al menos los registros físicos. Bueno, eso está cambiando ahora con el internet que es tan importante y tan emocionante para nosotros mientras crecemos en esta era digital. Así puedes tener un artista, un músico, que puede crear su propio registro, su propio vídeo. Acceso con iTunes. Ventas, a través de YouTube, por transmisión.\\
Usar un recopilador de contenido como CD Baby o Nimbit o TuneCore para tener todas las canciones en diferentes outlets digitales. Así que es importante, incluso para los músicos independientes, tener un contrato que realmente explique los derechos y obligaciones de su compañía con él mismo en caso de que sea músico independiente Por ejemplo, supón que decido grabar una canción y lo puedo hacer en mi portátil. Y también podría crear un vídeo. Bueno, estoy en el negocio entonces. Tengo música de John Kellogg. Y es importante entonces que este artista independiente firme contrato con esta compañía, ¿por qué? Porque los contratos implican ciertos derechos y obligaciones. que son muy, muy importantes para el negocio. Así que esta lección va a hablar de los derechos y obligaciones. Cómo cobra un artista y una vez más, cada vez que su música es reproducida, alguien recibe dinero y los contrato de grabación exclusiva determinan cómo. Cuánto y cómo vas a conseguir pagado. Por lo tanto es muy importante comprender los términos. No es ciencia espacial, no es fácil. Recuerdo cuando empecé a ejercer la ley y leí mi primer contrato de grabación y dije: no entiendo estas cosas, pero me dije: voy a leer más y más sobre estos contratos, y estoy seguro de que con el tiempo que voy a tener una mejor comprensión. Bien no es ciencia espacial, como dije. Así que voy a tratar de descifrar algunos de los términos y podrás ver los contratos en el texto. el texto está escrito de una manera donde tienes aspectos legales por un lado, y la anotación en otras palabras, intentando descifrar lo que aspectos legales están diciendo, en términos sencillos así que por lo menos lee los términos sencillos del contrato para que lo entiendas mejor. Ahora, también vamos a ver qué está sucediendo en la era digital, Cómo están evolucionando los contratos y lo que va a suceder en el futuro. a qué es lo que debes prestar atención, a qué debes ejercer presión para asegurarte de que los artistas, compañías y creadores sean compensados justamente por estos trabajos mientras la era digital crece y mientras estos trabajos se usen de tantas formas.
\section{La importancia de registrar acuerdos}
Cuando comencé a practicar leyes hace más de 30 años, Los acuerdos de grabación eran documentos bastante simples. Tal vez 10 o 12 páginas que básicamente hablaron el anticipo que recibiría el artista, las regalías que reciben. La configuración predominante de grabaciones, La venta de grabaciones en ese momento consistía en discos de vinilo y casetes. 10 o 12 páginas. Bueno, ha cambiado mucho. Los acuerdos de grabación han evolucionado a lo largo de los años, ahora los acuerdos llegan a las 70 páginas. Siempre traigo esto a clase y lo dejo caer en el podio para enfatizar, solo para que los estudiantes sepan que este es un documento bastante serio y muy grueso. ¿Ahora por qué, dices, por qué 70 páginas? ¿Por qué 70 páginas? Bueno, lo que pasó es por cada avance tecnológico. En la industria de la música, estos contratos se han alargado cada vez más. ¿Por qué? Porque hay más páginas agregadas,se agregaron más disposiciones para cubrir los derechos y obligaciones que resultan de estos nuevos avances tecnológicos. Por ejemplo, cuando comencé a practicar en 1980, Los CD ni siquiera habían ingresado al mercado, pero en un par de años lo hicieron. Se agregaron cuatro o cinco páginas para tratar los CD y los derechos y obligaciones relacionadas con que el CD se convierta en una nueva configuración en la industria de la música. También a principios de los 80 tuviste otro avance tecnológico, televisión por cable y un nuevo canal de cable, Music Television, MTV.
Entonces las compañías discográficas tuvieron que comenzar a producir vídeos musicales. Se agregaron otras cinco o seis páginas para dar cuenta de eso. Y siguió y siguió. A medida que entramos en los años 90 tuviste el desarrollo, en la última parte de los 90, de la era digital. Y párrafo por párrafo, avance tecnológico por avance, tenía páginas adicionales agregadas a estos contratos.\\
Muy, muy importante. ¿Por qué? Porque el propósito de un exclusivo acuerdo de grabación es establecer y documentar el valor del activo de la empresa. ¿Qué quiero decir con eso? ¿Qué son los activos? Los activos son cosas de valor. ¿Y qué hacen los contratos exclusivos de artistas de grabación? Bueno, transmiten derechos y obligaciones a ambas partes. Así que vamos a hablar sobre dos de las cosas que creo que crean valor para una compañía discográfica y para un artista en un acuerdo exclusivo de artista discográfico. \textbf{Primero es lo que llamo el activo más valioso en la industria de la música, cual es el copyright.} Ahora vamos a tratar los derechos de autor en la próxima lección en gran medida, así que no me detendré demasiado en serio esta vez. Pero al hablar sobre el acuerdo exclusivo para artistas de grabación, \textbf{Es importante tener en cuenta que las compañías discográficas quieren los derechos de los derechos de autor en las grabaciones de sonido.} \\
Hay varios tipos de derechos de autor bajo la ley, y trataremos con dos o tres en este curso que realmente se relacionan a la industria de la música tal como se encuentra actualmente. Las compañías discográficas están preocupadas por los derechos de autor de grabación de sonido. ¿Qué es el copyright de grabación de sonido? ¿Qué es una grabación de sonido? Esa es una grabación de una composición, y una composición es otro derecho de autor que trataremos en la próxima lección. Pero la grabación de sonido es la fijación de sonidos. que componen la grabación de una composición. ¿Qué quiero decir con eso?, bueno, tendrás una canción básica que grabará un artista. Un arreglista se involucrará, un productor se involucrará. Podrían entrar otros músicos. Es posible que tenga una sección de ritmo, es posible que entren cadenas o que use cadenas sintetizadas. Vas a tener un arreglo. Vas a tener a alguien entrenando al vocalista sobre la interpretación de la canción. Eso es básicamente lo que hace el productor. Todas estas personas están involucradas en la producción de esta grabación de sonido, La fijación de los sonidos que componen la versión de la composición. Las compañías discográficas quieren poseer los derechos de los derechos de autor en la grabación de sonido. ¿Qué es un copyright? Bueno, \textbf{un copyright es una propiedad intelectual.} Es como ser dueño de bienes raíces. Por supuesto, no es real en absoluto porque está en el aire. No puedes poner tu dedo encima. Se llama propiedad intelectual. Es la creación de la mente y el talento también. Entonces, un artista que actúa en una grabación, por ejemplo, Actué con el grupo Cameo, y las inflexiones de mi voz y mi trabajo creativo en el estudio, es algo a lo que la compañía discográfica quiere los derechos. Y bajo el acuerdo exclusivo de artista de grabación, eso es exactamente lo que sucede. El artista y todos los involucrados en el proceso de creación de los sonidos que están fijos en que la grabación de sonido debe transferirse a la compañía discográfica porque la compañía discográfica quiere poseer esos derechos en la grabación de sonido. ¿Por qué quieren poseer los derechos en la grabación de sonido? ¿Cuál es el valor de tener ese copyright de grabación de sonido? Bueno, los derechos de autor pueden durar toda la vida más 70 años. Hay otros derechos de los que hablaremos en la próxima lección sobre copyright. Entonces, si grabo y produzco mi propia canción en el estudio de mi casa y estoy listo para lanzarlo, y quiero lanzarlo a través de mi compañía, John Kellogg Music,lLuego voy a transferir mi interés de copyright a mi empresa, John Kellogg Music, y \textbf{serán dueños de los derechos de autor de mi vida más 70 años. Sujeto a algunas limitaciones de las que hablaremos, pero después de esa vida más un período de 70 años, entonces ese derecho de autor cae en lo que se llama dominio público, lo que significa que cualquiera puede usar ese copyright y no tiene que obtener permiso de nadie.} \textbf{Con un derecho de autor, eso te da el monopolio de esa propiedad intelectual. Nadie más puede usar esa propiedad. Nadie más puede usar esa grabación de sonido sin obtener su permiso durante La vida del copyright.} entonces, ¿por qué las empresas quieren tener grabaciones de sonido? ¿Por qué quieren tener ese derecho?, y  necesitas saber esto también. Las compañías discográficas no te requieren asignarles su interés en derechos de autor por nada. No, esta es la razón por la que tiene un acuerdo. La compañía discográfica dice asignarnos ese derecho de autor y a cambio de que nos asignes los derechos de autor, le pagaremos regalías por asignarle ese interés. Las compañías discográficas quieren poseer esas grabaciones de sonido porque durante la vida de derechos de autor, esas grabaciones de sonido se pueden utilizar de varias maneras. Solo te voy a dar un par de maneras. Uno, se pueden vender. Esas grabaciones de sonido se pueden vender en diferentes tipos de configuraciones. Descargas digitales. Se pueden vender en CD. También se pueden usar de otras maneras. Se pueden usar con fines de transmisión. Entonces, la compañía discográfica está en posición, al poseer ese copyright, negociar con las personas que quieren usar esas grabaciones. Por ejemplo, servicios de transmisión, Spotify, iRadio y otros, tienen que obtener el permiso de los propietarios de las grabaciones de sonido para transmitir esas grabaciones de sonido. \\
Y aquí hay otro uso que se está volviendo cada vez más importante en la música. industria hoy, y ese es el derecho a sincronizar esas grabaciones de sonido a una imagen de vídeo. Eso se llama la sincronización correcta. ¿Y a qué me refiero con sincronizar? Eso significa en secuencia de tiempo tomar una grabación y ponerlo en quizás un comercial, una película, el fondo de una película o un programa de televisión o un vídeo en Internet. Muy, muy importante derecho. ¿Por qué? Porque \textbf{las personas que quieren usar y firme estas licencias de sincronización para que puedan usar las grabaciones de sonido en Estos diversos medios generalmente tienen que pagar una tarifa.} Y es una fuente de ingresos muy, muy importante para las compañías discográficas, entonces, la transferencia de los derechos de autor es una de las cosas, una de las cosas de valor en tener un acuerdo de grabación. Establece el derecho en la compañía discográfica, y esa podría ser tu propia compañía discográfica como artista. ¿Por qué, como artista, deberías tener tu propia compañía discográfica?, porque en algún momento es posible que desee hacer un trato con una etiqueta importante, un distribuidor de internet, es posible que desee licenciar sus grabaciones a otras personas. Y en ese momento desea establecer esos intereses de propiedad en una entidad que trabaja para ti. ¿Por qué? Porque en el sistema estadounidense de capitalismo, A las empresas les gusta crear valor en sus activos, y Los derechos de autor son activos que generalmente crecen en valor. Cuanto más explotas esos  derechos de autor en varios medios, construye el valor cada vez más. Una de las compañías más notables en la historia de La industria musical actual es Motown Records. Las grabaciones maestras de Motown son muy, muy valioso, porque los escuchará en comerciales todo el tiempo. Los escucharás en las películas. Entonces, el acuerdo exclusivo para artistas de grabación, uno, transfiere la grabación de sonido interesa a la empresa por derechos de autor, lo cual es muy importante. Y dos, permite que el artista presente su exclusivo servicios de grabación por un período de tiempo o un determinado compromiso de producto. Eso podría ser de cuatro a seis álbumes. ¿Por qué es eso importante? Bueno, publico mi disco por mi cuenta a través de la distribución digital. Comienza a tener muchas redes sociales, se convierte en una sensación viral. Y digamos que una etiqueta importante se acerca a mí y me dice: realmente nos gusta lo que haces, John, y nos gustaría firmar un acuerdo. Bueno, es importante para mí decir, bueno, tengo mi propia compañía, John Kellogg Music, y tengo un contrato con mi compañía donde grabaré cinco álbumes. Ahora he grabado el primero y lo lancé y esa es la que generó todo el calor y la razón por la que quieres firmarme. Pero tengo cuatro álbumes más bajo mi contrato, así que tratemos. Mi compañía puede tratar con usted sobre esa base. Y te pone en una posición de negociación mucho mejor para tener eso. \textbf{¿A qué me refiero con servicios de grabación exclusivos? Bueno, eso significa que no puedo grabar para nadie más, cualquier otra empresa durante ese período de tiempo.} Eso es muy importante, muy importante. Y hablamos sobre la naturaleza evolutiva de los acuerdos de grabación. Creo que es importante hablar sobre el hecho de que las compañías discográficas vuelven a principios de los años 80 hizo cumplir esa disposición,\textbf{ La disposición de exclusividad en los contratos es bastante severa.} Recuerdo a principios de los 80 cuando comencé a practicar y Yo representaba a los O'Jays. Y tuvieron la oportunidad de grabar con Stevie Wonder, y de hecho, grabó con Stevie Wonder. Y en ese momento Stevie Wonder estaba en la cima de su juego, los O'Jays estaban en la cima de su juego, y fueron al estudio y Stevie grabó una canción llamada All I Do.\\
Todo lo que hago, excelente disco, por cierto, excelente, excelente disco. Stevie toca todos los instrumentos y, por supuesto, sabes que es un músico fabuloso, así como un gran, gran vocalista y compositor. Y los O'Jays no solo proporcionaron voces de fondo para él, pero durante algún momento de la grabación, los O'Jays hicieron lo que se llama en la industria, salieron. Y con eso queremos decir que comenzaron a improvisar y comenzaron a improvisar. sobre ciertas pistas, hasta el punto de que podrías reconocer eso, oye, ese es Eddie Levert en esa pista. Wow, los O'Jays están grabando con Stevie Wonder. Dos superestrellas, ¡esto es fantástico! Walter Williams, el otro cantante principal de los O'Jays, hizo su característica marca lame a lo largo de la canción. Fue fantástico. Recuerdo que los O'Jays trajeron el, en ese momento, cinta de casete con ellos y lo estábamos escuchando. Y todas las personas en su campamento estaban realmente emocionadas. Esto nunca se ha hecho antes, superestrellas que se unen para grabar pistas. Bueno, resulta que en ese momento las compañías discográficas estaban realmente preocupadas sobre permitir que sus artistas graben en las pistas de otros artistas porque pensaban que canibalizaría la venta de las grabaciones de sus propios artistas. Hasta el punto de que la compañía discográfica O'Jays no accedió a permitir las voces de los O'Jays para ser utilizadas, por lo que tuvieron que tirar de ellas. No podían mezclar esas pistas en él, lo que para mí fue lo que realmente hizo que la grabación fuera especial. Pero a los O'Jays se les permitió hacer la voz de fondo. Y el otro vocalista de fondo en esa canción fue Michael Jackson, pero fue firmado con Motown, o al menos Motown tuvo que darle el permiso para grabar con Stevie Wonder. Así que creo que eso muestra que hubo una vez que Las disposiciones de exclusividad en los contratos se cumplieron muy estrictamente. Y \textbf{es importante que sepa que si firma con una compañía discográfica, simplemente no puedes ir a grabar con nadie sin obtener el permiso de la compañía discográfica con la que has firmado.} \textbf{Eres su artista de grabación exclusivo. Ahora esa situación ha cambiado.} Una vez más, estos contratos evolucionan. A medida que te metiste en el género del rap, realmente ganaste una gran importancia en los años 90 y en la primera parte de este siglo y actualmente, comenzaste a tener los artistas proporcionan voces invitadas, voces de apoyo, en varias grabaciones. Y ahora, si miras en las listas actuales de Billboard, posiblemente encuentra entre los diez primeros, tal vez cinco o seis de los diez principales artistas invitados. Y \textbf{las compañías discográficas comenzaron a aflojar su restricción sobre las disposiciones de exclusividad, y realmente comenzaron a permitir que sus artistas aparecieran en las grabaciones de otros artistas. Por supuesto, exigieron una cierta regalía, o porcentaje de una regalía, cuando aparece su artista invitado.} Pero también creo que es importante tener en cuenta que las compañías discográficas encontraron valor en sus artista promovido como resultado de ser artistas invitados en otras grabaciones. Entonces, una vez más, la transferencia de los derechos de autor de grabación de sonido, dos, la provisión de los servicios de grabación exclusivos del artista a la compañía discográfica, son cosas muy importantes que construyen el valor del activo de la compañía discográfica.
\section{Disposiciones contractuales}
Muy bien, ahora es tiempo de que hablemos y examinar algunas disposiciones clave en el acuerdo exclusivo para artistas de grabación. Ahora, una vez más, realmente quiero que te refieras a tu texto, asegúrese de leer el texto del acuerdo exclusivo de artista de grabación. Y asegúrese de leer al menos el lado derecho, el lado anotado, de cada página de ese acuerdo para que te dé una idea. Por supuesto, si eres un artista o si recién estás comenzando tu propio disco compañía o tiene el deseo de comenzar su propia compañía discográfica, siempre tiene que haga que un abogado experimentado en entretenimiento revise el acuerdo por usted. Pero sigue siendo importante para tener una idea de lo que dicen algunas de estas disposiciones. Había un gran minorista, su nombre era Sy Syms, que comenzó las cadenas de tiendas por departamentos Syms, que eran muy grandes aquí en la costa este. Y Sy Syms acuñó un anzuelo muy interesante y una frase muy interesante. Sabes que soy grande en ganchos. Cada vez que suena la música, a alguien le pagan. Se les paga, no se juega. Bueno, a Sy Syms se le ocurrió un gran gancho, y es decir, un consumidor educado es mi mejor cliente. Los grandes almacenes Syms vendieron ropa y Grandes marcas a precios tremendamente rebajados. Eran realmente estilos que podrían haber sido una temporada pasada de moda, y los comprarían y los venderían en sus tiendas a precios reducidos. Algo arrinconó el mercado en el noreste. Fue fantástico. \textbf{Bueno, digo que un músico educado es mi mejor cliente.} ¿Y a qué me refiero con eso? Esa es una de las razones por las que escribí mi libro. Quería que mis clientes entendieran el tipo de problemas con los que estaba lidiando de su parte. Ahora ha habido algunos cambios significativos que han ocurrido en la grabación de acuerdos incluso desde principios de este siglo.\\
Como puedes imaginar, artistas, y no es ningún secreto, compositores, productores y otros talentos creativos, fueron tremendamente estafados en las primeras partes desde los años 20 hasta los años 50 y 60. No fue sino hasta la década de 1970 que incluso hubo una profesión de derecho del entretenimiento hasta cierto punto, para abogados de entretenimiento que representaban artistas para ayudarlos a obtener mayores derechos en sus grabaciones y en sus obras protegidas por derechos de autor. Pero durante ese período de tiempo, durante los 50 años entre 1950 y 2000, Los contratos de grabación fueron muy confusos. Eran muy confusos. Posiblemente intencionalmente confuso.\\
Muy no transparente fue muy difícil para los artistas en la lectura de contratos poder entender cómo les iban a pagar y cuáles son sus tasas reales de regalías. Así que vamos a discutir un par de las disposiciones clave. Vamos a hablar sobre el término. Vamos a examinar el término, cuánto dura el contrato. Vamos a hablar sobre los diferentes tipos de regalías. que los artistas tienen derecho bajo acuerdos exclusivos de artistas de grabación. Avances de los que hablaremos, los avances son muy importantes. \textbf{Los anticipos son montos de regalías que se pagan por adelantado.} \textbf{Los artistas pagan ciertos montos en dólares que se consideran anticipos contra regalías, lo que significa que cuando el artista comienza a ganar regalías, esas regalías son mantenidos por la compañía discográfica hasta que se recuperan o recuperan los anticipos.} Y también vamos a hablar sobre las \textbf{nuevas disposiciones en la grabación. acuerdos que realmente comenzaron a desarrollarse a principios de este siglo, Disposiciones de acuerdo de 360 grados, y hablaremos de eso a medida que pase el tiempo. Lo primero que debe saber es que hoy en el mercado actual, muchos artistas que son descubiertos por las compañías discográficas y no he tenido experiencia de grabación alguna vez se firman para acuerdos de desarrollo o incluso acuerdos individuales. Y esas ofertas generalmente proporcionan una cantidad muy baja de anticipo y Una promesa de una tasa de regalías. Y le da al artista tanto como un compromiso de pagar ciertos costos de grabación para un cierto número de demos o cierto número de sencillos para ver si el artista realmente vende. Si el material del artista se vende, entonces en ese momento la compañía discográfica tiene disposiciones que les permiten elegir opciones para varios álbumes.} Y hablaremos de eso un poco más tarde. Hablemos sobre el término, el término del acuerdo de grabación. \textbf{¿Cuánto dura el acuerdo de grabación? En los años 60 y 70, los artistas firmaron acuerdos de grabación con términos que fueron descritos en un plazo de años. Por ejemplo, el artista firmaría un contrato para un año con cuatro opciones de un año. Qué significa eso? Eso significa que el artista está firmado para grabar durante un año. Si a la compañía discográfica le gusta lo que está haciendo el artista y su producto se vende en grandes cantidades, entonces la compañía discográfica tiene la opción de retomar el segundo año.} Y si les gusta lo que hicieron en el segundo año, pueden retomar el tercer año, y así sucesivamente hasta el quinto año. Entonces, cuando los artistas en esos días miraban esos contratos, posiblemente decían, wow, tengo un contrato de cinco años con una compañía discográfica. Bueno, ese no es necesariamente el caso. La compañía discográfica no tiene que retomar el segundo año. Y aquí hay algo más que confundió al artista en ese momento. Aunque el contrato decía que era un año, el plazo era un año con cuatro opciones de un año, lo que el artista podría no haber visto en las siguientes páginas eran una disposición que establecía que durante ese año de contrato, la compañía discográfica tenía derecho a grabar un álbum en el artista. Pero también tuvieron una opción dentro de ese primer año para pedirle al artista que grabe un segundo álbum. Y si el artista no entregó ese segundo álbum dentro de ese período de un año, ese período de un año se extendió hasta que completaron el segundo álbum. Eso se volvió bastante problemático cuando entramos en los años de MTV y se hicieron vídeos. Porque cuando salió MTV y los vídeos se hicieron muy populares, grabar las compañías no querían que un artista lanzara un álbum, pero cada dos o tres años, porque querían publicar un vídeo en cada sencillo, tres o cuatro sencillos. Y trataron de vender ese álbum en el transcurso de dos años, y muchas veces las ventas del álbum continuarían desarrollándose durante un período de dos años. Entonces, algunos de esos artistas que firmaron ese tipo de contratos pensé que fueron firmados por cinco años. Como resultado, la compañía discográfica podría haber decidido grabar dos álbumes durante cada año, y el año se extendió hasta que se entregó el segundo álbum. Y el segundo álbum podría no haber sido entregado por dos años después del comienzo de ese período.\\
Los artistas comenzaron a involucrarse en contratos que duraron varios años, diez años o más De hecho, Prince, el gran artista Prince, firmó su primer contrato en 1977 o 1978, y el contrato no finalizó hasta que negoció un acuerdo del contrato en 1995 o 96 con Warner Bros Records, y en ese momento todavía le debía dos álbumes. Ahora, por supuesto, el contrato posiblemente había sido renegociado dando a Warner Bros los derechos de álbumes adicionales. Pero eso te da una idea de la duración de los contratos y por qué los artistas, particularmente a finales del siglo pasado y Al comienzo de este siglo, estaban realmente molestos por las disposiciones que posiblemente requirió que se quedaran con una compañía por un período prolongado de tiempo, posiblemente toda su carrera. Ese fue un ejemplo de la falta de transparencia de grabar acuerdos de artistas y la naturaleza confusa, particularmente de su disposición sobre el término del acuerdo exclusivo del artista discográfico. Ahora sí quiero decir que las compañías discográficas, y como resultado de ciertas demandas judiciales que se presentaron y ciertas decisiones judiciales que se dictaron, \textbf{las compañías discográficas en los años 80 y 90 comenzaron a cambiar esa disposición. En lugar de ser un año más cuatro opciones de un año, decidieron cambiar el período de tiempo a lo que se denominó períodos.} Tendrían varios períodos para el contrato, \textbf{lo que significa el primer período sería el tiempo que tomó grabar y lanzar un álbum.} Además, tal vez siete meses después para que la compañía discográfica vea cómo le fue en el álbum el mercado, en ese momento podrían elegir la opción para el segundo período. Y durante cada período, el artista debe grabar un álbum. Así que eso creó una situación en la que los artistas aún estaban atados tal vez de ocho a diez períodos para contratos celebrados en los años 80 y en los años 90, que todavía es un período de tiempo muy largo. Entonces las cosas han evolucionado en acuerdos de grabación exclusivos. En la última parte de los 90, los artistas realmente se frustraron mucho. Y Prince fue uno de los principales defensores.\\
Algunos de ustedes podrían haberlo visto en The Today Show a mediados de los 90, cuando dejó de llamarse Príncipe y solo pasó por un símbolo. Y cuando estaba en el Today Show, tenía un esclavo escrito a un lado de su cara. Y parte de su punto era que sentía que las compañías discográficas que te contrataron a acuerdos tan largos te trataban como a un esclavo porque tenían los derechos exclusivos de sus servicios de grabación para un período de tiempo tan extenso Para él realmente no tenía sentido. Eso encendió a la comunidad de artistas a finales de los 90 para comenzar realmente abogando por más justicia, más transparencia con acuerdos exclusivos para artistas de grabación. Y a principios de este siglo, hubo ciertas acciones tomado por la legislatura de California que cuestionó estos acuerdos y amenazó a la industria discográfica con legislación eso haría los contratos más transparentes, y como resultado de eso, las compañías discográficas entendieron el punto.\\
A principios de este siglo realmente comenzaron a reconocer que, bueno, tal vez sus contratos eran un poco irracionales, pidiendo posiblemente diez álbumes de un artista. Y voluntariamente, aunque realmente no fue voluntario, lo hicieron como resultado de la presión de la comunidad de artistas, las compañías discográficas comenzaron a cambiar las disposiciones de plazo de sus contratos. En lugar de pedir un potencial de diez álbumes de un nuevo artista, de ocho a diez álbumes de un artista, redujeron eso de cuatro a seis álbumes por término, lo que les daría varios períodos para grabar de cuatro a seis álbumes. ¿A qué me refiero con cuatro o seis álbumes? Bueno, un artista que tiene mucha influencia, ¿y qué quiero decir con influencia? Me refiero a la influencia. Y con eso te daré un ejemplo. Digamos que un artista tiene varias vistas de YouTube, millones de visitas a YouTube, y han creado una gran audiencia por su cuenta. Bueno, ese artista tendrá mucha influencia, mucha influencia, si entra una etiqueta importante y quiere negociar con ellos. El sello principal podría ofrecer primero firmarlos para un contrato de seis álbumes. Y, por supuesto, el abogado del artista en esa situación sería prudente decir que no, No queremos grabar seis. Grabaremos tres álbumes para usted. Y luego, por supuesto, la compañía discográfica va a contrarrestar tal vez cinco, y eventualmente podrían estar de acuerdo con cuatro. Entonces, depende de tu influencia en cuanto a cuántos álbumes vas a ir para terminar grabando bajo el acuerdo exclusivo de artista de grabación. Por supuesto, el sello discográfico querrá que grabe tantos como sea posible. Y, por supuesto, como artista, quieres grabar la menor cantidad posible, así que si tienes éxito durante ese período de cuatro o cinco álbumes, tendrás la oportunidad de probar el mercado y ver si hay otro compañías que están interesadas en grabarlo bajo un nuevo acuerdo de grabación. Entonces, el término, cuánto dura el acuerdo, es una disposición en evolución del acuerdo exclusivo para artistas de grabación. Recordarán cuando dije antes que los contratos de los años 60 y Los años 70 tenían un año más cuatro opciones de un año y durante cada año de contrato, el artista podría tener que grabar dos álbumes, un potencial de diez álbumes. ¿Ahora quién tiene la opción, quién tiene la opción? ¿Quién tiene derecho a elegir la opción? Por lo general, la compañía discográfica tendrá derecho a elegir la opción.\\
\textbf{Las compañías discográficas dicen que necesitan tener el derecho de recoger las opciones porque han invertido mucho dinero en el artista, en la grabación, produciendo, comercializando, distribuyendo el registro.} \textbf{Por lo tanto, deberían tener el derecho, si el registro es exitoso, a recoger la opción para que puedan intentar obtener algún beneficio de esa inversión inicial.} Sin embargo, como puede ver, \textbf{esta disposición del contrato en particular está evolucionando, y ha evolucionado y continúa evolucionando. Ahora su contrato con su empresa probablemente debería ser por la norma, de cuatro a seis álbumes.} Para que si quieres hacer un trato con una etiqueta importante, está en condiciones de asignar esos derechos, si eso es lo que tiene que hacer, a la etiqueta principal para la cantidad máxima de álbumes que tiene. Pero una vez más, cuando digo que esa es la norma, eso no significa que esa sea la única forma en que puede ocurrir la situación.\\
\textbf{Al igual que estos contratos han evolucionado en los últimos años, evolucionarán aún más en el futuro. Y el panorama digital realmente está cambiando la forma en que se hacen las cosas. Y la clave es que todo se reduce al apalancamiento.}  Si ha vendido muchos productos, si se vuelve muy popular y convertirse en un activo muy importante para la compañía discográfica, estás en condiciones de negociar mejores términos. \\
Por ejemplo, voy a usar Prince nuevamente, un artista revolucionario no solo como talento musical sino también como hombre de negocios. En 1994 estaba un poco perturbado con su compañía discográfica en el lanzamiento de singles, y había una canción que realmente quería lanzar por su cuenta. Y como resultado de Prince siendo Prince, siendo un artista multimillonario en ventas, pudo lanzar un sencillo, The Most Beautiful Girl in the World, independientemente, a través de un distribuidor independiente, Al Bell's Bellmark Records. Ese registro llegó a ser extremadamente exitoso, y fue uno solo. Fue único, como lo llaman. En otras palabras, le permitieron lanzar un single de forma independiente y se volvió tremendamente exitoso. Después de que terminó su contrato con Warner Bros Records, Price se hizo conocido como el único artista que podría tener eventos únicos con varios sellos discográficos importantes. Y con eso quiero decir que podría lanzar un álbum con una compañía diferente cada vez. Muy pocos artistas pueden hacer eso, pero mostró la posibilidad de que eso suceda. Tenga en cuenta que es posible que no tenga que celebrar un acuerdo a largo plazo con ninguna de las partes. Depende de tu apalancamiento. Y en la industria de la música actual, puede desarrollar su influencia en una serie de diferentes formas que no existían, incluso hace dos o tres años. \textbf{Si tiene un número, millones de visitas en YouTube, si tienes una presencia muy fuerte en tu cuenta de Instagram, Si tiene una cantidad de me gusta en Facebook, puede usar una de sus grabaciones en un comercial de televisión o en el fondo de una escena de película o un programa de televisión. Podría convertirse en un vídeo viral como resultado del éxito de esa escena en particular o ese comercial en particular. Todo eso te ayuda a aumentar el apalancamiento y te coloca en la posición de estar a la vanguardia de la industria musical en evolución. Realmente podrías negociar por términos que nunca se han negociado antes, lo que hace avanzar a esta industria.}
\section{Regalías}
Muy bien, hablemos de otra posición que es de vital importancia para el artista, y esa es la provisión de regalías. El artista asigna sus derechos de autor a la compañía discográfica y a cambio de esa asignación, reciben regalías de varios usos de esas grabaciones de sonido. Esta disposición, como todas las disposiciones han evolucionado durante un período de tiempo y continuar evolucionando\\
Hablemos de las regalías por los diversos usos de las grabaciones de sonido. \textbf{Uno de los usos de las grabaciones de sonido es la venta de la grabación,} La venta a través de distribución digital, la venta de CD físicos. Y las tasas de regalías para las grabaciones físicas se expresan en términos de puntos porcentuales. En otras palabras, el uno por ciento se considera un punto.\\
\textbf{La norma para las tasas de regalías para nuevos artistas para álbumes es entre 13 y 16\%.} Ahora, muchos contratos tendrán disposiciones que establecen que las tasas de regalías para los singles son tres cuartos de la tarifa del álbum, lo que significa, Si tiene una regalía del 16\%, su regalía para sencillos podría ser del 12\%. Yo personalmente tengo problemas con eso, pero vamos a hablar de eso, particularmente para sencillos digitales, pero hablaremos de eso un poco más tarde. 13 a 16\% es una buena tasa de regalías para nuevos artistas en el mercado actual. Pero surge la pregunta, ¿13 y 16\% de qué? ¿A qué se aplica ese 13 o 16\%? Digamos que tiene un acuerdo de 15 puntos. Bueno, esto vuelve a la evolución de los acuerdos de grabación. En años pasados, algunas marcas importantes aplicaban ese porcentaje al precio minorista. Y algunas etiquetas lo aplicarán al precio mayorista. ¿Cuál es el precio mayorista? Las compañías discográficas crean configuraciones de grabaciones de sonido, Digamos que un CD es una configuración de una grabación de sonido. Múltiples grabaciones de sonido están en un CD, los venden a minoristas por lo que se llama el precio mayorista, que es generalmente unos pocos dólares menos que el precio que el minorista vende al consumidor. Las compañías discográficas en general, particularmente en el pasado en días de distribución física, no se venden al público. Vendieron sus discos a minoristas que vendieron sus discos al público. Bueno, cuando las compañías discográficas basarían su cálculo de regalías en el comercio minorista precios, generalmente causaba confusión. Y aquí está la situación. Las compañías discográficas sugieren precios de lista minoristas. A minoristas para vender el producto. En otras palabras, hace años, volvamos a los años 90. Podría salir un álbum de Jay-Z y la compañía discográfica, aunque vendería el álbum al minorista por \$ 12, sugerirían que el álbum se vendiera por 15.98. Ese fue el precio minorista, el precio de lista minorista sugerido. A veces llamado SRLP, el precio de lista de venta sugerido. Y ese es el precio que sugiere la compañía discográfica los minoristas lo venden. Los minoristas no tienen que venderlo por ese precio. Los minoristas podrían venderlo por \$ 14. Y podrían determinar su grado de ganancia. Pagaron \$ 12 por él, podrían venderlo por \$ 12.98 y ganar un dólar. Podrían venderlo por \$ 14.98 y obtener una mayor ganancia. Por lo tanto, realmente depende de por qué lo vendió el minorista. Pero ese precio sugerido del minorista sería el precio inicial de determinar cómo se aplicaría esa tasa de regalías. Y aquí es donde crea mucha confusión. En la primera parte del contrato, la provisión de regalías diría que la tasa de regalías se aplicaría al precio base de regalías minoristas. Los artistas asumirían que ese es el precio de lista de venta sugerido, en este caso \$ 15.98 o \$ 16. Pero más adelante en el contrato, habría una disposición que definiera cuál es el precio base de la regalía minorista, y ese no sería el precio de lista de venta sugerido. Habría ciertas deducciones tomadas de ese precio de lista minorista sugerido, antes de que se aplicara esa tasa de regalías. Y vamos a hablar sobre algunas deducciones que se toman de los artistas regalías que realmente crean confusión, conducen a la falta de transparencia y realmente han evolucionado durante un período de tiempo. Entonces, la definición del precio base de regalías minoristas indicaría que es el precio de lista minorista sugerido menos una deducción de embalaje. Una deducción de embalaje. ¿Qué es una deducción de embalaje? Bueno, las compañías discográficas hace años realmente sintieron que la obra de arte  en el álbum, que pagaron para que los artistas crearan, Realmente creado valor en la venta del producto. Y no querían pagarle al artista una regalía basada en la obra de arte. Solo quieren pagarle al artista una regalía por la música. Entonces sintieron que a partir de ese precio de lista minorista sugerido, deberían sacar una cierta cantidad de eso, para dar cuenta de que pagan la obra de arte que ayudó a vender el álbum. Aquí está el problema. Al principio, para álbumes de vinilo, la deducción de empaque fue del 15\%. Y una vez más, esto entra en la evolución de la tecnología. Cada vez que se introduce una nueva tecnología, se introduce una nueva configuración En el mercado, parecía que la deducción de envases aumentaba cada vez más. Entonces, cuando los casetes entraron al mercado, la deducción de empaques para casetes fue del 20\%. Y luego, en los años 80, La deducción del embalaje fue del 25\%. Entonces, ¿qué significa todo esto? Volvamos a nuestro ejemplo con Jay-Z. El precio de lista minorista sugerido en los años 90 podría haber sido \$ 15.98, \$ 16, ese fue el precio de lista sugerido. El precio minorista basado en regalías, sin embargo incluido una deducción del 25\% de ese precio de \$ 16. Lo que redujo ese precio base de regalías de \$ 16 a \$ 12.\\
Es como descontar las regalías del artista, así que  la compañía discográfica tiene que pagar menos por eso. ¿Cuánto costó a las compañías discográficas no solo fabricar, sino paquete y sello de CD?. Se quitaron cuatro dólares con la deducción del embalaje, a veces se llama deducción de contenedor, pero la compañía discográfica solo gastó entre 60 y 75 centavos, no solo para fabricarlo, sino empaquételo, selle el CD, entre 60 y 75 centavos. Pero tomaron una deducción de cuatro dólares del sugerido precio del minorista para llegar al precio minorista de pared a base. Muy confuso. No transparente De eso se quejaban los artistas. Desafortunadamente, encuentra muchos contratos que aún están redactados en términos del precio de venta hoy. Sin embargo, \textbf{las regalías de las principales discográficas no se calculan sobre el precio minorista, se calculan sobre el precio mayorista. ¿Cuál es el precio mayorista? El precio mayorista a veces se denomina precio publicado por distribuidor o PPD. Así que tomemos un ejemplo aquí. Digamos que actualmente un álbum se vende a través de Itunes por \$ 9.99, y la compañía discográfica recibe \$ 7.50. Se lo venden a iTunes por \$ 7.50. Bueno, si aplicas la tasa de regalías del 15\% para ese álbum, encontrará que eso le deja al artista una regalía de aproximadamente \$ 1.12 por álbum.} Cuando hablas de sencillos, podría ser una historia diferente. Mencioné anteriormente, tradicionalmente, las tasas de regalías de los sencillos, muchas veces, son solo las tres cuartas partes de la tasa de regalías del álbum. Entonces, para hacerlo más fácil, si tuvieras un acuerdo de 16 puntos por álbumes, tendrías 12 puntos trato para sencillos, pero esa es la forma histórica de pagar regalías de ssencillos. Eso fue en el día en que se lanzaron CD singles, las compañías discográficas tuvieron que pagar mucho dinero para producir estos singles de CD, para distribuirlos a mayoristas y mayoristas que vendieron a minoristas, hubo mucha participación, una gran inversión para el lanzamiento de singles en CD.\\
Pero desde 2003, la configuración principal de los singles no ha sido el CD. Ha sido la descarga digital. ¿Cuáles son los costos para la compañía discográfica? para entregar estos archivos a iTunes, a Amazon? No están produciendo un CD. No tienen que distribuirlo y tienen almacenes para distribuirlo. No tienen todos esos costos múltiples. Entonces, ¿La tasa de regalías debería ser menor para un single digital que para un álbum digital? Yo argumento que no. Y aunque su contrato podría tener la propuesta original de la compañía discográfica declarando que quieren pagarle una tasa de tres cuartos, Descubrí que si realmente presionas el tema, las compañías discográficas posiblemente acordarán pagar la tarifa del álbum para singles digitales. ¿Por qué es eso importante?\\
Los singles se están convirtiendo en una gran configuración de venta de discos. Y tienes muchos millones de singles de venta en el mercado actual. Entonces, el single se está volviendo importante de nuevo al igual que en los años cincuenta, años sesenta y setenta. Entonces la deducción de empaquetado creó confusión, particularmente cuando trata con regalías que se pagan sobre el precio de lista minorista. Y una vez más, algunos contratos que encontrará hoy en día todavía afirman que son pagar regalías sobre un precio de lista minorista e impondrá una deducción de empaque, pero hay otras deducciones que debe preocuparle.
\subsection{Licencia de terceros}
\textbf{Aunque una tasa de regalías del 13 al 16\% de la venta al por mayor es común en el mundo de las principales etiquetas y sub etiquetas, y empresas de producción que se ocupan de las principales discográficas. Hay otras etiquetas independientes más pequeñas que pueden ofrecer una tasa de regalías diferente, una forma diferente de calcular las regalías. Es posible que algunas de las etiquetas más pequeñas estén dispuestas a firmar solo con uno o Acuerdo de dos álbumes. Y podrían estar dispuestos a pagarle un acuerdo de participación en las ganancias del 50\%.} ¿Entonces que significa eso? Bueno, \textbf{la etiqueta independiente pagará el costo de producción, fabricación, distribución, comercialización. Y tomarán todos esos costos, y después de recuperar todos esos costos del dinero que entra, dividirán las ganancias 50-50.} Ahora, ese es un arreglo muy, muy interesante. Una vez más, todas las etiquetas no hacen eso. Algunas etiquetas independientes más pequeñas podrían hacer eso. Pero realmente me pregunto si están basando eso en otro disposición en un acuerdo de etiqueta importante, del que hablaremos en este momento, Eso es muy importante. Hablamos sobre la tasa de regalías para la venta de grabaciones. \textbf{Pero en la mayoría de los acuerdos de etiquetas, cuando las grabaciones de sonido se usan en otros tipos de situaciones, donde la compañía discográfica licencia las grabaciones para otros usos, Las compañías discográficas suelen dividir lo que hacen 50-50 con el artista. Esos se llaman usos de terceros.} Déjame darte un ejemplo de usos de terceros. Licencias de sincronización, hablamos de eso un poco antes. Cuando las grabaciones de sonido se usan en comerciales, en programas de televisión, en la escena de fondo de una película, esos son usos de terceros. Y generalmente las compañías discográficas cobran una tarifa bastante considerable por ese uso, en función de cuánto tiempo se usa la canción en la película, o cuál es el contexto del uso, cuál es el presupuesto de la producción de la película. Pero esas tarifas pueden variar desde \$ 5,000, posiblemente incluso hasta \$ 1 millón de dólares por uso. Es una fuente de ingresos muy valiosa para las compañías discográficas.\\
Hablemos de algunas otras licencias de terceros. Es posible que haya comprado una tarjeta de felicitación recientemente que, cuando la abrió, Tenía una grabación que se reprodujo. Sé que compré uno para mi nieta hace un par de años. Era una de las canciones de Rick James, era Party All the Time o algo así. Era su cumpleaños, y lo abres, y fue la grabación de Rick James tocando. Esas tarjetas musicales se hicieron muy populares, es una licencia de terceros. Las compañías discográficas licencian esas grabaciones a American Greetings y la compañía de tarjetas Hallmark. Para que puedan insertarlos en las cartas, para que puedan jugar en las cartas. Se convirtió en una fuente muy valiosa de ingresos para las compañías discográficas, tres o hace cuatro años.\\
Otro uso de grabaciones por parte de terceros está otorgando licencias de las grabaciones a una compañía discográfica para armar una compilación, como la serie de álbumes NOW que se han vuelto tan populares en los últimos 15 o 20 años. Y esos son CD, más o menos, que tienen todas las grabaciones exitosas durante el año pasado más o menos. Y esas son grabaciones de varios sellos. Una etiqueta licencia esas grabaciones de todas las otras etiquetas, y póngalo en un CD y véndalo. Esos álbumes realmente se convirtieron en los más vendidos en los últimos cinco a diez años. Sin embargo, como las descargas digitales y iTunes se han vuelto muy populares, la gente básicamente puede armar su propia lista de reproducción. Entonces la popularidad de eso ha disminuido un poco. Pero eso es típico del uso de un tercero. Esas compañías discográficas acuerdan una división de regalías por licenciando esas grabaciones a esa compañía en particular. Y la compañía discográfica y el artista dividieron esas regalías 50-50. Ahora, aquí hay un gran problema en la industria de la música que me preocupa mucho, y muchas otras personas en la industria están preocupadas.\\
Otro uso de grabaciones por parte de terceros está otorgando licencias de las grabaciones a una compañía discográfica para armar una compilación, como la serie de álbumes NOW que se han vuelto tan populares en los últimos 15 o 20 años. Y esos son CD, más o menos, que tienen todas las grabaciones exitosas durante el año pasado más o menos. Y esas son grabaciones de varios sellos. Una etiqueta licencia esas grabaciones de todas las otras etiquetas, y póngalo en un CD y véndalo. Esos álbumes realmente se convirtieron en los más vendidos en los últimos cinco a diez años. Sin embargo, como las descargas digitales y iTunes se han vuelto muy populares, la gente básicamente puede armar su propia lista de reproducción. Entonces la popularidad de eso ha disminuido un poco. Pero eso es típico del uso de un tercero. Esas compañías discográficas acuerdan una división de regalías por licenciando esas grabaciones a esa compañía en particular. Y la compañía discográfica y el artista dividieron esas regalías 50-50. Ahora, aquí hay un gran problema en la industria de la música que me preocupa mucho, y muchas otras personas en la industria están preocupadas. ¿Es una descarga digital una venta de una grabación?  o es una licencia de terceros? ¿La compañía discográfica está vendiendo ese producto a iTunes, cuando entregan ese archivo para que iTunes lo ponga en su servidor y luego lo venda? ¿O están licenciando esa grabación a iTunes para vender? La diferencia entre licencia y venta es muy significativa. Entonces, \textbf{¿cuál es la diferencia entre una venta y una licencia de terceros? Bueno, generalmente en la situación de venta, la compañía discográfica se hace cargo de todos los costos de producción, fabricación, la distribución, la comercialización de ese producto. Y como resultado de todos esos costos en los que invierten, sienten que deberían obtener la mayor parte del dinero que se genera a partir de eso. Y esta es la razón por la cual a los artistas solo se les paga del 13 al 16\%.}\textbf{ Pero en una licencia de terceros, la compañía discográfica generalmente solo licencia el sonido grabación, entrega la grabación de sonido a cualquier compañía, Hallmark Cards. Quién hace las cartas, quién distribuye las cartas, comercializa esas tarjetas y asume todos esos gastos.} A esa compañía discográfica que reúne la serie Now. Esa compañía discográfica hace los álbumes, distribuye los álbumes, lleva todos esos costos. Muy bien, veamos las descargas digitales. ¿Qué hace una compañía discográfica? Con la descarga digital, la compañía discográfica básicamente entrega esa grabación de sonido a Amazon, a iTunes. Lo ponen en sus servidores, lo comercializan. Ahora la compañía discográfica lo hace, todavía hace un poco de marketing. Pero la compañía discográfica, en una descarga digital a través de iTunes o Amazon, no fabrican un CD. No lo distribuyen, no soportan el costo de ponerlo en un almacén. No tienen todos esos costos asociados con una venta. Entonces, ¿deberían las compañías discográficas aplicar una venta? regalías del 13 al 16\% para descargas digitales? ¿O deberían pagar el 50\% de la licencia de terceros? Bueno, esa ha sido una gran disputa a lo largo de los años. Como una cuestión de hecho, La compañía de producción de Eminem presentó una demanda en ese mismo punto. Y realmente tuve una decisión en ese caso, prestado por un juez indicando que el juez sintió que una descarga digital era, de hecho, como una licencia de un tercero y no una venta. Desafortunadamente, ese caso se resolvió, así que que no hay ningún precedente legal para eso en este momento. Pero sí creó una situación en la que al menos las compañías discográficas y los artistas están teniendo esa discusión. \textbf{Tal vez la descarga digital debería considerarse una licencia de terceros. Y por lo tanto, el artista debería tener derecho al 50\%, en lugar de una venta donde solo recibirían del 13 al 16\%.}
\subsection{Deducciones por rotura}
Una vez mas ,\textbf{ la deducción de embalaje se basa en el precio para el cual se aplica la tarifa de regalías.} pero aquí hay una deducción que esta basada en el numero de grabaciones vendidas, llamada la deducción de rotura. La deducción de rotura comenzó en verdad y fue introducida en los contratos de los años 40s y 50s, cuando las grabaciones en ese tiempo, principalmente en el año 78, se hacían en goma de laca. La goma de laca es una sustancia muy frágil, muy fácil de romper, muy fácil de rayar. Bien, las compañías discográficas tenían una preocupación legítima respecto al número de grabaciones que producían en la planta y enviaban a los distribuidores, que eran vendidas a los minoristas, que podrían romperse durante el proceso de envío, muchas veces lo fueron, las cajas se les soltaban a personas que las descargaban en los contenedores, podían caerse de esa área y se terminaba con varios discos rotos. y básicamente, los minoristas no tuvieron que pagar por los discos rotos. Los enviaron de regreso a la compañía discográfica y decían: miren, estos se rompieron, no estoy pagando por ellos. y por lo tanto, las compañías disqueras pudieron establecer que cerca del 10 al 15\% de los discos que enviaron se quebraron durante ese proceso. legitimo, en los 40s y 50s. Este es el problema. Estas disposiciones estuvieron en los contratos de grabación a comienzos de este siglo. La deducción de rotura básicamente dice que la compañía discográfica no tiene que pagar por los discos rotos y se llegó al punto de que las compañías discográficas decidieron que solo, desde el 10 al 15\% de los discos que eran quebrados durante este proceso de envío los contratos dirían, solo tenemos que pagar por el 85\% de los discos que enviemos o el 90\% de los discos que enviemos. Así, en otras palabras, si la compañía discográfica enviaba 1,000 discos, solo tendría que pagar por las regalías en 850, 85\% de las 1,000 que enviaba. La industria de la música evolucionó de los discos hechos en vinilo, no en goma de laca, el vinilo es mucho más maleable, flexible, irrompible, pero ¿las compañías discográficas eliminaron la disposición de los contratos de en los años 60s y los 70s, ya que se enviaron álbumes de 45's que fueron hechos en vinilo, se enviaron y llegaron a ser la configuración predominante? no, ¿ y que pasa con los casetes? se rompieron los casetes?, porque algunos pueden romperse. pero en el proceso de envío, no. quiero decir, recuerdo tener casetes que después de reproducirlos por un largo tiempo algunas veces la cinta se salia. , pero no se rompían. no durante el proceso de envío. en otras palabras, ¿la deducción por ruptura fuera en verdad significativa? si se hubiera aplicado a los casetes, las compañías discográficas hubieran tomado la posición de que, esta disposición permaneciera en el contrato, de 1,000 casetes enviados, solo se pagará el 850 Se aplicaría la deducción de rotura del 15\% CDs, creo que es el ejemplo mas ridículo de la aplicación de la deducción de rotura. aquí está la cosa CDs el gran punto de comercialización de los discos compactos fue que eran indestructibles. el vinilo se rayaba, algunas veces se doblaba Por lo tanto, las compañías discográficas inventaron este esquema de comercialización para los discos compactos, diciendo que este disco, nunca que quebrará, nunca se doblará y nunca se rayará y que era la mejor parte, la mejor parte de la comercialización de los discos compactos, y eso funcionó. ¿ debería la deducción de rotura  aplicarse a los discos compactos? Claramente, no. ¿ las compañías discográficas sacaron la deducción de rotura de los contratos de los años 80s y 90s. No. y fueron muy firme acerca de eso. recuerdo haber discutido este punto a inicios de los años 90s con una persona de asuntos comerciales, y dije, sabe este deducción de rotura es ridícula. estas solo vendiendo Discos Compactos ahora, principalmente CDs. ¿Por qué en el mundo tienes una deducción de rotura? quiero que reduzca o elimine esa deducción de rotura de una sola vez. y obtuve una respuesta negativa y la persona de asuntos comerciales dice, no, así es como se establece nuestro sistema, lo que nosotros debemos hacer cumplir sabe, puedo darle un gran adelanto, entiendo, lo compensare de alguna forma, pero las compañías discográficas decidieron mantener la deducción por rotura. y no fue hasta al rededor del 2004 o 2005 que el último sello importante eliminó la deducción por rotura de sus contratos. y ahora, todos los principales sellos discográficos pagan el 100\% de los discos vendidos. entonces, puedes decir, bien, ¿ por qué debería preocuparme por la deducción de rotura? ninguno de los mejores sellos discográficos usa ya la deducción de rotura Bien, los mejores sellos no son solo compañías que te firman acuerdos exclusivos de grabación. tienes muchas compañías independientes. tienes muchas compañías productoras que pueden firmarte como un artista o, grabarte y llegar a un acuerdo con uno de los principales sellos. y van a tener sus propios acuerdos exclusivos de grabación contigo como un artista. Muchas de estas compañías productoras y aun algunos sellos independientes, podrán basar sus contratos en antiguos contratos de sellos importantes que tengan estas deducciones de embalaje y tipos de disposiciones de deducciones de rotura en ellos. y tu necesitas ser consciente de esas disposiciones, si ellos tiene que en su contrato, están en posición de tomar esas deducciones de las regalías. Así, necesita ser capaz de reconocer si, se esta aplicando o no una deducción de embalaje, o una deducción por rotura al pago de sus regalías.
\subsection{bienes gratis}
Hablemos de la deducción de bienes gratis. La deducción de bienes gratis. En el mercado físico con CD, Las compañías discográficas venden estos productos a minoristas, y el personal de ventas en el trabajo de la compañía discográfica es vender tantos CD como sea posible a minoristas. A veces para hacer eso tienen que dar algo a los minoristas para atraerlos a comprar más CD., por ejemplo, digamos que un minorista quiere comprar 75 copias de un álbum. Porque piensan que el álbum se venderá en sus tiendas, y al menos podrían vender 75 copias. Entonces van a hacer un pedido con el vendedor de la compañía discográfica y Dicen que quiero comprar 75 copias de ese álbum en particular. Por lo general, los vendedores de la compañía discográfica están en condiciones de decir, espera un minuto, Te voy a ofrecer un trato. ¿Por qué no compras 100? Haga un pedido de 100 en lugar de 75 y si lo hace, le daré 15 copias más, te daré 115 copias, pero solo comprarás 100. Los 15 adicionales se denominan bienes gratuitos. Y las compañías discográficas no pagan regalías al artista por productos gratuitos. Ahora, las compañías discográficas tienen una preocupación legítima porque están tratando de obtener El minorista para comprar más. Y entonces sienten que deben tener este tipo de incentivos para vender más productos. Por lo tanto, las compañías discográficas tienen disposiciones que establecen que no tienen que pagar regalías por estos productos gratuitos. \textbf{Pueden regalar una cierta cantidad de bienes gratis que no tienen que pagar regalías. Por lo general, el abogado del artista intentará limitar El porcentaje de bienes gratuitos que entregan puede ser del 10\% o del 15\%. En otras palabras, no quieren que den 25 más o 30 más en un pedido de 100. Pero tal vez solo 15 más.}
\subsection{Reservas}
La deducción de envases, la deducción de rotura, la deducción de mercancías gratis. Voy a tratar con otra deducción que es significativa para los artistas hasta ahora como el pago de regalías y cuándo se pagan sus regalías, y Es una deducción de reserva, una deducción por reservas. ¿Qué son las reservas? Bueno, hablemos sobre el proceso, particularmente en el mundo físico de los CD. \\
\textbf{Las compañías discográficas venden CD a minoristas en consignación}. En otras palabras, si esos minoristas no venden esos CD, tienen derecho a devuelva esos CD a la compañía discográfica para obtener crédito para comprar otros CD. En otras palabras, cuando una compañía discográfica envía y vende un CD en particular, Pueden enviar 1,000 copias. No saben si el minorista va a vender 1,000, supongamos que solo venden 600 de los 1,000 para el artista. Ahora pueden no saber por dos años, esos registros podrían permanecer en la tienda durante dos años, es posible que no lo sepan durante dos años en realidad cuántos de esos, de ese CD en particular que vendieron. Teóricamente, la compañía discográfica tiene que rendir cuentas al artista en seis meses, diciendo que vendieron 1,000 CD y pagan regalías por 1,000 CD. Y aquí es donde se pone un poco complicado. Las compañías discográficas dirán, está bien, Te mostraré cuáles son tus regalías por vender 1,000 CD.  Y esta es la cantidad de regalías que le debemos por vender 1,000 CD. Sin embargo, la compañía discográfica tendrá un cierto porcentaje de esas regalías en reserva para devoluciones. Así que tomemos un ejemplo de eso. Digamos que han enviado 1,000 álbumes, y si eres un artista de 15 puntos y los CD se han vendido por el PPD, o precio al por mayor, de \$ 7.50, gana aproximadamente \$ 1.12 por registro. Sus regalías serían de aproximadamente \$ 1,100. Pero, la provisión de reserva dice que pueden retener el 25\% de esas regalías, en otras palabras, alrededor de 250, 260 dólares, por un cierto período de tiempo, hasta dos años, para determinar si van a obtener beneficios o no esos, ahora aquí está lo bueno de las reservas, si vende 1,000, eventualmente le darán cuenta de las cantidades que han mantenido en reserva. Suele ocurrir dos años después del hecho, pero al menos si está vendiendo, obtendrá esa cantidad.\\
Muy bien, así que resumamos de qué hablamos. Regalías basadas en la venta de grabaciones, por lo general eso se hace aplicando una tasa de regalías. La tasa de regalías podría estar entre 13 y 16\% de ya sea el precio minorista, en algunos contratos, o el precio mayorista, particularmente en relación con los principales contratos de etiquetas, y Hay ciertas deducciones o descuentos tomados de las regalías. ¿Cuáles son los descuentos de los que hablamos? La deducción de envases, la deducción de rotura, la deducción de bienes libres, reservas. Ahora hablemos sobre cómo están evolucionando los contratos de grabación a medida que avanzamos en la era digital. ¿Son relevantes esas deducciones? Las mismas deducciones que se tomaron para copias físicas de grabaciones, CD, ¿son relevantes para lo que está sucediendo en la era digital?.
Bueno, yo diría que hay preocupación, con respecto a si son relevantes o no. Ciertamente, la deducción de embalaje. ¿Hay algún embalaje? Para un álbum digital, sí, hacen obras de arte, pero no hay un paquete para una descarga digital. De hecho, a principios de siglo, como descarga digital recién comenzaban a desarrollarse, no en el cambio de siglo, 2003, 2004, tuve una conversación con una persona de negocios en un sello discográfico importante, porque vi una deducción de empaque, en lo que respecta a las descargas digitales. Empecé a reír por teléfono. Dije, espera un minuto. Espera un minuto. ¿Tiene una deducción de empaque para descarga digital? ¿Cómo se tiene un paquete? Eso es ridículo. Y ella dijo, bueno, una vez más, estas son nuestras políticas y tenemos que tener eso y allí, déjame compensarte de otra manera.\\
Las deducciones de empaque claramente no son relevantes para descargas digitales y Los contratos han evolucionado en los últimos diez años para eliminar eso.\\
Deducción por rotura, una vez más, las grandes etiquetas no aplican la deducción por rotura. Sin embargo, es posible que tenga una empresa independiente o incluso su La compañía de producción con la que firmas puede lanzar tu producto en iTunes. Y si tienen una deducción por rotura allí, le pagarán un número menor que los álbumes o singles que se vendieron realmente. Entonces realmente necesitas ver eso. \textbf{¿Puede haber alguna rotura de una descarga digital? Absolutamente no.}\\
Los registros se entregan en forma digital al distribuidor digital, y como se venden, salen de sus servidores al comprador, no tiene sentido. No hay deducción por rotura en el mundo digital. ¿Qué pasa con los productos gratuitos? En los mundos físicos, sí. Las compañías discográficas tuvieron que atraer a los minoristas para que compraran más copias posiblemente dándoles algunos gratis. Una vez más, en un mundo digital, el archivo se entrega a iTunes, a Amazon. Cuando se venden, el minorista o iTunes o Amazon no está dando copias gratis. ¿Y qué hay de las reservas? El mismo argumento. El archivo se entrega al distribuidor digital, quien lo vende directamente al consumidor. No hay devolución de descargas digitales. El distribuidor digital solo representa el número que realmente vende. No hay razón para que haya alguna provisión de reserva.\\
Entonces, una vez más, no es ciencia espacial, pero es importante si eres un artista estar al tanto de estas disposiciones, para poder negociarlas a su favor.\textbf{ Y si usted es propietario de una compañía discográfica en esta nueva era digital, reconozca que estos son disposiciones que pueden dejar de ser relevantes y desea tener una feria y contrato equitativo con el artista para avanzar en el desarrollo de su negocio.}
\subsection{Servicios de streaming}
Además de las compañías discográficas y los artistas que ganan dinero con todos esas áreas valiosas, como las ventas, tanto en las tiendas. Venta física de CD, venta de copias digitales en iTunes, y Amazon, y otras áreas. Hay otra fuente muy importante de ingresos para las compañías discográficas y artistas que están comenzando a hacer crecer la transmisión por Internet. Streaming por internet. \textbf{La transmisión de regalías se está volviendo cada vez más importante en el negocio.} Cada vez que una grabación se transmite digitalmente, una transmisión digital puede ocurrir a través de cable, radio satelital o transmisión web. La compañía discográfica y el artista tienen derecho a ser compensados por esos usos. ¿Cuál es el problema? En este momento el problema es que la cantidad que se paga esos usos son tan pequeños, \textbf{recuerdo que hace años alguien dijo que el negocio de la música es un negocio de centavos, un negocio de centavos.} Bueno, la situación es, en este momento que para la transmisión digital y la transmisión por Internet, esos centavos se están dividiendo en porciones aún más pequeñas. Decenas de centavos. Decenas de centavos. Entonces, la cantidad que recibirá, tanto la compañía discográfica como El artista recibirá, para este tipo de servicios dependen de los tipos de usos. \textbf{Hay un par de tipos de usos de los que quiero hablar. Primero está el flujo no interactivo, el uso no interactivo. ¿Qué son los no interactivos? Ahí es donde el consumidor, la persona que escucha el webcast No puedo decidir escuchar una canción en particular varias veces. En otras palabras, es similar a Pandora, golpeas o pones en el artista que te gusta y Pandora programar una estación de radio con otras canciones que suenen similares a las de este artista. En otras palabras, no tienes derecho a elegir esas canciones en particular que quieres escuchar.} Pandora por eso está pagando regalías por la cantidad de solo una décima de centavo. Una décima de centavo por flujo. está causando bastante rencor dentro de la industria de la música. Nuevos servicios llamados servicios interactivos. Y los servicios interactivos son donde puedes elegir una canción en particular y escúchalo varias veces. Puede elegir cualquier cosa del vasto catálogo de un servicio. Como, por ejemplo, Spotify, y escúchalo todo lo que quieras. Spotify también tiene un servicio similar a la radio, similar a Pandora. Y esas regalías varían. El uso no interactivo, o el servicio tipo radio, a través de Spotify, Las tasas de regalías pueden ser menores. Un poco más de una décima de centavo.\\
\textbf{Pero para uso interactivo, donde realmente eliges las canciones y las escuchas varias veces, esa tasa de regalías puede aumentar a quizás tres décimas de centavo.} Muy pequeña. Dices, bueno, ¿cómo puedes ganar dinero en esa área? Bueno, ese ha sido el problema actualmente. Muchos artistas están muy preocupados por la baja cantidad de regalías que están recibiendo en este punto. Pero como nuevos servicios, como nuevos emprendedores, como nuevos servicios innovadores están siendo desarrollado, encontrarás diferentes tipos de acuerdos que están siendo alcanzados por las compañías discográficas y estos proveedores de servicios de Internet. Mientras Pandora paga una cantidad específica por cada transmisión, algunos de los nuevos Los servicios de Internet están entrando en acuerdos con las principales discográficas donde lo harán no solo paga por tasa de transmisión, sino que pagará una parte de su publicidad y dinero de suscripción que hacen de sus diversos suscriptores. Aquellos de ustedes que están familiarizados con Spotify saben que pueden escucharlo durante gratis o puedes suscribirte mensualmente. Y si haces eso, no tienes que escuchar los anuncios, y Puedes escuchar la música que quieras. Tienes acceso completo a la música. La mayoría de los servicios de transmisión de Internet más nuevos, tanto no interactivos como Interactivo, basan su modelo de negocio en un anuncio y modelo de suscripción base. ¿Qué quiero decir con eso? Bueno, están vendiendo anuncios que se reproducirán tal vez cada tres o cuatro canciones o ofrecerán un servicio de suscripción en el que puede pagar una cantidad mensual. Entonces eso eliminará los anuncios y puedes reproducir la música que quieras en cualquier momento.\\
\textbf{Muchos de los sellos discográficos están entrando en acuerdos especiales que otorgan licencias catálogos a estos servicios de internet. iRadio, Spotify. Y están recibiendo un porcentaje de la publicidad y ingresos por suscripción generados por estos nuevos servicios.} \textbf{Y esas regalías se prorratean de acuerdo con el específico Vapor de las canciones en su catálogo., que actúan como personas intermedias que agregan o agrupan un número de artistas independientes, y licencian su música a estos servicios de Internet.} Por supuesto que les pagan una tarifa por eso, pero les permite a estos artistas ingresar al mercado de la transmisión por Internet.\\
Entonces, mientras que las regalías de transmisión por Internet se consideran muy bajas por varios artistas, sienten que es demasiado bajo. De hecho, varios artistas se han negado a permitir que su música sea transmitido por las bajas regalías. La esperanza es que a medida que crezca la base de suscriptores, actualmente son solo unos pocos millones, pero debería alcanzar la escala para Estos diversos servicios, tal vez 50 millones de personas o más. La cantidad de ingresos disponibles para estas regalías que se prorratearán también crecerán y los artistas, compositores, y las compañías discográficas recibirán una parte más justa de esos ingresos.\\
Ahora, \textbf{¿quién responde a la compañía discográfica y al artista por los ingresos generados por los servicios de transmisión? Bueno, la RIAA, la Asociación de la Industria de Grabación de América,} estableció una organización específica de derechos de desempeño para manejar este tipo de regalías. Esa organización se llama Sound Exchange y tiene su sede en Washington DC. Y cada compañía discográfica y cada artista necesita registrarse con Sound Exchange para recibir sus regalías de la transmisión por Internet y La transmisión digital de sus obras. Siempre estoy interesado en que los artistas reciban su parte justa del uso de su grabaciones, y estoy feliz de decir que las etiquetas de grabación, artistas de grabación y Sound Exchange llegó a un acuerdo hace años.\\
Asegurarse de que las regalías generadas por la transmisión digital de grabaciones se dividirían entre el artista y la compañía discográfica de manera justa e igualitaria.\textbf{ El desglose es del 50\% para la compañía discográfica, del 45\% para el artista destacado, y 5\% a los músicos o vocalistas no presentados o de sesión. Para que el dinero se pague directamente a la compañía discográfica, directamente al artista destacado, y directamente a los sindicatos. Pero artistas, si no te registras en Sound Exchange, No puedes recibir tus regalías.} Por lo tanto, es muy importante que se registre en Sound Exchange, para que pueda puede recibir sus regalías de transmisión de audio digital y transmisión por Internet. A medida que este modelo de transmisión por Internet continúa evolucionando, las compañías discográficas y Los proveedores de servicios de Internet están presentando diferentes tipos de ofertas ahora.\\
Una etiqueta importante podría licenciar todo su catálogo de grabaciones a Internet proveedor de servicios a cambio de que le paguen el 50\% de la regalía directamente a ellos, al tiempo que permite el 45\% al artista destacado y el 5\% para el artista no destacado que se pagará a través de Sound Exchange. Sin embargo, no pienses que esto es un hecho.\\
Si ha firmado con una etiqueta importante, es muy importante para revisas tu contrato con mucho cuidado para ver si incluso tienes derecho a regalías de estos diversos servicios. En algunos contratos con las principales discográficas, habrá una disposición que establece que licenciamiento de todo el catálogo de grabaciones sonoras de la discográfica y incluso los vídeos pueden no requerir que paguen regalías al artista.\\
Entonces, esto crea una gran situación para los artistas independientes. Y es un buen momento para ser un artista independiente y poseer los derechos de la grabación de sonido usted mismo. {\color{red}Para que pueda licenciar a través de agregadores, como Nimbit TuneCore u otros. Puede recibir las regalías de la compañía discográfica, así como las regalías de los artistas. Entonces, este es el mejor momento para DIO, hacerlo nosotros mismos. Cree un equipo de personas que pueda ayudarlo a crear su propia empresa independiente, poseer y desarrollar las grabaciones, tanto de audio como de vídeo. Explotar esos activos.}\\
En varias plataformas, webcast, transmisión por internet, ventas en iTunes, una licencia para varias películas y comerciales. Construyendo su audiencia y una vez más eso vuelve a ser un producto poderoso. Construyendo esos valiosos activos en sus grabaciones de sonidos y solo ese interés para explotarlo de varias maneras, para construir el valor en su empresa.
\section{Avances}
\subsection{Anticipos y presupuestos de registro}
Otra disposición importante del acuerdo del artista discográfico es la disposición sobre anticipos. \textbf{¿Qué son los anticipos? Bueno, los anticipos son pagos anticipados de regalías, y las compañías discográficas adelantan regalías por varias razones. Principalmente, para grabar el álbum, y muchas veces, los anticipos de cierta cantidad en dólares se pagarán como anticipo para producir un álbum. Esos se llaman Adelantos del Fondo del Álbum, lo que significa todos los costos de la grabación de los álbumes tiene que ser sacada de ese avance particular. En un nivel de etiqueta importante, un nuevo artista podría recibir un adelanto de fondos para álbumes de 50 a \$ 100,000.00. Para un artista de nivel medio, puede ser de 100 a \$ 200,000.00. En el caso de los artistas superestrellas, pueden recibir un adelanto de fondos para álbumes de 300 a \$ 1,000,000.00. Ahora, las etiquetas independientes también pueden pagar un anticipo, pero serán mucho más pequeños porque se venden en cantidades más pequeñas. Puede costar desde \$ 2,500 por un álbum hasta 25,000 dependiendo del género de música del que estés hablando. Los anticipos del fondo del álbum se pagan para compensar el costo de grabar el álbum. Eso significa pagar los costos de grabación, pagar los estudios de grabación, pagando a productores, arregladores, ingenieros. Una parte del fondo puede ser utilizada para gastos personales del artista. Entonces, digamos que un nuevo artista tiene un fondo de álbum de \$ 100,000. Fuera de eso, la compañía discográfica podría designar \$ 20,000 como adelanto personal para el artista. Y el resto de los \$ 80,000 se mantiene en reserva para ser utilizado para pagar todo El costo de grabación. Y en base al fondo del álbum, si queda algo después del pago de todos los costos de grabación, que también irán al artista. Entonces, el artista podría tener \$ 20,000 designados, \$ 80,000 para los costos de grabación, pero si los costos de grabación solo llegan a \$ 60,000, Eso es un sobrante adicional de \$ 20,000 que también podría ir al artista.} Entonces, el artista podría terminar recibiendo \$ 40,000. Está bien, pero veamos la instancia en la que los costos de grabación superan el presupuesto. En lugar de 80,000, necesitan \$ 90,000. Bueno, generalmente el artista tiene que pagar esa diferencia o La compañía discográfica tiene derecho a rescindir el contrato. Por lo tanto, es muy importante que los artistas se mantengan dentro de su presupuesto de grabación. Y aquí está el siguiente punto muy importante. \textbf{Todos los avances se recuperan de las regalías del artista.} Ahora, \textbf{mucha gente piensa que la compañía discográfica está prestando dinero al artista. para grabar su álbum. Bueno, no es un préstamo. Es una gran diferencia entre un anticipo y un préstamo. ¿Por qué? En un préstamo con un banco, si compra una casa,el banco te dará una hipoteca y Lo van a asegurar contra esa casa. Si no devuelve el dinero, el banco puede demandarlo personalmente. Pueden ir tras tus cuentas bancarias. Pueden vender tu auto. Pueden vender la casa para recuperar ese dinero. Ese no es el caso en un acuerdo de grabación, donde se hacen avances al artista. La única forma en que la compañía discográfica puede recuperar esa inversión es vendiendo más registros porque eso solo es recuperable de las regalías del artista. Si el artista no genera suficientes regalías para pagarlo, entonces la compañía discográfica tiene que vivir con eso.} No pueden perseguir al artista personalmente por regalías no solicitadas. Cuando comencé a practicar leyes me dijeron que es es importante obtener el mayor adelanto posible en regalías. Y en ese momento, la gente me dijo que si no recibes el dinero por adelantado, es poco probable que lo veas en el backend. Y a medida que pasó el tiempo y durante los últimos 20 o 30 años, Realmente he visto que ese es el caso. Ahora es más importante que nunca para artistas para tratar de obtener un avance por adelantado porque desafortunadamente, es poco probable que puedan recuperar todo el costo. ¿Por qué digo eso? Bueno, además de los fondos del álbum Avances para producir el álbum, \textbf{En el transcurso de los últimos 20 o 30 años, las compañías discográficas han agregó otros costos que consideran adelantos contra las regalías del artista. ¿Cuáles son esos costos? Uno, son el costo del vídeo, el costo de producir vídeos, Se consideran anticipos contra las regalías del artista.} Si su contrato es bien negociado por su abogado, solo el 50\% de los costos de vídeo serán recuperables de las regalías de su artista. El 50\% de los costos de su vídeo se pueden recuperar de los derechos de autor. Las empresas intentarán recuperar el 100\%, pero a veces, si luchas contra ellas en eso punto, aceptarán recuperar solo el 50\% del costo del vídeo. ¿Por qué es eso importante? Bueno, es muy importante porque los vídeos son bastante caros. En algún momento, en los años 90, algunos de los principales productores de vídeos cobraría \$ 1,000,000 en un fondo para producir un vídeo. Ahora, esos costos han bajado dramáticamente ahora, pero un buen vídeo probablemente al menos le costará \$ 50,000 o más. Y encontrará que para cada álbum, los artistas producirán tal vez tres o cuatro vídeos para tres o cuatro sencillos. Entonces, esos costos de vídeo pueden ser sustanciales. Entonces, los costos de vídeo ahora se pueden recuperar de los derechos de autor. Aquí hay otro punto. Costes de promoción independiente. Los costos de promoción independientes también se pueden recuperar de los derechos de autor. ¿Cuáles son los costos de promoción independientes? Bueno, los promotores independientes son personas que tienen especial relaciones con programadores de radio. Y aunque Internet realmente está despegando, y YouTube y Las sensaciones virales están creando una nueva ola de promoción en la industria de la música, radio, la radio terrestre sigue siendo una parte muy, muy importante de la mezcla de promoción. Entonces, es muy importante para compañías discográficas para poder contratar a estos promotores independientes que tienen relaciones especiales con programadores de radio para que suenen los singles. El costo de pagar a estos promotores independientes es bastante considerable. Puede oscilar entre \$ 50,000 y \$ 300,000 por persona. Entonces, ese es un costo muy costoso. Digamos que son \$ 50,000 por persona y Digamos que un artista tiene un vídeo que cuesta \$ 50,000. Por lo general, el costo de promoción independiente, si presiona la compañía discográfica, solo se puede recuperar a una tasa del 50\%. Entonces, quieres intentar obtener tu vídeo y promoción independiente costo recuperable a una tasa del 50\%, no una tasa del 100\%. Entonces, digamos que un artista tiene cuatro singles de su álbum. Cortaron cuatro vídeos a \$ 50,000 cada uno. Para cada single, se contrata a un promotor independiente a \$ 50,000 por single. Entonces, tiene un costo de \$ 400,000. La mitad de eso es recuperable de las regalías de su artista. Digamos que el artista tenía un fondo de grabación de \$ 100,000. Bueno, podrías haber pensado, inicialmente es que, todo lo que tengo que hacer es recuperar \$ 100,000 en costo por mi fondo de grabación, y luego tengo derecho a más regalías. Pero también debe tener en cuenta el costo del vídeo y el costo de la promoción independiente. Entonces, tienes otros 200,000 además de eso. Entonces, ahora tiene \$ 300,000 que debe recuperar. Entonces, como pueden ver, se vuelve bastante difícil para artistas para poder generar regalías más allá de ese avance inicial. Pero veamos la instancia de una pequeña etiqueta independiente que ha firmado con y tal vez su fondo de grabación es de solo \$ 25,000. Y sus costos de vídeo pueden ser de \$ 2,500 por vídeo, y puede producir cuatro vídeos, y los costos de promoción independientes serán mucho menores. También podrían ser de \$ 2,500 por persona. Entonces, eso es \$ 20,000 adicionales sobre el fondo de grabación de \$ 25,000. Aquí está la cosa con las etiquetas independientes. Recuerda cómo hablé, a veces las etiquetas independientes entran en 50, ¿50 relaciones de participación en las ganancias con su artista? Pero cuando entran en ese 50, 50 artista de participación en las ganancias, recuperan todos los costos, no el 50\% de los costos, como hacen las grandes discográficas costo de vídeo y costos de promoción independientes. Todos esos costos deben recuperarse antes de llegar a la división 50, 50. Entonces, \textbf{en la industria musical actual, es un poco difícil para un artista para estar en una posición de recuperación total, tener derecho a regalías más allá del anticipo inicial, y aquí es generalmente cómo funciona.} Si no estás contento después del primer álbum, ya sea que estés en una de las principales etiqueta o etiqueta independiente, pero vendió un número significativo de copias, por lo general, la etiqueta querrá elegir la opción para el próximo álbum, en ese momento, tienes derecho a un avance adicional para la producción de ese álbum. Entonces, aunque esa cuenta no solicitada se transfiera al segundo álbum, eso se llama colateralización cruzada de un álbum a otro, el artista aún habría ganado dinero con ese avance para el segundo álbum. Y la mayoría de los artistas ahora generan regalías a través de los avances. Y confía en mí, si vendiste una cantidad significativa de álbumes para tu primer álbum, su abogado estará en la línea con la persona de negocios en el registro compañía que defiende un avance renegociado y más alto para el segundo álbum. Y es por eso que digo que es importante obtener todo lo que pueda por adelantado. Esa teoría sigue siendo cierta. Era cierto hace 30 años, y sigue siendo cierto hoy.
\section{Ofertas de 360 grados}
\subsection{Ofertas 360 u ofertas $"$todo incluido$"$}
Y, por último, abordemos el tema de los acuerdos de 360 grados. \textbf{Las ofertas de 360 grados a veces se denominan acuerdos de todos los derechos, acuerdos de todos los derechos.} ¿Qué quiero decir con un acuerdo de 360 grados? Bueno, tenemos que mirar hacia atrás.\\
En el siglo pasado, las compañías discográficas firmaron artistas para grabar acuerdos y principalmente ingresos obtenidos estrictamente de la venta o uso de las grabaciones de artistas.\\
Así es como una compañía discográfica hizo su dinero. De vender o licenciar el uso de esas grabaciones de sonido. Ese era el único ingreso que les preocupaba. Y como resultado de las ventas de CD y las ventas de discos y álbumes que aumentaron a partir de los años 70, todos a finales de los 90, eso es todo el dinero que la compañía discográfica quería. Obtuvieron una ganancia ordenada de solo vender y usar grabaciones. Pero entonces, ¿qué pasó? Napster.  ¿Correcto? Intercambio de archivos punto a punto. Una disminución inmediata en las ventas de discos CD y gran parte de las ganancias que estaba obteniendo la compañía discográfica comenzaron a evaporarse. Así que las compañías discográficas tuvieron que buscar otras fuentes de ingresos para tratar de compensar ese déficit.\\
Recuerdo quizás 2002, 2003 que llamé un disco empresa que tenía un acuerdo con mis clientes, y estaba argumentando por un porcentaje de interés en los derechos de autor en las composiciones. Sentí que mis clientes habían contribuido a las canciones que grabaron y nunca se les dio crédito por ello. Ahora la compañía con la que estaba tratando también estaba compuesta por personas creativas. Productores que también fueron compositores. Y cuando presioné este punto, quiero que mis clientes participen En el ingreso de composición, el abogado se molestó un poco y dijo: espera un minuto, John, sabes que tus clientes han tenido una carrera exitosa durante más de 25 años. Mis clientes escribieron y produjeron la mayoría de las canciones exitosas, y como resultado de que mis clientes contribuyeron con esas grandes canciones exitosas, sus clientes han podido vender conciertos. Han podido actuar en programas de televisión y en películas. Han podido obtener patrocinios y avales. ¿Y sabes algo? No participamos en ese ingreso. Nunca lo hemos hecho. Él dijo, hm, John, ¿sabes algo? Quizás mis clientes deberían pedirles a sus clientes cierto porcentaje de sus ingresos en vivo. Un cierto porcentaje de su publicación, de su comercialización, de sus honorarios de actuación, de sus patrocinios. Hm, tal vez eso es algo en lo que deberíamos pensar. Y esa fue mi primera inclinación de que las personas en el nivel de etiqueta principal fueran pensando en participar en ingresos que no sean la venta y el uso de grabaciones. Y poco después, EMI entró en lo que se consideraba el primer contrato de 360 grados con Robbie Williams, quien fue un gran artista en todo el mundo, excepto en los Estados Unidos. Donde Robbie llegó a un acuerdo donde renunció a cierto interés en su publicando, renunció a un cierto porcentaje interés en su comercialización, patrocinio. Cualquier actividad dentro del área de entretenimiento, EMI iba a participar en las ganancias, y {\color{red} de eso se tratan generalmente estas ofertas de 360 grados. En otras palabras, la compañía discográfica no solo participa en los ingresos generados del uso y venta de esas grabaciones de sonido, pero participan en todo el círculo, 360 grados, de ingresos que genera el artista.} Así que ahora la mayoría de las ofertas que firmarás con las principales etiquetas, Al menos un sello importante intentará que firmes un acuerdo de 360 grados. Si tienes mucha influencia, Es posible que pueda negociar un acuerdo de 360 grados. Pero en la mayoría de los casos, los artistas firmarán acuerdos de 360 grados. Lo que significa que la compañía discográfica al menos tendrá opciones para recoger contratos de comercialización con el artista. \textbf{Para que participen en una cierta cantidad de los ingresos de comercialización, ingresos por patrocinio, ingresos por patrocinio, presentaciones en vivo. Tal vez un cierto porcentaje, 10, se pagará el 15\% de los ingresos netos de las actuaciones en vivo. Nunca antes se había hecho eso en el nivel de etiqueta principal.}\\
Pero se ha hecho a nivel de etiqueta independiente así como pequeñas empresas de producción. Eso se ha hecho durante años, durante los últimos 20 o 30 años, en virtud de acuerdos llamados el acuerdo general. El acuerdo general era un acuerdo en el que una compañía de producción firmaría un artista desconocido para no solo un acuerdo de grabación sino que también los firmarían para un acuerdo de publicación y participar en las regalías de sus compositores. Incluso podrían firmarlos en un acuerdo de gestión, donde también participar en cualquiera de los otros ingresos que generó el artista. Entonces esto ha estado sucediendo por años con etiquetas más pequeñas, etiquetas independientes y compañías de producción. Siempre lo han hecho ellos mismos. Hablamos de hacerlo usted mismo, y ahora veo que es más y Más importante para los artistas hacerlo nosotros mismos, tener un equipo que los rodee eso los ayudará a generar ingresos de varias fuentes diferentes. Bueno, etiquetas independientes y Las empresas productoras lo han estado haciendo durante varios años. Incluso Motown, allá por los años 60, tenían un acuerdo general.\\
Berry Gordy tenía una empresa de gestión que gestionaba todo el talento en Motown. Los compositores que resultaron ser artistas También tuvo que renunciar a sus intereses editoriales a Motown. Y suena, incluso lo abordo como abogado de un artista, Es como si las etiquetas fueran codiciosas. Están tomando dinero de todas las fuentes posibles. Pero en ciertas situaciones podría estar justificado. Ciertamente estaba justificado en los años de Motown. Berry Gordy estaba en condiciones de conducir más de 100 singles al número uno en las listas. Como resultado de eso, los artistas pudieron trabajar más. Pudieron obtener mayores oportunidades para televisores, películas. Diana Ross pasó a protagonizar Lady Sings the Blues, en Mahogany. Todas estas cosas donde los artistas se están transformando en otros tipos de carreras, el sello discográfico siente que comenzaron eso. Sin esa grabación inicial, y Es por eso que el acuerdo de grabación exclusivo es tan importante. Es la génesis de todos los otros tipos de actividades que un artista puede ser. involucrado en generar ingresos. Entonces las compañías discográficas quieren ser parte de eso y sienten que tienen derecho a participar.\\
Entonces la pregunta siempre ha sido, incluso lidiar con los tratos generales Además de los acuerdos de 360 grados, ¿son las empresas capaces de hacerlo ellos mismos? ¿Tienen las personas involucradas que entienden la actuación en vivo, que entienden el merchandising, que entienden los patrocinios y avales?.\\
Bueno, el hecho es que durante los últimos diez años desde estos acuerdos se han vuelto populares en las grandes discográficas, las grandes discográficas son ellas mismas comenzando a transformarse en compañías de entretenimiento en general. Van mucho más allá de producir grabaciones. Se están afiliando a empresas de gestión. Se están afiliando a la comercialización, a veces se asocian con esas compañías con el fin de crear oportunidades para que sus artistas generen ingresos.\\
Hace poco traté con una persona de A\& R en una discográfica importante. Y su asistente se puso en contacto conmigo, y el título del asistente era gerente de A\& R y desarrollo de artistas. Motown siempre tuvo un departamento de desarrollo de artistas. Pero en los últimos 30 años, las compañías discográficas realmente salieron de ese negocio. Pero ahora están de vuelta en ese negocio. \textbf{Están tratando de desarrollar a sus artistas para poder expandir sus carreras. más allá de solo grabar e incluso tocar en vivo en el escenario. Quieren que el artista actúe. Quieren que el artista sea una persona de tono, así que que eso puede generar más y más ingresos. Ahora los porcentajes que toman sobre todas estas fuentes pueden variar. Puede variar del 10\% hasta el 25\%, pero todos esos porcentajes tienen que ser negociados.}
\section{Resumen}
Hablamos de cómo a principios de este siglo, Las grandes discográficas hicieron un esfuerzo por hacer que sus contratos fueran más transparentes y simples. Sin embargo, los contratos todavía están evolucionando y, como resultado de la transmisión por Internet y A medida que la era digital toma forma, se vuelven aún más complejos.\\
Las regalías se basan en ventas minoristas o mayoristas. Regalías pagadas en licencias de terceros cuando las grabaciones se usan en comerciales, programas de televisión o películas. Transmitiendo regalías, aunque pequeñas ahora, convirtiéndose en un flujo de ingresos más importante a medida que pasa el tiempo.\\
Hablamos sobre los avances, por qué es importante obtener un avance si puedes. Es importante para compañías discográficas para comprender cuánto deben pagar por adelantado.\\
También hablamos sobre nuevas disposiciones en los acuerdos de grabación. El modelo de acuerdo de 360 grados, que es cada vez más frecuente en nuestra industria. Donde las compañías discográficas pueden participar en las ganancias del artista de varias fuentes de ingresos. Y, por último, este es el momento más importante a medida que evoluciona esta era digital. Es un buen momento para que los artistas construyan sus propios activos, posean sus propios activos. Poseer sus grabaciones de sonido, poseer sus grabaciones de vídeo, explotarlos a través de todas las diversas fuentes de Internet que pueden construir esa audiencia y hacer crecer el valor de los activos de su empresa. Y esto vuelve al punto que hice antes. En lugar de bricolaje, hágalo usted mismo, reconozca que necesitará un equipo. DIO. Hazlo nosotros mismos. Obtenga un equipo que pueda ayudarlo a construir y ser dueño de su propia empresa y poseer sus propios activos y explotarlos de la manera que construirá su negocio.
  
\chapter{Conceptos básicos de copyright}
Los derechos de autor son los más valiosos.
\section{Historia del derecho de autor}
Así que hablemos sobre la historia de los derechos de autor, La historia del copyright en los Estados Unidos. Como indiqué, la Ley de Copyright inicial se promulgó en 1790. Muy interesante que nuestros fundadores durante esta era de iluminación decidieron que querían crear un mecanismo donde los estadounidenses pudieran crear obras. Querían alentar a las personas a crear obras. Invenciones, patentes, trabajos creativos. Escribe libros, escribe música, que podrían poseer por un cierto período de tiempo. Fue muy importante durante ese período de tiempo porque, el desarrollo de la imprenta era muy importante para el motor económico de los Estados Unidos. Y quieren alentar a más estadounidenses a escribir libros que en el extranjero, libros con derechos de autor de Inglaterra. Y Europa.\\
Durante este período de tiempo desde 1790, ha habido una serie de enmiendas a la Ley de derechos de autor que afectan el plazo de los derechos de autor o la duración de la propiedad de los derechos de autor. Tasas de regalías, tipos y derechos de derechos de autor. \\
Así que hablemos sobre el término de derechos de autor. En 1790, en la Ley de Derechos de Autor original, la duración del derecho de autor era de solo 14 años. En otras palabras, si alguien creó una obra y presentada ante la Oficina de Derechos de Autor de los Estados Unidos, tendrían los derechos exclusivos y monopolio de los derechos de ese trabajo durante 14 años. Y al final de ese período de 14 años, y al final de un plazo de copyright, el trabajo luego cae en lo que se llama dominio público, lo que significa que cualquier persona en el público puede usar ese trabajo sin obtener el permiso del propietario original de los derechos de autor o cualquier otra persona para el caso. Ahora, en el transcurso de más de 220 años de historia de la Ley de Derechos de Autor, ha habido una serie de enmiendas, particularmente al término de derechos de autor. Y actualmente el término de copyright es la vida del autor, más 70 años, a menos que el trabajo sea por encargo. Y usted podría estar diciendo, ¿qué demonios es un trabajo por contrato? Bueno, un trabajo por contrato es cuando una corporación contrata a alguien para crear un trabajo dentro de su alcance de empleo, o si el trabajo cae en una de las nueve categorías de tipos de obras que puede considerarse un trabajo por contrato, si eso sucede, entonces la corporación No se considera que el creador de la obra sea el autor de la obra. Y la corporación será dueña del trabajo por 95 años. Ahora mencioné ese término actual de copyright si no es un trabajo para contratar es vida más 70 años. Bueno, ¿qué pasa si tienes más de un autor de la obra? Bueno, si tienes más de un autor del trabajo, ese período de 70 años no se extiende hasta la muerte del último autor sobreviviente. Tengo un ejemplo perfecto para eso. Mi cliente, Gerald Levert, el difunto gran Gerald Levert escribió canciones con un coguionista que todavía estaba vivo. Gerald lamentablemente murió en 2006. Pero su coguionista en varias de sus canciones todavía está vivo. Digamos que el coguionista vive hasta 2035. Bueno, el período de 70 años no comienza a correr hasta después de la muerte del último autor sobreviviente. Entonces, cuando fallezca, se extenderá ese período de 70 años. Lo que significa que su patrimonio, junto con el patrimonio de Gerald, poseería los derechos de autor de todas esas canciones hasta el año 2105. Entonces ves la importancia de la duración de los derechos de autor, y una de las razones por las que digo el copyright es uno de los activos más valiosos en la industria de la música. Si eres un estudiante fuera de los Estados Unidos. Reconozca que cada país tiene su propia ley de derechos de autor. Y esas leyes pueden ser diferentes de las leyes de los Estados Unidos. Así que asegúrese de considerar las leyes de derechos de autor de su nación. La Ley de Copyright también ha sido modificada para aumentar la cantidad de regalías mecánicas. Pagado a los propietarios de los derechos de autor en la composición. Recordarán en la lección uno, mencioné cómo los escritores de canciones y los editores se preocuparon mucho cuando la primera forma de reproducción mecánica de sus canciones, el piano roll, salió a principios de 1900. Como resultado de eso, los fabricantes de rollos de piano tuvieron que pagar a los propietarios de los derechos de autor y la composición, los compositores y editores una regalía de dos centavos por cada copia del rollo de piano que se vendió.\\
Con los años, esa tasa de regalías mecánicas ha aumentado a la actual tasa de 9.1 centavos por copia para grabaciones de menos de cinco minutos, o \$ 0.0175 por minuto para grabaciones de más de cinco minutos. Los libros y las canciones fueron la principal forma de derechos de autor protegidos por la Ley de Derechos de Autor hasta 1972 cuando el Congreso creó un nuevo tipo de derechos de autor.  a grabación de sonido, la grabación de sonido.\\
El copyright de la grabación de sonido debe diferenciarse del copyright en la composición, y hablemos de eso ahora mismo. Recordarán que al comienzo de esta lección canté el gancho de la canción, If I was Down and Out, la primera canción que escribí. Bueno, después de encontrar ese gancho me fui a casa y Escribí los versos, escribí un puente y armé una canción completa. No lo sabía en ese momento, por eso es tan importante que tomes este curso, para que pueda ser consciente de la importancia de proteger su trabajo. Podría haber copiado escrita esa canción. La música básica y las letras. Y para hacer eso habría tenido que presentar lo que se llama el registro de artes escénicas, o un registro de PA, que protege justamente eso. La música y las letras, la música básica y las letras de la canción.  \textbf{Entonces llamemos a eso el registro de PA. Los derechos de autor en la composición.}\\
Bueno, el copyright de grabación de sonido es diferente de los derechos de autor en la composición. El Copyright de grabación de sonido protege la fijación de sonidos que componen una grabación de la composición. La fijación del sonido, por supuesto, va a incluir todos los diversos instrumentos, todas las voces, toda la mezcla que se hace para crear esa grabación específica. Y el Congreso quería asegurarse de que las compañías discográficas y los artistas estaban protegidos de otras personas haciendo duplicados exactos. De esas grabaciones reales. Entonces aprobaron la enmienda de Copyright de grabación de sonido a la Ley de Copyright. \textbf{Por lo general, los propietarios de los derechos y la composición son los compositores y los editores. Los editores son personas que realmente explotan la música, explotan las canciones. Intenta que las canciones se utilicen de la mayor manera posible para generar ingresos.}\\
Por lo general, los artistas asignan sus derechos y los artistas tienen derechos en las grabaciones de sonido. Porque cuando crean ese trabajo en el estudio, Las inflexiones en su voz es un trabajo creativo, son trabajos creativos. Y asignan esos derechos de autor a las compañías discográficas y las compañías discográficas, por supuesto, explotan las grabaciones para ellos. Entonces me escuchaste cantar la canción If I Was Down And Out. Hice una grabación de sonido de esa canción también. Entonces podría haber tenido o presentado un registro de PA para esa composición y también pude haber presentado un formulario SR que significa grabación de sonido para proteger mis derechos en la grabación de sonido. \\
\textbf{Así que hay dos tipos de derechos de autor que son muy importantes para el músico, tanto para el compositor como para el artista. Los derechos de autor en una composición que está protegida mediante la presentación de las artes escénicas o Tipo de registro de PA. Y la grabación de sonido que está protegida mediante la presentación en el tipo de registro SR.}\\
El copyright de grabación de sonido también le permite proteger no solo la fijación de suena en esa grabación en particular. Pero también le permite proteger cualquier obra de arte que esté en el CD o álbum, o cualquier nota, el paquete completo, o la obra de arte o el trabajo creativo que entra en esa grabación de sonido puede protegerse con el registro SR. Y bajo la ley de derechos de autor de EE. UU. tanto la composición como la grabación de sonido al archivar un sonido registrar los derechos de autor e indicar y marcar la casilla de su propiedad todos los derechos en la composición que está en la grabación también. Y de esa manera puedes proteger tanto la composición como la grabación de sonido bajo el formulario de copyright de grabación de sonido. Una vez más, debe poseer todos los derechos en la composición y la grabación de sonido para poder hacer eso en una forma.\\
En 1976, el Congreso modificó la Ley de Derechos de Autor para incluir un nuevo derecho, eso sería de beneficio para los compositores y artistas. Este derecho otorga a los autores de trabajos no contratados creados después de 1978, los derechos de recuperar la transferencia y las asignaciones de aquellos intereses de derechos de autor 35 años después de la asignación original. Recordarán antes que mencioné cómo la mayoría de los compositores asignar su interés en derechos de autor a los editores. Y en el mercado actual, la mayoría de los editores ofrecen a los compositores lo que se llama coedición y acuerdos de administración donde los editores se encargan de todos Los aspectos administrativos de la explotación de la composición en nombre de los compositores a cambio de recibir el 25\% de los ingresos. El artista de grabación también transfiere su interés de copyright a las compañías discográficas. \textbf{Y como mencioné anteriormente, las compañías discográficas generalmente pagan regalías del 13 al 16\% para la venta de las grabaciones y generalmente 50\% para cualquier otro uso de las grabaciones.} \\
Bueno, puedes ver cómo este es un derecho tan importante. Como 35 años después de 1978, es principios de 2013. Y a principios de este año, varios compositores y artistas reclamados y recapturaron su interés de copyright que transfirieron en 1978. Y como resultado, ahora pueden recibir el 100\% de los ingresos en lugar de dando un 25\% a la editorial, o un 84 a 87\% a la compañía discográfica, o el 50\% que normalmente ceden a las compañías discográficas para otros tipos de usos. Otro nuevo derecho fue creado en copyright en 1995, cuando se aprobó la Ley de Rendimiento Digital en la Grabación de Sonido, dando a los propietarios de derechos de autor de la grabación de sonido, \textbf{las compañías discográficas y artistas, pagos por cualquier transmisión digital de sus derechos de autor. ¿Entonces que significa eso? Eso significa que el propietario de la grabación de sonido, las compañías discográficas y los artistas ahora cobran en cualquier momento esas grabaciones se juegan a través de tipos digitales de transmisiones. Las transmisiones digitales incluyen TV por cable, canales de radio. Canales de internet. Satellite radio. XM Sirius son todas transmisiones digitales.}\\
Mientras que los propietarios de los derechos de autor y composiciones, editores y a los compositores se les paga cuando sus canciones se realizan en la radio terrestre a través de organizaciones de derechos de desempeño. Los propietarios de los derechos de autor y las compañías discográficas de grabación de sonido y los artistas históricamente no lo han hecho. Si bien eso es cierto, la Ley de Derechos de Rendimiento Digital y Grabaciones de Sonido de 1995 creó una nueva fuente de ingresos para las compañías discográficas y los artistas a pagar. Las compañías discográficas y los artistas durante años se han quejado del hecho de que no se les paga cuando sus grabaciones se reproducen en la radio terrestre. Y, de hecho, he estado presionando al Congreso para que promulgue una legislación que lo haga posible. Bueno, no han tenido éxito en este esfuerzo pero al menos el derecho de rendimiento digital de 1995 y Ley de grabación de sonido está dando a las compañías discográficas y artistas la oportunidad de recibir un pago cuando se transmiten sus grabaciones y jugado a través de este creciente mercado de transmisión digital.
\section{Registro}
\subsection{Formas PA y SR}
Los creadores de obras tienen derechos de autor tan pronto a medida que ese trabajo se crea y se reduce a una forma tangible. Ya sea anotado o grabado en un CD, en un MP3 o en una cinta.





 
\chapter{Gerentes, Agentes y Abogados}
\end{document} 
