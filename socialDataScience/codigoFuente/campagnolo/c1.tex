
\pagenumbering{arabic} 
\section{Ciencia de Datos Sociales Xeniales}
El propósito de este documento es explorar la tensión entre lo análogo y lo digital como parte de un programa de investigación en evolución y explorar la secuencia de métodos dentro del mismo. Como veremos, la ciencia de los datos sociales debería situarse mejor dentro de la genealogía de las relaciones disciplinarias de largo plazo entre la sociología fenomenológica, la experiencia en informática asociada a la digitalización y el positivismo narrativo (Abbott 1992) vinculado al uso de la estadística en la investigación social.\\

Una parte considerable de lo que el CERS define como investigación social digital se ha inspirado en enfoques topológicos de redes (Moats and Borra, 2018).\\

Como en el caso del influyente estudio de Greiffenhagen et al. (2011), este libro llevará a cabo una investigación simétrica del proceso de trabajo científico en las tres disciplinas que constituyen la ciencia de los datos sociales, incluido su componente más cualitativo, con la esperanza de ayudar a más investigadores a apreciar la extraña alianza de la sociología fenomenológica, la informática y la estadística social.\\

\subsection{Sociología Fenomenológica, Informática y Estadística Social}
La ciencia de los datos sociales es un diálogo múltiple en el que intervienen diversos enfoques y subdisciplinas. Este libro se ocupará principalmente de dos aspectos: (i) la historia de la colaboración de la etnografía con la informática (Fele 2009; Suchman 2007[1987]; Dourish 2001) y (ii) el debate entre la fenomenología social y la estadística social (Snow 1959; Hindess 1973; Collins 1984; Mahoney y Goertz 2006), frente al uso de formas nuevas y emergentes de datos.

\subsection{Investigación sustantiva. El interés como método}

\section{Representación}
La innovación en los métodos de investigación social significaba la comprensión de cómo los cambios físicos en el lugar de trabajo son paralelos a los cambios en el método de investigación social (Law y Urry 2004).\\

Para Husserl, la representación es la experiencia de ver "más" que los detalles perspectivos presentados concretamente: ver que una casa tiene un interior y una parte trasera cuando se ve de frente. Lo que no está concretamente "presente" está implicado junto con lo que es inmediatamente aparente y, recíprocamente, determina el sentido de lo que se ve (una casa con un interior, etc.). Schutz desarrolló aún más la idea de aplicar a la interacción social el "emparejamiento apresentacional" (Schutz 1971: 248). Se trata de una forma de conceptualizar la "percepción del otro" recíproca: la apreciación por parte de un receptor de lo que un hablante pretende con una expresión manifiesta, junto con la orientación interna del hablante hacia la (posible) comprensión del receptor.


\subsection{Los efectos formativos de las prácticas digitales en la investigación social}
Dado que los datos de las redes sociales hacen que la interacción social ordinaria sea tan intensiva en datos la naturaleza cada vez más digital de las prácticas sociales está teniendo un efecto formativo sobre cómo (podemos) estudiarlas. El argumento sobre los efectos formativos de las prácticas de uso intensivo de datos en la investigación social se inspira en la discusión de Pollner sobre la relación entre la etnometodología y los negocios (Pollner 2012b). \\

los preceptos etnometodológicos fundamentales siguen siendo válidos y útiles. Sin embargo, mi atención a la habilidad de las personas para dar sentido a lo que está fuera de su alcance local (por ejemplo, la apresentación) también muestra que para lograr una descripción naturalista de los métodos de los miembros en situaciones cada vez más conectadas, es necesario reconsiderar supuestos prominentes en los escritos etnometodológicos como la idea de "setting" como un circuito de acción en cierto modo cerrado, autocompensante y autoterminante.\\

Al igual que Garfnkel incorporó la contabilidad, las cuentas y la rendición de cuentas en la propia especificación del objeto de la etnometodología, el intento de la etnometodología contemporánea debería consistir en ampliar este esfuerzo considerando las prácticas sociales basadas en plataformas digitales como, por ejemplo, las que se producen a través de los sitios de redes sociales y profesionales. El homo sociologicus de Garfnkel era un "contable" (Pollner 2012a: 25), que hacía que sus acciones fueran observables/reportables en un espacio pequeño, que implicaba hablar más que escribir, y que apuntaba al mundo de las señales no verbales que acompañan a dichos intercambios (Goodwin 1996). Nuestro homo sociologicus es una persona contemporánea que organiza sus asuntos sociales y profesionales globales con la ayuda de plataformas digitales, dejando tras de sí una cantidad desmesurada de rastros digitales.\\

Los sitios de redes sociales y profesionales revelan niveles inexplorados de posibilidad en la comprensión detallada de la integración de los hechos sociales. En situaciones digitales, las realidades de las pantallas y los algoritmos que las producen se convierten en "un socio interactivo para los participantes" (Knorr Cetina 2009: 72). 

\section{El mapeo analógico}

\subsection{Algunos casos prácticos de cartografía analógica}

\subsubsection{Ecologías vinculadas}
Para tener éxito en una ecología, un desarrollo tecnológico debe proporcionar resultados también a los aliados en ecologías vecinas. Donde existen bisagras, es decir estrategias que funcionan en ambas ecologías a la vez, pero puede percibirse de maneras diferentes en las dos. 

\subsubsection{Comparación de los mapas de situación y de red digital}
